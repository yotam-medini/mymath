% 

%%%%%%%%%%%%%%%%%%%%%%%%%%%%%%%%%%%%%%%%%%%%%%%%%%%%%%%%%%%%%%%%%%%%%%%%
%%%%%%%%%%%%%%%%%%%%%%%%%%%%%%%%%%%%%%%%%%%%%%%%%%%%%%%%%%%%%%%%%%%%%%%%
%%%%%%%%%%%%%%%%%%%%%%%%%%%%%%%%%%%%%%%%%%%%%%%%%%%%%%%%%%%%%%%%%%%%%%%%
\chapterTypeout{Background Material}

%%%%%%%%%%%%%%%%%%%%%%%%%%%%%%%%%%%%%%%%%%%%%%%%%%%%%%%%%%%%%%%%%%%%%%%%
%%%%%%%%%%%%%%%%%%%%%%%%%%%%%%%%%%%%%%%%%%%%%%%%%%%%%%%%%%%%%%%%%%%%%%%%
\section{Notes}

\begin{manuallemma}{I.7.8}
\(ON\) is a transitive class. That is, if \(\alpha \in ON\)
and \(z \in \alpha\),
then \(z \in ON\).
\end{manuallemma}
The proof says ``it is easy to check that $z$ is a transitive set.''
\\
Here is how. Let \(y \in z\), we need to show \(y \subseteq z\).
By transitivity, \(z \subseteq \alpha\) and so \(y \in \alpha\).
Again by transitivity, \(y \subseteq \alpha\).
Let \(x\in y\), thus \(x \in \alpha\) and actually \(x,y,z \in \alpha\).
By orering of \(\alpha\), we have \(x\in y \in z\) and hence \(x\in z\)
giving the desired \(y \subseteq z\).

%%%%%%%%%%%%%%%%%%%%%%%%%%%%%%%%%%%%%%%%%%%%%%%%%%%%%%%%%%%%%%%%%%%%%%%%
\section{Old Edition Exercises } % pages 111-115


%%%%%%%%%%%%%%%%%
\begin{enumerate}
%%%%%%%%%%%%%%%%%

%%%%%%% 1
\begin{excopy}
Write a formula expressing 
 \(z = \lrangle{\lrangle{x,y},\lrangle{v,w}}\)
using just \(\in\) and \(=\).
\end{excopy}

\unfinished

%%%%%%% 
\begin{excopy}
Show that \(\alpha < \beta\) implies that 
\(\gamma + \alpha < \gamma + \beta\) 
and
\(\alpha + \gamma \leq \beta + \gamma\).
Give an example to show that the \(\leq\)
cannot be replaced by $<$.
Also show:
\begin{equation*}
\alpha < \beta \ra \exists! \delta (\alpha + \delta = \beta).
\end{equation*}
\end{excopy}

Following Definition~7.17
\begin{equation*}
\gamma\times\{0\} \cup \alpha\times\{1\}
\subsetneq
\gamma\times\{0\} \cup \beta\times\{1\}.\
\end{equation*}
and the corresponding relations
\(R_{\gamma+\alpha} \subsetneq R_{\gamma+\beta}\).
By Theorem~6.3, the unique identity isomorphism of 
\(\gamma+\alpha\) to itself as a proper subset gives 
\(\gamma + \alpha < \gamma + \beta\),

By negation, \(\beta + \gamma < \alpha + \gamma\).
Consider the injectin isomorphism from 
\(\beta\times\{0\} \cup \gamma\times\{1\}\)
into a proper subset of 
\(\alpha\times\{0\} \cup \gamma\times\{1\}\).

The \(\leq\) is needed for \(0+\omega = 1+\omega\).
\unfinished

\end{enumerate}

%%%%%%%%%%%%%%%%%%%%%%%%%%%%%%%%%%%%%%%%%%%%%%%%%%%%%%%%%%%%%%%%%%%%%%%%
\section{Current Edition Exercises } % pages 111-115

\begin{lexcopy}{I.4.18}{24}
Derive \(\forall y[y \notin y]\) from the Axioms of Comprehension and
Foundation. Don't use the Pairing or Extensionality Axioms. Then find a
two element model for Foundation, Extensionality, Pairing, and Union, plus
\(\exists y\forall x [x \in y]\) (so, of course, \(y \in y\)).
\end{lexcopy}

By negation \(y\in y\).
By Comprehension, let \(\varphi(z)\) be \(z=y\). Apply
\begin{equation*}
  \exists x \forall z (z \in x \wedge \varphi(z))
\end{equation*}
gives \(x:=\{y\}\). By Foundation, $y$ being the (only) minimal
element for which
\mbox{\(\lnot\exists z(z\in y \wedge z \in \{y\})\)}
should hold, but \(z=y\) gives contradiction.

Let the model have two ``sets'': \(\mu\), \(\nu\).
with the following ``\(\in\)'' relations:
\(\mu\in\nu\), \(\nu\in\nu\).

\begin{lexcopy}{I.6.23}{31}
Define the relation $R$ on \(\N \times \N\) by saying: \((x_0,x_1)R(y_0,y_1)\)
iff \(x_0 \leq y_0\) and \(x_1 \leq y_1\)
and \((x_0,x_1)\neq (y_0,y_1)\).
Show that $R$ is well-founded
on \N\ but not a total order.
Show that every non-empty \(X \subseteq \N\times\N\) contains
only finitely many $R$-minimal elements, and, for each \(k \in\N\),
there is such an
$X$ with exactly $k$ $R$-minimal elements.
\end{lexcopy}

Let non-empty \(A \subset \N\times\N\).
Let \(P = \{n\in\N: \exists m((n,m)\in A\).
For each \(n\in P\) let \(Q_n = \{m\in\N: (n,m)\in A\}\).
Let \(p = \min(P)\) and \(q = \min(Q_p)\).
Clearly \((p,q)\in A\) is minimal in $A$.

Not total order since
\((0,1)\not{R}(1,0)\) and
\((1,0)\not{R}(0,1)\).

Given $k$, \(A=\{(n,k-n-1)\in\N\times\N: 0\leq n < k\}\)
has $k$ elements all of which are minimal.

\begin{lexcopy}{I.6.26}{31}
Show that \(\in\) is a well-order of $3$.
\end{lexcopy}

The set 
\(3 = \{\emptyset, \{\emptyset\}, \{\emptyset, \{\emptyset\}\}\}\).
has $3$ elements. Clearly for any two elements \(x,y\in 3\)
we have \(x=y \lor (x\in y \lor y \in x)\).
and any non empty subset has a minimal element.

\begin{lexcopy}{I.6.29}{33}
Show that the Axiom of Foundation implies that the successor
function $S$ is $1-1$ on $V$.
\end{lexcopy}
Assume \(x \neq y\). % , \wlogy\ \(b\in x \setminus y\).
If by negation \(S(x) = S(y)\), then % \(b = \{y\}\).
\(x \in S(y)\) and so \(x\in y\).
Similarly \(y \in x\). Thus \(\{x, y\}\) has no ``\(\in\)-minimal''
element contradiction to to the Axiom of Foundation.

\begin{lexcopy}{I.7.2}{34}
Assuming the Axiom of Foundation, show that every 
nonempty transitive set contains $0$ and show that every non-singleton
transitive
set contains $1$. Then show that $1$ is the only one-element transitive set,
and $2$
is the only two-element transitive set.
\end{lexcopy}

Let \(T\neq\emptyset\) be transitive set.
By Axiom of Foundation there is an
element \(e\in T\) if by negation \(x\in e\),
by transitivity \(x\in T\) contradicting the existance of $e$,
thus \(e = \emptyset \in T\).
Clearly \(1= \{\emptyset\}\) is the only singleton transitive set.

If $T$ is non-singleton, let \(S = T \setminus \{\emptyset\} \neq \emptyset\).
Since $S$ is not transitive, let \(u \in S\)
be such that \(u \not\subseteq S\).
Since \(u\subseteq T\), we have \(\emptyset \in u\)
that is \(1 = \{\emptyset\} \in u\).

If $T$ is two element set,
then \(S = T \setminus \{\emptyset\}=\{u\}\) is a singleton.
Also \(u \subset T = \{\emptyset, u\}\).
But \(u \neq \{\emptyset\}\),
so $u$ must be singleton conisting of some element of $T$.
Since \(u\notin u\) its element must be \(\emptyset\),
namely \(u = \{\emptyset\}\) and
\(T = \{\emptyset, \{\emptyset\}\} = 2\).

\begin{lexcopy}{I.7.25}{38}
  Assuming the Axiom of Foundation, $z$ is an ordinal iff $z$ is
  a transitive set and \(\in\) satisfies trichotomy on $z$.
\end{lexcopy}

If $z$ is an ordinal, then by definition it is transitive
and since  \(\in\) is an order, it satisfies trichotomy.

Conversely, assume that  $z$ is  a transitive set
and \(\in\) is trichotomy on $z$.
Let \(S\subseteq z\) be some non-empty set. By Axiom of Foundation,
we have \(y \in S\) such that \(\forall z\in y (z\notin S)\).
If by negation $y$ uis not \(\in\)-minimal in $S$,
then \(\exists m\in s(m\in y)\) is a contradiction.
Ths \(\in\) well orders $z$ and thus $z$ is an ordinal.


\begin{lexcopy}{I.7.26}{39}
  Assuming the Axiom of Foundation, $z$ is an ordinal iff $z$ is
a transitive set and all elements of $z$ are transitive sets.
\end{lexcopy}

If $z$ is an ordinal, the by definition it is transitive,
its elements are ordinals by Lemma~ I.7.8 thus transitive.

Conversely, we need to show that \(\in\) is well-order on \(\alpha\).
Assume by negation \(\in\) is not a total order.
Let \(x\in z\) be \(\in\)-minimal such that it is non-comparable
with some elemet in $z$.
Let \(y\in z\) be \(\in\)-minimal such that it is non-comparable
with $x$.
Now if \(u\in y\) then \(u\in x\) hence \(y \subseteq x\).
Similarly, if \(u\in x\) then \(u\in y\) hence \(x \subseteq y\),
thus \(x = y\) is a contradiction, and \(\in\) is a total-order on $z$
and by the Axiom of Foundation is well-order. Hence $z$ is an ordinal.

\begin{lexcopy}{I.7.27}{39}
  Show, in \(BST^{-}\), that if $A$,$B$ are finite,
  then \(A\cup B\) and \(A\times B\)
are finite, and \(\mathcal{P}(A)\) exists and is finite.
\end{lexcopy}
Here without Axiom of Foundation.

By induction on ordinal $n$ for which \(B \preccurlyeq n\).
Let \(B \preccurlyeq n + 1 = \{0, 1, \ldots n - 1\}\).
Pick the (last) element \(b\in B\) that is mapped to \(n-1\),
and let \(B' = B \setminus \{b\}\), so now \(B \preccurlyeq n\).

\(\mathbf{A\cup B}\):
Let \(f'\) be a 1-1 function \(A \cup B' \to n - 1\)
If \(b\in A\) the the same \(f'\) maps  \(A \cup B \to n - 1\),
Otherwise we define \(f: A \cup B \to n\), by
\(f(b) = n - 1\) and \(f(b') = f'(b')\) for \(b'\in B'\).

\(\mathbf{A\times B}\):
Let \(g: A \to m\) be 1-1 mapping, and let \(a_i = g^{-1}(i)\)
for \(0 \leq i < m\).
Let \(f'\) be a 1-1 function \(A \times B' \to M\)
It is easy to have 1-1- map
\(A \times B \to (A \times B') \cup (\{x\} \times B)\),
we can pick any \(x\notin A\) for example \(\{A\}\).
The finiteness is now derived from the previous result.

\(\mathbf{P(A)}\): Similarly we
set the disjoint union \(A = A' \cup \{a\}\).
Given a 1-1 map \(f': P(A') \to m\)
Define
\begin{equation*}
f: P(A) \to (\{0\}\times m) \cup (\{1\}\times m)
\end{equation*}
by:
\begin{equation*}
f(B) =
\left\{
\begin{array}{ll}
(\{0\}, f'(B)) & \textrm{if}\; B \subseteq A' \\
(\{1\}, f'(B \setminus \{a\})) & \textrm{if}\; B \not\subseteq A'
\end{array}
\right.
\end{equation*}

\begin{lexcopy}{I.9.6}{46}
  Derive the Axioms of Infinity and Replacement from (2) of Lemma I.9.5.
\end{lexcopy}
\textbf{Infinity:} Define $R$ by \(xRy\) iff \(S(x) = y\)
\\
\textbf{Replacement:} Define $R$ by \(xRy\) iff
\(\forall z(\varphi(x, z) \to y = z)\).

\begin{lexcopy}{I.9.8}{46}
  If $R$ is well-founded on $A$, then $R$ is acyclic on $A$; hence, \(R^*\)
  is a strict partial order on $A$.
  Conversely, if $A$ is finite and $R$ is acyclic on $A$,
then $R$ is well-founded on $A$.
\end{lexcopy}
If there was an $R$-cylce there would be no $R$ minimal element in $A$.
The definition of \(R^*\) via $R$-path gives transitivity, namely
\begin{equation*}
\forall x,y,z\,\left(xRy \wedge yRz \;\to\; xRz\right).
\end{equation*}

Conversely, each $R$-path must be finite for there are no cycles
and $A$ is finite. Thus there exists $R$-minimal element
(or \(A=\emptyset\)).

\begin{lexcopy}{I.9.27}{54}
  Assume that each element of \R\ has rank \(\omega\);
  this will be true
using our Definition 1.14-2. Let \(\mu\) be Lebesgue measure on \R. Prove that
\(\R,\mu \in WF\), with \(\textrm{rank}(\R) = \omega + 1\)
and \(\textrm{rank}(\mu) = \omega + 4\).
\end{lexcopy}
Since the members of each \(x\in \R\) all have rank of \(\omega\),
\begin{equation*}
\rank(\R)=\max(\{\rank(x)+1: x\in\R\|) = \omega+1.
\end{equation*}
Nonempty subsets of \(\R\) have rank of \(\omega+2\).
Rank of pairs where \(a\in\R\) and \(\emptyset \neq A \subset \R\)
\begin{equation*}
\rank(\langle A, a\rangle)
 = \max(\rank(A),\rank(a)) + 2
 = \max(\omega+1,\omega) + 2 = \omega + 3.
\end{equation*}
Now \(\mu\) is a nonempty set cosisting of such pairs,
Thus
\begin{equation*}
\omega + 3 + 1 \leq \rank(\mu)
\leq \rank\left(
  \mathcal{P}(\{\langle A, a\rangle): a\in\R \wedge A\subset\R\}
   \right)
= \omega + 4.
\end{equation*}
So the abpve inequalities are actually equalities.

\begin{lexcopy}{I.9.39}{59}
Assume that $R$ is well-founded and set-like on $A$. Then
\(\textrm{rank}_{A,R}\) is 1-1 iff \(R^*\) is a total order of $A$.
\end{lexcopy}

Assume \(\textrm{rank}_{A,R}\) is 1-1
and by negation \((x,y),(y,x) \notin R^*\).
Let \(W = \{x,y\}\,\cup\,\{w\in A: (w,x)\in R^*\;\lor\; (w,y)\in R^*\}\).


Let \(E=\{y'\in W: y'\neq x\;\land\; (y',x)\notin R^*\}\neq \emptyset\).
\(y\in E\), so let \(y_0\in E\) be \(R^*\)-minimal.
Similarly, 
let \(F=\{x'\in E: x'\neq y' \;\land\; (x',y')\notin R^*\}\neq \emptyset\)
and let \(x_0\in F\) be \(R^*\)-minimal and \(x_0 \neq y_0\).

Being minimal we now have
\begin{equation*}
\{z\in A: (z,x_0)\in R^*\} = \{z\in A: (z,y_o)\in R^*\}.
\end{equation*}
Hence \(\rank(x_0) = \rank(y_0)\) a contradiction.

Conversely, assume \(\R^*\) is a total order.
Let \(x,y\in A\), \wlogy \((x,y)\in R^*\) thus \(\rank(y)>\rank(x)\).


\begin{lexcopy}{I.9.40}{59}
If \(R_1 \subseteq R_2\) are both well-founded and set-like on $A$, then
\(\rank_{A,R_1}(y) \leq \rank_{A,R_2}(y)\)
for all \(y \in A\). Also, 
\(\rank_{A,R_1}(y) = \rank_{A,R_2}(y)\)
if \(R_1 \subseteq R_2 \subseteq (R_1)^*\).
\end{lexcopy}

By negation assume
\begin{equation} \label{eq:I.9.40:neg}
\rank_{A,R_1}(y) > \rank_{A,R_2}(y).
\end{equation}
By looking at \(\pred_{A,R_2}(y)\) we can assume that $y$ is \(R_2\)-mininal
for \eqref{eq:I.9.40:neg}.
Now we have the contradiction:
\begin{equation*}
\rank_{A,R_1}(y)
= \sup\{\rank_{A,R_1}(x): xR_1y\}
\leq 
  \sup\{\rank_{A,R_2}(x): xR_2y\} = \rank_{A,R_2}(y).
\end{equation*}

We will now show that for any well-founded and set-like $R$ on $A$
\begin{equation} \label{eq:I.9.40:ReqRstar}
\rank_{A,R}(x) = \rank_{A,R^*}(x) \quad \forall x\in A.
\end{equation}
By induction define \(R^0=\emptyset\), \(R^1=R\)
and for each \(n<\omega \land n > 1\)
\begin{equation*}
R^n = \left\{\langle x, y \rangle: \exists z, xR^{n-1}z \land zR^{n-1}y\right\}.
\end{equation*}
Clearly \(R^* = \cup_{n<\omega}R^n\), so it will
be sufficient to show by induction that
\begin{equation}  \label{eq:I.9.40:ReqRn}
\rank_{A,R}(x) = \rank_{A,R^n}(x) \quad \forall x\in A.
\end{equation}
Let \(n > 1\).
By previous result we have \(\rank_{A,R^{n-1}}(x) \leq \rank_{A,R^n}(x)\).
By negation assume
\begin{equation} \label{eq:I.9.40:negstar}
\rank_{A,R^{n-1}}(x) < \rank_{A,R^n}(x).
\end{equation}
By considereing \(\pred_{A,R^n}(x)\)
we can assume that $x$ is \(R^n\)-mininal for \eqref{eq:I.9.40:negstar}.
By induction, for each $y$ such that \(yR^{n-1}x\) we have
\(\rank_{A,R^{n-1}}(y) = \rank_{A,R^n}(y)\). Hence there is some $y$
such that \(\langle y, x \rangle \in R^n \setminus R^{n-1}\).
Thus there is in $R$-path of $n$ steps via $s$ function,
where \(s(0)=y\), \(s(k)\,R\,s(k+1)\) and \(s(n)=x\)
Put \(x'=s(n-1)\), so \(\rank_{A,R^{n-1}}(x') < \rank_{A,R^n}(x')\)
Since \(x'Rx\) and $y$ and $s$ are arbitrary \eqref{eq:I.9.40:ReqRn}
and so \eqref{eq:I.9.40:ReqRstar} is proved.

Back to the exercise's last part.  we already shown
\begin{equation*}
\rank_{A,R_1}(y) \leq \rank_{A,R_2}(y) \leq \rank_{A,R_*^*}(y)
\end{equation*}
by \eqref{eq:I.9.40:ReqRn} the last inequalities are actual equalities,
that is \(\rank_{A,R_1}(y) = \rank_{A,R_2}(y)\).

\begin{lexcopy}{I.9.41}{59}
Assume that $R$ is well-founded and set-like on $A$, and that $R$
is a transitive relation on $A$.
Then \(\mos_{A,R}(a) = \rank_{A,R}(a)\) for all \(a \in A\).
\end{lexcopy}

% using:  frederic-wang.fr/mathematics/set-theory/kunen/chapter_1.html
By transitivity \(R=R^*\).
By negation we have \(\mos_{A,R}(a) \neq \rank_{A,R}(a)\)
for some $R$-minimal $a$.
Since \(\mos(\emptyset)=\emptyset=\rank(\emptyset])\)
clearly \(a\neq\emptyset\).
By Lemma~I.7.13 union of ordinals that is a set --- is an ordinal.
By minimally
\begin{equation*}
\mos(a)
= \{\mos(x): xRa\}
= \{\rank(x): xRa\} % = \alpha. % = \beta = \rank(b).
\end{equation*}
and by Lemma~I.9.16
\begin{equation*}
\{\rank(x): xRa\} = \{\alpha: \alpha < \rank(a)\} = \rank(a),
\end{equation*}

\begin{lexcopy}{I.9.42}{59}
Assume that $R$ well-orders the set $A$. Then the three 
functions \(\textrm{mos}_{A,R}\), \(\textrm{rank}_{A,R}\),
and the isomorphism of $A$ onto an ordinal described in
Theorem~1.8.2, are all the same.
\end{lexcopy}
The proof of Theorem~1.8.2 dedines a map from \(a\in A\)
to a unique oridinal \(\xi\) so \((\xi,\in)\)
is isomorphic with \((a\kern -0.2em\downarrow,R)\).
By Lemma~I.9.17 \(rank_{ON,\in}(\xi) = \xi\)
and since the isomorphism is unique, it is equals to \(\rank\).
By previous exercise is equals to \(\mos\) as well.

\begin{lexcopy}{I.9.43}{59}
Work in \(ZF^-\). Let $K$ be any class and assume that every
subset of $K$ is a member of $K$. Prove that \(WF \subseteq K\).
\end{lexcopy}

By negation \(WF \not\subseteq K\) and by Lemma~I.6.28
let $a$ be \(\in\)-minimal element in \(WF\setminus K\).
Now
\begin{equation*}
a = \{x\in a\} = \{x \in a\cap K\} \subseteq K.
\end{equation*}
The last equality holds by minimality of $a$.
But \(a \in K\) by the assumption which is a contradiction.

\begin{lexcopy}{I.9.44}{59}
For any relation $R$ on a class $A$, and \(a \in A\): $R$
is well-founded on \(\textrm{pred}_{A,R^*}(a)\) iff $R$
is well-founded on \(\{a\} \cup \textrm{pred}_{A,R^*}(a)\).
\end{lexcopy}
Let \(A_0 = \textrm{pred}_{A,R^*}(a)\) and let \(A_1 = \{a\} \cup A_0\).

\(\rightarrow\): Let $B$ be a subset of \(A_1\).
If \(a \notin B\) then \(B \subset A_0\) and has a $R$-minimal element.
Otherwise \(a \in B\) then either $a$ is $R$-minimal in $B$
or \(B \setminus \{a\} \subset A_0\) has $R$-minimal element.

\(\leftarrow\):  Let $B$ be a subset of \(A_0 \subset A_1\),
and by assumption has a $R$-minimal element.

\begin{lexcopy}{I.9.46}{60}
Let $R$ be set-like on $A$, and let \(W = WF_{A,R}\) Then:
\begin{enumerate}
\item \(A\setminus W\) has no $R$-minimal elements.
\item $R$ is well-founded on $W$.
\item For \(a \in W\), \(\rank_{A,R}(a) = \rank_{W,R}(a)\).
\item For \(a\in A\): \(a\in W\) iff \(\pred_{A,R^*}(a) \subseteq W\)
      iff \(\pred_{A,R}(a) \subseteq W\).
\end{enumerate}
\end{lexcopy}

\begin{enumerate}
  \item
    Using item~2.\ If by negation $a$ is $R$-minimal in \(A \setminus W\)
    then \(P=\pred_{A,R^*}(b) \subset W\).
    Then by following \textbf{2.}, $R$ is well-founded on $P$
    contrdiction to \(a\notin W\).
  
  \item
    Let \(B \subset W\) be non-empty subset.
    Pick some \(b\in B\). Let \(P = \pred_{A,R^*}(b)\) which is well-founded.
    Now $R$ is well-founded on \(B \cap P\) and has a $R$-minimal
    element \(c \in B \cap P)\). But $c$ must also be $R$-minimal in $B$
    since otherwise there exists \(x\in B\) and \(xRc\), and then \(x\in P\)
    contradicting $R$-minimality of $c$ in \(B \cap P)\).
    Hence $W$ is well-founded.

  \item
    For \(a\in W\)
    \begin{align}
      \rank_{A,R}(a) &= \rank_{\{a\}\cup\pred_{A,R^*},R}(a) \nonumber \\
      &= \sup\{\rank_{\pred_{A,R^*},R}(x) + 1: xRa\} \nonumber \\
      &= \sup\{\rank_{W,R}(x) + 1: xRa\} = \rank_{W,R}(a) \label{eq:rank:W}
    \end{align}
    The \eqref{eq:rank:W} equality
    is justified by the fact the the \(\sup\) operator
    recursively runs in both cases on \(\pred_{A,R^*}(a)\).

  \item
    Assume \(a \in W\). Then $R$ is well founded on \(P_a=\pred_{A,R^*}(a)\).
    Clearly for each \(b\in P\)
    \begin{equation*}
      P_b = \pred_{A,R^*}(b) \subset \pred_{A,R^*}(a) = P_a.
    \end{equation*}
    and  $R$ is well founded on \(P_n\) and thus \(b\in W\)
    and so \(P_a\subset W\).

    Assume \(\pred_{A,R^*}(a) \subset W\). Clearly
    \begin{equation*}
      \pred_{A,R}(a) \subset \pred_{A,R^*}(a) \subset W
    \end{equation*}

    Assume \(P_a = \pred_{A,R}(a) \subset W\).
    For each \(b \in P_a\) we have $R$ well founded on\(P_b = \pred_{A,R^*}(b)\).
    \begin{align*}
       \pred_{A,R^*}(a) = \cup_{b\in P_a} \{b\} \cup \pred_{A,R^*}(b) \subset W.
    \end{align*}
    By item~2.\ $R$ is well-founded on $W$ and thus also on \(P_a\).
    Hence \(a\in W\).
\end{enumerate}

\begin{lexcopy}{I.9.47}{60}
Call a relation $R$ on a set $A$ anti-transitive iff it satisfies
\(\forall x y z \in A\, [xRy \land yRz \to xRz]\).
Prove that if $R$ is any well-founded relation
on a set $A$, and all elements of $A$ have finite rank,
then there is a unique anti-transitive \(H \subset R\)
such that \(H^* — R^*\). Then give a counter-example to this
where $R$ is a well-order of a countable set.
\end{lexcopy}

The exercise is wrong.
Consider \(A = \{0, 1, 2\}\) and the natural
\begin{align*}
  R &= \{(0, 1), (1, 2), (0, 2)\} \\
  H_0 &= \{(0, 1), (1, 2)\} \\
  H_1 &= R
\end{align*}
and \(H_0^* = H_1^* = R\) but \(H_0 \neq H_1\)..

\begin{lexcopy}{I.9.48}{60}
The requirements for a Ph.D. in math at \(\aleph\)U are only the
courses \framebox{\(\sigma\)} for each
\(\sigma \in \omega^{<\omega}\) (see Definition~1.6.11). The prerequisites for
\framebox{a} are all \framebox{\(\tau\)} such that
\(\tau\) is obtained from \(\tau\) by replacing one term $n$ from
\(\tau\)r by a finite (possibly empty) sequence of numbers less than $n$.
For example,
\framebox{\(2,1,7,1\)} teaches you all about the sequence \((2,1, 7,1)a\),
and has as prerequisites
\framebox{\(1,7,1\)},
\framebox{\(|0,1,1,0,1,7,1\)},
\framebox{\(2,1,5,5,4,0,1,1\)}, etc.
\framebox{\ } teaches you all about the empty sequence \(\emptyset\),
and has no prerequisite. No number is less than $0$, so
the only prerequisite for \framebox{$0$} is \framebox{\ }.

\begin{enumerate}
\renewcommand{\theenumi}{\alph{enumi}}
\item
Prove that the prerequisite relation is well-founded; so, you can graduate
in some ordinal number of semesters.
\item
Compute the \(\rank\) function; note that here, \(rank(x)\) is the first semester
in which you can possibly take course $x$.
\end{enumerate}
\end{lexcopy}

\begin{enumerate}
\renewcommand{\theenumi}{\alph{enumi}}
\item
 For each sequence \(\tau\) consider its canonical reduction
 \(tau'\) to strictly reducing sequence where for each $n$ in \(tau\),
 $n$ appears exactly once in \(tau\).
 Consider the lexicographic order on the canonical reductions.
 For each subset $S$ sequences, we can find some \(\sigma\in S\) with the minimal
 canonical reduction.
 By considering the first difference in the canonical reductions,
 it is easy to see that \(\sigma\)  has no strict prerequisite in $S$.
\item
Let $R$ be the prerequisite relation.
Let \(z_0 = \emptyset\) and for any natural \(n>1\)
\(z_n = \bigl(\overbrace{0,\ldots,0}^n\bigr)\).
Clearly \(\rank(z_n) = n\).
For any \(\tau\)
having \(m>0\) in its sequence we have \(z_n R \tau\)
and thus \(\rank(\tau)=\omega\).
\end{enumerate}

\begin{lexcopy}{I.9.49}{60}
Let $R$ be well-founded and set-like on $A$. Prove that there is
a class $M$ and an isomorphism from \((A,R)\) onto \((M, \in)\).
\end{lexcopy}
Following hint. For \(y\in A\)
let \(T(y) = \{\lrangle{0,y},\lrangle{1,y},\lrangle{2,y}\}\)
and let
\begin{equation*}
F(y) = \{F(x): x\in y\kdownarrow\} \cup T(y). %  = \textrm{mos}(y)\cup T(y).
\end{equation*}
% Similar to Mostowsky function which is not necessarily \(1-1\).
If \(xRy\) then by definition \(F(x)\in F(y)\).
Assume \(F(y_1) = F(y_2)\). Then \(z_1 = \lrangle{0,y_1} \in F(y_2)\).
But \(\lrangle{0,y_1}\) has at most two elements,
while \(F(x)\) has at least three.
Hence \(\lrangle{0,y_1} \in T(y_2)\) and by uniqueness of pairs (Lemma~I.4.12)
\(\lrangle{0,y_1} = \lrangle{0,y_2}\) and so \(y_1 = y_2\).
Thus $F$ is 1-1, and with \(M=F(A)\) we have isomorphism.

\begin{lexcopy}{I.9.50}{61}
Assume that $R$ is set-like on $A$ and not well-founded. Define
a \(G(x,s)\) that is an explicit counter-example to Theorem~I1.9.11.
\end{lexcopy}
Define
\begin{equation*}
\widehat{S}(y) =
 \left\{
 \begin{array}{ll}
 y + 1 & y \in ON \\
 \emptyset & \textnormal{otherwise}
 \end{array}\right.
\end{equation*}
Now define for \(x\in A\)
\begin{equation*}
G(x,s) =
 \left\{
 \begin{array}{ll}
  \bigcup \{\widehat{S}(s(y)) &
    s \;\textnormal{is a function and}\, \dom(s)=x\kdownarrow \\
 \emptyset & \textnormal{otherwise}
 \end{array}\right.
\end{equation*}
If by negation there exists $F$ such that
\(F(x) = G(x,F\vert x\kdownarrow)\).
Union of ordinals is an orfinal and so is \(F(x)\).
If \(xRy\) then
\begin{equation*}
F(x) < F(x) + 1 = \widehat{S}(x) \leq F(y).
\end{equation*}
By Lemma~I.9.9, $R$ is well-founded, a contradiction.

\begin{lexcopy}{I.9.51}{61}
Work in ZF. Assume that $R$ is well-founded on $A$, and that
$X$ is a non-empty sub-class of $A$. Prove that $X$ has an $R$-minimal element.
\end{lexcopy}
Using Definition~I.9.28 \(R(\alpha) = \{x \in WF : \rank(x) < \alpha\}\)
Assume by negation $X$ has no $R$-minimal element.
For each (set!) \(x\in X\) let
\begin{equation*}
f(x) = \min\{\rank(z) : z \in X \wedge zRx\}
\end{equation*}
Clearly \(f(x)\) are ordinals. Pick \(x_0 \in X\) and let
\(\gamma_0 = \rank(x_9) + 1\). By induction define for any natural \(n\in\omega\)
\begin{equation*}
\gamma_{n+1} = \sup\left(\{f⁢(x): x \in X\cap R⁢(\gamma_n)\} \cup
  \{\gamma_n + 1\}\right).
\end{equation*}
Let \(\gamma = \sup_{n\in\omega} \gamma_n\). Now
\begin{equation*}
x_0 \in X\cap R⁢(\gamma_0) \subseteq X\cap R⁢(\gamma).
\end{equation*}
and \(\rank⁡(x) < \gamma\) for all \(x\in X\cap R⁢(\gamma)\).
% So there is \(n<\omega\) such that \(\rank(x) < \gamma_n\).
% Then \(f⁢(x) \leq \gamma_n + 1 < \gamma\).

Now, let \(x_1\) be a $R$-minimal element of the set \(X\cap R⁢(\gamma)\).
By the definition of $f$, 
let \(z\in X\) be such that \(z⁢R⁢x_1\) and \(\rank⁡(z)=f⁢(x_1)\).
Then \(\rank⁡(z) < \gamma\) and so actually \(z \in X\cap R⁢(\gamma)\).
This contradicts the minimality of \(x_1\).

\begin{lexcopy}{I.9.52}{61}
Verify the statements made about exponentiation in Table~1.1
on page~40.
\end{lexcopy}
\unfinished

\begin{lexcopy}{I.9.53}{61}
Let \(\gamma\) be a limit ordinal. Show that the following are 
equivalent:
\begin{enumerate}
\renewcommand{\theenumi}{\alph{enumi}}
\item
\(\forall \alpha,\beta < \gamma [\alpha + \beta < \gamma]\).

\item
\(\forall \alpha < \gamma [ \alpha + \gamma = \gamma]\).

\item
\(\forall X \subseteq \gamma
  [\type(X) = \gamma \lor \type(\gamma\setminus X) = \gamma]\).

\item
\(\exists\delta[\gamma = \omega^\delta]\).
\end{enumerate}
\end{lexcopy}
\unfinished

\begin{lexcopy}{I.9.54}{61}
Prove the Cantor Normal Form Theorem: Each ordinal
\(\alpha > 0\) can be represented uniquely in the form:
\begin{equation*}
\alpha =\omega^{\beta_1}\cdot n_1 + \cdots + \omega^{\beta_k}\cdot n_k .
\end{equation*}
where \(k,n_1m\ldots,n_k \in \omega \setminus \{0\}\)
and
\(\alpha \geq \beta_1 > \cdots > \beta_k\).

If \(\alpha = \beta_1\) then \(\alpha = \omega^\alpha\) and \(k = n_1 = 1\).
\end{lexcopy}
\unfinished

\begin{lexcopy}{I.9.55}{62}
Let \(F(\alpha, \beta)\) be the set of all functions $f$ from
\(\beta\) to \(\alpha\) such
that \(\{\xi < \beta: f(\xi) \neq 0\}\) finite.
For \(f,g \in F(\alpha,\beta)\) with \(f \neq g\), define \(f \triangleleft g\)
iff \(f(\xi) < g(\xi)\), where \(\xi\) is the largest ordinal such
that \(f(\xi) \neq g(\xi)\). Prove that
\(\triangleleft\) well-orders \(F(\alpha, \beta)\) in type \(\alpha^\beta\).
\end{lexcopy}
\unfinished
