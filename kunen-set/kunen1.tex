% 

\section{Exercises} % pages 111-115

%%%%%%%%%%%%%%%%%
\begin{enumerate}
%%%%%%%%%%%%%%%%%

%%%%%%% 1
\begin{excopy}
Write a formula expressing 
 \(z = \lrangle{\lrangle{x,y},\lrangle{v,w}}\)
using just \(\in\) and \(=\).
\end{excopy}

\unfinished

%%%%%%% 
\begin{excopy}
Show that \(\alpha < \beta\) implies that 
\(\gamma + \alpha < \gamma + \beta\) 
and
\(\alpha + \gamma \leq \beta + \gamma\).
Give an example to show that the \(\leq\)
cannot be replaced by $<$.
Also show:
\begin{equation*}
\alpha < \beta \ra \exists! \delta (\alpha + \delta = \beta).
\end{equation*}
\end{excopy}

Following Definition~7.17
\begin{equation*}
\gamma\times\{0\} \cup \alpha\times\{1\}
\subsetneq
\gamma\times\{0\} \cup \beta\times\{1\}.\
\end{equation*}
and the corresponding relations
\(R_{\gamma+\alpha} \subsetneq R_{\gamma+\beta}\).
By Theorem~6.3, the unique identity isomorphism of 
\(\gamma+\alpha\) to itself as a proper subset gives 
\(\gamma + \alpha < \gamma + \beta\),

By negation, \(\beta + \gamma < \alpha + \gamma\).
Consider the injectin isomorphism from 
\(\beta\times\{0\} \cup \gamma\times\{1\}\)
into a proper subset of 
\(\alpha\times\{0\} \cup \gamma\times\{1\}\).

The \(\leq\) is needed for \(0+\omega = 1+\omega\).
\unfinished

\end{enumerate}

