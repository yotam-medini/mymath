% 

%%%%%%%%%%%%%%%%%%%%%%%%%%%%%%%%%%%%%%%%%%%%%%%%%%%%%%%%%%%%%%%%%%%%%%%%
%%%%%%%%%%%%%%%%%%%%%%%%%%%%%%%%%%%%%%%%%%%%%%%%%%%%%%%%%%%%%%%%%%%%%%%%
%%%%%%%%%%%%%%%%%%%%%%%%%%%%%%%%%%%%%%%%%%%%%%%%%%%%%%%%%%%%%%%%%%%%%%%%
\chapterTypeout{Background Material}

%%%%%%%%%%%%%%%%%%%%%%%%%%%%%%%%%%%%%%%%%%%%%%%%%%%%%%%%%%%%%%%%%%%%%%%%
%%%%%%%%%%%%%%%%%%%%%%%%%%%%%%%%%%%%%%%%%%%%%%%%%%%%%%%%%%%%%%%%%%%%%%%%
\section{Notes}

\begin{manuallemma}{I.7.8}
\(ON\) is a transitive class. That is, if \(\alpha \in ON\)
and \(z \in \alpha\),
then \(z \in ON\).
\end{manuallemma}
The proof says ``it is easy to check that $z$ is a transitive set.''
\\
Here is how. Let \(y \in z\), we need to show \(y \subseteq z\).
By transitivity, \(z \subseteq \alpha\) and so \(y \in \alpha\).
Again by transitivity, \(y \subseteq \alpha\).
Let \(x\in y\), thus \(x \in \alpha\) and actually \(x,y,z \in \alpha\).
By orering of \(\alpha\), we have \(x\in y \in z\) and hence \(x\in z\)
giving the desired \(y \subseteq z\).

%%%%%%%%%%%%%%%%%%%%%%%%%%%%%%%%%%%%%%%%%%%%%%%%%%%%%%%%%%%%%%%%%%%%%%%%
\section{Old Edition Exercises } % pages 111-115


%%%%%%%%%%%%%%%%%
\begin{enumerate}
%%%%%%%%%%%%%%%%%

%%%%%%% 1
\begin{excopy}
Write a formula expressing 
 \(z = \lrangle{\lrangle{x,y},\lrangle{v,w}}\)
using just \(\in\) and \(=\).
\end{excopy}

\unfinished

%%%%%%% 
\begin{excopy}
Show that \(\alpha < \beta\) implies that 
\(\gamma + \alpha < \gamma + \beta\) 
and
\(\alpha + \gamma \leq \beta + \gamma\).
Give an example to show that the \(\leq\)
cannot be replaced by $<$.
Also show:
\begin{equation*}
\alpha < \beta \ra \exists! \delta (\alpha + \delta = \beta).
\end{equation*}
\end{excopy}

Following Definition~7.17
\begin{equation*}
\gamma\times\{0\} \cup \alpha\times\{1\}
\subsetneq
\gamma\times\{0\} \cup \beta\times\{1\}.\
\end{equation*}
and the corresponding relations
\(R_{\gamma+\alpha} \subsetneq R_{\gamma+\beta}\).
By Theorem~6.3, the unique identity isomorphism of 
\(\gamma+\alpha\) to itself as a proper subset gives 
\(\gamma + \alpha < \gamma + \beta\),

By negation, \(\beta + \gamma < \alpha + \gamma\).
Consider the injectin isomorphism from 
\(\beta\times\{0\} \cup \gamma\times\{1\}\)
into a proper subset of 
\(\alpha\times\{0\} \cup \gamma\times\{1\}\).

The \(\leq\) is needed for \(0+\omega = 1+\omega\).
\unfinished

\end{enumerate}

%%%%%%%%%%%%%%%%%%%%%%%%%%%%%%%%%%%%%%%%%%%%%%%%%%%%%%%%%%%%%%%%%%%%%%%%
\section{Current Edition Exercises } % pages 111-115

\begin{lexcopy}{I.4.18}{24}
Derive \(\forall y[y \notin y]\) from the Axioms of Comprehension and
Foundation. Don't use the Pairing or Extensionality Axioms. Then find a
two element model for Foundation, Extensionality, Pairing, and Union, plus
\(\exists y\forall x [x \in y]\) (so, of course, \(y \in y\)).
\end{lexcopy}

By negation \(y\in y\).
By Comprehension, let \(\varphi(z)\) be \(z=y\). Apply
\begin{equation*}
  \exists x \forall z (z \in x \wedge \varphi(z))
\end{equation*}
gives \(x:=\{y\}\). By Foundation, $y$ being the (only) minimal
element for which
\mbox{\(\lnot\exists z(z\in y \wedge z \in \{y\})\)}
should hold, but \(z=y\) gives contradiction.

Let the model have two ``sets'': \(\mu\), \(\nu\).
with the following ``\(\in\)'' relations:
\(\mu\in\nu\), \(\nu\in\nu\).

\begin{lexcopy}{I.6.23}{31}
Define the relation $R$ on \(\N \times \N\) by saying: \((x_0,x_1)R(y_0,y_1)\)
iff \(x_0 \leq y_0\) and \(x_1 \leq y_1\)
and \((x_0,x_1)\neq (y_0,y_1)\).
Show that $R$ is well-founded
on \N\ but not a total order.
Show that every non-empty \(X \subseteq \N\times\N\) contains
only finitely many $R$-minimal elements, and, for each \(k \in\N\),
there is such an
$X$ with exactly $k$ $R$-minimal elements.
\end{lexcopy}

Let non-empty \(A \subset \N\times\N\).
Let \(P = \{n\in\N: \exists m((n,m)\in A\).
For each \(n\in P\) let \(Q_n = \{m\in\N: (n,m)\in A\}\).
Let \(p = \min(P)\) and \(q = \min(Q_p)\).
Clearly \((p,q)\in A\) is minimal in $A$.

Not total order since
\((0,1)\not{R}(1,0)\) and
\((1,0)\not{R}(0,1)\).

Given $k$, \(A=\{(n,k-n-1)\in\N\times\N: 0\leq n < k\}\)
has $k$ elements all of which are minimal.

\begin{lexcopy}{I.6.26}{31}
Show that \(\in\) is a well-order of $3$.
\end{lexcopy}

The set 
\(3 = \{\emptyset, \{\emptyset\}, \{\emptyset, \{\emptyset\}\}\}\).
has $3$ elements. Clearly for any two elements \(x,y\in 3\)
we have \(x=y \lor (x\in y \lor y \in x)\).
and any non empty subset has a minimal element.

\begin{lexcopy}{I.6.29}{33}
Show that the Axiom of Foundation implies that the successor
function $S$ is $1-1$ on $V$.
\end{lexcopy}
Assume \(x \neq y\). % , \wlogy\ \(b\in x \setminus y\).
If by negation \(S(x) = S(y)\), then % \(b = \{y\}\).
\(x \in S(y)\) and so \(x\in y\).
Similarly \(y \in x\). Thus \(\{x, y\}\) has no ``\(\in\)-minimal''
element contradiction to to the Axiom of Foundation.

\begin{lexcopy}{I.7.2}{34}
Assuming the Axiom of Foundation, show that every 
nonempty transitive set contains $0$ and show that every non-singleton
transitive
set contains $1$. Then show that $1$ is the only one-element transitive set,
and $2$
is the only two-element transitive set.
\end{lexcopy}

Let \(T\neq\emptyset\) be transitive set.
By Axiom of Foundation there is an
element \(e\in T\) if by negation \(x\in e\),
by transitivity \(x\in T\) contradicting the existance of $e$,
thus \(e = \emptyset \in T\).
Clearly \(1= \{\emptyset\}\) is the only singleton transitive set.

If $T$ is non-singleton, let \(S = T \setminus \{\emptyset\} \neq \emptyset\).
Since $S$ is not transitive, let \(u \in S\)
be such that \(u \not\subseteq S\).
Since \(u\subseteq T\), we have \(\emptyset \in u\)
that is \(1 = \{\emptyset\} \in u\).

If $T$ is two element set,
then \(S = T \setminus \{\emptyset\}=\{u\}\) is a singleton.
Also \(u \subset T = \{\emptyset, u\}\).
But \(u \neq \{\emptyset\}\),
so $u$ must be singleton conisting of some element of $T$.
Since \(u\notin u\) its element must be \(\emptyset\),
namely \(u = \{\emptyset\}\) and
\(T = \{\emptyset, \{\emptyset\}\} = 2\).

\begin{lexcopy}{I.7.25}{38}
  Assuming the Axiom of Foundation, $z$ is an ordinal iff $z$ is
  a transitive set and \(\in\) satisfies trichotomy on $z$.
\end{lexcopy}

If $z$ is an ordinal, then by definition it is transitive
and since  \(\in\) is an order, it satisfies trichotomy.

Conversely, assume that  $z$ is  a transitive set
and \(\in\) is trichotomy on $z$.
Let \(S\subseteq z\) be some non-empty set. By Axiom of Foundation,
we have \(y \in S\) such that \(\forall z\in y (z\notin S)\).
If by negation $y$ uis not \(\in\)-minimal in $S$,
then \(\exists m\in s(m\in y)\) is a contradiction.
Ths \(\in\) well orders $z$ and thus $z$ is an ordinal.


\begin{lexcopy}{I.7.26}{39}
  Assuming the Axiom of Foundation, $z$ is an ordinal iff $z$ is
a transitive set and all elements of $z$ are transitive sets.
\end{lexcopy}

If $z$ is an ordinal, the by definition it is transitive,
its elements are ordinals by Lemma~ I.7.8 thus transitive.

Conversely, we need to show that \(\in\) is well-order on \(\alpha\).
Assume by negation \(\in\) is not a total order.
Let \(x\in z\) be \(\in\)-minimal such that it is non-comparable
with some elemet in $z$.
Let \(y\in z\) be \(\in\)-minimal such that it is non-comparable
with $x$.
Now if \(u\in y\) then \(u\in x\) hence \(y \subseteq x\).
Similarly, if \(u\in x\) then \(u\in y\) hence \(x \subseteq y\),
thus \(x = y\) is a contradiction, and \(\in\) is a total-order on $z$
and by the Axiom of Foundation is well-order. Hence $z$ is an ordinal.

\begin{lexcopy}{I.7.27}{39}
  Show, in \(BST^{-}\), that if $A$,$B$ are finite,
  then \(A\cup B\) and \(A\times B\)
are finite, and \(\mathcal{P}(A)\) exists and is finite.
\end{lexcopy}
Here without Axiom of Foundation.

By induction on ordinal $n$ for which \(B \preccurlyeq n\).
Let \(B \preccurlyeq n + 1 = \{0, 1, \ldots n - 1\}\).
Pick the (last) element \(b\in B\) that is mapped to \(n-1\),
and let \(B' = B \setminus \{b\}\), so now \(B \preccurlyeq n\).

\(\mathbf{A\cup B}\):
Let \(f'\) be a 1-1 function \(A \cup B' \to n - 1\)
If \(b\in A\) the the same \(f'\) maps  \(A \cup B \to n - 1\),
Otherwise we define \(f: A \cup B \to n\), by
\(f(b) = n - 1\) and \(f(b') = f'(b')\) for \(b'\in B'\).

\(\mathbf{A\times B}\):
Let \(g: A \to m\) be 1-1 mapping, and let \(a_i = g^{-1}(i)\)
for \(0 \leq i < m\).
Let \(f'\) be a 1-1 function \(A \times B' \to M\)
It is easy to have 1-1- map
\(A \times B \to (A \times B') \cup (\{x\} \times B)\),
we can pick any \(x\notin A\) for example \(\{A\}\).
The finiteness is now derived from the previous result.

\(\mathbf{P(A)}\): Similarly we
set the disjoint union \(A = A' \cup \{a\}\).
Given a 1-1 map \(f': P(A') \to m\)
Define
\begin{equation*}
f: P(A) \to (\{0\}\times m) \cup (\{1\}\times m)
\end{equation*}
by:
\begin{equation*}
f(B) =
\left\{
\begin{array}{ll}
(\{0\}, f'(B)) & \textrm{if}\; B \subseteq A' \\
(\{1\}, f'(B \setminus \{a\})) & \textrm{if}\; B \not\subseteq A'
\end{array}
\right.
\end{equation*}

\begin{lexcopy}{I.9.6}{46}
  Derive the Axioms of Infinity and Replacement from (2) of Lemma I.9.5.
\end{lexcopy}
\textbf{Infinity:} Define $R$ by \(xRy\) iff \(S(x) = y\)
\\
\textbf{Replacement:} Define $R$ by \(xRy\) iff
\(\forall z(\varphi(x, z) \to y = z)\).

\begin{lexcopy}{I.9.8}{46}
  If $R$ is well-founded on $A$, then $R$ is acyclic on $A$; hence, \(R^*\)
  is a strict partial order on $A$.
  Conversely, if $A$ is finite and $R$ is acyclic on $A$,
then $R$ is well-founded on $A$.
\end{lexcopy}
If there was an $R$-cylce there would be no $R$ minimal element in $A$.
The definition of \(R^*\) via $R$-path gives transitivity, namely
\begin{equation*}
\forall x,y,z\,\left(xRy \wedge yRz \;\to\; xRz\right).
\end{equation*}

Conversely, each $R$-path must be finite for there are no cycles
and $A$ is finite. Thus there exists $R$-minimal element
(or \(A=\emptyset\)).

\begin{lexcopy}{I.9.27}{54}
  Assume that each element of \R\ has rank \(\omega\);
  this will be true
using our Definition 1.14-2. Let \(\mu\) be Lebesgue measure on \R. Prove that
\(\R,\mu \in WF\), with \(\textrm{rank}(\R) = \omega + 1\)
and \(\textrm{rank}(\mu) = \omega + 4\).
\end{lexcopy}
\unfinished

\begin{lexcopy}{I.9.39}{59}
Assume that $R$ is well-founded and set-like on $A$. Then
\(\textrm{rank}_{A,R}\) is 1-1 iff \(R^*\) is a total order of $A$.
\end{lexcopy}
\unfinished

\begin{lexcopy}{1.9.40}{59}
If \(R_1 \subseteq R_2\) are both well-founded and set-like on $A$, then
\(\textrm{rank}_{A,R}(y) \leq \textrm{rank}_{A,R}(y)\)
for all \(y \in A\). Also, 
\(\textrm{rank}_{A,R}(y) = \textrm{rank}_{A,R}(y)\)
if \(R_1 \subseteq R_2 \subseteq (R_1)^*\).
\end{lexcopy}
\unfinished

\begin{lexcopy}{1.9.41}{59}
Assume that $R$ is well-founded and set-like on $A$, and that $R$
is a transitive relation on $A$.
Then \(\textrm{mos}_{A,R}(a) = \textrm{rank}_{A,R}(a)\) for all \(a \in A\).
\end{lexcopy}
\unfinished

\begin{lexcopy}{I.9.42}{59}
Assume that $R$ well-orders the set $A$. Then the three 
functions \(\textrm{mos}_{A,R}\), \(\textrm{rank}_{A,R}\),
and the isomorphism of $A$ onto an ordinal described in
Theorem~1.8.2, are all the same.
\end{lexcopy}
\unfinished

\begin{lexcopy}{1.9.43}{59}
Work in \(ZF^-\). Let $K$ be any class and assume that every
subset of $K$ is a member of $K$. Prove that \(WF \subseteq K\).
\end{lexcopy}
\unfinished

\begin{lexcopy}{1.9.44}{59}
For any relation $R$ on a class $A$, and \(a \in A\): $R$
is well-founded on \(\textrm{pred}_{A,R^*}(a)\) iff $R$
is well-founded on \(\{a\} \cup \textrm{pred}_{A,R^*}(a)\).
\end{lexcopy}
\unfinished

