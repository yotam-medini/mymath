% $Id: langalg.tex,v 1.4 2001/05/04 12:24:45 yotam Exp yotam $
\documentclass[12pt]{book}
\usepackage{fullpage}
\usepackage{amsmath}
\usepackage{amssymb}
\usepackage{amsthm}
% \usepackage{amsthm}
\usepackage{makeidx}
\makeindex % enable

\usepackage{multicol,supertabular}

\setlength{\parindent}{0pt}

% \usepackage{amsmath}

\usepackage{enumerate}

% 'Inspired' by:
%% This is file `uwamaths.sty',
%%%     author   = "Greg Gamble",
%%%     email     = "gregg@csee.uq.edu.au (Internet)",

\makeatletter
\def\DOTSB{\relax}
\def\dotcup{\DOTSB\mathop{\overset{\textstyle.}\cup}}
 \def\@avr#1{\vrule height #1ex width 0pt}
 \def\@dotbigcupD{\smash\bigcup\@avr{2.1}}
 \def\@dotbigcupT{\smash\bigcup\@avr{1.5}}
 \def\dotbigcupD{\DOTSB\mathop{\overset{\textstyle.}\@dotbigcupD%
                               \vphantom{\bigcup}}}

\def\dotbigcupT{\DOTSB\smash{\mathop{\overset{\textstyle.}\@dotbigcupT%
                              \vphantom{\bigcup}}}%
                       \vphantom{\bigcup}\@avr{2.0}}
\def\dotbigcup{\mathop{\mathchoice{\dotbigcupD}{\dotbigcupT}
                                  {\dotbigcupT}{\dotbigcupT}}}
\let\disjunion\dotcup
\let\Disjunion\dotbigcup
\makeatother

\usepackage{amsmath}
\usepackage{amssymb}
% \usepackage{eucal}
\usepackage{mathrsfs}

% \usepackage{fullpage}

\usepackage{geometry}
\geometry{a4paper, left=2cm, right=2cm, top=2cm, bottom=2cm, includeheadfoot}

\setlength{\parindent}{0pt}
\setlength{\parskip}{6pt}


% are we in pdftex ????
\ifx\pdfoutput\undefined % We're not running pdftex
\else
\RequirePackage[colorlinks,hyperindex,plainpages=false]{hyperref}
\def\pdfBorderAttrs{/Border [0 0 0] } % No border arround Links
\fi

% \usepackage{fancyheadings}
\usepackage{fancyhdr}
\usepackage{pifont}

\pagestyle{fancy}
% \addtolength{\headwidth}{\marginparsep}
% \addtolength{\headwidth}{\marginparwidth}
%  \addtolength{\textheight}{2pt}

\newcommand{\ineqjton}{\overset{1\leq i,j \leq n}{i \neq j}}
\newcommand{\srightmark}{\rightmark}
\newcommand{\sfbfpg}{\sffamily\bfseries{\thepage}}
  \newcommand{\symenvelop}{%
     {\nullfont\ }\relax\lower.2ex\hbox{\large\Pisymbol{pzd}{41}}}
% \renewcommand{\chaptermark}[1]{\markboth{\thechapter.\ #1}}

\iffalse
% \lhead[\fancyplain{}{{\sfbfpg}}]{\fancyplain{}\bfseries\srightmark}
\lhead[\fancyplain{}{{\sfbfpg}}]{\fancyplain{}\sl\srightmark}
% \rhead[\fancyplain{}\bfseries\leftmark]{\fancyplain{}{{\sfbfpg}}}
\rhead[\fancyplain{}\sl\leftmark]{\fancyplain{}{{\sfbfpg}}}
\lfoot{\today}
\cfoot{Yotam Medini \copyright}
  \newcommand{\symenvelop}{%
     {\nullfont a}\relax\lower.2ex\hbox{\large\Pisymbol{pzd}{41}}}
\rfoot{\symenvelop\ \texttt{yotam.medini@gmail.com}}

\renewcommand{\headrulewidth}{0.4pt}
\renewcommand{\footrulewidth}{0.4pt}
\fi

\setlength{\headheight}{16pt}
\fancyplain{plain}{%
 \fancyhf{}
 \fancyhead[LE,RO]{\fancyplain{}{{\sfbfpg}}}
 \fancyhead[RE,LO]{\sl\leftmark}
 \fancyfoot[L]{\today}
 \fancyfoot[C]{Yotam Medini \copyright}
 \fancyfoot[R]{\symenvelop\ \texttt{yotam.medini@gmail.com}}
 \renewcommand{\headrulewidth}{0.4pt}
 \renewcommand{\footrulewidth}{0.4pt}
}

% \usepackage{amstex}
% \usepackage{amsmath}
% \usepackage{amssymb}
\usepackage{amsthm}
\usepackage{bm}
\usepackage{makeidx}
\makeindex % enable

% 'Inspired' by:
%% This is file `uwamaths.sty',
%%%     author   = "Greg Gamble",
%%%     email     = "gregg@csee.uq.edu.au (Internet)",

\makeatletter
\def\DOTSB{\relax}
\def\dotcup{\DOTSB\mathop{\overset{\textstyle.}\cup}}
 \def\@avr#1{\vrule height #1ex width 0pt}
 \def\@dotbigcupD{\smash\bigcup\@avr{2.1}}
 \def\@dotbigcupT{\smash\bigcup\@avr{1.5}}
 \def\dotbigcupD{\DOTSB\mathop{\overset{\textstyle.}\@dotbigcupD%
                               \vphantom{\bigcup}}}

\def\dotbigcupT{\DOTSB\smash{\mathop{\overset{\textstyle.}\@dotbigcupT%
                              \vphantom{\bigcup}}}%
                       \vphantom{\bigcup}\@avr{2.0}}
\def\dotbigcup{\mathop{\mathchoice{\dotbigcupD}{\dotbigcupT}
                                  {\dotbigcupT}{\dotbigcupT}}}
\let\disjunion\dotcup
\let\Disjunion\dotbigcup
\makeatother


\newcommand{\half}{\ensuremath{\frac{1}{2}}}



\newcommand{\C}{\ensuremath{\mathbb{C}}} % The Complex set
\newcommand{\aded}{\ensuremath{\textrm{a.e.}}} % almost everyehere
\newcommand{\chhi}{\raise2pt\hbox{\ensuremath\chi}}           %raise the chi
\newcommand{\calA}{\ensuremath{\mathcal{A}}}
\newcommand{\calB}{\ensuremath{\mathcal{B}}}
\newcommand{\calE}{\ensuremath{\mathcal{E}}}
\newcommand{\calF}{\ensuremath{\mathcal{F}}}
\newcommand{\calG}{\ensuremath{\mathcal{G}}}
\newcommand{\calM}{\ensuremath{\mathcal{M}}}
\newcommand{\calR}{\ensuremath{\mathcal{R}}}
\newcommand{\eqdef}{\ensuremath{\stackrel{\mbox{\upshape\tiny def}}{=}}}
\newcommand{\frakB}{\ensuremath{\mathfrak{B}}}
\newcommand{\frakC}{\ensuremath{\mathfrak{C}}}
\newcommand{\frakF}{\ensuremath{\mathfrak{F}}}
\newcommand{\frakG}{\ensuremath{\mathfrak{G}}}
\newcommand{\frakI}{\ensuremath{\mathfrak{I}}}
\newcommand{\frakM}{\ensuremath{\mathfrak{M}}}
\newcommand{\scrA}{\ensuremath{\mathscr{A}}}
\newcommand{\scrB}{\ensuremath{\mathscr{B}}}
\newcommand{\scrD}{\ensuremath{\mathscr{D}}}
\newcommand{\scrF}{\ensuremath{\mathscr{F}}}
\newcommand{\scrN}{\ensuremath{\mathscr{N}}}
\newcommand{\scrP}{\ensuremath{\mathscr{P}}}
\newcommand{\scrQ}{\ensuremath{\mathscr{Q}}}
\newcommand{\scrR}{\ensuremath{\mathscr{R}}}
\newcommand{\scrT}{\ensuremath{\mathscr{T}}}
\newcommand{\Lp}[1]{\ensuremath{\mathbf{L}^{#1}}} % Lp space
\newcommand{\N}{\ensuremath{\mathbb{N}}} % The Natural Set
\newcommand{\bbP}{\ensuremath{\mathbb{P}}} % Some partially ordered set
\newcommand{\Q}{\ensuremath{\mathbb{Q}}} % The Rational set
\newcommand{\R}{\ensuremath{\mathbb{R}}} % The Real Set
\newcommand{\T}{\ensuremath{\mathbb{T}}} % The Thorus [-pi,\pi)
\newcommand{\Z}{\ensuremath{\mathbb{Z}}} % The Integer Set
\newcommand{\intR}{\int_{-\infty}^{\infty}} % Integral over the reals
\newcommand{\posthat}[1]{#1{\,\hat{}\,}}

% sequences
\newcommand{\seq}[2]{\ensuremath{#1_1,\ldots,#1_{#2}}}
\newcommand{\seqn}[1]{\seq{#1}{n}}
\newcommand{\seqan}{\seq{a}{n}}
\newcommand{\seqxn}{\seq{x}{n}}
\newcommand{\seqalphn}{\seq{\alpha}{n}}

\newcommand{\mset}[1]{\ensuremath{\{#1\}}}


%%%%%%%%%%%%
%% math op's
\newcommand{\Alt}{\mathop{\rm Alt}\nolimits}
\newcommand{\Ang}{\mathop{\rm Ang}\nolimits}
\newcommand{\Arg}{\mathop{\rm Arg}\nolimits}
\newcommand{\co}{\mathop{\rm co}\nolimits}
\newcommand{\conv}{\mathop{\rm conv}\nolimits}
\newcommand{\diam}{\mathop{\rm diam}\nolimits}
\newcommand{\dom}{\mathop{\rm dom}\nolimits}
% \newcommand{\dim}{\mathop{\rm dim}\nolimits}
% \newcommand{\esssup}{\mathop{\rm ess\ sup}\nolimits}
\DeclareMathOperator*{\esssup}{ess\,sup}
\newcommand{\ext}{\mathop{\rm ext}\nolimits}
\newcommand{\Id}{\mathop{\rm Id}\nolimits}
\newcommand{\Image}{\mathop{\rm Im}\nolimits}
\newcommand{\Ind}{\mathop{\rm Ind}\nolimits}
\newcommand{\Lip}{\mathop{\rm Lip}\nolimits}
\newcommand{\lip}{\mathop{\rm lip}\nolimits}
\newcommand{\percB}{
  \mathbin{\ooalign{$\hidewidth\%\hidewidth$\cr$\phantom{+}$}}}
\newcommand{\bres}[2]{\ensuremath{#1 \percB #2}}

\newcommand{\Ker}{\mathop{\rm Ker}\nolimits}
\newcommand{\rank}{\mathop{\rm rank}\nolimits}
\newcommand{\rng}{\mathop{\rm rng}\nolimits}
\newcommand{\Res}{\mathop{\rm Res}\nolimits}
\newcommand{\supp}{\mathop{\rm supp}\nolimits}
\newcommand{\vol}{\mathop{\rm vol}\nolimits}
\newcommand{\vspan}{\mathop{\rm span}\nolimits}

% I wish this was more standardized
\renewcommand{\Re}{\mathop{\bf Re}\nolimits}
\renewcommand{\Im}{\mathop{\bf Im}\nolimits}

\newcommand{\inter}[1]{\ensuremath{#1^{\circ}}}  % interior
\newcommand{\closure}[1]{\ensuremath{\overline{#1}}} % closure
\newcommand{\boundary}[1]{\ensuremath{\partial #1}} % closure


\newcommand{\ich}[1]{(\textit{#1})}
\newcommand{\itemch}[1]{\item[\ich{#1}]}
\newcommand{\itemdim}{\item[\(\diamond\)]}

% Special names
\newcommand{\Cech}{\u{C}ech}

\author{Yotam Medini}


%%%%%%%%%%%
%% Theorems
%%
\makeatletter
\@ifclassloaded{book}{
 \newtheorem{thm}{Theorem}[chapter]
 \newtheorem{cor}[thm]{Corollary}
 \newtheorem{lem}[thm]{Lemma}
 \newtheorem{llem}[thm]{Local Lemma}
 \newtheorem{lthm}[thm]{Local Theorem}
 % \newtheorem{quotecor}{Corollary}
 % \newtheorem{quotelem}{Lemma}[section]
 \newtheorem{quotethm}{Theorem}[chapter]
}{}
\makeatother
\newtheorem{Def}{Definition}

\newtheorem{manualtheoreminner}{Theorem}
\newenvironment{manualtheorem}[1]{%
  \renewcommand\themanualtheoreminner{#1}%
  \manualtheoreminner
}{\endmanualtheoreminner}

\newtheorem{manuallemmainner}{Lemma}
\newenvironment{manuallemma}[1]{%
  \renewcommand\themanuallemmainner{#1}%
  \manuallemmainner
}{\endmanuallemmainner}

\newcommand{\loclemma}{Lemma}


% \newcommand{\proofend}{\(\bullet\)}
% \newcommand{\proofend}{\hfill\(\blacksquare\)}
\newcommand{\proofend}{\hfill\(\Box\)}
\newenvironment{thmproof}
{\textbf{Proof.}}
{\proofend}

\newcommand{\chapterTypeout}[1]{\typeout{#1} \chapter{#1}}
\newcommand{\sectionTypeout}[1]{\typeout{#1} \section{#1}}

% abbreviations, ensuremath
\newcommand{\fx}{\ensuremath{f(x)}}
\newcommand{\gx}{\ensuremath{g(x)}}
\newcommand{\lrangle}[1]{\ensuremath{\left\langle #1 \right\rangle}}
\newcommand{\lrbangle}[1]{\ensuremath{\left\langle #1 \right\rangle}}
\newcommand{\M}{\ensuremath{\mathfrak{M}}}
\newcommand{\mldots}{\ensuremath{\ldots}}
\newcommand{\salgebra}{\(\sigma\)-algebra}
\newcommand{\swedge}{\;\wedge\;}
\newcommand{\wlogy}{without loss of generality}
\newcommand{\Wlogy}{Without loss of generality}
\newcommand{\twopii}{\ensuremath{2\pi i}}
\newcommand{\dtwopii}{\ensuremath{\frac{1}{\twopii}}}

% https://tex.stackexchange.com/
% questions/22252/how-to-typeset-function-restrictions
\newcommand\restr[2]{\ensuremath{% we make the whole thing an ordinary symbol
  \left.\kern-\nulldelimiterspace % automatically resize the bar with \right
  #1 % the function
  \vphantom{\big|} % pretend it's a little taller at normal size
  \right|_{#2} % this is the delimiter
  }}

\newenvironment{excopyOLD}
{\item\begin{minipage}[t]{.8\textwidth}\footnotesize}
{\smallskip\hrule\end{minipage}}

\newenvironment{excopy}
{\item % \relax
 \begin{list}{}{
 \setlength{\topsep}{0pt}
 \setlength{\partopsep}{0pt}
 \setlength{\itemsep}{0pt}
 \setlength{\parsep}{0pt}
 \setlength{\leftmargin}{0pt}
 \setlength{\rightmargin}{20pt}
 \setlength{\listparindent}{0pt}
 \setlength{\itemindent}{0pt}
 % \setlength{\labelsep}{0pt}
 \setlength{\labelwidth}{0pt}
 \footnotesize
 }
 \item
}
{\par
 % {\nullfont 0}
 \hrulefill
 \end{list}
}


\title{
 Notes and Solutions to Exercises\\
 for\\
 ``Algebra'' \quad by\quad  Serge Lang}
\author{Yotam Medini\\\texttt{yotam\_medini@yahoo.com}}

\newcommand{\Zm}[1]{\Z/#1\Z} % The Cyclic group

% Trivial group
\newcommand{\eG}{\ensuremath{\{e\}}}

\newcommand{\UNFINISHED}{\large\textbf{UNFINISHED!}}

% \newcommand{\disjunion}{\.\cup}      % Some use \sqcup or \uplus
% \newcommand{\Disjunion}{\.\bigsqcup} % Some use \bigsqcup or \biguplus
% \newcommand{\disjunion}{{\bigsqcup}}
% \newcommand{\Disjunion}{\bigsqcup}

\def\Aut{\mathop{\rm Aut}\nolimits}
\def\card{\mathop{\rm card}\nolimits}
\def\Ch{\mathop{\rm Ch}\nolimits}
\def\Id{\mathop{\rm Id}\nolimits}
\def\id{\mathop{\rm id}\nolimits}
\def\Inn{\mathop{\rm Inn}\nolimits}
\def\Irr{\mathop{\rm Irr}\nolimits}
\def\Ker{\mathop{\rm Ker}\nolimits}
\def\Map{\mathop{\rm Map}\nolimits}
\def\Stab{\mathop{\rm Stab}\nolimits}
\def\subnormal{\vartriangleleft}

% \renewenvironment{excopy}
% {\begin{minipage}[t]{.8\textwidth}\footnotesize}
% {\smallskip\hrule\end{minipage}}


\newcounter{myenumi}
\newenvironment{myenumerate}
{\begin{enumerate}
 \setcounter{enumi}{\themyenumi}
}
{\setcounter{myenumi}{\theenumi}
 \end{enumerate}}

% End of proof
% \newcommand{\eop}{{\small\quad\(\square\)}}

% \newtheorem{thm}{Theorem}[chapter]
% \newtheorem{cor}[thm]{Corollary}
% \newtheorem{lem}[thm]{Lemma}
% \newtheorem{llem}[thm]{Local Lemma}


\begin{document}
\maketitle
\newpage
\tableofcontents
\newpage

%%%%%%%%%%%%%%%%%%%%%%%%%%%%%%%%%%%%%%%%%%%%%%%%%%%%%%%%%%%%%%%%%%%%%%%%
%%%%%%%%%%%%%%%%%%%%%%%%%%%%%%%%%%%%%%%%%%%%%%%%%%%%%%%%%%%%%%%%%%%%%%%%
%%%%%%%%%%%%%%%%%%%%%%%%%%%%%%%%%%%%%%%%%%%%%%%%%%%%%%%%%%%%%%%%%%%%%%%%
\chapter*{Introduction}

This document is a companion to \cite{Lan94}.

%%%%%%%%%%%%%%%%%%%%%%%%%%%%%%%%%%%%%%%%%%%%%%%%%%%%%%%%%%%%%%%%%%%%%%%%
%%%%%%%%%%%%%%%%%%%%%%%%%%%%%%%%%%%%%%%%%%%%%%%%%%%%%%%%%%%%%%%%%%%%%%%%
\section*{Notation}

For each natural \(n\in\N\) we define
\begin{equation*}
\N_n \eqdef \{m\in\N: 1\leq m \leq n\} \qquad
\Z_n \eqdef \{m\in\Z: 0\leq m < n\}.
\end{equation*}

We borrow \textbf{C}-programming language modulo operator 
to define the residue function
\begin{equation*}
\bres{n}{d} = n - n \left\lfloor \frac{n}{d} \right\rfloor
\end{equation*}


%%%%%%%%%%%%%%%%%%%%%%%%%%%%%%%%%%%%%%%%%%%%%%%%%%%%%%%%%%%%%%%%%%%%%%%%
%%%%%%%%%%%%%%%%%%%%%%%%%%%%%%%%%%%%%%%%%%%%%%%%%%%%%%%%%%%%%%%%%%%%%%%%
%%%%%%%%%%%%%%%%%%%%%%%%%%%%%%%%%%%%%%%%%%%%%%%%%%%%%%%%%%%%%%%%%%%%%%%%
\chapter{Groups}

%%%%%%%%%%%%%%%%%%%%%%%%%%%%%%%%%%%%%%%%%%%%%%%%%%%%%%%%%%%%%%%%%%%%%%%%
%%%%%%%%%%%%%%%%%%%%%%%%%%%%%%%%%%%%%%%%%%%%%%%%%%%%%%%%%%%%%%%%%%%%%%%%
\section{Notes}

%%%%%%%%%%%%%%%%%%%%%%%%%%%%%%%%%%%%%%%%%%%%%%%%%%%%%%%%%%%%%%%%%%%%%%%%
\subsection{Fixed proof of 5.5}

In an old edition:

Page~33 in the proof of Theorem~5.5.

The last paragraph of the proof deals with the case
in which the orbit of \(\langle\sigma\rangle\) has \(\geq3\) elements.
With the defined \(\tau = [krs]\) the text claims that
with \(\sigma' =  \tau\sigma\tau^{-1}\sigma^{-1}\) we have
\(\sigma'(i) = i\).

If we set \(\sigma = [ijkrs]\) then \(\sigma'(i)\neq i\) \emph{contrary}
to what is claimed in the text!

With the defined \(\tau = [krs]\) we actually have in the case
\begin{eqnarray}
\sigma'(i) & = & \tau\sigma\tau^{-1}\sigma^{-1}(i) \\
           & = & \tau\sigma\tau^{-1}(s) \\
           & = & \tau\sigma(r) \\
           & = & \tau(s) \\
           & = & k \neq i
\end{eqnarray}

In the current Third Edition, \(\tau = [rsk]\).

%%%%%%%%%%%%%%%%%%%%%%%%%%%%%%%%%%%%%%%%%%%%%%%%%%%%%%%%%%%%%%%%%%%%%%%%
\subsection{Lemma 8.3}
The \(c\in A_1\) may be explicitly taken as
\[c = p^{k-r}\mu a_1.\]

%%%%%%%%%%%%%%%%%%%%%%%%%%%%%%%%%%%%%%%%%%%%%%%%%%%%%%%%%%%%%%%%%%%%%%%%
%%%%%%%%%%%%%%%%%%%%%%%%%%%%%%%%%%%%%%%%%%%%%%%%%%%%%%%%%%%%%%%%%%%%%%%%
\section{Exercises (page 75)}

%%%%%%%%%%%%%%%%
\begin{myenumerate}
\addtolength{\itemsep}{10pt}

%%%%%
\begin{excopy}
Show that every group of order \(\leq 5\) is abelian.
\end{excopy}

If \(|G|=1\) then \(G=\eG\) and it is obvious.
For \(|G|\in\{2,3,5\}\) then the order is prime and $G$ is cyclic
and therefore abelian.

Now assume \(|G|=4\). If $G$ is cyclic then it is abelian.
If $G$ is not cyclic then \(G=(\Zm{2})\times(\Zm{2})\)
and a simple check verifies that $G$ is abelian.

%%%%%
\begin{excopy}
Show that there are two non-isomorphic groups of order $4$,
namely the cyclic one and the product of two cyclic groups of order $2$.
\end{excopy}

Note: This was actually used in the solution of previous exercise.
Let $G$ be a non cyclic group of order $4$ and
\(G = \{e, g_1, g_2, g_3\}\).
Since \(g_i^4=e\) for all \(i=1,2,3\) and therefore we must also
have \(g_i^2=e\) for all \(g_i\) since otherwise \(\{g_i^k\}_{k=0}^3\)
generates $G$ contradicting it being non-cyclic.
Denote \(h = g_1 g_2\). This product $h$
 cannot be equal to $e$ since \(g_1\)  and \(g_2\)
The product $h$ cannot be equal to either \(g_1\) nor \(g_2\)
Since neither of them is the unit.
Thus \(g_1 g_2 = g_3\). With this the homomorphism
of $G$ onto \((\Zm{2})\times(\Zm{2})\)
generated by:
\begin{eqnarray*}
 g_1 & \rightarrow &  (1,0) \\
 g_2 & \rightarrow &  (0,1) \\
\end{eqnarray*}
exists and satisfies isomorphism.


%%%%%
\begin{excopy}
Let $G$ be a group. A \textbf{commutator} in $G$ is
an element of the form \(aba^{-1}b^{-1}\) with \(a,b\in G\).
Let \(G^c\) be the subgroup generated by the commutators.
Then \(G^c\) is called the \textbf{commutator subgroup}.
Show that \(G^c\) is normal. Show that any homomorphism of
$G$ into an abelian group factors through \(G/G^c\).
\end{excopy}

Let \(g\in G\) and \(m\in G^c\).
By definition \(g^{-1}mgm^{-1} \in G^c\) and so
\(g^{-1}mg \in G^c m = G^c\) and \(G^c\) is normal.

Let $A$ be an abelian group and
\(h: G\rightarrow A\) a group homomorphism.
We need to show that \(G^c \subseteq \Ker h\).
It is sufficient to show it on the generators
\begin{eqnarray*}
h(aba^{-1}b^{-1}) & = & h(a)h(b)h(a^{-1})h(b^{-1}) \\
                  & = & h(a)h(a^{-1})h(b)h(b^{-1}) \\
                  & = & h(aa^{-1})h(bb^{-1}) \\
                  & = & h(e_G)h(e_G) = e_A\\
\end{eqnarray*}


%%%%%
\begin{excopy}
Let \(H,K\) be subgroups of a finite group $G$
with \(K \subset N_H\). Show that
\[\#(HK) = \frac{\#(H)\#(K)}{\#(H\cap K)}.\]
\end{excopy}

In example (\textbf{iv}) of \S 3,
% On page~17 (Example \textbf{iv})
it was shown that \emph{when} $H$ is contained in the normalizer of $K$, then
\[H/(H\cap K) \approx HK/K.\]
From this we get
\[\#(H)/\#(H\cap K) = \#(HK)/\#(K)\]
and the desired equality follows for the special case.

Now for the general case. Put \(G = H\cap K\).
For each \(h\in H\), \(k\in K\) and \(g\in G\)
we have \(hk = (hg^{-1})(gk)\).
Therefore
\begin{equation*}
\#(HK) \geq \frac{\#(H)\#(K)}{\#(G)}.
\end{equation*}
Conversely, given \(h_1\in H\) and \(k_1\in K\),
for any \(h_2\in H\) and \(k_2\in K\) satisfying:
\begin{equation*}
  h_1 k_1 = h_2 k_2 % \qquad\textnormal{where} \h_i\in H \land k_i = K\) 
\end{equation*}
we have \(g = h_2^{-1}h_1 = k_2k_1^{-1} \in G\).
Now \(h_2 = h_1 g^{-1}\) and \(k_2 = gk_1\).
Therefore
\begin{equation*}
\#(HK) \leq \frac{\#(H)\#(K)}{\#(G)}.
\end{equation*}
and the desired equality follows.

%%%%%
\begin{excopy}
{\normalsize (Note: Using different notation.)}\newline
\textbf{Goursat's Lemma.} Let \(G_1, G_2\) be groups,
and let $H$ be a subgroup of \(G_1\times G_2\) such that
the two projections
\(p_i:H\rightarrow G_i\) for \(i=1,2\) are surjective.
Let \(N_1\) be the kernel of \(p_2\)
and \(N_2\) be the kernel of \(p_1\).
One can identify \(N_i\) as a normal subgroup of \(G_i\) (\(i=1,2\)).
Show that the image of $H$
in \(G_1/N_1 \times G_2/N_2\) is the graph of an isomorphism
\[G_1/N_1 \approx G_2/N_2.\]
\end{excopy}

It is easy to see that
\begin{eqnarray*}
\Ker{p_1} & = & H \cap (\{e_G\}\times G')\\
\Ker{p_2} & = & H \cap (G\times\{e_{G'}\})\\
\end{eqnarray*}

We have the natural mappings
\[\widetilde{p_i}: H \rightarrow G_i/N_i\qquad(i=1,2).\]
We first need to show that the graph of \((\widetilde{p_1},\widetilde{p_2})\)
defines a surjective bijection function
\begin{equation}\label{eq:g1n1g2n2}
G_1/N_1 \rightarrow G_2/N_2.
\end{equation}
The natural mapping \(H\rightarrow G_1/N_1\) is surjective
and the graph covers \(G_1/N_1\).
Hence it suffices to show that for any element of \(G_1/N_1\)
there is only one associated element in \(G_2/N_2\).
Let
\((g_1,g_2),(g'_1,g'_2)\) be any elements of $H$ such that
\[\widetilde{p_1}((g_1,g_2)) = \widetilde{p_1}((g'_1,g'_2)).\]
It is obvious that
\((g_1^{-1}g'_1,g_2^{-1}g'_2) \in \Ker(\widetilde{p_1})\)
and so \(g_2^{-1}g'_2\in N_2\) and thus \(g_2 N_2 = g'_2 N_2\).
This shows
\[\widetilde{p_2}(\left(g_1,g_2\right)) =
  \widetilde{p_2}(\left(g'_1,g'_2\right))\]
and the graph defines the mapping of (\ref{eq:g1n1g2n2}) we had to show.
Similarly, we can show the inverse mapping
\[G_2/N_2 \rightarrow G_1/N_1\]
and thus the graph is an surjective bijection.

Now to show homomorphism. Let
\(x_1N_1,y_1N_1 \in G_1/N_1\).
For \(i=1,2\)
There must be some
% \((a_i,b_i)\in H\) so
\((a_1,a_2),(b_1,b_2)\in H\) so
\(a_iN = x_iN\) and
\(b_iN = y_iN\).
The mapping (\ref{eq:g1n1g2n2}) we have established has:
\(x_1N_1 \rightarrow a_2N2\) and
\(y_1N_1 \rightarrow b_2N2\).
By looking at
\((a_1 b_1,a_2 b_2)\in H\) we get
\(x_1 y_1 N_1 \rightarrow a_2 b_2 N2\).


%%%%%
\begin{excopy}
Prove that the group of inner automorphisms of a group $G$
is normal in \(Aut(G)\).
\end{excopy}

The \textbf{inner} automorphisms
(\(\Inn(G)\), See: \cite{Scott87})
are the conjunctions.

Let \(T\in \Aut(G)\) and
let \(\gamma_x\in \Inn(G)\).
For \(y\in G\) We have
\(\gamma_x(y) = xyx^{-1}\) and
\begin{eqnarray*}
(T\gamma_x T^{-1})(y)
  & = & T(xT^{-1}(y)x^{-1})\\
  & = & T(x)T(T^{-1}(y))T(x^{-1})\\
  & = & T(x)yT(x^{-1})\\
  & = & \gamma_{T(x)}(y).\\
\end{eqnarray*}
So \(T\gamma_x T^{-1} \in \Inn(G)\) and \(\Inn(G)\) is normal.

%%%%%
\begin{excopy}
Let $G$ be a group such that \(\Aut(G)\) is cyclic.
Prove that $G$ is abelian.
\end{excopy}

Let \(T\in\Aut(G)\) be a generator.
For all \(g\in G\) we denote the inner mapping \(\gamma_g(x)=gxg^{-1}\).
So for any \(g\in G\) there exists a minimal \(n(g)\geq 0\) such that
\(T^{n(g)} = \gamma_g\). G is abelian iff \(n(g)=0\) for all \(g\in G\).
In negation we can assume there is some \(t \in G\)
with minimal \(m=n(t)>0\). We will show that \(\gamma_t\)
generates all of \(\Inn(G)\).

Let \(g\in G\), \(d=\lfloor n(g)/m\rfloor\) and the residue
\(r = n(g) - md\) where \(0\leq r<m\).
% Now \(\gamma_g = T^{n(g)} = T^r(T^m)^{d} = T^r \gamma_t^d\).
Set \(g'=gt^{-d}\) and we get
\[\gamma_{g'}(x) = (T^{n(g)}(T^m)^d)(x) = T^{n(g)-md}(x) = T^r(x).\]
By minimality of $m$ we must have \(r=0\).

So for all \(g\in G\) there exists a \(d\geq0\) such that
\(T^{md}=\gamma_g\) that is for all \(x\in G\)
\(t^{md}xt^{-md} = gxg^{-1}\). Substituting $x$ with $t$ we get
\(t = t^{md}tt^{-md} = gtg^{-1}\) from which we get \(gt=tg\)
and \(T^m\) is the identity that generates \(\Inn(G)\).
Since \(\Inn(G)=\{\Id_G\}\) we conclude that $G$ is abelian.



%%%%%
\begin{excopy}
Let $G$ be a group and let \(H, H'\) be subgroups.
By a \textbf{double coset} of \(H, H'\) one means
a subset of $G$ of the form \(HxH'\).
\begin{itemize}
  \item[(a)] Show that $G$ is a disjoint union of double cosets.
  \item[(b)] Let $C$ be a family of representatives for
     double cosets. For each \(a \in G\) denote by \([a]H'\)
     the conjugate \(aH'a^{-1}\) of \(H'\).
     For each \(c\in C\) we have a decomposition into ordinary cosets
  \begin{equation} \label{eq:decoxcHHp}
    H = \Disjunion_{x\in X_c} x(H\cap[c]H'),
    % H = \Disjunion_{x\in X_c} x(H\cap[c]H'),
  \end{equation}
     where \(X_c\) is a family of elements of $H$, depending on $c$.
     Show that the elements
     \(\{xc: c\in C,\,x\in X_c\}\) form a family of left cosets
     representatives for \(H'\) in $G$; that is,
  \begin{equation} \label{eq:decoGxccHp}
    G = \Disjunion_{c\in C}\,\Disjunion_{x\in X_c} xcH',
  \end{equation}

  \begin{quote}
   \textbf{Note:} In the original text,
   equation (\ref{eq:decoxcHHp})  appears as
   \[ H = \bigcup_c x_c(H\cap[c]H'),\]
   and equation (\ref{eq:decoGxccHp}) appears as
   \[ G = \bigcup_{x_c}\,\bigcup_{x_c} x_c cH'.\]
  I believe these are notational mistakes
   or unclear indices style.
  \end{quote}

\end{itemize}
\end{excopy}

To prove (a) let us assume that for some \(x_1,x_2\in G\)
\((Hx_1H')\cap(Hx_2H')\neq\emptyset\).
So we have \(h_1,h_2\in H\) and \(h'_1,h'_2\in H'\) so
\(h_1 x_1 h'_1 = h_2 x_2 h'_2\).
So we
\(x_1 = ({h_1}^{-1}h_2) x_2 (h'_2{h'_1}^{-1}).\)
Now for any \(g\in Hx_1H'\) there are
\(h\in H,h'\in H'\) such that we have
\[g = hx_1h' = (h{h_1}^{-1}h_2) x_2 (h'_2{h'_1}^{-1}) \in Hx_2H'.\]
Thus \(Hx_1H'\subset Hx_2H'\). Similarly,
\(Hx_2H'\subset Hx_1H'\) and they are equal and so the double cosets
are disjoint.

Now we turn to (b). For each \(c\in C\)
\(H_c = H\cap(cH'c^{-1})\subseteq H\). So we have
cosets of \(H_c\) form a disjoint union
\[H = \disjunion_{x\in X_c} xH_c = \disjunion_{x\in X_c} xcH'c{-1}\]
with some family \(X_c\) of representatives.

We compute
\begin{eqnarray} \label{eq:hch}
 HcH' & = & \left(\Disjunion_{x\in X_c} xcH'c^{-1}\right)cH' \\
      & = & \left(\Disjunion_{x\in X_c} xcH'c^{-1}c\right)H' \nonumber \\
      & = & \left(\Disjunion_{x\in X_c} xcH'\right)H' \nonumber\\
      & = & \Disjunion_{x\in X_c} xcH'H' \nonumber\\
      & = & \Disjunion_{x\in X_c} xcH'  \nonumber
\end{eqnarray}

From (a) we have a family $C$ of representatives of the double cosets
and using \ref{eq:hch} for substitution we get:
\[
G = \Disjunion_{c\in C} HcH'\\
  = \Disjunion_{c\in C} \, \Disjunion_{x\in X_c} xcH'.\]



%%%%%
\begin{excopy}
\begin{itemize}
 \item[(a)] Let $G$ be a group and $H$ a subgroup of finite index.
   Show that there exists a normal subgroup $N$ of $G$ contained in $H$
   and also of a finite index. [\emph{Hint}: If \((G:H)=n\),
   find a homomorphism if $G$ into \(S_n\) whose kernel is contained in $H$.]
 \item[(b)] Let $G$ be a group and let \(H_1\), \(H_2\) be subgroups of
   finite index. Prove that \(H_1\cap H_2\) has finite index.
\end{itemize}
\end{excopy}

Let $G$ act on the set of cosets of $H$ with \(\pi_g(cH) = gcH\)
for every \(g\in G\) and ever coset \(cH\) with \(c\in G\).
We want to show that this definition is independent of choice of
the $c$ representative.
So let \(c_1H = c_2H\), % so there must be some \(h\in H\)
and by associativity of the group multiplication
\[\pi_g(c_1H) = g(c_1H) = g(c_2H) = \pi_g(c_2)\]
and for each \(g_1,g_2\in G\), \(c\in G\)
\begin{eqnarray} \label{eq:pig1g2}
\pi_{g_1g_2}(cH) = (g_1g_2)(cH) = g_1(g_2(cH)) =  g_1(\pi_{g_2}cH) \\
  \hfill = \pi_{g_1}(\pi_{g_2}cH) = (\pi_{g_1}\pi_{g_2})(cH). \nonumber
\end{eqnarray}
The set of $H$ has $n$ elements.
By labeling the cosets
and from (\ref{eq:pig1g2})
we see that the association \(g\rightarrow \pi_g\)
is homomorphism\(T:G\rightarrow S_n\).
We have \(\Ker T \subseteq H\) since
for every \(g\in G\setminus H\) we have \(gH\neq H\).
Obviously, \(G/\Ker(T) \equiv T(G) \subseteq S_n\) and so
\[[G:H] \leq [G:\Ker(T)] = |T(G)| \leq |S_n| = n! < \infty.\]

To prove (b) we can assume that \(H_1\), \(H_2\) are normal.
Otherwise we simply use (a) and substitute them with normal subgroups.
Now we can use Example~(iv) of \S~3 (page~17)
\begin{equation}
H_1/(H_1\cap H_2) = H_1 H_2 / H_2
\end{equation}
that gives
\begin{equation}
[H_1:(H_1\cap H_2)] =
|G/(H_1\cap H_2)| =
|H_1 H_2 / H_2| \leq
|G/H_2|
\end{equation}
and so
\begin{equation}
[G:(H_1\cap H_2)] =
[G:H_1]\cdot[H_1:(H_1\cap H_2)] \leq
|G/H_1|\cdot|G/H_2| < \infty.
\end{equation}



%%%%%
\begin{excopy}
Let $G$ be a group and let $H$ be a subgroup of a finite index.
Prove that there is only a finite number of right cosets of $H$, and that
the number of right cosets is equal to the number of left cosets.
\end{excopy}

We will show a one to one surjective mapping between
the left cosets to the right cosets. It is defined by
\begin{equation} \label{eq:xH2Hx}
xH \rightarrow Hx^{-1} \qquad\textrm{for\ } x\in G
\end{equation}

The map is obviously surjective since \(x\rightarrow x^{-1}\) is.
Now assume \(Hx^{-1} = Hy^{-1}\). Then  \(x^{-1}y\in H\)
and so \((x^{-1}y)^{-1} = y^{-1}x \in H\) and so \(xH = yH\)
and so (\ref{eq:xH2Hx}) is one to one.

%%%%%
\begin{excopy}
Let $G$ be a group, and $A$ a normal abelian subgroup.
Show that \(G/A\) operates on $A$ by conjunction,
and in this manner get a homomorphism of \(G/A\) into \(\Aut(A)\).
\end{excopy}

Let \(xA\in G/A\) operate as \(a\mapsto xax^{-1}\) for all \(a\in A\).
To show that this operation as well defined, assume \(xA=yA\).
So \(x^{-1}y, y^{-1}x\in A\) and using the fact that $A$ is abelian, we get
\begin{equation}
xax^{-1} =
xax^{-1}(xy^{-1})^{-1}(xy^{-1}) =
(xy^{-1})^{-1}xax^{-1}(xy^{-1}) =
yay^{-1}.
\end{equation}
Now since \(y(xax^{-1}y^{-1} = (yx)a(yx)^{-1}\) the homomorphism follows.

\end{myenumerate}
\textbf{Semidirect product}
\begin{myenumerate}


%%%%%
\begin{excopy}
Let $G$ be a group and let $H$, $N$ be subgroups with $N$ normal.
Let \(\gamma_x\) be conjunction by an element \(x\in G\).
\begin{itemize}
 \item[(a)] Show that \(x\rightarrow \gamma_x\) induces
    a homomorphism \(f:\,H\mapsto\Aut(N)\).
 \item[(b)] If \(H\cap N = \eG\), show that the map
    \(H \times N \rightarrow HN\) given by
    \((x,y) \mapsto xy\) is a bijection, and that this map
    is an isomorphism if and only if $f$ is trivial,
    i.e. \(f(x) = \id_N\) for all \(x\in H\).
\end{itemize}
We define $G$ to be the \textbf{semidirect product} of $H$ and $N$
if \(G=NH\) and \(H\cap N = \eG\).
\begin{itemize}
 \item[(c)] Conversely, let $H$, $N$ be groups,and let
   \(\psi:\,H\mapsto \Aut(N)\) be a given homomorphism.
  Construct a semidirect product as follows.
  Let $G$ be the set of pairs \((x,h)\) with \(x\in N\) and \(h\in H\).
  Define the composition law
  \begin{equation}
    (x_1,h_1)(x_2,h_2) = (x_1x_2^{\psi(h_1x_2)}, h_1h_2),
  \end{equation}
  Show that this is a group law, and yields a semidirect product of $N$ and $H$,
  identifying
       $N$ with the set of elements \((x,1)\)
   and $H$ with the set of elements \((1,h)\).
\end{itemize}
\end{excopy}

\begin{itemize}

 \item[(a)] Let \(x,y\in H\). For any \(u\in N\)
 \[\gamma_{xy}(u) = xyu(xy)^{-1} = x\gamma_y(u)x^{-1} = \gamma_x(\gamma_y(u)).\]
 \item[(b)]
    Assume \(x_1y_1 = x_2y_2\) where \((x_1,y_1),\,(x_2,y_2)\in H\times N\).
   Multiplying both sides with
    \(x_1^{-1}\) from the left and
    \(y_2^{-1}\) from the right, we get
     \[x_2^{-1}x_1 = y_2y_1^{-1} \in H\cap N = \eG\]
   and thus \(x_1 = x_2\)
   and  \(y_1 = y_2\).
 \item[(c)]
    Seems that the problem isnot well formed.


\end{itemize}

%%%%%
\begin{excopy}
\begin{itemize}
 \item[(a)]
    Let $H$, $N$, be normal subgroups of a finite group $G$.
    Assume that the orders  of $H$ and $N$ are relatively prime.
    Prove that \(xy=yx\) for all \(x\in H\) and \((y\in N\),
    and that \(H\times N=HN\).
 \item[(b)]
    Let \(H_1,\ldots\,H_r\) be normal subgroups of $G$ such that the order
    of \(H_i\) is relatively prime to the order of \(H_j\) for \(i\neq j\).
    Prove that
    \begin{equation}
      H_1\times \ldots \times H_r = H_1\cdots H_r.
    \end{equation}
\end{itemize}
\end{excopy}

\begin{itemize}
 \item[(a)]
    We look at \(N\cap H\) since its order must divide both that
    of $H$ and $N$ we have \(|N\cap H| = 1\) and \(N\cap H = \eG\).
    Now
   \[ xyx^{-1}y^{-1} = (xyx^{-1})y^{-1} = x(yx^{-1}y^{-1}) \in N\cap H.\]
   and thus \(xyx^{-1}y^{-1} = e\) and we get \(xy=yx\).
 \item[(b)]
    Trivial by induction on $r$.
\end{itemize}


%%%%%
\begin{excopy}
Let $G$ be a finite group and let $N$ be a normal subgroup such that
$N$ and \(G/N\) have relatively prime orders.
 \begin{itemize}
   \item[(a)]
      Let $H$ be a subgroup of $G$ having the same order as \(G/N\).
      Prove that \(G = HN\).
   \item[(b)]
      Let $G$ be an automorphism of $G$. Prove that \(g(N) = N\).
 \end{itemize}
\end{excopy}

\begin{itemize}
 \item[(a)]
    The subgroup \(H\cap N\) has on order that must divide
    both $H$ and $N$ and therefore is $1$ and so the subgroup is trivial.
    Now assume \(h_1g_1 = h_2g_2\)
    for \(h_i\in H\), \(g_i\in N\), \(i=1,2\).
    Then \(h_2^{-1}h_1 = g_2g_1^{-1} \in H\cap N\) and
    \(h_1 = h_2\) and \(g_1 = g_2\). Thus counting the elements
    \(|HN| = |H|\cdot|N| = |G|\) and thus \(HN=G\).
 \item[(b)]
    Let \(h: N \rightarrow G/N\) be defined by \(h(x)=g(x)+N\).
    We will show the $h$ must be trivial.
    Now \(H=h(N)\) is a subgroup of \(G/N\) and \(|H|\)
    divides both \(|N|\) and \(G/N\) and therefore \(H=\eG\)
    which means that \(g(N)\subseteq N\). Since $g$ is automorphism
    we have \(g(N) = N\).
\end{itemize}

%%%%%
\begin{excopy}
\label{gop:nofix}
Let $G$ be a finite group operating on a finite set $S$ with \(\#(S)\geq2\).
Assume that there is only one orbit. Prove that there exist an element
\(x\in G\) which has no fixed point,i.e. \(xs\neq s\) for all \(s\in S\).
\end{excopy}

Solution using \cite{Scott87}~10.1.5.

Define \(\Ch(g) = |\{s\in S: gs=s\}|\)  \label{def:Ch}.
\begin{llem}
If $G$ acts on a finite set $S$ has $N$ orbits, then
\begin{equation}
\sum_{g\in G} \Ch(g) = n\cdot |G|.
\end{equation}
\end{llem}

Let \(T\subseteq S\) be an orbit of $G$ and \(a,b\in T\).
It is clear that \(|G_a| = |G_b| = |G|/|T|\).

We may view \(g\in G\) as permutations of $S$.
Let \(F = \{(s,g)\in S\times G: gs=s\}\) be the set of fixed points.
Now
\begin{eqnarray}
\sum_{g\in G} \Ch(g) & = & |F|\\
 & = & \sum_{s\in S} |G_s| \\
 & = & \sum_{T\ \textrm{orbit}} \sum_{s\in T} |G_s| \\
 & = & \sum_{T\ \textrm{orbit}} |T|\cdot|G|\\
 & = & n\cdot|G|.
\end{eqnarray}

Now back to the exercise, if \(n=1\)
\index{transitive}
($G$ is \emph{transitive})
and $G$ is finite then
\begin{eqnarray}
\sum_{g\in G} \Ch(g) = |G|.
\end{eqnarray}
Now assume by negation that for all \(g\in G\) \(\Ch(g)\geq 1\)
(at least one fixed point), then
\begin{eqnarray*}
|G| & = & \sum_{g\in G} \Ch(g) \\
    & = & \Ch(e) + \sum_{g\in G\setminus\eG} \Ch(g) \\
    & = & |S| + \sum_{g\in G\setminus\eG} \Ch(g) \\
    & \geq & |S| + |G| - 1
\end{eqnarray*}
Hence, \(|S| \leq 1\) which contradicts the assumption.

%%%%%
\begin{excopy}
Let $H$ be a proper subgroup of a finite group $G$. Show that $G$
is not the union of all the conjugates of $H$.
\end{excopy}

We look at $G$ as a group operating on the finite sets of the conjugates of $H$.
From the previous Exercise~\ref{gop:nofix}, there must be some \(x\in G\)
for which \(x(gHg^{-1})x^{-1} \neq gHg^{-1}\) for all \(g\in G\).
That is \(x\notin gHg^{-1}\) for all \(g\in G\) and
\[x \notin \bigcup_{g\in G}gHg^{-1}.\]

%%%%%
\begin{excopy}
Let $X$,$Y$ be finite sets and let $C$ be a subset of \(X\times Y\).
For \(x\in X\) let \(\phi(x)=\) number of elements \(y\in Y\) such that
\((x,y)\in C\). Verify that \[\#(C) = \sum_{x\in X}\varphi(x).\]

\emph{Remark}. A subset $V$ as in the above exercise is often called
\index{correspondence}
a \textbf{correspondence}, and \(\varphi(x)\) is the number of elements in $Y$
which correspond to a given element \(x\in X\).
\end{excopy}

This is simple result of looking at the disjoint union:
\[C = \Disjunion_{x\in X} \{(x,y)\in X\times Y: (x,y)\in C\}.\]

%%%%%
\begin{excopy}
Let $S$, $T$ be finite sets. Show that \(\#\Map(S,T) = (\#T)^{\#(S)}\).
\end{excopy}

Simple induction on \(\#(S)\).
If \(S'=S\cup\{x\}\) then we can extend each map \(S\rightarrow T\)
to \(S'\) by assigning \(\#(T)\) different values to $x$.

%%%%%
\begin{excopy}
Let $G$ be a finite group operating on a finite set $S$.
 \begin{itemize}
  \item[(a)]
    For each \(s\in S\) show that \[\sum_{t \in Gs} {\frac{1}{\#(Gt)}} = 1.\]
  \item[(b)]
    For each \(x \in G\) define \(f(x)=\) number of element \(s\in S\)
    such that \(xs=s\). Prove that the number of orbits of $G$ in $S$
    is equal to
      \[\frac{1}{\#(G)}\sum_{x\in G} f(x).\]
 \end{itemize}
\end{excopy}

\begin{itemize}
 \item[(a)]
   For all \(t \in Gs\) we have \(|Gs|=|Gt|\) and so
   \begin{equation}
   \sum_{t \in Gs} {\frac{1}{\#(Gt)}} =
   |Gs|\cdot{\frac{1}{|Gs|}} = 1,
   \end{equation}
 \item[(b)]
  Let us compute the number of ``fixed occurrences''.
  \begin{eqnarray}
   \sum_{x\in G} f(x)
     & = & \sum_{x\in G} \Ch(x)                 \label{eq:f2Ch} \\
     & = & \sum_{s\in S} |\{g\in G: gs = s\}|    \label{eq:Ch2gss} \\
     & = & \sum_{s\in S} |G_s|                   \label{eq:gss2Gs} \\
     & = & \sum_{\textrm{orbit\ } T\subseteq S}
             \sum_{t\in T} |G_t|                 \label{eq:Gsrob} \\
     & = & \sum_{\textrm{orbit\ } T\subseteq S}
             \sum_{t\in T} |G|/|Gt|              \label{eq:GfsGGs} \\
     & = & |G|\cdot\sum_{\textrm{orbit\ } T\subseteq S}
             \sum_{t\in T} 1/|Gt|                \label{eq:1overGs} \\
     & = & |G|\cdot\sum_{\overset{s\in S}{\textrm{orbits repr.}}}
             \sum_{t\in Gs} 1/|Gt|                \label{eq:orbrep} \\
     & = & |G|\cdot\sum_{\overset{s\in S}{\textrm{orbits repr.}}} 1.
                                                  \label{eq:orb1}
  \end{eqnarray}

  Equalities explanation:
  \begin{itemize}
   \item[(\ref{eq:f2Ch})] --- simply using the definition
                              in Exercise~\ref{def:Ch}.
   \item[(\ref{eq:Ch2gss})] --- counting fixed points via $S$ instead of $G$.
   \item[(\ref{eq:gss2Gs})] --- definition of \(G_s\).
   \item[(\ref{eq:Gsrob})] --- separating the summation over orbits.
   \item[(\ref{eq:GfsGGs})] --- basic result of group operating on set.
   \item[(\ref{eq:1overGs})] --- factoring \(|G|\) out
   \item[(\ref{eq:orbrep})] ---  Looking at an orbit $T$ via
                                 a representative \(s\in S\).
   \item[(\ref{eq:orb1})] --- Using the previous item (a) of this exercise.
  \end{itemize}

  Now we simply divide both ends of the equation by \(|G|\)
  to get the desired result.

\end{itemize}

\end{myenumerate}

Throughout, $p$ is a prime number.

\iffalse
% Global remark - so fake an item
\item[]
 \setlength{\leftmargin}{0pt}
 \setlength{\labelwidth}{0pt}
 \setlength{\labelwidth}{0pt}
 Throughout, $p$ is a prime number.
% {\nullfont kaka}
\addtocounter{enumi}{-1}
\fi

\begin{myenumerate}
%%%%%
\begin{excopy}
Let $P$ be a $p$-group. Let $A$ be a normal subgroup of order $p$.
Prove that $A$ is contained in the center of $P$.
\end{excopy}

We can view $P$ as operating on $A$ by conjunction.
That is for any \(x\in P\),we have \(\gamma_x(a) = xax^{-1}\).
Let \(x\in P\) and \(a\in A\) be any elements.
Say \(|P|=p^n\)
and so we have \(\gamma_x^{p^n}=\Id_A\).
% Assume \(\gamma_x(a)\neq a\), so \(a\neq e\) for sure.
Note that
\[\underbrace{\gamma_x(\gamma_x(\ldots(\gamma_x(}_{n\ \textrm{times}}a)\ldots))
 =  \gamma_x^n(a).\]
So because of \(A\setminus\eG\)
let \(k>0\) be the minimal such that \(\gamma_x^k(a)=a\).
So we have \(k|p^n\) and \(k\neq p-1\) and so \(k=1\) and \(\gamma_x(a)=a\).
Thus \(xa=ax\) and $a$ is in the center of $P$.


%%%%%
\begin{excopy}
Let $G$ be a finite group and $H$ a subgroup. Let \(P_H\) be
a $P$-Sylow subgroup of $H$. Prove that there exists a $p$-Sylow subgroup $P$
  of $G$ such that \(P_H = P\cap H\).
\end{excopy}

Since any $p$-subgroup is contained in a $p$-Sylow subgroup,
We have a subgroup $P$ such that \(P_H\subseteq P\subseteq G\).
Obviously \(P_H\subseteq P\cap H\).
Now \(P\cap H\) has an order that divides \(|P|\) so it is a power of $p$.
But from the maximality of \(P_H\) the equality follows.

%%%%%
\begin{excopy}
Let $H$ be a normal subgroup of a finite group $G$
and assume that \(\#(H)=p\). Prove that $H$ is contained in every $p$-Sylow
subgroup of $G$
\end{excopy}

We know that $H$ is contained in some $p$-Sylow subgroup S.
All $p$-Sylow subgroups are conjugates. Now for all \(x\in G\)
\[H=xHx^{-1}\subseteq xSx^{-1}.\]


%%%%%
\begin{excopy}
Let $P$, \(P'\) be $p$-Sylow subgroups of a finite group $G$.
\begin{itemize}
 \item[(a)]  If \(P'\subset N(P)\) (normalizer of $P$), then \(P'=P\).
 \item[(b)]  If \(N(P')=N(P)\), then \(P'=P\).
 \item[(c)]  We have \(N(N(P))=N(P)\).
\end{itemize}
\end{excopy}

\begin{itemize}
 \item[(a)] (Following argument in the proof of Theorem~6.4).
     Since \(P'\subseteq N(P)\)
     we have \(P'P\) is a subgroup of \(N(P)\) and $P$ is normal in it.
     Now
     \begin{equation}\label{eq:pppcp}
     (P'P:P) = (P':P'\cap P)
     \end{equation}
     (see~(iv) page~17).
     Now if by negation \(P'\neq P\) then $p$ divides
     the right side of~(\ref{eq:pppcp}) and so \(P'P\) contains
     a~$p$-Sylow subgroup with higher power of $p$ than that of $P$
     contradicting the fact that $P$ itself is a $p$-Sylow subgroup.
 \item[(b)] Immediate from (a) and the fact that \(P'\subseteq N(P')\).
 \item[(c)] By negation, say \(x\in N(N(P))\setminus N(P)\).
   So \(P' = xPx^{-1}\) is a $p$-Sylow subgroup and \(P'\neq P\).
   Now \[P' = xPx^{-1} \subseteq xN(P)x^{-1} = N(P)\]
   and from (a) we get \(P'=P\) a contradiction.
\end{itemize}
%%%%%%%%%%%%%%
\end{myenumerate}

%%%%%%%%%%%%%%%%%%%%%%%%%%%%%%%%%%%%%%%%%%
\textbf{Explicit determination of groups}

Let us have some lemmas.

\begin{llem} \label{llem:npdiv}
Let $G$ be a group or order $m$,
let \(p^r\) be the highest power of~$p$ that divides~$m$
and let~\(n_p\) be the number of $p$-Sylow subgroups.
Then \(n_p \mid m/p^r\).
\end{llem}
%\textbf{Proof:}
\begin{proof}
Immediate result from Proposition~5.2.
\end{proof}

\begin{llem} \label{rose94:p2q}
\textnormal{\small [See \cite{Rose94} Theorem~5.19]}
Let $G$ be a group and \(|G|=p^2q\) where $p$, $q$ are distinct primes,
then $G$ has a normal  Sylow subgroup and so $G$ is not simple.
\end{llem}
\begin{proof}
Indeed, if $G$ has a normal Sylow subgroup $H$ then
\[\eG\subnormal H \subnormal G\]
is an abelian tower by exercise \ref{ex:p2abel}.
This is true for either \(|H|=p^2\) or \(|H|=q\).
Thus $G$ is simple.

Now let's show the existence of a normal Sylow subgroup.

If \(q<p\) then by Lemma~6.7 the Sylow $p$-subgroup is normal.
So we now can assume \(p<q\).
Let \(n_p\) and \(n_q\) be the numbers of
Sylow $p$-subgroup and $q$-subgroups. By negation we assume
\(n_p > 1\) and \(n_q > 1\).
By local-lemma~\ref{llem:npdiv}
\begin{itemize}
 \item
   \(n_p\mid q\) and so \(n_p=q\).
   % Also \(n_q \equiv 1 \bmod\).
 \item
   \(n_q\mid p^2\), hence \(n_q=p\) or \(n_q=p^2\).
   But  if \(n_q=p\) then by \(n_q\equiv 1 \bmod q\) we have \(p>q\)
   contradicting our assumption and so \(n_q=p^2\).
\end{itemize}
Now any two different $q$-subgroups intersect in \eG
and so the number of elements of order $q$ is \(n_q(q-1)\).
The number of the ``non $q$ order'' elements in $G$ is \(p^2q - n_q(q-1)=p^2\).
Now a Sylow $p$-subgroup has an order \(p^2\) and all its elements
have order different than $q$ and so such subgroup is determined
by its \(p^2\) ``non $q$ order'' elements
and thus it is unique and normal.
\end{proof}

\begin{llem} \label{rose94:pqr}
\textnormal{\small [See \cite{Rose94} Theorem~5.20.]}
Let $G$ be a group and \(|G|=pqr\) where $p$, $q$, $r$ are distinct primes,
then $G$ is not simple.
\end{llem}
\begin{proof}
Let \(n_p\),\(n_q\) and \(n_r\) be the respective
numbers of Sylow subgroups.
Assume by negation that these three numbers are \(\>1\)
since otherwise
we have a normal Sylow subgroup and we are done.
Assume \(p>q>r\)
So any two distinct Sylow subgroups intersect in~\eG.
So the numbers of elements in $G$ of order $p$, $q$ and $r$
are
\(n_p(p-1)\), \(n_q(q-1)\) and \(n_r(r-1)\) respectively.
Therefore
\begin{equation}
|G|=pqr\geq 1 + n_p(p-1) + n_q(q-1) + n_r(r-1).
\end{equation}
By Sylow theorem, \(n_p \mid qr\) and \(n_p\equiv 1\bmod p\).
Since \(n_p>q\) and \(p>q\), \(p>r\), it follows that \(n_p=qr\).

Also \(n_q\mid pr\) and \(n_q\equiv 1\bmod q\)..
Since \(n_q>1\) and \(q>r\), it follows that \(n_q\geq p\).

Finally, \(n_r\mid pq\) so \(n_r\geq q\). Now we have
\begin{equation}
pqr \geq 1 + qr(p-1) + p(q-1) + q(r-1) = pqr + pq + qr - p - q  + 1,
\end{equation}
and hence \((p-1)(q-1)\leq 0\) which is impossible.
\end{proof}

\textbf{Definition:}
\textnormal{\small [See \cite{Rose94}~Exercise~90]}
Let $H$ be a subgroup of $G$.
\index{core!of group}
\index{normal interior}
Define the \emph{core} or \emph{normal interior} of $H$ in $G$ as
\begin{equation}
H_G = \bigcap_{g\in G} g^{-1}Hg
\end{equation}

It is clear that \(H_G\) is the largest normal subgroup of $G$ that
is contained in $H$.

\begin{llem} \label{llem:GsCoreSn}
\textnormal{\small [See \cite{Rose94}~Theorem~4.13]}
if $H$ is a subgroup of $G$ of finite index \(n=[G:H]\) then \(G/H_G\)
can be embedded in \(S_n\).
\end{llem}
\begin{proof}
Let \(\hat{H}\) be the set of $n$ left cosets of $H$ in $G$.
Let $G$ operate on this set by left multiplication.
Each \(g\in G\) permutates \(\hat{H}\).
By enumerating the cosets we identify \(S_n\)
with the permutations of \(\hat{H}\) and we have a mapping
\(\rho: G \rightarrow S_n\).
The kernel of \(\rho\) is the elements \(g\in G\)
for which each coset is fixed.

Let's first compute the \index{stabilizer} \index{Stab}
of a coset \(xH\)
\begin{eqnarray}
\Stab_G(xH)
  & = & \{g\in G: gxH=xH\} \\
  & = & \{g\in G: x^{-1}gxH=H\} \\
  & = & \{g\in G: x^{-1}gx \in H\} \\
  & = & \{g\in G: g \in xHx^{-1}\} \\
  & = & xHx^{-1}
\end{eqnarray}
We compute:
\begin{eqnarray}
\Ker\rho
  & = & \bigcap_{xH\in\hat{H}} \Stab_G(xH) \\
  & = & \bigcap_{g\in G} \Stab_G(gH) \\
  & = & \bigcap_{g\in G} gHg^{-1} \\
  & = & H_G. \\
\end{eqnarray}
And so \(G/H_G\approx \rho(G)\) a subgroup of \(S_n\).
\end{proof}


\begin{llem} \label{llem:pmr:divfac}
\textnormal{\small [See \cite{Rose94}~Exercise~279]}
Let $G$ be a simple group of order \(p^m r\) where \(p\nmid r\).
Then \(p^m \mid (r-1)!\).
\end{llem}
\begin{proof}
Let $H$ be a $p$-Sylow subgroup if $G$. Since $G$ is simple \(H_G = \eG\)
and from local-lemma~\ref{llem:GsCoreSn} $G$ can be embedded in \(S_{[G:H]}\).
Hence \(p^m r \mid r!\) and therefore \(p^m \mid (r-1)!\).
\end{proof}

\begin{myenumerate}

%%%%%
\begin{excopy}
Let $p$ be a prime number. Show that a group of order \(p^2\)
is abelian, and that there are only two such groups up to isomorphism.
\end{excopy}  \label{ex:p2abel}

Let $G$ be the group and $Z$ its center and \(Z\subnormal G\).
If \(G=Z\) then clearly $G$ is abelian.
Assume by negation that \(Z\subsetneq G\).
Since $Z$ is not trivial by Theorem~6.5 it must be of order $p$
and so \(G/Z\) has an order of $p$ as well and is cyclic generated by \(a+Z\).
Now let  \(a_1,a_2 \in G\) be any elements in $G$. For \(i=1,2\) we can
have the representation \(a_i=a^{n_i}g_i\) where \(n_i\neq 0\) and \(g_i\in Z\).
Now
\begin{equation} \label{eq:a1a2}
a_1a_2 = a^{n_1}g_1 a^{n_2}g_2 =
  a^{n_1+n_2}g_1g_2 =
  a^{n_2}a^{n_1}g_2g_1 =
  a^{n_2}g_2a^{n_1}g_1 = a_2a_1.
\end{equation}
Thus $G$ is abelian.

Now by Theorem~8.2, $G$ is isomorphic to a product of cyclic $p$-group.
Hence, isomorphic to\, \(\Zm{p^2}\) \, or \, \(\Zm{p}\times\Zm{p}\).


%%%%%
\begin{excopy}
Let $G$ be a group of order \(p^3\), where $p$ is prime, and $G$ is not abelian.
Let $Z$ be its center. Let $C$ be a cyclic group of order $p$.
\begin{enumerate}[(a)]
\item Show that \(Z \approx C\) and \(G/Z \approx C \times C\).
\item) Every subgroup of $G$ of order \(p^2\) contains $Z$ and is normal.
\item Suppose \(x^p = 1\) for all \(x \in G\),
 Show that $G$ contains a normal subgroup \hbox{\(H \approx C \times C\)}.
\end{enumerate}
\end{excopy}

\begin{itemize}
 \item[(a)]
     The $Z$ subgroup cannot be the whole $G$ since $G$ is not abelian.
     It cannot be trivial because of Theorem~6.5. That leaves the possibilities
     for its order to be $p$ or \(p^2\). If by negation the order is \(p^2\),
     then \(G/Z\) is cyclic and as in exercise~\ref{ex:p2abel}
     similar arguments like in (\ref{eq:a1a2}) gives a contradiction
     by showing that $G$ is abelian. Thus $Z$ is cyclic of order $p$
     and isomorphic to $C$ and \(G/Z\) is of order \(p^2\).

     Now from the previous exercise we know that groups of order \(p^2\)
     must be isomorphic to either \(\Zm{p}\times\Zm{p}\)
     or to the cyclic
     \(\Zm{p^2}\). The latter leads to contradiction that $g$ is abelian
     using the same arguments with \(G/Z\) cyclic.
     Thus \(G/Z\) is isomorphic to \(\Zm{p}\times\Zm{p}\) which is
     isomorphic to \(C\times C\).

 \item[(b)]
     Say $H$ is a subgroup of order \(p^2\). By Lemma~6.7 $H$ is normal.
     The subgroup \(H\cap Z\)
     could be or order $p$ or $1$.
     If by negation it is the latter case, then \(H\cap Z = \eG\).
     To show that \(HZ = \{hc: h\in H \, \textrm{and} \, c\in Z\}\)
     has exactly \(|H|\cdot|Z|=p^3\) we will show that the products
     differ. If
     \(h_1 c_1 = h_2 c_2\) with \(h_i\in H\), \(c_i\in Z\) we get
     \(c_1c_2^{-1} = h_1^{-1}h_2 \in H\cap Z\) and so this product equals $e$
     and \(h_1=h_2\), \(c_1=c_2\). Hence \(HZ=G\) and we can
     represent any \(a_1,a_2\in G\) by \(a_i=h_i c_i\) where
      \(h_i\in H\), \(c_i\in Z\) for \(i=1,2\).
      From exercise~\ref{ex:p2abel} $H$ is abelian and so
      \begin{equation}
      a_1a_2 = h_1 c_1 h_2 c_2 = h_1 h_2  c_1 c_2 =
          h_2 h_1  c_2 c_1 = h_2 c_2 h_1 c_1 = a_2a_1
      \end{equation}
      contradicting the fact that $G$ is abelian.
      Thus  \(H\cap Z\) is of order $p$ and $H$ must contain $Z$.

 \item[(c)]
      Examining the proof of Corollary~6.6 we see that
      in the sequence
      \[\eG=G_0 \subset G_1 \cdots \subset G_n = G\]
      every $p$-group $G$ has, the subgroup \(G_1\) is in the center.
      So in our case \(G_1 = Z\) and  let $H$ be \(G_2\)
      that has order of \(p^2\)
      and again by Lemma~6.7 is normal.
      Now $H$ cannot be cyclic, since if it were then
      its generator $x$ would not satisfy the required
      \(x^p=1\) equation.
      Now by exercise~\ref{ex:p2abel}  $H$
      must be isomorphic to \(C\times C\) and not to the cyclic \(\Zm{p^2}\).
\end{itemize}

%%%%%
\begin{excopy}
\begin{itemize}
 \item[(a)] Let $G$ be a group of order \(pq\), where $p$, $q$ are primes
            and \(p<q\). Assume that \(q\not\equiv 1 \bmod p\).
            Prove that  $G$ is cyclic.
 \item[(b)] Show that every group of order \(15\) is cyclic.
\end{itemize}
\end{excopy}  \label{ex:GpLTq}

\begin{itemize}
 \item[(a)]
   [Similar to the example on page~36 with $G$ of \(35\)].
   Let \(H_p\) and \(H_q\) be a $p$-Sylow and $q$-Sylow subgroups respectively.
   Then \(H_q\) is normal by Lemma~6.7.
   Now \(H_p\) operates by conjunction on \(H_q\) and we have
   a homomorphism \(H_p\rightarrow \Aut(H_q) \approx \Zm{(q-1)}\).
   So the image order must divide $p$ and \(q-1\).
   Since \(q-1\not\equiv 0 \bmod p\) clearly \(p\nmid q-1\)
   and so the image is trivial and so elements of \(H_p\) and \(H_q\)
   commutes with each other.

   We will show that \(H_pH_q = G\).
   The set \(H_{pq}=\{x_p^m x_q^n: 0\leq m<p, 0\leq n<q\}\)
   contains \(pq\) elements
   since if \(x_p^{m_1} x_q^{n_1} = x_p^{m_2} x_q^{n_2}\)
   we use the commutativity and the fact that \(H_p\cap H_q=\eG\)
   to get \(x_p^{m_1-m_2} = x_q^{n_2-n_1} = e\) and so \(H_pq=G\)
   and $G$ is abelian. By Proposition 4.3(\textbf{v}) $G$ is cyclic.

   Let \(x_p\) and \(x_q\) be generators of \(H_p\) and \(H_q\) respectively.
   % Then these generators commutes with each other and th
 \item[(b)] By (a) with \(p=3\), \(q=5\) and we have
      \(5\equiv 2\not\equiv 1 \bmod 3\).

\end{itemize}

%%%%%
\begin{excopy}
Show that every group of order \(<60\) is solvable.
We use the results from Corollary~6.6 and
exercises \ref{ex:GpLTq}, \ref{ex:p2q} and \ref{ex:2pq}.
\end{excopy}

{
% \begin{multicols}{2}

\tablefirsthead{\hline \(|G|\)   &   $=$ & $p$   &   $q$ & $r$ \\ \hline}
\tablehead{\hline \multicolumn{5}{|c|}{\small\textsl{continuation}} \\ \hline}
\tabletail{\hline \multicolumn{5}{|c|}{\small\textsl{to be continued}}\\ \hline}
\tablelasttail{\hline}
\begin{supertabular}{|r|c|r|r|r|}
 1 & \multicolumn{4}{|l|}{Trivial}  \\ \hline
 2 & $p$       & $2$  &      &   \\ \hline
 3 & $p$       & $3$  &      &   \\ \hline
 4 & \(p^2\)   & $3$  &      &   \\ \hline
 5 & $p$       & $5$  &      &   \\ \hline
 6 & $pq$      & $2$  & $3$  &   \\ \hline
 7 & $p$       & $7$  &      &   \\ \hline
 8 & \(p^n\)   & $2$  &      &   \\ \hline
 9 & \(p^n\)   & $3$  &      &   \\ \hline
10 & $pq$      & $2$  & $5$  &   \\ \hline
11 & $p$       & $11$ &      &   \\ \hline
12 & \(p^2q\)  & $2$  & $3$  &   \\ \hline
13 & $p$       & $13$ &      &   \\ \hline
14 & $pq$      & $2$  & $7$  &   \\ \hline
15 & $pq$      & $3$  & $5$  &   \\ \hline
16 & \(p^n\)   & $2$  &      &   \\ \hline
17 & $p$       & $17$ &      &   \\ \hline
18 & $p^2q$    & $3$  & $2$  &   \\ \hline
19 & $p$       & $19$ &      &   \\ \hline
20 & $p^2q$    & $2$  & $5$  &   \\ \hline
21 & $pq$      & $3$  & $7$  &   \\ \hline
22 & $pq$      & $2$  & $11$ &   \\ \hline
23 & $p$       & $23$ &      &   \\ \hline
% 24 &           &      &      &   \\ \hline
\hline
25 & \(p^n\)   & $5$  &      &   \\ \hline
26 & $pq$      & $2$  & $13$ &   \\ \hline
27 & \(p^n\)   & $3$  &      &   \\ \hline
28 & \(p^2q\)  & $2$  & $7$  &   \\ \hline
29 & $p$       & $29$ &      &   \\ \hline
30 & \(pqr\)   & $2$  & $3$  & $5$  \\ \hline
31 & $p$       & $31$ &      &   \\ \hline
32 & \(p^n\)   & $2$  &      &   \\ \hline
33 & $pq$      & $3$  & $11$ &   \\ \hline
34 & $pq$      & $2$  & $17$ &   \\ \hline
35 & $pq$      & $5$  & $7$  &   \\ \hline
\hline
37 & $p$       & $37$ &      &   \\ \hline
38 & $pq$      & $2$  & $19$ &   \\ \hline
39 & $pq$      & $3$  & $13$ &   \\ \hline
41 & $p$       & $41$ &      &   \\ \hline
42 & \(pqr\)   & $2$  & $3$  & $7$  \\ \hline
43 & $p$       & $43$ &      &   \\ \hline
44 & \(p^2q\)  & $2$  & $11$ &   \\ \hline
45 & \(p^2q\)  & $3$  & $5$  &   \\ \hline
46 & $pq$      & $2$  & $23$ &   \\ \hline
47 & $p$       & $47$ &      &   \\ \hline
\hline
49 & \(p^n\)   & $7$  &      &   \\ \hline
50 & \(p^2q\)  & $5$  & $2$  &   \\ \hline
51 & $p$       & $51$ &      &   \\ \hline
52 & \(p^2q\)  & $2$  & $13$ &   \\ \hline
53 & $p$       & $53$ &      &   \\ \hline
\hline
55 & $pq$      & $5$  & $11$ &   \\ \hline
\hline
57 & $pq$      & $3$  & $19$ &   \\ \hline
58 & $pq$      & $2$  & $29$ &   \\ \hline
59 & $p$       & $59$ &      &   \\ \hline
\end{supertabular}
% \end{multicols}
}

Now we need to solve some cases specifically.
Let $G$ be a group. For most orders upto $60$ solvability was shown
in the above table. In each of the following remaining cases
we will show the existence of some proper normal subgroup $H$.
Because of  results for lower orders of $G$,
A sequence
\(\eG\subnormal H \subnormal G\) can be completed to an abelian tower.

\begin{itemize}
 \item Assume \(|G|=24=2^3\cdot3\).\\
    From local-lemma \ref{llem:GsCoreSn} $G$ is not simple since otherwise
    \(2^3\mid(3-1)!\).
 \item Assume \(|G|=36=2^2\cdot3^2\).
    From local-lemma \ref{llem:GsCoreSn} $G$ is not simple since otherwise
    \(3^2\mid(4-1)!\).
 \item Assume \(|G|=40=2^3\cdot5\).
    Exercise~\ref{ex:G40G12} shows that $G$ is not simple.
 \item Assume \(|G|=48=2^4\cdot3\).
    From local-lemma \ref{llem:GsCoreSn} $G$ is not simple since otherwise
    \(2^4\mid (3-1)!\).
 \item Assume \(|G|=54=2\cdot3^3\).
    From Lemma~6.7 \(H_3\) is normal and $G$ is not simple
 \item Assume \(|G|=56=2^3\cdot7\).
    Let \(n_p\) be the number of $p$-Sylow subgroups for \(p=2,7\).
    Now \(n_7\equiv 1 \bmod 7\) and \(n_7\mid 8\).
    If \(n_7=1\) then such \(H_7\) is normal and we are done.
    % if by negation $G$ is simple, then \(n_7=8\). % and \(n_2=7\)

    Otherwise, we can assume \(n_7=8\). % and \(n_2=7\)
    Since such $7$-Sylow subgroups intersect in \eG,
    the number of elements
    in $G$ with order $7$ is \(n_7(7-1)=48\).
    The elements of any $2$-Sylow subgroup are of order \(\neq7\).
    And since \(56-48=8\) there could be only
    one such subgroup \(H_2\{g\in G: g^7\neq e\}\) whose order is $8$
    and must be normal.
\end{itemize}

%%%%%
\begin{excopy}
Let $p$, $q$ be distinct primes. Prove that a group of order \(p^2q\)
is solvable, and that one of its Sylow subgroups is normal.
\end{excopy}  \label{ex:p2q}

By Local Lemma~\ref{rose94:p2q} one of its Sylow subgroups is normal.
Then one of the normal towers
\begin{itemize}
 \item[] \(\eG \subnormal H_p \subnormal G\)
 \item[] \(\eG \subnormal H_q \subnormal G\)
\end{itemize}
exist and can be refined into abelian (and even cyclic) tower.

%%%%%
\begin{excopy}
Let $p$, $q$ be odd primes. Prove that a group of order \(2pq\) is solvable.
\end{excopy}  \label{ex:2pq}

By Local Lemma~\ref{rose94:pqr} such group $G$ has a normal subgroup $H$.

Now \(|H|\in \{p,q,r,pq,pr,qr\}\) and
by Proposition~6.8 the tower \(\eG\subnormal H\subnormal G\)
can be refined to abelian (and cyclic) tower.



%%%%%
\begin{excopy}
\begin{itemize}
 \item[(a)]
   Prove that one of the Sylow subgroups of a group of order $40$ is normal.
 \item[(b)]
   Prove that one of the Sylow subgroups of a group of order $12$ is normal.
\end{itemize}
\end{excopy} \label{ex:G40G12}

\begin{itemize}
\item[(a)] We have \(40=2^3\cdot5\) so the number of 5-Sylow subgroups
  must satisfy \(n_5\mid 8\) and \(n_5\equiv 1 \bmod 5\)
  and so \(n_5=1\) and the unique 5-Sylow subgroup is normal.
\item[(b)]
 From exercise~\ref{ex:p2q}  with \(p^2q=12\) where \(p=2\) and \(q=3\).
\end{itemize}

%%%%%
\begin{excopy}
Determine all groups of order \(\leq 10\) up to isomorphism.
In particular, show that a non-abelian group of order $6$
is isomorphic to \(S_3\).
\end{excopy}

The groups with prime order are cyclic. For other case of a group $G$:
\begin{itemize}
  \item Assume \(|G|=6=2\cdot3\). It could be isomorphic to:
    \begin{itemize}
       \item \(\Zm{6}\) cyclic.
       \item \(\Zm{2}\times\Zm{3}\) abelian.
       \item \(S_3\).
    \end{itemize}
  \item Assume \(|G|=8=2^3\). It could be isomorphic to:
    \begin{itemize}
       \item \(\Zm{8}\) cyclic.
       \item \(\Zm{2}\times\Zm{2}\times\Zm{2}\) abelian.
       \item \(\Zm{2}\times\Zm{4}\) abelian.
    \end{itemize}
  \item Assume \(|G|=9=3^2\). It could be isomorphic to:
    \begin{itemize}
       \item \(\Zm{9}\) cyclic.
       \item \(\Zm{3}\times\Zm{3}\) abelian.
    \end{itemize}
  \item Assume \(|G|=10=2\cdot5\). It could be isomorphic to:
    \begin{itemize}
       \item \(\Zm{10}\) cyclic.
       \item \(\Zm{2}\times\Zm{5}\) abelian.
    \end{itemize}

\end{itemize}

%%%%%
\begin{excopy}
Let \(S_n\) be the permutation group on $n$ elements.
Determine the $p$-Sylow subgroups of
\(S_3\), \(S_4\), \(S_5\) for \(p=2\) and \(p=3\).
\end{excopy}

\begin{itemize}
 \item[\(S_3\)]
    The $2$-Sylow subgroups % generated by transposition and they
    are:
    \(\{e,(12)\}\), \(\{e,(13)\}\) and \(\{e,(23)\}\).
    The $3$-Sylow subgroup is
    \(\{e,(123),(132)\}\).
 \item[\(S_4\)]
    The $2$-Sylow subgroup is \(S_4\) itself and no $3$-Sylow subgroups.
 \item[\(S_5\)] No $2$-Sylow and no $3$-Sylow subgroups.
\end{itemize}

%%%%%
\begin{excopy}
Let \(\sigma\) be a permutation of a finite set $I$ having $n$ elements.
Define \(e(\sigma)\) to be \((-1)^m\) where
\[m = n - \textrm{number of orbits of }\, \sigma.\]
If \(I_1,\ldots,I_r\) are orbits of \(\sigma\), then $m$ is also equal
to the sum
\[ m= \sum_{v=1}^r [\card(I_v)-1].\]
If \(\tau\)  is a transposition, show that \(e(\sigma\tau) = -e(\sigma)\)
be considering the two cases where $i$, $j$ lie in the same orbit of \(\sigma\),
or lie in different orbits. In the first case, \(\sigma\tau\) has one more
orbit and in theses case one less orbit that \(\sigma\).
In particular, the sign of a transposition is \(-1\).
Prove that \(e(\sigma)=\epsilon(\sigma)\) is the sign of the permutation.
\end{excopy}

We show the equality of $m$,
\[ \sum_{v=1}^r [\card(I_v)-1] =
   \sum_{v=1}^r \card(I_v) - \sum_{v=1}^r 1 =
   \sum_{v=1}^r \card(I_v) - \sum_{v=1}^r 1 =
   n - r.\]

We now show \(e(\sigma\tau)= -e(\sigma)\). Let \(\tau=(ij)\).
There are two cases:
\begin{itemize}
 \item
   The elements $i$, $j$  lie in the same orbit $I$ of \(\sigma\).
   Let \(l=|I|\geq 2\) and \(1\leq k<l\) such that \(\sigma^k(i)=j\).
   It is clear that \(\sigma^{l-k}(j)=i\).
   Now \(\sigma\tau\) has all the orbits of \(\sigma\)
   with $I$ split into two orbits:
     \((i, \sigma(j), \ldots \sigma^{l-k-1}(j))\)
   and
     \((j, \sigma(i), \ldots \sigma^{k-1}(j))\).
 \item
   The elements $i$, $j$  lie in different orbits \(I_i\) and \(I_j\)
   respectively of \(\sigma\).
   Now \(\sigma\tau\) has all the orbits of \(\sigma\)
   but with \(I_i\) and \(I_j\) united.
\end{itemize}
In both cases the number of orbits of \(\sigma\tau\) differs by $1$
from that of \(\sigma\).
Hence
\(e(\sigma\tau) = (-1)^{m+1} = -(-1)^m = -e(\sigma)\).

Since the transpositions  generates \(S_n\) and both $e$ and \(\epsilon\)
agree on the transpositions and the identity (\(e(\id)=\epsilon(\id)=1\))
they agree on all permutations.

%%%%%
\begin{excopy}
\begin{itemize}
 \item[(a)]
   Let $n$ be an even positive integer. Show that there exists  a group
   of order \(2n\), generated by two elements \(\sigma\), \(\tau\)
   such that \(\sigma^n=e=\tau^2\), and \(\sigma\tau=\tau\sigma^{n-1}\).
   (Draw a picture of a regular $n$-gon, number the vertices,
   and use the picture as an inspiration to get \(\sigma\), \(\tau\).)
   Thus group is called the
   \index{dihedral} \index{group!dihedral}
   \textbf{dihedral group}.
 \item[(b)]
   Let $n$ be an odd positive integer. Let \(D_{4n}\) be the group generated
   by the matrices
   \begin{equation}
     \left(
      \begin{array}{lr}
       0 & -1 \\
       1 & 0 \\
      \end{array}
     \right)
     \quad\textrm{and}\quad
     \left(
      \begin{array}{lc}
       \zeta & 0 \\
       0 & \zeta^{-1} \\
      \end{array}
     \right)
   \end{equation}
   where \(\zeta\) is a primitive $n$-th root of unity. Show that \(D_{4n}\)
   has order \(4n\), and give the commutation relations between the above
   generators.
\end{itemize}
\end{excopy}

\begin{itemize}
 \item[(a)]
 Let \(\theta=2\pi/n\). Now let\(\sigma\) be a \(1/n\) rotation
 and \(\tau\) a reflection. More formally:
   \begin{equation}
     \sigma = \left(
      \begin{array}{rl}
       \cos\theta & \sin\theta \\
       -\sin\theta & \cos\theta \\
      \end{array}
     \right)
     \quad\textrm{and}\quad
     \tau = \left(
      \begin{array}{lr}
       1 & 0 \\
       0 & -1 \\
      \end{array}
     \right)
   \end{equation}

 \item[(b)]
  Denote
   \begin{equation}
     \sigma = \left(
      \begin{array}{lr}
       0 & -1 \\
       1 & 0 \\
      \end{array}
     \right)
     \quad\textrm{and}\quad
     \tau = \left(
      \begin{array}{lc}
       \zeta & 0 \\
       0 & \zeta^{-1} \\
      \end{array}
     \right)
   \end{equation}

  From that we compute \(\sigma^2 = -\Id\) and \(\sigma^4 = \tau^n = \Id\).
  Thus \(\sigma^2\tau=\tau\sigma^2\) and \(\tau^n\sigma=\sigma\tau^n\).
\end{itemize}

\iffalse
\begin{itemize}
  \item[(a)] Rotation and mirroring. Consider the subgroup of \(S_n\)
     where
   \begin{equation*}
     \sigma(i) =
       \left\{
         \begin{array}{ll}
         i + i \;& \textnormal{if}\; i < n \\
         0     \;& \textnormal{if}\; i = n
         \end{array}
       \right.
   \end{equation*}
   and \(\tau(i) = (n - i + 1)\).

  \item[(b)]
  Say $J$ is the first matrix. Then
  \begin{equation*}
    J^2 =
      \left(
        \begin{array}{rr}
        -1 & 0 \\
        0  & -1
        \end{array}
      \right)
     \quad\textrm{and}\quad
    J^3 =
      \left(
        \begin{array}{rr}
        0 & 1 \\
        -1 & 0
        \end{array}
      \right)
     \quad\textrm{and}\quad
    J^4 =
      \left(
        \begin{array}{rr}
        1 & 0 \\
        0 & 1
        \end{array}
      \right).
  \end{equation*}
\end{itemize}
\fi


%%%%%
\begin{excopy}
Show that there are exactly two non-isomorphic non-abelian groups of order~$8$.
(one of them is given by the generators \(\sigma\), \(\tau\) with the relations
\begin{equation*}
\sigma^4 = 1, \qquad \tau^2 = 1, \qquad \tau\sigma\tau = \sigma^3.
\end{equation*}
The other is the quaternion group.)
\end{excopy}

Let $G$ be non-abelian group of order~$8$.
Let $m$ be the maximal period of the elements of $G$.
Since \(m|8\) we must have \(m\in\{1,2,4,8\}\).

We will show that \(m=4\).
If \(m=1\) then \(|G|=1\), contradiction.
If \(m=2\) then for any \(a,b\in G\)
\begin{equation*}
1 = (ab)(ab) = a(bb)a = (ab)(ba)
\end{equation*}
and so
\begin{equation*}
(ab)^{-1} = ab = ba
\end{equation*}
and $G$ is abelian, contradiction.
If \(m=8\) then $G$ is cyclic, contradiction.

Let \(\sigma\in G\) be of period $4$. It generates
the subgroup \(H=\{\sigma^1,\sigma^2,\sigma^3,1\}\).
\newcommand{\coH}{\ensuremath{\tilde{H}}}
Put  \(\coH = \coH\).
Since \([G:H]=2\) there are two cosets of $H$ and \coH\ in $G$.
Now for any \(g_1,g_2\in \coH\) we have
\begin{equation} \label{eq:H=g1g2H}
H = g_1 g_2 H = g_1 H g_2 = H g_1 g_2.
\end{equation}
and in particular \(H \triangleleft G\).

Since conjunction is automorphism, \(g\sigma g^{-1} \in H\) is
of period $4$ for each \(g\in G\).
Thus
\begin{equation*}
\forall g\in G,\; g\sigma g^{-1} \in \{\sigma^1,\sigma^3\}.
\end{equation*}

Assume by negation \(\exists y\in \coH,\; g\sigma g^{-1}=\sigma\).
Then
\begin{equation*}
\exists y\in \coH\, \forall k\in\{0,1,2,3\},\; g\sigma^kg^{-1}=\sigma^k.
\end{equation*}
Thus
\begin{equation*}
\exists y\in \coH\, \forall h\in H,\; yh=hy.
\end{equation*}
But any \(z \in \coH\) is of the form \(z = yh'\) for some \(h'\in H\)
and so for any \(h\in H\) we have
\begin{equation*}
zh = (yh')h = y(h'h) = (h'h)y = (hh')y = h(h'y) = hz.
\end{equation*}
and now
\begin{equation*}
\forall y\in \coH\, \forall h\in H,\; yh=hy.
\end{equation*}

Let \(y_1,y_2\in \coH\). By looking at \(\coH\) as a coset,
\(y_2 = h y_1 = y_1 h\) for some \(h\in H\).
Now since \((y_1)^2\in H\) as we saw in \eqref{eq:H=g1g2H}
\begin{equation*}
y_1 y_2 = y_1 (h y_1) = y_1 (y_1 h) = (y_1)^2 h = h (y_1)^2 = (h y_1)y_1
= (y_1 h) y_1 = y_2 y_1.
\end{equation*}
Thus $G$ is abelian, and by contradiction
\begin{equation*}
\forall y\in \coH,\; y\sigma y^{-1}=\sigma^3.
\end{equation*}
Similarly,
\begin{eqnarray*}
\forall y\in \coH,\; & y^{-1}\sigma y &= \sigma^3 \\
\forall y\in \coH,\; & y\sigma^3 y^{-1} &= \sigma \\
\forall y\in \coH,\; & y^{-1}\sigma^3 y &= \sigma.
\end{eqnarray*}

Clearly the periods of \(y\in \coH\) could be $2$ or $4$.
If all these periods are $2$ then

Assume the index of $y$ is $2$ for some \(y\in\coH\).
Then \(y^2=1\) and \(u=y^{-1}\) and so
\begin{equation*}
(\sigma y)^2 = \sigma(y \sigma y^{-1}) = \sigma^{1+3}=1.
\end{equation*}
Since conjunction is automorphism, \(y\sigma^2 y^{-1} = \sigma^2\)
begin the only element of order $2$ in $H$.
Noting that \(y\cdot 1 \cdot y{-1} = 1\)
we have
\begin{equation*}
\left\{y\sigma^n y^{-1}: n\in\{0,1,2\}\right\}
=
\left\{y\sigma^n y^{-1}: n\in\{0,3,2\}\right\}.
\end{equation*}
So by looking at the reminder
\(y\sigma^3 y^{-1} = \sigma\).
\begin{equation*}
(\sigma^3 y)^2 = \sigma^3 (y \sigma^3) y = \sigma^{3+1} = 1.
\end{equation*}
Thus all elements of \coH\ are of \emph{equal} period, $2$ or $4$.

Thus we have two possibilities.
\begin{enumerate}

\item If the order of \(y\in\coH\) is $2$ then $G$ is the diehedral group
with the relations specified in the exercise.

\item If the order of \(y\in\coH\) is $8$ then $G$ is the quaternion group.
We put \(i=\sigma\), pick arbitrary \(j\in\coH\), and put \(k=ij\).
Now since the orders of $j$ and $k$ are also $4$,
we have \(|\{i^3,j^3,k^3\}|=3\) (different elements) and
\begin{equation*}
rsr^{-1}=s^3 \qquad \textnormal{where} \qquad
(r,s) \in \left\{(i,j), (i,k), (j,i), (j, k), (k,i), k,j)\right\}.
\end{equation*}
and \(i^2 = j^2 = k^2\) which we conveniently denote as \((-1)\).
With this we have the 3 elements
\begin{eqnarray*}
(-i) &=& i^{3} = (-1)i = i(-1) \\
(-j) &=& j^{3} = (-1)j = j(-1) \\
(-k) &=& k^{3} = (-1)k = k(-1).
\end{eqnarray*}
Also
\begin{eqnarray*}
ji &=& ji(j^{-1}j) = (jij^{-1})j = i^{3}j = i^2k = (-1)k \\
jk &=& j(ij)(i^{-1}i) = (ii^{-1})j(ij) = i(i^{-1}ji)j = ij^{3+1} = i \\
kj &=& (kj)(k^{-1}k) = (kjk^{-1})k = j^{2+1}k = (-1)jk = (-1)i   \\
ik &=& ik(jj^3) = i(kj)j^3 = i^{1+2+1}j^{2+1} = (-1)j \\
ki &=& ki(k^{-1}k) = (kik^{-1})k = i^{2+1}k = (-1)(ik) = (-1)j
\end{eqnarray*}
\end{enumerate}


%%%%%
\begin{excopy}
Let \(\sigma = [123 \ldots n]\) in \(S_n\).
Show that the conjugacy class of \(\sigma\) has \((n - 1)!\) elements.
Show that the centralizer of \(\sigma\) is the cyclic group generated by
\(\sigma\).
\end{excopy}

The following relation on \(S_n\)
\begin{equation*}
\tau_1 \sim \tau_2 
\qquad \textnormal{iff} \qquad
\exists k\in \N_n,\, \tau_1 \sigma^k = \tau_2.
\end{equation*}
is an equivalence relation, since
\(k=0\) gives reflexivity, symmetry by considering 
\(\tau_2 \sigma^{n-k} = \tau_1\) and associativity
since if \(\tau_1 \sim \tau_2\)
and if \(\tau_2 \sim \tau_3\)
with \(k_1\) and \(k_2\) respectively, then
\begin{equation*}
\tau_1 \sigma^{k_3} = \tau_3
\qquad \textnormal{where}\; k_3 = k_1+k_2.
\end{equation*}

For any \(\tau\in S_n\) and \(k \in \N_n\)
\begin{equation*}
(\tau\sigma^k)\sigma\left(\tau\sigma^k\right)^{-1}
= \tau\sigma^{k+1-k}\tau^{-1}
= \tau\sigma\tau^{-1}.
\end{equation*}

Assume \(\tau_1 \nsim \tau_2\).
We choose \(\tau'_1 \sim \tau_1\)
and \(\tau'_2 \sim \tau_2\)
such that 
\(\tau'_1(1) = \tau'_2(1)\).

Clearly we can find some \(j\in\N_n\) such that
\(\tau'_1(j) = \tau'_2(j) = j\)
and
\({\tau'_1}^{-1}(j + 1) \neq {\tau'_2}^{-1}(j + 1)\).
Now
\begin{equation*}
\left(\tau'_1 \sigma {\tau'_1}^{-1}\right)(j')
\neq
\left(\tau'_2 \sigma {\tau'_2}^{-1}\right)(j')
\end{equation*}
Thus 
\begin{equation*}
\tau'_1 \sigma {\tau'_1}^{-1}
=
\tau'_2 \sigma {\tau'_2}^{-1}
\qquad \textnormal{iff} \qquad
\tau'_1 \sim \tau'_2.
\end{equation*}
Thus the size of the conjugacy class of \(\sigma\)
is the same as the number of classes of the equivalence relation \(\sim\)
which is \(n!/n = (n-1)!\).

%%%%%
\begin{excopy}
\begin{enumerate}[(a)]
\item
Let \(\sigma = [i_1\cdots i_m]\) be a cycle.
Let \(\gamma \in S_n\). Show that \(\gamma\sigma\gamma^{-1}\)
is  the cycle \([\gamma(i_1)\cdots \gamma(i_m)]\).
\item
Suppose that a permutation \(\sigma\) in \(S_n\) can be written
as a product of $r$ disjoint
cycles, and let \(d_1,\ldots,d_r\) be the number of elements in each cycle,
in increasing order.
 Let \(\tau\) be another permutation which can be written as a product of
disjoint cycles, whose cardinalities are
\(d'_1,\ldots,d'_r\)
in increasing order. Prove
that \(\tau\) is conjugate to \(\tau\) in \(S_n\) if and only if
\(r = s\) and \(d_i = d'_i\) for all \(i=1,\ldots,r\).
\end{enumerate}
\end{excopy}

\begin{enumerate}[(a)]
\item
\begin{equation*}
\left(\gamma\sigma\gamma^{-1}\right)(\gamma(i_j)) 
= \left(\gamma\sigma\right)\left(\gamma^{-1}\gamma\right)(i_j)
= \left(\gamma\sigma\right)(i_j)
= \gamma(\sigma(i_j))
= \gamma(i_{\bres{(j+1}{m}}).
\end{equation*}
\item
If \(\tau\) is conjugate to \(\sigma\) then the equalities
of the cycles sizes follow from (\emph{a}).
Conversely, if the cycles sizes agree then
if \([i_1,\ldots,i_m]\) is a cycle of \(\sigma\)
and  \([j_1,\ldots,j_m]\) is a cycle of \(\tau\)
we define \(\mu(i_k)=j_k\) for \(k\in \N_n\).
Since \(\N_n\) is a disjoint union of any permutation in \(S_n\)
we have \(\mu\in S_n\) well defined 
and \(\tau = \mu\sigma\mu^{-1}\).
\end{enumerate}

%%%%%
\begin{excopy}
\begin{enumerate}[(a)]
\item
Show that \(S_n\) is generated by the transpositions
\([12]\), \([13]\),\(\ldots\),\([1n]\).
\item
Show that \(S_n\) is generated by the transpositions
\([12]\), \([23]\), \([34]\),\(\ldots\),\([n-1,n]\).
\item
Show that \(S_n\) is generated by the cycles \([12]\) and \([1 2 3 \ldots n]\).
\item
Assume that $n$ is prime. 
Let \(\sigma = [1 2 3 \ldots n]\) and let \(\tau = [rs]\) be any transposition.
Show that \(\sigma\), \(\tau\) generate \(S_n\).
\end{enumerate}
\end{excopy}

Note that given a generator \(g\in S_n\),
we always have \(\Id\in S_n\) gernerated by \(\Id = g^k\) for some \(k\in\N\).
Given \(\sigma,\tau\in S_n\). We define the distance
\begin{equation*}
d(\sigma,\tau) = \left\|\{i\in\N_n: \sigma(i) \neq \tau(i)\}\right|.
\end{equation*}
\begin{enumerate}[(a)]
\item Let \(\sigma\in S_n\). Let \(G\subset S_n\) be the group generated
by the given generators. Let \(g\in G\) be with the minimal
distance with \(sigma\).
If by negation \(d(\sigma,g) > 0\) then let \(j\in\N_n\)
be the minimal index such that \(\sigma(j)\neq g(j)\).
Define
\begin{equation*}
g' = [1 j][1 g^{-1}\sigma(j)][1 j]g
\end{equation*}
and now \(d(\sigma, g') < d(\sigma, g)\) contrdicting the minimal choice.
\item
For each \(k \in \N_n\) we have
\begin{equation*}
[1 k] = [1 2]\cdot[2 3]\cdots[(k-2),(k-1)]\cdot[(k-1),k]\cdots[2 3]\cdot[1 2]
\end{equation*}
Thus we get the generators of previous case (a).
\item
For each \(k \in \N_n\) we have
\begin{equation*}
[k,(k+1)] = [1 2 3\ldots n]^{(k-1)}\cdot[12]\cdot [1 2 3\ldots n]^{n - (k-1)}.
\end{equation*}
Thus we get the generators of previous case (b).
\item
Assume \(d=r-s>0\). for any \(k\in \N_n\)
\begin{equation*}
[k,\bres{(k+d)}{n}] = \sigma^{k-r}[r s]\cdot\sigma^{r-k}
\end{equation*}
and
\begin{equation*}
\tau_k = [\bres{1 + kd}{n}, \bres{1 + (k+1)d}{n}].
\end{equation*}
are all generated.
Since $n$ is prime, for any \(m < n\) there exists \(q\in\N\) 
such that \(d^q = (m - 1) \bmod n\). Thus
\begin{equation*}
[1 m] = \tau_0\cdot\tau_1\cdots\tau_q\cdots\tau_1\cdots\tau_0.
\end{equation*}
Hence the generators of (b) are generated.
\end{enumerate}

\end{myenumerate}

Let $G$ be a finite group operating on a set $S$.
Then $G$ operates in a natural way on
the Cartesian product \(S^{(n)}\) for each positive integer $n$ .
We define the operation on $S$
to be \hbox{\boldmath$n$\textbf{-transitive}} if given $n$ distinct elements 
\((s_1,\ldots,s_n)\) and $n$ distinct elements
\((s'_1,\ldots,s'_n)\) of $S$, there exists \(\sigma\in G\)
such that \(\sigma s_i = s'_i\) for all \(i = 1,\ldots,n\).

\begin{myenumerate}
%%%%%
\begin{excopy}
Show that the action of the alternating group \(A_n\)
on \(\{1,\ldots,n\}\) is \((n - 2)\)-transitive.
\end{excopy}

Given arbitrary
\(n-2\) elements \((s_1,\ldots,s_{n-2})\)  in \(\N_n\)
and
\(n-2\) elements \((s'_1,\ldots,s'_{n-2})\)  in \(\N_n\).
We have a pair of two remaining elements
\begin{eqnarray*}
\{s_{n-1}, s_n\} &= \N_n \setminus \{(s_1,\ldots,s_{n-2}\}\\
\{s'_{n-1}, s'_n\} &= \N_n \setminus \{(s'_1,\ldots,s'_{n-2}\}.
\end{eqnarray*}
For \(i\in\N_n\) define  \(\sigma_1(s_i) = s'_i\) and
\begin{equation*}
  \sigma_2(s_i) = 
    \left\{
      \begin{array}{ll}
        s'_i \quad &\textnormal{iff}\; i \leq n - 2 \\
        s'_n \quad &\textnormal{iff}\; i = n - 1 \\
        s'_{n-1} \quad &\textnormal{iff}\; i = n \\
      \end{array}
    \right.
\end{equation*}
Clearly \(\sigma_1,\sigma_2\in\S_n\) having different signs, thus
(exactly) one of them \(\in A_n\).

%%%%% ex 40
\begin{excopy}
Let \(A_n\) be the alternating group of even permutations of
\(\{1,\ldots,n\}\), For \(j = 1,\ldots,n\)
let \(H_j\) be the subgroup of \(A_n\) fixing $j$, 
so \(H_j \approx A_{n-1}\), and \((A_n: H_j) = n\) for \(n > 3\).
Let \(n \geq 3\) and let $H$ be a subgroup of index $n$ in \(A_n\).
\begin{enumerate}[(a)]
\item
Show that the action of \(A_n\) on cosets of $H$ by left translation
gives an isomorphism \(A_n\) with the alternating group of permutations
of \(A_n/H\).
\item
Show that there exists an automorphism of \(A_n\) mapping \(H_1\) on $H$,
and that
such an automorphism is induced by an inner automorphism of \(S_n\) if and only
if \(H = H_i\) for some~$i$.
\end{enumerate}
\end{excopy}

Note that 
\begin{equation*}
(A_n: H_j) = |A_n|/|A_{n-1}| = (n!/2)/\left((n-1)!/2)\right) = n!/(n-1)! = n.
\end{equation*}
Let \(A_n\) act on cosets \(\tau H_j\) by multiplication from left.
We need to show that this action is well defined.
Let \(\tau_1 H_j = \tau_2 H_j\),
Thus we have \(h_1,h_2\in H_j\) such that  
\(\tau_1 h_1 = \tau_2 h_2\)
and so
\(\sigma\tau_1 h_1 = \sigma\tau_2 h_2\) for all \(\sigma \in S_n\)
and so 
\(\sigma\tau_1 H_j = \sigma\tau_2 H_j\) for all \(\sigma \in A_n\)
showing that the action is well defined.
\begin{enumerate}[(a)]
\item
With similar argunent we showed that the action is well definded,
we can also show that the action is an homomorphism of \(A_n\)
on the permutations of \(A_n/H\).
We still need to show it is an injection and surjection..

We check manually for n=3,4.
\begin{equation*}
A_3 = \{e, [1,2,3], [1, 3, 2]\}
\end{equation*}
and thus \(|H|=1\) so \(H=\{e\}\) and the isomorphism is clear.

\begin{align*}
A_4 = \{&e, \\
  &[1,2,3], [1, 3, 2], [1,2,4], [1, 4, 2], [1, 3, 4], [1, 4, 3], 
    [2, 3, 4], [2, 4, 3], \\
  &[1, 2][3, 4], [1, 3][2, 4], [1, 4][2, 3]\}.
\end{align*}
and this \(|H|=3\) the possible subgroups are of the 
form \(\{e, h, h^2\}\) where 
\(h \in \{[i,j,k]\in A_4: i\neq j \neq k\}\).
Note we always have \(h^3=e\).
Now there are 4 disjoint cosets \(eH, \tau_1 H, \tau_2 H, \tau_3 H\)
where \(\tau_i \in A_n\) for \(i=1,2,3\), and put \(\tau_0 = e\).
To show innjection, let \(\sigma_1, \sigma_2 \in A_4\) 
and assume \(\sigma_2\tau_i H = \sigma_2\tau_i H\) for \(i=0,1,2,3\).
Looking at \(i=0\), we have 
\(\sigma_1(e) \in \{\sigma_2 e, \sigma_2 h, \sigma_2 h^2\}\).\\
Cases:
\begin{itemize}
\item \(\sigma_1 e = \sigma_2 e\). Clearly \(\sigma_1 = \sigma_2\).
\item \(\sigma_1 e = \sigma_2 h\). Then \(\sigma_2^{-1}\sigma_1 = h\).
\item \(\sigma_1 e = \sigma_2 h^2\). 
  Putting \(h' = h^2\) and then \(h'^2 = h^4 = h\).
  So we can apply the previous case.
\end{itemize}

\UNFINISHED

Assume \(\sigma \in A_n\) is in the kernel
of the left multiplication mapping.Then 
\begin{equation} \label{eq:sigma-in-kern:An}
\forall \tau\in A_n\quad \sigma\tau H = \tau H.
\end{equation}
Now
\begin{equation*}
\sigma\tau H = \tau H
\;\Leftrightarrow\;
\tau^{-1}\sigma\tau H = H
\;\Leftrightarrow\;
\tau^{-1}\sigma\tau \in H
\;\Leftrightarrow\;
\sigma \in \tau H \tau^{-1}
\end{equation*}
Thus \eqref{eq:sigma-in-kern:An} gives
\begin{equation*}
\sigma \in \bigcap_{\tau\in A_n} \tau H \tau^{-1}
\end{equation*}
The latter intersection is clearly a normal subgroup of \(A_n\)
Since it is a subgroup of $H$ it is a proper subgroup of \(A_n\)
so it must be the trivial \(\{e\}\) since \(A_n\) is simple.
Thus the left multiplication is injective.


\begin{llem} \label{llem:half:normal}
Let $H$ be a subgroup of $G$. If \([G:H]=2\) then \(H \subnormal G\).
\end{llem}
\begin{proof}
Let \(g \in G\setminus H\) then clearly  \(gH\cap H = \emptyset\)
and since \(|gH| + |H| = |G|\) we have
\begin{equation*}
G = H \dotcup gH = H \dotcup g^{-1}H = H \dotcup Hg = H \dotcup Hg^{-1}.
\end{equation*}
Thus \(gH = Hg\) and so \(gHg^{-1}= (Hg)g^{-1} = H\).
\end{proof}

\begin{llem} \label{llem:unique:Sn:half}
For \(n\geq 2\) there exists a unique subgroup of \(S_n\) of order \(n!/2\)
namely \(A_n\).
\end{llem}
\begin{proof}
Let \(\sigma\;S_n\to \{+1,-1\}\) be the sign group homomorphism.
Its kernal is clearly \(A_n\).
Now let \(H\subset S_n\) with \(|H|=n!/2\).
By local-lemma~\ref{llem:half:normal} \(H\subnormal G\).
% Since \(G/H \simeq C_2 = \{+1,=1\}\) 
% Consider the natural homomorphism \(\Lambda G \to G/H \{+1,=\{+1,=1\}\).
As was shown in the text, \(A_n\) is generated by all 3-cycles
using \([ij][rs] = [ijr]Urs]\).
If by negation \(H \neq A_n\) then both $H$ and \(S_n\setminus H\)
contains some 3-cycles. Say
\begin{equation*}
[ijk] \in H \qquad [xyz]\in S_n\setminus H
\end{equation*}
Say \(I = \{i,j,k\}\cap \{x,y,z\}\).
Without loss of generality we may assume
that if \(|I|\geq 1\) then \(i=x\)
and  if \(|I|=2\) then \(j=y\)
Now consider 
\begin{equation*}
\sigma = 
 \left\{
  \begin{array}{ll}
   {[ix]}[jy][kz] \quad &\textnormal{if}\; |I|=0 \\
   {[jy]}[kz] \quad &\textnormal{if}\; |I|=1 \\
   {[kz]} \quad &\textnormal{if}\; |I|=2 \\
  \end{array}
 \right.
\end{equation*}
Now \([xyz] = \sigma [ijk] \sigma^{-1}\).
And by \(H \subnormal S_n\) we have the contrdiction \([xyz] \in H\).
\end{proof}

Back to the exercise.
The left multiplication $L$ maps \(A_n\) to permutations
of the $n$ cosets of $H$.
Thus we have a one-to-one mapping \(L: A_n \rightarrow S_n\).
Thus \(|L(A_n)|=|A_n|=|S_n|/2 = n!/2\).
By local-lemma~\ref{llem:unique:Sn:half} \(L(A_n)=A_n\).

\item
\end{enumerate}

\UNFINISHED

%%%%%
\begin{excopy}
Let $H$ be a simple group of order \(60\).
\begin{enumerate}[(a)]
\item
Show that the action of $H$ by conjugation on the set of its Sylow subgroups
gives an imbedding \(H \hookrightarrow A_6\).
\item
Using the preceding exercise, show that \(H \approx A_5\).
\item
Show that \(A_6\) has an automorphism which is not induced by an inner
automorphism of \(S_6\).
\end{enumerate}
\end{excopy}

%%%%%
\begin{excopy}
\end{excopy}

%%%%%
\begin{excopy}
\end{excopy}

\end{myenumerate}

%%%%%%%%%%%%%%%%%%%%%%%%%%%%%%%%%%%%%%%%%%%%%%%%%%%%%%%%%%%%%%%%%%%%%%%%
%%%%%%%%%%%%%%%%%%%%%%%%%%%%%%%%%%%%%%%%%%%%%%%%%%%%%%%%%%%%%%%%%%%%%%%%
%%%%%%%%%%%%%%%%%%%%%%%%%%%%%%%%%%%%%%%%%%%%%%%%%%%%%%%%%%%%%%%%%%%%%%%%
\bibliographystyle{plain}
\bibliography{langalg}

%%%%%%%%%%%%%%%%%%%%%%%%%%%%%%%%%%%%%%%%%%%%%%%%%%%%%%%%%%%%%%%%%%%%%%%%
%%%%%%%%%%%%%%%%%%%%%%%%%%%%%%%%%%%%%%%%%%%%%%%%%%%%%%%%%%%%%%%%%%%%%%%%
%%%%%%%%%%%%%%%%%%%%%%%%%%%%%%%%%%%%%%%%%%%%%%%%%%%%%%%%%%%%%%%%%%%%%%%%
% % $Id: langalg.tex,v 1.4 2001/05/04 12:24:45 yotam Exp yotam $
\documentclass[12pt]{book}
\usepackage{fullpage}
\usepackage{amsmath}
\usepackage{amssymb}
\usepackage{amsthm}
% \usepackage{amsthm}
\usepackage{makeidx}
\makeindex % enable

\usepackage{multicol,supertabular}

\setlength{\parindent}{0pt}

% \usepackage{amsmath}

\usepackage{enumerate}

% 'Inspired' by:
%% This is file `uwamaths.sty',
%%%     author   = "Greg Gamble",
%%%     email     = "gregg@csee.uq.edu.au (Internet)",

\makeatletter
\def\DOTSB{\relax}
\def\dotcup{\DOTSB\mathop{\overset{\textstyle.}\cup}}
 \def\@avr#1{\vrule height #1ex width 0pt}
 \def\@dotbigcupD{\smash\bigcup\@avr{2.1}}
 \def\@dotbigcupT{\smash\bigcup\@avr{1.5}}
 \def\dotbigcupD{\DOTSB\mathop{\overset{\textstyle.}\@dotbigcupD%
                               \vphantom{\bigcup}}}

\def\dotbigcupT{\DOTSB\smash{\mathop{\overset{\textstyle.}\@dotbigcupT%
                              \vphantom{\bigcup}}}%
                       \vphantom{\bigcup}\@avr{2.0}}
\def\dotbigcup{\mathop{\mathchoice{\dotbigcupD}{\dotbigcupT}
                                  {\dotbigcupT}{\dotbigcupT}}}
\let\disjunion\dotcup
\let\Disjunion\dotbigcup
\makeatother

\usepackage{amsmath}
\usepackage{amssymb}
% \usepackage{eucal}
\usepackage{mathrsfs}

% \usepackage{fullpage}

\usepackage{geometry}
\geometry{a4paper, left=2cm, right=2cm, top=2cm, bottom=2cm, includeheadfoot}

\setlength{\parindent}{0pt}
\setlength{\parskip}{6pt}


% are we in pdftex ????
\ifx\pdfoutput\undefined % We're not running pdftex
\else
\RequirePackage[colorlinks,hyperindex,plainpages=false]{hyperref}
\def\pdfBorderAttrs{/Border [0 0 0] } % No border arround Links
\fi

% \usepackage{fancyheadings}
\usepackage{fancyhdr}
\usepackage{pifont}

\pagestyle{fancy}
% \addtolength{\headwidth}{\marginparsep}
% \addtolength{\headwidth}{\marginparwidth}
%  \addtolength{\textheight}{2pt}

\newcommand{\ineqjton}{\overset{1\leq i,j \leq n}{i \neq j}}
\newcommand{\srightmark}{\rightmark}
\newcommand{\sfbfpg}{\sffamily\bfseries{\thepage}}
  \newcommand{\symenvelop}{%
     {\nullfont\ }\relax\lower.2ex\hbox{\large\Pisymbol{pzd}{41}}}
% \renewcommand{\chaptermark}[1]{\markboth{\thechapter.\ #1}}

\iffalse
% \lhead[\fancyplain{}{{\sfbfpg}}]{\fancyplain{}\bfseries\srightmark}
\lhead[\fancyplain{}{{\sfbfpg}}]{\fancyplain{}\sl\srightmark}
% \rhead[\fancyplain{}\bfseries\leftmark]{\fancyplain{}{{\sfbfpg}}}
\rhead[\fancyplain{}\sl\leftmark]{\fancyplain{}{{\sfbfpg}}}
\lfoot{\today}
\cfoot{Yotam Medini \copyright}
  \newcommand{\symenvelop}{%
     {\nullfont a}\relax\lower.2ex\hbox{\large\Pisymbol{pzd}{41}}}
\rfoot{\symenvelop\ \texttt{yotam.medini@gmail.com}}

\renewcommand{\headrulewidth}{0.4pt}
\renewcommand{\footrulewidth}{0.4pt}
\fi

\setlength{\headheight}{16pt}
\fancyplain{plain}{%
 \fancyhf{}
 \fancyhead[LE,RO]{\fancyplain{}{{\sfbfpg}}}
 \fancyhead[RE,LO]{\sl\leftmark}
 \fancyfoot[L]{\today}
 \fancyfoot[C]{Yotam Medini \copyright}
 \fancyfoot[R]{\symenvelop\ \texttt{yotam.medini@gmail.com}}
 \renewcommand{\headrulewidth}{0.4pt}
 \renewcommand{\footrulewidth}{0.4pt}
}

% \usepackage{amstex}
% \usepackage{amsmath}
% \usepackage{amssymb}
\usepackage{amsthm}
\usepackage{bm}
\usepackage{makeidx}
\makeindex % enable

% 'Inspired' by:
%% This is file `uwamaths.sty',
%%%     author   = "Greg Gamble",
%%%     email     = "gregg@csee.uq.edu.au (Internet)",

\makeatletter
\def\DOTSB{\relax}
\def\dotcup{\DOTSB\mathop{\overset{\textstyle.}\cup}}
 \def\@avr#1{\vrule height #1ex width 0pt}
 \def\@dotbigcupD{\smash\bigcup\@avr{2.1}}
 \def\@dotbigcupT{\smash\bigcup\@avr{1.5}}
 \def\dotbigcupD{\DOTSB\mathop{\overset{\textstyle.}\@dotbigcupD%
                               \vphantom{\bigcup}}}

\def\dotbigcupT{\DOTSB\smash{\mathop{\overset{\textstyle.}\@dotbigcupT%
                              \vphantom{\bigcup}}}%
                       \vphantom{\bigcup}\@avr{2.0}}
\def\dotbigcup{\mathop{\mathchoice{\dotbigcupD}{\dotbigcupT}
                                  {\dotbigcupT}{\dotbigcupT}}}
\let\disjunion\dotcup
\let\Disjunion\dotbigcup
\makeatother


\newcommand{\half}{\ensuremath{\frac{1}{2}}}



\newcommand{\C}{\ensuremath{\mathbb{C}}} % The Complex set
\newcommand{\aded}{\ensuremath{\textrm{a.e.}}} % almost everyehere
\newcommand{\chhi}{\raise2pt\hbox{\ensuremath\chi}}           %raise the chi
\newcommand{\calA}{\ensuremath{\mathcal{A}}}
\newcommand{\calB}{\ensuremath{\mathcal{B}}}
\newcommand{\calE}{\ensuremath{\mathcal{E}}}
\newcommand{\calF}{\ensuremath{\mathcal{F}}}
\newcommand{\calG}{\ensuremath{\mathcal{G}}}
\newcommand{\calM}{\ensuremath{\mathcal{M}}}
\newcommand{\calR}{\ensuremath{\mathcal{R}}}
\newcommand{\eqdef}{\ensuremath{\stackrel{\mbox{\upshape\tiny def}}{=}}}
\newcommand{\frakB}{\ensuremath{\mathfrak{B}}}
\newcommand{\frakC}{\ensuremath{\mathfrak{C}}}
\newcommand{\frakF}{\ensuremath{\mathfrak{F}}}
\newcommand{\frakG}{\ensuremath{\mathfrak{G}}}
\newcommand{\frakI}{\ensuremath{\mathfrak{I}}}
\newcommand{\frakM}{\ensuremath{\mathfrak{M}}}
\newcommand{\scrA}{\ensuremath{\mathscr{A}}}
\newcommand{\scrB}{\ensuremath{\mathscr{B}}}
\newcommand{\scrD}{\ensuremath{\mathscr{D}}}
\newcommand{\scrF}{\ensuremath{\mathscr{F}}}
\newcommand{\scrN}{\ensuremath{\mathscr{N}}}
\newcommand{\scrP}{\ensuremath{\mathscr{P}}}
\newcommand{\scrQ}{\ensuremath{\mathscr{Q}}}
\newcommand{\scrR}{\ensuremath{\mathscr{R}}}
\newcommand{\scrT}{\ensuremath{\mathscr{T}}}
\newcommand{\Lp}[1]{\ensuremath{\mathbf{L}^{#1}}} % Lp space
\newcommand{\N}{\ensuremath{\mathbb{N}}} % The Natural Set
\newcommand{\bbP}{\ensuremath{\mathbb{P}}} % Some partially ordered set
\newcommand{\Q}{\ensuremath{\mathbb{Q}}} % The Rational set
\newcommand{\R}{\ensuremath{\mathbb{R}}} % The Real Set
\newcommand{\T}{\ensuremath{\mathbb{T}}} % The Thorus [-pi,\pi)
\newcommand{\Z}{\ensuremath{\mathbb{Z}}} % The Integer Set
\newcommand{\intR}{\int_{-\infty}^{\infty}} % Integral over the reals
\newcommand{\posthat}[1]{#1{\,\hat{}\,}}

% sequences
\newcommand{\seq}[2]{\ensuremath{#1_1,\ldots,#1_{#2}}}
\newcommand{\seqn}[1]{\seq{#1}{n}}
\newcommand{\seqan}{\seq{a}{n}}
\newcommand{\seqxn}{\seq{x}{n}}
\newcommand{\seqalphn}{\seq{\alpha}{n}}

\newcommand{\mset}[1]{\ensuremath{\{#1\}}}


%%%%%%%%%%%%
%% math op's
\newcommand{\Alt}{\mathop{\rm Alt}\nolimits}
\newcommand{\Ang}{\mathop{\rm Ang}\nolimits}
\newcommand{\Arg}{\mathop{\rm Arg}\nolimits}
\newcommand{\co}{\mathop{\rm co}\nolimits}
\newcommand{\conv}{\mathop{\rm conv}\nolimits}
\newcommand{\diam}{\mathop{\rm diam}\nolimits}
\newcommand{\dom}{\mathop{\rm dom}\nolimits}
% \newcommand{\dim}{\mathop{\rm dim}\nolimits}
% \newcommand{\esssup}{\mathop{\rm ess\ sup}\nolimits}
\DeclareMathOperator*{\esssup}{ess\,sup}
\newcommand{\ext}{\mathop{\rm ext}\nolimits}
\newcommand{\Id}{\mathop{\rm Id}\nolimits}
\newcommand{\Image}{\mathop{\rm Im}\nolimits}
\newcommand{\Ind}{\mathop{\rm Ind}\nolimits}
\newcommand{\Lip}{\mathop{\rm Lip}\nolimits}
\newcommand{\lip}{\mathop{\rm lip}\nolimits}
\newcommand{\percB}{
  \mathbin{\ooalign{$\hidewidth\%\hidewidth$\cr$\phantom{+}$}}}
\newcommand{\bres}[2]{\ensuremath{#1 \percB #2}}

\newcommand{\Ker}{\mathop{\rm Ker}\nolimits}
\newcommand{\rank}{\mathop{\rm rank}\nolimits}
\newcommand{\rng}{\mathop{\rm rng}\nolimits}
\newcommand{\Res}{\mathop{\rm Res}\nolimits}
\newcommand{\supp}{\mathop{\rm supp}\nolimits}
\newcommand{\vol}{\mathop{\rm vol}\nolimits}
\newcommand{\vspan}{\mathop{\rm span}\nolimits}

% I wish this was more standardized
\renewcommand{\Re}{\mathop{\bf Re}\nolimits}
\renewcommand{\Im}{\mathop{\bf Im}\nolimits}

\newcommand{\inter}[1]{\ensuremath{#1^{\circ}}}  % interior
\newcommand{\closure}[1]{\ensuremath{\overline{#1}}} % closure
\newcommand{\boundary}[1]{\ensuremath{\partial #1}} % closure


\newcommand{\ich}[1]{(\textit{#1})}
\newcommand{\itemch}[1]{\item[\ich{#1}]}
\newcommand{\itemdim}{\item[\(\diamond\)]}

% Special names
\newcommand{\Cech}{\u{C}ech}

\author{Yotam Medini}


%%%%%%%%%%%
%% Theorems
%%
\makeatletter
\@ifclassloaded{book}{
 \newtheorem{thm}{Theorem}[chapter]
 \newtheorem{cor}[thm]{Corollary}
 \newtheorem{lem}[thm]{Lemma}
 \newtheorem{llem}[thm]{Local Lemma}
 \newtheorem{lthm}[thm]{Local Theorem}
 % \newtheorem{quotecor}{Corollary}
 % \newtheorem{quotelem}{Lemma}[section]
 \newtheorem{quotethm}{Theorem}[chapter]
}{}
\makeatother
\newtheorem{Def}{Definition}

\newtheorem{manualtheoreminner}{Theorem}
\newenvironment{manualtheorem}[1]{%
  \renewcommand\themanualtheoreminner{#1}%
  \manualtheoreminner
}{\endmanualtheoreminner}

\newtheorem{manuallemmainner}{Lemma}
\newenvironment{manuallemma}[1]{%
  \renewcommand\themanuallemmainner{#1}%
  \manuallemmainner
}{\endmanuallemmainner}

\newcommand{\loclemma}{Lemma}


% \newcommand{\proofend}{\(\bullet\)}
% \newcommand{\proofend}{\hfill\(\blacksquare\)}
\newcommand{\proofend}{\hfill\(\Box\)}
\newenvironment{thmproof}
{\textbf{Proof.}}
{\proofend}

\newcommand{\chapterTypeout}[1]{\typeout{#1} \chapter{#1}}
\newcommand{\sectionTypeout}[1]{\typeout{#1} \section{#1}}

% abbreviations, ensuremath
\newcommand{\fx}{\ensuremath{f(x)}}
\newcommand{\gx}{\ensuremath{g(x)}}
\newcommand{\lrangle}[1]{\ensuremath{\left\langle #1 \right\rangle}}
\newcommand{\lrbangle}[1]{\ensuremath{\left\langle #1 \right\rangle}}
\newcommand{\M}{\ensuremath{\mathfrak{M}}}
\newcommand{\mldots}{\ensuremath{\ldots}}
\newcommand{\salgebra}{\(\sigma\)-algebra}
\newcommand{\swedge}{\;\wedge\;}
\newcommand{\wlogy}{without loss of generality}
\newcommand{\Wlogy}{Without loss of generality}
\newcommand{\twopii}{\ensuremath{2\pi i}}
\newcommand{\dtwopii}{\ensuremath{\frac{1}{\twopii}}}

% https://tex.stackexchange.com/
% questions/22252/how-to-typeset-function-restrictions
\newcommand\restr[2]{\ensuremath{% we make the whole thing an ordinary symbol
  \left.\kern-\nulldelimiterspace % automatically resize the bar with \right
  #1 % the function
  \vphantom{\big|} % pretend it's a little taller at normal size
  \right|_{#2} % this is the delimiter
  }}

\newenvironment{excopyOLD}
{\item\begin{minipage}[t]{.8\textwidth}\footnotesize}
{\smallskip\hrule\end{minipage}}

\newenvironment{excopy}
{\item % \relax
 \begin{list}{}{
 \setlength{\topsep}{0pt}
 \setlength{\partopsep}{0pt}
 \setlength{\itemsep}{0pt}
 \setlength{\parsep}{0pt}
 \setlength{\leftmargin}{0pt}
 \setlength{\rightmargin}{20pt}
 \setlength{\listparindent}{0pt}
 \setlength{\itemindent}{0pt}
 % \setlength{\labelsep}{0pt}
 \setlength{\labelwidth}{0pt}
 \footnotesize
 }
 \item
}
{\par
 % {\nullfont 0}
 \hrulefill
 \end{list}
}


\title{
 Notes and Solutions to Exercises\\
 for\\
 ``Algebra'' \quad by\quad  Serge Lang}
\author{Yotam Medini\\\texttt{yotam\_medini@yahoo.com}}

\newcommand{\Zm}[1]{\Z/#1\Z} % The Cyclic group

% Trivial group
\newcommand{\eG}{\ensuremath{\{e\}}}

\newcommand{\UNFINISHED}{\large\textbf{UNFINISHED!}}

% \newcommand{\disjunion}{\.\cup}      % Some use \sqcup or \uplus
% \newcommand{\Disjunion}{\.\bigsqcup} % Some use \bigsqcup or \biguplus
% \newcommand{\disjunion}{{\bigsqcup}}
% \newcommand{\Disjunion}{\bigsqcup}

\def\Aut{\mathop{\rm Aut}\nolimits}
\def\card{\mathop{\rm card}\nolimits}
\def\Ch{\mathop{\rm Ch}\nolimits}
\def\Id{\mathop{\rm Id}\nolimits}
\def\id{\mathop{\rm id}\nolimits}
\def\Inn{\mathop{\rm Inn}\nolimits}
\def\Irr{\mathop{\rm Irr}\nolimits}
\def\Ker{\mathop{\rm Ker}\nolimits}
\def\Map{\mathop{\rm Map}\nolimits}
\def\Stab{\mathop{\rm Stab}\nolimits}
\def\subnormal{\vartriangleleft}

% \renewenvironment{excopy}
% {\begin{minipage}[t]{.8\textwidth}\footnotesize}
% {\smallskip\hrule\end{minipage}}


\newcounter{myenumi}
\newenvironment{myenumerate}
{\begin{enumerate}
 \setcounter{enumi}{\themyenumi}
}
{\setcounter{myenumi}{\theenumi}
 \end{enumerate}}

% End of proof
% \newcommand{\eop}{{\small\quad\(\square\)}}

% \newtheorem{thm}{Theorem}[chapter]
% \newtheorem{cor}[thm]{Corollary}
% \newtheorem{lem}[thm]{Lemma}
% \newtheorem{llem}[thm]{Local Lemma}


\begin{document}
\maketitle
\newpage
\tableofcontents
\newpage

%%%%%%%%%%%%%%%%%%%%%%%%%%%%%%%%%%%%%%%%%%%%%%%%%%%%%%%%%%%%%%%%%%%%%%%%
%%%%%%%%%%%%%%%%%%%%%%%%%%%%%%%%%%%%%%%%%%%%%%%%%%%%%%%%%%%%%%%%%%%%%%%%
%%%%%%%%%%%%%%%%%%%%%%%%%%%%%%%%%%%%%%%%%%%%%%%%%%%%%%%%%%%%%%%%%%%%%%%%
\chapter*{Introduction}

This document is a companion to \cite{Lan94}.

%%%%%%%%%%%%%%%%%%%%%%%%%%%%%%%%%%%%%%%%%%%%%%%%%%%%%%%%%%%%%%%%%%%%%%%%
%%%%%%%%%%%%%%%%%%%%%%%%%%%%%%%%%%%%%%%%%%%%%%%%%%%%%%%%%%%%%%%%%%%%%%%%
\section*{Notation}

For each natural \(n\in\N\) we define
\begin{equation*}
\N_n \eqdef \{m\in\N: 1\leq m \leq n\} \qquad
\Z_n \eqdef \{m\in\Z: 0\leq m < n\}.
\end{equation*}

We borrow \textbf{C}-programming language modulo operator
to define the residue function
\begin{equation*}
\bres{n}{d} = n - n \left\lfloor \frac{n}{d} \right\rfloor
\end{equation*}


%%%%%%%%%%%%%%%%%%%%%%%%%%%%%%%%%%%%%%%%%%%%%%%%%%%%%%%%%%%%%%%%%%%%%%%%
%%%%%%%%%%%%%%%%%%%%%%%%%%%%%%%%%%%%%%%%%%%%%%%%%%%%%%%%%%%%%%%%%%%%%%%%
%%%%%%%%%%%%%%%%%%%%%%%%%%%%%%%%%%%%%%%%%%%%%%%%%%%%%%%%%%%%%%%%%%%%%%%%
\chapter{Groups}

%%%%%%%%%%%%%%%%%%%%%%%%%%%%%%%%%%%%%%%%%%%%%%%%%%%%%%%%%%%%%%%%%%%%%%%%
%%%%%%%%%%%%%%%%%%%%%%%%%%%%%%%%%%%%%%%%%%%%%%%%%%%%%%%%%%%%%%%%%%%%%%%%
\section{Notes}

%%%%%%%%%%%%%%%%%%%%%%%%%%%%%%%%%%%%%%%%%%%%%%%%%%%%%%%%%%%%%%%%%%%%%%%%
\subsection{Fixed proof of 5.5}

In an old edition:

Page~33 in the proof of Theorem~5.5.

The last paragraph of the proof deals with the case
in which the orbit of \(\langle\sigma\rangle\) has \(\geq3\) elements.
With the defined \(\tau = [krs]\) the text claims that
with \(\sigma' =  \tau\sigma\tau^{-1}\sigma^{-1}\) we have
\(\sigma'(i) = i\).

If we set \(\sigma = [ijkrs]\) then \(\sigma'(i)\neq i\) \emph{contrary}
to what is claimed in the text!

With the defined \(\tau = [krs]\) we actually have in the case
\begin{eqnarray}
\sigma'(i) & = & \tau\sigma\tau^{-1}\sigma^{-1}(i) \\
           & = & \tau\sigma\tau^{-1}(s) \\
           & = & \tau\sigma(r) \\
           & = & \tau(s) \\
           & = & k \neq i
\end{eqnarray}

In the current Third Edition, \(\tau = [rsk]\).

%%%%%%%%%%%%%%%%%%%%%%%%%%%%%%%%%%%%%%%%%%%%%%%%%%%%%%%%%%%%%%%%%%%%%%%%
\subsection{Lemma 8.3}
The \(c\in A_1\) may be explicitly taken as
\[c = p^{k-r}\mu a_1.\]

%%%%%%%%%%%%%%%%%%%%%%%%%%%%%%%%%%%%%%%%%%%%%%%%%%%%%%%%%%%%%%%%%%%%%%%%
%%%%%%%%%%%%%%%%%%%%%%%%%%%%%%%%%%%%%%%%%%%%%%%%%%%%%%%%%%%%%%%%%%%%%%%%
\section{Exercises (page 75)}

%%%%%%%%%%%%%%%%
\begin{myenumerate}
\addtolength{\itemsep}{10pt}

%%%%%
\begin{excopy}
Show that every group of order \(\leq 5\) is abelian.
\end{excopy}

If \(|G|=1\) then \(G=\eG\) and it is obvious.
For \(|G|\in\{2,3,5\}\) then the order is prime and $G$ is cyclic
and therefore abelian.

Now assume \(|G|=4\). If $G$ is cyclic then it is abelian.
If $G$ is not cyclic then \(G=(\Zm{2})\times(\Zm{2})\)
and a simple check verifies that $G$ is abelian.

%%%%%
\begin{excopy}
Show that there are two non-isomorphic groups of order $4$,
namely the cyclic one and the product of two cyclic groups of order $2$.
\end{excopy}

Note: This was actually used in the solution of previous exercise.
Let $G$ be a non cyclic group of order $4$ and
\(G = \{e, g_1, g_2, g_3\}\).
Since \(g_i^4=e\) for all \(i=1,2,3\) and therefore we must also
have \(g_i^2=e\) for all \(g_i\) since otherwise \(\{g_i^k\}_{k=0}^3\)
generates $G$ contradicting it being non-cyclic.
Denote \(h = g_1 g_2\). This product $h$
 cannot be equal to $e$ since \(g_1\)  and \(g_2\)
The product $h$ cannot be equal to either \(g_1\) nor \(g_2\)
Since neither of them is the unit.
Thus \(g_1 g_2 = g_3\). With this the homomorphism
of $G$ onto \((\Zm{2})\times(\Zm{2})\)
generated by:
\begin{eqnarray*}
 g_1 & \rightarrow &  (1,0) \\
 g_2 & \rightarrow &  (0,1) \\
\end{eqnarray*}
exists and satisfies isomorphism.


%%%%%
\begin{excopy}
Let $G$ be a group. A \textbf{commutator} in $G$ is
an element of the form \(aba^{-1}b^{-1}\) with \(a,b\in G\).
Let \(G^c\) be the subgroup generated by the commutators.
Then \(G^c\) is called the \textbf{commutator subgroup}.
Show that \(G^c\) is normal. Show that any homomorphism of
$G$ into an abelian group factors through \(G/G^c\).
\end{excopy}

Let \(g\in G\) and \(m\in G^c\).
By definition \(g^{-1}mgm^{-1} \in G^c\) and so
\(g^{-1}mg \in G^c m = G^c\) and \(G^c\) is normal.

Let $A$ be an abelian group and
\(h: G\rightarrow A\) a group homomorphism.
We need to show that \(G^c \subseteq \Ker h\).
It is sufficient to show it on the generators
\begin{eqnarray*}
h(aba^{-1}b^{-1}) & = & h(a)h(b)h(a^{-1})h(b^{-1}) \\
                  & = & h(a)h(a^{-1})h(b)h(b^{-1}) \\
                  & = & h(aa^{-1})h(bb^{-1}) \\
                  & = & h(e_G)h(e_G) = e_A\\
\end{eqnarray*}


%%%%%
\begin{excopy}
Let \(H,K\) be subgroups of a finite group $G$
with \(K \subset N_H\). Show that
\[\#(HK) = \frac{\#(H)\#(K)}{\#(H\cap K)}.\]
\end{excopy}

In example (\textbf{iv}) of \S 3,
% On page~17 (Example \textbf{iv})
it was shown that \emph{when} $H$ is contained in the normalizer of $K$, then
\[H/(H\cap K) \approx HK/K.\]
From this we get
\[\#(H)/\#(H\cap K) = \#(HK)/\#(K)\]
and the desired equality follows for the special case.

Now for the general case. Put \(G = H\cap K\).
For each \(h\in H\), \(k\in K\) and \(g\in G\)
we have \(hk = (hg^{-1})(gk)\).
Therefore
\begin{equation*}
\#(HK) \geq \frac{\#(H)\#(K)}{\#(G)}.
\end{equation*}
Conversely, given \(h_1\in H\) and \(k_1\in K\),
for any \(h_2\in H\) and \(k_2\in K\) satisfying:
\begin{equation*}
  h_1 k_1 = h_2 k_2 % \qquad\textnormal{where} \h_i\in H \land k_i = K\)
\end{equation*}
we have \(g = h_2^{-1}h_1 = k_2k_1^{-1} \in G\).
Now \(h_2 = h_1 g^{-1}\) and \(k_2 = gk_1\).
Therefore
\begin{equation*}
\#(HK) \leq \frac{\#(H)\#(K)}{\#(G)}.
\end{equation*}
and the desired equality follows.

%%%%%
\begin{excopy}
{\normalsize (Note: Using different notation.)}\newline
\textbf{Goursat's Lemma.} Let \(G_1, G_2\) be groups,
and let $H$ be a subgroup of \(G_1\times G_2\) such that
the two projections
\(p_i:H\rightarrow G_i\) for \(i=1,2\) are surjective.
Let \(N_1\) be the kernel of \(p_2\)
and \(N_2\) be the kernel of \(p_1\).
One can identify \(N_i\) as a normal subgroup of \(G_i\) (\(i=1,2\)).
Show that the image of $H$
in \(G_1/N_1 \times G_2/N_2\) is the graph of an isomorphism
\[G_1/N_1 \approx G_2/N_2.\]
\end{excopy}

It is easy to see that
\begin{eqnarray*}
\Ker{p_1} & = & H \cap (\{e_G\}\times G')\\
\Ker{p_2} & = & H \cap (G\times\{e_{G'}\})\\
\end{eqnarray*}

We have the natural mappings
\[\widetilde{p_i}: H \rightarrow G_i/N_i\qquad(i=1,2).\]
We first need to show that the graph of \((\widetilde{p_1},\widetilde{p_2})\)
defines a surjective bijection function
\begin{equation}\label{eq:g1n1g2n2}
G_1/N_1 \rightarrow G_2/N_2.
\end{equation}
The natural mapping \(H\rightarrow G_1/N_1\) is surjective
and the graph covers \(G_1/N_1\).
Hence it suffices to show that for any element of \(G_1/N_1\)
there is only one associated element in \(G_2/N_2\).
Let
\((g_1,g_2),(g'_1,g'_2)\) be any elements of $H$ such that
\[\widetilde{p_1}((g_1,g_2)) = \widetilde{p_1}((g'_1,g'_2)).\]
It is obvious that
\((g_1^{-1}g'_1,g_2^{-1}g'_2) \in \Ker(\widetilde{p_1})\)
and so \(g_2^{-1}g'_2\in N_2\) and thus \(g_2 N_2 = g'_2 N_2\).
This shows
\[\widetilde{p_2}(\left(g_1,g_2\right)) =
  \widetilde{p_2}(\left(g'_1,g'_2\right))\]
and the graph defines the mapping of (\ref{eq:g1n1g2n2}) we had to show.
Similarly, we can show the inverse mapping
\[G_2/N_2 \rightarrow G_1/N_1\]
and thus the graph is an surjective bijection.

Now to show homomorphism. Let
\(x_1N_1,y_1N_1 \in G_1/N_1\).
For \(i=1,2\)
There must be some
% \((a_i,b_i)\in H\) so
\((a_1,a_2),(b_1,b_2)\in H\) so
\(a_iN = x_iN\) and
\(b_iN = y_iN\).
The mapping (\ref{eq:g1n1g2n2}) we have established has:
\(x_1N_1 \rightarrow a_2N2\) and
\(y_1N_1 \rightarrow b_2N2\).
By looking at
\((a_1 b_1,a_2 b_2)\in H\) we get
\(x_1 y_1 N_1 \rightarrow a_2 b_2 N2\).


%%%%%
\begin{excopy}
Prove that the group of inner automorphisms of a group $G$
is normal in \(Aut(G)\).
\end{excopy}

The \textbf{inner} automorphisms
(\(\Inn(G)\), See: \cite{Scott87})
are the conjunctions.

Let \(T\in \Aut(G)\) and
let \(\gamma_x\in \Inn(G)\).
For \(y\in G\) We have
\(\gamma_x(y) = xyx^{-1}\) and
\begin{eqnarray*}
(T\gamma_x T^{-1})(y)
  & = & T(xT^{-1}(y)x^{-1})\\
  & = & T(x)T(T^{-1}(y))T(x^{-1})\\
  & = & T(x)yT(x^{-1})\\
  & = & \gamma_{T(x)}(y).\\
\end{eqnarray*}
So \(T\gamma_x T^{-1} \in \Inn(G)\) and \(\Inn(G)\) is normal.

%%%%%
\begin{excopy}
Let $G$ be a group such that \(\Aut(G)\) is cyclic.
Prove that $G$ is abelian.
\end{excopy}

Let \(T\in\Aut(G)\) be a generator.
For all \(g\in G\) we denote the inner mapping \(\gamma_g(x)=gxg^{-1}\).
So for any \(g\in G\) there exists a minimal \(n(g)\geq 0\) such that
\(T^{n(g)} = \gamma_g\). G is abelian iff \(n(g)=0\) for all \(g\in G\).
In negation we can assume there is some \(t \in G\)
with minimal \(m=n(t)>0\). We will show that \(\gamma_t\)
generates all of \(\Inn(G)\).

Let \(g\in G\), \(d=\lfloor n(g)/m\rfloor\) and the residue
\(r = n(g) - md\) where \(0\leq r<m\).
% Now \(\gamma_g = T^{n(g)} = T^r(T^m)^{d} = T^r \gamma_t^d\).
Set \(g'=gt^{-d}\) and we get
\[\gamma_{g'}(x) = (T^{n(g)}(T^m)^d)(x) = T^{n(g)-md}(x) = T^r(x).\]
By minimality of $m$ we must have \(r=0\).

So for all \(g\in G\) there exists a \(d\geq0\) such that
\(T^{md}=\gamma_g\) that is for all \(x\in G\)
\(t^{md}xt^{-md} = gxg^{-1}\). Substituting $x$ with $t$ we get
\(t = t^{md}tt^{-md} = gtg^{-1}\) from which we get \(gt=tg\)
and \(T^m\) is the identity that generates \(\Inn(G)\).
Since \(\Inn(G)=\{\Id_G\}\) we conclude that $G$ is abelian.



%%%%%
\begin{excopy}
Let $G$ be a group and let \(H, H'\) be subgroups.
By a \textbf{double coset} of \(H, H'\) one means
a subset of $G$ of the form \(HxH'\).
\begin{itemize}
  \item[(a)] Show that $G$ is a disjoint union of double cosets.
  \item[(b)] Let $C$ be a family of representatives for
     double cosets. For each \(a \in G\) denote by \([a]H'\)
     the conjugate \(aH'a^{-1}\) of \(H'\).
     For each \(c\in C\) we have a decomposition into ordinary cosets
  \begin{equation} \label{eq:decoxcHHp}
    H = \Disjunion_{x\in X_c} x(H\cap[c]H'),
    % H = \Disjunion_{x\in X_c} x(H\cap[c]H'),
  \end{equation}
     where \(X_c\) is a family of elements of $H$, depending on $c$.
     Show that the elements
     \(\{xc: c\in C,\,x\in X_c\}\) form a family of left cosets
     representatives for \(H'\) in $G$; that is,
  \begin{equation} \label{eq:decoGxccHp}
    G = \Disjunion_{c\in C}\,\Disjunion_{x\in X_c} xcH',
  \end{equation}

  \begin{quote}
   \textbf{Note:} In the original text,
   equation (\ref{eq:decoxcHHp})  appears as
   \[ H = \bigcup_c x_c(H\cap[c]H'),\]
   and equation (\ref{eq:decoGxccHp}) appears as
   \[ G = \bigcup_{x_c}\,\bigcup_{x_c} x_c cH'.\]
  I believe these are notational mistakes
   or unclear indices style.
  \end{quote}

\end{itemize}
\end{excopy}

To prove (a) let us assume that for some \(x_1,x_2\in G\)
\((Hx_1H')\cap(Hx_2H')\neq\emptyset\).
So we have \(h_1,h_2\in H\) and \(h'_1,h'_2\in H'\) so
\(h_1 x_1 h'_1 = h_2 x_2 h'_2\).
So we
\(x_1 = ({h_1}^{-1}h_2) x_2 (h'_2{h'_1}^{-1}).\)
Now for any \(g\in Hx_1H'\) there are
\(h\in H,h'\in H'\) such that we have
\[g = hx_1h' = (h{h_1}^{-1}h_2) x_2 (h'_2{h'_1}^{-1}) \in Hx_2H'.\]
Thus \(Hx_1H'\subset Hx_2H'\). Similarly,
\(Hx_2H'\subset Hx_1H'\) and they are equal and so the double cosets
are disjoint.

Now we turn to (b). For each \(c\in C\)
\(H_c = H\cap(cH'c^{-1})\subseteq H\). So we have
cosets of \(H_c\) form a disjoint union
\[H = \disjunion_{x\in X_c} xH_c = \disjunion_{x\in X_c} xcH'c{-1}\]
with some family \(X_c\) of representatives.

We compute
\begin{eqnarray} \label{eq:hch}
 HcH' & = & \left(\Disjunion_{x\in X_c} xcH'c^{-1}\right)cH' \\
      & = & \left(\Disjunion_{x\in X_c} xcH'c^{-1}c\right)H' \nonumber \\
      & = & \left(\Disjunion_{x\in X_c} xcH'\right)H' \nonumber\\
      & = & \Disjunion_{x\in X_c} xcH'H' \nonumber\\
      & = & \Disjunion_{x\in X_c} xcH'  \nonumber
\end{eqnarray}

From (a) we have a family $C$ of representatives of the double cosets
and using \ref{eq:hch} for substitution we get:
\[
G = \Disjunion_{c\in C} HcH'\\
  = \Disjunion_{c\in C} \, \Disjunion_{x\in X_c} xcH'.\]



%%%%%
\begin{excopy}
\begin{itemize}
 \item[(a)] Let $G$ be a group and $H$ a subgroup of finite index.
   Show that there exists a normal subgroup $N$ of $G$ contained in $H$
   and also of a finite index. [\emph{Hint}: If \((G:H)=n\),
   find a homomorphism if $G$ into \(S_n\) whose kernel is contained in $H$.]
 \item[(b)] Let $G$ be a group and let \(H_1\), \(H_2\) be subgroups of
   finite index. Prove that \(H_1\cap H_2\) has finite index.
\end{itemize}
\end{excopy}

Let $G$ act on the set of cosets of $H$ with \(\pi_g(cH) = gcH\)
for every \(g\in G\) and ever coset \(cH\) with \(c\in G\).
We want to show that this definition is independent of choice of
the $c$ representative.
So let \(c_1H = c_2H\), % so there must be some \(h\in H\)
and by associativity of the group multiplication
\[\pi_g(c_1H) = g(c_1H) = g(c_2H) = \pi_g(c_2)\]
and for each \(g_1,g_2\in G\), \(c\in G\)
\begin{eqnarray} \label{eq:pig1g2}
\pi_{g_1g_2}(cH) = (g_1g_2)(cH) = g_1(g_2(cH)) =  g_1(\pi_{g_2}cH) \\
  \hfill = \pi_{g_1}(\pi_{g_2}cH) = (\pi_{g_1}\pi_{g_2})(cH). \nonumber
\end{eqnarray}
The set of $H$ has $n$ elements.
By labeling the cosets
and from (\ref{eq:pig1g2})
we see that the association \(g\rightarrow \pi_g\)
is homomorphism\(T:G\rightarrow S_n\).
We have \(\Ker T \subseteq H\) since
for every \(g\in G\setminus H\) we have \(gH\neq H\).
Obviously, \(G/\Ker(T) \equiv T(G) \subseteq S_n\) and so
\[[G:H] \leq [G:\Ker(T)] = |T(G)| \leq |S_n| = n! < \infty.\]

To prove (b) we can assume that \(H_1\), \(H_2\) are normal.
Otherwise we simply use (a) and substitute them with normal subgroups.
Now we can use Example~(iv) of \S~3 (page~17)
\begin{equation}
H_1/(H_1\cap H_2) = H_1 H_2 / H_2
\end{equation}
that gives
\begin{equation}
[H_1:(H_1\cap H_2)] =
|G/(H_1\cap H_2)| =
|H_1 H_2 / H_2| \leq
|G/H_2|
\end{equation}
and so
\begin{equation}
[G:(H_1\cap H_2)] =
[G:H_1]\cdot[H_1:(H_1\cap H_2)] \leq
|G/H_1|\cdot|G/H_2| < \infty.
\end{equation}



%%%%%
\begin{excopy}
Let $G$ be a group and let $H$ be a subgroup of a finite index.
Prove that there is only a finite number of right cosets of $H$, and that
the number of right cosets is equal to the number of left cosets.
\end{excopy}

We will show a one to one surjective mapping between
the left cosets to the right cosets. It is defined by
\begin{equation} \label{eq:xH2Hx}
xH \rightarrow Hx^{-1} \qquad\textrm{for\ } x\in G
\end{equation}

The map is obviously surjective since \(x\rightarrow x^{-1}\) is.
Now assume \(Hx^{-1} = Hy^{-1}\). Then  \(x^{-1}y\in H\)
and so \((x^{-1}y)^{-1} = y^{-1}x \in H\) and so \(xH = yH\)
and so (\ref{eq:xH2Hx}) is one to one.

%%%%%
\begin{excopy}
Let $G$ be a group, and $A$ a normal abelian subgroup.
Show that \(G/A\) operates on $A$ by conjunction,
and in this manner get a homomorphism of \(G/A\) into \(\Aut(A)\).
\end{excopy}

Let \(xA\in G/A\) operate as \(a\mapsto xax^{-1}\) for all \(a\in A\).
To show that this operation as well defined, assume \(xA=yA\).
So \(x^{-1}y, y^{-1}x\in A\) and using the fact that $A$ is abelian, we get
\begin{equation}
xax^{-1} =
xax^{-1}(xy^{-1})^{-1}(xy^{-1}) =
(xy^{-1})^{-1}xax^{-1}(xy^{-1}) =
yay^{-1}.
\end{equation}
Now since \(y(xax^{-1}y^{-1} = (yx)a(yx)^{-1}\) the homomorphism follows.

\end{myenumerate}
\textbf{Semidirect product}
\begin{myenumerate}


%%%%%
\begin{excopy}
Let $G$ be a group and let $H$, $N$ be subgroups with $N$ normal.
Let \(\gamma_x\) be conjunction by an element \(x\in G\).
\begin{itemize}
 \item[(a)] Show that \(x\rightarrow \gamma_x\) induces
    a homomorphism \(f:\,H\mapsto\Aut(N)\).
 \item[(b)] If \(H\cap N = \eG\), show that the map
    \(H \times N \rightarrow HN\) given by
    \((x,y) \mapsto xy\) is a bijection, and that this map
    is an isomorphism if and only if $f$ is trivial,
    i.e. \(f(x) = \id_N\) for all \(x\in H\).
\end{itemize}
We define $G$ to be the \textbf{semidirect product} of $H$ and $N$
if \(G=NH\) and \(H\cap N = \eG\).
\begin{itemize}
 \item[(c)] Conversely, let $H$, $N$ be groups,and let
   \(\psi:\,H\mapsto \Aut(N)\) be a given homomorphism.
  Construct a semidirect product as follows.
  Let $G$ be the set of pairs \((x,h)\) with \(x\in N\) and \(h\in H\).
  Define the composition law
  \begin{equation}
    (x_1,h_1)(x_2,h_2) = (x_1x_2^{\psi(h_1x_2)}, h_1h_2),
  \end{equation}
  Show that this is a group law, and yields a semidirect product of $N$ and $H$,
  identifying
       $N$ with the set of elements \((x,1)\)
   and $H$ with the set of elements \((1,h)\).
\end{itemize}
\end{excopy}

\begin{itemize}

 \item[(a)] Let \(x,y\in H\). For any \(u\in N\)
 \[\gamma_{xy}(u) = xyu(xy)^{-1} = x\gamma_y(u)x^{-1} = \gamma_x(\gamma_y(u)).\]
 \item[(b)]
    Assume \(x_1y_1 = x_2y_2\) where \((x_1,y_1),\,(x_2,y_2)\in H\times N\).
   Multiplying both sides with
    \(x_1^{-1}\) from the left and
    \(y_2^{-1}\) from the right, we get
     \[x_2^{-1}x_1 = y_2y_1^{-1} \in H\cap N = \eG\]
   and thus \(x_1 = x_2\)
   and  \(y_1 = y_2\).
 \item[(c)]
    Seems that the problem isnot well formed.


\end{itemize}

%%%%%
\begin{excopy}
\begin{itemize}
 \item[(a)]
    Let $H$, $N$, be normal subgroups of a finite group $G$.
    Assume that the orders  of $H$ and $N$ are relatively prime.
    Prove that \(xy=yx\) for all \(x\in H\) and \((y\in N\),
    and that \(H\times N=HN\).
 \item[(b)]
    Let \(H_1,\ldots\,H_r\) be normal subgroups of $G$ such that the order
    of \(H_i\) is relatively prime to the order of \(H_j\) for \(i\neq j\).
    Prove that
    \begin{equation}
      H_1\times \ldots \times H_r = H_1\cdots H_r.
    \end{equation}
\end{itemize}
\end{excopy}

\begin{itemize}
 \item[(a)]
    We look at \(N\cap H\) since its order must divide both that
    of $H$ and $N$ we have \(|N\cap H| = 1\) and \(N\cap H = \eG\).
    Now
   \[ xyx^{-1}y^{-1} = (xyx^{-1})y^{-1} = x(yx^{-1}y^{-1}) \in N\cap H.\]
   and thus \(xyx^{-1}y^{-1} = e\) and we get \(xy=yx\).
 \item[(b)]
    Trivial by induction on $r$.
\end{itemize}


%%%%%
\begin{excopy}
Let $G$ be a finite group and let $N$ be a normal subgroup such that
$N$ and \(G/N\) have relatively prime orders.
 \begin{itemize}
   \item[(a)]
      Let $H$ be a subgroup of $G$ having the same order as \(G/N\).
      Prove that \(G = HN\).
   \item[(b)]
      Let $G$ be an automorphism of $G$. Prove that \(g(N) = N\).
 \end{itemize}
\end{excopy}

\begin{itemize}
 \item[(a)]
    The subgroup \(H\cap N\) has on order that must divide
    both $H$ and $N$ and therefore is $1$ and so the subgroup is trivial.
    Now assume \(h_1g_1 = h_2g_2\)
    for \(h_i\in H\), \(g_i\in N\), \(i=1,2\).
    Then \(h_2^{-1}h_1 = g_2g_1^{-1} \in H\cap N\) and
    \(h_1 = h_2\) and \(g_1 = g_2\). Thus counting the elements
    \(|HN| = |H|\cdot|N| = |G|\) and thus \(HN=G\).
 \item[(b)]
    Let \(h: N \rightarrow G/N\) be defined by \(h(x)=g(x)+N\).
    We will show the $h$ must be trivial.
    Now \(H=h(N)\) is a subgroup of \(G/N\) and \(|H|\)
    divides both \(|N|\) and \(G/N\) and therefore \(H=\eG\)
    which means that \(g(N)\subseteq N\). Since $g$ is automorphism
    we have \(g(N) = N\).
\end{itemize}

%%%%%
\begin{excopy}
\label{gop:nofix}
Let $G$ be a finite group operating on a finite set $S$ with \(\#(S)\geq2\).
Assume that there is only one orbit. Prove that there exist an element
\(x\in G\) which has no fixed point,i.e. \(xs\neq s\) for all \(s\in S\).
\end{excopy}

Solution using \cite{Scott87}~10.1.5.

Define \(\Ch(g) = |\{s\in S: gs=s\}|\)  \label{def:Ch}.
\begin{llem}
If $G$ acts on a finite set $S$ has $N$ orbits, then
\begin{equation}
\sum_{g\in G} \Ch(g) = n\cdot |G|.
\end{equation}
\end{llem}

Let \(T\subseteq S\) be an orbit of $G$ and \(a,b\in T\).
It is clear that \(|G_a| = |G_b| = |G|/|T|\).

We may view \(g\in G\) as permutations of $S$.
Let \(F = \{(s,g)\in S\times G: gs=s\}\) be the set of fixed points.
Now
\begin{eqnarray}
\sum_{g\in G} \Ch(g) & = & |F|\\
 & = & \sum_{s\in S} |G_s| \\
 & = & \sum_{T\ \textrm{orbit}} \sum_{s\in T} |G_s| \\
 & = & \sum_{T\ \textrm{orbit}} |T|\cdot|G|\\
 & = & n\cdot|G|.
\end{eqnarray}

Now back to the exercise, if \(n=1\)
\index{transitive}
($G$ is \emph{transitive})
and $G$ is finite then
\begin{eqnarray}
\sum_{g\in G} \Ch(g) = |G|.
\end{eqnarray}
Now assume by negation that for all \(g\in G\) \(\Ch(g)\geq 1\)
(at least one fixed point), then
\begin{eqnarray*}
|G| & = & \sum_{g\in G} \Ch(g) \\
    & = & \Ch(e) + \sum_{g\in G\setminus\eG} \Ch(g) \\
    & = & |S| + \sum_{g\in G\setminus\eG} \Ch(g) \\
    & \geq & |S| + |G| - 1
\end{eqnarray*}
Hence, \(|S| \leq 1\) which contradicts the assumption.

%%%%%
\begin{excopy}
Let $H$ be a proper subgroup of a finite group $G$. Show that $G$
is not the union of all the conjugates of $H$.
\end{excopy}

We look at $G$ as a group operating on the finite sets of the conjugates of $H$.
From the previous Exercise~\ref{gop:nofix}, there must be some \(x\in G\)
for which \(x(gHg^{-1})x^{-1} \neq gHg^{-1}\) for all \(g\in G\).
That is \(x\notin gHg^{-1}\) for all \(g\in G\) and
\[x \notin \bigcup_{g\in G}gHg^{-1}.\]

%%%%%
\begin{excopy}
Let $X$,$Y$ be finite sets and let $C$ be a subset of \(X\times Y\).
For \(x\in X\) let \(\phi(x)=\) number of elements \(y\in Y\) such that
\((x,y)\in C\). Verify that \[\#(C) = \sum_{x\in X}\varphi(x).\]

\emph{Remark}. A subset $V$ as in the above exercise is often called
\index{correspondence}
a \textbf{correspondence}, and \(\varphi(x)\) is the number of elements in $Y$
which correspond to a given element \(x\in X\).
\end{excopy}

This is simple result of looking at the disjoint union:
\[C = \Disjunion_{x\in X} \{(x,y)\in X\times Y: (x,y)\in C\}.\]

%%%%%
\begin{excopy}
Let $S$, $T$ be finite sets. Show that \(\#\Map(S,T) = (\#T)^{\#(S)}\).
\end{excopy}

Simple induction on \(\#(S)\).
If \(S'=S\cup\{x\}\) then we can extend each map \(S\rightarrow T\)
to \(S'\) by assigning \(\#(T)\) different values to $x$.

%%%%%
\begin{excopy}
Let $G$ be a finite group operating on a finite set $S$.
 \begin{itemize}
  \item[(a)]
    For each \(s\in S\) show that \[\sum_{t \in Gs} {\frac{1}{\#(Gt)}} = 1.\]
  \item[(b)]
    For each \(x \in G\) define \(f(x)=\) number of element \(s\in S\)
    such that \(xs=s\). Prove that the number of orbits of $G$ in $S$
    is equal to
      \[\frac{1}{\#(G)}\sum_{x\in G} f(x).\]
 \end{itemize}
\end{excopy}

\begin{itemize}
 \item[(a)]
   For all \(t \in Gs\) we have \(|Gs|=|Gt|\) and so
   \begin{equation}
   \sum_{t \in Gs} {\frac{1}{\#(Gt)}} =
   |Gs|\cdot{\frac{1}{|Gs|}} = 1,
   \end{equation}
 \item[(b)]
  Let us compute the number of ``fixed occurrences''.
  \begin{eqnarray}
   \sum_{x\in G} f(x)
     & = & \sum_{x\in G} \Ch(x)                 \label{eq:f2Ch} \\
     & = & \sum_{s\in S} |\{g\in G: gs = s\}|    \label{eq:Ch2gss} \\
     & = & \sum_{s\in S} |G_s|                   \label{eq:gss2Gs} \\
     & = & \sum_{\textrm{orbit\ } T\subseteq S}
             \sum_{t\in T} |G_t|                 \label{eq:Gsrob} \\
     & = & \sum_{\textrm{orbit\ } T\subseteq S}
             \sum_{t\in T} |G|/|Gt|              \label{eq:GfsGGs} \\
     & = & |G|\cdot\sum_{\textrm{orbit\ } T\subseteq S}
             \sum_{t\in T} 1/|Gt|                \label{eq:1overGs} \\
     & = & |G|\cdot\sum_{\overset{s\in S}{\textrm{orbits repr.}}}
             \sum_{t\in Gs} 1/|Gt|                \label{eq:orbrep} \\
     & = & |G|\cdot\sum_{\overset{s\in S}{\textrm{orbits repr.}}} 1.
                                                  \label{eq:orb1}
  \end{eqnarray}

  Equalities explanation:
  \begin{itemize}
   \item[(\ref{eq:f2Ch})] --- simply using the definition
                              in Exercise~\ref{def:Ch}.
   \item[(\ref{eq:Ch2gss})] --- counting fixed points via $S$ instead of $G$.
   \item[(\ref{eq:gss2Gs})] --- definition of \(G_s\).
   \item[(\ref{eq:Gsrob})] --- separating the summation over orbits.
   \item[(\ref{eq:GfsGGs})] --- basic result of group operating on set.
   \item[(\ref{eq:1overGs})] --- factoring \(|G|\) out
   \item[(\ref{eq:orbrep})] ---  Looking at an orbit $T$ via
                                 a representative \(s\in S\).
   \item[(\ref{eq:orb1})] --- Using the previous item (a) of this exercise.
  \end{itemize}

  Now we simply divide both ends of the equation by \(|G|\)
  to get the desired result.

\end{itemize}

\end{myenumerate}

Throughout, $p$ is a prime number.

\iffalse
% Global remark - so fake an item
\item[]
 \setlength{\leftmargin}{0pt}
 \setlength{\labelwidth}{0pt}
 \setlength{\labelwidth}{0pt}
 Throughout, $p$ is a prime number.
% {\nullfont kaka}
\addtocounter{enumi}{-1}
\fi

\begin{myenumerate}
%%%%%
\begin{excopy}
Let $P$ be a $p$-group. Let $A$ be a normal subgroup of order $p$.
Prove that $A$ is contained in the center of $P$.
\end{excopy}

We can view $P$ as operating on $A$ by conjunction.
That is for any \(x\in P\),we have \(\gamma_x(a) = xax^{-1}\).
Let \(x\in P\) and \(a\in A\) be any elements.
Say \(|P|=p^n\)
and so we have \(\gamma_x^{p^n}=\Id_A\).
% Assume \(\gamma_x(a)\neq a\), so \(a\neq e\) for sure.
Note that
\[\underbrace{\gamma_x(\gamma_x(\ldots(\gamma_x(}_{n\ \textrm{times}}a)\ldots))
 =  \gamma_x^n(a).\]
So because of \(A\setminus\eG\)
let \(k>0\) be the minimal such that \(\gamma_x^k(a)=a\).
So we have \(k|p^n\) and \(k\neq p-1\) and so \(k=1\) and \(\gamma_x(a)=a\).
Thus \(xa=ax\) and $a$ is in the center of $P$.


%%%%%
\begin{excopy}
Let $G$ be a finite group and $H$ a subgroup. Let \(P_H\) be
a $P$-Sylow subgroup of $H$. Prove that there exists a $p$-Sylow subgroup $P$
  of $G$ such that \(P_H = P\cap H\).
\end{excopy}

Since any $p$-subgroup is contained in a $p$-Sylow subgroup,
We have a subgroup $P$ such that \(P_H\subseteq P\subseteq G\).
Obviously \(P_H\subseteq P\cap H\).
Now \(P\cap H\) has an order that divides \(|P|\) so it is a power of $p$.
But from the maximality of \(P_H\) the equality follows.

%%%%%
\begin{excopy}
Let $H$ be a normal subgroup of a finite group $G$
and assume that \(\#(H)=p\). Prove that $H$ is contained in every $p$-Sylow
subgroup of $G$
\end{excopy}

We know that $H$ is contained in some $p$-Sylow subgroup S.
All $p$-Sylow subgroups are conjugates. Now for all \(x\in G\)
\[H=xHx^{-1}\subseteq xSx^{-1}.\]


%%%%%
\begin{excopy}
Let $P$, \(P'\) be $p$-Sylow subgroups of a finite group $G$.
\begin{itemize}
 \item[(a)]  If \(P'\subset N(P)\) (normalizer of $P$), then \(P'=P\).
 \item[(b)]  If \(N(P')=N(P)\), then \(P'=P\).
 \item[(c)]  We have \(N(N(P))=N(P)\).
\end{itemize}
\end{excopy}

\begin{itemize}
 \item[(a)] (Following argument in the proof of Theorem~6.4).
     Since \(P'\subseteq N(P)\)
     we have \(P'P\) is a subgroup of \(N(P)\) and $P$ is normal in it.
     Now
     \begin{equation}\label{eq:pppcp}
     (P'P:P) = (P':P'\cap P)
     \end{equation}
     (see~(iv) page~17).
     Now if by negation \(P'\neq P\) then $p$ divides
     the right side of~(\ref{eq:pppcp}) and so \(P'P\) contains
     a~$p$-Sylow subgroup with higher power of $p$ than that of $P$
     contradicting the fact that $P$ itself is a $p$-Sylow subgroup.
 \item[(b)] Immediate from (a) and the fact that \(P'\subseteq N(P')\).
 \item[(c)] By negation, say \(x\in N(N(P))\setminus N(P)\).
   So \(P' = xPx^{-1}\) is a $p$-Sylow subgroup and \(P'\neq P\).
   Now \[P' = xPx^{-1} \subseteq xN(P)x^{-1} = N(P)\]
   and from (a) we get \(P'=P\) a contradiction.
\end{itemize}
%%%%%%%%%%%%%%
\end{myenumerate}

%%%%%%%%%%%%%%%%%%%%%%%%%%%%%%%%%%%%%%%%%%
\textbf{Explicit determination of groups}

Let us have some lemmas.

\begin{llem} \label{llem:npdiv}
Let $G$ be a group or order $m$,
let \(p^r\) be the highest power of~$p$ that divides~$m$
and let~\(n_p\) be the number of $p$-Sylow subgroups.
Then \(n_p \mid m/p^r\).
\end{llem}
%\textbf{Proof:}
\begin{proof}
Immediate result from Proposition~5.2.
\end{proof}

\begin{llem} \label{rose94:p2q}
\textnormal{\small [See \cite{Rose94} Theorem~5.19]}
Let $G$ be a group and \(|G|=p^2q\) where $p$, $q$ are distinct primes,
then $G$ has a normal  Sylow subgroup and so $G$ is not simple.
\end{llem}
\begin{proof}
Indeed, if $G$ has a normal Sylow subgroup $H$ then
\[\eG\subnormal H \subnormal G\]
is an abelian tower by exercise \ref{ex:p2abel}.
This is true for either \(|H|=p^2\) or \(|H|=q\).
Thus $G$ is simple.

Now let's show the existence of a normal Sylow subgroup.

If \(q<p\) then by Lemma~6.7 the Sylow $p$-subgroup is normal.
So we now can assume \(p<q\).
Let \(n_p\) and \(n_q\) be the numbers of
Sylow $p$-subgroup and $q$-subgroups. By negation we assume
\(n_p > 1\) and \(n_q > 1\).
By local-lemma~\ref{llem:npdiv}
\begin{itemize}
 \item
   \(n_p\mid q\) and so \(n_p=q\).
   % Also \(n_q \equiv 1 \bmod\).
 \item
   \(n_q\mid p^2\), hence \(n_q=p\) or \(n_q=p^2\).
   But  if \(n_q=p\) then by \(n_q\equiv 1 \bmod q\) we have \(p>q\)
   contradicting our assumption and so \(n_q=p^2\).
\end{itemize}
Now any two different $q$-subgroups intersect in \eG
and so the number of elements of order $q$ is \(n_q(q-1)\).
The number of the ``non $q$ order'' elements in $G$ is \(p^2q - n_q(q-1)=p^2\).
Now a Sylow $p$-subgroup has an order \(p^2\) and all its elements
have order different than $q$ and so such subgroup is determined
by its \(p^2\) ``non $q$ order'' elements
and thus it is unique and normal.
\end{proof}

\begin{llem} \label{rose94:pqr}
\textnormal{\small [See \cite{Rose94} Theorem~5.20.]}
Let $G$ be a group and \(|G|=pqr\) where $p$, $q$, $r$ are distinct primes,
then $G$ is not simple.
\end{llem}
\begin{proof}
Let \(n_p\),\(n_q\) and \(n_r\) be the respective
numbers of Sylow subgroups.
Assume by negation that these three numbers are \(\>1\)
since otherwise
we have a normal Sylow subgroup and we are done.
Assume \(p>q>r\)
So any two distinct Sylow subgroups intersect in~\eG.
So the numbers of elements in $G$ of order $p$, $q$ and $r$
are
\(n_p(p-1)\), \(n_q(q-1)\) and \(n_r(r-1)\) respectively.
Therefore
\begin{equation}
|G|=pqr\geq 1 + n_p(p-1) + n_q(q-1) + n_r(r-1).
\end{equation}
By Sylow theorem, \(n_p \mid qr\) and \(n_p\equiv 1\bmod p\).
Since \(n_p>q\) and \(p>q\), \(p>r\), it follows that \(n_p=qr\).

Also \(n_q\mid pr\) and \(n_q\equiv 1\bmod q\)..
Since \(n_q>1\) and \(q>r\), it follows that \(n_q\geq p\).

Finally, \(n_r\mid pq\) so \(n_r\geq q\). Now we have
\begin{equation}
pqr \geq 1 + qr(p-1) + p(q-1) + q(r-1) = pqr + pq + qr - p - q  + 1,
\end{equation}
and hence \((p-1)(q-1)\leq 0\) which is impossible.
\end{proof}

\textbf{Definition:}
\textnormal{\small [See \cite{Rose94}~Exercise~90]}
Let $H$ be a subgroup of $G$.
\index{core!of group}
\index{normal interior}
Define the \emph{core} or \emph{normal interior} of $H$ in $G$ as
\begin{equation}
H_G = \bigcap_{g\in G} g^{-1}Hg
\end{equation}

It is clear that \(H_G\) is the largest normal subgroup of $G$ that
is contained in $H$.

\begin{llem} \label{llem:GsCoreSn}
\textnormal{\small [See \cite{Rose94}~Theorem~4.13]}
if $H$ is a subgroup of $G$ of finite index \(n=[G:H]\) then \(G/H_G\)
can be embedded in \(S_n\).
\end{llem}
\begin{proof}
Let \(\hat{H}\) be the set of $n$ left cosets of $H$ in $G$.
Let $G$ operate on this set by left multiplication.
Each \(g\in G\) permutates \(\hat{H}\).
By enumerating the cosets we identify \(S_n\)
with the permutations of \(\hat{H}\) and we have a mapping
\(\rho: G \rightarrow S_n\).
The kernel of \(\rho\) is the elements \(g\in G\)
for which each coset is fixed.

Let's first compute the \index{stabilizer} \index{Stab}
of a coset \(xH\)
\begin{eqnarray}
\Stab_G(xH)
  & = & \{g\in G: gxH=xH\} \\
  & = & \{g\in G: x^{-1}gxH=H\} \\
  & = & \{g\in G: x^{-1}gx \in H\} \\
  & = & \{g\in G: g \in xHx^{-1}\} \\
  & = & xHx^{-1}
\end{eqnarray}
We compute:
\begin{eqnarray}
\Ker\rho
  & = & \bigcap_{xH\in\hat{H}} \Stab_G(xH) \\
  & = & \bigcap_{g\in G} \Stab_G(gH) \\
  & = & \bigcap_{g\in G} gHg^{-1} \\
  & = & H_G. \\
\end{eqnarray}
And so \(G/H_G\approx \rho(G)\) a subgroup of \(S_n\).
\end{proof}


\begin{llem} \label{llem:pmr:divfac}
\textnormal{\small [See \cite{Rose94}~Exercise~279]}
Let $G$ be a simple group of order \(p^m r\) where \(p\nmid r\).
Then \(p^m \mid (r-1)!\).
\end{llem}
\begin{proof}
Let $H$ be a $p$-Sylow subgroup if $G$. Since $G$ is simple \(H_G = \eG\)
and from local-lemma~\ref{llem:GsCoreSn} $G$ can be embedded in \(S_{[G:H]}\).
Hence \(p^m r \mid r!\) and therefore \(p^m \mid (r-1)!\).
\end{proof}

\begin{myenumerate}

%%%%%
\begin{excopy}
Let $p$ be a prime number. Show that a group of order \(p^2\)
is abelian, and that there are only two such groups up to isomorphism.
\end{excopy}  \label{ex:p2abel}

Let $G$ be the group and $Z$ its center and \(Z\subnormal G\).
If \(G=Z\) then clearly $G$ is abelian.
Assume by negation that \(Z\subsetneq G\).
Since $Z$ is not trivial by Theorem~6.5 it must be of order $p$
and so \(G/Z\) has an order of $p$ as well and is cyclic generated by \(a+Z\).
Now let  \(a_1,a_2 \in G\) be any elements in $G$. For \(i=1,2\) we can
have the representation \(a_i=a^{n_i}g_i\) where \(n_i\neq 0\) and \(g_i\in Z\).
Now
\begin{equation} \label{eq:a1a2}
a_1a_2 = a^{n_1}g_1 a^{n_2}g_2 =
  a^{n_1+n_2}g_1g_2 =
  a^{n_2}a^{n_1}g_2g_1 =
  a^{n_2}g_2a^{n_1}g_1 = a_2a_1.
\end{equation}
Thus $G$ is abelian.

Now by Theorem~8.2, $G$ is isomorphic to a product of cyclic $p$-group.
Hence, isomorphic to\, \(\Zm{p^2}\) \, or \, \(\Zm{p}\times\Zm{p}\).


%%%%%
\begin{excopy}
Let $G$ be a group of order \(p^3\), where $p$ is prime, and $G$ is not abelian.
Let $Z$ be its center. Let $C$ be a cyclic group of order $p$.
\begin{enumerate}[(a)]
\item Show that \(Z \approx C\) and \(G/Z \approx C \times C\).
\item) Every subgroup of $G$ of order \(p^2\) contains $Z$ and is normal.
\item Suppose \(x^p = 1\) for all \(x \in G\),
 Show that $G$ contains a normal subgroup \hbox{\(H \approx C \times C\)}.
\end{enumerate}
\end{excopy}

\begin{itemize}
 \item[(a)]
     The $Z$ subgroup cannot be the whole $G$ since $G$ is not abelian.
     It cannot be trivial because of Theorem~6.5. That leaves the possibilities
     for its order to be $p$ or \(p^2\). If by negation the order is \(p^2\),
     then \(G/Z\) is cyclic and as in exercise~\ref{ex:p2abel}
     similar arguments like in (\ref{eq:a1a2}) gives a contradiction
     by showing that $G$ is abelian. Thus $Z$ is cyclic of order $p$
     and isomorphic to $C$ and \(G/Z\) is of order \(p^2\).

     Now from the previous exercise we know that groups of order \(p^2\)
     must be isomorphic to either \(\Zm{p}\times\Zm{p}\)
     or to the cyclic
     \(\Zm{p^2}\). The latter leads to contradiction that $g$ is abelian
     using the same arguments with \(G/Z\) cyclic.
     Thus \(G/Z\) is isomorphic to \(\Zm{p}\times\Zm{p}\) which is
     isomorphic to \(C\times C\).

 \item[(b)]
     Say $H$ is a subgroup of order \(p^2\). By Lemma~6.7 $H$ is normal.
     The subgroup \(H\cap Z\)
     could be or order $p$ or $1$.
     If by negation it is the latter case, then \(H\cap Z = \eG\).
     To show that \(HZ = \{hc: h\in H \, \textrm{and} \, c\in Z\}\)
     has exactly \(|H|\cdot|Z|=p^3\) we will show that the products
     differ. If
     \(h_1 c_1 = h_2 c_2\) with \(h_i\in H\), \(c_i\in Z\) we get
     \(c_1c_2^{-1} = h_1^{-1}h_2 \in H\cap Z\) and so this product equals $e$
     and \(h_1=h_2\), \(c_1=c_2\). Hence \(HZ=G\) and we can
     represent any \(a_1,a_2\in G\) by \(a_i=h_i c_i\) where
      \(h_i\in H\), \(c_i\in Z\) for \(i=1,2\).
      From exercise~\ref{ex:p2abel} $H$ is abelian and so
      \begin{equation}
      a_1a_2 = h_1 c_1 h_2 c_2 = h_1 h_2  c_1 c_2 =
          h_2 h_1  c_2 c_1 = h_2 c_2 h_1 c_1 = a_2a_1
      \end{equation}
      contradicting the fact that $G$ is abelian.
      Thus  \(H\cap Z\) is of order $p$ and $H$ must contain $Z$.

 \item[(c)]
      Examining the proof of Corollary~6.6 we see that
      in the sequence
      \[\eG=G_0 \subset G_1 \cdots \subset G_n = G\]
      every $p$-group $G$ has, the subgroup \(G_1\) is in the center.
      So in our case \(G_1 = Z\) and  let $H$ be \(G_2\)
      that has order of \(p^2\)
      and again by Lemma~6.7 is normal.
      Now $H$ cannot be cyclic, since if it were then
      its generator $x$ would not satisfy the required
      \(x^p=1\) equation.
      Now by exercise~\ref{ex:p2abel}  $H$
      must be isomorphic to \(C\times C\) and not to the cyclic \(\Zm{p^2}\).
\end{itemize}

%%%%%
\begin{excopy}
\begin{itemize}
 \item[(a)] Let $G$ be a group of order \(pq\), where $p$, $q$ are primes
            and \(p<q\). Assume that \(q\not\equiv 1 \bmod p\).
            Prove that  $G$ is cyclic.
 \item[(b)] Show that every group of order \(15\) is cyclic.
\end{itemize}
\end{excopy}  \label{ex:GpLTq}

\begin{itemize}
 \item[(a)]
   [Similar to the example on page~36 with $G$ of \(35\)].
   Let \(H_p\) and \(H_q\) be a $p$-Sylow and $q$-Sylow subgroups respectively.
   Then \(H_q\) is normal by Lemma~6.7.
   Now \(H_p\) operates by conjunction on \(H_q\) and we have
   a homomorphism \(H_p\rightarrow \Aut(H_q) \approx \Zm{(q-1)}\).
   So the image order must divide $p$ and \(q-1\).
   Since \(q-1\not\equiv 0 \bmod p\) clearly \(p\nmid q-1\)
   and so the image is trivial and so elements of \(H_p\) and \(H_q\)
   commutes with each other.

   We will show that \(H_pH_q = G\).
   The set \(H_{pq}=\{x_p^m x_q^n: 0\leq m<p, 0\leq n<q\}\)
   contains \(pq\) elements
   since if \(x_p^{m_1} x_q^{n_1} = x_p^{m_2} x_q^{n_2}\)
   we use the commutativity and the fact that \(H_p\cap H_q=\eG\)
   to get \(x_p^{m_1-m_2} = x_q^{n_2-n_1} = e\) and so \(H_pq=G\)
   and $G$ is abelian. By Proposition 4.3(\textbf{v}) $G$ is cyclic.

   Let \(x_p\) and \(x_q\) be generators of \(H_p\) and \(H_q\) respectively.
   % Then these generators commutes with each other and th
 \item[(b)] By (a) with \(p=3\), \(q=5\) and we have
      \(5\equiv 2\not\equiv 1 \bmod 3\).

\end{itemize}

%%%%%
\begin{excopy}
Show that every group of order \(<60\) is solvable.
We use the results from Corollary~6.6 and
exercises \ref{ex:GpLTq}, \ref{ex:p2q} and \ref{ex:2pq}.
\end{excopy}

{
% \begin{multicols}{2}

\tablefirsthead{\hline \(|G|\)   &   $=$ & $p$   &   $q$ & $r$ \\ \hline}
\tablehead{\hline \multicolumn{5}{|c|}{\small\textsl{continuation}} \\ \hline}
\tabletail{\hline \multicolumn{5}{|c|}{\small\textsl{to be continued}}\\ \hline}
\tablelasttail{\hline}
\begin{supertabular}{|r|c|r|r|r|}
 1 & \multicolumn{4}{|l|}{Trivial}  \\ \hline
 2 & $p$       & $2$  &      &   \\ \hline
 3 & $p$       & $3$  &      &   \\ \hline
 4 & \(p^2\)   & $3$  &      &   \\ \hline
 5 & $p$       & $5$  &      &   \\ \hline
 6 & $pq$      & $2$  & $3$  &   \\ \hline
 7 & $p$       & $7$  &      &   \\ \hline
 8 & \(p^n\)   & $2$  &      &   \\ \hline
 9 & \(p^n\)   & $3$  &      &   \\ \hline
10 & $pq$      & $2$  & $5$  &   \\ \hline
11 & $p$       & $11$ &      &   \\ \hline
12 & \(p^2q\)  & $2$  & $3$  &   \\ \hline
13 & $p$       & $13$ &      &   \\ \hline
14 & $pq$      & $2$  & $7$  &   \\ \hline
15 & $pq$      & $3$  & $5$  &   \\ \hline
16 & \(p^n\)   & $2$  &      &   \\ \hline
17 & $p$       & $17$ &      &   \\ \hline
18 & $p^2q$    & $3$  & $2$  &   \\ \hline
19 & $p$       & $19$ &      &   \\ \hline
20 & $p^2q$    & $2$  & $5$  &   \\ \hline
21 & $pq$      & $3$  & $7$  &   \\ \hline
22 & $pq$      & $2$  & $11$ &   \\ \hline
23 & $p$       & $23$ &      &   \\ \hline
% 24 &           &      &      &   \\ \hline
\hline
25 & \(p^n\)   & $5$  &      &   \\ \hline
26 & $pq$      & $2$  & $13$ &   \\ \hline
27 & \(p^n\)   & $3$  &      &   \\ \hline
28 & \(p^2q\)  & $2$  & $7$  &   \\ \hline
29 & $p$       & $29$ &      &   \\ \hline
30 & \(pqr\)   & $2$  & $3$  & $5$  \\ \hline
31 & $p$       & $31$ &      &   \\ \hline
32 & \(p^n\)   & $2$  &      &   \\ \hline
33 & $pq$      & $3$  & $11$ &   \\ \hline
34 & $pq$      & $2$  & $17$ &   \\ \hline
35 & $pq$      & $5$  & $7$  &   \\ \hline
\hline
37 & $p$       & $37$ &      &   \\ \hline
38 & $pq$      & $2$  & $19$ &   \\ \hline
39 & $pq$      & $3$  & $13$ &   \\ \hline
41 & $p$       & $41$ &      &   \\ \hline
42 & \(pqr\)   & $2$  & $3$  & $7$  \\ \hline
43 & $p$       & $43$ &      &   \\ \hline
44 & \(p^2q\)  & $2$  & $11$ &   \\ \hline
45 & \(p^2q\)  & $3$  & $5$  &   \\ \hline
46 & $pq$      & $2$  & $23$ &   \\ \hline
47 & $p$       & $47$ &      &   \\ \hline
\hline
49 & \(p^n\)   & $7$  &      &   \\ \hline
50 & \(p^2q\)  & $5$  & $2$  &   \\ \hline
51 & $p$       & $51$ &      &   \\ \hline
52 & \(p^2q\)  & $2$  & $13$ &   \\ \hline
53 & $p$       & $53$ &      &   \\ \hline
\hline
55 & $pq$      & $5$  & $11$ &   \\ \hline
\hline
57 & $pq$      & $3$  & $19$ &   \\ \hline
58 & $pq$      & $2$  & $29$ &   \\ \hline
59 & $p$       & $59$ &      &   \\ \hline
\end{supertabular}
% \end{multicols}
}

Now we need to solve some cases specifically.
Let $G$ be a group. For most orders upto $60$ solvability was shown
in the above table. In each of the following remaining cases
we will show the existence of some proper normal subgroup $H$.
Because of  results for lower orders of $G$,
A sequence
\(\eG\subnormal H \subnormal G\) can be completed to an abelian tower.

\begin{itemize}
 \item Assume \(|G|=24=2^3\cdot3\).\\
    From local-lemma \ref{llem:GsCoreSn} $G$ is not simple since otherwise
    \(2^3\mid(3-1)!\).
 \item Assume \(|G|=36=2^2\cdot3^2\).
    From local-lemma \ref{llem:GsCoreSn} $G$ is not simple since otherwise
    \(3^2\mid(4-1)!\).
 \item Assume \(|G|=40=2^3\cdot5\).
    Exercise~\ref{ex:G40G12} shows that $G$ is not simple.
 \item Assume \(|G|=48=2^4\cdot3\).
    From local-lemma \ref{llem:GsCoreSn} $G$ is not simple since otherwise
    \(2^4\mid (3-1)!\).
 \item Assume \(|G|=54=2\cdot3^3\).
    From Lemma~6.7 \(H_3\) is normal and $G$ is not simple
 \item Assume \(|G|=56=2^3\cdot7\).
    Let \(n_p\) be the number of $p$-Sylow subgroups for \(p=2,7\).
    Now \(n_7\equiv 1 \bmod 7\) and \(n_7\mid 8\).
    If \(n_7=1\) then such \(H_7\) is normal and we are done.
    % if by negation $G$ is simple, then \(n_7=8\). % and \(n_2=7\)

    Otherwise, we can assume \(n_7=8\). % and \(n_2=7\)
    Since such $7$-Sylow subgroups intersect in \eG,
    the number of elements
    in $G$ with order $7$ is \(n_7(7-1)=48\).
    The elements of any $2$-Sylow subgroup are of order \(\neq7\).
    And since \(56-48=8\) there could be only
    one such subgroup \(H_2\{g\in G: g^7\neq e\}\) whose order is $8$
    and must be normal.
\end{itemize}

%%%%%
\begin{excopy}
Let $p$, $q$ be distinct primes. Prove that a group of order \(p^2q\)
is solvable, and that one of its Sylow subgroups is normal.
\end{excopy}  \label{ex:p2q}

By Local Lemma~\ref{rose94:p2q} one of its Sylow subgroups is normal.
Then one of the normal towers
\begin{itemize}
 \item[] \(\eG \subnormal H_p \subnormal G\)
 \item[] \(\eG \subnormal H_q \subnormal G\)
\end{itemize}
exist and can be refined into abelian (and even cyclic) tower.

%%%%%
\begin{excopy}
Let $p$, $q$ be odd primes. Prove that a group of order \(2pq\) is solvable.
\end{excopy}  \label{ex:2pq}

By Local Lemma~\ref{rose94:pqr} such group $G$ has a normal subgroup $H$.

Now \(|H|\in \{p,q,r,pq,pr,qr\}\) and
by Proposition~6.8 the tower \(\eG\subnormal H\subnormal G\)
can be refined to abelian (and cyclic) tower.



%%%%%
\begin{excopy}
\begin{itemize}
 \item[(a)]
   Prove that one of the Sylow subgroups of a group of order $40$ is normal.
 \item[(b)]
   Prove that one of the Sylow subgroups of a group of order $12$ is normal.
\end{itemize}
\end{excopy} \label{ex:G40G12}

\begin{itemize}
\item[(a)] We have \(40=2^3\cdot5\) so the number of 5-Sylow subgroups
  must satisfy \(n_5\mid 8\) and \(n_5\equiv 1 \bmod 5\)
  and so \(n_5=1\) and the unique 5-Sylow subgroup is normal.
\item[(b)]
 From exercise~\ref{ex:p2q}  with \(p^2q=12\) where \(p=2\) and \(q=3\).
\end{itemize}

%%%%%
\begin{excopy}
Determine all groups of order \(\leq 10\) up to isomorphism.
In particular, show that a non-abelian group of order $6$
is isomorphic to \(S_3\).
\end{excopy}

The groups with prime order are cyclic. For other case of a group $G$:
\begin{itemize}
  \item Assume \(|G|=6=2\cdot3\). It could be isomorphic to:
    \begin{itemize}
       \item \(\Zm{6}\) cyclic.
       \item \(\Zm{2}\times\Zm{3}\) abelian.
       \item \(S_3\).
    \end{itemize}
  \item Assume \(|G|=8=2^3\). It could be isomorphic to:
    \begin{itemize}
       \item \(\Zm{8}\) cyclic.
       \item \(\Zm{2}\times\Zm{2}\times\Zm{2}\) abelian.
       \item \(\Zm{2}\times\Zm{4}\) abelian.
    \end{itemize}
  \item Assume \(|G|=9=3^2\). It could be isomorphic to:
    \begin{itemize}
       \item \(\Zm{9}\) cyclic.
       \item \(\Zm{3}\times\Zm{3}\) abelian.
    \end{itemize}
  \item Assume \(|G|=10=2\cdot5\). It could be isomorphic to:
    \begin{itemize}
       \item \(\Zm{10}\) cyclic.
       \item \(\Zm{2}\times\Zm{5}\) abelian.
    \end{itemize}

\end{itemize}

%%%%%
\begin{excopy}
Let \(S_n\) be the permutation group on $n$ elements.
Determine the $p$-Sylow subgroups of
\(S_3\), \(S_4\), \(S_5\) for \(p=2\) and \(p=3\).
\end{excopy}

\begin{itemize}
 \item[\(S_3\)]
    The $2$-Sylow subgroups % generated by transposition and they
    are:
    \(\{e,(12)\}\), \(\{e,(13)\}\) and \(\{e,(23)\}\).
    The $3$-Sylow subgroup is
    \(\{e,(123),(132)\}\).
 \item[\(S_4\)]
    The $2$-Sylow subgroup is \(S_4\) itself and no $3$-Sylow subgroups.
 \item[\(S_5\)] No $2$-Sylow and no $3$-Sylow subgroups.
\end{itemize}

%%%%%
\begin{excopy}
Let \(\sigma\) be a permutation of a finite set $I$ having $n$ elements.
Define \(e(\sigma)\) to be \((-1)^m\) where
\[m = n - \textrm{number of orbits of }\, \sigma.\]
If \(I_1,\ldots,I_r\) are orbits of \(\sigma\), then $m$ is also equal
to the sum
\[ m= \sum_{v=1}^r [\card(I_v)-1].\]
If \(\tau\)  is a transposition, show that \(e(\sigma\tau) = -e(\sigma)\)
be considering the two cases where $i$, $j$ lie in the same orbit of \(\sigma\),
or lie in different orbits. In the first case, \(\sigma\tau\) has one more
orbit and in theses case one less orbit that \(\sigma\).
In particular, the sign of a transposition is \(-1\).
Prove that \(e(\sigma)=\epsilon(\sigma)\) is the sign of the permutation.
\end{excopy}

We show the equality of $m$,
\[ \sum_{v=1}^r [\card(I_v)-1] =
   \sum_{v=1}^r \card(I_v) - \sum_{v=1}^r 1 =
   \sum_{v=1}^r \card(I_v) - \sum_{v=1}^r 1 =
   n - r.\]

We now show \(e(\sigma\tau)= -e(\sigma)\). Let \(\tau=(ij)\).
There are two cases:
\begin{itemize}
 \item
   The elements $i$, $j$  lie in the same orbit $I$ of \(\sigma\).
   Let \(l=|I|\geq 2\) and \(1\leq k<l\) such that \(\sigma^k(i)=j\).
   It is clear that \(\sigma^{l-k}(j)=i\).
   Now \(\sigma\tau\) has all the orbits of \(\sigma\)
   with $I$ split into two orbits:
     \((i, \sigma(j), \ldots \sigma^{l-k-1}(j))\)
   and
     \((j, \sigma(i), \ldots \sigma^{k-1}(j))\).
 \item
   The elements $i$, $j$  lie in different orbits \(I_i\) and \(I_j\)
   respectively of \(\sigma\).
   Now \(\sigma\tau\) has all the orbits of \(\sigma\)
   but with \(I_i\) and \(I_j\) united.
\end{itemize}
In both cases the number of orbits of \(\sigma\tau\) differs by $1$
from that of \(\sigma\).
Hence
\(e(\sigma\tau) = (-1)^{m+1} = -(-1)^m = -e(\sigma)\).

Since the transpositions  generates \(S_n\) and both $e$ and \(\epsilon\)
agree on the transpositions and the identity (\(e(\id)=\epsilon(\id)=1\))
they agree on all permutations.

%%%%%
\begin{excopy}
\begin{itemize}
 \item[(a)]
   Let $n$ be an even positive integer. Show that there exists  a group
   of order \(2n\), generated by two elements \(\sigma\), \(\tau\)
   such that \(\sigma^n=e=\tau^2\), and \(\sigma\tau=\tau\sigma^{n-1}\).
   (Draw a picture of a regular $n$-gon, number the vertices,
   and use the picture as an inspiration to get \(\sigma\), \(\tau\).)
   Thus group is called the
   \index{dihedral} \index{group!dihedral}
   \textbf{dihedral group}.
 \item[(b)]
   Let $n$ be an odd positive integer. Let \(D_{4n}\) be the group generated
   by the matrices
   \begin{equation}
     \left(
      \begin{array}{lr}
       0 & -1 \\
       1 & 0 \\
      \end{array}
     \right)
     \quad\textrm{and}\quad
     \left(
      \begin{array}{lc}
       \zeta & 0 \\
       0 & \zeta^{-1} \\
      \end{array}
     \right)
   \end{equation}
   where \(\zeta\) is a primitive $n$-th root of unity. Show that \(D_{4n}\)
   has order \(4n\), and give the commutation relations between the above
   generators.
\end{itemize}
\end{excopy}

\begin{itemize}
 \item[(a)]
 Let \(\theta=2\pi/n\). Now let\(\sigma\) be a \(1/n\) rotation
 and \(\tau\) a reflection. More formally:
   \begin{equation}
     \sigma = \left(
      \begin{array}{rl}
       \cos\theta & \sin\theta \\
       -\sin\theta & \cos\theta \\
      \end{array}
     \right)
     \quad\textrm{and}\quad
     \tau = \left(
      \begin{array}{lr}
       1 & 0 \\
       0 & -1 \\
      \end{array}
     \right)
   \end{equation}

 \item[(b)]
  Denote
   \begin{equation}
     \sigma = \left(
      \begin{array}{lr}
       0 & -1 \\
       1 & 0 \\
      \end{array}
     \right)
     \quad\textrm{and}\quad
     \tau = \left(
      \begin{array}{lc}
       \zeta & 0 \\
       0 & \zeta^{-1} \\
      \end{array}
     \right)
   \end{equation}

  From that we compute \(\sigma^2 = -\Id\) and \(\sigma^4 = \tau^n = \Id\).
  Thus \(\sigma^2\tau=\tau\sigma^2\) and \(\tau^n\sigma=\sigma\tau^n\).
\end{itemize}

\iffalse
\begin{itemize}
  \item[(a)] Rotation and mirroring. Consider the subgroup of \(S_n\)
     where
   \begin{equation*}
     \sigma(i) =
       \left\{
         \begin{array}{ll}
         i + i \;& \textnormal{if}\; i < n \\
         0     \;& \textnormal{if}\; i = n
         \end{array}
       \right.
   \end{equation*}
   and \(\tau(i) = (n - i + 1)\).

  \item[(b)]
  Say $J$ is the first matrix. Then
  \begin{equation*}
    J^2 =
      \left(
        \begin{array}{rr}
        -1 & 0 \\
        0  & -1
        \end{array}
      \right)
     \quad\textrm{and}\quad
    J^3 =
      \left(
        \begin{array}{rr}
        0 & 1 \\
        -1 & 0
        \end{array}
      \right)
     \quad\textrm{and}\quad
    J^4 =
      \left(
        \begin{array}{rr}
        1 & 0 \\
        0 & 1
        \end{array}
      \right).
  \end{equation*}
\end{itemize}
\fi


%%%%%
\begin{excopy}
Show that there are exactly two non-isomorphic non-abelian groups of order~$8$.
(one of them is given by the generators \(\sigma\), \(\tau\) with the relations
\begin{equation*}
\sigma^4 = 1, \qquad \tau^2 = 1, \qquad \tau\sigma\tau = \sigma^3.
\end{equation*}
The other is the quaternion group.)
\end{excopy}

Let $G$ be non-abelian group of order~$8$.
Let $m$ be the maximal period of the elements of $G$.
Since \(m|8\) we must have \(m\in\{1,2,4,8\}\).

We will show that \(m=4\).
If \(m=1\) then \(|G|=1\), contradiction.
If \(m=2\) then for any \(a,b\in G\)
\begin{equation*}
1 = (ab)(ab) = a(bb)a = (ab)(ba)
\end{equation*}
and so
\begin{equation*}
(ab)^{-1} = ab = ba
\end{equation*}
and $G$ is abelian, contradiction.
If \(m=8\) then $G$ is cyclic, contradiction.

Let \(\sigma\in G\) be of period $4$. It generates
the subgroup \(H=\{\sigma^1,\sigma^2,\sigma^3,1\}\).
\newcommand{\coH}{\ensuremath{\tilde{H}}}
Put  \(\coH = \coH\).
Since \([G:H]=2\) there are two cosets of $H$ and \coH\ in $G$.
Now for any \(g_1,g_2\in \coH\) we have
\begin{equation} \label{eq:H=g1g2H}
H = g_1 g_2 H = g_1 H g_2 = H g_1 g_2.
\end{equation}
and in particular \(H \triangleleft G\).

Since conjunction is automorphism, \(g\sigma g^{-1} \in H\) is
of period $4$ for each \(g\in G\).
Thus
\begin{equation*}
\forall g\in G,\; g\sigma g^{-1} \in \{\sigma^1,\sigma^3\}.
\end{equation*}

Assume by negation \(\exists y\in \coH,\; g\sigma g^{-1}=\sigma\).
Then
\begin{equation*}
\exists y\in \coH\, \forall k\in\{0,1,2,3\},\; g\sigma^kg^{-1}=\sigma^k.
\end{equation*}
Thus
\begin{equation*}
\exists y\in \coH\, \forall h\in H,\; yh=hy.
\end{equation*}
But any \(z \in \coH\) is of the form \(z = yh'\) for some \(h'\in H\)
and so for any \(h\in H\) we have
\begin{equation*}
zh = (yh')h = y(h'h) = (h'h)y = (hh')y = h(h'y) = hz.
\end{equation*}
and now
\begin{equation*}
\forall y\in \coH\, \forall h\in H,\; yh=hy.
\end{equation*}

Let \(y_1,y_2\in \coH\). By looking at \(\coH\) as a coset,
\(y_2 = h y_1 = y_1 h\) for some \(h\in H\).
Now since \((y_1)^2\in H\) as we saw in \eqref{eq:H=g1g2H}
\begin{equation*}
y_1 y_2 = y_1 (h y_1) = y_1 (y_1 h) = (y_1)^2 h = h (y_1)^2 = (h y_1)y_1
= (y_1 h) y_1 = y_2 y_1.
\end{equation*}
Thus $G$ is abelian, and by contradiction
\begin{equation*}
\forall y\in \coH,\; y\sigma y^{-1}=\sigma^3.
\end{equation*}
Similarly,
\begin{eqnarray*}
\forall y\in \coH,\; & y^{-1}\sigma y &= \sigma^3 \\
\forall y\in \coH,\; & y\sigma^3 y^{-1} &= \sigma \\
\forall y\in \coH,\; & y^{-1}\sigma^3 y &= \sigma.
\end{eqnarray*}

Clearly the periods of \(y\in \coH\) could be $2$ or $4$.
If all these periods are $2$ then

Assume the index of $y$ is $2$ for some \(y\in\coH\).
Then \(y^2=1\) and \(u=y^{-1}\) and so
\begin{equation*}
(\sigma y)^2 = \sigma(y \sigma y^{-1}) = \sigma^{1+3}=1.
\end{equation*}
Since conjunction is automorphism, \(y\sigma^2 y^{-1} = \sigma^2\)
begin the only element of order $2$ in $H$.
Noting that \(y\cdot 1 \cdot y{-1} = 1\)
we have
\begin{equation*}
\left\{y\sigma^n y^{-1}: n\in\{0,1,2\}\right\}
=
\left\{y\sigma^n y^{-1}: n\in\{0,3,2\}\right\}.
\end{equation*}
So by looking at the reminder
\(y\sigma^3 y^{-1} = \sigma\).
\begin{equation*}
(\sigma^3 y)^2 = \sigma^3 (y \sigma^3) y = \sigma^{3+1} = 1.
\end{equation*}
Thus all elements of \coH\ are of \emph{equal} period, $2$ or $4$.

Thus we have two possibilities.
\begin{enumerate}

\item If the order of \(y\in\coH\) is $2$ then $G$ is the diehedral group
with the relations specified in the exercise.

\item If the order of \(y\in\coH\) is $8$ then $G$ is the quaternion group.
We put \(i=\sigma\), pick arbitrary \(j\in\coH\), and put \(k=ij\).
Now since the orders of $j$ and $k$ are also $4$,
we have \(|\{i^3,j^3,k^3\}|=3\) (different elements) and
\begin{equation*}
rsr^{-1}=s^3 \qquad \textnormal{where} \qquad
(r,s) \in \left\{(i,j), (i,k), (j,i), (j, k), (k,i), k,j)\right\}.
\end{equation*}
and \(i^2 = j^2 = k^2\) which we conveniently denote as \((-1)\).
With this we have the 3 elements
\begin{eqnarray*}
(-i) &=& i^{3} = (-1)i = i(-1) \\
(-j) &=& j^{3} = (-1)j = j(-1) \\
(-k) &=& k^{3} = (-1)k = k(-1).
\end{eqnarray*}
Also
\begin{eqnarray*}
ji &=& ji(j^{-1}j) = (jij^{-1})j = i^{3}j = i^2k = (-1)k \\
jk &=& j(ij)(i^{-1}i) = (ii^{-1})j(ij) = i(i^{-1}ji)j = ij^{3+1} = i \\
kj &=& (kj)(k^{-1}k) = (kjk^{-1})k = j^{2+1}k = (-1)jk = (-1)i   \\
ik &=& ik(jj^3) = i(kj)j^3 = i^{1+2+1}j^{2+1} = (-1)j \\
ki &=& ki(k^{-1}k) = (kik^{-1})k = i^{2+1}k = (-1)(ik) = (-1)j
\end{eqnarray*}
\end{enumerate}


%%%%%
\begin{excopy}
Let \(\sigma = [123 \ldots n]\) in \(S_n\).
Show that the conjugacy class of \(\sigma\) has \((n - 1)!\) elements.
Show that the centralizer of \(\sigma\) is the cyclic group generated by
\(\sigma\).
\end{excopy}

The following relation on \(S_n\)
\begin{equation*}
\tau_1 \sim \tau_2
\qquad \textnormal{iff} \qquad
\exists k\in \N_n,\, \tau_1 \sigma^k = \tau_2.
\end{equation*}
is an equivalence relation, since
\(k=0\) gives reflexivity, symmetry by considering
\(\tau_2 \sigma^{n-k} = \tau_1\) and associativity
since if \(\tau_1 \sim \tau_2\)
and if \(\tau_2 \sim \tau_3\)
with \(k_1\) and \(k_2\) respectively, then
\begin{equation*}
\tau_1 \sigma^{k_3} = \tau_3
\qquad \textnormal{where}\; k_3 = k_1+k_2.
\end{equation*}

For any \(\tau\in S_n\) and \(k \in \N_n\)
\begin{equation*}
(\tau\sigma^k)\sigma\left(\tau\sigma^k\right)^{-1}
= \tau\sigma^{k+1-k}\tau^{-1}
= \tau\sigma\tau^{-1}.
\end{equation*}

Assume \(\tau_1 \nsim \tau_2\).
We choose \(\tau'_1 \sim \tau_1\)
and \(\tau'_2 \sim \tau_2\)
such that
\(\tau'_1(1) = \tau'_2(1)\).

Clearly we can find some \(j\in\N_n\) such that
\(\tau'_1(j) = \tau'_2(j) = j\)
and
\({\tau'_1}^{-1}(j + 1) \neq {\tau'_2}^{-1}(j + 1)\).
Now
\begin{equation*}
\left(\tau'_1 \sigma {\tau'_1}^{-1}\right)(j')
\neq
\left(\tau'_2 \sigma {\tau'_2}^{-1}\right)(j')
\end{equation*}
Thus
\begin{equation*}
\tau'_1 \sigma {\tau'_1}^{-1}
=
\tau'_2 \sigma {\tau'_2}^{-1}
\qquad \textnormal{iff} \qquad
\tau'_1 \sim \tau'_2.
\end{equation*}
Thus the size of the conjugacy class of \(\sigma\)
is the same as the number of classes of the equivalence relation \(\sim\)
which is \(n!/n = (n-1)!\).

%%%%%
\begin{excopy}
\begin{enumerate}[(a)]
\item
Let \(\sigma = [i_1\cdots i_m]\) be a cycle.
Let \(\gamma \in S_n\). Show that \(\gamma\sigma\gamma^{-1}\)
is  the cycle \([\gamma(i_1)\cdots \gamma(i_m)]\).
\item
Suppose that a permutation \(\sigma\) in \(S_n\) can be written
as a product of $r$ disjoint
cycles, and let \(d_1,\ldots,d_r\) be the number of elements in each cycle,
in increasing order.
 Let \(\tau\) be another permutation which can be written as a product of
disjoint cycles, whose cardinalities are
\(d'_1,\ldots,d'_r\)
in increasing order. Prove
that \(\tau\) is conjugate to \(\tau\) in \(S_n\) if and only if
\(r = s\) and \(d_i = d'_i\) for all \(i=1,\ldots,r\).
\end{enumerate}
\end{excopy}

\begin{enumerate}[(a)]
\item
\begin{equation*}
\left(\gamma\sigma\gamma^{-1}\right)(\gamma(i_j))
= \left(\gamma\sigma\right)\left(\gamma^{-1}\gamma\right)(i_j)
= \left(\gamma\sigma\right)(i_j)
= \gamma(\sigma(i_j))
= \gamma(i_{\bres{(j+1}{m}}).
\end{equation*}
\item
If \(\tau\) is conjugate to \(\sigma\) then the equalities
of the cycles sizes follow from (\emph{a}).
Conversely, if the cycles sizes agree then
if \([i_1,\ldots,i_m]\) is a cycle of \(\sigma\)
and  \([j_1,\ldots,j_m]\) is a cycle of \(\tau\)
we define \(\mu(i_k)=j_k\) for \(k\in \N_n\).
Since \(\N_n\) is a disjoint union of any permutation in \(S_n\)
we have \(\mu\in S_n\) well defined
and \(\tau = \mu\sigma\mu^{-1}\).
\end{enumerate}

%%%%%
\begin{excopy}
\begin{enumerate}[(a)]
\item
Show that \(S_n\) is generated by the transpositions
\([12]\), \([13]\),\(\ldots\),\([1n]\).
\item
Show that \(S_n\) is generated by the transpositions
\([12]\), \([23]\), \([34]\),\(\ldots\),\([n-1,n]\).
\item
Show that \(S_n\) is generated by the cycles \([12]\) and \([1 2 3 \ldots n]\).
\item
Assume that $n$ is prime.
Let \(\sigma = [1 2 3 \ldots n]\) and let \(\tau = [rs]\) be any transposition.
Show that \(\sigma\), \(\tau\) generate \(S_n\).
\end{enumerate}
\end{excopy}

Note that given a generator \(g\in S_n\),
we always have \(\Id\in S_n\) gernerated by \(\Id = g^k\) for some \(k\in\N\).
Given \(\sigma,\tau\in S_n\). We define the distance
\begin{equation*}
d(\sigma,\tau) = \left\|\{i\in\N_n: \sigma(i) \neq \tau(i)\}\right|.
\end{equation*}
\begin{enumerate}[(a)]
\item Let \(\sigma\in S_n\). Let \(G\subset S_n\) be the group generated
by the given generators. Let \(g\in G\) be with the minimal
distance with \(sigma\).
If by negation \(d(\sigma,g) > 0\) then let \(j\in\N_n\)
be the minimal index such that \(\sigma(j)\neq g(j)\).
Define
\begin{equation*}
g' = [1 j][1 g^{-1}\sigma(j)][1 j]g
\end{equation*}
and now \(d(\sigma, g') < d(\sigma, g)\) contrdicting the minimal choice.
\item
For each \(k \in \N_n\) we have
\begin{equation*}
[1 k] = [1 2]\cdot[2 3]\cdots[(k-2),(k-1)]\cdot[(k-1),k]\cdots[2 3]\cdot[1 2]
\end{equation*}
Thus we get the generators of previous case (a).
\item
For each \(k \in \N_n\) we have
\begin{equation*}
[k,(k+1)] = [1 2 3\ldots n]^{(k-1)}\cdot[12]\cdot [1 2 3\ldots n]^{n - (k-1)}.
\end{equation*}
Thus we get the generators of previous case (b).
\item
Assume \(d=r-s>0\). for any \(k\in \N_n\)
\begin{equation*}
[k,\bres{(k+d)}{n}] = \sigma^{k-r}[r s]\cdot\sigma^{r-k}
\end{equation*}
and
\begin{equation*}
\tau_k = [\bres{1 + kd}{n}, \bres{1 + (k+1)d}{n}].
\end{equation*}
are all generated.
Since $n$ is prime, for any \(m < n\) there exists \(q\in\N\)
such that \(d^q = (m - 1) \bmod n\). Thus
\begin{equation*}
[1 m] = \tau_0\cdot\tau_1\cdots\tau_q\cdots\tau_1\cdots\tau_0.
\end{equation*}
Hence the generators of (b) are generated.
\end{enumerate}

\end{myenumerate}

Let $G$ be a finite group operating on a set $S$.
Then $G$ operates in a natural way on
the Cartesian product \(S^{(n)}\) for each positive integer $n$ .
We define the operation on $S$
to be \hbox{\boldmath$n$\textbf{-transitive}} if given $n$ distinct elements
\((s_1,\ldots,s_n)\) and $n$ distinct elements
\((s'_1,\ldots,s'_n)\) of $S$, there exists \(\sigma\in G\)
such that \(\sigma s_i = s'_i\) for all \(i = 1,\ldots,n\).

\begin{myenumerate}
%%%%%
\begin{excopy}
Show that the action of the alternating group \(A_n\)
on \(\{1,\ldots,n\}\) is \((n - 2)\)-transitive.
\end{excopy}

Given arbitrary
\(n-2\) elements \((s_1,\ldots,s_{n-2})\)  in \(\N_n\)
and
\(n-2\) elements \((s'_1,\ldots,s'_{n-2})\)  in \(\N_n\).
We have a pair of two remaining elements
\begin{eqnarray*}
\{s_{n-1}, s_n\} &= \N_n \setminus \{(s_1,\ldots,s_{n-2}\}\\
\{s'_{n-1}, s'_n\} &= \N_n \setminus \{(s'_1,\ldots,s'_{n-2}\}.
\end{eqnarray*}
For \(i\in\N_n\) define  \(\sigma_1(s_i) = s'_i\) and
\begin{equation*}
  \sigma_2(s_i) =
    \left\{
      \begin{array}{ll}
        s'_i \quad &\textnormal{iff}\; i \leq n - 2 \\
        s'_n \quad &\textnormal{iff}\; i = n - 1 \\
        s'_{n-1} \quad &\textnormal{iff}\; i = n \\
      \end{array}
    \right.
\end{equation*}
Clearly \(\sigma_1,\sigma_2\in\S_n\) having different signs, thus
(exactly) one of them \(\in A_n\).

%%%%% ex 40
\begin{excopy}
Let \(A_n\) be the alternating group of even permutations of
\(\{1,\ldots,n\}\), For \(j = 1,\ldots,n\)
let \(H_j\) be the subgroup of \(A_n\) fixing $j$,
so \(H_j \approx A_{n-1}\), and \((A_n: H_j) = n\) for \(n > 3\).
Let \(n \geq 3\) and let $H$ be a subgroup of index $n$ in \(A_n\).
\begin{enumerate}[(a)]
\item
Show that the action of \(A_n\) on cosets of $H$ by left translation
gives an isomorphism \(A_n\) with the alternating group of permutations
of \(A_n/H\).
\item
Show that there exists an automorphism of \(A_n\) mapping \(H_1\) on $H$,
and that
such an automorphism is induced by an inner automorphism of \(S_n\) if and only
if \(H = H_i\) for some~$i$.
\end{enumerate}
\end{excopy}

Note that
\begin{equation*}
(A_n: H_j) = |A_n|/|A_{n-1}| = (n!/2)/\left((n-1)!/2)\right) = n!/(n-1)! = n.
\end{equation*}
Let \(A_n\) act on cosets \(\tau H_j\) by multiplication from left.
We need to show that this action is well defined.
Let \(\tau_1 H_j = \tau_2 H_j\),
Thus we have \(h_1,h_2\in H_j\) such that
\(\tau_1 h_1 = \tau_2 h_2\)
and so
\(\sigma\tau_1 h_1 = \sigma\tau_2 h_2\) for all \(\sigma \in S_n\)
and so
\(\sigma\tau_1 H_j = \sigma\tau_2 H_j\) for all \(\sigma \in A_n\)
showing that the action is well defined.
\begin{enumerate}[(a)]
\item
With similar argunent we showed that the action is well definded,
we can also show that the action is an homomorphism of \(A_n\)
on the permutations of \(A_n/H\).
We still need to show it is an injection and surjection..

We check manually for n=3,4.
\begin{equation*}
A_3 = \{e, [1,2,3], [1, 3, 2]\}
\end{equation*}
and thus \(|H|=1\) so \(H=\{e\}\) and the isomorphism is clear.

\begin{align*}
A_4 = \{&e, \\
  &[1,2,3], [1, 3, 2], [1,2,4], [1, 4, 2], [1, 3, 4], [1, 4, 3],
    [2, 3, 4], [2, 4, 3], \\
  &[1, 2][3, 4], [1, 3][2, 4], [1, 4][2, 3]\}.
\end{align*}
\(|A_4|=4!/2=12\) and \(|H|=|A_4|/4=3\).
The possible subgroups are of the
form \(\{e, h, h^2\}\) where
\(h \in \{[i,j,k]\in A_4: i\neq j \neq k\}\).
Note we always have \(h^3=e\).
Now there are 4 disjoint cosets \(eH, \tau_1 H, \tau_2 H, \tau_3 H\)
where \(\tau_i \in A_n\) for \(i=1,2,3\), and put \(\tau_0 = e\).
To show innjection, let \(\sigma_1, \sigma_2 \in A_4\)
and assume \(\sigma_2\tau_i H = \sigma_2\tau_i H\) for \(i=0,1,2,3\).
Looking at \(i=0\), we have
\(\sigma_1(e) \in \{\sigma_2 e, \sigma_2 h, \sigma_2 h^2\}\).\\
Cases:
\begin{itemize}
\item \(\sigma_1 e = \sigma_2 e\). Clearly \(\sigma_1 = \sigma_2\).
\item \(\sigma_1 e = \sigma_2 h\). Then \(\sigma_2^{-1}\sigma_1 = h\).
 \Wlogy we may assume \(h=[1,2,3]\).
Manually calculating permutations cycles,
 \(H = \{[e, [1, 2, 3], [1, 3, 2]\}\). The 4 cosets are:
\begin{align*}
H_1 &= \{e, [1, 2, 3], [1, 3, 2]\} = H \\
H_2 &= \{[2, 3, 4], [1, 3][2, 4], [1, 4, 2]\} \\
H_3 &= \{[2, 4, 3], [1, 2][3, 4], [1, 4, 3]\} \\
H_4 &= \{[1, 2, 4], [1, 3, 4], [1, 4][2, 3]\}.
\end{align*}
Now multiplying from left by all \(A_4\) permutations:
\begin{center}
\begin{tabular}{lll}
 i & \(\alpha \in A_4\) & $H$ indices of \(\alpha H_{1,2,3,4}\) \\
 1 & [1, 2, 3, 4] & [1, 2, 3, 4] \\
 2 & [1, 3, 4, 2] & [2, 3, 1, 4] \\
 3 & [1, 4, 2, 3] & [3, 1, 2, 4] \\
 4 & [2, 1, 4, 3] & [3, 4, 1, 2] \\
 5 & [2, 3, 1, 4] & [1, 3, 4, 2] \\
 6 & [2, 4, 3, 1] & [4, 1, 3, 2] \\
 7 & [3, 1, 2, 4] & [1, 4, 2, 3] \\
 8 & [3, 2, 4, 1] & [4, 2, 1, 3] \\
 9 & [3, 4, 1, 2] & [2, 1, 4, 3] \\
10 & [4, 1, 3, 2] & [2, 4, 3, 1] \\
11 & [4, 2, 1, 3] & [3, 2, 4, 1] \\
12 & [4, 3, 2, 1] & [4, 3, 2, 1]
\end{tabular}
\end{center}
The table shows the left- multiplication is injective, 
thus \(\sigma_1=\sigma_2\).

\item \(\sigma_1 e = \sigma_2 h^2\).
  Putting \(h' = h^2\) and then \(h'^2 = h^4 = h\).
  So we can apply the previous case.
\end{itemize}
That the end of handling \(A_3\).

Now we may assume \(n \geq 5\).
Assume \(\sigma \in A_n\) is in the kernel
of the left multiplication mapping.Then
\begin{equation} \label{eq:sigma-in-kern:An}
\forall \tau\in A_n\quad \sigma\tau H = \tau H.
\end{equation}
Now
\begin{equation*}
\sigma\tau H = \tau H
\;\Leftrightarrow\;
\tau^{-1}\sigma\tau H = H
\;\Leftrightarrow\;
\tau^{-1}\sigma\tau \in H
\;\Leftrightarrow\;
\sigma \in \tau H \tau^{-1}
\end{equation*}
Thus \eqref{eq:sigma-in-kern:An} gives
\begin{equation*}
\sigma \in \bigcap_{\tau\in A_n} \tau H \tau^{-1}
\end{equation*}
The latter intersection is clearly a normal subgroup of \(A_n\)
Since it is a subgroup of $H$ it is a proper subgroup of \(A_n\)
so it must be the trivial \(\{e\}\) since \(A_n\) is simple.
Thus the left multiplication is injective.


\begin{llem} \label{llem:half:normal}
Let $H$ be a subgroup of $G$. If \([G:H]=2\) then \(H \subnormal G\).
\end{llem}
\begin{proof}
Let \(g \in G\setminus H\) then clearly  \(gH\cap H = \emptyset\)
and since \(|gH| + |H| = |G|\) we have
\begin{equation*}
G = H \dotcup gH = H \dotcup g^{-1}H = H \dotcup Hg = H \dotcup Hg^{-1}.
\end{equation*}
Thus \(gH = Hg\) and so \(gHg^{-1}= (Hg)g^{-1} = H\).
\end{proof}

\begin{llem} \label{llem:unique:Sn:half}
For \(n\geq 2\) there exists a unique subgroup of \(S_n\) of order \(n!/2\)
namely \(A_n\).
\end{llem}
\begin{proof}
Let \(\sigma\;S_n\to \{+1,-1\}\) be the sign group homomorphism.
Its kernel is clearly \(A_n\).
Now let \(H\subset S_n\) with \(|H|=n!/2\).
By local-lemma~\ref{llem:half:normal} \(H\subnormal G\).
If \(n<5\) then we manually verify that  \(H==A_n\). 
We may now assume \(n\geq 5\).
Clearly \(H\cap A_n \subnormal A_n\). By Theorem~5.5
\(H\cap A_n = A_n\) and we are done, 
or by negation \(H\cap A_n = \{e\}\). Now since \(|H\cup A_n| = n! -1\)
clearly $H$ Contains \(n!/2 - 1\) odd permutations.
Pick 3 odd permutations, at least 2 of them \(\pi_1, \pi_2\)
satisfy \(\kappa=\pi_1\pi_2 \neq e\) and \(\kappa\) is even
thus \(\kappa \in A_n\) which gives a contradiction.
\iffalse
% Since \(G/H \simeq C_2 = \{+1,=1\}\)
% Consider the natural homomorphism \(\Lambda G \to G/H \{+1,=\{+1,=1\}\).
As was shown in the text, \(A_n\) is generated by all 3-cycles
using \([ij][rs] = [ijr][irs]\).
If by negation \(H \neq A_n\) then both $H$ and \(S_n\setminus H\)
contains some 3-cycles. Say
\begin{equation*}
[ijk] \in H \qquad [xyz]\in S_n\setminus H
\end{equation*}
Say \(I = \{i,j,k\}\cap \{x,y,z\}\).
Without loss of generality we may assume
that if \(|I|\geq 1\) then \(i=x\)
and  if \(|I|=2\) then \(j=y\)
Now consider
\begin{equation*}
\sigma =
 \left\{
  \begin{array}{ll}
   {[ix]}[jy][kz] \quad &\textnormal{if}\; |I|=0 \\
   {[jy]}[kz] \quad &\textnormal{if}\; |I|=1 \\
   {[kz]} \quad &\textnormal{if}\; |I|=2 \\
  \end{array}
 \right.
\end{equation*}
Now \([xyz] = \sigma [ijk] \sigma^{-1}\).
And by \(H \subnormal S_n\) we have the contrdiction \([xyz] \in H\).
\fi
\end{proof}

Back to the exercise.
The left multiplication $L$ maps \(A_n\) to permutations
of the $n$ cosets of $H$.
Thus we have a one-to-one mapping \(L: A_n \rightarrow S_n\).
Thus \(|L(A_n)|=|A_n|=|S_n|/2 = n!/2\).
By local-lemma~\ref{llem:unique:Sn:half} \(L(A_n)=A_n\).

\item
From the previous part, we have a 1-1 onto mapping
 \(L: A_n \to \Alt(A_n/H)\simeq A_n\).
Now for each \(\alpha \in H\)
\begin{equation*}
 L(\alpha) = eH, \tau_2 H, \ldots \tau_n H
\end{equation*}
Thus \(L(H) = H_1\).

Let \(\\sigma \in H_i\) and \(\tau\in A_n\) such that \(\tau(i)=j\)
then \(\tau\) acting by conjunction on \((H_1,H_2,\ldots,h_n\)
moves \(\tau H_i \tau^{-1} = H_j\).
Conversely, if \(H = \tau H_i \tau^{-1}\) then clearly \(H = H_j\).
\end{enumerate}

%%%%%
\begin{excopy}
Let $H$ be a simple group of order \(60\).
\begin{enumerate}[(a)]
\item
Show that the action of $H$ by conjugation on the set of its Sylow subgroups
gives an imbedding \(H \hookrightarrow A_6\).
\item
Using the preceding exercise, show that \(H \approx A_5\).
\item
Show that \(A_6\) has an automorphism which is not induced by an inner
automorphism of \(S_6\).
\end{enumerate}
\end{excopy}

%%%%%
\begin{excopy}
\end{excopy}

%%%%%
\begin{excopy}
\end{excopy}

\end{myenumerate}

%%%%%%%%%%%%%%%%%%%%%%%%%%%%%%%%%%%%%%%%%%%%%%%%%%%%%%%%%%%%%%%%%%%%%%%%
%%%%%%%%%%%%%%%%%%%%%%%%%%%%%%%%%%%%%%%%%%%%%%%%%%%%%%%%%%%%%%%%%%%%%%%%
%%%%%%%%%%%%%%%%%%%%%%%%%%%%%%%%%%%%%%%%%%%%%%%%%%%%%%%%%%%%%%%%%%%%%%%%
\bibliographystyle{plain}
\bibliography{langalg}

%%%%%%%%%%%%%%%%%%%%%%%%%%%%%%%%%%%%%%%%%%%%%%%%%%%%%%%%%%%%%%%%%%%%%%%%
%%%%%%%%%%%%%%%%%%%%%%%%%%%%%%%%%%%%%%%%%%%%%%%%%%%%%%%%%%%%%%%%%%%%%%%%
%%%%%%%%%%%%%%%%%%%%%%%%%%%%%%%%%%%%%%%%%%%%%%%%%%%%%%%%%%%%%%%%%%%%%%%%
% % $Id: langalg.tex,v 1.4 2001/05/04 12:24:45 yotam Exp yotam $
\documentclass[12pt]{book}
\usepackage{fullpage}
\usepackage{amsmath}
\usepackage{amssymb}
\usepackage{amsthm}
% \usepackage{amsthm}
\usepackage{makeidx}
\makeindex % enable

\usepackage{multicol,supertabular}

\setlength{\parindent}{0pt}

% \usepackage{amsmath}

\usepackage{enumerate}

% 'Inspired' by:
%% This is file `uwamaths.sty',
%%%     author   = "Greg Gamble",
%%%     email     = "gregg@csee.uq.edu.au (Internet)",

\makeatletter
\def\DOTSB{\relax}
\def\dotcup{\DOTSB\mathop{\overset{\textstyle.}\cup}}
 \def\@avr#1{\vrule height #1ex width 0pt}
 \def\@dotbigcupD{\smash\bigcup\@avr{2.1}}
 \def\@dotbigcupT{\smash\bigcup\@avr{1.5}}
 \def\dotbigcupD{\DOTSB\mathop{\overset{\textstyle.}\@dotbigcupD%
                               \vphantom{\bigcup}}}

\def\dotbigcupT{\DOTSB\smash{\mathop{\overset{\textstyle.}\@dotbigcupT%
                              \vphantom{\bigcup}}}%
                       \vphantom{\bigcup}\@avr{2.0}}
\def\dotbigcup{\mathop{\mathchoice{\dotbigcupD}{\dotbigcupT}
                                  {\dotbigcupT}{\dotbigcupT}}}
\let\disjunion\dotcup
\let\Disjunion\dotbigcup
\makeatother

\usepackage{amsmath}
\usepackage{amssymb}
% \usepackage{eucal}
\usepackage{mathrsfs}

% \usepackage{fullpage}

\usepackage{geometry}
\geometry{a4paper, left=2cm, right=2cm, top=2cm, bottom=2cm, includeheadfoot}

\setlength{\parindent}{0pt}
\setlength{\parskip}{6pt}


% are we in pdftex ????
\ifx\pdfoutput\undefined % We're not running pdftex
\else
\RequirePackage[colorlinks,hyperindex,plainpages=false]{hyperref}
\def\pdfBorderAttrs{/Border [0 0 0] } % No border arround Links
\fi

% \usepackage{fancyheadings}
\usepackage{fancyhdr}
\usepackage{pifont}

\pagestyle{fancy}
% \addtolength{\headwidth}{\marginparsep}
% \addtolength{\headwidth}{\marginparwidth}
%  \addtolength{\textheight}{2pt}

\newcommand{\ineqjton}{\overset{1\leq i,j \leq n}{i \neq j}}
\newcommand{\srightmark}{\rightmark}
\newcommand{\sfbfpg}{\sffamily\bfseries{\thepage}}
  \newcommand{\symenvelop}{%
     {\nullfont\ }\relax\lower.2ex\hbox{\large\Pisymbol{pzd}{41}}}
% \renewcommand{\chaptermark}[1]{\markboth{\thechapter.\ #1}}

\iffalse
% \lhead[\fancyplain{}{{\sfbfpg}}]{\fancyplain{}\bfseries\srightmark}
\lhead[\fancyplain{}{{\sfbfpg}}]{\fancyplain{}\sl\srightmark}
% \rhead[\fancyplain{}\bfseries\leftmark]{\fancyplain{}{{\sfbfpg}}}
\rhead[\fancyplain{}\sl\leftmark]{\fancyplain{}{{\sfbfpg}}}
\lfoot{\today}
\cfoot{Yotam Medini \copyright}
  \newcommand{\symenvelop}{%
     {\nullfont a}\relax\lower.2ex\hbox{\large\Pisymbol{pzd}{41}}}
\rfoot{\symenvelop\ \texttt{yotam.medini@gmail.com}}

\renewcommand{\headrulewidth}{0.4pt}
\renewcommand{\footrulewidth}{0.4pt}
\fi

\setlength{\headheight}{16pt}
\fancyplain{plain}{%
 \fancyhf{}
 \fancyhead[LE,RO]{\fancyplain{}{{\sfbfpg}}}
 \fancyhead[RE,LO]{\sl\leftmark}
 \fancyfoot[L]{\today}
 \fancyfoot[C]{Yotam Medini \copyright}
 \fancyfoot[R]{\symenvelop\ \texttt{yotam.medini@gmail.com}}
 \renewcommand{\headrulewidth}{0.4pt}
 \renewcommand{\footrulewidth}{0.4pt}
}

% \usepackage{amstex}
% \usepackage{amsmath}
% \usepackage{amssymb}
\usepackage{amsthm}
\usepackage{bm}
\usepackage{makeidx}
\makeindex % enable

% 'Inspired' by:
%% This is file `uwamaths.sty',
%%%     author   = "Greg Gamble",
%%%     email     = "gregg@csee.uq.edu.au (Internet)",

\makeatletter
\def\DOTSB{\relax}
\def\dotcup{\DOTSB\mathop{\overset{\textstyle.}\cup}}
 \def\@avr#1{\vrule height #1ex width 0pt}
 \def\@dotbigcupD{\smash\bigcup\@avr{2.1}}
 \def\@dotbigcupT{\smash\bigcup\@avr{1.5}}
 \def\dotbigcupD{\DOTSB\mathop{\overset{\textstyle.}\@dotbigcupD%
                               \vphantom{\bigcup}}}

\def\dotbigcupT{\DOTSB\smash{\mathop{\overset{\textstyle.}\@dotbigcupT%
                              \vphantom{\bigcup}}}%
                       \vphantom{\bigcup}\@avr{2.0}}
\def\dotbigcup{\mathop{\mathchoice{\dotbigcupD}{\dotbigcupT}
                                  {\dotbigcupT}{\dotbigcupT}}}
\let\disjunion\dotcup
\let\Disjunion\dotbigcup
\makeatother


\newcommand{\half}{\ensuremath{\frac{1}{2}}}



\newcommand{\C}{\ensuremath{\mathbb{C}}} % The Complex set
\newcommand{\aded}{\ensuremath{\textrm{a.e.}}} % almost everyehere
\newcommand{\chhi}{\raise2pt\hbox{\ensuremath\chi}}           %raise the chi
\newcommand{\calA}{\ensuremath{\mathcal{A}}}
\newcommand{\calB}{\ensuremath{\mathcal{B}}}
\newcommand{\calE}{\ensuremath{\mathcal{E}}}
\newcommand{\calF}{\ensuremath{\mathcal{F}}}
\newcommand{\calG}{\ensuremath{\mathcal{G}}}
\newcommand{\calM}{\ensuremath{\mathcal{M}}}
\newcommand{\calR}{\ensuremath{\mathcal{R}}}
\newcommand{\eqdef}{\ensuremath{\stackrel{\mbox{\upshape\tiny def}}{=}}}
\newcommand{\frakB}{\ensuremath{\mathfrak{B}}}
\newcommand{\frakC}{\ensuremath{\mathfrak{C}}}
\newcommand{\frakF}{\ensuremath{\mathfrak{F}}}
\newcommand{\frakG}{\ensuremath{\mathfrak{G}}}
\newcommand{\frakI}{\ensuremath{\mathfrak{I}}}
\newcommand{\frakM}{\ensuremath{\mathfrak{M}}}
\newcommand{\scrA}{\ensuremath{\mathscr{A}}}
\newcommand{\scrB}{\ensuremath{\mathscr{B}}}
\newcommand{\scrD}{\ensuremath{\mathscr{D}}}
\newcommand{\scrF}{\ensuremath{\mathscr{F}}}
\newcommand{\scrN}{\ensuremath{\mathscr{N}}}
\newcommand{\scrP}{\ensuremath{\mathscr{P}}}
\newcommand{\scrQ}{\ensuremath{\mathscr{Q}}}
\newcommand{\scrR}{\ensuremath{\mathscr{R}}}
\newcommand{\scrT}{\ensuremath{\mathscr{T}}}
\newcommand{\Lp}[1]{\ensuremath{\mathbf{L}^{#1}}} % Lp space
\newcommand{\N}{\ensuremath{\mathbb{N}}} % The Natural Set
\newcommand{\bbP}{\ensuremath{\mathbb{P}}} % Some partially ordered set
\newcommand{\Q}{\ensuremath{\mathbb{Q}}} % The Rational set
\newcommand{\R}{\ensuremath{\mathbb{R}}} % The Real Set
\newcommand{\T}{\ensuremath{\mathbb{T}}} % The Thorus [-pi,\pi)
\newcommand{\Z}{\ensuremath{\mathbb{Z}}} % The Integer Set
\newcommand{\intR}{\int_{-\infty}^{\infty}} % Integral over the reals
\newcommand{\posthat}[1]{#1{\,\hat{}\,}}

% sequences
\newcommand{\seq}[2]{\ensuremath{#1_1,\ldots,#1_{#2}}}
\newcommand{\seqn}[1]{\seq{#1}{n}}
\newcommand{\seqan}{\seq{a}{n}}
\newcommand{\seqxn}{\seq{x}{n}}
\newcommand{\seqalphn}{\seq{\alpha}{n}}

\newcommand{\mset}[1]{\ensuremath{\{#1\}}}


%%%%%%%%%%%%
%% math op's
\newcommand{\Alt}{\mathop{\rm Alt}\nolimits}
\newcommand{\Ang}{\mathop{\rm Ang}\nolimits}
\newcommand{\Arg}{\mathop{\rm Arg}\nolimits}
\newcommand{\co}{\mathop{\rm co}\nolimits}
\newcommand{\conv}{\mathop{\rm conv}\nolimits}
\newcommand{\diam}{\mathop{\rm diam}\nolimits}
\newcommand{\dom}{\mathop{\rm dom}\nolimits}
% \newcommand{\dim}{\mathop{\rm dim}\nolimits}
% \newcommand{\esssup}{\mathop{\rm ess\ sup}\nolimits}
\DeclareMathOperator*{\esssup}{ess\,sup}
\newcommand{\ext}{\mathop{\rm ext}\nolimits}
\newcommand{\Id}{\mathop{\rm Id}\nolimits}
\newcommand{\Image}{\mathop{\rm Im}\nolimits}
\newcommand{\Ind}{\mathop{\rm Ind}\nolimits}
\newcommand{\Lip}{\mathop{\rm Lip}\nolimits}
\newcommand{\lip}{\mathop{\rm lip}\nolimits}
\newcommand{\percB}{
  \mathbin{\ooalign{$\hidewidth\%\hidewidth$\cr$\phantom{+}$}}}
\newcommand{\bres}[2]{\ensuremath{#1 \percB #2}}

\newcommand{\Ker}{\mathop{\rm Ker}\nolimits}
\newcommand{\rank}{\mathop{\rm rank}\nolimits}
\newcommand{\rng}{\mathop{\rm rng}\nolimits}
\newcommand{\Res}{\mathop{\rm Res}\nolimits}
\newcommand{\supp}{\mathop{\rm supp}\nolimits}
\newcommand{\vol}{\mathop{\rm vol}\nolimits}
\newcommand{\vspan}{\mathop{\rm span}\nolimits}

% I wish this was more standardized
\renewcommand{\Re}{\mathop{\bf Re}\nolimits}
\renewcommand{\Im}{\mathop{\bf Im}\nolimits}

\newcommand{\inter}[1]{\ensuremath{#1^{\circ}}}  % interior
\newcommand{\closure}[1]{\ensuremath{\overline{#1}}} % closure
\newcommand{\boundary}[1]{\ensuremath{\partial #1}} % closure


\newcommand{\ich}[1]{(\textit{#1})}
\newcommand{\itemch}[1]{\item[\ich{#1}]}
\newcommand{\itemdim}{\item[\(\diamond\)]}

% Special names
\newcommand{\Cech}{\u{C}ech}

\author{Yotam Medini}


%%%%%%%%%%%
%% Theorems
%%
\makeatletter
\@ifclassloaded{book}{
 \newtheorem{thm}{Theorem}[chapter]
 \newtheorem{cor}[thm]{Corollary}
 \newtheorem{lem}[thm]{Lemma}
 \newtheorem{llem}[thm]{Local Lemma}
 \newtheorem{lthm}[thm]{Local Theorem}
 % \newtheorem{quotecor}{Corollary}
 % \newtheorem{quotelem}{Lemma}[section]
 \newtheorem{quotethm}{Theorem}[chapter]
}{}
\makeatother
\newtheorem{Def}{Definition}

\newtheorem{manualtheoreminner}{Theorem}
\newenvironment{manualtheorem}[1]{%
  \renewcommand\themanualtheoreminner{#1}%
  \manualtheoreminner
}{\endmanualtheoreminner}

\newtheorem{manuallemmainner}{Lemma}
\newenvironment{manuallemma}[1]{%
  \renewcommand\themanuallemmainner{#1}%
  \manuallemmainner
}{\endmanuallemmainner}

\newcommand{\loclemma}{Lemma}


% \newcommand{\proofend}{\(\bullet\)}
% \newcommand{\proofend}{\hfill\(\blacksquare\)}
\newcommand{\proofend}{\hfill\(\Box\)}
\newenvironment{thmproof}
{\textbf{Proof.}}
{\proofend}

\newcommand{\chapterTypeout}[1]{\typeout{#1} \chapter{#1}}
\newcommand{\sectionTypeout}[1]{\typeout{#1} \section{#1}}

% abbreviations, ensuremath
\newcommand{\fx}{\ensuremath{f(x)}}
\newcommand{\gx}{\ensuremath{g(x)}}
\newcommand{\lrangle}[1]{\ensuremath{\left\langle #1 \right\rangle}}
\newcommand{\lrbangle}[1]{\ensuremath{\left\langle #1 \right\rangle}}
\newcommand{\M}{\ensuremath{\mathfrak{M}}}
\newcommand{\mldots}{\ensuremath{\ldots}}
\newcommand{\salgebra}{\(\sigma\)-algebra}
\newcommand{\swedge}{\;\wedge\;}
\newcommand{\wlogy}{without loss of generality}
\newcommand{\Wlogy}{Without loss of generality}
\newcommand{\twopii}{\ensuremath{2\pi i}}
\newcommand{\dtwopii}{\ensuremath{\frac{1}{\twopii}}}

% https://tex.stackexchange.com/
% questions/22252/how-to-typeset-function-restrictions
\newcommand\restr[2]{\ensuremath{% we make the whole thing an ordinary symbol
  \left.\kern-\nulldelimiterspace % automatically resize the bar with \right
  #1 % the function
  \vphantom{\big|} % pretend it's a little taller at normal size
  \right|_{#2} % this is the delimiter
  }}

\newenvironment{excopyOLD}
{\item\begin{minipage}[t]{.8\textwidth}\footnotesize}
{\smallskip\hrule\end{minipage}}

\newenvironment{excopy}
{\item % \relax
 \begin{list}{}{
 \setlength{\topsep}{0pt}
 \setlength{\partopsep}{0pt}
 \setlength{\itemsep}{0pt}
 \setlength{\parsep}{0pt}
 \setlength{\leftmargin}{0pt}
 \setlength{\rightmargin}{20pt}
 \setlength{\listparindent}{0pt}
 \setlength{\itemindent}{0pt}
 % \setlength{\labelsep}{0pt}
 \setlength{\labelwidth}{0pt}
 \footnotesize
 }
 \item
}
{\par
 % {\nullfont 0}
 \hrulefill
 \end{list}
}


\title{
 Notes and Solutions to Exercises\\
 for\\
 ``Algebra'' \quad by\quad  Serge Lang}
\author{Yotam Medini\\\texttt{yotam\_medini@yahoo.com}}

\newcommand{\Zm}[1]{\Z/#1\Z} % The Cyclic group

% Trivial group
\newcommand{\eG}{\ensuremath{\{e\}}}

\newcommand{\UNFINISHED}{\large\textbf{UNFINISHED!}}

% \newcommand{\disjunion}{\.\cup}      % Some use \sqcup or \uplus
% \newcommand{\Disjunion}{\.\bigsqcup} % Some use \bigsqcup or \biguplus
% \newcommand{\disjunion}{{\bigsqcup}}
% \newcommand{\Disjunion}{\bigsqcup}

\def\Aut{\mathop{\rm Aut}\nolimits}
\def\card{\mathop{\rm card}\nolimits}
\def\Ch{\mathop{\rm Ch}\nolimits}
\def\Id{\mathop{\rm Id}\nolimits}
\def\id{\mathop{\rm id}\nolimits}
\def\Inn{\mathop{\rm Inn}\nolimits}
\def\Irr{\mathop{\rm Irr}\nolimits}
\def\Ker{\mathop{\rm Ker}\nolimits}
\def\Map{\mathop{\rm Map}\nolimits}
\def\Stab{\mathop{\rm Stab}\nolimits}
\def\subnormal{\vartriangleleft}

% \renewenvironment{excopy}
% {\begin{minipage}[t]{.8\textwidth}\footnotesize}
% {\smallskip\hrule\end{minipage}}


\newcounter{myenumi}
\newenvironment{myenumerate}
{\begin{enumerate}
 \setcounter{enumi}{\themyenumi}
}
{\setcounter{myenumi}{\theenumi}
 \end{enumerate}}

% End of proof
% \newcommand{\eop}{{\small\quad\(\square\)}}

% \newtheorem{thm}{Theorem}[chapter]
% \newtheorem{cor}[thm]{Corollary}
% \newtheorem{lem}[thm]{Lemma}
% \newtheorem{llem}[thm]{Local Lemma}


\begin{document}
\maketitle
\newpage
\tableofcontents
\newpage

%%%%%%%%%%%%%%%%%%%%%%%%%%%%%%%%%%%%%%%%%%%%%%%%%%%%%%%%%%%%%%%%%%%%%%%%
%%%%%%%%%%%%%%%%%%%%%%%%%%%%%%%%%%%%%%%%%%%%%%%%%%%%%%%%%%%%%%%%%%%%%%%%
%%%%%%%%%%%%%%%%%%%%%%%%%%%%%%%%%%%%%%%%%%%%%%%%%%%%%%%%%%%%%%%%%%%%%%%%
\chapter*{Introduction}

This document is a companion to \cite{Lan94}.

%%%%%%%%%%%%%%%%%%%%%%%%%%%%%%%%%%%%%%%%%%%%%%%%%%%%%%%%%%%%%%%%%%%%%%%%
%%%%%%%%%%%%%%%%%%%%%%%%%%%%%%%%%%%%%%%%%%%%%%%%%%%%%%%%%%%%%%%%%%%%%%%%
\section*{Notation}

For each natural \(n\in\N\) we define
\begin{equation*}
\N_n \eqdef \{m\in\N: 1\leq m \leq n\} \qquad
\Z_n \eqdef \{m\in\Z: 0\leq m < n\}.
\end{equation*}

We borrow \textbf{C}-programming language modulo operator
to define the residue function
\begin{equation*}
\bres{n}{d} = n - n \left\lfloor \frac{n}{d} \right\rfloor
\end{equation*}


%%%%%%%%%%%%%%%%%%%%%%%%%%%%%%%%%%%%%%%%%%%%%%%%%%%%%%%%%%%%%%%%%%%%%%%%
%%%%%%%%%%%%%%%%%%%%%%%%%%%%%%%%%%%%%%%%%%%%%%%%%%%%%%%%%%%%%%%%%%%%%%%%
%%%%%%%%%%%%%%%%%%%%%%%%%%%%%%%%%%%%%%%%%%%%%%%%%%%%%%%%%%%%%%%%%%%%%%%%
\chapter{Groups}

%%%%%%%%%%%%%%%%%%%%%%%%%%%%%%%%%%%%%%%%%%%%%%%%%%%%%%%%%%%%%%%%%%%%%%%%
%%%%%%%%%%%%%%%%%%%%%%%%%%%%%%%%%%%%%%%%%%%%%%%%%%%%%%%%%%%%%%%%%%%%%%%%
\section{Notes}

%%%%%%%%%%%%%%%%%%%%%%%%%%%%%%%%%%%%%%%%%%%%%%%%%%%%%%%%%%%%%%%%%%%%%%%%
\subsection{Fixed proof of 5.5}

In an old edition:

Page~33 in the proof of Theorem~5.5.

The last paragraph of the proof deals with the case
in which the orbit of \(\langle\sigma\rangle\) has \(\geq3\) elements.
With the defined \(\tau = [krs]\) the text claims that
with \(\sigma' =  \tau\sigma\tau^{-1}\sigma^{-1}\) we have
\(\sigma'(i) = i\).

If we set \(\sigma = [ijkrs]\) then \(\sigma'(i)\neq i\) \emph{contrary}
to what is claimed in the text!

With the defined \(\tau = [krs]\) we actually have in the case
\begin{eqnarray}
\sigma'(i) & = & \tau\sigma\tau^{-1}\sigma^{-1}(i) \\
           & = & \tau\sigma\tau^{-1}(s) \\
           & = & \tau\sigma(r) \\
           & = & \tau(s) \\
           & = & k \neq i
\end{eqnarray}

In the current Third Edition, \(\tau = [rsk]\).

%%%%%%%%%%%%%%%%%%%%%%%%%%%%%%%%%%%%%%%%%%%%%%%%%%%%%%%%%%%%%%%%%%%%%%%%
\subsection{Lemma 8.3}
The \(c\in A_1\) may be explicitly taken as
\[c = p^{k-r}\mu a_1.\]

%%%%%%%%%%%%%%%%%%%%%%%%%%%%%%%%%%%%%%%%%%%%%%%%%%%%%%%%%%%%%%%%%%%%%%%%
%%%%%%%%%%%%%%%%%%%%%%%%%%%%%%%%%%%%%%%%%%%%%%%%%%%%%%%%%%%%%%%%%%%%%%%%
\section{Exercises (page 75)}

%%%%%%%%%%%%%%%%
\begin{myenumerate}
\addtolength{\itemsep}{10pt}

%%%%%
\begin{excopy}
Show that every group of order \(\leq 5\) is abelian.
\end{excopy}

If \(|G|=1\) then \(G=\eG\) and it is obvious.
For \(|G|\in\{2,3,5\}\) then the order is prime and $G$ is cyclic
and therefore abelian.

Now assume \(|G|=4\). If $G$ is cyclic then it is abelian.
If $G$ is not cyclic then \(G=(\Zm{2})\times(\Zm{2})\)
and a simple check verifies that $G$ is abelian.

%%%%%
\begin{excopy}
Show that there are two non-isomorphic groups of order $4$,
namely the cyclic one and the product of two cyclic groups of order $2$.
\end{excopy}

Note: This was actually used in the solution of previous exercise.
Let $G$ be a non cyclic group of order $4$ and
\(G = \{e, g_1, g_2, g_3\}\).
Since \(g_i^4=e\) for all \(i=1,2,3\) and therefore we must also
have \(g_i^2=e\) for all \(g_i\) since otherwise \(\{g_i^k\}_{k=0}^3\)
generates $G$ contradicting it being non-cyclic.
Denote \(h = g_1 g_2\). This product $h$
 cannot be equal to $e$ since \(g_1\)  and \(g_2\)
The product $h$ cannot be equal to either \(g_1\) nor \(g_2\)
Since neither of them is the unit.
Thus \(g_1 g_2 = g_3\). With this the homomorphism
of $G$ onto \((\Zm{2})\times(\Zm{2})\)
generated by:
\begin{eqnarray*}
 g_1 & \rightarrow &  (1,0) \\
 g_2 & \rightarrow &  (0,1) \\
\end{eqnarray*}
exists and satisfies isomorphism.


%%%%%
\begin{excopy}
Let $G$ be a group. A \textbf{commutator} in $G$ is
an element of the form \(aba^{-1}b^{-1}\) with \(a,b\in G\).
Let \(G^c\) be the subgroup generated by the commutators.
Then \(G^c\) is called the \textbf{commutator subgroup}.
Show that \(G^c\) is normal. Show that any homomorphism of
$G$ into an abelian group factors through \(G/G^c\).
\end{excopy}

Let \(g\in G\) and \(m\in G^c\).
By definition \(g^{-1}mgm^{-1} \in G^c\) and so
\(g^{-1}mg \in G^c m = G^c\) and \(G^c\) is normal.

Let $A$ be an abelian group and
\(h: G\rightarrow A\) a group homomorphism.
We need to show that \(G^c \subseteq \Ker h\).
It is sufficient to show it on the generators
\begin{eqnarray*}
h(aba^{-1}b^{-1}) & = & h(a)h(b)h(a^{-1})h(b^{-1}) \\
                  & = & h(a)h(a^{-1})h(b)h(b^{-1}) \\
                  & = & h(aa^{-1})h(bb^{-1}) \\
                  & = & h(e_G)h(e_G) = e_A\\
\end{eqnarray*}


%%%%%
\begin{excopy}
Let \(H,K\) be subgroups of a finite group $G$
with \(K \subset N_H\). Show that
\[\#(HK) = \frac{\#(H)\#(K)}{\#(H\cap K)}.\]
\end{excopy}

In example (\textbf{iv}) of \S 3,
% On page~17 (Example \textbf{iv})
it was shown that \emph{when} $H$ is contained in the normalizer of $K$, then
\[H/(H\cap K) \approx HK/K.\]
From this we get
\[\#(H)/\#(H\cap K) = \#(HK)/\#(K)\]
and the desired equality follows for the special case.

Now for the general case. Put \(G = H\cap K\).
For each \(h\in H\), \(k\in K\) and \(g\in G\)
we have \(hk = (hg^{-1})(gk)\).
Therefore
\begin{equation*}
\#(HK) \geq \frac{\#(H)\#(K)}{\#(G)}.
\end{equation*}
Conversely, given \(h_1\in H\) and \(k_1\in K\),
for any \(h_2\in H\) and \(k_2\in K\) satisfying:
\begin{equation*}
  h_1 k_1 = h_2 k_2 % \qquad\textnormal{where} \h_i\in H \land k_i = K\)
\end{equation*}
we have \(g = h_2^{-1}h_1 = k_2k_1^{-1} \in G\).
Now \(h_2 = h_1 g^{-1}\) and \(k_2 = gk_1\).
Therefore
\begin{equation*}
\#(HK) \leq \frac{\#(H)\#(K)}{\#(G)}.
\end{equation*}
and the desired equality follows.

%%%%%
\begin{excopy}
{\normalsize (Note: Using different notation.)}\newline
\textbf{Goursat's Lemma.} Let \(G_1, G_2\) be groups,
and let $H$ be a subgroup of \(G_1\times G_2\) such that
the two projections
\(p_i:H\rightarrow G_i\) for \(i=1,2\) are surjective.
Let \(N_1\) be the kernel of \(p_2\)
and \(N_2\) be the kernel of \(p_1\).
One can identify \(N_i\) as a normal subgroup of \(G_i\) (\(i=1,2\)).
Show that the image of $H$
in \(G_1/N_1 \times G_2/N_2\) is the graph of an isomorphism
\[G_1/N_1 \approx G_2/N_2.\]
\end{excopy}

It is easy to see that
\begin{eqnarray*}
\Ker{p_1} & = & H \cap (\{e_G\}\times G')\\
\Ker{p_2} & = & H \cap (G\times\{e_{G'}\})\\
\end{eqnarray*}

We have the natural mappings
\[\widetilde{p_i}: H \rightarrow G_i/N_i\qquad(i=1,2).\]
We first need to show that the graph of \((\widetilde{p_1},\widetilde{p_2})\)
defines a surjective bijection function
\begin{equation}\label{eq:g1n1g2n2}
G_1/N_1 \rightarrow G_2/N_2.
\end{equation}
The natural mapping \(H\rightarrow G_1/N_1\) is surjective
and the graph covers \(G_1/N_1\).
Hence it suffices to show that for any element of \(G_1/N_1\)
there is only one associated element in \(G_2/N_2\).
Let
\((g_1,g_2),(g'_1,g'_2)\) be any elements of $H$ such that
\[\widetilde{p_1}((g_1,g_2)) = \widetilde{p_1}((g'_1,g'_2)).\]
It is obvious that
\((g_1^{-1}g'_1,g_2^{-1}g'_2) \in \Ker(\widetilde{p_1})\)
and so \(g_2^{-1}g'_2\in N_2\) and thus \(g_2 N_2 = g'_2 N_2\).
This shows
\[\widetilde{p_2}(\left(g_1,g_2\right)) =
  \widetilde{p_2}(\left(g'_1,g'_2\right))\]
and the graph defines the mapping of (\ref{eq:g1n1g2n2}) we had to show.
Similarly, we can show the inverse mapping
\[G_2/N_2 \rightarrow G_1/N_1\]
and thus the graph is an surjective bijection.

Now to show homomorphism. Let
\(x_1N_1,y_1N_1 \in G_1/N_1\).
For \(i=1,2\)
There must be some
% \((a_i,b_i)\in H\) so
\((a_1,a_2),(b_1,b_2)\in H\) so
\(a_iN = x_iN\) and
\(b_iN = y_iN\).
The mapping (\ref{eq:g1n1g2n2}) we have established has:
\(x_1N_1 \rightarrow a_2N2\) and
\(y_1N_1 \rightarrow b_2N2\).
By looking at
\((a_1 b_1,a_2 b_2)\in H\) we get
\(x_1 y_1 N_1 \rightarrow a_2 b_2 N2\).


%%%%%
\begin{excopy}
Prove that the group of inner automorphisms of a group $G$
is normal in \(Aut(G)\).
\end{excopy}

The \textbf{inner} automorphisms
(\(\Inn(G)\), See: \cite{Scott87})
are the conjunctions.

Let \(T\in \Aut(G)\) and
let \(\gamma_x\in \Inn(G)\).
For \(y\in G\) We have
\(\gamma_x(y) = xyx^{-1}\) and
\begin{eqnarray*}
(T\gamma_x T^{-1})(y)
  & = & T(xT^{-1}(y)x^{-1})\\
  & = & T(x)T(T^{-1}(y))T(x^{-1})\\
  & = & T(x)yT(x^{-1})\\
  & = & \gamma_{T(x)}(y).\\
\end{eqnarray*}
So \(T\gamma_x T^{-1} \in \Inn(G)\) and \(\Inn(G)\) is normal.

%%%%%
\begin{excopy}
Let $G$ be a group such that \(\Aut(G)\) is cyclic.
Prove that $G$ is abelian.
\end{excopy}

Let \(T\in\Aut(G)\) be a generator.
For all \(g\in G\) we denote the inner mapping \(\gamma_g(x)=gxg^{-1}\).
So for any \(g\in G\) there exists a minimal \(n(g)\geq 0\) such that
\(T^{n(g)} = \gamma_g\). G is abelian iff \(n(g)=0\) for all \(g\in G\).
In negation we can assume there is some \(t \in G\)
with minimal \(m=n(t)>0\). We will show that \(\gamma_t\)
generates all of \(\Inn(G)\).

Let \(g\in G\), \(d=\lfloor n(g)/m\rfloor\) and the residue
\(r = n(g) - md\) where \(0\leq r<m\).
% Now \(\gamma_g = T^{n(g)} = T^r(T^m)^{d} = T^r \gamma_t^d\).
Set \(g'=gt^{-d}\) and we get
\[\gamma_{g'}(x) = (T^{n(g)}(T^m)^d)(x) = T^{n(g)-md}(x) = T^r(x).\]
By minimality of $m$ we must have \(r=0\).

So for all \(g\in G\) there exists a \(d\geq0\) such that
\(T^{md}=\gamma_g\) that is for all \(x\in G\)
\(t^{md}xt^{-md} = gxg^{-1}\). Substituting $x$ with $t$ we get
\(t = t^{md}tt^{-md} = gtg^{-1}\) from which we get \(gt=tg\)
and \(T^m\) is the identity that generates \(\Inn(G)\).
Since \(\Inn(G)=\{\Id_G\}\) we conclude that $G$ is abelian.



%%%%%
\begin{excopy}
Let $G$ be a group and let \(H, H'\) be subgroups.
By a \textbf{double coset} of \(H, H'\) one means
a subset of $G$ of the form \(HxH'\).
\begin{itemize}
  \item[(a)] Show that $G$ is a disjoint union of double cosets.
  \item[(b)] Let $C$ be a family of representatives for
     double cosets. For each \(a \in G\) denote by \([a]H'\)
     the conjugate \(aH'a^{-1}\) of \(H'\).
     For each \(c\in C\) we have a decomposition into ordinary cosets
  \begin{equation} \label{eq:decoxcHHp}
    H = \Disjunion_{x\in X_c} x(H\cap[c]H'),
    % H = \Disjunion_{x\in X_c} x(H\cap[c]H'),
  \end{equation}
     where \(X_c\) is a family of elements of $H$, depending on $c$.
     Show that the elements
     \(\{xc: c\in C,\,x\in X_c\}\) form a family of left cosets
     representatives for \(H'\) in $G$; that is,
  \begin{equation} \label{eq:decoGxccHp}
    G = \Disjunion_{c\in C}\,\Disjunion_{x\in X_c} xcH',
  \end{equation}

  \begin{quote}
   \textbf{Note:} In the original text,
   equation (\ref{eq:decoxcHHp})  appears as
   \[ H = \bigcup_c x_c(H\cap[c]H'),\]
   and equation (\ref{eq:decoGxccHp}) appears as
   \[ G = \bigcup_{x_c}\,\bigcup_{x_c} x_c cH'.\]
  I believe these are notational mistakes
   or unclear indices style.
  \end{quote}

\end{itemize}
\end{excopy}

To prove (a) let us assume that for some \(x_1,x_2\in G\)
\((Hx_1H')\cap(Hx_2H')\neq\emptyset\).
So we have \(h_1,h_2\in H\) and \(h'_1,h'_2\in H'\) so
\(h_1 x_1 h'_1 = h_2 x_2 h'_2\).
So we
\(x_1 = ({h_1}^{-1}h_2) x_2 (h'_2{h'_1}^{-1}).\)
Now for any \(g\in Hx_1H'\) there are
\(h\in H,h'\in H'\) such that we have
\[g = hx_1h' = (h{h_1}^{-1}h_2) x_2 (h'_2{h'_1}^{-1}) \in Hx_2H'.\]
Thus \(Hx_1H'\subset Hx_2H'\). Similarly,
\(Hx_2H'\subset Hx_1H'\) and they are equal and so the double cosets
are disjoint.

Now we turn to (b). For each \(c\in C\)
\(H_c = H\cap(cH'c^{-1})\subseteq H\). So we have
cosets of \(H_c\) form a disjoint union
\[H = \disjunion_{x\in X_c} xH_c = \disjunion_{x\in X_c} xcH'c{-1}\]
with some family \(X_c\) of representatives.

We compute
\begin{eqnarray} \label{eq:hch}
 HcH' & = & \left(\Disjunion_{x\in X_c} xcH'c^{-1}\right)cH' \\
      & = & \left(\Disjunion_{x\in X_c} xcH'c^{-1}c\right)H' \nonumber \\
      & = & \left(\Disjunion_{x\in X_c} xcH'\right)H' \nonumber\\
      & = & \Disjunion_{x\in X_c} xcH'H' \nonumber\\
      & = & \Disjunion_{x\in X_c} xcH'  \nonumber
\end{eqnarray}

From (a) we have a family $C$ of representatives of the double cosets
and using \ref{eq:hch} for substitution we get:
\[
G = \Disjunion_{c\in C} HcH'\\
  = \Disjunion_{c\in C} \, \Disjunion_{x\in X_c} xcH'.\]



%%%%%
\begin{excopy}
\begin{itemize}
 \item[(a)] Let $G$ be a group and $H$ a subgroup of finite index.
   Show that there exists a normal subgroup $N$ of $G$ contained in $H$
   and also of a finite index. [\emph{Hint}: If \((G:H)=n\),
   find a homomorphism if $G$ into \(S_n\) whose kernel is contained in $H$.]
 \item[(b)] Let $G$ be a group and let \(H_1\), \(H_2\) be subgroups of
   finite index. Prove that \(H_1\cap H_2\) has finite index.
\end{itemize}
\end{excopy}

Let $G$ act on the set of cosets of $H$ with \(\pi_g(cH) = gcH\)
for every \(g\in G\) and ever coset \(cH\) with \(c\in G\).
We want to show that this definition is independent of choice of
the $c$ representative.
So let \(c_1H = c_2H\), % so there must be some \(h\in H\)
and by associativity of the group multiplication
\[\pi_g(c_1H) = g(c_1H) = g(c_2H) = \pi_g(c_2)\]
and for each \(g_1,g_2\in G\), \(c\in G\)
\begin{eqnarray} \label{eq:pig1g2}
\pi_{g_1g_2}(cH) = (g_1g_2)(cH) = g_1(g_2(cH)) =  g_1(\pi_{g_2}cH) \\
  \hfill = \pi_{g_1}(\pi_{g_2}cH) = (\pi_{g_1}\pi_{g_2})(cH). \nonumber
\end{eqnarray}
The set of $H$ has $n$ elements.
By labeling the cosets
and from (\ref{eq:pig1g2})
we see that the association \(g\rightarrow \pi_g\)
is homomorphism\(T:G\rightarrow S_n\).
We have \(\Ker T \subseteq H\) since
for every \(g\in G\setminus H\) we have \(gH\neq H\).
Obviously, \(G/\Ker(T) \equiv T(G) \subseteq S_n\) and so
\[[G:H] \leq [G:\Ker(T)] = |T(G)| \leq |S_n| = n! < \infty.\]

To prove (b) we can assume that \(H_1\), \(H_2\) are normal.
Otherwise we simply use (a) and substitute them with normal subgroups.
Now we can use Example~(iv) of \S~3 (page~17)
\begin{equation}
H_1/(H_1\cap H_2) = H_1 H_2 / H_2
\end{equation}
that gives
\begin{equation}
[H_1:(H_1\cap H_2)] =
|G/(H_1\cap H_2)| =
|H_1 H_2 / H_2| \leq
|G/H_2|
\end{equation}
and so
\begin{equation}
[G:(H_1\cap H_2)] =
[G:H_1]\cdot[H_1:(H_1\cap H_2)] \leq
|G/H_1|\cdot|G/H_2| < \infty.
\end{equation}



%%%%%
\begin{excopy}
Let $G$ be a group and let $H$ be a subgroup of a finite index.
Prove that there is only a finite number of right cosets of $H$, and that
the number of right cosets is equal to the number of left cosets.
\end{excopy}

We will show a one to one surjective mapping between
the left cosets to the right cosets. It is defined by
\begin{equation} \label{eq:xH2Hx}
xH \rightarrow Hx^{-1} \qquad\textrm{for\ } x\in G
\end{equation}

The map is obviously surjective since \(x\rightarrow x^{-1}\) is.
Now assume \(Hx^{-1} = Hy^{-1}\). Then  \(x^{-1}y\in H\)
and so \((x^{-1}y)^{-1} = y^{-1}x \in H\) and so \(xH = yH\)
and so (\ref{eq:xH2Hx}) is one to one.

%%%%%
\begin{excopy}
Let $G$ be a group, and $A$ a normal abelian subgroup.
Show that \(G/A\) operates on $A$ by conjunction,
and in this manner get a homomorphism of \(G/A\) into \(\Aut(A)\).
\end{excopy}

Let \(xA\in G/A\) operate as \(a\mapsto xax^{-1}\) for all \(a\in A\).
To show that this operation as well defined, assume \(xA=yA\).
So \(x^{-1}y, y^{-1}x\in A\) and using the fact that $A$ is abelian, we get
\begin{equation}
xax^{-1} =
xax^{-1}(xy^{-1})^{-1}(xy^{-1}) =
(xy^{-1})^{-1}xax^{-1}(xy^{-1}) =
yay^{-1}.
\end{equation}
Now since \(y(xax^{-1}y^{-1} = (yx)a(yx)^{-1}\) the homomorphism follows.

\end{myenumerate}
\textbf{Semidirect product}
\begin{myenumerate}


%%%%%
\begin{excopy}
Let $G$ be a group and let $H$, $N$ be subgroups with $N$ normal.
Let \(\gamma_x\) be conjunction by an element \(x\in G\).
\begin{itemize}
 \item[(a)] Show that \(x\rightarrow \gamma_x\) induces
    a homomorphism \(f:\,H\mapsto\Aut(N)\).
 \item[(b)] If \(H\cap N = \eG\), show that the map
    \(H \times N \rightarrow HN\) given by
    \((x,y) \mapsto xy\) is a bijection, and that this map
    is an isomorphism if and only if $f$ is trivial,
    i.e. \(f(x) = \id_N\) for all \(x\in H\).
\end{itemize}
We define $G$ to be the \textbf{semidirect product} of $H$ and $N$
if \(G=NH\) and \(H\cap N = \eG\).
\begin{itemize}
 \item[(c)] Conversely, let $H$, $N$ be groups,and let
   \(\psi:\,H\mapsto \Aut(N)\) be a given homomorphism.
  Construct a semidirect product as follows.
  Let $G$ be the set of pairs \((x,h)\) with \(x\in N\) and \(h\in H\).
  Define the composition law
  \begin{equation}
    (x_1,h_1)(x_2,h_2) = (x_1x_2^{\psi(h_1x_2)}, h_1h_2),
  \end{equation}
  Show that this is a group law, and yields a semidirect product of $N$ and $H$,
  identifying
       $N$ with the set of elements \((x,1)\)
   and $H$ with the set of elements \((1,h)\).
\end{itemize}
\end{excopy}

\begin{itemize}

 \item[(a)] Let \(x,y\in H\). For any \(u\in N\)
 \[\gamma_{xy}(u) = xyu(xy)^{-1} = x\gamma_y(u)x^{-1} = \gamma_x(\gamma_y(u)).\]
 \item[(b)]
    Assume \(x_1y_1 = x_2y_2\) where \((x_1,y_1),\,(x_2,y_2)\in H\times N\).
   Multiplying both sides with
    \(x_1^{-1}\) from the left and
    \(y_2^{-1}\) from the right, we get
     \[x_2^{-1}x_1 = y_2y_1^{-1} \in H\cap N = \eG\]
   and thus \(x_1 = x_2\)
   and  \(y_1 = y_2\).
 \item[(c)]
    Seems that the problem isnot well formed.


\end{itemize}

%%%%%
\begin{excopy}
\begin{itemize}
 \item[(a)]
    Let $H$, $N$, be normal subgroups of a finite group $G$.
    Assume that the orders  of $H$ and $N$ are relatively prime.
    Prove that \(xy=yx\) for all \(x\in H\) and \((y\in N\),
    and that \(H\times N=HN\).
 \item[(b)]
    Let \(H_1,\ldots\,H_r\) be normal subgroups of $G$ such that the order
    of \(H_i\) is relatively prime to the order of \(H_j\) for \(i\neq j\).
    Prove that
    \begin{equation}
      H_1\times \ldots \times H_r = H_1\cdots H_r.
    \end{equation}
\end{itemize}
\end{excopy}

\begin{itemize}
 \item[(a)]
    We look at \(N\cap H\) since its order must divide both that
    of $H$ and $N$ we have \(|N\cap H| = 1\) and \(N\cap H = \eG\).
    Now
   \[ xyx^{-1}y^{-1} = (xyx^{-1})y^{-1} = x(yx^{-1}y^{-1}) \in N\cap H.\]
   and thus \(xyx^{-1}y^{-1} = e\) and we get \(xy=yx\).
 \item[(b)]
    Trivial by induction on $r$.
\end{itemize}


%%%%%
\begin{excopy}
Let $G$ be a finite group and let $N$ be a normal subgroup such that
$N$ and \(G/N\) have relatively prime orders.
 \begin{itemize}
   \item[(a)]
      Let $H$ be a subgroup of $G$ having the same order as \(G/N\).
      Prove that \(G = HN\).
   \item[(b)]
      Let $G$ be an automorphism of $G$. Prove that \(g(N) = N\).
 \end{itemize}
\end{excopy}

\begin{itemize}
 \item[(a)]
    The subgroup \(H\cap N\) has on order that must divide
    both $H$ and $N$ and therefore is $1$ and so the subgroup is trivial.
    Now assume \(h_1g_1 = h_2g_2\)
    for \(h_i\in H\), \(g_i\in N\), \(i=1,2\).
    Then \(h_2^{-1}h_1 = g_2g_1^{-1} \in H\cap N\) and
    \(h_1 = h_2\) and \(g_1 = g_2\). Thus counting the elements
    \(|HN| = |H|\cdot|N| = |G|\) and thus \(HN=G\).
 \item[(b)]
    Let \(h: N \rightarrow G/N\) be defined by \(h(x)=g(x)+N\).
    We will show the $h$ must be trivial.
    Now \(H=h(N)\) is a subgroup of \(G/N\) and \(|H|\)
    divides both \(|N|\) and \(G/N\) and therefore \(H=\eG\)
    which means that \(g(N)\subseteq N\). Since $g$ is automorphism
    we have \(g(N) = N\).
\end{itemize}

%%%%%
\begin{excopy}
\label{gop:nofix}
Let $G$ be a finite group operating on a finite set $S$ with \(\#(S)\geq2\).
Assume that there is only one orbit. Prove that there exist an element
\(x\in G\) which has no fixed point,i.e. \(xs\neq s\) for all \(s\in S\).
\end{excopy}

Solution using \cite{Scott87}~10.1.5.

Define \(\Ch(g) = |\{s\in S: gs=s\}|\)  \label{def:Ch}.
\begin{llem}
If $G$ acts on a finite set $S$ has $N$ orbits, then
\begin{equation}
\sum_{g\in G} \Ch(g) = n\cdot |G|.
\end{equation}
\end{llem}

Let \(T\subseteq S\) be an orbit of $G$ and \(a,b\in T\).
It is clear that \(|G_a| = |G_b| = |G|/|T|\).

We may view \(g\in G\) as permutations of $S$.
Let \(F = \{(s,g)\in S\times G: gs=s\}\) be the set of fixed points.
Now
\begin{eqnarray}
\sum_{g\in G} \Ch(g) & = & |F|\\
 & = & \sum_{s\in S} |G_s| \\
 & = & \sum_{T\ \textrm{orbit}} \sum_{s\in T} |G_s| \\
 & = & \sum_{T\ \textrm{orbit}} |T|\cdot|G|\\
 & = & n\cdot|G|.
\end{eqnarray}

Now back to the exercise, if \(n=1\)
\index{transitive}
($G$ is \emph{transitive})
and $G$ is finite then
\begin{eqnarray}
\sum_{g\in G} \Ch(g) = |G|.
\end{eqnarray}
Now assume by negation that for all \(g\in G\) \(\Ch(g)\geq 1\)
(at least one fixed point), then
\begin{eqnarray*}
|G| & = & \sum_{g\in G} \Ch(g) \\
    & = & \Ch(e) + \sum_{g\in G\setminus\eG} \Ch(g) \\
    & = & |S| + \sum_{g\in G\setminus\eG} \Ch(g) \\
    & \geq & |S| + |G| - 1
\end{eqnarray*}
Hence, \(|S| \leq 1\) which contradicts the assumption.

%%%%%
\begin{excopy}
Let $H$ be a proper subgroup of a finite group $G$. Show that $G$
is not the union of all the conjugates of $H$.
\end{excopy}

We look at $G$ as a group operating on the finite sets of the conjugates of $H$.
From the previous Exercise~\ref{gop:nofix}, there must be some \(x\in G\)
for which \(x(gHg^{-1})x^{-1} \neq gHg^{-1}\) for all \(g\in G\).
That is \(x\notin gHg^{-1}\) for all \(g\in G\) and
\[x \notin \bigcup_{g\in G}gHg^{-1}.\]

%%%%%
\begin{excopy}
Let $X$,$Y$ be finite sets and let $C$ be a subset of \(X\times Y\).
For \(x\in X\) let \(\phi(x)=\) number of elements \(y\in Y\) such that
\((x,y)\in C\). Verify that \[\#(C) = \sum_{x\in X}\varphi(x).\]

\emph{Remark}. A subset $V$ as in the above exercise is often called
\index{correspondence}
a \textbf{correspondence}, and \(\varphi(x)\) is the number of elements in $Y$
which correspond to a given element \(x\in X\).
\end{excopy}

This is simple result of looking at the disjoint union:
\[C = \Disjunion_{x\in X} \{(x,y)\in X\times Y: (x,y)\in C\}.\]

%%%%%
\begin{excopy}
Let $S$, $T$ be finite sets. Show that \(\#\Map(S,T) = (\#T)^{\#(S)}\).
\end{excopy}

Simple induction on \(\#(S)\).
If \(S'=S\cup\{x\}\) then we can extend each map \(S\rightarrow T\)
to \(S'\) by assigning \(\#(T)\) different values to $x$.

%%%%%
\begin{excopy}
Let $G$ be a finite group operating on a finite set $S$.
 \begin{itemize}
  \item[(a)]
    For each \(s\in S\) show that \[\sum_{t \in Gs} {\frac{1}{\#(Gt)}} = 1.\]
  \item[(b)]
    For each \(x \in G\) define \(f(x)=\) number of element \(s\in S\)
    such that \(xs=s\). Prove that the number of orbits of $G$ in $S$
    is equal to
      \[\frac{1}{\#(G)}\sum_{x\in G} f(x).\]
 \end{itemize}
\end{excopy}

\begin{itemize}
 \item[(a)]
   For all \(t \in Gs\) we have \(|Gs|=|Gt|\) and so
   \begin{equation}
   \sum_{t \in Gs} {\frac{1}{\#(Gt)}} =
   |Gs|\cdot{\frac{1}{|Gs|}} = 1,
   \end{equation}
 \item[(b)]
  Let us compute the number of ``fixed occurrences''.
  \begin{eqnarray}
   \sum_{x\in G} f(x)
     & = & \sum_{x\in G} \Ch(x)                 \label{eq:f2Ch} \\
     & = & \sum_{s\in S} |\{g\in G: gs = s\}|    \label{eq:Ch2gss} \\
     & = & \sum_{s\in S} |G_s|                   \label{eq:gss2Gs} \\
     & = & \sum_{\textrm{orbit\ } T\subseteq S}
             \sum_{t\in T} |G_t|                 \label{eq:Gsrob} \\
     & = & \sum_{\textrm{orbit\ } T\subseteq S}
             \sum_{t\in T} |G|/|Gt|              \label{eq:GfsGGs} \\
     & = & |G|\cdot\sum_{\textrm{orbit\ } T\subseteq S}
             \sum_{t\in T} 1/|Gt|                \label{eq:1overGs} \\
     & = & |G|\cdot\sum_{\overset{s\in S}{\textrm{orbits repr.}}}
             \sum_{t\in Gs} 1/|Gt|                \label{eq:orbrep} \\
     & = & |G|\cdot\sum_{\overset{s\in S}{\textrm{orbits repr.}}} 1.
                                                  \label{eq:orb1}
  \end{eqnarray}

  Equalities explanation:
  \begin{itemize}
   \item[(\ref{eq:f2Ch})] --- simply using the definition
                              in Exercise~\ref{def:Ch}.
   \item[(\ref{eq:Ch2gss})] --- counting fixed points via $S$ instead of $G$.
   \item[(\ref{eq:gss2Gs})] --- definition of \(G_s\).
   \item[(\ref{eq:Gsrob})] --- separating the summation over orbits.
   \item[(\ref{eq:GfsGGs})] --- basic result of group operating on set.
   \item[(\ref{eq:1overGs})] --- factoring \(|G|\) out
   \item[(\ref{eq:orbrep})] ---  Looking at an orbit $T$ via
                                 a representative \(s\in S\).
   \item[(\ref{eq:orb1})] --- Using the previous item (a) of this exercise.
  \end{itemize}

  Now we simply divide both ends of the equation by \(|G|\)
  to get the desired result.

\end{itemize}

\end{myenumerate}

Throughout, $p$ is a prime number.

\iffalse
% Global remark - so fake an item
\item[]
 \setlength{\leftmargin}{0pt}
 \setlength{\labelwidth}{0pt}
 \setlength{\labelwidth}{0pt}
 Throughout, $p$ is a prime number.
% {\nullfont kaka}
\addtocounter{enumi}{-1}
\fi

\begin{myenumerate}
%%%%%
\begin{excopy}
Let $P$ be a $p$-group. Let $A$ be a normal subgroup of order $p$.
Prove that $A$ is contained in the center of $P$.
\end{excopy}

We can view $P$ as operating on $A$ by conjunction.
That is for any \(x\in P\),we have \(\gamma_x(a) = xax^{-1}\).
Let \(x\in P\) and \(a\in A\) be any elements.
Say \(|P|=p^n\)
and so we have \(\gamma_x^{p^n}=\Id_A\).
% Assume \(\gamma_x(a)\neq a\), so \(a\neq e\) for sure.
Note that
\[\underbrace{\gamma_x(\gamma_x(\ldots(\gamma_x(}_{n\ \textrm{times}}a)\ldots))
 =  \gamma_x^n(a).\]
So because of \(A\setminus\eG\)
let \(k>0\) be the minimal such that \(\gamma_x^k(a)=a\).
So we have \(k|p^n\) and \(k\neq p-1\) and so \(k=1\) and \(\gamma_x(a)=a\).
Thus \(xa=ax\) and $a$ is in the center of $P$.


%%%%%
\begin{excopy}
Let $G$ be a finite group and $H$ a subgroup. Let \(P_H\) be
a $P$-Sylow subgroup of $H$. Prove that there exists a $p$-Sylow subgroup $P$
  of $G$ such that \(P_H = P\cap H\).
\end{excopy}

Since any $p$-subgroup is contained in a $p$-Sylow subgroup,
We have a subgroup $P$ such that \(P_H\subseteq P\subseteq G\).
Obviously \(P_H\subseteq P\cap H\).
Now \(P\cap H\) has an order that divides \(|P|\) so it is a power of $p$.
But from the maximality of \(P_H\) the equality follows.

%%%%%
\begin{excopy}
Let $H$ be a normal subgroup of a finite group $G$
and assume that \(\#(H)=p\). Prove that $H$ is contained in every $p$-Sylow
subgroup of $G$
\end{excopy}

We know that $H$ is contained in some $p$-Sylow subgroup S.
All $p$-Sylow subgroups are conjugates. Now for all \(x\in G\)
\[H=xHx^{-1}\subseteq xSx^{-1}.\]


%%%%%
\begin{excopy}
Let $P$, \(P'\) be $p$-Sylow subgroups of a finite group $G$.
\begin{itemize}
 \item[(a)]  If \(P'\subset N(P)\) (normalizer of $P$), then \(P'=P\).
 \item[(b)]  If \(N(P')=N(P)\), then \(P'=P\).
 \item[(c)]  We have \(N(N(P))=N(P)\).
\end{itemize}
\end{excopy}

\begin{itemize}
 \item[(a)] (Following argument in the proof of Theorem~6.4).
     Since \(P'\subseteq N(P)\)
     we have \(P'P\) is a subgroup of \(N(P)\) and $P$ is normal in it.
     Now
     \begin{equation}\label{eq:pppcp}
     (P'P:P) = (P':P'\cap P)
     \end{equation}
     (see~(iv) page~17).
     Now if by negation \(P'\neq P\) then $p$ divides
     the right side of~(\ref{eq:pppcp}) and so \(P'P\) contains
     a~$p$-Sylow subgroup with higher power of $p$ than that of $P$
     contradicting the fact that $P$ itself is a $p$-Sylow subgroup.
 \item[(b)] Immediate from (a) and the fact that \(P'\subseteq N(P')\).
 \item[(c)] By negation, say \(x\in N(N(P))\setminus N(P)\).
   So \(P' = xPx^{-1}\) is a $p$-Sylow subgroup and \(P'\neq P\).
   Now \[P' = xPx^{-1} \subseteq xN(P)x^{-1} = N(P)\]
   and from (a) we get \(P'=P\) a contradiction.
\end{itemize}
%%%%%%%%%%%%%%
\end{myenumerate}

%%%%%%%%%%%%%%%%%%%%%%%%%%%%%%%%%%%%%%%%%%
\textbf{Explicit determination of groups}

Let us have some lemmas.

\begin{llem} \label{llem:npdiv}
Let $G$ be a group or order $m$,
let \(p^r\) be the highest power of~$p$ that divides~$m$
and let~\(n_p\) be the number of $p$-Sylow subgroups.
Then \(n_p \mid m/p^r\).
\end{llem}
%\textbf{Proof:}
\begin{proof}
Immediate result from Proposition~5.2.
\end{proof}

\begin{llem} \label{rose94:p2q}
\textnormal{\small [See \cite{Rose94} Theorem~5.19]}
Let $G$ be a group and \(|G|=p^2q\) where $p$, $q$ are distinct primes,
then $G$ has a normal  Sylow subgroup and so $G$ is not simple.
\end{llem}
\begin{proof}
Indeed, if $G$ has a normal Sylow subgroup $H$ then
\[\eG\subnormal H \subnormal G\]
is an abelian tower by exercise \ref{ex:p2abel}.
This is true for either \(|H|=p^2\) or \(|H|=q\).
Thus $G$ is simple.

Now let's show the existence of a normal Sylow subgroup.

If \(q<p\) then by Lemma~6.7 the Sylow $p$-subgroup is normal.
So we now can assume \(p<q\).
Let \(n_p\) and \(n_q\) be the numbers of
Sylow $p$-subgroup and $q$-subgroups. By negation we assume
\(n_p > 1\) and \(n_q > 1\).
By local-lemma~\ref{llem:npdiv}
\begin{itemize}
 \item
   \(n_p\mid q\) and so \(n_p=q\).
   % Also \(n_q \equiv 1 \bmod\).
 \item
   \(n_q\mid p^2\), hence \(n_q=p\) or \(n_q=p^2\).
   But  if \(n_q=p\) then by \(n_q\equiv 1 \bmod q\) we have \(p>q\)
   contradicting our assumption and so \(n_q=p^2\).
\end{itemize}
Now any two different $q$-subgroups intersect in \eG
and so the number of elements of order $q$ is \(n_q(q-1)\).
The number of the ``non $q$ order'' elements in $G$ is \(p^2q - n_q(q-1)=p^2\).
Now a Sylow $p$-subgroup has an order \(p^2\) and all its elements
have order different than $q$ and so such subgroup is determined
by its \(p^2\) ``non $q$ order'' elements
and thus it is unique and normal.
\end{proof}

\begin{llem} \label{rose94:pqr}
\textnormal{\small [See \cite{Rose94} Theorem~5.20.]}
Let $G$ be a group and \(|G|=pqr\) where $p$, $q$, $r$ are distinct primes,
then $G$ is not simple.
\end{llem}
\begin{proof}
Let \(n_p\),\(n_q\) and \(n_r\) be the respective
numbers of Sylow subgroups.
Assume by negation that these three numbers are \(\>1\)
since otherwise
we have a normal Sylow subgroup and we are done.
Assume \(p>q>r\)
So any two distinct Sylow subgroups intersect in~\eG.
So the numbers of elements in $G$ of order $p$, $q$ and $r$
are
\(n_p(p-1)\), \(n_q(q-1)\) and \(n_r(r-1)\) respectively.
Therefore
\begin{equation}
|G|=pqr\geq 1 + n_p(p-1) + n_q(q-1) + n_r(r-1).
\end{equation}
By Sylow theorem, \(n_p \mid qr\) and \(n_p\equiv 1\bmod p\).
Since \(n_p>q\) and \(p>q\), \(p>r\), it follows that \(n_p=qr\).

Also \(n_q\mid pr\) and \(n_q\equiv 1\bmod q\)..
Since \(n_q>1\) and \(q>r\), it follows that \(n_q\geq p\).

Finally, \(n_r\mid pq\) so \(n_r\geq q\). Now we have
\begin{equation}
pqr \geq 1 + qr(p-1) + p(q-1) + q(r-1) = pqr + pq + qr - p - q  + 1,
\end{equation}
and hence \((p-1)(q-1)\leq 0\) which is impossible.
\end{proof}

\textbf{Definition:}
\textnormal{\small [See \cite{Rose94}~Exercise~90]}
Let $H$ be a subgroup of $G$.
\index{core!of group}
\index{normal interior}
Define the \emph{core} or \emph{normal interior} of $H$ in $G$ as
\begin{equation}
H_G = \bigcap_{g\in G} g^{-1}Hg
\end{equation}

It is clear that \(H_G\) is the largest normal subgroup of $G$ that
is contained in $H$.

\begin{llem} \label{llem:GsCoreSn}
\textnormal{\small [See \cite{Rose94}~Theorem~4.13]}
if $H$ is a subgroup of $G$ of finite index \(n=[G:H]\) then \(G/H_G\)
can be embedded in \(S_n\).
\end{llem}
\begin{proof}
Let \(\hat{H}\) be the set of $n$ left cosets of $H$ in $G$.
Let $G$ operate on this set by left multiplication.
Each \(g\in G\) permutates \(\hat{H}\).
By enumerating the cosets we identify \(S_n\)
with the permutations of \(\hat{H}\) and we have a mapping
\(\rho: G \rightarrow S_n\).
The kernel of \(\rho\) is the elements \(g\in G\)
for which each coset is fixed.

Let's first compute the \index{stabilizer} \index{Stab}
of a coset \(xH\)
\begin{eqnarray}
\Stab_G(xH)
  & = & \{g\in G: gxH=xH\} \\
  & = & \{g\in G: x^{-1}gxH=H\} \\
  & = & \{g\in G: x^{-1}gx \in H\} \\
  & = & \{g\in G: g \in xHx^{-1}\} \\
  & = & xHx^{-1}
\end{eqnarray}
We compute:
\begin{eqnarray}
\Ker\rho
  & = & \bigcap_{xH\in\hat{H}} \Stab_G(xH) \\
  & = & \bigcap_{g\in G} \Stab_G(gH) \\
  & = & \bigcap_{g\in G} gHg^{-1} \\
  & = & H_G. \\
\end{eqnarray}
And so \(G/H_G\approx \rho(G)\) a subgroup of \(S_n\).
\end{proof}


\begin{llem} \label{llem:pmr:divfac}
\textnormal{\small [See \cite{Rose94}~Exercise~279]}
Let $G$ be a simple group of order \(p^m r\) where \(p\nmid r\).
Then \(p^m \mid (r-1)!\).
\end{llem}
\begin{proof}
Let $H$ be a $p$-Sylow subgroup if $G$. Since $G$ is simple \(H_G = \eG\)
and from local-lemma~\ref{llem:GsCoreSn} $G$ can be embedded in \(S_{[G:H]}\).
Hence \(p^m r \mid r!\) and therefore \(p^m \mid (r-1)!\).
\end{proof}

\begin{myenumerate}

%%%%%
\begin{excopy}
Let $p$ be a prime number. Show that a group of order \(p^2\)
is abelian, and that there are only two such groups up to isomorphism.
\end{excopy}  \label{ex:p2abel}

Let $G$ be the group and $Z$ its center and \(Z\subnormal G\).
If \(G=Z\) then clearly $G$ is abelian.
Assume by negation that \(Z\subsetneq G\).
Since $Z$ is not trivial by Theorem~6.5 it must be of order $p$
and so \(G/Z\) has an order of $p$ as well and is cyclic generated by \(a+Z\).
Now let  \(a_1,a_2 \in G\) be any elements in $G$. For \(i=1,2\) we can
have the representation \(a_i=a^{n_i}g_i\) where \(n_i\neq 0\) and \(g_i\in Z\).
Now
\begin{equation} \label{eq:a1a2}
a_1a_2 = a^{n_1}g_1 a^{n_2}g_2 =
  a^{n_1+n_2}g_1g_2 =
  a^{n_2}a^{n_1}g_2g_1 =
  a^{n_2}g_2a^{n_1}g_1 = a_2a_1.
\end{equation}
Thus $G$ is abelian.

Now by Theorem~8.2, $G$ is isomorphic to a product of cyclic $p$-group.
Hence, isomorphic to\, \(\Zm{p^2}\) \, or \, \(\Zm{p}\times\Zm{p}\).


%%%%%
\begin{excopy}
Let $G$ be a group of order \(p^3\), where $p$ is prime, and $G$ is not abelian.
Let $Z$ be its center. Let $C$ be a cyclic group of order $p$.
\begin{enumerate}[(a)]
\item Show that \(Z \approx C\) and \(G/Z \approx C \times C\).
\item) Every subgroup of $G$ of order \(p^2\) contains $Z$ and is normal.
\item Suppose \(x^p = 1\) for all \(x \in G\),
 Show that $G$ contains a normal subgroup \hbox{\(H \approx C \times C\)}.
\end{enumerate}
\end{excopy}

\begin{itemize}
 \item[(a)]
     The $Z$ subgroup cannot be the whole $G$ since $G$ is not abelian.
     It cannot be trivial because of Theorem~6.5. That leaves the possibilities
     for its order to be $p$ or \(p^2\). If by negation the order is \(p^2\),
     then \(G/Z\) is cyclic and as in exercise~\ref{ex:p2abel}
     similar arguments like in (\ref{eq:a1a2}) gives a contradiction
     by showing that $G$ is abelian. Thus $Z$ is cyclic of order $p$
     and isomorphic to $C$ and \(G/Z\) is of order \(p^2\).

     Now from the previous exercise we know that groups of order \(p^2\)
     must be isomorphic to either \(\Zm{p}\times\Zm{p}\)
     or to the cyclic
     \(\Zm{p^2}\). The latter leads to contradiction that $g$ is abelian
     using the same arguments with \(G/Z\) cyclic.
     Thus \(G/Z\) is isomorphic to \(\Zm{p}\times\Zm{p}\) which is
     isomorphic to \(C\times C\).

 \item[(b)]
     Say $H$ is a subgroup of order \(p^2\). By Lemma~6.7 $H$ is normal.
     The subgroup \(H\cap Z\)
     could be or order $p$ or $1$.
     If by negation it is the latter case, then \(H\cap Z = \eG\).
     To show that \(HZ = \{hc: h\in H \, \textrm{and} \, c\in Z\}\)
     has exactly \(|H|\cdot|Z|=p^3\) we will show that the products
     differ. If
     \(h_1 c_1 = h_2 c_2\) with \(h_i\in H\), \(c_i\in Z\) we get
     \(c_1c_2^{-1} = h_1^{-1}h_2 \in H\cap Z\) and so this product equals $e$
     and \(h_1=h_2\), \(c_1=c_2\). Hence \(HZ=G\) and we can
     represent any \(a_1,a_2\in G\) by \(a_i=h_i c_i\) where
      \(h_i\in H\), \(c_i\in Z\) for \(i=1,2\).
      From exercise~\ref{ex:p2abel} $H$ is abelian and so
      \begin{equation}
      a_1a_2 = h_1 c_1 h_2 c_2 = h_1 h_2  c_1 c_2 =
          h_2 h_1  c_2 c_1 = h_2 c_2 h_1 c_1 = a_2a_1
      \end{equation}
      contradicting the fact that $G$ is abelian.
      Thus  \(H\cap Z\) is of order $p$ and $H$ must contain $Z$.

 \item[(c)]
      Examining the proof of Corollary~6.6 we see that
      in the sequence
      \[\eG=G_0 \subset G_1 \cdots \subset G_n = G\]
      every $p$-group $G$ has, the subgroup \(G_1\) is in the center.
      So in our case \(G_1 = Z\) and  let $H$ be \(G_2\)
      that has order of \(p^2\)
      and again by Lemma~6.7 is normal.
      Now $H$ cannot be cyclic, since if it were then
      its generator $x$ would not satisfy the required
      \(x^p=1\) equation.
      Now by exercise~\ref{ex:p2abel}  $H$
      must be isomorphic to \(C\times C\) and not to the cyclic \(\Zm{p^2}\).
\end{itemize}

%%%%%
\begin{excopy}
\begin{itemize}
 \item[(a)] Let $G$ be a group of order \(pq\), where $p$, $q$ are primes
            and \(p<q\). Assume that \(q\not\equiv 1 \bmod p\).
            Prove that  $G$ is cyclic.
 \item[(b)] Show that every group of order \(15\) is cyclic.
\end{itemize}
\end{excopy}  \label{ex:GpLTq}

\begin{itemize}
 \item[(a)]
   [Similar to the example on page~36 with $G$ of \(35\)].
   Let \(H_p\) and \(H_q\) be a $p$-Sylow and $q$-Sylow subgroups respectively.
   Then \(H_q\) is normal by Lemma~6.7.
   Now \(H_p\) operates by conjunction on \(H_q\) and we have
   a homomorphism \(H_p\rightarrow \Aut(H_q) \approx \Zm{(q-1)}\).
   So the image order must divide $p$ and \(q-1\).
   Since \(q-1\not\equiv 0 \bmod p\) clearly \(p\nmid q-1\)
   and so the image is trivial and so elements of \(H_p\) and \(H_q\)
   commutes with each other.

   We will show that \(H_pH_q = G\).
   The set \(H_{pq}=\{x_p^m x_q^n: 0\leq m<p, 0\leq n<q\}\)
   contains \(pq\) elements
   since if \(x_p^{m_1} x_q^{n_1} = x_p^{m_2} x_q^{n_2}\)
   we use the commutativity and the fact that \(H_p\cap H_q=\eG\)
   to get \(x_p^{m_1-m_2} = x_q^{n_2-n_1} = e\) and so \(H_pq=G\)
   and $G$ is abelian. By Proposition 4.3(\textbf{v}) $G$ is cyclic.

   Let \(x_p\) and \(x_q\) be generators of \(H_p\) and \(H_q\) respectively.
   % Then these generators commutes with each other and th
 \item[(b)] By (a) with \(p=3\), \(q=5\) and we have
      \(5\equiv 2\not\equiv 1 \bmod 3\).

\end{itemize}

%%%%%
\begin{excopy}
Show that every group of order \(<60\) is solvable.
We use the results from Corollary~6.6 and
exercises \ref{ex:GpLTq}, \ref{ex:p2q} and \ref{ex:2pq}.
\end{excopy}

{
% \begin{multicols}{2}

\tablefirsthead{\hline \(|G|\)   &   $=$ & $p$   &   $q$ & $r$ \\ \hline}
\tablehead{\hline \multicolumn{5}{|c|}{\small\textsl{continuation}} \\ \hline}
\tabletail{\hline \multicolumn{5}{|c|}{\small\textsl{to be continued}}\\ \hline}
\tablelasttail{\hline}
\begin{supertabular}{|r|c|r|r|r|}
 1 & \multicolumn{4}{|l|}{Trivial}  \\ \hline
 2 & $p$       & $2$  &      &   \\ \hline
 3 & $p$       & $3$  &      &   \\ \hline
 4 & \(p^2\)   & $3$  &      &   \\ \hline
 5 & $p$       & $5$  &      &   \\ \hline
 6 & $pq$      & $2$  & $3$  &   \\ \hline
 7 & $p$       & $7$  &      &   \\ \hline
 8 & \(p^n\)   & $2$  &      &   \\ \hline
 9 & \(p^n\)   & $3$  &      &   \\ \hline
10 & $pq$      & $2$  & $5$  &   \\ \hline
11 & $p$       & $11$ &      &   \\ \hline
12 & \(p^2q\)  & $2$  & $3$  &   \\ \hline
13 & $p$       & $13$ &      &   \\ \hline
14 & $pq$      & $2$  & $7$  &   \\ \hline
15 & $pq$      & $3$  & $5$  &   \\ \hline
16 & \(p^n\)   & $2$  &      &   \\ \hline
17 & $p$       & $17$ &      &   \\ \hline
18 & $p^2q$    & $3$  & $2$  &   \\ \hline
19 & $p$       & $19$ &      &   \\ \hline
20 & $p^2q$    & $2$  & $5$  &   \\ \hline
21 & $pq$      & $3$  & $7$  &   \\ \hline
22 & $pq$      & $2$  & $11$ &   \\ \hline
23 & $p$       & $23$ &      &   \\ \hline
% 24 &           &      &      &   \\ \hline
\hline
25 & \(p^n\)   & $5$  &      &   \\ \hline
26 & $pq$      & $2$  & $13$ &   \\ \hline
27 & \(p^n\)   & $3$  &      &   \\ \hline
28 & \(p^2q\)  & $2$  & $7$  &   \\ \hline
29 & $p$       & $29$ &      &   \\ \hline
30 & \(pqr\)   & $2$  & $3$  & $5$  \\ \hline
31 & $p$       & $31$ &      &   \\ \hline
32 & \(p^n\)   & $2$  &      &   \\ \hline
33 & $pq$      & $3$  & $11$ &   \\ \hline
34 & $pq$      & $2$  & $17$ &   \\ \hline
35 & $pq$      & $5$  & $7$  &   \\ \hline
\hline
37 & $p$       & $37$ &      &   \\ \hline
38 & $pq$      & $2$  & $19$ &   \\ \hline
39 & $pq$      & $3$  & $13$ &   \\ \hline
41 & $p$       & $41$ &      &   \\ \hline
42 & \(pqr\)   & $2$  & $3$  & $7$  \\ \hline
43 & $p$       & $43$ &      &   \\ \hline
44 & \(p^2q\)  & $2$  & $11$ &   \\ \hline
45 & \(p^2q\)  & $3$  & $5$  &   \\ \hline
46 & $pq$      & $2$  & $23$ &   \\ \hline
47 & $p$       & $47$ &      &   \\ \hline
\hline
49 & \(p^n\)   & $7$  &      &   \\ \hline
50 & \(p^2q\)  & $5$  & $2$  &   \\ \hline
51 & $p$       & $51$ &      &   \\ \hline
52 & \(p^2q\)  & $2$  & $13$ &   \\ \hline
53 & $p$       & $53$ &      &   \\ \hline
\hline
55 & $pq$      & $5$  & $11$ &   \\ \hline
\hline
57 & $pq$      & $3$  & $19$ &   \\ \hline
58 & $pq$      & $2$  & $29$ &   \\ \hline
59 & $p$       & $59$ &      &   \\ \hline
\end{supertabular}
% \end{multicols}
}

Now we need to solve some cases specifically.
Let $G$ be a group. For most orders upto $60$ solvability was shown
in the above table. In each of the following remaining cases
we will show the existence of some proper normal subgroup $H$.
Because of  results for lower orders of $G$,
A sequence
\(\eG\subnormal H \subnormal G\) can be completed to an abelian tower.

\begin{itemize}
 \item Assume \(|G|=24=2^3\cdot3\).\\
    From local-lemma \ref{llem:GsCoreSn} $G$ is not simple since otherwise
    \(2^3\mid(3-1)!\).
 \item Assume \(|G|=36=2^2\cdot3^2\).
    From local-lemma \ref{llem:GsCoreSn} $G$ is not simple since otherwise
    \(3^2\mid(4-1)!\).
 \item Assume \(|G|=40=2^3\cdot5\).
    Exercise~\ref{ex:G40G12} shows that $G$ is not simple.
 \item Assume \(|G|=48=2^4\cdot3\).
    From local-lemma \ref{llem:GsCoreSn} $G$ is not simple since otherwise
    \(2^4\mid (3-1)!\).
 \item Assume \(|G|=54=2\cdot3^3\).
    From Lemma~6.7 \(H_3\) is normal and $G$ is not simple
 \item Assume \(|G|=56=2^3\cdot7\).
    Let \(n_p\) be the number of $p$-Sylow subgroups for \(p=2,7\).
    Now \(n_7\equiv 1 \bmod 7\) and \(n_7\mid 8\).
    If \(n_7=1\) then such \(H_7\) is normal and we are done.
    % if by negation $G$ is simple, then \(n_7=8\). % and \(n_2=7\)

    Otherwise, we can assume \(n_7=8\). % and \(n_2=7\)
    Since such $7$-Sylow subgroups intersect in \eG,
    the number of elements
    in $G$ with order $7$ is \(n_7(7-1)=48\).
    The elements of any $2$-Sylow subgroup are of order \(\neq7\).
    And since \(56-48=8\) there could be only
    one such subgroup \(H_2\{g\in G: g^7\neq e\}\) whose order is $8$
    and must be normal.
\end{itemize}

%%%%%
\begin{excopy}
Let $p$, $q$ be distinct primes. Prove that a group of order \(p^2q\)
is solvable, and that one of its Sylow subgroups is normal.
\end{excopy}  \label{ex:p2q}

By Local Lemma~\ref{rose94:p2q} one of its Sylow subgroups is normal.
Then one of the normal towers
\begin{itemize}
 \item[] \(\eG \subnormal H_p \subnormal G\)
 \item[] \(\eG \subnormal H_q \subnormal G\)
\end{itemize}
exist and can be refined into abelian (and even cyclic) tower.

%%%%%
\begin{excopy}
Let $p$, $q$ be odd primes. Prove that a group of order \(2pq\) is solvable.
\end{excopy}  \label{ex:2pq}

By Local Lemma~\ref{rose94:pqr} such group $G$ has a normal subgroup $H$.

Now \(|H|\in \{p,q,r,pq,pr,qr\}\) and
by Proposition~6.8 the tower \(\eG\subnormal H\subnormal G\)
can be refined to abelian (and cyclic) tower.



%%%%%
\begin{excopy}
\begin{itemize}
 \item[(a)]
   Prove that one of the Sylow subgroups of a group of order $40$ is normal.
 \item[(b)]
   Prove that one of the Sylow subgroups of a group of order $12$ is normal.
\end{itemize}
\end{excopy} \label{ex:G40G12}

\begin{itemize}
\item[(a)] We have \(40=2^3\cdot5\) so the number of 5-Sylow subgroups
  must satisfy \(n_5\mid 8\) and \(n_5\equiv 1 \bmod 5\)
  and so \(n_5=1\) and the unique 5-Sylow subgroup is normal.
\item[(b)]
 From exercise~\ref{ex:p2q}  with \(p^2q=12\) where \(p=2\) and \(q=3\).
\end{itemize}

%%%%%
\begin{excopy}
Determine all groups of order \(\leq 10\) up to isomorphism.
In particular, show that a non-abelian group of order $6$
is isomorphic to \(S_3\).
\end{excopy}

The groups with prime order are cyclic. For other case of a group $G$:
\begin{itemize}
  \item Assume \(|G|=6=2\cdot3\). It could be isomorphic to:
    \begin{itemize}
       \item \(\Zm{6}\) cyclic.
       \item \(\Zm{2}\times\Zm{3}\) abelian.
       \item \(S_3\).
    \end{itemize}
  \item Assume \(|G|=8=2^3\). It could be isomorphic to:
    \begin{itemize}
       \item \(\Zm{8}\) cyclic.
       \item \(\Zm{2}\times\Zm{2}\times\Zm{2}\) abelian.
       \item \(\Zm{2}\times\Zm{4}\) abelian.
    \end{itemize}
  \item Assume \(|G|=9=3^2\). It could be isomorphic to:
    \begin{itemize}
       \item \(\Zm{9}\) cyclic.
       \item \(\Zm{3}\times\Zm{3}\) abelian.
    \end{itemize}
  \item Assume \(|G|=10=2\cdot5\). It could be isomorphic to:
    \begin{itemize}
       \item \(\Zm{10}\) cyclic.
       \item \(\Zm{2}\times\Zm{5}\) abelian.
    \end{itemize}

\end{itemize}

%%%%%
\begin{excopy}
Let \(S_n\) be the permutation group on $n$ elements.
Determine the $p$-Sylow subgroups of
\(S_3\), \(S_4\), \(S_5\) for \(p=2\) and \(p=3\).
\end{excopy}

\begin{itemize}
 \item[\(S_3\)]
    The $2$-Sylow subgroups % generated by transposition and they
    are:
    \(\{e,(12)\}\), \(\{e,(13)\}\) and \(\{e,(23)\}\).
    The $3$-Sylow subgroup is
    \(\{e,(123),(132)\}\).
 \item[\(S_4\)]
    The $2$-Sylow subgroup is \(S_4\) itself and no $3$-Sylow subgroups.
 \item[\(S_5\)] No $2$-Sylow and no $3$-Sylow subgroups.
\end{itemize}

%%%%%
\begin{excopy}
Let \(\sigma\) be a permutation of a finite set $I$ having $n$ elements.
Define \(e(\sigma)\) to be \((-1)^m\) where
\[m = n - \textrm{number of orbits of }\, \sigma.\]
If \(I_1,\ldots,I_r\) are orbits of \(\sigma\), then $m$ is also equal
to the sum
\[ m= \sum_{v=1}^r [\card(I_v)-1].\]
If \(\tau\)  is a transposition, show that \(e(\sigma\tau) = -e(\sigma)\)
be considering the two cases where $i$, $j$ lie in the same orbit of \(\sigma\),
or lie in different orbits. In the first case, \(\sigma\tau\) has one more
orbit and in theses case one less orbit that \(\sigma\).
In particular, the sign of a transposition is \(-1\).
Prove that \(e(\sigma)=\epsilon(\sigma)\) is the sign of the permutation.
\end{excopy}

We show the equality of $m$,
\[ \sum_{v=1}^r [\card(I_v)-1] =
   \sum_{v=1}^r \card(I_v) - \sum_{v=1}^r 1 =
   \sum_{v=1}^r \card(I_v) - \sum_{v=1}^r 1 =
   n - r.\]

We now show \(e(\sigma\tau)= -e(\sigma)\). Let \(\tau=(ij)\).
There are two cases:
\begin{itemize}
 \item
   The elements $i$, $j$  lie in the same orbit $I$ of \(\sigma\).
   Let \(l=|I|\geq 2\) and \(1\leq k<l\) such that \(\sigma^k(i)=j\).
   It is clear that \(\sigma^{l-k}(j)=i\).
   Now \(\sigma\tau\) has all the orbits of \(\sigma\)
   with $I$ split into two orbits:
     \((i, \sigma(j), \ldots \sigma^{l-k-1}(j))\)
   and
     \((j, \sigma(i), \ldots \sigma^{k-1}(j))\).
 \item
   The elements $i$, $j$  lie in different orbits \(I_i\) and \(I_j\)
   respectively of \(\sigma\).
   Now \(\sigma\tau\) has all the orbits of \(\sigma\)
   but with \(I_i\) and \(I_j\) united.
\end{itemize}
In both cases the number of orbits of \(\sigma\tau\) differs by $1$
from that of \(\sigma\).
Hence
\(e(\sigma\tau) = (-1)^{m+1} = -(-1)^m = -e(\sigma)\).

Since the transpositions  generates \(S_n\) and both $e$ and \(\epsilon\)
agree on the transpositions and the identity (\(e(\id)=\epsilon(\id)=1\))
they agree on all permutations.

%%%%%
\begin{excopy}
\begin{itemize}
 \item[(a)]
   Let $n$ be an even positive integer. Show that there exists  a group
   of order \(2n\), generated by two elements \(\sigma\), \(\tau\)
   such that \(\sigma^n=e=\tau^2\), and \(\sigma\tau=\tau\sigma^{n-1}\).
   (Draw a picture of a regular $n$-gon, number the vertices,
   and use the picture as an inspiration to get \(\sigma\), \(\tau\).)
   Thus group is called the
   \index{dihedral} \index{group!dihedral}
   \textbf{dihedral group}.
 \item[(b)]
   Let $n$ be an odd positive integer. Let \(D_{4n}\) be the group generated
   by the matrices
   \begin{equation}
     \left(
      \begin{array}{lr}
       0 & -1 \\
       1 & 0 \\
      \end{array}
     \right)
     \quad\textrm{and}\quad
     \left(
      \begin{array}{lc}
       \zeta & 0 \\
       0 & \zeta^{-1} \\
      \end{array}
     \right)
   \end{equation}
   where \(\zeta\) is a primitive $n$-th root of unity. Show that \(D_{4n}\)
   has order \(4n\), and give the commutation relations between the above
   generators.
\end{itemize}
\end{excopy}

\begin{itemize}
 \item[(a)]
 Let \(\theta=2\pi/n\). Now let\(\sigma\) be a \(1/n\) rotation
 and \(\tau\) a reflection. More formally:
   \begin{equation}
     \sigma = \left(
      \begin{array}{rl}
       \cos\theta & \sin\theta \\
       -\sin\theta & \cos\theta \\
      \end{array}
     \right)
     \quad\textrm{and}\quad
     \tau = \left(
      \begin{array}{lr}
       1 & 0 \\
       0 & -1 \\
      \end{array}
     \right)
   \end{equation}

 \item[(b)]
  Denote
   \begin{equation}
     \sigma = \left(
      \begin{array}{lr}
       0 & -1 \\
       1 & 0 \\
      \end{array}
     \right)
     \quad\textrm{and}\quad
     \tau = \left(
      \begin{array}{lc}
       \zeta & 0 \\
       0 & \zeta^{-1} \\
      \end{array}
     \right)
   \end{equation}

  From that we compute \(\sigma^2 = -\Id\) and \(\sigma^4 = \tau^n = \Id\).
  Thus \(\sigma^2\tau=\tau\sigma^2\) and \(\tau^n\sigma=\sigma\tau^n\).
\end{itemize}

\iffalse
\begin{itemize}
  \item[(a)] Rotation and mirroring. Consider the subgroup of \(S_n\)
     where
   \begin{equation*}
     \sigma(i) =
       \left\{
         \begin{array}{ll}
         i + i \;& \textnormal{if}\; i < n \\
         0     \;& \textnormal{if}\; i = n
         \end{array}
       \right.
   \end{equation*}
   and \(\tau(i) = (n - i + 1)\).

  \item[(b)]
  Say $J$ is the first matrix. Then
  \begin{equation*}
    J^2 =
      \left(
        \begin{array}{rr}
        -1 & 0 \\
        0  & -1
        \end{array}
      \right)
     \quad\textrm{and}\quad
    J^3 =
      \left(
        \begin{array}{rr}
        0 & 1 \\
        -1 & 0
        \end{array}
      \right)
     \quad\textrm{and}\quad
    J^4 =
      \left(
        \begin{array}{rr}
        1 & 0 \\
        0 & 1
        \end{array}
      \right).
  \end{equation*}
\end{itemize}
\fi


%%%%%
\begin{excopy}
Show that there are exactly two non-isomorphic non-abelian groups of order~$8$.
(one of them is given by the generators \(\sigma\), \(\tau\) with the relations
\begin{equation*}
\sigma^4 = 1, \qquad \tau^2 = 1, \qquad \tau\sigma\tau = \sigma^3.
\end{equation*}
The other is the quaternion group.)
\end{excopy}

Let $G$ be non-abelian group of order~$8$.
Let $m$ be the maximal period of the elements of $G$.
Since \(m|8\) we must have \(m\in\{1,2,4,8\}\).

We will show that \(m=4\).
If \(m=1\) then \(|G|=1\), contradiction.
If \(m=2\) then for any \(a,b\in G\)
\begin{equation*}
1 = (ab)(ab) = a(bb)a = (ab)(ba)
\end{equation*}
and so
\begin{equation*}
(ab)^{-1} = ab = ba
\end{equation*}
and $G$ is abelian, contradiction.
If \(m=8\) then $G$ is cyclic, contradiction.

Let \(\sigma\in G\) be of period $4$. It generates
the subgroup \(H=\{\sigma^1,\sigma^2,\sigma^3,1\}\).
\newcommand{\coH}{\ensuremath{\tilde{H}}}
Put  \(\coH = \coH\).
Since \([G:H]=2\) there are two cosets of $H$ and \coH\ in $G$.
Now for any \(g_1,g_2\in \coH\) we have
\begin{equation} \label{eq:H=g1g2H}
H = g_1 g_2 H = g_1 H g_2 = H g_1 g_2.
\end{equation}
and in particular \(H \triangleleft G\).

Since conjunction is automorphism, \(g\sigma g^{-1} \in H\) is
of period $4$ for each \(g\in G\).
Thus
\begin{equation*}
\forall g\in G,\; g\sigma g^{-1} \in \{\sigma^1,\sigma^3\}.
\end{equation*}

Assume by negation \(\exists y\in \coH,\; g\sigma g^{-1}=\sigma\).
Then
\begin{equation*}
\exists y\in \coH\, \forall k\in\{0,1,2,3\},\; g\sigma^kg^{-1}=\sigma^k.
\end{equation*}
Thus
\begin{equation*}
\exists y\in \coH\, \forall h\in H,\; yh=hy.
\end{equation*}
But any \(z \in \coH\) is of the form \(z = yh'\) for some \(h'\in H\)
and so for any \(h\in H\) we have
\begin{equation*}
zh = (yh')h = y(h'h) = (h'h)y = (hh')y = h(h'y) = hz.
\end{equation*}
and now
\begin{equation*}
\forall y\in \coH\, \forall h\in H,\; yh=hy.
\end{equation*}

Let \(y_1,y_2\in \coH\). By looking at \(\coH\) as a coset,
\(y_2 = h y_1 = y_1 h\) for some \(h\in H\).
Now since \((y_1)^2\in H\) as we saw in \eqref{eq:H=g1g2H}
\begin{equation*}
y_1 y_2 = y_1 (h y_1) = y_1 (y_1 h) = (y_1)^2 h = h (y_1)^2 = (h y_1)y_1
= (y_1 h) y_1 = y_2 y_1.
\end{equation*}
Thus $G$ is abelian, and by contradiction
\begin{equation*}
\forall y\in \coH,\; y\sigma y^{-1}=\sigma^3.
\end{equation*}
Similarly,
\begin{eqnarray*}
\forall y\in \coH,\; & y^{-1}\sigma y &= \sigma^3 \\
\forall y\in \coH,\; & y\sigma^3 y^{-1} &= \sigma \\
\forall y\in \coH,\; & y^{-1}\sigma^3 y &= \sigma.
\end{eqnarray*}

Clearly the periods of \(y\in \coH\) could be $2$ or $4$.
If all these periods are $2$ then

Assume the index of $y$ is $2$ for some \(y\in\coH\).
Then \(y^2=1\) and \(u=y^{-1}\) and so
\begin{equation*}
(\sigma y)^2 = \sigma(y \sigma y^{-1}) = \sigma^{1+3}=1.
\end{equation*}
Since conjunction is automorphism, \(y\sigma^2 y^{-1} = \sigma^2\)
begin the only element of order $2$ in $H$.
Noting that \(y\cdot 1 \cdot y{-1} = 1\)
we have
\begin{equation*}
\left\{y\sigma^n y^{-1}: n\in\{0,1,2\}\right\}
=
\left\{y\sigma^n y^{-1}: n\in\{0,3,2\}\right\}.
\end{equation*}
So by looking at the reminder
\(y\sigma^3 y^{-1} = \sigma\).
\begin{equation*}
(\sigma^3 y)^2 = \sigma^3 (y \sigma^3) y = \sigma^{3+1} = 1.
\end{equation*}
Thus all elements of \coH\ are of \emph{equal} period, $2$ or $4$.

Thus we have two possibilities.
\begin{enumerate}

\item If the order of \(y\in\coH\) is $2$ then $G$ is the diehedral group
with the relations specified in the exercise.

\item If the order of \(y\in\coH\) is $8$ then $G$ is the quaternion group.
We put \(i=\sigma\), pick arbitrary \(j\in\coH\), and put \(k=ij\).
Now since the orders of $j$ and $k$ are also $4$,
we have \(|\{i^3,j^3,k^3\}|=3\) (different elements) and
\begin{equation*}
rsr^{-1}=s^3 \qquad \textnormal{where} \qquad
(r,s) \in \left\{(i,j), (i,k), (j,i), (j, k), (k,i), k,j)\right\}.
\end{equation*}
and \(i^2 = j^2 = k^2\) which we conveniently denote as \((-1)\).
With this we have the 3 elements
\begin{eqnarray*}
(-i) &=& i^{3} = (-1)i = i(-1) \\
(-j) &=& j^{3} = (-1)j = j(-1) \\
(-k) &=& k^{3} = (-1)k = k(-1).
\end{eqnarray*}
Also
\begin{eqnarray*}
ji &=& ji(j^{-1}j) = (jij^{-1})j = i^{3}j = i^2k = (-1)k \\
jk &=& j(ij)(i^{-1}i) = (ii^{-1})j(ij) = i(i^{-1}ji)j = ij^{3+1} = i \\
kj &=& (kj)(k^{-1}k) = (kjk^{-1})k = j^{2+1}k = (-1)jk = (-1)i   \\
ik &=& ik(jj^3) = i(kj)j^3 = i^{1+2+1}j^{2+1} = (-1)j \\
ki &=& ki(k^{-1}k) = (kik^{-1})k = i^{2+1}k = (-1)(ik) = (-1)j
\end{eqnarray*}
\end{enumerate}


%%%%%
\begin{excopy}
Let \(\sigma = [123 \ldots n]\) in \(S_n\).
Show that the conjugacy class of \(\sigma\) has \((n - 1)!\) elements.
Show that the centralizer of \(\sigma\) is the cyclic group generated by
\(\sigma\).
\end{excopy}

The following relation on \(S_n\)
\begin{equation*}
\tau_1 \sim \tau_2
\qquad \textnormal{iff} \qquad
\exists k\in \N_n,\, \tau_1 \sigma^k = \tau_2.
\end{equation*}
is an equivalence relation, since
\(k=0\) gives reflexivity, symmetry by considering
\(\tau_2 \sigma^{n-k} = \tau_1\) and associativity
since if \(\tau_1 \sim \tau_2\)
and if \(\tau_2 \sim \tau_3\)
with \(k_1\) and \(k_2\) respectively, then
\begin{equation*}
\tau_1 \sigma^{k_3} = \tau_3
\qquad \textnormal{where}\; k_3 = k_1+k_2.
\end{equation*}

For any \(\tau\in S_n\) and \(k \in \N_n\)
\begin{equation*}
(\tau\sigma^k)\sigma\left(\tau\sigma^k\right)^{-1}
= \tau\sigma^{k+1-k}\tau^{-1}
= \tau\sigma\tau^{-1}.
\end{equation*}

Assume \(\tau_1 \nsim \tau_2\).
We choose \(\tau'_1 \sim \tau_1\)
and \(\tau'_2 \sim \tau_2\)
such that
\(\tau'_1(1) = \tau'_2(1)\).

Clearly we can find some \(j\in\N_n\) such that
\(\tau'_1(j) = \tau'_2(j) = j\)
and
\({\tau'_1}^{-1}(j + 1) \neq {\tau'_2}^{-1}(j + 1)\).
Now
\begin{equation*}
\left(\tau'_1 \sigma {\tau'_1}^{-1}\right)(j')
\neq
\left(\tau'_2 \sigma {\tau'_2}^{-1}\right)(j')
\end{equation*}
Thus
\begin{equation*}
\tau'_1 \sigma {\tau'_1}^{-1}
=
\tau'_2 \sigma {\tau'_2}^{-1}
\qquad \textnormal{iff} \qquad
\tau'_1 \sim \tau'_2.
\end{equation*}
Thus the size of the conjugacy class of \(\sigma\)
is the same as the number of classes of the equivalence relation \(\sim\)
which is \(n!/n = (n-1)!\).

%%%%%
\begin{excopy}
\begin{enumerate}[(a)]
\item
Let \(\sigma = [i_1\cdots i_m]\) be a cycle.
Let \(\gamma \in S_n\). Show that \(\gamma\sigma\gamma^{-1}\)
is  the cycle \([\gamma(i_1)\cdots \gamma(i_m)]\).
\item
Suppose that a permutation \(\sigma\) in \(S_n\) can be written
as a product of $r$ disjoint
cycles, and let \(d_1,\ldots,d_r\) be the number of elements in each cycle,
in increasing order.
 Let \(\tau\) be another permutation which can be written as a product of
disjoint cycles, whose cardinalities are
\(d'_1,\ldots,d'_r\)
in increasing order. Prove
that \(\tau\) is conjugate to \(\tau\) in \(S_n\) if and only if
\(r = s\) and \(d_i = d'_i\) for all \(i=1,\ldots,r\).
\end{enumerate}
\end{excopy}

\begin{enumerate}[(a)]
\item
\begin{equation*}
\left(\gamma\sigma\gamma^{-1}\right)(\gamma(i_j))
= \left(\gamma\sigma\right)\left(\gamma^{-1}\gamma\right)(i_j)
= \left(\gamma\sigma\right)(i_j)
= \gamma(\sigma(i_j))
= \gamma(i_{\bres{(j+1}{m}}).
\end{equation*}
\item
If \(\tau\) is conjugate to \(\sigma\) then the equalities
of the cycles sizes follow from (\emph{a}).
Conversely, if the cycles sizes agree then
if \([i_1,\ldots,i_m]\) is a cycle of \(\sigma\)
and  \([j_1,\ldots,j_m]\) is a cycle of \(\tau\)
we define \(\mu(i_k)=j_k\) for \(k\in \N_n\).
Since \(\N_n\) is a disjoint union of any permutation in \(S_n\)
we have \(\mu\in S_n\) well defined
and \(\tau = \mu\sigma\mu^{-1}\).
\end{enumerate}

%%%%%
\begin{excopy}
\begin{enumerate}[(a)]
\item
Show that \(S_n\) is generated by the transpositions
\([12]\), \([13]\),\(\ldots\),\([1n]\).
\item
Show that \(S_n\) is generated by the transpositions
\([12]\), \([23]\), \([34]\),\(\ldots\),\([n-1,n]\).
\item
Show that \(S_n\) is generated by the cycles \([12]\) and \([1 2 3 \ldots n]\).
\item
Assume that $n$ is prime.
Let \(\sigma = [1 2 3 \ldots n]\) and let \(\tau = [rs]\) be any transposition.
Show that \(\sigma\), \(\tau\) generate \(S_n\).
\end{enumerate}
\end{excopy}

Note that given a generator \(g\in S_n\),
we always have \(\Id\in S_n\) gernerated by \(\Id = g^k\) for some \(k\in\N\).
Given \(\sigma,\tau\in S_n\). We define the distance
\begin{equation*}
d(\sigma,\tau) = \left\|\{i\in\N_n: \sigma(i) \neq \tau(i)\}\right|.
\end{equation*}
\begin{enumerate}[(a)]
\item Let \(\sigma\in S_n\). Let \(G\subset S_n\) be the group generated
by the given generators. Let \(g\in G\) be with the minimal
distance with \(sigma\).
If by negation \(d(\sigma,g) > 0\) then let \(j\in\N_n\)
be the minimal index such that \(\sigma(j)\neq g(j)\).
Define
\begin{equation*}
g' = [1 j][1 g^{-1}\sigma(j)][1 j]g
\end{equation*}
and now \(d(\sigma, g') < d(\sigma, g)\) contrdicting the minimal choice.
\item
For each \(k \in \N_n\) we have
\begin{equation*}
[1 k] = [1 2]\cdot[2 3]\cdots[(k-2),(k-1)]\cdot[(k-1),k]\cdots[2 3]\cdot[1 2]
\end{equation*}
Thus we get the generators of previous case (a).
\item
For each \(k \in \N_n\) we have
\begin{equation*}
[k,(k+1)] = [1 2 3\ldots n]^{(k-1)}\cdot[12]\cdot [1 2 3\ldots n]^{n - (k-1)}.
\end{equation*}
Thus we get the generators of previous case (b).
\item
Assume \(d=r-s>0\). for any \(k\in \N_n\)
\begin{equation*}
[k,\bres{(k+d)}{n}] = \sigma^{k-r}[r s]\cdot\sigma^{r-k}
\end{equation*}
and
\begin{equation*}
\tau_k = [\bres{1 + kd}{n}, \bres{1 + (k+1)d}{n}].
\end{equation*}
are all generated.
Since $n$ is prime, for any \(m < n\) there exists \(q\in\N\)
such that \(d^q = (m - 1) \bmod n\). Thus
\begin{equation*}
[1 m] = \tau_0\cdot\tau_1\cdots\tau_q\cdots\tau_1\cdots\tau_0.
\end{equation*}
Hence the generators of (b) are generated.
\end{enumerate}

\end{myenumerate}

Let $G$ be a finite group operating on a set $S$.
Then $G$ operates in a natural way on
the Cartesian product \(S^{(n)}\) for each positive integer $n$ .
We define the operation on $S$
to be \hbox{\boldmath$n$\textbf{-transitive}} if given $n$ distinct elements
\((s_1,\ldots,s_n)\) and $n$ distinct elements
\((s'_1,\ldots,s'_n)\) of $S$, there exists \(\sigma\in G\)
such that \(\sigma s_i = s'_i\) for all \(i = 1,\ldots,n\).

\begin{myenumerate}
%%%%%
\begin{excopy}
Show that the action of the alternating group \(A_n\)
on \(\{1,\ldots,n\}\) is \((n - 2)\)-transitive.
\end{excopy}

Given arbitrary
\(n-2\) elements \((s_1,\ldots,s_{n-2})\)  in \(\N_n\)
and
\(n-2\) elements \((s'_1,\ldots,s'_{n-2})\)  in \(\N_n\).
We have a pair of two remaining elements
\begin{eqnarray*}
\{s_{n-1}, s_n\} &= \N_n \setminus \{(s_1,\ldots,s_{n-2}\}\\
\{s'_{n-1}, s'_n\} &= \N_n \setminus \{(s'_1,\ldots,s'_{n-2}\}.
\end{eqnarray*}
For \(i\in\N_n\) define  \(\sigma_1(s_i) = s'_i\) and
\begin{equation*}
  \sigma_2(s_i) =
    \left\{
      \begin{array}{ll}
        s'_i \quad &\textnormal{iff}\; i \leq n - 2 \\
        s'_n \quad &\textnormal{iff}\; i = n - 1 \\
        s'_{n-1} \quad &\textnormal{iff}\; i = n \\
      \end{array}
    \right.
\end{equation*}
Clearly \(\sigma_1,\sigma_2\in\S_n\) having different signs, thus
(exactly) one of them \(\in A_n\).

%%%%% ex 40
\begin{excopy}
Let \(A_n\) be the alternating group of even permutations of
\(\{1,\ldots,n\}\), For \(j = 1,\ldots,n\)
let \(H_j\) be the subgroup of \(A_n\) fixing $j$,
so \(H_j \approx A_{n-1}\), and \((A_n: H_j) = n\) for \(n > 3\).
Let \(n \geq 3\) and let $H$ be a subgroup of index $n$ in \(A_n\).
\begin{enumerate}[(a)]
\item
Show that the action of \(A_n\) on cosets of $H$ by left translation
gives an isomorphism \(A_n\) with the alternating group of permutations
of \(A_n/H\).
\item
Show that there exists an automorphism of \(A_n\) mapping \(H_1\) on $H$,
and that
such an automorphism is induced by an inner automorphism of \(S_n\) if and only
if \(H = H_i\) for some~$i$.
\end{enumerate}
\end{excopy}

Note that
\begin{equation*}
(A_n: H_j) = |A_n|/|A_{n-1}| = (n!/2)/\left((n-1)!/2)\right) = n!/(n-1)! = n.
\end{equation*}
Let \(A_n\) act on cosets \(\tau H_j\) by multiplication from left.
We need to show that this action is well defined.
Let \(\tau_1 H_j = \tau_2 H_j\),
Thus we have \(h_1,h_2\in H_j\) such that
\(\tau_1 h_1 = \tau_2 h_2\)
and so
\(\sigma\tau_1 h_1 = \sigma\tau_2 h_2\) for all \(\sigma \in S_n\)
and so
\(\sigma\tau_1 H_j = \sigma\tau_2 H_j\) for all \(\sigma \in A_n\)
showing that the action is well defined.
\begin{enumerate}[(a)]
\item
With similar argunent we showed that the action is well definded,
we can also show that the action is an homomorphism of \(A_n\)
on the permutations of \(A_n/H\).
We still need to show it is an injection and surjection..

We check manually for n=3,4.
\begin{equation*}
A_3 = \{e, [1,2,3], [1, 3, 2]\}
\end{equation*}
and thus \(|H|=1\) so \(H=\{e\}\) and the isomorphism is clear.

\begin{align*}
A_4 = \{&e, \\
  &[1,2,3], [1, 3, 2], [1,2,4], [1, 4, 2], [1, 3, 4], [1, 4, 3],
    [2, 3, 4], [2, 4, 3], \\
  &[1, 2][3, 4], [1, 3][2, 4], [1, 4][2, 3]\}.
\end{align*}
\(|A_4|=4!/2=12\) and \(|H|=|A_4|/4=3\).
The possible subgroups are of the
form \(\{e, h, h^2\}\) where
\(h \in \{[i,j,k]\in A_4: i\neq j \neq k\}\).
Note we always have \(h^3=e\).
Now there are 4 disjoint cosets \(eH, \tau_1 H, \tau_2 H, \tau_3 H\)
where \(\tau_i \in A_n\) for \(i=1,2,3\), and put \(\tau_0 = e\).
To show innjection, let \(\sigma_1, \sigma_2 \in A_4\)
and assume \(\sigma_2\tau_i H = \sigma_2\tau_i H\) for \(i=0,1,2,3\).
Looking at \(i=0\), we have
\(\sigma_1(e) \in \{\sigma_2 e, \sigma_2 h, \sigma_2 h^2\}\).\\
Cases:
\begin{itemize}
\item \(\sigma_1 e = \sigma_2 e\). Clearly \(\sigma_1 = \sigma_2\).
\item \(\sigma_1 e = \sigma_2 h\). Then \(\sigma_2^{-1}\sigma_1 = h\).
 \Wlogy we may assume \(h=[1,2,3]\).
Manually calculating permutations cycles,
 \(H = \{[e, [1, 2, 3], [1, 3, 2]\}\). The 4 cosets are:
\begin{align*}
H_1 &= \{e, [1, 2, 3], [1, 3, 2]\} = H \\
H_2 &= \{[2, 3, 4], [1, 3][2, 4], [1, 4, 2]\} \\
H_3 &= \{[2, 4, 3], [1, 2][3, 4], [1, 4, 3]\} \\
H_4 &= \{[1, 2, 4], [1, 3, 4], [1, 4][2, 3]\}.
\end{align*}
Now multiplying from left by all \(A_4\) permutations:
\begin{center}
\begin{tabular}{lll}
 i & \(\alpha \in A_4\) & $H$ indices of \(\alpha H_{1,2,3,4}\) \\
 1 & [1, 2, 3, 4] & [1, 2, 3, 4] \\
 2 & [1, 3, 4, 2] & [2, 3, 1, 4] \\
 3 & [1, 4, 2, 3] & [3, 1, 2, 4] \\
 4 & [2, 1, 4, 3] & [3, 4, 1, 2] \\
 5 & [2, 3, 1, 4] & [1, 3, 4, 2] \\
 6 & [2, 4, 3, 1] & [4, 1, 3, 2] \\
 7 & [3, 1, 2, 4] & [1, 4, 2, 3] \\
 8 & [3, 2, 4, 1] & [4, 2, 1, 3] \\
 9 & [3, 4, 1, 2] & [2, 1, 4, 3] \\
10 & [4, 1, 3, 2] & [2, 4, 3, 1] \\
11 & [4, 2, 1, 3] & [3, 2, 4, 1] \\
12 & [4, 3, 2, 1] & [4, 3, 2, 1]
\end{tabular}
\end{center}
The table shows the left- multiplication is injective, 
thus \(\sigma_1=\sigma_2\).

\item \(\sigma_1 e = \sigma_2 h^2\).
  Putting \(h' = h^2\) and then \(h'^2 = h^4 = h\).
  So we can apply the previous case.
\end{itemize}
That the end of handling \(A_3\).

Now we may assume \(n \geq 5\).
Assume \(\sigma \in A_n\) is in the kernel
of the left multiplication mapping.Then
\begin{equation} \label{eq:sigma-in-kern:An}
\forall \tau\in A_n\quad \sigma\tau H = \tau H.
\end{equation}
Now
\begin{equation*}
\sigma\tau H = \tau H
\;\Leftrightarrow\;
\tau^{-1}\sigma\tau H = H
\;\Leftrightarrow\;
\tau^{-1}\sigma\tau \in H
\;\Leftrightarrow\;
\sigma \in \tau H \tau^{-1}
\end{equation*}
Thus \eqref{eq:sigma-in-kern:An} gives
\begin{equation*}
\sigma \in \bigcap_{\tau\in A_n} \tau H \tau^{-1}
\end{equation*}
The latter intersection is clearly a normal subgroup of \(A_n\)
Since it is a subgroup of $H$ it is a proper subgroup of \(A_n\)
so it must be the trivial \(\{e\}\) since \(A_n\) is simple.
Thus the left multiplication is injective.


\begin{llem} \label{llem:half:normal}
Let $H$ be a subgroup of $G$. If \([G:H]=2\) then \(H \subnormal G\).
\end{llem}
\begin{proof}
Let \(g \in G\setminus H\) then clearly  \(gH\cap H = \emptyset\)
and since \(|gH| + |H| = |G|\) we have
\begin{equation*}
G = H \dotcup gH = H \dotcup g^{-1}H = H \dotcup Hg = H \dotcup Hg^{-1}.
\end{equation*}
Thus \(gH = Hg\) and so \(gHg^{-1}= (Hg)g^{-1} = H\).
\end{proof}

\begin{llem} \label{llem:unique:Sn:half}
For \(n\geq 2\) there exists a unique subgroup of \(S_n\) of order \(n!/2\)
namely \(A_n\).
\end{llem}
\begin{proof}
Let \(\sigma\;S_n\to \{+1,-1\}\) be the sign group homomorphism.
Its kernel is clearly \(A_n\).
Now let \(H\subset S_n\) with \(|H|=n!/2\).
By local-lemma~\ref{llem:half:normal} \(H\subnormal G\).
If \(n<5\) then we manually verify that  \(H==A_n\). 
We may now assume \(n\geq 5\).
Clearly \(H\cap A_n \subnormal A_n\). By Theorem~5.5
\(H\cap A_n = A_n\) and we are done, 
or by negation \(H\cap A_n = \{e\}\). Now since \(|H\cup A_n| = n! -1\)
clearly $H$ Contains \(n!/2 - 1\) odd permutations.
Pick 3 odd permutations, at least 2 of them \(\pi_1, \pi_2\)
satisfy \(\kappa=\pi_1\pi_2 \neq e\) and \(\kappa\) is even
thus \(\kappa \in A_n\) which gives a contradiction.
\iffalse
% Since \(G/H \simeq C_2 = \{+1,=1\}\)
% Consider the natural homomorphism \(\Lambda G \to G/H \{+1,=\{+1,=1\}\).
As was shown in the text, \(A_n\) is generated by all 3-cycles
using \([ij][rs] = [ijr][irs]\).
If by negation \(H \neq A_n\) then both $H$ and \(S_n\setminus H\)
contains some 3-cycles. Say
\begin{equation*}
[ijk] \in H \qquad [xyz]\in S_n\setminus H
\end{equation*}
Say \(I = \{i,j,k\}\cap \{x,y,z\}\).
Without loss of generality we may assume
that if \(|I|\geq 1\) then \(i=x\)
and  if \(|I|=2\) then \(j=y\)
Now consider
\begin{equation*}
\sigma =
 \left\{
  \begin{array}{ll}
   {[ix]}[jy][kz] \quad &\textnormal{if}\; |I|=0 \\
   {[jy]}[kz] \quad &\textnormal{if}\; |I|=1 \\
   {[kz]} \quad &\textnormal{if}\; |I|=2 \\
  \end{array}
 \right.
\end{equation*}
Now \([xyz] = \sigma [ijk] \sigma^{-1}\).
And by \(H \subnormal S_n\) we have the contrdiction \([xyz] \in H\).
\fi
\end{proof}

Back to the exercise.
The left multiplication $L$ maps \(A_n\) to permutations
of the $n$ cosets of $H$.
Thus we have a one-to-one mapping \(L: A_n \rightarrow S_n\).
Thus \(|L(A_n)|=|A_n|=|S_n|/2 = n!/2\).
By local-lemma~\ref{llem:unique:Sn:half} \(L(A_n)=A_n\).

\item
From the previous part, we have a 1-1 onto mapping
 \(L: A_n \to \Alt(A_n/H)\simeq A_n\).
Now for each \(\alpha \in H\)
\begin{equation*}
 L(\alpha) = eH, \tau_2 H, \ldots \tau_n H
\end{equation*}
Thus \(L(H) = H_1\).

Let \(\\sigma \in H_i\) and \(\tau\in A_n\) such that \(\tau(i)=j\)
then \(\tau\) acting by conjunction on \((H_1,H_2,\ldots,h_n\)
moves \(\tau H_i \tau^{-1} = H_j\).
Conversely, if \(H = \tau H_i \tau^{-1}\) then clearly \(H = H_j\).
\end{enumerate}

%%%%%
\begin{excopy}
Let $H$ be a simple group of order \(60\).
\begin{enumerate}[(a)]
\item
Show that the action of $H$ by conjugation on the set of its Sylow subgroups
gives an imbedding \(H \hookrightarrow A_6\).
\item
Using the preceding exercise, show that \(H \approx A_5\).
\item
Show that \(A_6\) has an automorphism which is not induced by an inner
automorphism of \(S_6\).
\end{enumerate}
\end{excopy}

%%%%%
\begin{excopy}
\end{excopy}

%%%%%
\begin{excopy}
\end{excopy}

\end{myenumerate}

%%%%%%%%%%%%%%%%%%%%%%%%%%%%%%%%%%%%%%%%%%%%%%%%%%%%%%%%%%%%%%%%%%%%%%%%
%%%%%%%%%%%%%%%%%%%%%%%%%%%%%%%%%%%%%%%%%%%%%%%%%%%%%%%%%%%%%%%%%%%%%%%%
%%%%%%%%%%%%%%%%%%%%%%%%%%%%%%%%%%%%%%%%%%%%%%%%%%%%%%%%%%%%%%%%%%%%%%%%
\bibliographystyle{plain}
\bibliography{langalg}

%%%%%%%%%%%%%%%%%%%%%%%%%%%%%%%%%%%%%%%%%%%%%%%%%%%%%%%%%%%%%%%%%%%%%%%%
%%%%%%%%%%%%%%%%%%%%%%%%%%%%%%%%%%%%%%%%%%%%%%%%%%%%%%%%%%%%%%%%%%%%%%%%
%%%%%%%%%%%%%%%%%%%%%%%%%%%%%%%%%%%%%%%%%%%%%%%%%%%%%%%%%%%%%%%%%%%%%%%%
% % $Id: langalg.tex,v 1.4 2001/05/04 12:24:45 yotam Exp yotam $
\documentclass[12pt]{book}
\usepackage{fullpage}
\usepackage{amsmath}
\usepackage{amssymb}
\usepackage{amsthm}
% \usepackage{amsthm}
\usepackage{makeidx}
\makeindex % enable

\usepackage{multicol,supertabular}

\setlength{\parindent}{0pt}

% \usepackage{amsmath}

\usepackage{enumerate}

% 'Inspired' by:
%% This is file `uwamaths.sty',
%%%     author   = "Greg Gamble",
%%%     email     = "gregg@csee.uq.edu.au (Internet)",

\makeatletter
\def\DOTSB{\relax}
\def\dotcup{\DOTSB\mathop{\overset{\textstyle.}\cup}}
 \def\@avr#1{\vrule height #1ex width 0pt}
 \def\@dotbigcupD{\smash\bigcup\@avr{2.1}}
 \def\@dotbigcupT{\smash\bigcup\@avr{1.5}}
 \def\dotbigcupD{\DOTSB\mathop{\overset{\textstyle.}\@dotbigcupD%
                               \vphantom{\bigcup}}}

\def\dotbigcupT{\DOTSB\smash{\mathop{\overset{\textstyle.}\@dotbigcupT%
                              \vphantom{\bigcup}}}%
                       \vphantom{\bigcup}\@avr{2.0}}
\def\dotbigcup{\mathop{\mathchoice{\dotbigcupD}{\dotbigcupT}
                                  {\dotbigcupT}{\dotbigcupT}}}
\let\disjunion\dotcup
\let\Disjunion\dotbigcup
\makeatother

\input{mymacs}

\title{
 Notes and Solutions to Exercises\\
 for\\
 ``Algebra'' \quad by\quad  Serge Lang}
\author{Yotam Medini\\\texttt{yotam\_medini@yahoo.com}}

\newcommand{\Zm}[1]{\Z/#1\Z} % The Cyclic group

% Trivial group
\newcommand{\eG}{\ensuremath{\{e\}}}

\newcommand{\UNFINISHED}{\large\textbf{UNFINISHED!}}

% \newcommand{\disjunion}{\.\cup}      % Some use \sqcup or \uplus
% \newcommand{\Disjunion}{\.\bigsqcup} % Some use \bigsqcup or \biguplus
% \newcommand{\disjunion}{{\bigsqcup}}
% \newcommand{\Disjunion}{\bigsqcup}

\def\Aut{\mathop{\rm Aut}\nolimits}
\def\card{\mathop{\rm card}\nolimits}
\def\Ch{\mathop{\rm Ch}\nolimits}
\def\Id{\mathop{\rm Id}\nolimits}
\def\id{\mathop{\rm id}\nolimits}
\def\Inn{\mathop{\rm Inn}\nolimits}
\def\Irr{\mathop{\rm Irr}\nolimits}
\def\Ker{\mathop{\rm Ker}\nolimits}
\def\Map{\mathop{\rm Map}\nolimits}
\def\Stab{\mathop{\rm Stab}\nolimits}
\def\subnormal{\vartriangleleft}

% \renewenvironment{excopy}
% {\begin{minipage}[t]{.8\textwidth}\footnotesize}
% {\smallskip\hrule\end{minipage}}


\newcounter{myenumi}
\newenvironment{myenumerate}
{\begin{enumerate}
 \setcounter{enumi}{\themyenumi}
}
{\setcounter{myenumi}{\theenumi}
 \end{enumerate}}

% End of proof
% \newcommand{\eop}{{\small\quad\(\square\)}}

% \newtheorem{thm}{Theorem}[chapter]
% \newtheorem{cor}[thm]{Corollary}
% \newtheorem{lem}[thm]{Lemma}
% \newtheorem{llem}[thm]{Local Lemma}


\begin{document}
\maketitle
\newpage
\tableofcontents
\newpage

%%%%%%%%%%%%%%%%%%%%%%%%%%%%%%%%%%%%%%%%%%%%%%%%%%%%%%%%%%%%%%%%%%%%%%%%
%%%%%%%%%%%%%%%%%%%%%%%%%%%%%%%%%%%%%%%%%%%%%%%%%%%%%%%%%%%%%%%%%%%%%%%%
%%%%%%%%%%%%%%%%%%%%%%%%%%%%%%%%%%%%%%%%%%%%%%%%%%%%%%%%%%%%%%%%%%%%%%%%
\chapter*{Introduction}

This document is a companion to \cite{Lan94}.

%%%%%%%%%%%%%%%%%%%%%%%%%%%%%%%%%%%%%%%%%%%%%%%%%%%%%%%%%%%%%%%%%%%%%%%%
%%%%%%%%%%%%%%%%%%%%%%%%%%%%%%%%%%%%%%%%%%%%%%%%%%%%%%%%%%%%%%%%%%%%%%%%
\section*{Notation}

For each natural \(n\in\N\) we define
\begin{equation*}
\N_n \eqdef \{m\in\N: 1\leq m \leq n\} \qquad
\Z_n \eqdef \{m\in\Z: 0\leq m < n\}.
\end{equation*}

We borrow \textbf{C}-programming language modulo operator
to define the residue function
\begin{equation*}
\bres{n}{d} = n - n \left\lfloor \frac{n}{d} \right\rfloor
\end{equation*}


%%%%%%%%%%%%%%%%%%%%%%%%%%%%%%%%%%%%%%%%%%%%%%%%%%%%%%%%%%%%%%%%%%%%%%%%
%%%%%%%%%%%%%%%%%%%%%%%%%%%%%%%%%%%%%%%%%%%%%%%%%%%%%%%%%%%%%%%%%%%%%%%%
%%%%%%%%%%%%%%%%%%%%%%%%%%%%%%%%%%%%%%%%%%%%%%%%%%%%%%%%%%%%%%%%%%%%%%%%
\chapter{Groups}

%%%%%%%%%%%%%%%%%%%%%%%%%%%%%%%%%%%%%%%%%%%%%%%%%%%%%%%%%%%%%%%%%%%%%%%%
%%%%%%%%%%%%%%%%%%%%%%%%%%%%%%%%%%%%%%%%%%%%%%%%%%%%%%%%%%%%%%%%%%%%%%%%
\section{Notes}

%%%%%%%%%%%%%%%%%%%%%%%%%%%%%%%%%%%%%%%%%%%%%%%%%%%%%%%%%%%%%%%%%%%%%%%%
\subsection{Fixed proof of 5.5}

In an old edition:

Page~33 in the proof of Theorem~5.5.

The last paragraph of the proof deals with the case
in which the orbit of \(\langle\sigma\rangle\) has \(\geq3\) elements.
With the defined \(\tau = [krs]\) the text claims that
with \(\sigma' =  \tau\sigma\tau^{-1}\sigma^{-1}\) we have
\(\sigma'(i) = i\).

If we set \(\sigma = [ijkrs]\) then \(\sigma'(i)\neq i\) \emph{contrary}
to what is claimed in the text!

With the defined \(\tau = [krs]\) we actually have in the case
\begin{eqnarray}
\sigma'(i) & = & \tau\sigma\tau^{-1}\sigma^{-1}(i) \\
           & = & \tau\sigma\tau^{-1}(s) \\
           & = & \tau\sigma(r) \\
           & = & \tau(s) \\
           & = & k \neq i
\end{eqnarray}

In the current Third Edition, \(\tau = [rsk]\).

%%%%%%%%%%%%%%%%%%%%%%%%%%%%%%%%%%%%%%%%%%%%%%%%%%%%%%%%%%%%%%%%%%%%%%%%
\subsection{Lemma 8.3}
The \(c\in A_1\) may be explicitly taken as
\[c = p^{k-r}\mu a_1.\]

%%%%%%%%%%%%%%%%%%%%%%%%%%%%%%%%%%%%%%%%%%%%%%%%%%%%%%%%%%%%%%%%%%%%%%%%
%%%%%%%%%%%%%%%%%%%%%%%%%%%%%%%%%%%%%%%%%%%%%%%%%%%%%%%%%%%%%%%%%%%%%%%%
\section{Exercises (page 75)}

%%%%%%%%%%%%%%%%
\begin{myenumerate}
\addtolength{\itemsep}{10pt}

%%%%%
\begin{excopy}
Show that every group of order \(\leq 5\) is abelian.
\end{excopy}

If \(|G|=1\) then \(G=\eG\) and it is obvious.
For \(|G|\in\{2,3,5\}\) then the order is prime and $G$ is cyclic
and therefore abelian.

Now assume \(|G|=4\). If $G$ is cyclic then it is abelian.
If $G$ is not cyclic then \(G=(\Zm{2})\times(\Zm{2})\)
and a simple check verifies that $G$ is abelian.

%%%%%
\begin{excopy}
Show that there are two non-isomorphic groups of order $4$,
namely the cyclic one and the product of two cyclic groups of order $2$.
\end{excopy}

Note: This was actually used in the solution of previous exercise.
Let $G$ be a non cyclic group of order $4$ and
\(G = \{e, g_1, g_2, g_3\}\).
Since \(g_i^4=e\) for all \(i=1,2,3\) and therefore we must also
have \(g_i^2=e\) for all \(g_i\) since otherwise \(\{g_i^k\}_{k=0}^3\)
generates $G$ contradicting it being non-cyclic.
Denote \(h = g_1 g_2\). This product $h$
 cannot be equal to $e$ since \(g_1\)  and \(g_2\)
The product $h$ cannot be equal to either \(g_1\) nor \(g_2\)
Since neither of them is the unit.
Thus \(g_1 g_2 = g_3\). With this the homomorphism
of $G$ onto \((\Zm{2})\times(\Zm{2})\)
generated by:
\begin{eqnarray*}
 g_1 & \rightarrow &  (1,0) \\
 g_2 & \rightarrow &  (0,1) \\
\end{eqnarray*}
exists and satisfies isomorphism.


%%%%%
\begin{excopy}
Let $G$ be a group. A \textbf{commutator} in $G$ is
an element of the form \(aba^{-1}b^{-1}\) with \(a,b\in G\).
Let \(G^c\) be the subgroup generated by the commutators.
Then \(G^c\) is called the \textbf{commutator subgroup}.
Show that \(G^c\) is normal. Show that any homomorphism of
$G$ into an abelian group factors through \(G/G^c\).
\end{excopy}

Let \(g\in G\) and \(m\in G^c\).
By definition \(g^{-1}mgm^{-1} \in G^c\) and so
\(g^{-1}mg \in G^c m = G^c\) and \(G^c\) is normal.

Let $A$ be an abelian group and
\(h: G\rightarrow A\) a group homomorphism.
We need to show that \(G^c \subseteq \Ker h\).
It is sufficient to show it on the generators
\begin{eqnarray*}
h(aba^{-1}b^{-1}) & = & h(a)h(b)h(a^{-1})h(b^{-1}) \\
                  & = & h(a)h(a^{-1})h(b)h(b^{-1}) \\
                  & = & h(aa^{-1})h(bb^{-1}) \\
                  & = & h(e_G)h(e_G) = e_A\\
\end{eqnarray*}


%%%%%
\begin{excopy}
Let \(H,K\) be subgroups of a finite group $G$
with \(K \subset N_H\). Show that
\[\#(HK) = \frac{\#(H)\#(K)}{\#(H\cap K)}.\]
\end{excopy}

In example (\textbf{iv}) of \S 3,
% On page~17 (Example \textbf{iv})
it was shown that \emph{when} $H$ is contained in the normalizer of $K$, then
\[H/(H\cap K) \approx HK/K.\]
From this we get
\[\#(H)/\#(H\cap K) = \#(HK)/\#(K)\]
and the desired equality follows for the special case.

Now for the general case. Put \(G = H\cap K\).
For each \(h\in H\), \(k\in K\) and \(g\in G\)
we have \(hk = (hg^{-1})(gk)\).
Therefore
\begin{equation*}
\#(HK) \geq \frac{\#(H)\#(K)}{\#(G)}.
\end{equation*}
Conversely, given \(h_1\in H\) and \(k_1\in K\),
for any \(h_2\in H\) and \(k_2\in K\) satisfying:
\begin{equation*}
  h_1 k_1 = h_2 k_2 % \qquad\textnormal{where} \h_i\in H \land k_i = K\)
\end{equation*}
we have \(g = h_2^{-1}h_1 = k_2k_1^{-1} \in G\).
Now \(h_2 = h_1 g^{-1}\) and \(k_2 = gk_1\).
Therefore
\begin{equation*}
\#(HK) \leq \frac{\#(H)\#(K)}{\#(G)}.
\end{equation*}
and the desired equality follows.

%%%%%
\begin{excopy}
{\normalsize (Note: Using different notation.)}\newline
\textbf{Goursat's Lemma.} Let \(G_1, G_2\) be groups,
and let $H$ be a subgroup of \(G_1\times G_2\) such that
the two projections
\(p_i:H\rightarrow G_i\) for \(i=1,2\) are surjective.
Let \(N_1\) be the kernel of \(p_2\)
and \(N_2\) be the kernel of \(p_1\).
One can identify \(N_i\) as a normal subgroup of \(G_i\) (\(i=1,2\)).
Show that the image of $H$
in \(G_1/N_1 \times G_2/N_2\) is the graph of an isomorphism
\[G_1/N_1 \approx G_2/N_2.\]
\end{excopy}

It is easy to see that
\begin{eqnarray*}
\Ker{p_1} & = & H \cap (\{e_G\}\times G')\\
\Ker{p_2} & = & H \cap (G\times\{e_{G'}\})\\
\end{eqnarray*}

We have the natural mappings
\[\widetilde{p_i}: H \rightarrow G_i/N_i\qquad(i=1,2).\]
We first need to show that the graph of \((\widetilde{p_1},\widetilde{p_2})\)
defines a surjective bijection function
\begin{equation}\label{eq:g1n1g2n2}
G_1/N_1 \rightarrow G_2/N_2.
\end{equation}
The natural mapping \(H\rightarrow G_1/N_1\) is surjective
and the graph covers \(G_1/N_1\).
Hence it suffices to show that for any element of \(G_1/N_1\)
there is only one associated element in \(G_2/N_2\).
Let
\((g_1,g_2),(g'_1,g'_2)\) be any elements of $H$ such that
\[\widetilde{p_1}((g_1,g_2)) = \widetilde{p_1}((g'_1,g'_2)).\]
It is obvious that
\((g_1^{-1}g'_1,g_2^{-1}g'_2) \in \Ker(\widetilde{p_1})\)
and so \(g_2^{-1}g'_2\in N_2\) and thus \(g_2 N_2 = g'_2 N_2\).
This shows
\[\widetilde{p_2}(\left(g_1,g_2\right)) =
  \widetilde{p_2}(\left(g'_1,g'_2\right))\]
and the graph defines the mapping of (\ref{eq:g1n1g2n2}) we had to show.
Similarly, we can show the inverse mapping
\[G_2/N_2 \rightarrow G_1/N_1\]
and thus the graph is an surjective bijection.

Now to show homomorphism. Let
\(x_1N_1,y_1N_1 \in G_1/N_1\).
For \(i=1,2\)
There must be some
% \((a_i,b_i)\in H\) so
\((a_1,a_2),(b_1,b_2)\in H\) so
\(a_iN = x_iN\) and
\(b_iN = y_iN\).
The mapping (\ref{eq:g1n1g2n2}) we have established has:
\(x_1N_1 \rightarrow a_2N2\) and
\(y_1N_1 \rightarrow b_2N2\).
By looking at
\((a_1 b_1,a_2 b_2)\in H\) we get
\(x_1 y_1 N_1 \rightarrow a_2 b_2 N2\).


%%%%%
\begin{excopy}
Prove that the group of inner automorphisms of a group $G$
is normal in \(Aut(G)\).
\end{excopy}

The \textbf{inner} automorphisms
(\(\Inn(G)\), See: \cite{Scott87})
are the conjunctions.

Let \(T\in \Aut(G)\) and
let \(\gamma_x\in \Inn(G)\).
For \(y\in G\) We have
\(\gamma_x(y) = xyx^{-1}\) and
\begin{eqnarray*}
(T\gamma_x T^{-1})(y)
  & = & T(xT^{-1}(y)x^{-1})\\
  & = & T(x)T(T^{-1}(y))T(x^{-1})\\
  & = & T(x)yT(x^{-1})\\
  & = & \gamma_{T(x)}(y).\\
\end{eqnarray*}
So \(T\gamma_x T^{-1} \in \Inn(G)\) and \(\Inn(G)\) is normal.

%%%%%
\begin{excopy}
Let $G$ be a group such that \(\Aut(G)\) is cyclic.
Prove that $G$ is abelian.
\end{excopy}

Let \(T\in\Aut(G)\) be a generator.
For all \(g\in G\) we denote the inner mapping \(\gamma_g(x)=gxg^{-1}\).
So for any \(g\in G\) there exists a minimal \(n(g)\geq 0\) such that
\(T^{n(g)} = \gamma_g\). G is abelian iff \(n(g)=0\) for all \(g\in G\).
In negation we can assume there is some \(t \in G\)
with minimal \(m=n(t)>0\). We will show that \(\gamma_t\)
generates all of \(\Inn(G)\).

Let \(g\in G\), \(d=\lfloor n(g)/m\rfloor\) and the residue
\(r = n(g) - md\) where \(0\leq r<m\).
% Now \(\gamma_g = T^{n(g)} = T^r(T^m)^{d} = T^r \gamma_t^d\).
Set \(g'=gt^{-d}\) and we get
\[\gamma_{g'}(x) = (T^{n(g)}(T^m)^d)(x) = T^{n(g)-md}(x) = T^r(x).\]
By minimality of $m$ we must have \(r=0\).

So for all \(g\in G\) there exists a \(d\geq0\) such that
\(T^{md}=\gamma_g\) that is for all \(x\in G\)
\(t^{md}xt^{-md} = gxg^{-1}\). Substituting $x$ with $t$ we get
\(t = t^{md}tt^{-md} = gtg^{-1}\) from which we get \(gt=tg\)
and \(T^m\) is the identity that generates \(\Inn(G)\).
Since \(\Inn(G)=\{\Id_G\}\) we conclude that $G$ is abelian.



%%%%%
\begin{excopy}
Let $G$ be a group and let \(H, H'\) be subgroups.
By a \textbf{double coset} of \(H, H'\) one means
a subset of $G$ of the form \(HxH'\).
\begin{itemize}
  \item[(a)] Show that $G$ is a disjoint union of double cosets.
  \item[(b)] Let $C$ be a family of representatives for
     double cosets. For each \(a \in G\) denote by \([a]H'\)
     the conjugate \(aH'a^{-1}\) of \(H'\).
     For each \(c\in C\) we have a decomposition into ordinary cosets
  \begin{equation} \label{eq:decoxcHHp}
    H = \Disjunion_{x\in X_c} x(H\cap[c]H'),
    % H = \Disjunion_{x\in X_c} x(H\cap[c]H'),
  \end{equation}
     where \(X_c\) is a family of elements of $H$, depending on $c$.
     Show that the elements
     \(\{xc: c\in C,\,x\in X_c\}\) form a family of left cosets
     representatives for \(H'\) in $G$; that is,
  \begin{equation} \label{eq:decoGxccHp}
    G = \Disjunion_{c\in C}\,\Disjunion_{x\in X_c} xcH',
  \end{equation}

  \begin{quote}
   \textbf{Note:} In the original text,
   equation (\ref{eq:decoxcHHp})  appears as
   \[ H = \bigcup_c x_c(H\cap[c]H'),\]
   and equation (\ref{eq:decoGxccHp}) appears as
   \[ G = \bigcup_{x_c}\,\bigcup_{x_c} x_c cH'.\]
  I believe these are notational mistakes
   or unclear indices style.
  \end{quote}

\end{itemize}
\end{excopy}

To prove (a) let us assume that for some \(x_1,x_2\in G\)
\((Hx_1H')\cap(Hx_2H')\neq\emptyset\).
So we have \(h_1,h_2\in H\) and \(h'_1,h'_2\in H'\) so
\(h_1 x_1 h'_1 = h_2 x_2 h'_2\).
So we
\(x_1 = ({h_1}^{-1}h_2) x_2 (h'_2{h'_1}^{-1}).\)
Now for any \(g\in Hx_1H'\) there are
\(h\in H,h'\in H'\) such that we have
\[g = hx_1h' = (h{h_1}^{-1}h_2) x_2 (h'_2{h'_1}^{-1}) \in Hx_2H'.\]
Thus \(Hx_1H'\subset Hx_2H'\). Similarly,
\(Hx_2H'\subset Hx_1H'\) and they are equal and so the double cosets
are disjoint.

Now we turn to (b). For each \(c\in C\)
\(H_c = H\cap(cH'c^{-1})\subseteq H\). So we have
cosets of \(H_c\) form a disjoint union
\[H = \disjunion_{x\in X_c} xH_c = \disjunion_{x\in X_c} xcH'c{-1}\]
with some family \(X_c\) of representatives.

We compute
\begin{eqnarray} \label{eq:hch}
 HcH' & = & \left(\Disjunion_{x\in X_c} xcH'c^{-1}\right)cH' \\
      & = & \left(\Disjunion_{x\in X_c} xcH'c^{-1}c\right)H' \nonumber \\
      & = & \left(\Disjunion_{x\in X_c} xcH'\right)H' \nonumber\\
      & = & \Disjunion_{x\in X_c} xcH'H' \nonumber\\
      & = & \Disjunion_{x\in X_c} xcH'  \nonumber
\end{eqnarray}

From (a) we have a family $C$ of representatives of the double cosets
and using \ref{eq:hch} for substitution we get:
\[
G = \Disjunion_{c\in C} HcH'\\
  = \Disjunion_{c\in C} \, \Disjunion_{x\in X_c} xcH'.\]



%%%%%
\begin{excopy}
\begin{itemize}
 \item[(a)] Let $G$ be a group and $H$ a subgroup of finite index.
   Show that there exists a normal subgroup $N$ of $G$ contained in $H$
   and also of a finite index. [\emph{Hint}: If \((G:H)=n\),
   find a homomorphism if $G$ into \(S_n\) whose kernel is contained in $H$.]
 \item[(b)] Let $G$ be a group and let \(H_1\), \(H_2\) be subgroups of
   finite index. Prove that \(H_1\cap H_2\) has finite index.
\end{itemize}
\end{excopy}

Let $G$ act on the set of cosets of $H$ with \(\pi_g(cH) = gcH\)
for every \(g\in G\) and ever coset \(cH\) with \(c\in G\).
We want to show that this definition is independent of choice of
the $c$ representative.
So let \(c_1H = c_2H\), % so there must be some \(h\in H\)
and by associativity of the group multiplication
\[\pi_g(c_1H) = g(c_1H) = g(c_2H) = \pi_g(c_2)\]
and for each \(g_1,g_2\in G\), \(c\in G\)
\begin{eqnarray} \label{eq:pig1g2}
\pi_{g_1g_2}(cH) = (g_1g_2)(cH) = g_1(g_2(cH)) =  g_1(\pi_{g_2}cH) \\
  \hfill = \pi_{g_1}(\pi_{g_2}cH) = (\pi_{g_1}\pi_{g_2})(cH). \nonumber
\end{eqnarray}
The set of $H$ has $n$ elements.
By labeling the cosets
and from (\ref{eq:pig1g2})
we see that the association \(g\rightarrow \pi_g\)
is homomorphism\(T:G\rightarrow S_n\).
We have \(\Ker T \subseteq H\) since
for every \(g\in G\setminus H\) we have \(gH\neq H\).
Obviously, \(G/\Ker(T) \equiv T(G) \subseteq S_n\) and so
\[[G:H] \leq [G:\Ker(T)] = |T(G)| \leq |S_n| = n! < \infty.\]

To prove (b) we can assume that \(H_1\), \(H_2\) are normal.
Otherwise we simply use (a) and substitute them with normal subgroups.
Now we can use Example~(iv) of \S~3 (page~17)
\begin{equation}
H_1/(H_1\cap H_2) = H_1 H_2 / H_2
\end{equation}
that gives
\begin{equation}
[H_1:(H_1\cap H_2)] =
|G/(H_1\cap H_2)| =
|H_1 H_2 / H_2| \leq
|G/H_2|
\end{equation}
and so
\begin{equation}
[G:(H_1\cap H_2)] =
[G:H_1]\cdot[H_1:(H_1\cap H_2)] \leq
|G/H_1|\cdot|G/H_2| < \infty.
\end{equation}



%%%%%
\begin{excopy}
Let $G$ be a group and let $H$ be a subgroup of a finite index.
Prove that there is only a finite number of right cosets of $H$, and that
the number of right cosets is equal to the number of left cosets.
\end{excopy}

We will show a one to one surjective mapping between
the left cosets to the right cosets. It is defined by
\begin{equation} \label{eq:xH2Hx}
xH \rightarrow Hx^{-1} \qquad\textrm{for\ } x\in G
\end{equation}

The map is obviously surjective since \(x\rightarrow x^{-1}\) is.
Now assume \(Hx^{-1} = Hy^{-1}\). Then  \(x^{-1}y\in H\)
and so \((x^{-1}y)^{-1} = y^{-1}x \in H\) and so \(xH = yH\)
and so (\ref{eq:xH2Hx}) is one to one.

%%%%%
\begin{excopy}
Let $G$ be a group, and $A$ a normal abelian subgroup.
Show that \(G/A\) operates on $A$ by conjunction,
and in this manner get a homomorphism of \(G/A\) into \(\Aut(A)\).
\end{excopy}

Let \(xA\in G/A\) operate as \(a\mapsto xax^{-1}\) for all \(a\in A\).
To show that this operation as well defined, assume \(xA=yA\).
So \(x^{-1}y, y^{-1}x\in A\) and using the fact that $A$ is abelian, we get
\begin{equation}
xax^{-1} =
xax^{-1}(xy^{-1})^{-1}(xy^{-1}) =
(xy^{-1})^{-1}xax^{-1}(xy^{-1}) =
yay^{-1}.
\end{equation}
Now since \(y(xax^{-1}y^{-1} = (yx)a(yx)^{-1}\) the homomorphism follows.

\end{myenumerate}
\textbf{Semidirect product}
\begin{myenumerate}


%%%%%
\begin{excopy}
Let $G$ be a group and let $H$, $N$ be subgroups with $N$ normal.
Let \(\gamma_x\) be conjunction by an element \(x\in G\).
\begin{itemize}
 \item[(a)] Show that \(x\rightarrow \gamma_x\) induces
    a homomorphism \(f:\,H\mapsto\Aut(N)\).
 \item[(b)] If \(H\cap N = \eG\), show that the map
    \(H \times N \rightarrow HN\) given by
    \((x,y) \mapsto xy\) is a bijection, and that this map
    is an isomorphism if and only if $f$ is trivial,
    i.e. \(f(x) = \id_N\) for all \(x\in H\).
\end{itemize}
We define $G$ to be the \textbf{semidirect product} of $H$ and $N$
if \(G=NH\) and \(H\cap N = \eG\).
\begin{itemize}
 \item[(c)] Conversely, let $H$, $N$ be groups,and let
   \(\psi:\,H\mapsto \Aut(N)\) be a given homomorphism.
  Construct a semidirect product as follows.
  Let $G$ be the set of pairs \((x,h)\) with \(x\in N\) and \(h\in H\).
  Define the composition law
  \begin{equation}
    (x_1,h_1)(x_2,h_2) = (x_1x_2^{\psi(h_1x_2)}, h_1h_2),
  \end{equation}
  Show that this is a group law, and yields a semidirect product of $N$ and $H$,
  identifying
       $N$ with the set of elements \((x,1)\)
   and $H$ with the set of elements \((1,h)\).
\end{itemize}
\end{excopy}

\begin{itemize}

 \item[(a)] Let \(x,y\in H\). For any \(u\in N\)
 \[\gamma_{xy}(u) = xyu(xy)^{-1} = x\gamma_y(u)x^{-1} = \gamma_x(\gamma_y(u)).\]
 \item[(b)]
    Assume \(x_1y_1 = x_2y_2\) where \((x_1,y_1),\,(x_2,y_2)\in H\times N\).
   Multiplying both sides with
    \(x_1^{-1}\) from the left and
    \(y_2^{-1}\) from the right, we get
     \[x_2^{-1}x_1 = y_2y_1^{-1} \in H\cap N = \eG\]
   and thus \(x_1 = x_2\)
   and  \(y_1 = y_2\).
 \item[(c)]
    Seems that the problem isnot well formed.


\end{itemize}

%%%%%
\begin{excopy}
\begin{itemize}
 \item[(a)]
    Let $H$, $N$, be normal subgroups of a finite group $G$.
    Assume that the orders  of $H$ and $N$ are relatively prime.
    Prove that \(xy=yx\) for all \(x\in H\) and \((y\in N\),
    and that \(H\times N=HN\).
 \item[(b)]
    Let \(H_1,\ldots\,H_r\) be normal subgroups of $G$ such that the order
    of \(H_i\) is relatively prime to the order of \(H_j\) for \(i\neq j\).
    Prove that
    \begin{equation}
      H_1\times \ldots \times H_r = H_1\cdots H_r.
    \end{equation}
\end{itemize}
\end{excopy}

\begin{itemize}
 \item[(a)]
    We look at \(N\cap H\) since its order must divide both that
    of $H$ and $N$ we have \(|N\cap H| = 1\) and \(N\cap H = \eG\).
    Now
   \[ xyx^{-1}y^{-1} = (xyx^{-1})y^{-1} = x(yx^{-1}y^{-1}) \in N\cap H.\]
   and thus \(xyx^{-1}y^{-1} = e\) and we get \(xy=yx\).
 \item[(b)]
    Trivial by induction on $r$.
\end{itemize}


%%%%%
\begin{excopy}
Let $G$ be a finite group and let $N$ be a normal subgroup such that
$N$ and \(G/N\) have relatively prime orders.
 \begin{itemize}
   \item[(a)]
      Let $H$ be a subgroup of $G$ having the same order as \(G/N\).
      Prove that \(G = HN\).
   \item[(b)]
      Let $G$ be an automorphism of $G$. Prove that \(g(N) = N\).
 \end{itemize}
\end{excopy}

\begin{itemize}
 \item[(a)]
    The subgroup \(H\cap N\) has on order that must divide
    both $H$ and $N$ and therefore is $1$ and so the subgroup is trivial.
    Now assume \(h_1g_1 = h_2g_2\)
    for \(h_i\in H\), \(g_i\in N\), \(i=1,2\).
    Then \(h_2^{-1}h_1 = g_2g_1^{-1} \in H\cap N\) and
    \(h_1 = h_2\) and \(g_1 = g_2\). Thus counting the elements
    \(|HN| = |H|\cdot|N| = |G|\) and thus \(HN=G\).
 \item[(b)]
    Let \(h: N \rightarrow G/N\) be defined by \(h(x)=g(x)+N\).
    We will show the $h$ must be trivial.
    Now \(H=h(N)\) is a subgroup of \(G/N\) and \(|H|\)
    divides both \(|N|\) and \(G/N\) and therefore \(H=\eG\)
    which means that \(g(N)\subseteq N\). Since $g$ is automorphism
    we have \(g(N) = N\).
\end{itemize}

%%%%%
\begin{excopy}
\label{gop:nofix}
Let $G$ be a finite group operating on a finite set $S$ with \(\#(S)\geq2\).
Assume that there is only one orbit. Prove that there exist an element
\(x\in G\) which has no fixed point,i.e. \(xs\neq s\) for all \(s\in S\).
\end{excopy}

Solution using \cite{Scott87}~10.1.5.

Define \(\Ch(g) = |\{s\in S: gs=s\}|\)  \label{def:Ch}.
\begin{llem}
If $G$ acts on a finite set $S$ has $N$ orbits, then
\begin{equation}
\sum_{g\in G} \Ch(g) = n\cdot |G|.
\end{equation}
\end{llem}

Let \(T\subseteq S\) be an orbit of $G$ and \(a,b\in T\).
It is clear that \(|G_a| = |G_b| = |G|/|T|\).

We may view \(g\in G\) as permutations of $S$.
Let \(F = \{(s,g)\in S\times G: gs=s\}\) be the set of fixed points.
Now
\begin{eqnarray}
\sum_{g\in G} \Ch(g) & = & |F|\\
 & = & \sum_{s\in S} |G_s| \\
 & = & \sum_{T\ \textrm{orbit}} \sum_{s\in T} |G_s| \\
 & = & \sum_{T\ \textrm{orbit}} |T|\cdot|G|\\
 & = & n\cdot|G|.
\end{eqnarray}

Now back to the exercise, if \(n=1\)
\index{transitive}
($G$ is \emph{transitive})
and $G$ is finite then
\begin{eqnarray}
\sum_{g\in G} \Ch(g) = |G|.
\end{eqnarray}
Now assume by negation that for all \(g\in G\) \(\Ch(g)\geq 1\)
(at least one fixed point), then
\begin{eqnarray*}
|G| & = & \sum_{g\in G} \Ch(g) \\
    & = & \Ch(e) + \sum_{g\in G\setminus\eG} \Ch(g) \\
    & = & |S| + \sum_{g\in G\setminus\eG} \Ch(g) \\
    & \geq & |S| + |G| - 1
\end{eqnarray*}
Hence, \(|S| \leq 1\) which contradicts the assumption.

%%%%%
\begin{excopy}
Let $H$ be a proper subgroup of a finite group $G$. Show that $G$
is not the union of all the conjugates of $H$.
\end{excopy}

We look at $G$ as a group operating on the finite sets of the conjugates of $H$.
From the previous Exercise~\ref{gop:nofix}, there must be some \(x\in G\)
for which \(x(gHg^{-1})x^{-1} \neq gHg^{-1}\) for all \(g\in G\).
That is \(x\notin gHg^{-1}\) for all \(g\in G\) and
\[x \notin \bigcup_{g\in G}gHg^{-1}.\]

%%%%%
\begin{excopy}
Let $X$,$Y$ be finite sets and let $C$ be a subset of \(X\times Y\).
For \(x\in X\) let \(\phi(x)=\) number of elements \(y\in Y\) such that
\((x,y)\in C\). Verify that \[\#(C) = \sum_{x\in X}\varphi(x).\]

\emph{Remark}. A subset $V$ as in the above exercise is often called
\index{correspondence}
a \textbf{correspondence}, and \(\varphi(x)\) is the number of elements in $Y$
which correspond to a given element \(x\in X\).
\end{excopy}

This is simple result of looking at the disjoint union:
\[C = \Disjunion_{x\in X} \{(x,y)\in X\times Y: (x,y)\in C\}.\]

%%%%%
\begin{excopy}
Let $S$, $T$ be finite sets. Show that \(\#\Map(S,T) = (\#T)^{\#(S)}\).
\end{excopy}

Simple induction on \(\#(S)\).
If \(S'=S\cup\{x\}\) then we can extend each map \(S\rightarrow T\)
to \(S'\) by assigning \(\#(T)\) different values to $x$.

%%%%%
\begin{excopy}
Let $G$ be a finite group operating on a finite set $S$.
 \begin{itemize}
  \item[(a)]
    For each \(s\in S\) show that \[\sum_{t \in Gs} {\frac{1}{\#(Gt)}} = 1.\]
  \item[(b)]
    For each \(x \in G\) define \(f(x)=\) number of element \(s\in S\)
    such that \(xs=s\). Prove that the number of orbits of $G$ in $S$
    is equal to
      \[\frac{1}{\#(G)}\sum_{x\in G} f(x).\]
 \end{itemize}
\end{excopy}

\begin{itemize}
 \item[(a)]
   For all \(t \in Gs\) we have \(|Gs|=|Gt|\) and so
   \begin{equation}
   \sum_{t \in Gs} {\frac{1}{\#(Gt)}} =
   |Gs|\cdot{\frac{1}{|Gs|}} = 1,
   \end{equation}
 \item[(b)]
  Let us compute the number of ``fixed occurrences''.
  \begin{eqnarray}
   \sum_{x\in G} f(x)
     & = & \sum_{x\in G} \Ch(x)                 \label{eq:f2Ch} \\
     & = & \sum_{s\in S} |\{g\in G: gs = s\}|    \label{eq:Ch2gss} \\
     & = & \sum_{s\in S} |G_s|                   \label{eq:gss2Gs} \\
     & = & \sum_{\textrm{orbit\ } T\subseteq S}
             \sum_{t\in T} |G_t|                 \label{eq:Gsrob} \\
     & = & \sum_{\textrm{orbit\ } T\subseteq S}
             \sum_{t\in T} |G|/|Gt|              \label{eq:GfsGGs} \\
     & = & |G|\cdot\sum_{\textrm{orbit\ } T\subseteq S}
             \sum_{t\in T} 1/|Gt|                \label{eq:1overGs} \\
     & = & |G|\cdot\sum_{\overset{s\in S}{\textrm{orbits repr.}}}
             \sum_{t\in Gs} 1/|Gt|                \label{eq:orbrep} \\
     & = & |G|\cdot\sum_{\overset{s\in S}{\textrm{orbits repr.}}} 1.
                                                  \label{eq:orb1}
  \end{eqnarray}

  Equalities explanation:
  \begin{itemize}
   \item[(\ref{eq:f2Ch})] --- simply using the definition
                              in Exercise~\ref{def:Ch}.
   \item[(\ref{eq:Ch2gss})] --- counting fixed points via $S$ instead of $G$.
   \item[(\ref{eq:gss2Gs})] --- definition of \(G_s\).
   \item[(\ref{eq:Gsrob})] --- separating the summation over orbits.
   \item[(\ref{eq:GfsGGs})] --- basic result of group operating on set.
   \item[(\ref{eq:1overGs})] --- factoring \(|G|\) out
   \item[(\ref{eq:orbrep})] ---  Looking at an orbit $T$ via
                                 a representative \(s\in S\).
   \item[(\ref{eq:orb1})] --- Using the previous item (a) of this exercise.
  \end{itemize}

  Now we simply divide both ends of the equation by \(|G|\)
  to get the desired result.

\end{itemize}

\end{myenumerate}

Throughout, $p$ is a prime number.

\iffalse
% Global remark - so fake an item
\item[]
 \setlength{\leftmargin}{0pt}
 \setlength{\labelwidth}{0pt}
 \setlength{\labelwidth}{0pt}
 Throughout, $p$ is a prime number.
% {\nullfont kaka}
\addtocounter{enumi}{-1}
\fi

\begin{myenumerate}
%%%%%
\begin{excopy}
Let $P$ be a $p$-group. Let $A$ be a normal subgroup of order $p$.
Prove that $A$ is contained in the center of $P$.
\end{excopy}

We can view $P$ as operating on $A$ by conjunction.
That is for any \(x\in P\),we have \(\gamma_x(a) = xax^{-1}\).
Let \(x\in P\) and \(a\in A\) be any elements.
Say \(|P|=p^n\)
and so we have \(\gamma_x^{p^n}=\Id_A\).
% Assume \(\gamma_x(a)\neq a\), so \(a\neq e\) for sure.
Note that
\[\underbrace{\gamma_x(\gamma_x(\ldots(\gamma_x(}_{n\ \textrm{times}}a)\ldots))
 =  \gamma_x^n(a).\]
So because of \(A\setminus\eG\)
let \(k>0\) be the minimal such that \(\gamma_x^k(a)=a\).
So we have \(k|p^n\) and \(k\neq p-1\) and so \(k=1\) and \(\gamma_x(a)=a\).
Thus \(xa=ax\) and $a$ is in the center of $P$.


%%%%%
\begin{excopy}
Let $G$ be a finite group and $H$ a subgroup. Let \(P_H\) be
a $P$-Sylow subgroup of $H$. Prove that there exists a $p$-Sylow subgroup $P$
  of $G$ such that \(P_H = P\cap H\).
\end{excopy}

Since any $p$-subgroup is contained in a $p$-Sylow subgroup,
We have a subgroup $P$ such that \(P_H\subseteq P\subseteq G\).
Obviously \(P_H\subseteq P\cap H\).
Now \(P\cap H\) has an order that divides \(|P|\) so it is a power of $p$.
But from the maximality of \(P_H\) the equality follows.

%%%%%
\begin{excopy}
Let $H$ be a normal subgroup of a finite group $G$
and assume that \(\#(H)=p\). Prove that $H$ is contained in every $p$-Sylow
subgroup of $G$
\end{excopy}

We know that $H$ is contained in some $p$-Sylow subgroup S.
All $p$-Sylow subgroups are conjugates. Now for all \(x\in G\)
\[H=xHx^{-1}\subseteq xSx^{-1}.\]


%%%%%
\begin{excopy}
Let $P$, \(P'\) be $p$-Sylow subgroups of a finite group $G$.
\begin{itemize}
 \item[(a)]  If \(P'\subset N(P)\) (normalizer of $P$), then \(P'=P\).
 \item[(b)]  If \(N(P')=N(P)\), then \(P'=P\).
 \item[(c)]  We have \(N(N(P))=N(P)\).
\end{itemize}
\end{excopy}

\begin{itemize}
 \item[(a)] (Following argument in the proof of Theorem~6.4).
     Since \(P'\subseteq N(P)\)
     we have \(P'P\) is a subgroup of \(N(P)\) and $P$ is normal in it.
     Now
     \begin{equation}\label{eq:pppcp}
     (P'P:P) = (P':P'\cap P)
     \end{equation}
     (see~(iv) page~17).
     Now if by negation \(P'\neq P\) then $p$ divides
     the right side of~(\ref{eq:pppcp}) and so \(P'P\) contains
     a~$p$-Sylow subgroup with higher power of $p$ than that of $P$
     contradicting the fact that $P$ itself is a $p$-Sylow subgroup.
 \item[(b)] Immediate from (a) and the fact that \(P'\subseteq N(P')\).
 \item[(c)] By negation, say \(x\in N(N(P))\setminus N(P)\).
   So \(P' = xPx^{-1}\) is a $p$-Sylow subgroup and \(P'\neq P\).
   Now \[P' = xPx^{-1} \subseteq xN(P)x^{-1} = N(P)\]
   and from (a) we get \(P'=P\) a contradiction.
\end{itemize}
%%%%%%%%%%%%%%
\end{myenumerate}

%%%%%%%%%%%%%%%%%%%%%%%%%%%%%%%%%%%%%%%%%%
\textbf{Explicit determination of groups}

Let us have some lemmas.

\begin{llem} \label{llem:npdiv}
Let $G$ be a group or order $m$,
let \(p^r\) be the highest power of~$p$ that divides~$m$
and let~\(n_p\) be the number of $p$-Sylow subgroups.
Then \(n_p \mid m/p^r\).
\end{llem}
%\textbf{Proof:}
\begin{proof}
Immediate result from Proposition~5.2.
\end{proof}

\begin{llem} \label{rose94:p2q}
\textnormal{\small [See \cite{Rose94} Theorem~5.19]}
Let $G$ be a group and \(|G|=p^2q\) where $p$, $q$ are distinct primes,
then $G$ has a normal  Sylow subgroup and so $G$ is not simple.
\end{llem}
\begin{proof}
Indeed, if $G$ has a normal Sylow subgroup $H$ then
\[\eG\subnormal H \subnormal G\]
is an abelian tower by exercise \ref{ex:p2abel}.
This is true for either \(|H|=p^2\) or \(|H|=q\).
Thus $G$ is simple.

Now let's show the existence of a normal Sylow subgroup.

If \(q<p\) then by Lemma~6.7 the Sylow $p$-subgroup is normal.
So we now can assume \(p<q\).
Let \(n_p\) and \(n_q\) be the numbers of
Sylow $p$-subgroup and $q$-subgroups. By negation we assume
\(n_p > 1\) and \(n_q > 1\).
By local-lemma~\ref{llem:npdiv}
\begin{itemize}
 \item
   \(n_p\mid q\) and so \(n_p=q\).
   % Also \(n_q \equiv 1 \bmod\).
 \item
   \(n_q\mid p^2\), hence \(n_q=p\) or \(n_q=p^2\).
   But  if \(n_q=p\) then by \(n_q\equiv 1 \bmod q\) we have \(p>q\)
   contradicting our assumption and so \(n_q=p^2\).
\end{itemize}
Now any two different $q$-subgroups intersect in \eG
and so the number of elements of order $q$ is \(n_q(q-1)\).
The number of the ``non $q$ order'' elements in $G$ is \(p^2q - n_q(q-1)=p^2\).
Now a Sylow $p$-subgroup has an order \(p^2\) and all its elements
have order different than $q$ and so such subgroup is determined
by its \(p^2\) ``non $q$ order'' elements
and thus it is unique and normal.
\end{proof}

\begin{llem} \label{rose94:pqr}
\textnormal{\small [See \cite{Rose94} Theorem~5.20.]}
Let $G$ be a group and \(|G|=pqr\) where $p$, $q$, $r$ are distinct primes,
then $G$ is not simple.
\end{llem}
\begin{proof}
Let \(n_p\),\(n_q\) and \(n_r\) be the respective
numbers of Sylow subgroups.
Assume by negation that these three numbers are \(\>1\)
since otherwise
we have a normal Sylow subgroup and we are done.
Assume \(p>q>r\)
So any two distinct Sylow subgroups intersect in~\eG.
So the numbers of elements in $G$ of order $p$, $q$ and $r$
are
\(n_p(p-1)\), \(n_q(q-1)\) and \(n_r(r-1)\) respectively.
Therefore
\begin{equation}
|G|=pqr\geq 1 + n_p(p-1) + n_q(q-1) + n_r(r-1).
\end{equation}
By Sylow theorem, \(n_p \mid qr\) and \(n_p\equiv 1\bmod p\).
Since \(n_p>q\) and \(p>q\), \(p>r\), it follows that \(n_p=qr\).

Also \(n_q\mid pr\) and \(n_q\equiv 1\bmod q\)..
Since \(n_q>1\) and \(q>r\), it follows that \(n_q\geq p\).

Finally, \(n_r\mid pq\) so \(n_r\geq q\). Now we have
\begin{equation}
pqr \geq 1 + qr(p-1) + p(q-1) + q(r-1) = pqr + pq + qr - p - q  + 1,
\end{equation}
and hence \((p-1)(q-1)\leq 0\) which is impossible.
\end{proof}

\textbf{Definition:}
\textnormal{\small [See \cite{Rose94}~Exercise~90]}
Let $H$ be a subgroup of $G$.
\index{core!of group}
\index{normal interior}
Define the \emph{core} or \emph{normal interior} of $H$ in $G$ as
\begin{equation}
H_G = \bigcap_{g\in G} g^{-1}Hg
\end{equation}

It is clear that \(H_G\) is the largest normal subgroup of $G$ that
is contained in $H$.

\begin{llem} \label{llem:GsCoreSn}
\textnormal{\small [See \cite{Rose94}~Theorem~4.13]}
if $H$ is a subgroup of $G$ of finite index \(n=[G:H]\) then \(G/H_G\)
can be embedded in \(S_n\).
\end{llem}
\begin{proof}
Let \(\hat{H}\) be the set of $n$ left cosets of $H$ in $G$.
Let $G$ operate on this set by left multiplication.
Each \(g\in G\) permutates \(\hat{H}\).
By enumerating the cosets we identify \(S_n\)
with the permutations of \(\hat{H}\) and we have a mapping
\(\rho: G \rightarrow S_n\).
The kernel of \(\rho\) is the elements \(g\in G\)
for which each coset is fixed.

Let's first compute the \index{stabilizer} \index{Stab}
of a coset \(xH\)
\begin{eqnarray}
\Stab_G(xH)
  & = & \{g\in G: gxH=xH\} \\
  & = & \{g\in G: x^{-1}gxH=H\} \\
  & = & \{g\in G: x^{-1}gx \in H\} \\
  & = & \{g\in G: g \in xHx^{-1}\} \\
  & = & xHx^{-1}
\end{eqnarray}
We compute:
\begin{eqnarray}
\Ker\rho
  & = & \bigcap_{xH\in\hat{H}} \Stab_G(xH) \\
  & = & \bigcap_{g\in G} \Stab_G(gH) \\
  & = & \bigcap_{g\in G} gHg^{-1} \\
  & = & H_G. \\
\end{eqnarray}
And so \(G/H_G\approx \rho(G)\) a subgroup of \(S_n\).
\end{proof}


\begin{llem} \label{llem:pmr:divfac}
\textnormal{\small [See \cite{Rose94}~Exercise~279]}
Let $G$ be a simple group of order \(p^m r\) where \(p\nmid r\).
Then \(p^m \mid (r-1)!\).
\end{llem}
\begin{proof}
Let $H$ be a $p$-Sylow subgroup if $G$. Since $G$ is simple \(H_G = \eG\)
and from local-lemma~\ref{llem:GsCoreSn} $G$ can be embedded in \(S_{[G:H]}\).
Hence \(p^m r \mid r!\) and therefore \(p^m \mid (r-1)!\).
\end{proof}

\begin{myenumerate}

%%%%%
\begin{excopy}
Let $p$ be a prime number. Show that a group of order \(p^2\)
is abelian, and that there are only two such groups up to isomorphism.
\end{excopy}  \label{ex:p2abel}

Let $G$ be the group and $Z$ its center and \(Z\subnormal G\).
If \(G=Z\) then clearly $G$ is abelian.
Assume by negation that \(Z\subsetneq G\).
Since $Z$ is not trivial by Theorem~6.5 it must be of order $p$
and so \(G/Z\) has an order of $p$ as well and is cyclic generated by \(a+Z\).
Now let  \(a_1,a_2 \in G\) be any elements in $G$. For \(i=1,2\) we can
have the representation \(a_i=a^{n_i}g_i\) where \(n_i\neq 0\) and \(g_i\in Z\).
Now
\begin{equation} \label{eq:a1a2}
a_1a_2 = a^{n_1}g_1 a^{n_2}g_2 =
  a^{n_1+n_2}g_1g_2 =
  a^{n_2}a^{n_1}g_2g_1 =
  a^{n_2}g_2a^{n_1}g_1 = a_2a_1.
\end{equation}
Thus $G$ is abelian.

Now by Theorem~8.2, $G$ is isomorphic to a product of cyclic $p$-group.
Hence, isomorphic to\, \(\Zm{p^2}\) \, or \, \(\Zm{p}\times\Zm{p}\).


%%%%%
\begin{excopy}
Let $G$ be a group of order \(p^3\), where $p$ is prime, and $G$ is not abelian.
Let $Z$ be its center. Let $C$ be a cyclic group of order $p$.
\begin{enumerate}[(a)]
\item Show that \(Z \approx C\) and \(G/Z \approx C \times C\).
\item) Every subgroup of $G$ of order \(p^2\) contains $Z$ and is normal.
\item Suppose \(x^p = 1\) for all \(x \in G\),
 Show that $G$ contains a normal subgroup \hbox{\(H \approx C \times C\)}.
\end{enumerate}
\end{excopy}

\begin{itemize}
 \item[(a)]
     The $Z$ subgroup cannot be the whole $G$ since $G$ is not abelian.
     It cannot be trivial because of Theorem~6.5. That leaves the possibilities
     for its order to be $p$ or \(p^2\). If by negation the order is \(p^2\),
     then \(G/Z\) is cyclic and as in exercise~\ref{ex:p2abel}
     similar arguments like in (\ref{eq:a1a2}) gives a contradiction
     by showing that $G$ is abelian. Thus $Z$ is cyclic of order $p$
     and isomorphic to $C$ and \(G/Z\) is of order \(p^2\).

     Now from the previous exercise we know that groups of order \(p^2\)
     must be isomorphic to either \(\Zm{p}\times\Zm{p}\)
     or to the cyclic
     \(\Zm{p^2}\). The latter leads to contradiction that $g$ is abelian
     using the same arguments with \(G/Z\) cyclic.
     Thus \(G/Z\) is isomorphic to \(\Zm{p}\times\Zm{p}\) which is
     isomorphic to \(C\times C\).

 \item[(b)]
     Say $H$ is a subgroup of order \(p^2\). By Lemma~6.7 $H$ is normal.
     The subgroup \(H\cap Z\)
     could be or order $p$ or $1$.
     If by negation it is the latter case, then \(H\cap Z = \eG\).
     To show that \(HZ = \{hc: h\in H \, \textrm{and} \, c\in Z\}\)
     has exactly \(|H|\cdot|Z|=p^3\) we will show that the products
     differ. If
     \(h_1 c_1 = h_2 c_2\) with \(h_i\in H\), \(c_i\in Z\) we get
     \(c_1c_2^{-1} = h_1^{-1}h_2 \in H\cap Z\) and so this product equals $e$
     and \(h_1=h_2\), \(c_1=c_2\). Hence \(HZ=G\) and we can
     represent any \(a_1,a_2\in G\) by \(a_i=h_i c_i\) where
      \(h_i\in H\), \(c_i\in Z\) for \(i=1,2\).
      From exercise~\ref{ex:p2abel} $H$ is abelian and so
      \begin{equation}
      a_1a_2 = h_1 c_1 h_2 c_2 = h_1 h_2  c_1 c_2 =
          h_2 h_1  c_2 c_1 = h_2 c_2 h_1 c_1 = a_2a_1
      \end{equation}
      contradicting the fact that $G$ is abelian.
      Thus  \(H\cap Z\) is of order $p$ and $H$ must contain $Z$.

 \item[(c)]
      Examining the proof of Corollary~6.6 we see that
      in the sequence
      \[\eG=G_0 \subset G_1 \cdots \subset G_n = G\]
      every $p$-group $G$ has, the subgroup \(G_1\) is in the center.
      So in our case \(G_1 = Z\) and  let $H$ be \(G_2\)
      that has order of \(p^2\)
      and again by Lemma~6.7 is normal.
      Now $H$ cannot be cyclic, since if it were then
      its generator $x$ would not satisfy the required
      \(x^p=1\) equation.
      Now by exercise~\ref{ex:p2abel}  $H$
      must be isomorphic to \(C\times C\) and not to the cyclic \(\Zm{p^2}\).
\end{itemize}

%%%%%
\begin{excopy}
\begin{itemize}
 \item[(a)] Let $G$ be a group of order \(pq\), where $p$, $q$ are primes
            and \(p<q\). Assume that \(q\not\equiv 1 \bmod p\).
            Prove that  $G$ is cyclic.
 \item[(b)] Show that every group of order \(15\) is cyclic.
\end{itemize}
\end{excopy}  \label{ex:GpLTq}

\begin{itemize}
 \item[(a)]
   [Similar to the example on page~36 with $G$ of \(35\)].
   Let \(H_p\) and \(H_q\) be a $p$-Sylow and $q$-Sylow subgroups respectively.
   Then \(H_q\) is normal by Lemma~6.7.
   Now \(H_p\) operates by conjunction on \(H_q\) and we have
   a homomorphism \(H_p\rightarrow \Aut(H_q) \approx \Zm{(q-1)}\).
   So the image order must divide $p$ and \(q-1\).
   Since \(q-1\not\equiv 0 \bmod p\) clearly \(p\nmid q-1\)
   and so the image is trivial and so elements of \(H_p\) and \(H_q\)
   commutes with each other.

   We will show that \(H_pH_q = G\).
   The set \(H_{pq}=\{x_p^m x_q^n: 0\leq m<p, 0\leq n<q\}\)
   contains \(pq\) elements
   since if \(x_p^{m_1} x_q^{n_1} = x_p^{m_2} x_q^{n_2}\)
   we use the commutativity and the fact that \(H_p\cap H_q=\eG\)
   to get \(x_p^{m_1-m_2} = x_q^{n_2-n_1} = e\) and so \(H_pq=G\)
   and $G$ is abelian. By Proposition 4.3(\textbf{v}) $G$ is cyclic.

   Let \(x_p\) and \(x_q\) be generators of \(H_p\) and \(H_q\) respectively.
   % Then these generators commutes with each other and th
 \item[(b)] By (a) with \(p=3\), \(q=5\) and we have
      \(5\equiv 2\not\equiv 1 \bmod 3\).

\end{itemize}

%%%%%
\begin{excopy}
Show that every group of order \(<60\) is solvable.
We use the results from Corollary~6.6 and
exercises \ref{ex:GpLTq}, \ref{ex:p2q} and \ref{ex:2pq}.
\end{excopy}

{
% \begin{multicols}{2}

\tablefirsthead{\hline \(|G|\)   &   $=$ & $p$   &   $q$ & $r$ \\ \hline}
\tablehead{\hline \multicolumn{5}{|c|}{\small\textsl{continuation}} \\ \hline}
\tabletail{\hline \multicolumn{5}{|c|}{\small\textsl{to be continued}}\\ \hline}
\tablelasttail{\hline}
\begin{supertabular}{|r|c|r|r|r|}
 1 & \multicolumn{4}{|l|}{Trivial}  \\ \hline
 2 & $p$       & $2$  &      &   \\ \hline
 3 & $p$       & $3$  &      &   \\ \hline
 4 & \(p^2\)   & $3$  &      &   \\ \hline
 5 & $p$       & $5$  &      &   \\ \hline
 6 & $pq$      & $2$  & $3$  &   \\ \hline
 7 & $p$       & $7$  &      &   \\ \hline
 8 & \(p^n\)   & $2$  &      &   \\ \hline
 9 & \(p^n\)   & $3$  &      &   \\ \hline
10 & $pq$      & $2$  & $5$  &   \\ \hline
11 & $p$       & $11$ &      &   \\ \hline
12 & \(p^2q\)  & $2$  & $3$  &   \\ \hline
13 & $p$       & $13$ &      &   \\ \hline
14 & $pq$      & $2$  & $7$  &   \\ \hline
15 & $pq$      & $3$  & $5$  &   \\ \hline
16 & \(p^n\)   & $2$  &      &   \\ \hline
17 & $p$       & $17$ &      &   \\ \hline
18 & $p^2q$    & $3$  & $2$  &   \\ \hline
19 & $p$       & $19$ &      &   \\ \hline
20 & $p^2q$    & $2$  & $5$  &   \\ \hline
21 & $pq$      & $3$  & $7$  &   \\ \hline
22 & $pq$      & $2$  & $11$ &   \\ \hline
23 & $p$       & $23$ &      &   \\ \hline
% 24 &           &      &      &   \\ \hline
\hline
25 & \(p^n\)   & $5$  &      &   \\ \hline
26 & $pq$      & $2$  & $13$ &   \\ \hline
27 & \(p^n\)   & $3$  &      &   \\ \hline
28 & \(p^2q\)  & $2$  & $7$  &   \\ \hline
29 & $p$       & $29$ &      &   \\ \hline
30 & \(pqr\)   & $2$  & $3$  & $5$  \\ \hline
31 & $p$       & $31$ &      &   \\ \hline
32 & \(p^n\)   & $2$  &      &   \\ \hline
33 & $pq$      & $3$  & $11$ &   \\ \hline
34 & $pq$      & $2$  & $17$ &   \\ \hline
35 & $pq$      & $5$  & $7$  &   \\ \hline
\hline
37 & $p$       & $37$ &      &   \\ \hline
38 & $pq$      & $2$  & $19$ &   \\ \hline
39 & $pq$      & $3$  & $13$ &   \\ \hline
41 & $p$       & $41$ &      &   \\ \hline
42 & \(pqr\)   & $2$  & $3$  & $7$  \\ \hline
43 & $p$       & $43$ &      &   \\ \hline
44 & \(p^2q\)  & $2$  & $11$ &   \\ \hline
45 & \(p^2q\)  & $3$  & $5$  &   \\ \hline
46 & $pq$      & $2$  & $23$ &   \\ \hline
47 & $p$       & $47$ &      &   \\ \hline
\hline
49 & \(p^n\)   & $7$  &      &   \\ \hline
50 & \(p^2q\)  & $5$  & $2$  &   \\ \hline
51 & $p$       & $51$ &      &   \\ \hline
52 & \(p^2q\)  & $2$  & $13$ &   \\ \hline
53 & $p$       & $53$ &      &   \\ \hline
\hline
55 & $pq$      & $5$  & $11$ &   \\ \hline
\hline
57 & $pq$      & $3$  & $19$ &   \\ \hline
58 & $pq$      & $2$  & $29$ &   \\ \hline
59 & $p$       & $59$ &      &   \\ \hline
\end{supertabular}
% \end{multicols}
}

Now we need to solve some cases specifically.
Let $G$ be a group. For most orders upto $60$ solvability was shown
in the above table. In each of the following remaining cases
we will show the existence of some proper normal subgroup $H$.
Because of  results for lower orders of $G$,
A sequence
\(\eG\subnormal H \subnormal G\) can be completed to an abelian tower.

\begin{itemize}
 \item Assume \(|G|=24=2^3\cdot3\).\\
    From local-lemma \ref{llem:GsCoreSn} $G$ is not simple since otherwise
    \(2^3\mid(3-1)!\).
 \item Assume \(|G|=36=2^2\cdot3^2\).
    From local-lemma \ref{llem:GsCoreSn} $G$ is not simple since otherwise
    \(3^2\mid(4-1)!\).
 \item Assume \(|G|=40=2^3\cdot5\).
    Exercise~\ref{ex:G40G12} shows that $G$ is not simple.
 \item Assume \(|G|=48=2^4\cdot3\).
    From local-lemma \ref{llem:GsCoreSn} $G$ is not simple since otherwise
    \(2^4\mid (3-1)!\).
 \item Assume \(|G|=54=2\cdot3^3\).
    From Lemma~6.7 \(H_3\) is normal and $G$ is not simple
 \item Assume \(|G|=56=2^3\cdot7\).
    Let \(n_p\) be the number of $p$-Sylow subgroups for \(p=2,7\).
    Now \(n_7\equiv 1 \bmod 7\) and \(n_7\mid 8\).
    If \(n_7=1\) then such \(H_7\) is normal and we are done.
    % if by negation $G$ is simple, then \(n_7=8\). % and \(n_2=7\)

    Otherwise, we can assume \(n_7=8\). % and \(n_2=7\)
    Since such $7$-Sylow subgroups intersect in \eG,
    the number of elements
    in $G$ with order $7$ is \(n_7(7-1)=48\).
    The elements of any $2$-Sylow subgroup are of order \(\neq7\).
    And since \(56-48=8\) there could be only
    one such subgroup \(H_2\{g\in G: g^7\neq e\}\) whose order is $8$
    and must be normal.
\end{itemize}

%%%%%
\begin{excopy}
Let $p$, $q$ be distinct primes. Prove that a group of order \(p^2q\)
is solvable, and that one of its Sylow subgroups is normal.
\end{excopy}  \label{ex:p2q}

By Local Lemma~\ref{rose94:p2q} one of its Sylow subgroups is normal.
Then one of the normal towers
\begin{itemize}
 \item[] \(\eG \subnormal H_p \subnormal G\)
 \item[] \(\eG \subnormal H_q \subnormal G\)
\end{itemize}
exist and can be refined into abelian (and even cyclic) tower.

%%%%%
\begin{excopy}
Let $p$, $q$ be odd primes. Prove that a group of order \(2pq\) is solvable.
\end{excopy}  \label{ex:2pq}

By Local Lemma~\ref{rose94:pqr} such group $G$ has a normal subgroup $H$.

Now \(|H|\in \{p,q,r,pq,pr,qr\}\) and
by Proposition~6.8 the tower \(\eG\subnormal H\subnormal G\)
can be refined to abelian (and cyclic) tower.



%%%%%
\begin{excopy}
\begin{itemize}
 \item[(a)]
   Prove that one of the Sylow subgroups of a group of order $40$ is normal.
 \item[(b)]
   Prove that one of the Sylow subgroups of a group of order $12$ is normal.
\end{itemize}
\end{excopy} \label{ex:G40G12}

\begin{itemize}
\item[(a)] We have \(40=2^3\cdot5\) so the number of 5-Sylow subgroups
  must satisfy \(n_5\mid 8\) and \(n_5\equiv 1 \bmod 5\)
  and so \(n_5=1\) and the unique 5-Sylow subgroup is normal.
\item[(b)]
 From exercise~\ref{ex:p2q}  with \(p^2q=12\) where \(p=2\) and \(q=3\).
\end{itemize}

%%%%%
\begin{excopy}
Determine all groups of order \(\leq 10\) up to isomorphism.
In particular, show that a non-abelian group of order $6$
is isomorphic to \(S_3\).
\end{excopy}

The groups with prime order are cyclic. For other case of a group $G$:
\begin{itemize}
  \item Assume \(|G|=6=2\cdot3\). It could be isomorphic to:
    \begin{itemize}
       \item \(\Zm{6}\) cyclic.
       \item \(\Zm{2}\times\Zm{3}\) abelian.
       \item \(S_3\).
    \end{itemize}
  \item Assume \(|G|=8=2^3\). It could be isomorphic to:
    \begin{itemize}
       \item \(\Zm{8}\) cyclic.
       \item \(\Zm{2}\times\Zm{2}\times\Zm{2}\) abelian.
       \item \(\Zm{2}\times\Zm{4}\) abelian.
    \end{itemize}
  \item Assume \(|G|=9=3^2\). It could be isomorphic to:
    \begin{itemize}
       \item \(\Zm{9}\) cyclic.
       \item \(\Zm{3}\times\Zm{3}\) abelian.
    \end{itemize}
  \item Assume \(|G|=10=2\cdot5\). It could be isomorphic to:
    \begin{itemize}
       \item \(\Zm{10}\) cyclic.
       \item \(\Zm{2}\times\Zm{5}\) abelian.
    \end{itemize}

\end{itemize}

%%%%%
\begin{excopy}
Let \(S_n\) be the permutation group on $n$ elements.
Determine the $p$-Sylow subgroups of
\(S_3\), \(S_4\), \(S_5\) for \(p=2\) and \(p=3\).
\end{excopy}

\begin{itemize}
 \item[\(S_3\)]
    The $2$-Sylow subgroups % generated by transposition and they
    are:
    \(\{e,(12)\}\), \(\{e,(13)\}\) and \(\{e,(23)\}\).
    The $3$-Sylow subgroup is
    \(\{e,(123),(132)\}\).
 \item[\(S_4\)]
    The $2$-Sylow subgroup is \(S_4\) itself and no $3$-Sylow subgroups.
 \item[\(S_5\)] No $2$-Sylow and no $3$-Sylow subgroups.
\end{itemize}

%%%%%
\begin{excopy}
Let \(\sigma\) be a permutation of a finite set $I$ having $n$ elements.
Define \(e(\sigma)\) to be \((-1)^m\) where
\[m = n - \textrm{number of orbits of }\, \sigma.\]
If \(I_1,\ldots,I_r\) are orbits of \(\sigma\), then $m$ is also equal
to the sum
\[ m= \sum_{v=1}^r [\card(I_v)-1].\]
If \(\tau\)  is a transposition, show that \(e(\sigma\tau) = -e(\sigma)\)
be considering the two cases where $i$, $j$ lie in the same orbit of \(\sigma\),
or lie in different orbits. In the first case, \(\sigma\tau\) has one more
orbit and in theses case one less orbit that \(\sigma\).
In particular, the sign of a transposition is \(-1\).
Prove that \(e(\sigma)=\epsilon(\sigma)\) is the sign of the permutation.
\end{excopy}

We show the equality of $m$,
\[ \sum_{v=1}^r [\card(I_v)-1] =
   \sum_{v=1}^r \card(I_v) - \sum_{v=1}^r 1 =
   \sum_{v=1}^r \card(I_v) - \sum_{v=1}^r 1 =
   n - r.\]

We now show \(e(\sigma\tau)= -e(\sigma)\). Let \(\tau=(ij)\).
There are two cases:
\begin{itemize}
 \item
   The elements $i$, $j$  lie in the same orbit $I$ of \(\sigma\).
   Let \(l=|I|\geq 2\) and \(1\leq k<l\) such that \(\sigma^k(i)=j\).
   It is clear that \(\sigma^{l-k}(j)=i\).
   Now \(\sigma\tau\) has all the orbits of \(\sigma\)
   with $I$ split into two orbits:
     \((i, \sigma(j), \ldots \sigma^{l-k-1}(j))\)
   and
     \((j, \sigma(i), \ldots \sigma^{k-1}(j))\).
 \item
   The elements $i$, $j$  lie in different orbits \(I_i\) and \(I_j\)
   respectively of \(\sigma\).
   Now \(\sigma\tau\) has all the orbits of \(\sigma\)
   but with \(I_i\) and \(I_j\) united.
\end{itemize}
In both cases the number of orbits of \(\sigma\tau\) differs by $1$
from that of \(\sigma\).
Hence
\(e(\sigma\tau) = (-1)^{m+1} = -(-1)^m = -e(\sigma)\).

Since the transpositions  generates \(S_n\) and both $e$ and \(\epsilon\)
agree on the transpositions and the identity (\(e(\id)=\epsilon(\id)=1\))
they agree on all permutations.

%%%%%
\begin{excopy}
\begin{itemize}
 \item[(a)]
   Let $n$ be an even positive integer. Show that there exists  a group
   of order \(2n\), generated by two elements \(\sigma\), \(\tau\)
   such that \(\sigma^n=e=\tau^2\), and \(\sigma\tau=\tau\sigma^{n-1}\).
   (Draw a picture of a regular $n$-gon, number the vertices,
   and use the picture as an inspiration to get \(\sigma\), \(\tau\).)
   Thus group is called the
   \index{dihedral} \index{group!dihedral}
   \textbf{dihedral group}.
 \item[(b)]
   Let $n$ be an odd positive integer. Let \(D_{4n}\) be the group generated
   by the matrices
   \begin{equation}
     \left(
      \begin{array}{lr}
       0 & -1 \\
       1 & 0 \\
      \end{array}
     \right)
     \quad\textrm{and}\quad
     \left(
      \begin{array}{lc}
       \zeta & 0 \\
       0 & \zeta^{-1} \\
      \end{array}
     \right)
   \end{equation}
   where \(\zeta\) is a primitive $n$-th root of unity. Show that \(D_{4n}\)
   has order \(4n\), and give the commutation relations between the above
   generators.
\end{itemize}
\end{excopy}

\begin{itemize}
 \item[(a)]
 Let \(\theta=2\pi/n\). Now let\(\sigma\) be a \(1/n\) rotation
 and \(\tau\) a reflection. More formally:
   \begin{equation}
     \sigma = \left(
      \begin{array}{rl}
       \cos\theta & \sin\theta \\
       -\sin\theta & \cos\theta \\
      \end{array}
     \right)
     \quad\textrm{and}\quad
     \tau = \left(
      \begin{array}{lr}
       1 & 0 \\
       0 & -1 \\
      \end{array}
     \right)
   \end{equation}

 \item[(b)]
  Denote
   \begin{equation}
     \sigma = \left(
      \begin{array}{lr}
       0 & -1 \\
       1 & 0 \\
      \end{array}
     \right)
     \quad\textrm{and}\quad
     \tau = \left(
      \begin{array}{lc}
       \zeta & 0 \\
       0 & \zeta^{-1} \\
      \end{array}
     \right)
   \end{equation}

  From that we compute \(\sigma^2 = -\Id\) and \(\sigma^4 = \tau^n = \Id\).
  Thus \(\sigma^2\tau=\tau\sigma^2\) and \(\tau^n\sigma=\sigma\tau^n\).
\end{itemize}

\iffalse
\begin{itemize}
  \item[(a)] Rotation and mirroring. Consider the subgroup of \(S_n\)
     where
   \begin{equation*}
     \sigma(i) =
       \left\{
         \begin{array}{ll}
         i + i \;& \textnormal{if}\; i < n \\
         0     \;& \textnormal{if}\; i = n
         \end{array}
       \right.
   \end{equation*}
   and \(\tau(i) = (n - i + 1)\).

  \item[(b)]
  Say $J$ is the first matrix. Then
  \begin{equation*}
    J^2 =
      \left(
        \begin{array}{rr}
        -1 & 0 \\
        0  & -1
        \end{array}
      \right)
     \quad\textrm{and}\quad
    J^3 =
      \left(
        \begin{array}{rr}
        0 & 1 \\
        -1 & 0
        \end{array}
      \right)
     \quad\textrm{and}\quad
    J^4 =
      \left(
        \begin{array}{rr}
        1 & 0 \\
        0 & 1
        \end{array}
      \right).
  \end{equation*}
\end{itemize}
\fi


%%%%%
\begin{excopy}
Show that there are exactly two non-isomorphic non-abelian groups of order~$8$.
(one of them is given by the generators \(\sigma\), \(\tau\) with the relations
\begin{equation*}
\sigma^4 = 1, \qquad \tau^2 = 1, \qquad \tau\sigma\tau = \sigma^3.
\end{equation*}
The other is the quaternion group.)
\end{excopy}

Let $G$ be non-abelian group of order~$8$.
Let $m$ be the maximal period of the elements of $G$.
Since \(m|8\) we must have \(m\in\{1,2,4,8\}\).

We will show that \(m=4\).
If \(m=1\) then \(|G|=1\), contradiction.
If \(m=2\) then for any \(a,b\in G\)
\begin{equation*}
1 = (ab)(ab) = a(bb)a = (ab)(ba)
\end{equation*}
and so
\begin{equation*}
(ab)^{-1} = ab = ba
\end{equation*}
and $G$ is abelian, contradiction.
If \(m=8\) then $G$ is cyclic, contradiction.

Let \(\sigma\in G\) be of period $4$. It generates
the subgroup \(H=\{\sigma^1,\sigma^2,\sigma^3,1\}\).
\newcommand{\coH}{\ensuremath{\tilde{H}}}
Put  \(\coH = \coH\).
Since \([G:H]=2\) there are two cosets of $H$ and \coH\ in $G$.
Now for any \(g_1,g_2\in \coH\) we have
\begin{equation} \label{eq:H=g1g2H}
H = g_1 g_2 H = g_1 H g_2 = H g_1 g_2.
\end{equation}
and in particular \(H \triangleleft G\).

Since conjunction is automorphism, \(g\sigma g^{-1} \in H\) is
of period $4$ for each \(g\in G\).
Thus
\begin{equation*}
\forall g\in G,\; g\sigma g^{-1} \in \{\sigma^1,\sigma^3\}.
\end{equation*}

Assume by negation \(\exists y\in \coH,\; g\sigma g^{-1}=\sigma\).
Then
\begin{equation*}
\exists y\in \coH\, \forall k\in\{0,1,2,3\},\; g\sigma^kg^{-1}=\sigma^k.
\end{equation*}
Thus
\begin{equation*}
\exists y\in \coH\, \forall h\in H,\; yh=hy.
\end{equation*}
But any \(z \in \coH\) is of the form \(z = yh'\) for some \(h'\in H\)
and so for any \(h\in H\) we have
\begin{equation*}
zh = (yh')h = y(h'h) = (h'h)y = (hh')y = h(h'y) = hz.
\end{equation*}
and now
\begin{equation*}
\forall y\in \coH\, \forall h\in H,\; yh=hy.
\end{equation*}

Let \(y_1,y_2\in \coH\). By looking at \(\coH\) as a coset,
\(y_2 = h y_1 = y_1 h\) for some \(h\in H\).
Now since \((y_1)^2\in H\) as we saw in \eqref{eq:H=g1g2H}
\begin{equation*}
y_1 y_2 = y_1 (h y_1) = y_1 (y_1 h) = (y_1)^2 h = h (y_1)^2 = (h y_1)y_1
= (y_1 h) y_1 = y_2 y_1.
\end{equation*}
Thus $G$ is abelian, and by contradiction
\begin{equation*}
\forall y\in \coH,\; y\sigma y^{-1}=\sigma^3.
\end{equation*}
Similarly,
\begin{eqnarray*}
\forall y\in \coH,\; & y^{-1}\sigma y &= \sigma^3 \\
\forall y\in \coH,\; & y\sigma^3 y^{-1} &= \sigma \\
\forall y\in \coH,\; & y^{-1}\sigma^3 y &= \sigma.
\end{eqnarray*}

Clearly the periods of \(y\in \coH\) could be $2$ or $4$.
If all these periods are $2$ then

Assume the index of $y$ is $2$ for some \(y\in\coH\).
Then \(y^2=1\) and \(u=y^{-1}\) and so
\begin{equation*}
(\sigma y)^2 = \sigma(y \sigma y^{-1}) = \sigma^{1+3}=1.
\end{equation*}
Since conjunction is automorphism, \(y\sigma^2 y^{-1} = \sigma^2\)
begin the only element of order $2$ in $H$.
Noting that \(y\cdot 1 \cdot y{-1} = 1\)
we have
\begin{equation*}
\left\{y\sigma^n y^{-1}: n\in\{0,1,2\}\right\}
=
\left\{y\sigma^n y^{-1}: n\in\{0,3,2\}\right\}.
\end{equation*}
So by looking at the reminder
\(y\sigma^3 y^{-1} = \sigma\).
\begin{equation*}
(\sigma^3 y)^2 = \sigma^3 (y \sigma^3) y = \sigma^{3+1} = 1.
\end{equation*}
Thus all elements of \coH\ are of \emph{equal} period, $2$ or $4$.

Thus we have two possibilities.
\begin{enumerate}

\item If the order of \(y\in\coH\) is $2$ then $G$ is the diehedral group
with the relations specified in the exercise.

\item If the order of \(y\in\coH\) is $8$ then $G$ is the quaternion group.
We put \(i=\sigma\), pick arbitrary \(j\in\coH\), and put \(k=ij\).
Now since the orders of $j$ and $k$ are also $4$,
we have \(|\{i^3,j^3,k^3\}|=3\) (different elements) and
\begin{equation*}
rsr^{-1}=s^3 \qquad \textnormal{where} \qquad
(r,s) \in \left\{(i,j), (i,k), (j,i), (j, k), (k,i), k,j)\right\}.
\end{equation*}
and \(i^2 = j^2 = k^2\) which we conveniently denote as \((-1)\).
With this we have the 3 elements
\begin{eqnarray*}
(-i) &=& i^{3} = (-1)i = i(-1) \\
(-j) &=& j^{3} = (-1)j = j(-1) \\
(-k) &=& k^{3} = (-1)k = k(-1).
\end{eqnarray*}
Also
\begin{eqnarray*}
ji &=& ji(j^{-1}j) = (jij^{-1})j = i^{3}j = i^2k = (-1)k \\
jk &=& j(ij)(i^{-1}i) = (ii^{-1})j(ij) = i(i^{-1}ji)j = ij^{3+1} = i \\
kj &=& (kj)(k^{-1}k) = (kjk^{-1})k = j^{2+1}k = (-1)jk = (-1)i   \\
ik &=& ik(jj^3) = i(kj)j^3 = i^{1+2+1}j^{2+1} = (-1)j \\
ki &=& ki(k^{-1}k) = (kik^{-1})k = i^{2+1}k = (-1)(ik) = (-1)j
\end{eqnarray*}
\end{enumerate}


%%%%%
\begin{excopy}
Let \(\sigma = [123 \ldots n]\) in \(S_n\).
Show that the conjugacy class of \(\sigma\) has \((n - 1)!\) elements.
Show that the centralizer of \(\sigma\) is the cyclic group generated by
\(\sigma\).
\end{excopy}

The following relation on \(S_n\)
\begin{equation*}
\tau_1 \sim \tau_2
\qquad \textnormal{iff} \qquad
\exists k\in \N_n,\, \tau_1 \sigma^k = \tau_2.
\end{equation*}
is an equivalence relation, since
\(k=0\) gives reflexivity, symmetry by considering
\(\tau_2 \sigma^{n-k} = \tau_1\) and associativity
since if \(\tau_1 \sim \tau_2\)
and if \(\tau_2 \sim \tau_3\)
with \(k_1\) and \(k_2\) respectively, then
\begin{equation*}
\tau_1 \sigma^{k_3} = \tau_3
\qquad \textnormal{where}\; k_3 = k_1+k_2.
\end{equation*}

For any \(\tau\in S_n\) and \(k \in \N_n\)
\begin{equation*}
(\tau\sigma^k)\sigma\left(\tau\sigma^k\right)^{-1}
= \tau\sigma^{k+1-k}\tau^{-1}
= \tau\sigma\tau^{-1}.
\end{equation*}

Assume \(\tau_1 \nsim \tau_2\).
We choose \(\tau'_1 \sim \tau_1\)
and \(\tau'_2 \sim \tau_2\)
such that
\(\tau'_1(1) = \tau'_2(1)\).

Clearly we can find some \(j\in\N_n\) such that
\(\tau'_1(j) = \tau'_2(j) = j\)
and
\({\tau'_1}^{-1}(j + 1) \neq {\tau'_2}^{-1}(j + 1)\).
Now
\begin{equation*}
\left(\tau'_1 \sigma {\tau'_1}^{-1}\right)(j')
\neq
\left(\tau'_2 \sigma {\tau'_2}^{-1}\right)(j')
\end{equation*}
Thus
\begin{equation*}
\tau'_1 \sigma {\tau'_1}^{-1}
=
\tau'_2 \sigma {\tau'_2}^{-1}
\qquad \textnormal{iff} \qquad
\tau'_1 \sim \tau'_2.
\end{equation*}
Thus the size of the conjugacy class of \(\sigma\)
is the same as the number of classes of the equivalence relation \(\sim\)
which is \(n!/n = (n-1)!\).

%%%%%
\begin{excopy}
\begin{enumerate}[(a)]
\item
Let \(\sigma = [i_1\cdots i_m]\) be a cycle.
Let \(\gamma \in S_n\). Show that \(\gamma\sigma\gamma^{-1}\)
is  the cycle \([\gamma(i_1)\cdots \gamma(i_m)]\).
\item
Suppose that a permutation \(\sigma\) in \(S_n\) can be written
as a product of $r$ disjoint
cycles, and let \(d_1,\ldots,d_r\) be the number of elements in each cycle,
in increasing order.
 Let \(\tau\) be another permutation which can be written as a product of
disjoint cycles, whose cardinalities are
\(d'_1,\ldots,d'_r\)
in increasing order. Prove
that \(\tau\) is conjugate to \(\tau\) in \(S_n\) if and only if
\(r = s\) and \(d_i = d'_i\) for all \(i=1,\ldots,r\).
\end{enumerate}
\end{excopy}

\begin{enumerate}[(a)]
\item
\begin{equation*}
\left(\gamma\sigma\gamma^{-1}\right)(\gamma(i_j))
= \left(\gamma\sigma\right)\left(\gamma^{-1}\gamma\right)(i_j)
= \left(\gamma\sigma\right)(i_j)
= \gamma(\sigma(i_j))
= \gamma(i_{\bres{(j+1}{m}}).
\end{equation*}
\item
If \(\tau\) is conjugate to \(\sigma\) then the equalities
of the cycles sizes follow from (\emph{a}).
Conversely, if the cycles sizes agree then
if \([i_1,\ldots,i_m]\) is a cycle of \(\sigma\)
and  \([j_1,\ldots,j_m]\) is a cycle of \(\tau\)
we define \(\mu(i_k)=j_k\) for \(k\in \N_n\).
Since \(\N_n\) is a disjoint union of any permutation in \(S_n\)
we have \(\mu\in S_n\) well defined
and \(\tau = \mu\sigma\mu^{-1}\).
\end{enumerate}

%%%%%
\begin{excopy}
\begin{enumerate}[(a)]
\item
Show that \(S_n\) is generated by the transpositions
\([12]\), \([13]\),\(\ldots\),\([1n]\).
\item
Show that \(S_n\) is generated by the transpositions
\([12]\), \([23]\), \([34]\),\(\ldots\),\([n-1,n]\).
\item
Show that \(S_n\) is generated by the cycles \([12]\) and \([1 2 3 \ldots n]\).
\item
Assume that $n$ is prime.
Let \(\sigma = [1 2 3 \ldots n]\) and let \(\tau = [rs]\) be any transposition.
Show that \(\sigma\), \(\tau\) generate \(S_n\).
\end{enumerate}
\end{excopy}

Note that given a generator \(g\in S_n\),
we always have \(\Id\in S_n\) gernerated by \(\Id = g^k\) for some \(k\in\N\).
Given \(\sigma,\tau\in S_n\). We define the distance
\begin{equation*}
d(\sigma,\tau) = \left\|\{i\in\N_n: \sigma(i) \neq \tau(i)\}\right|.
\end{equation*}
\begin{enumerate}[(a)]
\item Let \(\sigma\in S_n\). Let \(G\subset S_n\) be the group generated
by the given generators. Let \(g\in G\) be with the minimal
distance with \(sigma\).
If by negation \(d(\sigma,g) > 0\) then let \(j\in\N_n\)
be the minimal index such that \(\sigma(j)\neq g(j)\).
Define
\begin{equation*}
g' = [1 j][1 g^{-1}\sigma(j)][1 j]g
\end{equation*}
and now \(d(\sigma, g') < d(\sigma, g)\) contrdicting the minimal choice.
\item
For each \(k \in \N_n\) we have
\begin{equation*}
[1 k] = [1 2]\cdot[2 3]\cdots[(k-2),(k-1)]\cdot[(k-1),k]\cdots[2 3]\cdot[1 2]
\end{equation*}
Thus we get the generators of previous case (a).
\item
For each \(k \in \N_n\) we have
\begin{equation*}
[k,(k+1)] = [1 2 3\ldots n]^{(k-1)}\cdot[12]\cdot [1 2 3\ldots n]^{n - (k-1)}.
\end{equation*}
Thus we get the generators of previous case (b).
\item
Assume \(d=r-s>0\). for any \(k\in \N_n\)
\begin{equation*}
[k,\bres{(k+d)}{n}] = \sigma^{k-r}[r s]\cdot\sigma^{r-k}
\end{equation*}
and
\begin{equation*}
\tau_k = [\bres{1 + kd}{n}, \bres{1 + (k+1)d}{n}].
\end{equation*}
are all generated.
Since $n$ is prime, for any \(m < n\) there exists \(q\in\N\)
such that \(d^q = (m - 1) \bmod n\). Thus
\begin{equation*}
[1 m] = \tau_0\cdot\tau_1\cdots\tau_q\cdots\tau_1\cdots\tau_0.
\end{equation*}
Hence the generators of (b) are generated.
\end{enumerate}

\end{myenumerate}

Let $G$ be a finite group operating on a set $S$.
Then $G$ operates in a natural way on
the Cartesian product \(S^{(n)}\) for each positive integer $n$ .
We define the operation on $S$
to be \hbox{\boldmath$n$\textbf{-transitive}} if given $n$ distinct elements
\((s_1,\ldots,s_n)\) and $n$ distinct elements
\((s'_1,\ldots,s'_n)\) of $S$, there exists \(\sigma\in G\)
such that \(\sigma s_i = s'_i\) for all \(i = 1,\ldots,n\).

\begin{myenumerate}
%%%%%
\begin{excopy}
Show that the action of the alternating group \(A_n\)
on \(\{1,\ldots,n\}\) is \((n - 2)\)-transitive.
\end{excopy}

Given arbitrary
\(n-2\) elements \((s_1,\ldots,s_{n-2})\)  in \(\N_n\)
and
\(n-2\) elements \((s'_1,\ldots,s'_{n-2})\)  in \(\N_n\).
We have a pair of two remaining elements
\begin{eqnarray*}
\{s_{n-1}, s_n\} &= \N_n \setminus \{(s_1,\ldots,s_{n-2}\}\\
\{s'_{n-1}, s'_n\} &= \N_n \setminus \{(s'_1,\ldots,s'_{n-2}\}.
\end{eqnarray*}
For \(i\in\N_n\) define  \(\sigma_1(s_i) = s'_i\) and
\begin{equation*}
  \sigma_2(s_i) =
    \left\{
      \begin{array}{ll}
        s'_i \quad &\textnormal{iff}\; i \leq n - 2 \\
        s'_n \quad &\textnormal{iff}\; i = n - 1 \\
        s'_{n-1} \quad &\textnormal{iff}\; i = n \\
      \end{array}
    \right.
\end{equation*}
Clearly \(\sigma_1,\sigma_2\in\S_n\) having different signs, thus
(exactly) one of them \(\in A_n\).

%%%%% ex 40
\begin{excopy}
Let \(A_n\) be the alternating group of even permutations of
\(\{1,\ldots,n\}\), For \(j = 1,\ldots,n\)
let \(H_j\) be the subgroup of \(A_n\) fixing $j$,
so \(H_j \approx A_{n-1}\), and \((A_n: H_j) = n\) for \(n > 3\).
Let \(n \geq 3\) and let $H$ be a subgroup of index $n$ in \(A_n\).
\begin{enumerate}[(a)]
\item
Show that the action of \(A_n\) on cosets of $H$ by left translation
gives an isomorphism \(A_n\) with the alternating group of permutations
of \(A_n/H\).
\item
Show that there exists an automorphism of \(A_n\) mapping \(H_1\) on $H$,
and that
such an automorphism is induced by an inner automorphism of \(S_n\) if and only
if \(H = H_i\) for some~$i$.
\end{enumerate}
\end{excopy}

Note that
\begin{equation*}
(A_n: H_j) = |A_n|/|A_{n-1}| = (n!/2)/\left((n-1)!/2)\right) = n!/(n-1)! = n.
\end{equation*}
Let \(A_n\) act on cosets \(\tau H_j\) by multiplication from left.
We need to show that this action is well defined.
Let \(\tau_1 H_j = \tau_2 H_j\),
Thus we have \(h_1,h_2\in H_j\) such that
\(\tau_1 h_1 = \tau_2 h_2\)
and so
\(\sigma\tau_1 h_1 = \sigma\tau_2 h_2\) for all \(\sigma \in S_n\)
and so
\(\sigma\tau_1 H_j = \sigma\tau_2 H_j\) for all \(\sigma \in A_n\)
showing that the action is well defined.
\begin{enumerate}[(a)]
\item
With similar argunent we showed that the action is well definded,
we can also show that the action is an homomorphism of \(A_n\)
on the permutations of \(A_n/H\).
We still need to show it is an injection and surjection..

We check manually for n=3,4.
\begin{equation*}
A_3 = \{e, [1,2,3], [1, 3, 2]\}
\end{equation*}
and thus \(|H|=1\) so \(H=\{e\}\) and the isomorphism is clear.

\begin{align*}
A_4 = \{&e, \\
  &[1,2,3], [1, 3, 2], [1,2,4], [1, 4, 2], [1, 3, 4], [1, 4, 3],
    [2, 3, 4], [2, 4, 3], \\
  &[1, 2][3, 4], [1, 3][2, 4], [1, 4][2, 3]\}.
\end{align*}
\(|A_4|=4!/2=12\) and \(|H|=|A_4|/4=3\).
The possible subgroups are of the
form \(\{e, h, h^2\}\) where
\(h \in \{[i,j,k]\in A_4: i\neq j \neq k\}\).
Note we always have \(h^3=e\).
Now there are 4 disjoint cosets \(eH, \tau_1 H, \tau_2 H, \tau_3 H\)
where \(\tau_i \in A_n\) for \(i=1,2,3\), and put \(\tau_0 = e\).
To show innjection, let \(\sigma_1, \sigma_2 \in A_4\)
and assume \(\sigma_2\tau_i H = \sigma_2\tau_i H\) for \(i=0,1,2,3\).
Looking at \(i=0\), we have
\(\sigma_1(e) \in \{\sigma_2 e, \sigma_2 h, \sigma_2 h^2\}\).\\
Cases:
\begin{itemize}
\item \(\sigma_1 e = \sigma_2 e\). Clearly \(\sigma_1 = \sigma_2\).
\item \(\sigma_1 e = \sigma_2 h\). Then \(\sigma_2^{-1}\sigma_1 = h\).
 \Wlogy we may assume \(h=[1,2,3]\).
Manually calculating permutations cycles,
 \(H = \{[e, [1, 2, 3], [1, 3, 2]\}\). The 4 cosets are:
\begin{align*}
H_1 &= \{e, [1, 2, 3], [1, 3, 2]\} = H \\
H_2 &= \{[2, 3, 4], [1, 3][2, 4], [1, 4, 2]\} \\
H_3 &= \{[2, 4, 3], [1, 2][3, 4], [1, 4, 3]\} \\
H_4 &= \{[1, 2, 4], [1, 3, 4], [1, 4][2, 3]\}.
\end{align*}
Now multiplying from left by all \(A_4\) permutations:
\begin{center}
\begin{tabular}{lll}
 i & \(\alpha \in A_4\) & $H$ indices of \(\alpha H_{1,2,3,4}\) \\
 1 & [1, 2, 3, 4] & [1, 2, 3, 4] \\
 2 & [1, 3, 4, 2] & [2, 3, 1, 4] \\
 3 & [1, 4, 2, 3] & [3, 1, 2, 4] \\
 4 & [2, 1, 4, 3] & [3, 4, 1, 2] \\
 5 & [2, 3, 1, 4] & [1, 3, 4, 2] \\
 6 & [2, 4, 3, 1] & [4, 1, 3, 2] \\
 7 & [3, 1, 2, 4] & [1, 4, 2, 3] \\
 8 & [3, 2, 4, 1] & [4, 2, 1, 3] \\
 9 & [3, 4, 1, 2] & [2, 1, 4, 3] \\
10 & [4, 1, 3, 2] & [2, 4, 3, 1] \\
11 & [4, 2, 1, 3] & [3, 2, 4, 1] \\
12 & [4, 3, 2, 1] & [4, 3, 2, 1]
\end{tabular}
\end{center}
The table shows the left- multiplication is injective, 
thus \(\sigma_1=\sigma_2\).

\item \(\sigma_1 e = \sigma_2 h^2\).
  Putting \(h' = h^2\) and then \(h'^2 = h^4 = h\).
  So we can apply the previous case.
\end{itemize}
That the end of handling \(A_3\).

Now we may assume \(n \geq 5\).
Assume \(\sigma \in A_n\) is in the kernel
of the left multiplication mapping.Then
\begin{equation} \label{eq:sigma-in-kern:An}
\forall \tau\in A_n\quad \sigma\tau H = \tau H.
\end{equation}
Now
\begin{equation*}
\sigma\tau H = \tau H
\;\Leftrightarrow\;
\tau^{-1}\sigma\tau H = H
\;\Leftrightarrow\;
\tau^{-1}\sigma\tau \in H
\;\Leftrightarrow\;
\sigma \in \tau H \tau^{-1}
\end{equation*}
Thus \eqref{eq:sigma-in-kern:An} gives
\begin{equation*}
\sigma \in \bigcap_{\tau\in A_n} \tau H \tau^{-1}
\end{equation*}
The latter intersection is clearly a normal subgroup of \(A_n\)
Since it is a subgroup of $H$ it is a proper subgroup of \(A_n\)
so it must be the trivial \(\{e\}\) since \(A_n\) is simple.
Thus the left multiplication is injective.


\begin{llem} \label{llem:half:normal}
Let $H$ be a subgroup of $G$. If \([G:H]=2\) then \(H \subnormal G\).
\end{llem}
\begin{proof}
Let \(g \in G\setminus H\) then clearly  \(gH\cap H = \emptyset\)
and since \(|gH| + |H| = |G|\) we have
\begin{equation*}
G = H \dotcup gH = H \dotcup g^{-1}H = H \dotcup Hg = H \dotcup Hg^{-1}.
\end{equation*}
Thus \(gH = Hg\) and so \(gHg^{-1}= (Hg)g^{-1} = H\).
\end{proof}

\begin{llem} \label{llem:unique:Sn:half}
For \(n\geq 2\) there exists a unique subgroup of \(S_n\) of order \(n!/2\)
namely \(A_n\).
\end{llem}
\begin{proof}
Let \(\sigma\;S_n\to \{+1,-1\}\) be the sign group homomorphism.
Its kernel is clearly \(A_n\).
Now let \(H\subset S_n\) with \(|H|=n!/2\).
By local-lemma~\ref{llem:half:normal} \(H\subnormal G\).
If \(n<5\) then we manually verify that  \(H==A_n\). 
We may now assume \(n\geq 5\).
Clearly \(H\cap A_n \subnormal A_n\). By Theorem~5.5
\(H\cap A_n = A_n\) and we are done, 
or by negation \(H\cap A_n = \{e\}\). Now since \(|H\cup A_n| = n! -1\)
clearly $H$ Contains \(n!/2 - 1\) odd permutations.
Pick 3 odd permutations, at least 2 of them \(\pi_1, \pi_2\)
satisfy \(\kappa=\pi_1\pi_2 \neq e\) and \(\kappa\) is even
thus \(\kappa \in A_n\) which gives a contradiction.
\iffalse
% Since \(G/H \simeq C_2 = \{+1,=1\}\)
% Consider the natural homomorphism \(\Lambda G \to G/H \{+1,=\{+1,=1\}\).
As was shown in the text, \(A_n\) is generated by all 3-cycles
using \([ij][rs] = [ijr][irs]\).
If by negation \(H \neq A_n\) then both $H$ and \(S_n\setminus H\)
contains some 3-cycles. Say
\begin{equation*}
[ijk] \in H \qquad [xyz]\in S_n\setminus H
\end{equation*}
Say \(I = \{i,j,k\}\cap \{x,y,z\}\).
Without loss of generality we may assume
that if \(|I|\geq 1\) then \(i=x\)
and  if \(|I|=2\) then \(j=y\)
Now consider
\begin{equation*}
\sigma =
 \left\{
  \begin{array}{ll}
   {[ix]}[jy][kz] \quad &\textnormal{if}\; |I|=0 \\
   {[jy]}[kz] \quad &\textnormal{if}\; |I|=1 \\
   {[kz]} \quad &\textnormal{if}\; |I|=2 \\
  \end{array}
 \right.
\end{equation*}
Now \([xyz] = \sigma [ijk] \sigma^{-1}\).
And by \(H \subnormal S_n\) we have the contrdiction \([xyz] \in H\).
\fi
\end{proof}

Back to the exercise.
The left multiplication $L$ maps \(A_n\) to permutations
of the $n$ cosets of $H$.
Thus we have a one-to-one mapping \(L: A_n \rightarrow S_n\).
Thus \(|L(A_n)|=|A_n|=|S_n|/2 = n!/2\).
By local-lemma~\ref{llem:unique:Sn:half} \(L(A_n)=A_n\).

\item
From the previous part, we have a 1-1 onto mapping
 \(L: A_n \to \Alt(A_n/H)\simeq A_n\).
Now for each \(\alpha \in H\)
\begin{equation*}
 L(\alpha) = eH, \tau_2 H, \ldots \tau_n H
\end{equation*}
Thus \(L(H) = H_1\).

Let \(\\sigma \in H_i\) and \(\tau\in A_n\) such that \(\tau(i)=j\)
then \(\tau\) acting by conjunction on \((H_1,H_2,\ldots,h_n\)
moves \(\tau H_i \tau^{-1} = H_j\).
Conversely, if \(H = \tau H_i \tau^{-1}\) then clearly \(H = H_j\).
\end{enumerate}

%%%%%
\begin{excopy}
Let $H$ be a simple group of order \(60\).
\begin{enumerate}[(a)]
\item
Show that the action of $H$ by conjugation on the set of its Sylow subgroups
gives an imbedding \(H \hookrightarrow A_6\).
\item
Using the preceding exercise, show that \(H \approx A_5\).
\item
Show that \(A_6\) has an automorphism which is not induced by an inner
automorphism of \(S_6\).
\end{enumerate}
\end{excopy}

%%%%%
\begin{excopy}
\end{excopy}

%%%%%
\begin{excopy}
\end{excopy}

\end{myenumerate}

%%%%%%%%%%%%%%%%%%%%%%%%%%%%%%%%%%%%%%%%%%%%%%%%%%%%%%%%%%%%%%%%%%%%%%%%
%%%%%%%%%%%%%%%%%%%%%%%%%%%%%%%%%%%%%%%%%%%%%%%%%%%%%%%%%%%%%%%%%%%%%%%%
%%%%%%%%%%%%%%%%%%%%%%%%%%%%%%%%%%%%%%%%%%%%%%%%%%%%%%%%%%%%%%%%%%%%%%%%
\bibliographystyle{plain}
\bibliography{langalg}

%%%%%%%%%%%%%%%%%%%%%%%%%%%%%%%%%%%%%%%%%%%%%%%%%%%%%%%%%%%%%%%%%%%%%%%%
%%%%%%%%%%%%%%%%%%%%%%%%%%%%%%%%%%%%%%%%%%%%%%%%%%%%%%%%%%%%%%%%%%%%%%%%
%%%%%%%%%%%%%%%%%%%%%%%%%%%%%%%%%%%%%%%%%%%%%%%%%%%%%%%%%%%%%%%%%%%%%%%%
% \input{langalg.ind}
\printindex

%%%%%%%%%%%%%%%%%%%%%%%%%%%%%%%%%%%%%%%%%%%%%%%%%%%%%%%%%%%%%%%%%%%%%%%%
%%%%%%%%%%%%%%%%%%%%%%%%%%%%%%%%%%%%%%%%%%%%%%%%%%%%%%%%%%%%%%%%%%%%%%%%
%%%%%%%%%%%%%%%%%%%%%%%%%%%%%%%%%%%%%%%%%%%%%%%%%%%%%%%%%%%%%%%%%%%%%%%%
\end{document}

\printindex

%%%%%%%%%%%%%%%%%%%%%%%%%%%%%%%%%%%%%%%%%%%%%%%%%%%%%%%%%%%%%%%%%%%%%%%%
%%%%%%%%%%%%%%%%%%%%%%%%%%%%%%%%%%%%%%%%%%%%%%%%%%%%%%%%%%%%%%%%%%%%%%%%
%%%%%%%%%%%%%%%%%%%%%%%%%%%%%%%%%%%%%%%%%%%%%%%%%%%%%%%%%%%%%%%%%%%%%%%%
\end{document}

\printindex

%%%%%%%%%%%%%%%%%%%%%%%%%%%%%%%%%%%%%%%%%%%%%%%%%%%%%%%%%%%%%%%%%%%%%%%%
%%%%%%%%%%%%%%%%%%%%%%%%%%%%%%%%%%%%%%%%%%%%%%%%%%%%%%%%%%%%%%%%%%%%%%%%
%%%%%%%%%%%%%%%%%%%%%%%%%%%%%%%%%%%%%%%%%%%%%%%%%%%%%%%%%%%%%%%%%%%%%%%%
\end{document}

\printindex

%%%%%%%%%%%%%%%%%%%%%%%%%%%%%%%%%%%%%%%%%%%%%%%%%%%%%%%%%%%%%%%%%%%%%%%%
%%%%%%%%%%%%%%%%%%%%%%%%%%%%%%%%%%%%%%%%%%%%%%%%%%%%%%%%%%%%%%%%%%%%%%%%
%%%%%%%%%%%%%%%%%%%%%%%%%%%%%%%%%%%%%%%%%%%%%%%%%%%%%%%%%%%%%%%%%%%%%%%%
\end{document}
