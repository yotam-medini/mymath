%%%%%%%%%%%%%%%%%%%%%%%%%%%%%%%%%%%%%%%%%%%%%%%%%%%%%%%%%%%%%%%%%%%%%%%%
%%%%%%%%%%%%%%%%%%%%%%%%%%%%%%%%%%%%%%%%%%%%%%%%%%%%%%%%%%%%%%%%%%%%%%%%
%%%%%%%%%%%%%%%%%%%%%%%%%%%%%%%%%%%%%%%%%%%%%%%%%%%%%%%%%%%%%%%%%%%%%%%%
\chapterTypeout{II. Infinitary combinatorics}

%%%%%%%%%%%%%%%%%%%%%%%%%%%%%%%%%%%%%%%%%%%%%%%%%%%%%%%%%%%%%%%%%%%%%%%%
%%%%%%%%%%%%%%%%%%%%%%%%%%%%%%%%%%%%%%%%%%%%%%%%%%%%%%%%%%%%%%%%%%%%%%%%
\section{Notes}

%%%%%%%%%%%%%%%%%%%%%%%%%%%%%%%%%%%%%%%%%%%%%%%%%%%%%%%%%%%%%%%%%%%%%%%%
\subsection{Theorem 1.3}

In the proof of Theorem~1.3
\begin{equation*}
  A_X = \{X\cap \alpha: \alpha<\kappa\}.
\end{equation*}
It claims: If \(|X|=\kappa\) then \(|A_X|=\kappa\). Why?\\
For each distinct \(\alpha,\beta < \kappa\)
we \emph{cannot} simply show \(X\cap \alpha \neq X\cap\beta\), since
if we pick \(X' = X \setminus (\alpha \cup \beta)\)
we still have \(|X'|=\kappa\) but
\(X' = X'\cap \alpha = X' \cap \beta\).

We enumerate \(X=\{x_\beta: \beta < \kappa\}\).
For each \(\beta < \kappa\), we have \(\{x\in X: x< x_\beta\} = X\cap \alpha\)
for some \(\alpha\) (not necessarily unique \(\alpha\) for \(\beta\).
So the families:
\begin{itemize}
\item \(\{x\in X: x< x_\beta\}\) for each \(\beta < \kappa\)
\item \(X\cap \alpha\) for each \(\alpha < \kappa\)
\end{itemize}
are the same, and so \(|X|=\kappa\).

%%%%%%%%%%%%%%%%%%%%%%%%%%%%%%%%%%%%%%%%%%%%%%%%%%%%%%%%%%%%%%%%%%%%%%%%
\subsection{Theorem 1.6}

In the proof of Theorem~1.6 it says:
\begin{quote}
Since \(\theta\) is regular and \(\theta > \kappa\), 
there is some \(\rho < \kappa\), such that
 \(\scrA_1 = \{x\in\scrA: x \;\textnormal{has type}\; \rho\}\)
has cardinality \(\theta\).
\end{quote}
To Justify:
\begin{equation*}
\scrA = \mathsmaller{\Disjunion_{\rho<\kappa}} 
  \{x\in\scrA: x \;\textnormal{has type}\; \rho\}.
\end{equation*}

In the proof of Theorem~1.6 it also says:
\begin{quote}
For each \(\alpha < \theta\), \(|\alpha^{<\kappa}| < \theta\)
implies that less than \(\theta\) elements of \(\scrA_1\) are
subsets of \(\alpha\) Thus, \(\cup \scrA_1\) is unbounded in \(\theta\).
\end{quote}


%%%%%%%%%%%%%%%%%%%%%%%%%%%%%%%%%%%%%%%%%%%%%%%%%%%%%%%%%%%%%%%%%%%%%%%%
%%%%%%%%%%%%%%%%%%%%%%%%%%%%%%%%%%%%%%%%%%%%%%%%%%%%%%%%%%%%%%%%%%%%%%%%
\section{Exercises}

%%%%%%%%%%%%%%%%%
\begin{enumerate}
%%%%%%%%%%%%%%%%%

%%%%%%% 1
\begin{excopy}
Prove the \(\Delta\)-system lemma (1.5) directly. 
\emph{Hint}. One may assume that for
some $n$, \(\forall x \in \scrA (|x| =n)\). 
Now proceed by induction on $n$. Observe that
there is an uncountable \(\scrB \subset \scrA\) such that either 
\(\cap \scrB \neq \emptyset\) or \scrB\ is pairwise
disjoint.
\end{excopy}

The Theorem~1.5 says:
\begin{quote}
If \scrA\ is any uncountable family of finite sets, there is an uncountable
\(\scrB \subset \scrA\) which forms a \(\Delta\)-system.
\end{quote}
Put \(\scrA_n = \{A \in \scrA: |A|=n\}\).
Now the must exists some $n$ such that \(\scrA_n\) is uncountable.

Assume \(n=1\). Then take \(\scrB = \scrA_1\) 
and for each distinct \(B_1,B_2 \in scrB\), \(B_1\cap B_2 = \emptyset\)
thus \scrB\ is \(\Delta\)-system.

Induction step.
Assume that if all sets in \scrA\ are of the same size $s$ and \(s<n\)
then there exists \(\scrB_s\subset \scrA\) which is \(\Delta\)-system.
and assumee now that \(\scrA_n\) is uncountable.
Put \(R = \cup \scrA_n\) and well-order $R$.

\unfinished


\end{enumerate}
