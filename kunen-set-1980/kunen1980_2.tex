%%%%%%%%%%%%%%%%%%%%%%%%%%%%%%%%%%%%%%%%%%%%%%%%%%%%%%%%%%%%%%%%%%%%%%%%
%%%%%%%%%%%%%%%%%%%%%%%%%%%%%%%%%%%%%%%%%%%%%%%%%%%%%%%%%%%%%%%%%%%%%%%%
%%%%%%%%%%%%%%%%%%%%%%%%%%%%%%%%%%%%%%%%%%%%%%%%%%%%%%%%%%%%%%%%%%%%%%%%
\chapterTypeout{II. Infinitary combinatorics}

%%%%%%%%%%%%%%%%%%%%%%%%%%%%%%%%%%%%%%%%%%%%%%%%%%%%%%%%%%%%%%%%%%%%%%%%
%%%%%%%%%%%%%%%%%%%%%%%%%%%%%%%%%%%%%%%%%%%%%%%%%%%%%%%%%%%%%%%%%%%%%%%%
\section{Notes}

%%%%%%%%%%%%%%%%%%%%%%%%%%%%%%%%%%%%%%%%%%%%%%%%%%%%%%%%%%%%%%%%%%%%%%%%
\subsection{Theorem 1.3}

In the proof of Theorem~1.3
\begin{equation*}
  A_X = \{X\cap \alpha: \alpha<\kappa\}.
\end{equation*}
It claims: If \(|X|=\kappa\) then \(|A_X|=\kappa\). Why?\\
For each distinct \(\alpha,\beta < \kappa\)
we \emph{cannot} simply show \(X\cap \alpha \neq X\cap\beta\), since
if we pick \(X' = X \setminus (\alpha \cup \beta)\)
we still have \(|X'|=\kappa\) but
\(X' = X'\cap \alpha = X' \cap \beta\).

We enumerate \(X=\{x_\beta: \beta < \kappa\}\).
For each \(\beta < \kappa\), we have \(\{x\in X: x< x_\beta\} = X\cap \alpha\)
for some \(\alpha\) (not necessarily unique \(\alpha\) for \(\beta\).
So the families:
\begin{itemize}
\item \(\{x\in X: x< x_\beta\}\) for each \(\beta < \kappa\)
\item \(X\cap \alpha\) for each \(\alpha < \kappa\)
\end{itemize}
are the same, and so \(|X|=\kappa\).

%%%%%%%%%%%%%%%%%%%%%%%%%%%%%%%%%%%%%%%%%%%%%%%%%%%%%%%%%%%%%%%%%%%%%%%%
\subsection{Theorem 1.6}

\iffalse
\begin{quote}
1.6. THEOREM. 
Let \(\kappa\) be any infinite cardinal. 
Let \(\theta > \kappa\) be regular and satisfy
\(\forall \alpha < \theta (|\alpha^{-\kappa}| < \theta\).
Assume \(|A| \geq \theta\) and \(\forall x\in A (|x|<\kappa)\),
then there is a
\(B \subset A\), such that \(|B| = \theta\) and $B$ forms a \(\Delta\)-system,
namely \(\exists r, \forall x,y\in B (x\neq y \implies x\cap y = r\).

Proof. By shrinking $A$ if necessary, we may assume \(|A| = \theta\). Then
\(|\cup A| \leq \theta\). Since what the elements of $A$ are as individuals
is irrelevant, we
may assume \(\cup A \subset \theta\). 
  Then each \(x \in A\)/ has some order type \(<\kappa\) as a subset
of \(\theta\), Since \(\theta\) is regular and \(\theta > \kappa\), 
  there is some \(\rho < \kappa\), such that 
\(A_1 = \{x\in A: x \;\textnormal{has type}\; \rho\}\) has cardinality \(\theta\). 
  We now fix such a \(\rho\) and deal only with \(A_1\)

For each \(a <\theta\), \(|\alpha^{<\kappa}| < \theta\)
  implies that less than \(\theta\) elements of \(A_1\) are
subsets of \(\alpha\).
  Thus, \(\cup A_1\) is unbounded in \(\theta\). 
If \(x\in A_1\) and \(\xi < \rho\), let \(x(\xi)\)
be the \(\xi\)-th element of $x$. 
  Since \(\theta\) is regular, there is some \(\xi\) such that
\(\{x(\xi): x\in A_1\}\) is unbounded in \(\theta\). 
  Now fix \(\xi_0\) to be the least such \(\xi\) 
  (\(\xi_0\) may be $0$). Let
\begin{equation*}
\alpha_0  = \sup\{x(\eta) + 1: x\in A_1 \land \eta < \xi_o\};
\end{equation*}
then \(\alpha_0 < \theta\) and \(x(\eta) < \alpha_0\) 
  for all \(x \in A_1\) and all \(n < \xi_0\).
By transfinite recursion on \(\mu < \theta\),
  pick \(x_\mu \in A_1\) sO that \(x_\mu(\xi_0) > \alpha_0\) and
\(x_\mu(\xi_0)\) is above all elements of earlier \(x_\nu\); i.e.
\begin{equation*}
X_\mu(\xi_0) > \max(\alpha_0, \sup \{x_\nu(\eta): \eta<\rho \land \nu<\mu\}).
\end{equation*}
Let \(A_2 = \{x_\mu: \mu < \theta\}\). 
  Then \(|A_2| = \theta\) and \(x\cap y \subset \alpha_0\) whenever $x$ and $y$
are distinct elements of \(A_2\). Since \(|\alpha*{<\kappa}| < \theta\), 
  there is an \(r\subset \alpha_0\) and a
\(B \subset A_2\)of, with \(|B| = \theta\) 
  and \(\forall x \in B (x \cap \alpha_0 = r)\),
  whence $B$ forms a \(\Delta\)-system with root $r$.
\end{quote}
\fi

\subsubsection{Cardinality \(\theta\)}
In the proof of Theorem~1.6 it says:
\begin{quote}
Since \(\theta\) is regular and \(\theta > \kappa\), 
there is some \(\rho < \kappa\), such that
 \(\scrA_1 = \{x\in\scrA: x \;\textnormal{has type}\; \rho\}\)
has cardinality \(\theta\).
\end{quote}
To Justify:
\begin{equation*}
\scrA = \mathsmaller{\Disjunion_{\rho<\kappa}} 
  \{x\in\scrA: x \;\textnormal{has type}\; \rho\}.
\end{equation*}

\subsubsection{Subsets Estimation}

In the proof of Theorem~1.6 it also says:
\begin{quote}
For each \(\alpha < \theta\), \(|\alpha^{<\kappa}| < \theta\)
implies that less than \(\theta\) elements of \(\scrA_1\) are
subsets of \(\alpha\) Thus, \(\cup \scrA_1\) is unbounded in \(\theta\).
\end{quote}
Estimation of subsets is justified by:
\begin{equation*}
\left|\alpha^{<\kappa}\right| 
  = \left|\cup_{\gamma<\kappa}\alpha^\gamma\right|
  \geq \left|\cup_{\gamma<\min(\alpha,\kappa)}\alpha^\gamma\right|
  \geq \left|\cup_{\gamma<\min\alpha,\kappa)} 
    \{\rng(f): f\in \alpha^\gamma\}\right|
  = \left|\{\alpha': \alpha'\subseteq \alpha\}\right|.
\end{equation*}
It is bounded, since \(\theta\) is a limit ordinal, 
\(|\cup \scrA_1|=\theta\) and \(\forall x\in\scrA (|x|<\theta\).

\subsection{Final existence of $r$ and \scrB}

The proof of Theorem~1.6 ends with
\begin{quote}
Since \(|\alpha_0^{<\kappa}| < \theta\)
there is an \(r \subset \alpha_0\) and a
\(\scrB \subset \scrA_2\) with \(|\scrB| = \theta\)
  and \(\forall x \in \scrB (x \cap \alpha_0 = r)\)
\end{quote}

\B{Justifyication:} \\
\(x_\mu(\xi_0)> \alpha_0\) and is above all elements of earlier \(x_\nu\).
Hence \(x\cap y\subset\alpha_0\) for distinct \(x,y\in\scrA_2\).
\\
\(|\scrA_2|=\theta\), so among \(\{x\cap \alpha_0:x\in\scrA_2\)
(Whose size \(\leq \alpha_0\)
there there are \(\theta\) with intersection to some $r$.
So we pick such \(\scrB \subset \scrA_2\)
so \(|\{x\cap \alpha_0:x\in\scrB|=\theta\)
and \(\forall x\in\scrB (r = x\cap \alpha)\).
\\
By the transfinite construction of \(x_\mu\)
\(x\cap y = r\) for all distinct \(x,y\in\scrB\),
since they differ above \(x(\xi_0)\) and \(y(\xi_0)\).

%%%%%%%%%%%%%%%%%%%%%%%%%%%%%%%%%%%%%%%%%%%%%%%%%%%%%%%%%%%%%%%%%%%%%%%%
\subsection{Definition 2.5 of \(\MA(\kappa)\)}

In the definition \(\MA(\kappa)\)
why can't we always simply take filter \(G=\bbP\)?\\
Because it may fail the \ich{a} condition.

%%%%%%%%%%%%%%%%%%%%%%%%%%%%%%%%%%%%%%%%%%%%%%%%%%%%%%%%%%%%%%%%%%%%%%%%
\subsection{Example 5 \(\bbP\)}

Example 5 defines
\begin{equation} \label{eq:example5}
\bbP = \{p: p \subset \omega\times 2 \land |p|<\omega 
  \land p \;\textnormal{is a function}\}.
\end{equation}

and says: \bbP\ clearly has \ccc. \\
\emph{why?:} Simply because there are \(\aleph_0\) finite subsets of \(\omega\).
and there could be at most countable functions from 
such finite subsets to $2$.

%%%%%%%%%%%%%%%%%%%%%%%%%%%%%%%%%%%%%%%%%%%%%%%%%%%%%%%%%%%%%%%%%%%%%%%%
\subsection{Lemma 2.6(b)}

The proof of Lemma~2.6 says: ``\ich{b} was just done''.
But in the preceding text, the discussion deals with 
a specific \bbP\ of \eqref{eq:example5}.
To show that \(\MA(2^\omega)\) is false, it is sufficient
to show failure for some case.

%%%%%%%%%%%%%%%%%%%%%%%%%%%%%%%%%%%%%%%%%%%%%%%%%%%%%%%%%%%%%%%%%%%%%%%%
\subsection{Lemma 2.7}

The lemma says:\\
Let \(e_i=\lrangle{S_i,F_i}\) for \(i=1,2\). Put
\begin{equation*}
R(i) = \forall x\in F_i(x \cap s_{3-i} \leq s_i) \qquad (i=1,2).
\end{equation*}
Show \(e_1\|e_2\) iff \(R(1) \land R(2)\)
with \(\lrangle{s_1\cup s_2,F_1\cup F_2}\) a common extension.

we need to show (for \(i=1,2\)) that (here for \(i=1\))
\(\forall x \in F_1\;(x\cap (s_1\cup s_2) \subset s_1)\).

%%%%%%%%%%%%%%%%%%%%%%%%%%%%%%%%%%%%%%%%%%%%%%%%%%%%%%%%%%%%%%%%%%%%%%%%
%%%%%%%%%%%%%%%%%%%%%%%%%%%%%%%%%%%%%%%%%%%%%%%%%%%%%%%%%%%%%%%%%%%%%%%%
\section{Exercises}

%%%%%%%%%%%%%%%%%
\begin{enumerate}
%%%%%%%%%%%%%%%%%

%%%%%%% 1
\begin{excopy}
Prove the \(\Delta\)-system lemma (1.5) directly. 
\emph{Hint}. One may assume that for
some $n$, \(\forall x \in \scrA (|x| =n)\). 
Now proceed by induction on $n$. Observe that
there is an uncountable \(\scrB \subset \scrA\) such that either 
\(\cap \scrB \neq \emptyset\) or \scrB\ is pairwise
disjoint.
\end{excopy}

The Theorem~1.5 says:
\begin{quote}
If \scrA\ is any uncountable family of finite sets, there is an uncountable
\(\scrB \subset \scrA\) which forms a \(\Delta\)-system.
\end{quote}
Put \(\scrA_n = \{A \in \scrA: |A|=n\}\).
Now the must exists some $n$ such that \(\scrA_n\) is uncountable.

Assume \(n=1\). Then take \(\scrB = \scrA_1\) 
and for each distinct \(B_1,B_2 \in scrB\), \(B_1\cap B_2 = \emptyset\)
thus \scrB\ is \(\Delta\)-system.

Induction step.
Assume that if all sets in \scrA\ are of the same size $s$ and \(s<n\)
then there exists \(\scrB_s\subset \scrA\) which is \(\Delta\)-system.
and assumee now that \(\scrA_n\) is uncountable.
Put \(R = \cup \scrA_n\) and well-order $R$.

\unfinished


\end{enumerate}
