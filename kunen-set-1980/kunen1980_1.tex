%%%%%%%%%%%%%%%%%%%%%%%%%%%%%%%%%%%%%%%%%%%%%%%%%%%%%%%%%%%%%%%%%%%%%%%%
%%%%%%%%%%%%%%%%%%%%%%%%%%%%%%%%%%%%%%%%%%%%%%%%%%%%%%%%%%%%%%%%%%%%%%%%
%%%%%%%%%%%%%%%%%%%%%%%%%%%%%%%%%%%%%%%%%%%%%%%%%%%%%%%%%%%%%%%%%%%%%%%%
\chapterTypeout{I. The foundation of set theory}

%%%%%%%%%%%%%%%%%%%%%%%%%%%%%%%%%%%%%%%%%%%%%%%%%%%%%%%%%%%%%%%%%%%%%%%%
%%%%%%%%%%%%%%%%%%%%%%%%%%%%%%%%%%%%%%%%%%%%%%%%%%%%%%%%%%%%%%%%%%%%%%%%
\section{Notes}

%%%%%%%%%%%%%%%%%%%%%%%%%%%%%%%%%%%%%%%%%%%%%%%%%%%%%%%%%%%%%%%%%%%%%%%%
\subsection{Lemma 7.5}

The following 
\begin{quote}
  7.5 \textsc{Lemma.} If $A$ is a set of ordinals
  and \(\forall x\in A\, \forall y\in x\, (y\in A)\),
  then $A$ is an ordinal.
\end{quote}
is given without proof.
\begin{proof}
Let \(x\in A\).
Define \(Y=\{y\in A: y\in x\}\)
Since \(orall y\in x\, (y\in A)\) we have \(x=Y\subset A\).
Hence $A$ is transitive.
By \textsc{Theorem}~7.3(3) the \(\in\) is a total ordering.
If \(A\neq\emptyset\) then by the foundation axiom (see \S7 \textsc{Axiom 2}),
applying $A$ to $x$, there exists \(y\in X\)
which is \(\in\)-minimal, that is there is no \(z\in A\cap y\).
Hence $A$ is well-ordered.
With these two conclusions, the requirements of \textsc{Definition}~7.1
are met and thus $A$ is an ordinal.
\end{proof}
