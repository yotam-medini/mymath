%%%%%%%%%%%%%%%%%%%%%%%%%%%%%%%%%%%%%%%%%%%%%%%%%%%%%%%%%%%%%%%%%%%%%%%%
%%%%%%%%%%%%%%%%%%%%%%%%%%%%%%%%%%%%%%%%%%%%%%%%%%%%%%%%%%%%%%%%%%%%%%%%
%%%%%%%%%%%%%%%%%%%%%%%%%%%%%%%%%%%%%%%%%%%%%%%%%%%%%%%%%%%%%%%%%%%%%%%%
\chapterTypeout{I. The foundation of set theory}

%%%%%%%%%%%%%%%%%%%%%%%%%%%%%%%%%%%%%%%%%%%%%%%%%%%%%%%%%%%%%%%%%%%%%%%%
%%%%%%%%%%%%%%%%%%%%%%%%%%%%%%%%%%%%%%%%%%%%%%%%%%%%%%%%%%%%%%%%%%%%%%%%
\section{Notes}

%%%%%%%%%%%%%%%%%%%%%%%%%%%%%%%%%%%%%%%%%%%%%%%%%%%%%%%%%%%%%%%%%%%%%%%%
\subsection{Lemma 7.5}

The following 
\begin{quote}
  7.5 \textsc{Lemma.} If $A$ is a set of ordinals
  and \(\forall x\in A\, \forall y\in x\, (y\in A)\),
  then $A$ is an ordinal.
\end{quote}
is given without proof.
\begin{proof}
Let \(x\in A\).
Define \(Y=\{y\in A: y\in x\}\)
Since \(orall y\in x\, (y\in A)\) we have \(x=Y\subset A\).
Hence $A$ is transitive.
By \textsc{Theorem}~7.3(3) the \(\in\) is a total ordering.
If \(A\neq\emptyset\) then by the foundation axiom (see \S7 \textsc{Axiom 2}),
applying $A$ to $x$, there exists \(y\in X\)
which is \(\in\)-minimal, that is there is no \(z\in A\cap y\).
Hence $A$ is well-ordered.
With these two conclusions, the requirements of \textsc{Definition}~7.1
are met and thus $A$ is an ordinal.
\end{proof}


%%%%%%%%%%%%%%%%%%%%%%%%%%%%%%%%%%%%%%%%%%%%%%%%%%%%%%%%%%%%%%%%%%%%%%%%
%%%%%%%%%%%%%%%%%%%%%%%%%%%%%%%%%%%%%%%%%%%%%%%%%%%%%%%%%%%%%%%%%%%%%%%%
\section{Exercises}

%%%%%%%%%%%%%%%%%%%%%%%%%%%%%%%%%%%%%%%%%%%%%%%%%%%%%%%%%%%%%%%%%%%%%%%%
\begin{enumerate}

%%%%%%% 1
\begin{excopy}
Write a formula expressing
\(z = \lrangle{\lrangle{x, y}, \lrangle{v, w}}\).
using just \(\in\) and $=$.
\end{excopy}

Let's start with a simpler challenge, expressing \(z = \lrangle{x, y}\),
meaning \(z = \{x, \{x,y\}\}\).
\begin{equation*}
 a \in z\;\iff\; (a=x \lor (b \in a \iff ((b=x) \lor (b=y)))
\end{equation*}
\unfinished

%%%%%%% 2
\begin{excopy}
Show that
\(\alpha < \beta\) implies that \(\gamma + \alpha < \gamma + \beta\)
and \(\alpha + \gamma \leq \beta + \gamma\).
Give
 example to show that the \(\leq\) cannot be replaced by $<$. Also, show:
\begin{equation*}
 \alpha \leq \beta \,\implies\, \exists!\delta(\alpha + \delta = \beta).
\end{equation*}
\end{excopy}
By definition and by \textsc{Theorem}~6.3
\(\exists y\in\beta\,(\alpha \simeq \pred(\beta,y)\).
So we have unique isomorphism
\(\varphi: \alpha \to \pred(\beta,y)\).
Following \textsc{Definition}~7.17
define
\begin{align*}
\tilde{\varphi} &:
  \type\left(\gamma\times\{0\}\cup \alpha\times\{1\}\right)
  \to
  \type\left(\gamma\times\{0\}\cup \beta\times\{1\}\right) \\
\tilde{\varphi}(x) &=
  \left\{
  \begin{array}{ll}
    x &\quad x\in \gamma\times\{0\} \\
    \lrangle{\varphi(a),1} &\quad x \in\alpha\times\{1\}\,\land\,x=\lrangle{a,1}
  \end{array}
  \right.
\end{align*}
Clearly \(\tilde{\varphi}\) is injective but not surjective
since \(\varphi\) is not. Thus \(\gamma + \alpha < \gamma + \beta\).

If \(\alpha\leq\beta\) then \(\alpha \subsetneq \beta\).
Put \(d=\beta \setminus \alpha\).
Now $d$ is well ordered and \(\delta = \type(d)\)
satisfies the \(\alpha + \delta = \beta\) equality.
If \(\alpha + \eta = \beta\)
then there is an isomorphism \(\varphi: \alpha + \delta \to \alpha + \eta\),
By lookin at \textsc{Definition}~7.16 we can ``reduce''
\(\varphi\) to \(\delta \to \eta\) and by \textsc{Theorem}~7.3(2)
\(\eta = \delta\), thus uniqueness of \(\delta\) was shown.

If \(\alpha=0\), \(\beta=1\) and \(\gamma=\omega\)
then \(\alpha+\gamma = 0+\omega = \omega = 1 + \omega = \beta+\gamma\).
So \(\leq\) is needed.

% we \(\alpha \in \beta\) and by 
% transitivity
% \(\alpha \subset\) and again by definition \(\alpha \subsetneq \beta\).
% So there exists a unique injective mapping \(\phi:\alpha\to\beta\).
\unfinished

%%%%%%% N
\begin{excopy}
\end{excopy}
\unfinished


%%%%%%%%%%%%%%%%%%%%%%%%%%%%%%%%%%%%%%%%%%%%%%%%%%%%%%%%%%%%%%%%%%%%%%%%
\end{enumerate}

