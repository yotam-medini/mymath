% $Id$
% To be included by [pdf]LaTeX main document
%
%%%%%%%%%%%%%%%%%%%%%%%%%%

\usepackage{fullpage}

% are we in pdftex ????
\ifx\pdfoutput\undefined % We're not running pdftex
\else
\RequirePackage[colorlinks,hyperindex,plainpages=false]{hyperref}
\def\pdfBorderAttrs{/Border [0 0 0] } % No border arround Links
\fi

% \usepackage{fancyheadings}
\usepackage{fancyhdr}
\usepackage{pifont}

\pagestyle{fancy}
% \addtolength{\headwidth}{\marginparsep}
% \addtolength{\headwidth}{\marginparwidth}
%  \addtolength{\textheight}{2pt}
\newcommand{\srightmark}{\rightmark}
\newcommand{\sfbfpg}{\sffamily\bfseries{\thepage}}
  \newcommand{\symenvelop}{%
     {\nullfont a}\relax\lower.2ex\hbox{\large\Pisymbol{pzd}{41}}}
% \renewcommand{\chaptermark}[1]{\markboth{\thechapter.\ #1}}

\iffalse
% \lhead[\fancyplain{}{{\sfbfpg}}]{\fancyplain{}\bfseries\srightmark}
\lhead[\fancyplain{}{{\sfbfpg}}]{\fancyplain{}\sl\srightmark}
% \rhead[\fancyplain{}\bfseries\leftmark]{\fancyplain{}{{\sfbfpg}}}
\rhead[\fancyplain{}\sl\leftmark]{\fancyplain{}{{\sfbfpg}}}
\lfoot{\today}
\cfoot{Yotam Medini \copyright}
  \newcommand{\symenvelop}{%
     {\nullfont a}\relax\lower.2ex\hbox{\large\Pisymbol{pzd}{41}}}
\rfoot{\symenvelop\ \texttt{yotam\_medini@yahoo.com}}

\renewcommand{\headrulewidth}{0.4pt}
\renewcommand{\footrulewidth}{0.4pt}
\fi

\fancyplain{plain}{%
 \fancyhf{}
 \fancyhead[LE,RO]{\fancyplain{}{{\sfbfpg}}}
 \fancyhead[RE,LO]{\sl\leftmark}
 \fancyfoot[C]{Yotam Medini \copyright}
 \fancyfoot[R]{\symenvelop\ \texttt{yotam\_medini@yahoo.com}}
 \renewcommand{\headrulewidth}{0.4pt}
 \renewcommand{\footrulewidth}{0.4pt}
}


% \usepackage{amstex}
\usepackage{amsmath}
\usepackage{amssymb}
\usepackage{amsthm}
\usepackage{bm}
% \usepackage{amsthm}
\usepackage{makeidx}
\makeindex % enable

% \usepackage{multicol,supertabular}

\setlength{\parindent}{0pt}
\setlength{\parskip}{6pt}

% \usepackage{amsmath}

% 'Inspired' by:
%% This is file `uwamaths.sty',
%%%     author   = "Greg Gamble",
%%%     email     = "gregg@csee.uq.edu.au (Internet)",

\makeatletter
\def\DOTSB{\relax}
\def\dotcup{\DOTSB\mathop{\overset{\textstyle.}\cup}}
 \def\@avr#1{\vrule height #1ex width 0pt}
 \def\@dotbigcupD{\smash\bigcup\@avr{2.1}}
 \def\@dotbigcupT{\smash\bigcup\@avr{1.5}}
 \def\dotbigcupD{\DOTSB\mathop{\overset{\textstyle.}\@dotbigcupD%
                               \vphantom{\bigcup}}}

\def\dotbigcupT{\DOTSB\smash{\mathop{\overset{\textstyle.}\@dotbigcupT%
                              \vphantom{\bigcup}}}%
                       \vphantom{\bigcup}\@avr{2.0}}
\def\dotbigcup{\mathop{\mathchoice{\dotbigcupD}{\dotbigcupT}
                                  {\dotbigcupT}{\dotbigcupT}}}
\let\disjunion\dotcup
\let\Disjunion\dotbigcup
\makeatother

%%%%%%%%%%%%%%%%%%%%%%%%%%%%%%%%%%%%%%%%%%%%%%%%%%%%
% Column type: SeeThe LaTeX Companion 5.3.3 page 112
\usepackage{array}
\newcolumntype{C}{>{$}c<{$}}
\newcolumntype{L}{>{$}l<{$}}
\newcolumntype{R}{>{$}r<{$}}

\usepackage{moreverb}

%%%%%%%%%%%%%%%%
%% Abbreviations
%%
\newcommand{\wlogy}{without loss of generality}
\newcommand{\Wlogy}{Without loss of generality}

\title{
 Notes and Solutions to Exercises\\
 for\\
 ``Galois Theory'' \\
 by \quad Joseph Rotman}

\author{Yotam Medini\\\texttt{yotam\_medini@yahoo.com}}

\newcommand{\A}{\ensuremath{\mathbb{A}}} % The Algebraic numbers set
\newcommand{\C}{\mathbb{C}} % The Complex set
\newcommand{\R}{\mathbb{R}} % The Real set
\newcommand{\N}{\ensuremath{\mathbb{N}}} % The Rational set
\newcommand{\Q}{\ensuremath{\mathbb{Q}}} % The Rational set
\newcommand{\Z}{\ensuremath{\mathbb{Z}}} % The Integer Set
\newcommand{\Zn}[1]{\ensuremath{\Z_{#1}}} % The Cyclic group
\newcommand{\Zp}{\ensuremath{\Z_p}} % The Cyclic group
\newcommand{\Zm}[1]{\Z/#1\Z} % The Cyclic group

% Trivial group
\newcommand{\eG}{\ensuremath{\{e\}}}

% Degree of polynomial - as used in Rotman
\newcommand{\gdeg}{\partial}
\newcommand{\half}{\frac{1}{2}}


% sequences
\newcommand{\seq}[2]{\ensuremath{#1_1,\ldots,#1_{#2}}}
\newcommand{\seqn}[1]{\seq{#1}{n}}
\newcommand{\seqan}{\seq{a}{n}}
\newcommand{\seqxn}{\seq{x}{n}}
\newcommand{\seqalphn}{\seq{\alpha}{n}}

% \newcommand{\disjunion}{\.\cup}      % Some use \sqcup or \uplus
% \newcommand{\Disjunion}{\.\bigsqcup} % Some use \bigsqcup or \biguplus
% \newcommand{\disjunion}{{\bigsqcup}}
% \newcommand{\Disjunion}{\bigsqcup}

%%%%%%%%%%%%
%% math op's
%%
\def\Aut{\mathop{\rm Aut}\nolimits}
\def\card{\mathop{\rm card}\nolimits}
\def\Ch{\mathop{\rm Ch}\nolimits}
% \def\char{\mathop{\rm char}\nolimits}
\def\dim{\mathop{\rm dim}\nolimits}
\def\fchar{\mathop{\rm char}\nolimits}
\def\Frac{\mathop{\rm Frac}\nolimits}
\def\Gal{\mathop{\rm Gal}\nolimits}
\def\GF{\mathop{\rm GF}\nolimits}
\def\gcd{\mathop{\rm gcd}\nolimits}
\def\Id{\mathop{\rm Id}\nolimits}
\def\id{\mathop{\rm id}\nolimits}
\def\Im{\mathop{\rm Im}\nolimits}
\def\im{\mathop{\rm im}\nolimits}
\def\Inn{\mathop{\rm Inn}\nolimits}
\def\Irr{\mathop{\rm Irr}\nolimits}
\def\Ker{\mathop{\rm Ker}\nolimits}
\def\ker{\mathop{\rm ker}\nolimits}
\def\lcm{\mathop{\rm lcm}\nolimits}
\def\Map{\mathop{\rm Map}\nolimits}
\def\Stab{\mathop{\rm Stab}\nolimits}
\def\Vert{\mathop{\rm Vert}\nolimits}
\def\subnormal{\vartriangleleft}

% abbreviations, ensuremath
\newcommand{\fx}{\ensuremath{f(x)}}
\newcommand{\gx}{\ensuremath{g(x)}}
\newcommand{\lrangle}[1]{\ensuremath{\langle #1 \rangle}}
\newcommand{\mldots}{\ensuremath{\ldots}}
\newcommand{\px}{\ensuremath{p(x)}}

\newenvironment{excopy}
{\begin{minipage}[t]{.8\textwidth}\footnotesize}
{\smallskip\hrule\end{minipage}}


\newcounter{myenumi}
\newenvironment{myenumerate}
{\begin{enumerate}
 \setcounter{enumi}{\themyenumi}
 \addtolength{\itemsep}{10pt}
}
{\setcounter{myenumi}{\theenumi}
 \end{enumerate}}

% End of proof
% \newcommand{\eop}{{\small\quad\(\square\)}}

%%%%%%%%%%%
%% Theorems
%%
\newtheorem{thm}{Theorem}[chapter]
\newtheorem{cor}[thm]{Corollary}
\newtheorem{Def}{Definition}
\newtheorem{lem}[thm]{Lemma}
\newtheorem{llem}[thm]{Local Lemma}
\newtheorem{lthm}[thm]{Local Theorem}

% \newtheorem{quotecor}{Corollary}
\newtheorem{quotelem}{Lemma}[section]
\newtheorem{quotethm}{Theorem}
\newtheorem{quotelems}{Lemma}[section]

% \newcommand{\proofend}{\(\bullet\)}
\newcommand{\proofend}{\hfill\(\blacksquare\)}

\newcommand{\chapterTypeout}[1]{\typeout{#1} \chapter{#1}}
\newcommand{\sectionTypeout}[1]{\typeout{#1} \section{#1}}

\begin{document}
\maketitle
\newpage
\tableofcontents
\newpage

%%%%%%%%%%%%%%%%%%%%%%%%%%%%%%%%%%%%%%%%%%%%%%%%%%%%%%%%%%%%%%%%%%%%%%%%
%%%%%%%%%%%%%%%%%%%%%%%%%%%%%%%%%%%%%%%%%%%%%%%%%%%%%%%%%%%%%%%%%%%%%%%%
%%%%%%%%%%%%%%%%%%%%%%%%%%%%%%%%%%%%%%%%%%%%%%%%%%%%%%%%%%%%%%%%%%%%%%%%
\setcounter{chapter}{-1}
\chapter{Introduction}
\typeout{Introduction}

This document is a companion to the book:
\begin{center}
\textbf{Galois Theory}\\
by\\
Joseph Rotman's (\texttt{rotman@math.uiuc.edu}).
\cite{Rotman98}.
\end{center}
For me, this is the first text on Galois Theory that is so focused
and yet rigorous that I succeed to discipline myself
to use for self study.
I owe a lot to the beautiful exposition and deep remarks.

I use the following document for the following purposes:
\begin{itemize}
 \item Place for making myself do the exercises which are
       essential part of the text. Some proofs are dependent on them.
 \item Errata. Also meant to be sent to the author.
 \item Suggestions.  Clarifications and simplifications.
       Also meant to be sent to the author.
 \item Notes, hints and ideas guiding myself where I got (and still getting)
       lost.
\end{itemize}
Of course, some of my suggestions and claiming for simplifications
may result by me missing a point. If this is the case
please educate me, so I could remove them and more important,
understand the real idea I have missed.

%%%%%%%%%%%%%%%%%%%%%%%%%%%%%%%%%%%%%%%%%%%%%%%%%%%%%%%%%%%%%%%%%%%%%%%%
%%%%%%%%%%%%%%%%%%%%%%%%%%%%%%%%%%%%%%%%%%%%%%%%%%%%%%%%%%%%%%%%%%%%%%%%
\section{Notes}

The following notes apply for the book as a whole.

%%%%%%%%%%%%%%%%%%%%%%%%%%%%%%%%%%%%%%%%%%%%%%%%%%%%%%%%%%%%%%%%%%%%%%%%
\subsection{Hint Style}

Frequently, exercises are followed by hints.
There are two styles:
\begin{itemize} % do:    grep -n Hint rotgal.tex
 \item  \mldots (Hint: \mldots)\\
        as in exercises: 6, .
 \item  \mldots (Hint. \mldots)\\
        as in exercises: 9, .
\end{itemize}
It is desired to use a consistent style.

%%%%%%%%%%%%%%%%%%%%%%%%%%%%%%%%%%%%%%%%%%%%%%%%%%%%%%%%%%%%%%%%%%%%%%%%
%%%%%%%%%%%%%%%%%%%%%%%%%%%%%%%%%%%%%%%%%%%%%%%%%%%%%%%%%%%%%%%%%%%%%%%%
%%%%%%%%%%%%%%%%%%%%%%%%%%%%%%%%%%%%%%%%%%%%%%%%%%%%%%%%%%%%%%%%%%%%%%%%
\chapterTypeout{Symmetry}

%%%%%%%%%%%%%%%%%%%%%%%%%%%%%%%%%%%%%%%%%%%%%%%%%%%%%%%%%%%%%%%%%%%%%%%%
%%%%%%%%%%%%%%%%%%%%%%%%%%%%%%%%%%%%%%%%%%%%%%%%%%%%%%%%%%%%%%%%%%%%%%%%
\section{Notes}

Page~5 the proof of Lemma~2:
\index{polygon}
\begin{quote}
But if \(V'\) is not a vertex, then \(\angle M'V'N'=180^{\circ}\),
a contradiction.
\end{quote}
Seems like more argumentation is needed. Since it is not clear
at this stage that \(M'\) and \(N'\) are in the perimeter.
Thus at this point it is not clear that \(\angle M'V'N'=180^{\circ}\).




%%%%%%%%%%%%%%%%%%%%%%%%%%%%%%%%%%%%%%%%%%%%%%%%%%%%%%%%%%%%%%%%%%%%%%%%
%%%%%%%%%%%%%%%%%%%%%%%%%%%%%%%%%%%%%%%%%%%%%%%%%%%%%%%%%%%%%%%%%%%%%%%%
\section{Exercises (page 7)}

%%%%%%%%%%%%%%%%
\begin{myenumerate}
% \addtolength{\itemsep}{10pt}

%%%%%
\item
\begin{excopy}
\begin{itemize}
 \item[(i)]
   If $F$ is a square, prove that \(\Sigma(F)\cong D_8\),
   the diehedral group of order $8$.
 \item[(ii)]
   If $F$ is a rectangle that is not a square, prove that
   \(\Sigma(F)\cong V\), where $V$ denotes
   the $4$-group (\(V=\Z_2\times\Z_2\)).
 \item[(iii)]
   Give an example of quadrukaterals $Q$ and $Q'$ with
   \(\Sigma(Q)\cong \Z_2\) and \(\Sigma(Q')=1\).
\end{itemize}
\end{excopy}

\begin{itemize}
 \item[(i)]
   \textbf{Intuitively:}
   Say the square vertices are \(V=(V_i)_{i=1}^4\).
   Then \(V_1\) could be mapped into any of  it could be mapped to
   any of the four vertices $V$, say \(V_i\).
   Now \(V_2\) could be mapped to either
   \(V_{i^{+}}\) or \(V_{i^{-}}\) where
   \(i^{+} = i + 1 - 4\lfloor(i+1)/4\rfloor\) and
   \(i^{-} = i + 3 - 4\lfloor(i+3)/4\rfloor\).
   These \(4\times 2\) choices give \(D_8\).

   \textbf{Formally:}
   Say $F$'s 4 vertices are \(\{(\pm1,\pm1)\}\).
   Look at the following generators of \(\Sigma(F)\)
   \begin{equation}
     \rho = \sqrt{2}/2\left(
      \begin{array}{rl}
       1 & 1 \\
       -1 & 1 \\
      \end{array}
     \right)
     \quad\textrm{and}\quad
     \phi = \left(
      \begin{array}{lr}
       1 & 0 \\
       0 & -1 \\
      \end{array}
     \right)
   \end{equation}
   Now \(\rho^4=\id\), \(\phi^2=\id\) and \(\rho\phi = \phi\rho^3\).
 \item[(ii)]
   Say $F$'s 4 vertices are \(\{(\pm a,\pm b\}\).
   The following are the generators of \(\Sigma(F)\)
   \begin{equation}
     \phi_x = \left(
      \begin{array}{rl}
       -1 & 0 \\
       0  & 1 \\
      \end{array}
     \right)
     \quad\textrm{and}\quad
     \phi_y = \left(
      \begin{array}{lr}
       1 &  0 \\
       0 & -1 \\
      \end{array}
     \right)
   \end{equation}
 \item[(iii)]
    Let $Q$ be a trapezoid with the vertices \(\{(\pm a, 0), (\pm b, 1)\}\).
    Now the symmetries are the identity and \((x,y)\rightarrow (-x,y)\).

    Let $Q$ be (unsymmetric) with vertices \(\{(0,0), (3,0), (2,1), (0,1)\}\).
    It can easily be seen that the edges  are mutually different in length.
\end{itemize}

%%%%%
\item
\begin{excopy}
A polygon is
\index{regular!polygon}
\textbf{regular} if all the angles at its vertices are equal.
Prove that a polygon $P$ is regular if and only if \(\Sigma(P)\)
\index{transitively}
acts transitively on \(\Vert(P)\).
\end{excopy}

Say a polygon is of $n$ vertices.
If it is regular then rotations \(2\pi k/n\) where \(0\leq k < n\)
bring each vertex to any other vertex and thus \(\Sigma(P)\)
acts transitively.
Conversely, assume  \(\Sigma(P)\) acts transitively.
then for any two vertices \(V_i\) and \(V_j\)
there is \(\sigma\in\Sigma(P)\) such that
\(\sigma(V_i)=V_j\). Now let
\begin{eqnarray}
   i^{+} & = & i + 1 - n\lfloor(i+1)/n\rfloor \\
   i^{-} & = & i + (n - 1) - n\lfloor(i+n-1)/n\rfloor
\end{eqnarray}
and now the neighboring vertices
\(V_{i^{+}}\) and \(V_{i^{-}}\) are mapped to
neighboring vertices of \(V_j\).
Hence
\(\angle V_{i^{-}} V_i V_{i^{+}} =
  \angle \sigma(V_{i^{-}})\sigma(V_i)\sigma(V_{i^{+}})\)
and the polygon is regular.

%%%%%
\item
\begin{excopy}
Prove that if \(P_n\) is a regular polygon with $n$ vertices,
then \(\Sigma(P_n)\cong D_{2n}\),
where \(D_{2n}\) is the dihedral group of order \(2n\).
\end{excopy}

Without loss of generality,
let the vertices be \(\{(\cos 2\pi k/n, \sin 2\pi k/n)\}_{k=0}^{n-1}\).
The generators of \(\Sigma(P_n)\) are
\begin{equation}
  \rho = \left(
   \begin{array}{rl}
    \cos 2\pi/n & \sin 2\pi/n \\
    -\sin 2\pi/n  & \cos 2\pi/n \\
   \end{array}
  \right)
  \quad\textrm{and}\quad
  \phi_y = \left(
   \begin{array}{lr}
    1 &  0 \\
    0 & -1 \\
   \end{array}
  \right).
\end{equation}
Now \(\rho^n=\phi^2=\id\) and \(\phi\rho = \rho^{n-1}\phi\).
These equalities determine \(D_{2n}\) upto isomorphism.

%%%%%
\item
\begin{excopy}
Prove that if $F$ is a circular disk then \(\Sigma(F)\) is infinite.
\end{excopy}

Simply:
\begin{equation}
\Sigma(F) =
   \left\{
      \left(
      \begin{array}{rl}
        \cos\theta & \sin\theta \\
       -\sin\theta & \cos\theta \\
      \end{array}
      \right): 0\leq\theta<2\pi
   \right\}
\end{equation}

    %%%%%%%%%%%%%
\end{myenumerate}
%%%%%%%%%%%%%%%%%

%%%%%%%%%%%%%%%%%%%%%%%%%%%%%%%%%%%%%%%%%%%%%%%%%%%%%%%%%%%%%%%%%%%%%%%%
%%%%%%%%%%%%%%%%%%%%%%%%%%%%%%%%%%%%%%%%%%%%%%%%%%%%%%%%%%%%%%%%%%%%%%%%
%%%%%%%%%%%%%%%%%%%%%%%%%%%%%%%%%%%%%%%%%%%%%%%%%%%%%%%%%%%%%%%%%%%%%%%%
\chapterTypeout{Rings}

%%%%%%%%%%%%%%%%%%%%%%%%%%%%%%%%%%%%%%%%%%%%%%%%%%%%%%%%%%%%%%%%%%%%%%%%
%%%%%%%%%%%%%%%%%%%%%%%%%%%%%%%%%%%%%%%%%%%%%%%%%%%%%%%%%%%%%%%%%%%%%%%%
\section{Exercises (pages 12--13)}

%%%%%%%%%%%%%%%%
\begin{myenumerate}


%%%%%
\item
\begin{excopy}
Show that the intersection if any family of subrings $R$ is a subring.
\end{excopy}

The zero and one elements are there.
Closure of additions and multiplication is trivial.

%%%%%
\item
\begin{excopy}
Prove that the
\label{ex:binomial}
\index{binomial theorem}
\textbf{binomial theorem} holds in ant ring $R$: if \(n\geq 1\), then
\begin{equation}\label{eq:binom}
(a+b)^n = \sum {\binom{n}{i}} a^ib^{n-i},
\end{equation}
where \(\binom{n}{i}\)
denotes the binomial coefficient \(n!/(i!(n-i)!)\). (Hint: First prove
that
\begin{equation} \label{eq:binom:nm1}
\binom{n-1}{i-1} + \binom{n-1}{i} = \binom{n}{i}.)
\end{equation}
\end{excopy}

First let us show equation~\ref{eq:binom:nm1}.
\begin{eqnarray*}
\binom{n-1}{i-1} + \binom{n-1}{i}
  & = & (n-1)!/((i-1)!((n-1)-(i-1))!) + \\
  &   & \qquad (n-1)!/(i!((n-1)-i)!) \\
  & = & (n-1)!/((i-1)!(n-i)!) + (n-1)!/(i!((n-i-1)!) \\
  & = & i(n-1)!/i!(n-i)!) + (n-i)(n-1)!/(i!(n-i)!) \\
  & = & (i(n-1)! + (n-i)(n-1)!)/(i!(n-i)!) \\
  & = & (i + (n-i))(n-1)!)/(i!(n-i)!) \\
  & = & n(n-1)!)/(i!(n-i)!) = n!/(i!(n-i)!) \\
  & = & \binom{n}{i}
\end{eqnarray*}

Now equation~\ref{eq:binom} surely holds for \(n=1\).\\
By induction assume it is true for \(n=k-1\geq 1\).
So now
\begin{eqnarray*}
(a+b)^k
  & = & (a+b)(a+b)^{k-1}\\
  & = & (a+b) \sum_{i=0}^{k-1} {\binom{k-1}{i}} a^ib^{k-1-i} \\
  & = & a \sum_{i=0}^{k-1} {\binom{k-1}{i}} a^ib^{k-1-i} +
        b \sum_{i=0}^{k-1} {\binom{k-1}{i}} a^ib^{k-1-i} \\
  & = & \sum_{i=1}^k {\binom{k-1}{i-1}} a^ib^{k-i} +
        \sum_{i=0}^{k-1} {\binom{k-1}{i}} a^ib^{k-i} \\
  & = & 1\cdot a^k + 1\cdot b^k +
        \sum_{i=1}^{k-1}
          \left(\binom{k-1}{i-1} + \binom{k-1}{i}\right)a^ib^{k-i} \\
  & = & a^k + b^k +
        \sum_{i=1}^{k-1} \binom{k}{i}a^ib^{k-i} \\
  & = & \sum_i^k \binom{k}{i}a^ib^{k-i}
\end{eqnarray*}


%%%%%
\item
\begin{excopy}
If $p$ is a prime, prove that $p$ is a divisor of \(\binom{p}{i}\)
 for \(i\neq 0\) and \(i\neq p\).
(Note that $4$ is not a divisor of \(\binom{4}{2}=6\).)
\end{excopy}

We look at the definition
\[\binom{p}{i} = \frac{p!}{i!(p-i)!}.\]
Now clearly $p$ divides the numerator \(p!\) but not the
denominator \(i!(p-i)!\).

%%%%%
\item
\begin{excopy}
If $R$ is any ring and \(f(x)\in R[x]\), say,
\(f(x)=r_0+r_1x+\ldots+r_nx^n\), define is
\index{derivative}
\textbf{derivative} by
\begin{equation}
f'(x) = r_1+2r_2x+\ldots+nr_n x^{n-1}.
\end{equation}
Prove that
\begin{equation} \label{eq:deriv:add}
[f(x)+g(x)]'=f'(x)+g'(x)
\end{equation}
and
\begin{equation} \label{eq:deriv:mult}
[f(x)g(x)]' = f(x)g'(x) +f'(x)g(x).
\end{equation}
\end{excopy}

For notational convinience, we can assume
by filling with zero coefficients that the degrees of $f$ and $g$ agree.
Now let
\(f(x)=\sum_{i=0}^n a_ix^i\) and
\(g(x)=\sum_{i=0}^n b_ix^i\).

Showing (\ref{eq:deriv:add}):

\begin{eqnarray*}
[f(x)+g(x)]'
  & = & \left(\sum_{i=0}^n (a_i+b_i)x^i\right)' \\
  & = & \sum_{i=1}^n (a_i+b_i)ix^{i-1}) \\
  & = & \sum_{i=1}^n a_i i x^{i-1}) +
        \sum_{i=1}^n b_i i x^{i-1}) \\
  & = & f'(x)+g'(x)
\end{eqnarray*}
We show (\ref{eq:deriv:mult}) first when $f$ and $g$ each has single term.
Let \(f(x)=ax^m\) and \(g(x)=bx^n\).
\begin{eqnarray*}
f(x)g'(x) + f'(x)g(x)
 & = & ax^m nbx^{n-1} + max^{m-1}bx^n = (m+n)abx^{m+n-1} \\
 & = & (m+n)abx^{m+n-1} = (abx^{m+n})' \\
 & = & \left(f(x)g(x)\right)'
\end{eqnarray*}

The formula (\ref{eq:deriv:mult}) is additive.
That is, assume it holds for the $4$ pairs of polynomials:
\((f_i(x), g_j(x))_{i,j=1,2}\) then using \ref{eq:deriv:add}
\begin{multline*}
 \left((f_1+f_2)(x)\cdot(g_1+g_2)(x)\right)' =
   \sum_{\substack{i=1,2\\j=1,2}} \left(f_i(x)g_j(x)\right)'
      =   \sum_{\substack{i=1,2\\j=1,2}} f_i(x){g_j}'(x) + {f_i}'(x)g_j(x) \\
   = (f_1+f_2)(x)\cdot(g_1+g_2)'(x) + (f_1+f_2)'(x)\cdot(g_1+g_2)(x)
\end{multline*}

Now by induction on the number of terms in $f$, $g$
equation (\ref{eq:deriv:mult}) holds for all polynomials.


%%%%%
\item
\begin{excopy}
If $R$ is a ring and $S$ is a set, let\(R^S\) denote the set of all
functions \(S\rightarrow R\).
Equip \(R^S\) with the operations of pointwise addition and multiplication;
that is, if \(f,g\,:\,S\rightarrow R\), then
\[f+g:s \mapsto f(s) + g(s),\]
and
\[fg:s \mapsto f(s)g(s).\]
Prove that \(R^S\) is a ring. (Hint. ``Zero'' is the constant function $z$
with \(z(s)=0\) for all \(s\in S\),  and ``one'' is the constant for
$e$ with \(e(s)=1\) for all \(s\in S\)).
\end{excopy}

It is clear that the commutative, associative and distributive rules
for \(R^S\) derived from $R$.
It is else clear that \(f+z=f\) and  \(fe=f\)
for all \(f\in R^S\).

    %%%%%%%%%%%%%
\end{myenumerate}
%%%%%%%%%%%%%%%%%

%%%%%%%%%%%%%%%%%%%%%%%%%%%%%%%%%%%%%%%%%%%%%%%%%%%%%%%%%%%%%%%%%%%%%%%%
%%%%%%%%%%%%%%%%%%%%%%%%%%%%%%%%%%%%%%%%%%%%%%%%%%%%%%%%%%%%%%%%%%%%%%%%
%%%%%%%%%%%%%%%%%%%%%%%%%%%%%%%%%%%%%%%%%%%%%%%%%%%%%%%%%%%%%%%%%%%%%%%%
\chapterTypeout{Domains and Fields}

%%%%%%%%%%%%%%%%%%%%%%%%%%%%%%%%%%%%%%%%%%%%%%%%%%%%%%%%%%%%%%%%%%%%%%%%
%%%%%%%%%%%%%%%%%%%%%%%%%%%%%%%%%%%%%%%%%%%%%%%%%%%%%%%%%%%%%%%%%%%%%%%%
\section{Exercises (pages 16--17)}

%%%%%%%%%%%%%%%%
\begin{myenumerate}

%%%%%
\item
\begin{excopy}
\begin{itemize}
 \item[(i)]
   If $R$ is a ring, prove that \(U(R)\), the set of all units in $R$,
   is a group under multiplication. One calls \(U(R)\) the
   \index{units!group of}
   \textbf{group of units} of $R$.
 \item[(ii)]
   Prove that a ring $R$ is a field if and only if \(R^{\#}=R-0\)
   is a group under multiplication. (Of course, \(U(R)=R^{\#}\) here.)
\end{itemize}
\end{excopy}

\begin{itemize}
 \item[(i)]
   By definition, $R$ is closed under inverse operation.
   Now for any \(u,v\in R\) we have \((uv)^{-1} = v^{-1}u^{-1}\) and so
   $R$ is closed also under multiplication.
 \item[(ii)]
   If $R$ is a field then \(R^{\#}=\) is a multiplicative group.
   Conversely, if \(R^{\#}=\) is a multiplicative group,
   then for any non zero element in $R$ there's an inverse
   in \(R^{\#}\) which is a multiplicative inverse in $R$ and so $R$
   is a field.
\end{itemize}


%%%%%
\item
\begin{excopy}
Prove that if \(a \in \Z\), then \([a]\) is a unit in \(\Z_n\)
if and only if \((a,n)=1\).
Conclude that the group of units,  \(U(\Zn{n})\), has order \(\varphi(n)\),
where \(\varphi\) is
\index{Euler's function}
\textbf{Euler's function}: \(\varphi(1)=1\) and, if \(n>1\), then
\(\varphi(n)=|\{k\in\Z: 1\leq k\leq n \quad\textrm{and}\quad(k,n)=1\}|\).
\end{excopy}

Assume \([a]\) is a unit in \(\Z_n\), then there exists \([b]\in \Z_n\)
such that \([a][b]=[1]\) which means \(ab=1\bmod n\).
Hence there exists an integer $m$ so \(ab+mn=1\) and thus \((a,n)=1\).

Conversely, if \((a,n)=1\) then there are \(b,m\in\Z\) such that
\(ab+mn=1\) and so \(ab=1\bmod n\) and so \([a][b]=[1]\) and \([a]\)
is a unit.

%%%%%
\item
\begin{excopy}
Let \(f(x),\, (g(x) \in R[x]\). Show that the constant term of
\(f(x)g(x)\)  is the product of the constant terms of
\(f(x)\) and of \(g(x)\).
\end{excopy}

Trivial.

%%%%%
\item
\begin{excopy}
\begin{itemize}
 \item[(i)]
   If $R$ is a domain, then the leading coefficient of
   \(f(x)\) and of \(g(x)\). Conclude that if
   \(f(x)\) and \(g(x)\) are nonzero polynomials in \(R[x]\),
   where $R$ is a domain, then
     \[\gdeg(fg)=\gdeg(f) + \gdeg(g).\]
 \item[(ii)]
   Prove that if $R$ is a domain, then \(R[x]\) is also a domain.
 \item[(iii)] \label{ex:2xp1}
   If \(R = \Z_4[x]\), show that \((2x+1)^2=1\).
   Conclude that the formula
   \(\gdeg(fg)=\gdeg(f)+\gdeg(g)\) may be false in \(R[x]\) when $R$
   is not a domain.
 \item[(vi)]
   Show that there is a factorization \(x=f(x)g(x)\) in
   \(R=\Z_4[x]\) in which neither \(f(x)\) nor \(g(x)\) is a constant.

\end{itemize}
\end{excopy}

\begin{itemize}
 \item[(i)]
   Say \(f(x)=(c_0,\ldots,c_m,0,\ldots)\)
   and \(g(x)=(d_0,\ldots,d_n,0,\ldots)\).
   Then is is clear the leading coefficient of
   \(f(x)g(x)\) is
   \[\sum_{i+j=\gdeg(f)+\gdeg(g)} c_i d_j = c_m d_n.\]
   Since $R$ is a domain \(c_m d_n\neq 0\)
   and we have \(\gdeg(fg)=\gdeg(f) + \gdeg(g)\).
 \item[(ii)]
   Say \(f(x), g(x)\) in a domain \(R[x]\) and \(f(x)g(x)=0\).
   If by negation both \(f(x), g(x)\) are nonzero, then
   from (ii) \(f(x)g(x)\) has a nonzero finite degree which is
   a contradiction.
 \item[(iii)]
   In \(\Z_4[x]\) we have \((2x+1)^2=4x^2+4x+1=1\).
   If the formula were true then \(2+2=0\) (as natural numbers!)
   --- a contradiction.
 \item[(vi)]
   Let \(f(x)=2x^2+x\), \(g(x)=2x+1\), then
   in \(\Z_4[x]\) we have \[(f(x)g(x)=(2x^2+x)(2x+1)=4x^3+(2+2)x^2+x=x.\]
\end{itemize}

%%%%%
\item
\begin{excopy}
Define the ring of \textbf{polynomials in two variables} over $R$,
denoted by \(R[x,y]\), as \(A[y]\), where \(A=R[x]\).
Define polynomials in several variables over $R$ by induction, and show that
if $R$ is a domain, then so is \(R[\seqxn]\) (one usually denotes
\(\Frac(F[x_1,\ldots,x_n])\) by \(F(\seqxn)\) when $F$ is a field).
\end{excopy}

By induction let \(n > 1\). The ring \(R[\seqxn]\) of
\textbf{polynomials in $n$ variables} is the ring of polynomials
\(A[x_n]\) where \(A=R[\seq{x}{n-1}]\) (the coefficients).

Assume $R$ is a domain. By induction, assume $A$ is a domain
and by (iii) we have \(R[\seqxn]\).

%%%%%
\item
\begin{excopy}
Let $R$ be a domain, and let \(f,g\in R\) be nonzero elements satisfying
\[f=ug  \qquad\textrm{and}\qquad g=vf,\]
where \(u,v\in R\). Prove that \(uv=1\) and that $u$ and $v$ are units.
\end{excopy}

We have: \(f=ug=uvf\) and so \((uv-1)f=0\). Since $f$ is nonzero
and $R$ is a domain we have \((uv-1)=0\) and thus \(uv=1\)
and so  $u$ and $v$ are units.

%%%%%
\item
\begin{excopy}
\begin{itemize}
 \item[(i)]
   Prove that if $F$ is a field, then the units in \(F[x]\) are the nonzero
   constants.
 \item[(ii)]
   Show that \(\Z_2[x]\) is an infinite ring having exactly one unit.
 \item[(iii)]
   Give an example of a nonconstant polynomial in \(\Z_4[x]\)
   that is a unit.
\end{itemize}
\end{excopy}

\begin{itemize}
 \item[(i)]
    The constants of \(F[x]\) are obviously units.
    Now say \(f(x)\in F[x]\) is a unit with an inverse \(g(x)\).
    Since \(f(x)g(x)=1\) both polynomials are nonzero
    by the degree formula \(\gdeg(f)+\gdeg(g)=0\)
    and so both polynomials and in particular \(f(x)\) is constant.
 \item[(ii)]
    The polynomials \(x^n\) are all different and in \(\Z_2[x]\).
    From (i) the units are the nonzero constant, and in \(\Z_2[x]\)
    only \(f(x)=1\) is such.
 \item[(iii)]
    In exercise \ref{ex:2xp1}~(iii) it was show that \((2x+1)\) is unit
    and actually is its self inverse.
\end{itemize}

%%%%%
\item
\begin{excopy}
\begin{itemize}
 \item[(i)]
   Prove that the
   \index{division algorithm}
   \textbf{division algorithm} for polynomials: If $R$ is a ring,
   if \(f(x),g(x)\in R[x]\), and if the leading coefficient of \(g(x)\) is
   a unit in particular, if \(g(x)\) is monic], then there are polynomials
   \(q(x)\) and \(r(x)\in R[x]\) (\textbf{quotient} and \textbf{reminder}) with
   \[f(x) = q(x)g(x) + r(x)\]
   and either \(r(x)=0\) or \(\gdeg(r)<\gdeg(g)\).
 \item[(ii)]
   If $R$ is a domain, then the quotient and reminder occurring in the
   division algorithm are unique. (There are rings $R$, e.g.\ \(\Z_4\),
   for which the corresponding assertion is false.)
\end{itemize}
\end{excopy}

\begin{itemize}
 \item[(i)]
    Denote
    \(m=\gdeg(f)\) and  \(n=\gdeg(g)\) and the polynomials formal
    representations:\\
    \(f=(a_0,a_1,\ldots,a_m)\) and
    \(g=(b_0,b_1,\ldots,a_n)\).

    First let us show a weaker claim. That is
    if \(m\leq n\) then there exists \(\tilde{q}(x),\tilde{r}(x)\in R[x]\)
    such that \(f(x) = \tilde{q}(x)g(x) + \tilde{r}(x)\)
    and either \(\tilde{r}(x)=0\) or \(\gdeg(\tilde{r})<\gdeg(f)\).
    Let \(c_{m-n}=a_m/b_n.\) and now put \(\tilde{q}(x)=c_{m-n}x^{m-n}\)
    and \(\tilde{r}(x)=f(x)-\tilde{q}(x)g(x)\).
    The  coefficient of power \(x^m\) of  \(\tilde{r}(x)\) is canceled
    and indeed   \(\tilde{r}(x)=0\) or \(\gdeg(\tilde{r})<\gdeg(f)\).

    Now show that if there exist $k$, \(n\leq k<m\) and
    a polynomials \(q_k(x),r_k(x)\in R[x]\)
    such that \(f(x) = q_k(x)g(x) + r_k(x)\) and
    \(r_k(x)=0\) or \(\gdeg(r_k)\leq k\),
    then  there exist \(q_{k-1}(x),r_{k-1}(x)\in R[x]\)
    such that \(f(x) = q_{k-1}(x)g(x) + r_{k-1}(x)\) and
    \(r_{k-1}(x)=0\) or \(\gdeg(r_{k-1})\leq k-1\).

    If \(r_k(x)=0\) or \(\gdeg(r_k)\leq k-1\)
    then we are done, since we can use \(q_k(x),r_k(x)\)
    for  \(q_{k-1}(x),r_{k-1}(x)\).
    Otherwise we have \(\gdeg(r_k)=k\) and use the weaker claim
    with \(r_k(x)\) in place of \(f(x)\)
    getting \(\tilde{q}(x),\tilde{r}(x)\) so
    \(r_k(x)=\tilde{q}(x)g(x)+\tilde{r}(x)\) and \(\tilde{r}(x)=0\) or
    \(\gdeg(\tilde{r})<k\) and so putting \(q_{k-1}(x)=q_k+\tilde{q}(x)\)
    and \(r_{k-1}=\tilde{r}\) we get
    \[f(x) = (q_k(x)+\tilde{q}(x))g(x)+\tilde{r}_1(x) =
             \tilde{q}(x)g(x)+\tilde{r}(x).\]
    Getting to \(q_{n-1}(x)\) and \(r_{n-1}(x)\)
    is results with the goal of the division algorithm.

 \item[(ii)]
    Say \(f(x)=q_i(x)g(x)+r_i(x)\) and
    \(r_i(x)=0\) or \(\gdeg(r_i)<\gdeg(g)\) for \(i=1,2\).
    Since \(d(x)=(q_1(x)-q_2(x))g(x)=r_2(x)-r1(x)\) it is clear
    that \(g(x)\mid d(x)=r_2(x)-r1(x)\) but \(\gdeg(d)\leq\gdeg(g)\)
    or \(d(x)=0\)
    and only the latter is possible and so \(r_1(x)=r_2(x)\)
    and because \(R[x]\) is a domain we have  \(q_1(x)=q_2(x)\).
\end{itemize}

%%%%%
\item
\begin{excopy}
A \textbf{subfield} $F$ of a ring $R$ is a subring of $R$ that is a field.
Show that a subset $X$ of a ring $R$ is a subfield if and only if $X$
contains $1$ and $X$ is closed under subtraction, multiplication and inverses.
\end{excopy}

Assume  \(X\subseteq R\) is a a subfield. By definition is is a subring
and thus closed under subtraction and multiplication.
It must have $1$ and by being field  closed under inverses for nonzero
elements.

Assume  \(1\in X\subseteq R\) and $X$
is closed under subtraction, multiplication and inverses.
Thus is is a subtraction and a field.
%%%%%

\item
\begin{excopy}
Prove that the intersection of any family of subfields is itself a subfield.
(Note that this intersection is not \(\{0\}\) because it contains $1$.)
\end{excopy}

Closure under (finite) operations is maintained in the intersection.
So is the $0$ and $1$ elements.

%%%%%
\item
\begin{excopy}
\begin{itemize}
 \item[(i)]
   Show that \(\Z_p[x]\) is an infinite domain containing \(Z_p\)
   as a subfield.
 \item[(ii)]
   Show that there exists an infinite field containing \(Z_p\) as a subfield.
\end{itemize}
\end{excopy}

\begin{itemize}
 \item[(i)]
   The ring \(\Z_p[x]\) is  infinite since it contains the different
   polynomials \(x^n\)  for all \(n\in Z\). \(Z_p\) is a field
   and in particular domain. Hence from Exercise~13~(iii)
   \(\Z_p[x]\) is a domain whose constants are naturally
   isomorphic to \(\Z_p\).
 \item[(ii)]
   The fraction field \(\Frac(\Z_p[x])\) of the infinite ring \(\Z_p[x]\)
   is infinite and its constant polynomials with fraction $1$
   is isomorphic to \(Z_p\)
\end{itemize}


%%%%%
\item
\begin{excopy}
Show that \(R[x]\) is never a field.
\end{excopy}

The polynomial \(f(x)=x\) has no inverses in \(R[x]\) hence
it is not a field.

%%%%%
\item
\begin{excopy}
Show that \(\Z_n\) is a field if and only if $n$ is prime.
\end{excopy}

The ring \(\Z_n\) is a field iff
for all $m$ such that \(0<m<n\) there exists an inverse to \(m+n\Z\).
Equivalently,  iff for all $m$ such that \(0<m<n\) there exists
\(i\in\Z\) such that \(im=1\bmod n\).
Equivalently,  iff for all $m$ such that \(0<m<n\) there exists
\(i,d\in\Z\) such that \(im+dn=1\) which happens
iff \((m,n)=1\) for all $m$ such that \(0<m<n\)
or equivalently, $n$ is prime.

   %%%%%%%%%%%%%
\end{myenumerate}
%%%%%%%%%%%%%%%%%


%%%%%%%%%%%%%%%%%%%%%%%%%%%%%%%%%%%%%%%%%%%%%%%%%%%%%%%%%%%%%%%%%%%%%%%%
%%%%%%%%%%%%%%%%%%%%%%%%%%%%%%%%%%%%%%%%%%%%%%%%%%%%%%%%%%%%%%%%%%%%%%%%
%%%%%%%%%%%%%%%%%%%%%%%%%%%%%%%%%%%%%%%%%%%%%%%%%%%%%%%%%%%%%%%%%%%%%%%%
\chapterTypeout{Homomorphisms and Ideals}

%%%%%%%%%%%%%%%%%%%%%%%%%%%%%%%%%%%%%%%%%%%%%%%%%%%%%%%%%%%%%%%%%%%%%%%%
%%%%%%%%%%%%%%%%%%%%%%%%%%%%%%%%%%%%%%%%%%%%%%%%%%%%%%%%%%%%%%%%%%%%%%%%
\section{Exercises (pages 19--21)}

%%%%%%%%%%%%%%%%
\begin{myenumerate}

%%%%%
\item
\begin{excopy}
If $R$ is a field, prove that the map \(R\rightarrow \Frac(R)\),
given by \(a\mapsto a/1\), is an isomorphism. Conversely, prove that
if $R$ is a domain and the map \(a\mapsto a/1\) is an isomorphism,
then $R$ is a field.
\end{excopy}

The kernel of this map is clearly \(\{0\}\) and so
the map is injective. Each element in \(\Frac(R)\)
is of the form \(a/b\) where \(a,b\in R\) and it is easy to see
that \(ab^-1 \mapsto (ab^-1)/1 \sim a/b\).

%%%%%
\item
\begin{excopy}
If \(\varphi: R \rightarrow S\) is an isomorphism between domains,
prove that there is an isomorphism \(\Frac(R)\rightarrow\Frac(S)\),
namely, \(a/b \mapsto \varphi(a)/\varphi(b)\).
\end{excopy}

Denote the mapping by \(\phi\).

Well defined:
Say \(a_1/b_1\sim a_2/b_2\) and so \(a_1b_2=a_2b_1\).
Hence
\[ \varphi(a_1)\varphi(b_2) =
   \varphi(a_1b_2) = \varphi(a_2b_1)
   \varphi(a_2)\varphi(b_1). \]
And so \(\varphi(a_1)/\varphi(b_1) \sim \varphi(a_2)/\varphi(b_2)\),
thus
\[ \phi(a_1/b_1)
 = \varphi(a_1)/\varphi(b_1)\
 = \varphi(a_2)/\varphi(b_2)\
   \phi(a_2/b_2).\]

Additivity:
\begin{eqnarray*}
\phi(a_1/b_1 + a_2/b_2)
 & = & \phi( (a_1b_2 + a_2b_1)/(b_1b_2) ) \\
 & = & \varphi(a_1b_2 + a_2b_1)/\varphi(b_1b_2) \\
 & = & (\varphi(a_1b_2) + \varphi(a_2b_1))/\varphi(b_1b_2) \\
 & = &  \varphi(a_1b_2) / \varphi(b_1b_2)
      + \varphi(a_2b_1) / \varphi(b_1b_2) \\
 & = &  \varphi(a_1) / \varphi(b_1) + \varphi(a_2) / \varphi(b_2) \\
 & = & \phi(a_1/b_1) + \phi(a_2/b_2)
\end{eqnarray*}

Multiplicative:
\begin{eqnarray*}
\phi(a_1/b_1 \cdot a_2/b_2)
 & = & \phi((a_1a_2)/(b_1b_2))\\
 & = & \varphi(a_1a_2)/\varphi(b_1b_2) \\
 & = & (\varphi(a_1)\varphi(a_2)) / (\varphi(b_1)\varphi(b_2)) \\
 & = & (\varphi(a_1)/\varphi(b_1)) \cdot (\varphi(a_2)/\varphi(b_2)) \\
 & = & \phi(a_1/b_1) \cdot \phi(a_2/b_2)
\end{eqnarray*}

%%%%%
\item
\begin{excopy}
Let $R$ be a subring of a field $F$, and let  $K$ be the intersection
of all the subfields of $F$ that contain $R$.
Prove that \(K\cong \Frac(R)\).
\end{excopy}

Let \(a/b\) be an arbitrary element of \(\Frac(R)\),
then \(ab^-1\in F\) and must also be in all subfields containing $R$
and so \(ab^-1\in K\). Thus, the mapping
\(\phi:\Frac(R) \rightarrow  K\) defined by
\(a/b\mapsto ab^{-1}\) is well defined homomorphism
as shown in previous exercise.
The kernel is \(\{0\}\) and so the mapping is injective.
The image \(\phi(\Frac(R))\) is a subfield of $K$ and by
the definition of $K$ as intersection, the image equals $K$
and thus \(\phi\) is actually an isomorphism.

%%%%%
\item
\begin{excopy}
\begin{itemize}
 \item[(i)]
   If \(\varphi: R \rightarrow S\) is an isomorphism, then
   \(\varphi^{-1}: S \rightarrow R\) is also an isomorphism.
 \item[(ii)]
   If \(\varphi: R \rightarrow S\) and
   If \(\psi: S \rightarrow T\) are ring  homomorphisms, then so is
   the composite \(\psi\varphi: R \rightarrow T\).
\end{itemize}
\end{excopy}

Trivial.

%%%%%
\item
\begin{excopy}
If \(a\in R\) is a unit and if
\(\varphi: R \rightarrow S\) is a ring map, then \(\varphi(a)\)
is a unit in $S$.
\end{excopy}

Ther must be an \(a'\in R\) so \(aa'=1\) in $R$ and since
\(\varphi(1)=1\) in $S$ we have \(\varphi(a)\varphi(a')=1\)
and  \(\varphi(a)\) is a unit in $S$.


%%%%%
\item
\begin{excopy}
\begin{itemize}
 \item[(i)]
   If $R$ is a ring, prove that \(\varphi: R[x]\rightarrow R\),
   where \(\varphi: f(x) \mapsto c_0\), the constant term of \(f(x)\),
   is a ring map.
 \item[(ii)]
   What is \(\ker\varphi\)?
\end{itemize}
\end{excopy}

\begin{itemize}
 \item[(i)]
   Let $r$ be and arbitrary element in $R$
   and  \(f(x)=\sum a_ix^i\),  \(g=(x)=\sum b_ix^i\)
   arbitrary polynomials in \(R[x]\).
   Checking the definition in \cite{Rotman98}~page~17:

   \begin{itemize}
    \item
      Sum:
      \(\varphi(f(x)+g(x)) = a_0 + b_0 = \varphi(f(x))+ \varphi(g(x))\).
    \item
      Product:
      \(\varphi(f(x)+g(x)) = a_0 \cdot b_0 = \varphi(f(x))\varphi(g(x))\).
    \item
      One:
      \(\varphi(1+\sum_{i\neq 0} a_ix^i) = \varphi(1) = 1\).
   \end{itemize}
 \item[(ii)]
    The kernel is all polynomials with zero constant term.
\end{itemize}


%%%%%
\item
\begin{excopy} \label{ex:sigstar}
\begin{itemize}
 \item[(i)]
   If \(\sigma: R \rightarrow S\) is a ring map, prove that
   \(\sigma^*: R[x] \rightarrow s[x]\), defined by
   \[ \sum r_i x^i \mapsto \sum \varphi(r_i)x^i,\]
   is also a ring map.
 \item[(ii)]
   If \(\tau: S \rightarrow T\) is a ring map, prove that
   \((\tau\sigma)^*: R[x] \rightarrow T[x]\) is equal to \(\tau^* \sigma^*\).
 \item[(iii)]
   Prove that if \(\sigma\) is an isomorphism, then so is \(\sigma^*\).
\end{itemize}
\end{excopy}

\begin{itemize}
 \item[(i)]
   Let $r$ be and arbitrary element in $R$
   and  \(f(x)=\sum a_ix^i\),  \(g=(x)=\sum b_ix^i\)
   arbitrary polynomials in \(R[x]\).
   \begin{itemize}
    \item
      Sum:
      \begin{eqnarray*}
        \varphi^*(f(x)+g(x))
        & = & \varphi^*\left( \sum a_i x^i +  \sum b_i x^i\right) \\
        & = & \varphi^*( \sum (a_i + b_i) x^i ) \\
        & = & \sum \varphi(a_i + b_i) x^i  \\
        & = & \sum \varphi(a_i) x^i + \sum \varphi(b_i) x^i  \\
        & = & \varphi^*(f(x))+ \varphi^*(g(x)).
      \end{eqnarray*}
    \item
      Product:
      \begin{eqnarray*}
        \varphi^*(f(x)g(x))
        & = & \varphi^*\left(
                 \sum_i \left(\sum_{k+j=i} (a_k+b_j)\right) x^i
                       \right) \\
        & = & \sum_i \varphi^*\left(\sum_{k+j=i} (a_k+b_j)\right) x^i ) \\
        & = & \sum_i \left(\sum_{k+j=i} \varphi(a_k+b_j)\right) x^i ) \\
        & = & \left(\sum \varphi(a_i) x^i\right) \cdot
              \left(\sum \varphi(b_i) x^i\right) \\
        & = & \varphi^*\left(\sum a_i x^i\right) \cdot
              \varphi^*\left(\sum b_i x^i\right) \\
        & = & \varphi^*(f(x)) \varphi^*(g(x)).
      \end{eqnarray*}
    \item
      One --- Trivially:  \(\varphi(1) = 1\).
   \end{itemize}
 \item[(ii)]
    Let \(f(x)=\sum a_ix^i\in R[x]\).
    \begin{eqnarray*}
    (\tau\sigma)^*(f(x))
     & = & (\tau\sigma)^*\left(\sum a_i x^i\right) \\
     & = & \sum (\tau\sigma)(a_i) x^i \\
     & = & \tau^*\left(\sum \sigma(a_i) x^i\right) \\
     & = & \tau^*\left(\sigma^*\left(\sum a_i x^i\right)\right) \\
     & = & (\tau^*\sigma^*)\left(\sum a_i x^i\right)
    \end{eqnarray*}
    and so \((\tau\sigma)^* = \tau^*\sigma^*\).
 \item[(iii)]
    In (i) it was shown that \(\sigma^*\) is homomorphism.
    Now since the kernel is \(\{0\}\) (the constant zero polynomial)
    it is injective. It is also surjective since each
    \(g(x)=\sum b_ix^i \in S[x]\) has an inverse
    \(\sum \sigma^{-1}(b_i)x^i \in R[x]\).

\end{itemize}

%%%%%
\item
\begin{excopy}
\begin{itemize}
 \item[(i)]
   The intersection of any family of ideals in $R$ is an ideal is an ideal
   in $R$. Conclude that if $X$ is any subset of a ring $R$, there is a smallest
   ideal, denoted by \((X)\), containing $X$. Once calls \((X)\)
   \index{ideal generated by}
   \textbf{ideal generated by} $X$, namely, the intersection of all ideals
   in $R$ that contains $X$.
 \item[(ii)]
   Prove that \((X)\) is the ``smallest'' ideal containing $X$ in
   the following sense: \((X)\) is an ideal containing $X$ and,
   if $J$ is any ideal in $R$ containing $X$, then \((X)\subset J\).
\end{itemize}
\end{excopy}

\begin{itemize}
 \item[(i)]
   Let \(I = \cap_\alpha I_\alpha\) an intersection of ideals.
   It is clear that:
   \begin{itemize}
    \item \(0\in I\).
    \item
      For any \(r\in R\) and \(a\in I\),
      we have \(ra \in I_\alpha\) for all \(\alpha\)
      and so \(ra\in I\).
    \item
      For any \(a,b\in I\), we have \(a-b\in I_\alpha\) for all \(\alpha\)
      and so \(a-b\in I\).
   \end{itemize}
 \item[(ii)]
   If by negation there exist an ideal $I$ that contains $X$
   then it is one of the ideals participating in the intersection defining
   $X$, hence \(X\subseteq I\).
\end{itemize}

%%%%%
\item
\begin{excopy}
\begin{itemize}
 \item[(i)]
   If \(a\in R\), prove that \(\{ra:r\in R\}\) is an ideal generated by $a$;
   it is called the
   \index{generated!principal ideal}
   \index{principal ideal generated}
   \textbf{principal ideal generated by } $a$, and is denoted by \((a)\).
 \item[(ii)]
   If \(\seqan\) are elements in a ring $R$, prove that the set of
   all linear combinations,
   \[I = \{r_1a_1 + \cdots + r_na_n: r_i\in R, i=1,\ldots,n\},\]
   is equal to \((\seqan)\), the ideal generated by
   \(\{\seqan\}\).
\end{itemize}
\end{excopy}

\begin{itemize}
 \item[(i)]
   By definition, \(\{ra:r\in R\} \subseteq (\{a\})\).
   Since clearly \((a)=\{ra:r\in R\}\) is an ideal, it is the ideal
   generated by $a$.
 \item[(ii)]
   An ideal $I$ containing \(\seqan\in R\) must also contain
   \(ra_i\) for any \(r\in R\). Since an ideal must be closed
   under subtraction, and scalar multiplication, it contains
   all linear combinations \(r_1a_1 + \cdots + r_na_n\), where
   \(r_i\in R\).
   On the other hand, It is easy to see that the set of these
   linear combinations:
   \begin{itemize}
     \item contains the zero and element.
     \item closed under subtraction.
     \item closed under scalar multiplication.
   \end{itemize}
   and therefore this set is an ideal. By minimality shown before,
   it is the ideal generated by \(\seqan\).
\end{itemize}
\label{ex:ei:ideal:generated}
\label{ex:ideal:generated}


%%%%%
\item
\begin{excopy}
Let $u$ be a unit in a ring $R$.
\begin{itemize}
 \item[(i)]
   Prove that if an ideal $I$ contains $u$, then \(I=R\).
 \item[(ii)]
   If \(r\in R\), then \((ur)=(r)\).
   In particular, every nonzero principal ideal
   \((f(x))\) in \(R=F[x]\), where $F$ is a field, can be generated by a monic
   polynomial.
 \item[(iii)]
   If $R$ is a domain and \(r,s\in R\), then \((r)=(s)\) if and only if
   \(s=ur\) for some unit $u$ in $R$.
\end{itemize}
\end{excopy}

\begin{itemize}
 \item[(i)]
    Since \(u^{-1}I=I\) we have \(1=u^{-1}u\in I\) and so
    for any \(r\in R\), we have \(r=r\cdot 1\in I\).
 \item[(ii)]
   Since \(r\in(r)\), we have \(ur\in(r)\) and so \((ur)\subseteq(r)\).
   Conversely, \(r=u^{-1}(ur)\) and thus \(r\in (ur)\)
   and so \((r)\subseteq(ur)\).

   Given \(f(x)\in F[x]\), and \(a\neq 0\) is the leading coefficient,
   we have \((a^{-1}f(x))=(f(x))\) and \(a^{-1}f(x)\) is monic.
 \item[(iii)]
\end{itemize}


%%%%%
\item
\begin{excopy}
Prove that a ring $R$ is a field if and only if it has only one proper ideal,
namely \(\{0\}\).
\end{excopy}

Say a ring $R$ is a field. If,
by negation, there is a non trivial proper ideal \(I\neq\{0\}\)
then we have a nonzero \(a\in I\). But then
\(1=a^{-1}a\in I\) and \(I=R\) a contradiction.

Conversely, assume \(\{0\}\) is the only proper ideal.
Let $a$ be an arbitrary nonzero element of $R$.
Then, \((a)=\{ra:r\in R\}\)
is not a proper ideal, and there must be some \(r\in R\) so \(ra=1\)
and thus \(a^{-1}\) exists. Thus $R$ is a field.


%%%%%
\item
\begin{excopy}
\begin{itemize}
 \item[(i)]
   The set $I$ if all \(f(x)\in \Z[x]\) having even constant term is an ideal
   in \(\Z[x]\); it consists of all linear combinations of $x$ and $2$;
   that is, \(I=(x,2)\).
 \item[(ii)]
   Prove that \((x,2)\) is not a principal ideal in \(\Z[x]\).
\end{itemize}
\end{excopy}

\begin{itemize}
 \item[(i)]
   It is clear that $I$ contains $0$, and closed under
   subtraction and scalar multiplication. Thus $I$ is an ideal.
   Any \(f(x)=\sum a_i x^i\in \Z[x]\) with \(2\mid a_0\)
   can be expressed as the required linear combination:
   \[f(x) = \left(\sum_{i>0} a_ix^{i-1}\right)x + (a_0/2)2.\]
 \item[(ii)]
   Say by negation there is \(g(x)\in \Z[x]\) so \((g(x))=(x,2)\).
   If \(g(x)\) is a constant, it cannot generate \(f(x)=x\), otherwise
   it cannot generate the constant \(f(x)=2\).
\end{itemize}


%%%%%
\item
\begin{excopy}
Prove that if $F$ is a field  and $S$ is a ring,
then a ring map
\(\varphi :F \rightarrow S\) must be an injection  and \(\im \varphi\)
is a subfield of $S$ isomorphic to $F$.
\end{excopy}

Let $a$ be a nonzero in $F$.
Clearly, \[\varphi(a)\varphi(a^{-1})=\varphi(aa^{-1})=\varphi(1)=1.\]
Thus \(\varphi(a)\neq 0\) and \(\ker \varphi=\{0\}\) and thus
\(\varphi\) is an injection.


    %%%%%%%%%%%%%
\end{myenumerate}
%%%%%%%%%%%%%%%%%


%%%%%%%%%%%%%%%%%%%%%%%%%%%%%%%%%%%%%%%%%%%%%%%%%%%%%%%%%%%%%%%%%%%%%%%%
%%%%%%%%%%%%%%%%%%%%%%%%%%%%%%%%%%%%%%%%%%%%%%%%%%%%%%%%%%%%%%%%%%%%%%%%
%%%%%%%%%%%%%%%%%%%%%%%%%%%%%%%%%%%%%%%%%%%%%%%%%%%%%%%%%%%%%%%%%%%%%%%%
\chapterTypeout{Quotient Rings}

%%%%%%%%%%%%%%%%%%%%%%%%%%%%%%%%%%%%%%%%%%%%%%%%%%%%%%%%%%%%%%%%%%%%%%%%
%%%%%%%%%%%%%%%%%%%%%%%%%%%%%%%%%%%%%%%%%%%%%%%%%%%%%%%%%%%%%%%%%%%%%%%%
\section{Exercises (page 23)}



%%%%%%%%%%%%%%%%
\begin{myenumerate}


%%%%%
\item
\begin{excopy}
Let $n$ be a positive integer and let \(I=(n)\) be a principal ideal
in \Z\ generated by~$n$. Show that the quotient  ring \(\Z/I\) is equal
to \(\Z_n\), the ring of integers modulu~$n$.
(Hint. These rings have the same elements (\([a]=a+I\))
and the same addition and multiplication.
\end{excopy}

The equality of elements \([a]\in \Z/I\) and \(\{a+kn: k\in \Z\}\in\Z_n\)
is obvious:
\begin{itemize}
 \item
    If \(x\in[a]\)then \(x-a\in I\) and so there is an integer $k$
    such that \(x-a=kn\) and \(x\in \{a+kn: k\in \Z\}\).
 \item
    Conversely, if \(x\in \{a+kn: k\in \Z\}\)
    then there is an integer $k$  such that \(x-a=kn\).
    Thus, \(x-a\in I\) and \(x\in[a]\).

\end{itemize}

%%%%%
\item
\begin{excopy}
Prove that if $R$ is a ring and \(I=(x)\) is the principal ideal in \(R[x]\)
generated by $x$, then \(R[x]/I \cong R\).
\end{excopy}

We define \(\varphi:R[x] \rightarrow R\)
by \(\varphi(\sum c_ix^i)=c_0\).
It is clear that this is a ring homomorphism.
The kernel consists of all polynomial with zero constant term
which is exactly $I$ and so \(R[x]/I = R[x]/\ker\varphi \cong R\).

%%%%%
\item
\begin{excopy}
Prove the
\index{correspondence theorem for rings}
\textbf{Correspondence Theorem for Rings}.
If $I$ is a proper ideal in a ring $R$, then there is a bijection
from the family of all intermediate ideals $J$, where
\(I\subset J\subset R\), to the family of all ideals in \(R/I\), given by
 \[J\mapsto \pi(J) = J/I = \{a+I: a\in J\}, \]
where \(\pi :R \rightarrow R/I\) is the natural map.
Moreover, if \(J\subset J'\) are intermediate ideals, then
\(\pi(J)\subset \pi(J')\). (Compare with theorem G.9.)
\end{excopy}

First let us show that \(\pi(J)\) is indeed an ideal in \(R/I\).
Since \(0\in J\) and \(0\in I\), we have \(\pi(0)=0_{R/I}\in\pi(J)\).
For any element \(a+I\) in \(\pi(J)\), we have \(a\in J\).
Thus for any two elements \(a+I\), \(b+I\) in \(\pi(J)\)
we have \(a,b\in J\) and so \(a-b \in J\) therefore
\(a+I)-(b+I)=(a-b)+I\in \pi(J)\).

By mapping definition, it is clear
that if  \(J\subseteq J'\) then \(\pi(J)\subseteq \pi(J')\).
\iffalse
Now let assume that \(J\varsubsetneqq J'\)
and let \(a\in J'\setminus J\).
Clearly \(a+I\) is in \(\pi(J')\). If by negation
\(a+I\in \pi(J)\) then there must be some
\(x\in J\) such that \(a+I=x+I\), hence \(a-x\in I\subset J\).
Since $J$ is an ideal, \(a\in J\) a contradiction.
Thus \(\pi(J)\varsubsetneqq \pi(J')\).
\fi

Now for any two differenet ideals \(I\subset J,K \subset R\),
without loss of generality, we can assume that
there exists \(a\in J'\setminus J\).
Clearly \(a+I\) is in \(\pi(J)\). If by negation
\(a+I\in \pi(K)\) then there must be some
\(x\in K\) such that \(a+I=x+I\), hence \(a-x\in I\subset K\).
Since $K$ is an ideal, \(a=(a-x)+x\in J\) a contradiction.

Thus \(\pi(J)\varsubsetneqq \pi(J')\).
\(a+I\in \pi(J)\setminus \pi(K)\) and
\(\pi(J)\neq \pi(K)\) and so \(\pi\) is a bijection.


%%%%%
\item
\begin{excopy}
Let $I$ be an ideal in a ring $R$, let $J$ be an ideal in a ring $S$,
and let \(\varphi:R \rightarrow S\) be a ring isomorphism with
\(\varphi(I)=J\). Prove that the function
\(\bar{\varphi}:r+I \mapsto \varphi(r)+J\) is a (well defined) isomorphism
\(R/I \rightarrow S/J\).
\end{excopy}

If \(r_1+I = r_2+I\) are two representations then \(r_1-r_2\in I\)
and so \(\varphi(r_1)-\varphi(r_2)\in J\) and thus
\(\bar{\varphi}\) is well defined.

Clearly it is an homomorphism.

Injection: \(\ker(\bar\varphi)=\ker(\varphi)+I=0+I=0_{R/I}\)

Surjection:  For any \(s+J\in S/J\), there exists
\(r=\varphi^-1(s)\in R\), and
\(\bar{\varphi}(r)+I=\varphi(r)+J=s+J\).

   %%%%%%%%%%%%%
\end{myenumerate}
%%%%%%%%%%%%%%%%%


%%%%%%%%%%%%%%%%%%%%%%%%%%%%%%%%%%%%%%%%%%%%%%%%%%%%%%%%%%%%%%%%%%%%%%%%
%%%%%%%%%%%%%%%%%%%%%%%%%%%%%%%%%%%%%%%%%%%%%%%%%%%%%%%%%%%%%%%%%%%%%%%%
%%%%%%%%%%%%%%%%%%%%%%%%%%%%%%%%%%%%%%%%%%%%%%%%%%%%%%%%%%%%%%%%%%%%%%%%
\chapterTypeout{Polynomial Rings over Fields}

%%%%%%%%%%%%%%%%%%%%%%%%%%%%%%%%%%%%%%%%%%%%%%%%%%%%%%%%%%%%%%%%%%%%%%%%
%%%%%%%%%%%%%%%%%%%%%%%%%%%%%%%%%%%%%%%%%%%%%%%%%%%%%%%%%%%%%%%%%%%%%%%%
\section{Exercises (pages 30--31)}

%%%%%%%%%%%%%%%%
\begin{myenumerate}

%%%%%
\item
\begin{excopy}
Prove that there are domains $R$ containing a pair of elements
having no gcd.
(Hint. Let $F$ be a field and let $R$ be the subring of \(F[x]\)
consisting of all polynomials having no linear term;
i.e., \(f(x)\in R\) if and only if
\[f(x)=a_0+a_2x^2+a_3x^3+\cdots.\]
Show that \(x^5\) and \(x^6\) have no gcd by noting that their
common monic divisors are $1$, \(x^2\) and \(x^3\), none of which is divisible
in $R$ by the other two.)
\end{excopy}

Following the hint, $R$ is indeed a domain since \(F[x]\) is.
In $R$ the monic divisors of \(x^5\) are:
$1$, \(x^2\), \(x^3\) and \(x^5\),
while the monic divisors of \(x^6\) are:
$1$, \(x^2\), \(x^3\), \(x^4\) and \(x^6\).
So the common monic divisors are: $1$, \(x^2\) and \(x^3\).
None of which can be a gcd since none of these three divides
the other two.

%%%%%
\item
\begin{excopy}
\begin{itemize}
 \item[(i)]
   Define the gcd of integers \seqan\ to be a positive integers $d$
   which is a
   \index{common divisor}
   \emph{common divisor},
   i.e., \(d\mid a_i\) for all $i$, that is divisible by every common divisor.
   Prove that the gcd $d$ of \seqan\ exists, and that $d$ is a linear
   combination of \seqan. (Hint. Let $d$ be a positive generator of the
   ideal in \Z\  generated by \seqan.)
 \item[(ii)]
   Define the gcd of polynomials \(\seqn{f}\in F[x]\), where $F$ is a field,
   to be a monic polynomial $d$ which is a common divisor.
   Prove the generalization of Corollary~16 that the gcd of \seqn{f}
   exists, and that $d$ is a linear combination of \seqn{f}.
\end{itemize}
\end{excopy}


\begin{itemize}
 \item[(i)]
   Following the hint, let $d$ be a positive generator of the
   ideal $I$ in \Z\  generated by \seqan.
   For every \(x\in I\), we have \(d\mid x\)
   and so \(x\mid a_i\) for all \(i=1,\ldots,n\).
   By exercise~\ref{ex:ei:ideal:generated} (page~\pageref{ex:ideal:generated})
   $d$ is a linear combination of \seqan. Now, say $k$ is an integer
   that also divides all \(\{a_i\}_{i=1}^n\), then it divides any
   linear combination of \seqan, and thus it divides $d$. Hence
   $d$ is the gcd.
 \item[(ii)]
   Let $I$ be the ideal in \(F[x]\) generated by \seqn{f}.
   Since \(F[x]\) is a principal ideal domain it has a generator
   \(d(x)\) which is equal to a linear combination of \seqn{f}
   and so \(d(x)\mid f_i(x)\) for all \(i=1,\ldots,n\).
   By dividing, by its leading coefficient, we can assume that \(d(x)\)
   is monic.
   If \(h(x)\) is any polynomial dividing all \(f_i(x)\)
   it must divides any linear combination of them and in particular
   \(h(x)\mid d(x)\), thus \(d(x)\) is the gcd of \seqn{f}.
\end{itemize}


%%%%%
\item
\begin{excopy}
Prove that if \seqan\ are distinct elements in a field $F$, then for all $i$,
the polynomials \(x-a_{i+1}\) and \((x-a_1)(x-a_2)\cdots(x-a_i)\)
are relatively prime.
\end{excopy}

Assume by negation that there is a non constant common divisor \(d(x)\),
so it must be a degree $1$. Without loss of generality,
we can assume that \(d(x)\) is monic, now since
\(d(x)\mid (x-a_{i+1})\) looking at the leading coefficient,
we must have \(d(x) = (x-a_{i+1})\).
Since \(d(x)\) is a common divisor, there exists \(q(x)\in F[x]\)
such that
\begin{equation} \label{eq:xa1toi}
(x-a_1)(x-a_2)\cdots(x-a_i) = q(x)(x-a_{i+1}).
\end{equation}
Evaluating both sides of equation~(\ref{eq:xa1toi})
at \(x=a_{i+1}\) we get a nonzero product on the left
being equal to $0$, a contradiction.

%%%%%
\item
\begin{excopy}
In the ring \(R=\Z[x]\), show that $x$ and $2$ are relatively prime,
but there are no polynomials \(f(x)\) and \(g(x)\in \Z[x]\) with
\(1=xf(x)+2g(x)\).
\end{excopy}

The divisors of $x$ are: \(\{1,x\}\).
The divisors of $2$ are: \(\{1,2\}\).
Thus \(\gcd(x,2)=1\) and so $x$ and $2$ are relatively prime.

For any  \(f(x),g(x)\in \Z[x]\)
the constant term of \(h(x)=1=xf(x)+2g(x)\) is even.
In particular \(h(x)\neq 1\).

%%%%% 44
\item
\begin{excopy}
Let \(f(x)=\prod(x-a_i)\in F[x]\), where $F$ is a field and
\(a_i\in F\) for all $i$.
Show that \(f(x)\) has
\index{repeated roots}
\textbf{no repeated roots} [i.e., \(f(x)\) is not a multiple of \((x-a)^2\)
for any \(a\in F\)] if and only if \((f(x),f'(x))=1\),
where \(f'(x)\) is the derivative of \(f(x)\).
\end{excopy}

Equivalently, we will show that
\(f(x)\) has repeated roots iff \((f(x),f'(x))\neq1\).

First let us prove
\begin{equation} \label{eq:derivn}
\left(\prod_{i=1}^n(x-a_i)\right)' =
  \sum_{i=j}^n \prod_{\overset{1\leq i\leq n}{i\neq j}}(x-a_i)
\end{equation}

By induction on $n$. For \(n=1\) we have on the left side
of equation~\ref{eq:derivn} \((x-a_i)'=1\).
On the right side of that equation, we have a single empty product
that trivially evaluates to $1$.
Now assume equation~(\ref{eq:derivn}) holds for \(n=k\).
Using the
product derivative formula~(\ref{eq:deriv:mult})
(page \pageref{eq:deriv:mult}) we compute:

\begin{eqnarray*}
\left(\prod_{i=1}^{k+1}(x-a_i)\right)'
 & = & \left(\left(\prod_{i=1}^{k}(x-a_i)\right)(x-a_{k+1})\right)' \\
 & = & \left(\sum_{j=1}^k \prod_{\overset{1\leq i\leq k}{i\neq j}}(x-a_i)\right)
       (x-a_{k+1}) +
        \left(\prod_{i=1}^{k}(x-a_i)\right)\cdot 1 \\
 & = & \sum_{j=1}^k \prod_{\overset{1\leq i\leq k+1}{i\neq j}}(x-a_i) +
        \prod_{i=1}^k(x-a_i) \\
 & = & \sum_{j=1}^{k+1} \prod_{\overset{1\leq i\leq k+1}{i\neq j}}(x-a_i)
\end{eqnarray*}

If \(f(x)\) has a repeated root, say \(a_r=a_s\) where
\(1\neq r < s\neq n\) then the term \((x-a_r)\) appears in
all the products in equation~(\ref{eq:derivn}).
Thus \(x-a_r\) divides both \(f(x)\) and \(f'(x)\) and \((f(x),f'(x))\neq 1\).

Before showing the converse direction,
let us prove that if \(d(x)\mid f(x)\) and \(\gdeg(d)\geq 1\)
then \(a_i\) is a root of \(d(x)\) for some $i$.
By \(d(x)\) being a divisor, there exists a monic polynomial \(q(x)\)
such that \(d(x)q(x)=f(x)\).
For each \(i=1,\ldots,n\) the root \(a_i\) of \(f(x)\)
is certainly a root of \(d(x)\) or \(q(x)\).
If by negation all the roots \(\{a_i:1\leq i\leq n\}\) of \(f(x)\)
are roots of \(q(x)\)
then \(\gdeg(q)\geq n\)  but this contraditcs
\(n+1\leq \gdeg(q)+\gdeg(d)=\gdeg(f)=n\).
Thus \(a_i\) is a root of \(d(x)\) for some $i$, where \(0\leq i \leq n\).

Now say \(d(x)=(f(x),f'(x))\neq 1\).
Then some \(a_k\) is a root of \(d(x)\) where \(1\leq k\leq n\)
and thus \(f'(a_k)=0\). Using equation~(\ref{eq:derivn}) we compute
\begin{eqnarray}
0 & = & f'(a_k) \notag \\
  & = & \sum_{j=1}^n \prod_{\overset{1\leq i\leq n}{i\neq j}}(a_k-a_i) \notag \\
  & = & \prod_{\overset{1\leq i\leq n}{i\neq k}}(a_k-a_i) \label{eq:ak}
\end{eqnarray}

The equality~(\ref{eq:ak}) follows from the fact that all other products
are zerod by the \((a_k-a_k)\) that they have.
Now the last product is zero and so there exists some \(i\neq k\) such
that \(a_k-a_i=0\) and so \(f(x)\) has a repeated root \(a_i=a_k\).


%%%%%
\item
\begin{excopy}
Find the gcd of \(x^3-2x^2+1\) and \(x^2-x-3\) in \(\Q[x]\)
and express it as a linear combination.
\end{excopy}

We do two divisions:
\begin{equation} \label{eq:gcd:div1}
x^3-2x^2+1 = (x+1)(x^2-x-3)+2(x+2)
\end{equation}

followed by:
\begin{equation}\label{eq:gcd:div2}
x^2-x-3 = (x-3)(x-2)+3
\end{equation}

From equation~(\ref{eq:gcd:div1}) we have:
\begin{equation}
x+2 = (1/2)(x^3-2x^2+1) - (1/2)(x+1)(x^2-x-3)
\end{equation}

Using this and (\ref{eq:gcd:div2}):
\begin{eqnarray*}
3 &=& (x^2-x-3) - (x-3)(x-2) \\
 &=& (x^2-x-3) \\
 & & \quad - (x-3)\left(\frac{1}{2}(x^3-2x^2+1)
                        -\frac{1}{2}(x+1)(x^2-x-3)\right)
\end{eqnarray*}

Dividing by $3$ and collecting terms of original polynomials we get:
\begin{equation*}
\gcd(x^3-2x^2+1, x^2-x-3) = 1 =
   -\frac{1}{6}(x-3)(x^3-2x^2+1) + \frac{1}{6}(x^2-2x-1)(x^2-x-3)
\end{equation*}

%%%%%
\item
\begin{excopy}
Prove that \(\Z_2[x]/I\) is a field, where \(p(x)=x^3+x+1\in\Z_2[x]\)
and \(I=(p(x))\).
\end{excopy}

We actually need to show that all nonzero elements of \(\Z_2[x]/I\)
have an inverse.
Without loss of generality, we can look only on polynomials
of degrees \(<3\) since we can always divide by \(x^3+x+1\)
and take ressidue.
All such  polynomials are \(a_2x^2+a_1x+a_0\)
where \(a_2,a_1,a_0\) can assume values of \(0,1\) only
and so there are \(3^2=8\) of them:
\[0 \quad 1 \quad
  x \quad x+1 \quad
  x^2 \qquad x^2+1 \quad x^2+x \quad x^2+x+1.\]
Let's show that all of them have an inverse as required.
\begin{itemize}

\item The $0$ polynomial should not and does not have an inverse.
\item The $1$ polynomial is trivially its own inverse.
\item The $x$ and \(x^2+1\) polynomials:
  \[  x(x^2+1) = x^3 +x \equiv 2x^3 + 2x + 1 \equiv 1 \bmod (x^3+x+1) \]
\item The \(x+1\) and \(x^2+x\) polynomials:
  \[  (x+1)(x^2+x) = x^3 + 2x^2 + x \equiv x^3 + x \equiv 1 \bmod (x^3+x+1) \]
\item The \(x^2\) and \(x^2+x+1\) polynomials:
\begin{multline*}
   x^2(x^2+x+1) = x^4 + x^3 + x^2 \\
      \equiv x^4 + x^3 + x^2 (1-x)(x^3+x+1)
      = 2x^3+2x+1 \equiv  1 \bmod (x^3+x+1)
\end{multline*}
\end{itemize}


%%%%%
\item
\begin{excopy}
If $R$ is a ring and \(a \in R\), let \(e_a:R[x]\rightarrow R\)
be evaluation at $a$.
Prove that \(\ker e_a\) consists of all polynomials over $R$ having $a$ as
a root, and so \(\ker e_a=(x-a)\), the principal ideal generated
by \(x-a\).
\end{excopy}

Directly by definitions.


%%%%%
\item
\begin{excopy}
Let $F$ be a field, and let \(f(x),g(x)\in F[x]\).
Prove that if \(\gdeg(f)\leq\gdeg(g)=n\)
and if \(f(a)=g(a)\) for \(n+1\) elements \(a\in F\), then \(f(x)=g(x)\).
\end{excopy}

Look at the polynomial \(h(x)=f(x)-g(x)\).
If h(x) is not identically zero, then clearly
\(0\leq \gdeg(h)\leq n\). But \(h(x)\) has at least \(n+1\) roots in $F$
contradicting \cite{Rotman98}~Theorem~22.

   %%%%%%%%%%%%%
\end{myenumerate}
%%%%%%%%%%%%%%%%%


%%%%%%%%%%%%%%%%%%%%%%%%%%%%%%%%%%%%%%%%%%%%%%%%%%%%%%%%%%%%%%%%%%%%%%%%
%%%%%%%%%%%%%%%%%%%%%%%%%%%%%%%%%%%%%%%%%%%%%%%%%%%%%%%%%%%%%%%%%%%%%%%%
%%%%%%%%%%%%%%%%%%%%%%%%%%%%%%%%%%%%%%%%%%%%%%%%%%%%%%%%%%%%%%%%%%%%%%%%
\chapterTypeout{Prime Ideals and Maximal Ideals}

%%%%%%%%%%%%%%%%%%%%%%%%%%%%%%%%%%%%%%%%%%%%%%%%%%%%%%%%%%%%%%%%%%%%%%%%
%%%%%%%%%%%%%%%%%%%%%%%%%%%%%%%%%%%%%%%%%%%%%%%%%%%%%%%%%%%%%%%%%%%%%%%%
\section{Exercises (pages 37--38)}

%%%%%%%%%%%%%%%%
\begin{myenumerate}

%%%%%
\item
\begin{excopy}
A polynomial \(p(x)\in F[x]\) of degree $2$, or $3$ is irreducible over $F$
\label{ex:irr23}
if and only if $F$ contains no root of \(p(x)\).
(This is false for degree $4$: the polynomial \((x^2+1)^2\)
factors in \(\R[x]\), but it has no real roots.)
\end{excopy}

Equivalently, we show that \(p(x)\) is reducible in \(F[x]\) iff
it contains a root  in $F$.

If \(p(x)\) is reducible, then there are \(f(x),g(x)\in F[x]\)
with degrees \(<3\)
such that \(p(x)=f(x)g(x)\).
Since  \(\gdeg(f)+\gdeg(g)=2,3\)
then, \(\gdeg(f)=1\) or  \(\gdeg(g)=1\) must have a degree of $1$
and trivially have a root which must also be a root of \(p(x)\).

Conversely, if \(a\in F\) is a root of \(p(x)\) then \(f(x)=x-a\)
divides \(p(x)\) and \(p(x)\) is reducible.

%%%%%
\item
\begin{excopy}
Let \(p(x)\in F[x]\) be irreducible. If \(g(x)\in F[x]\) is not constant,
then either \((p(x),g(x))=1\) or \(p(x)\mid g(x)\).
\end{excopy}

\Wlogy, or by dividing by leading coefficient,
we can assume that both \(p(x)\) and \(g(x)\) are monic.
Let \(d(x)=\gcd(p(x),g(x))\). Since \(d(x)\mid p(x)\) and \(p(x)\)
is irreducible, \(d(x)=1\) or \(d(x)=p(x)\).
If \(d(x)=1\) we are done, otherwise, \(p(x)=d(x)\mid g(x)\).

% \vfill

%%%%%
\item
\begin{excopy}
\begin{itemize}
 \item[(i)]
   Every nonzero polynomial \(f(x)\) in \(F[x]\) has a factorization
   of the form
   \[f(x) = a p_1(x)\cdots p_t(x),\]
   where $a$ is a nonzero constant and the \(p_i(x)\) are (not necessarily
   distinct) monic irreducible polynomial;
 \item[(ii)]
   the factors and their multiplicities in this factorization are uniquely
   determined.
\end{itemize}

(Thus analogue of the fundamental theorem of arithmetic has the same
proof of the theorem: if also \(f(x)=bq_1(x)\cdots q_s(x)\),
where $b$ is constant and the \(q_j(x)\) are monic and irreducible,
the uniqueness is proved by Euclid's lemma and induction on \(\max\{t,s\}\).
One calls \(f[x]\)
\index{unique factorization domain}
\index{domain!unique factorization}
\index{factorization!unique}
a \textbf{unique factorization domain} when one wishes to call
attention to this property of it.)
\end{excopy}

\begin{itemize}
 \item[(i)]
   By induction on the degree of \(f(x)\).
   If \(\gdeg(f)=1\) then dividing by its leading coefficient
   we get the desired representation.
   Let us assume that \(\gdeg(f)=n\) and that
   a factorization exists for any polynomial
   of degree \(<n\).
   If \(f(x)\) is irreducible then again by dividing with
   its leading coefficient  we get a factorization.
   Otherwise, \(f(x)=g_1(x)g_2(x)\) where both \(\gdeg(g_i)<n\)
   and thus both \(g_i(x)\) have factorization.
   By combining their factorizations --- multiplying the constants
   and appending the monic irreducible polynomials ---
   we get a factorization for \(f(x)\).
 \item[(ii)]
   Let
   \begin{equation} \label{eq:produniq}
    f(x)=a\prod_{i=1}^t p_i(x) = b\prod_{i=1}^s q_i(x)
   \end{equation}
   two factorizations, with \(p_i(x)\) and \(q_i(x)\) monic irreducible
   polynomials. Obviously, \(a=b\) the leading coefficient of \(f(x)\).
   Now \(p_t(x)\) divides \(f(x)\). By Euclid's lemma
   (\cite{Rotman98}~Corollary~15, page~26) \(p_i(x)\mid q_j(x)\)
   for some $j$. Since \(q_j(x)\) is also irreducible, \(p_i(x)=q_j(x)\).
   Now by induction on the degree of \(f(x)\)
   we have a unique factorization of
    \[g(x) = f(x)/(ap_t(x))
      = \prod_{i=1}^{t-1} p_i(x)
      = \prod_{\overset{1\leq i\leq s}{i\neq j}} q_i(x).\]
   Multiplying back by \(ap_t(x)=bq_j(x)\) shows the
   uniqueness of the representation in (\ref{eq:produniq}).
\end{itemize}


%%%%%
\item
\begin{excopy}
Let \(f(x)=ap_1(x)^{k_1} \cdots p_t(x)^{k_t}\)
and \(g(x)=bp_1(x)^{n_1} \cdots p_t(x)^{n_t}\),
where \(k_i\geq 0\), \(n_i\geq 0\),\, \(a,b\) are nonzero constants,
and \(p_i(x)\) are distinct monic irreducible polynomials (zero exponents
allow one to  have the same \(p_i(x)\) in both factorizations).
Prove that
\[\gcd(f,g) = p_1(x)^{m_1} \cdots p_t(x)^{m_t}\]
and
\[\lcm(f,g) = p_1(x)^{M_1} \cdots p_t(x)^{M_t},\]
where \(m_i=\min(k_i,n_i)\)
and \(M_i=\max(k_i,n_i)\).
\end{excopy}


Clearly \(k_i-m_i\geq 0\) and \(n_i-m_i\geq 0\)
and also \(M_i-m_i\geq 0\) and \(M_i-n_i\geq 0\).
Thus we have

{
\newcommand{\prodit}[1]{\prod_{i=1}^t p_i^{#1}(x)}
\begin{eqnarray*}
f(x) & = & \prodit{k_i} = \prodit{m_i} \cdot \prodit{k_i-m_i}
       = \gcd(f,g)\cdot\prodit{k_i-m_i} \\
g(x) & = & \prodit{n_i} = \prodit{m_i} \cdot \prodit{n_i-m_i}
       = \gcd(f,g)\cdot\prodit{n_i-m_i} \\
\end{eqnarray*}

and
\begin{eqnarray*}
\lcm(f,g)
  & = & \prodit{M_i} \\
  & = & \prodit{k_i} \cdot \prodit{M_i-k_i} = f(x)\cdot \prodit{M_i-k_i} \\
  & = & \prodit{n_i} \cdot \prodit{M_i-n_i} = g(x)\cdot \prodit{M_i-n_i}
\end{eqnarray*}
}

Thus \(\gcd(f,g)\) is well defined and divides both \(f(x)\) and \(g(x)\).
Also  \(\lcm(f,g)\) is well defined and is a multiple
of both \(f(x)\) and \(g(x)\).

It remains to show that \(\gcd(f,g)\) is the greatest
and \(\lcm(f,g)\) is the lowest.

\textbf{Maximality of \(\gcd(f,g)\) :}
Let \(d(x)\) be a common divisor of \(f(x)\) and \(g(x)\).
Again, by allowing zero exponents, and extending $t$ if necessary,
we have the representation: \(d(x)=\prod_{i=1}^t p_j^{l_j}(x)\).
By negation, assume that for some $j$, \(l_j>m_j\).
Since  $f$ and $g$ play symmetric role here,
we can assume, \wlogy, that \(m_j=\min(k_j,n_j)=k_j\).
Set \(u=l_j-k_j>0\), and
now since \(d(x)\mid f(x)\), dividing by \(p_j^{m_j}(x)\) we get
\begin{equation}
p_j(x)
  \mid p_j^u(x)
  \mid d(x)/p_j^{m_i}(x)
  = \prod_{\overset{1\leq i\leq t}{i\neq j}} p_i^{l_i}(x)
\end{equation}

By Euclid's lemma (\cite{Rotman98}~Corollary~15, page~26),
\(p_j(x)\mid p_{j'}(x)\) for some \(j\neq j'\).
But then \(p_j(x)=p_{j'}(x)\) since both are irreducible and monic.
A contradiction to the fact that the \(p_i\) are distinct.

Hence, \(l_i\leq m_i\) for all $i$,
and thus \(d(x)\mid\gcd(f,g)\) and \(\gcd(f,g)\) is the greatest.

\textbf{Minimality of \(\lcm(f,g)\):}
Now let \(\mu(x)\) be a common multiple of \(f(x)\) and \(g(x)\).
Again, by allowing zero exponents, and extending $t$ if necessary,
we have the representation: \(\mu(x)=\prod_{i=1}^t p_j^{L_j}(x)\).
By negation, assume that for some $j$, \(L_j<M_j\).

Since  $f$ and $g$ play symmetric role here,
we can assume, \wlogy, that \(M_j=\max(k_j,n_j)=k_j\).
Set \(u=k_j-L_j>0\), and
now since \(f(x)\mid \mu(x)\), % dividing by \(p_j^{k_j}(x)\)
we get
\begin{equation}
p_j(x)
  \mid p_j^u(x)
  \mid f(x)/p_j^{k_i}(x)
  \mid \mu(x)/p_j^{k_i}(x)
  = \prod_{\overset{1\leq i\leq t}{i\neq j}} p_i^{k_i}(x)
\end{equation}

Similarly utilizing Euclid's lemma we conclude
that for some\(j'\neq j\) \(p_j(x)\mid p_{j'}(x)\)
and as before \(p_j(x)=p_{j'}(x)\) contradicting the fact
that \(p_i(x)\) are distinct.

Hence, \(L_i\geq M_i\) for all $i$,
and thus \(\lcm(f,g)\mid\mu(x)\) and \(\lcm(f,g)\) is the lowest.

%%%%%
\item
\begin{excopy}
\begin{itemize}
 \item[(i)]
   Prove that the zero ideal in ring $R$ is a prime ideal if and only if
   $R$ is a domain.
 \item[(ii)]
   Prove that the zero ideal in a ring $R$ is a maximal ideal if and only if
   $R$ is a field.
\end{itemize}
\end{excopy}

Since \(R\cong R/\{0\}\) both claims are specific cases
of \cite{Rotman98}'s theorems 25 and 26 with \(I=\{0\}\).

\begin{itemize}
 \item[(i)]

   Using \cite{Rotman98}~Theorem~25
   Assume the zero ideal is a prime and let \(ab=0\) for some \(a,b\in R\).
   Now \(ab\in\{0\}\) and so \(a\in\{0\}\) or \(b\in\{0\}\)
   which means that \(a=0\) or \(b=0\) and $R$ is a domain.

   Conversely, assume $R$ is a domain and let \(ab\in \{0\}\) the zero ideal.
   Now \(a=0\) or \(b=0\) and so \(a\in\{0\}\) or \(b\in\{0\}\)
   and the zero ideal is prime.
 \item[(ii)]
   Assume zero ideal is a maximal ideal and let \(a\in R\) be a nonzero element.
   Now \(aR\) cannot be a proper ideal since the zero ideal is maximal.
   Thus \(aR=R\) and \(1\in aR\) and an particular \(a^{-1}\) exists
   and $R$ is a field.

   Conversely, assume $R$ is a field.
   Let $I$ be any ideal of $R$.
   if $I$ contains an nonzero element $a$ then
   since \(a^{-1}\in R\) then \(a^{-1}a=1\in R\) and so \(I=R\)
   Hence the only proper ideal is the zero ideal and it is a maximal ideal.
\end{itemize}


%%%%%
\item
\begin{excopy}
The ideal $I$ in \(\Z[x]\) consisting of all polynomials having an even
constant term is a maximal ideal.
\end{excopy}

Assume there is an ideal $J$, such that
\(I \varsubsetneqq J \subset \Z[x]\).
Thus there  must be some polynomial \(g(x)=\sum a_ix^i \in J\)
with an odd constant term, \(a_0=2k+1\) for some integer $k$.
Now, the polynomial \(\tilde{g}(x)=2k+\sum_{i>0}  a_ix^i \in I\subset J\)
and so \(g(x)-\tilde{g}(x)=1\in J\).
Therefore, \(J=\Z[x]\) and $I$ is maximal.

%%%%%
\item
\begin{excopy}
Let \(f(x),g(x)\in F[x]\). Then \((f,g)\neq 1\) if and only if there is a field
$E$ containing both $F$ and a common root of \(f(x)\) and \(g(x)\).
\end{excopy}

\index{Kronecker's theorem}
Using Kronecker's theorem (\cite{Rotman98} theorem~30),
we extend $F$ twice to $E$ so both \(f(x),g(x)\) split.

Now  \((f(x),g(x))\in F[x]\subset E[x]\) and then \((f,g)\)
splits in $E$.
Now \((f,g)\neq 1\) iff \((f,g)\) has a root in $E$.


%%%%%
\item
\begin{excopy}
\begin{itemize}
 \item[(i)]
   Prove that if \(f(x)\in \Z_p[x]\), then \((f(x))^p=f(x^p)\).
   \index{Fermat}
   (Hint: Use Fermat's theorem: \(a^p\equiv a \bmod p\).)
 \item[(ii)]
   Show that the first part of the exercise may be false if
   \(\Z_p\) is replaced by an inifinite field of characteristic $p$.
\end{itemize}
\end{excopy}

\begin{itemize}
 \item[(i)]
   By induction on the degree of \(f(x)\).
   Say \(f(x)\) is of degree 0, then \(f(x)=a_0\) is a constant
   and so \[(f(x))^p = a_0^p \equiv a_0 = f(x^p) \bmod p.\]

   Now assume the desired equalityis true for polynomial
   of degree \(<n\) and let \(f(x)=\sum_{i=0}^n a_ix^i\).

   We use Newton's binomial formula (see exercise~\ref{ex:binomial}).
   The following computation is done \(\mod \Z_p[x]\).
   \begin{eqnarray}
   f(x)^p
    & = & \left(\sum_{i=0}^n a_ix^i\right)^p \notag \\
    & = & \left(\left(\sum_{i=0}^{n-1} a_ix^i\right) + a_nx^n\right)^p \notag \\
    & = & \sum_{k=0}^p \binom{p}{k}
                       \left(\sum_{i=0}^{n-1} a_ix^i\right)^k
                       (a_nx^n)^{p-k} \notag \\
    & \equiv & \label{eq:binom:modp}
           (a_nx^n)^p +  \left(\sum_{i=0}^{n-1} a_ix^i\right)^p \\
    & = &      \label{eq:binom:induc}
           a_n (x^p)^n + \sum_{i=0}^{n-1} a_i(x^p)^i \\
    & = & \sum_{i=0}^n a_i(x^p)^i \notag \\
    & = & f(x^p) \notag
   \end{eqnarray}

   \textbf{Notes:}
   The equality in (\ref{eq:binom:modp}) is derived
   from \(\binom{p}{k}=0\) in \(\Z_p\) for \(0<k<p\) since $p$ is prime.
   \index{Fermat}
   The equality in (\ref{eq:binom:induc}) is by Fermat's theorem
   (\(a^p\equiv a \bmod p\).) and by induction.

 \item[(ii)]
   Consider the field \(F=\Frac(\Z_p[t])\) and let
  \(f(x)=tx\in F[x]\) where $t$ is a polynomial in \(\Z_p[t]) \subset F\).
  Obviously \(t^p \neq t\) in \(F[x]\), and so
  \[f(x)^p = t^px^p \neq tx^p = f(x^p).\]
\end{itemize}

%%%%%
\item
\begin{excopy}
Exhibit an infinite field of characteristic $p$. (Hint: Exercise~20.)
\end{excopy}

The fractions field of \(\Frac(\Z_p[x])\) is infinite and of characteristic $p$.

%%%%%
\item
\begin{excopy}
If $F$ is a field, prove that the kernel of any evaluation map
\(F[x]\rightarrow F\) is a maximal ideal.
\end{excopy}

Any evaluation map \(\varphi\) induces the isomorphism:
\(\tilde{\varphi}: F[x]/\ker(\varphi) \rightarrow F\).
By \cite{Rotman98}'s theorem~26,
since $F$ is a field the ideal \(\ker(\varphi)\) must be maximal.

%%%%%
\item
\begin{excopy}
If $F$ is a field of characteristic $0$ and \(p(x)\in F[x]\) is irreducible,
then \(p(x)\) has no repeated roots. (Hint: Consider \((p(x),p'(x))\).)
\end{excopy}

If \(\gdeg(p)<2\) then trivially, there are no repeated roots.
We can assume that \(\gdeg(p)\geq 2\).

Let $E$ be an extension field of $F$ where \(p(x)\) splits.
The existence of such $E$ is given by
\index{Kronecker's theorem}
Kronecker's theorem
(\cite{Rotman98} page~34).
Let
\(p(x)=\prod_{i=1}^n(x-a_i)\) where \(a_i\in E\) be the splitting polynomial
representation.
By negation, assume \(p(x)\) has a repeated root and
so \(a_j=a_k\) for some \(0\leq j < k \leq n\).
By the derivative formula (\ref{eq:derivn}) (page~\pageref{eq:derivn}),
\(p'(a_j)=p(a_k)=0\). Now the leading coefficient of \(p'(x)\)
is $n$ ($n$ times 1) and since \(\fchar F=0\), so in $F$ \(n\neq 0\)
and thus \(p'(x)\) has a degree of \(n-1>0\).
Note that \(\gcd(p,p')\in F[x]\) but \((x-a_j)\mid\gcd(p,p')\) in \(E[x]\)
and so \(gcd(p,p')\neq 1\) contradicting the fact that \(p(x)\)
is irreducible in \(F[x]\).

%% (possibly in some extension field).

%%%%% 60
\item
\begin{excopy}
Use
\index{Kronecker's theorem}
Kronecker's theorem to construct a field with four elements
by adjoining a suitable root of \(x^4-x\) to \(\Z_2\).
\end{excopy}

First, we factorize  \(x^4-x = x(x-1)(x^2+x+1)\).
Clearly $0$ and $1$ or obvious roots
and we need to find a root for \((x^2+x+1)\).
Let \(I=(x^2+x+1)\) be the ideal in \(\Z_2[x]\) generated
by \(x^2+x+1\) and
\(E=\Z_2[x]/(x^2+x+1)\) be the extension field
with the solution \(\theta = x+I\).
Since we can always divide any representative polynomial by \(x^2+x+1\)
and get a residue, \(\{1, x\}\) span $E$ over \(\Z_2\).
So for additions we have:

% \begin{tabular}{|>{$}l<{$}|l|l|l|l|} \hline
\begin{tabular}{R||C|C|C|C|}
+    &  0    &  1   &  x  &  x+1 \\ \hline\hline
0    &  0    &      &     &      \\ \hline
1    &  1    &  0   &     &      \\ \hline
x    &  x    &  x+1 &  0  &      \\ \hline
x+1  &  x+1  &  x   &  1  &   0  \\ \hline
\end{tabular}

And multiplications:

% \begin{tabular}{|>{$}l<{$}|l|l|l|l|} \hline
\begin{tabular}{R||C|C|C|C|}
\mathbf{\cdot}
     &  0  &  1    &  x    &  x+1 \\ \hline\hline
0    &  0  &       &       &      \\ \hline
1    &  0  &  1    &       &      \\ \hline
x    &  0  &  x    &  x+1  &      \\ \hline
x+1  &  0  &  x+1  &  1    &   x  \\ \hline
\end{tabular}

%%%%%
\item
\begin{excopy}
Give the addition and multiplication tables of a field having eight elements.
(hint: Factor \(x^8-x\) over \(Z_2\).
\end{excopy}

First, it is easy to see that
\[x^8-x = x(x-1)(x^7-1) = x(x-1)(x^6+x^5+x^4+x^3+x^2+x+1).\]
We guess that the sub-polynomial \(\sum_{i=1}^6 x^i\)
is reducible by polynomials of degrees
$3$ over \(\Z_2\).
Solving the \emph{coefficients}
\(\{a_i\}_{i=0}^3\) and
\(\{b_i\}_{i=0}^3\)
in:
\[\sum_{i=0}^6 1\cdot x^i
  = \left(\sum_{i=0}^3 a_i x^i\right)\left(\sum_{i=0}^3 b_i x^i\right)\]
gives the irreducible polynomials \((x^3+x^2+1)\), \((x^3+x+1)\)
(see exercise~\ref{ex:irr23}).
Hence
\begin{equation}
 x^8-x = x(x-1)(x^3+x^2+1)(x^3+x+1).
\end{equation}

The splitting field is \(\Z_2[x]/(x^3+x+1)\).
By dividing with \(x^3+x+1\) we can look at polynomial representatives
of degrees \(\leq2\). Thus we have the $8$ polynomials \(\sum_{i=0}^2 a_i x^i\)
with \(a_i=0,1\).


The addition is simply of that of a vector space.
Thus
\[\left(\sum_{i=0}^2 a_i x^i\right) + \left(\sum_{i=0}^2 b_i x^i\right)
 = \sum_{i=0}^2 (a_i+b_i) x^i\]
where \(a_i+b_i\) is done modulo $2$ in \(\Z_2\).

For the multiplication, we multiply these polynomials
and take the residue after dividing by \(x^3+x+1\).
We will denote each polynomial element
as a 3-binary digit combination \(a_{_2}a_{_1}a_{_0}\).

\begin{center}
\begin{tabular}{R||C|C|C|C|C|C|C|C|}
\mathbf{\cdot}
      & 000 &   001 & 010 & 011 & 100 & 101 & 110 & 111  \\ \hline\hline
 000  &  0  &    0  &  0  &  0  &  0  &  0  &  0  &  0   \\ \hline

 001  &  0  &   001 & 010 & 011 & 100 & 101 & 110 & 111  \\ \hline
 010  &  0  &   010 & 100 & 110 & 011 & 001 & 111 & 101  \\ \hline
 011  &  0  &   011 & 110 & 101 & 111 & 100 & 001 & 010  \\ \hline
 100  &  0  &   100 & 011 & 111 & 110 & 010 & 101 & 001  \\ \hline
 101  &  0  &   101 & 001 & 100 & 010 & 111 & 011 & 110  \\ \hline
 110  &  0  &   110 & 111 & 001 & 101 & 011 & 010 & 100  \\ \hline
 111  &  0  &   111 & 101 & 010 & 001 & 110 & 100 & 011  \\ \hline
\end{tabular}
\end{center}


%%%%%
\item
\begin{excopy}
Show that a field with four elements is not (isomorphic to)
a subfield of a field with eight elements.
\end{excopy}

Let $G$ be a (the) field of $8$ elements. Its multiplicative group
of $7$ elements is cyclic and each element is a generator.
Thus no element except $1$ can belong to a subgroup, in particular
to a multiplicative group of a subfield of $4$ elements
that should include $3$ elements and not just $1$.

   %%%%%%%%%%%%%
\end{myenumerate}
%%%%%%%%%%%%%%%%%



%%%%%%%%%%%%%%%%%%%%%%%%%%%%%%%%%%%%%%%%%%%%%%%%%%%%%%%%%%%%%%%%%%%%%%%%
%%%%%%%%%%%%%%%%%%%%%%%%%%%%%%%%%%%%%%%%%%%%%%%%%%%%%%%%%%%%%%%%%%%%%%%%
%%%%%%%%%%%%%%%%%%%%%%%%%%%%%%%%%%%%%%%%%%%%%%%%%%%%%%%%%%%%%%%%%%%%%%%%
\chapterTypeout{Irreducible Polynomials}

%%%%%%%%%%%%%%%%%%%%%%%%%%%%%%%%%%%%%%%%%%%%%%%%%%%%%%%%%%%%%%%%%%%%%%%%
%%%%%%%%%%%%%%%%%%%%%%%%%%%%%%%%%%%%%%%%%%%%%%%%%%%%%%%%%%%%%%%%%%%%%%%%
\section{Notes}


%%%%%%%%%%%%%%%%%%%%%%%%%%%%%%%%%%%%%%%%%%%%%%%%%%%%%%%%%%%%%%%%%%%%%%%%
\subsection{Syntax}

A footnote (6) in page~40 reads:
\begin{quotation}
This elegant proof of Gauss's  lemma was shown me by Peter Cameron.
\end{quotation}

I think it should say:
\begin{quotation}
This elegant proof of Gauss's lemma was shown \textbf{to} me by Peter Cameron.
\end{quotation}

%%%%%%%%%%%%%%%%%%%%%%%%%%%%%%%%%%%%%%%%%%%%%%%%%%%%%%%%%%%%%%%%%%%%%%%%
\subsection{More Elegent Proof}

In page 42 there is a remark:
\begin{quotation}
\textbf{Remark.} The following more elegant proof of Gauss's lemma
is due to Peter Cameron.
\end{quotation}

Isn't it confusing and somewhat incosistent with the footnote
of previous page crediting the proof \emph{there} to Peter Cameron?

%%%%%%%%%%%%%%%%%%%%%%%%%%%%%%%%%%%%%%%%%%%%%%%%%%%%%%%%%%%%%%%%%%%%%%%%
\subsection{Wrong Equality}

On page~43 there is an equality in \(\Q[x]\):
\begin{equation*}
x^3+x^2+x+1 = (x+1)(x^2+x+1)
\end{equation*}
which is wrong! the factorization shoud be:
\begin{equation*}
x^3+x^2+x+1 = (x+1)(x^2+1)
\end{equation*}



%%%%%%%%%%%%%%%%%%%%%%%%%%%%%%%%%%%%%%%%%%%%%%%%%%%%%%%%%%%%%%%%%%%%%%%%
%%%%%%%%%%%%%%%%%%%%%%%%%%%%%%%%%%%%%%%%%%%%%%%%%%%%%%%%%%%%%%%%%%%%%%%%
\section{Exercises (page 43)}

%%%%%%%%%%%%%%%%
\begin{myenumerate}

%%%%%
\item
\begin{excopy}
Let \label{ex:QZx:rs}
\(f(x)= a_0+a_1x+\cdots + a_nx^n\in\Z[x]\).
If \(r/s\) is a rational root of \(f(x)\), where \(r/s\)
is in lowest terms, i.e., \((r,s)=1\),
then \(r\mid a_0\) and \(s\mid a_n\).
Conclude that any rational root of a monic polynomial in \(\Z[x]\)
must be an integer.
\end{excopy}

Since \(r/s\) is a root we have
\begin{equation*}
\sum_{i=0}^n a_i(r/s)^i = 0.
\end{equation*}
Multiplying by \(s^n\) gives:
\begin{equation*}
\sum_{i=0}^n a_i r^i s^{n-i} = 0.
\end{equation*}

Clearly \(r\mid 0\) and \(s\mid 0\).
Looking at the term of \(i=0\) we get \(r\mid a_0 s^n\).
Since \((r,s)=1\) we have \(r\mid a_0\).

Looking at the term of \(i=n\) we get \(s\mid a_n r^n\).
Since \((r,s)=1\) we have \(s\mid a_n\).

Monic polynomial means that \(a_n=1\) and so \(s=1\)
and the rootmust be an integer.

%%%%%
\item
\begin{excopy}
Test
\label{ex:4polys:inQ}
whether the following polynomials factor in \(\Q[x]\):
\begin{itemize}
 \item[(i)]   \(3x^2-7x-5\);
 \item[(ii)]  \(6x^3-3x-18\);
 \item[(iii)] \(x^3-7x+1\).
 \item[(iv)]  \(x^3-9x-9\).
\end{itemize}
\end{excopy}

We use mappings to \(\Z_p[x]\) and the technique used in
\cite{Rotman98} Example~15 on page~39
to show that the polynomials cannot be factored into
lower degrees.
By using
\index{Gauss}
\cite{Rotman98}~Theorem~39 (Gauss) we conclude
that the polynomials are irreducible in \(\Q[x]\).

\begin{itemize}
 \item[(i)]
   Mapping to \(\Z_2[x]\) gives a polynomial \(x^2+x+1\)
   for which both $0$ and $1$ evaluates to $1$
   thus do not have solution in \Zn{2}.
 \item[(ii)]
   First we switch to (factoring by $3$) \(f(x)=2x^3-x-6\).
   we evaluate in \Zn{11}:
   \begin{center}
   \begin{tabular}{|R|R|R|} \hline
      % Thank python for the results ...
      n  &   f(n) & f(n) \bmod 11 \\ \hline\hline
      0  &    -6  &  5  \\ \hline
      1  &    -5  &  6  \\ \hline
      2  &     8  &  8  \\ \hline
      3  &    45  &  1  \\ \hline
      4  &   118  &  8  \\ \hline
      5  &   239  &  8  \\ \hline
      6  &   420  &  2  \\ \hline
      7  &   673  &  2  \\ \hline
      8  &  1010  &  9  \\ \hline
      9  &  1443  &  2  \\ \hline
     10  &  1984  &  4  \\ \hline
   \end{tabular}
   \end{center}
   Thus \(f(x)\) has no solution in \Zn{11}.

\end{itemize}
The last two polynomials (of (iii) and (iv)\,)
are mapped to \(x^3+x+1\in\Z_2[x]\) and clearly have no solution in \Zn{2}.


%%%%%
\item
\begin{excopy}
Let $F$ be a field. Prove that if \(a_0+a_1 x +\cdots + a_n x^n \in F[x]\)
is irreducible then so \(a_n+a_{n-1}x +\cdots a_0x^n\).
\end{excopy}

Because of symmetry, this is actually an equivalence.

Assume \(f(x)=\sum_{i=0}^n a_i x^i\) can be factored by
\begin{equation}
f(x) = \sum_{i=0}^n a_i x^i
  = \left(\sum_{i=0}^l b_i x^i\right)\cdot\left(\sum_{i=0}^m c_i x^i \right)\,,
\end{equation}
where \(l+m=n\),
then since for every $j$, where \(0\leq j\leq  n\)
we have
\[a_i = \sum_{j+k=i} b_j c_k\]
and \(j+k=i\) iff \(n-i= n-(j+k)=(l-j)+(m-k)\)
and therefore:
\[a_{n-i} = \sum_{(l-j)+(m-k)=n-i} b_{l-j} c_{m-k}\]
and we get the factorization:
\begin{equation}
\hat{f}(x) = \sum_{i=0}^n a_{n-i} x^i
  = \left(\sum_{j=0}^l b_{l-j} x^j\right)\cdot
    \left(\sum_{k=0}^m c_{m-k} x^k \right)\,.
\end{equation}

%%%%%
\item
\begin{excopy}
If \(c\in R\), where $R$ is a ring, then the map
\(f(x) \mapsto f(x+c)\) is an isomorphism of the ring \(R[x]\) with
itself. Conclude, when $R$ is a field, that \(p(x)\)
is irreducible if and only if \(p(x+c)\) is irreducible.
\end{excopy}

Say $R$ is a field and \(f(x)\in R[x]\) is reducible
by a factorization \(f(x)=g(x)h(x)\) where \(\gdeg(g),\gdeg(h)<\gdeg(f)\).
Now let us define \(f_c(x)=f(x+c)\) and similarly
\(g_c(x)=g(x+c)\) and
\(h_c(x)=h(x+c)\). Clearly, \(f_c(x)=g_c(x)h_c(x)\)
and so \(f_c(x)\) is reducible.
Conversely we simply use an inverse mapping \(f_{-c}(x)=f(x-c)\).

%%%%%
\item
\begin{excopy}
Prove that \(f(x)=x^4 - 10x^2 + 1\) is irreducible in \(\Q[x]\).
(Hint. Use Exercise~63 to show that \(f(x)\) has no rational roots;
then show that there are no rationals $a$, $b$ and $c$ with
\begin{equation*}
x^4 - 10x^2+1 = (x^2+ax+b)(x^2-ax+c)\,.)
\end{equation*}
\end{excopy}

Becuase of Exercise~63 it is sufficient to show that there are
no integer solution to \(f(x)\).
Since \(f(x)=f(-x)\) an even function is is sufficient to check
the non negative integers.
Also from that exercise, an integer root must divide
the constant term which is $1$, so we end up checking $0$ and $1$.
But \(f(0)=1\) and \(f(1)=-8\)
and thus \(f(x)\) has no roots in \Q.

Assume \(f(x)\) is reducible in \(\Q[x]\).
It must then be factored with polynomials of degree $2$,
since otherwise it would need to factored
by polynomials of degrees $3$ and $1$ and have a root in \Q.

So the factorization is
\begin{equation} \label{eq:x67:a}
  f(x)= x^4 - 10x^2 + 1 = (x^2+a_1x + b)(x^2+a_2x + c).
\end{equation}
But the coefficient of \(x^3\) is \(a_1+a_2=0\).
Setting
\(a\rightarrow a\), \(a_2\rightarrow -a\)
and (\ref{eq:x67:a}) becomes:
\begin{equation}
  f(x) = x^4 - 10x^2 + 1 = (x^2 + ax + b)(x^2 - ax + c).
\end{equation}
Looking at the linear coefficient we get \(ac-ab=0\) and so \(a=0\)
or \(b=c\).

If \(a=0\) then
\begin{eqnarray} \label{eq:bc10}
b+c & = & -10 \\
bc  & = & 1 \notag
\end{eqnarray}
and so \(b^2+10b+1=0\) and the solutions to the (\ref{eq:bc10}) system
are symmetric: \(\{b,c\}=\{-5\pm4\sqrt{6}\}\)
contradicting \(b,c\in\Q\).

Otherwise, \(a\neq 0\)
% and looking at the coefficient of \(x^2\) we have \(b=c=-5\)
and looking at the linear coefficient we have \(b=c=\pm 1\)
Then,  looking at the coefficient of \(x^2\) we have
\(-a^2 \pm 2 = -10\) which is \(a^2\in\{8,12\}\) contradicting \(a\in\Q\).

   %%%%%%%%%%%%%
\end{myenumerate}
%%%%%%%%%%%%%%%%%


%%%%%%%%%%%%%%%%%%%%%%%%%%%%%%%%%%%%%%%%%%%%%%%%%%%%%%%%%%%%%%%%%%%%%%%%
%%%%%%%%%%%%%%%%%%%%%%%%%%%%%%%%%%%%%%%%%%%%%%%%%%%%%%%%%%%%%%%%%%%%%%%%
%%%%%%%%%%%%%%%%%%%%%%%%%%%%%%%%%%%%%%%%%%%%%%%%%%%%%%%%%%%%%%%%%%%%%%%%
\chapterTypeout{Classical Formulas} \label{chap:class:formulas}

%%%%%%%%%%%%%%%%%%%%%%%%%%%%%%%%%%%%%%%%%%%%%%%%%%%%%%%%%%%%%%%%%%%%%%%%
%%%%%%%%%%%%%%%%%%%%%%%%%%%%%%%%%%%%%%%%%%%%%%%%%%%%%%%%%%%%%%%%%%%%%%%%
\section{Notes}

%%%%%%%%%%%%%%%%%%%%%%%%%%%%%%%%%%%%%%%%%%%%%%%%%%%%%%%%%%%%%%%%%%%%%%%%
\subsection{Fix Shift to Reduce Polynomial}

On page~44 we have:
\begin{quotation}
\setcounter{quotelem}{42} % to get 43
  \begin{quotelem}
    If \(f(x) = a_nX^n + a_{n-1}X^{n-1} + a_{n-2}X^{n-2} + \cdots\),
    then replacing $X$ by \(x - a_{n-1}/n\) gives a reduced polynomial
    \begin{equation*}
      \tilde{f}(x) = f(x - a_{n-1}/n);
    \end{equation*}
    \mldots
  \end{quotelem}
\end{quotation}

But this works only if \(a_n = 1\), so it should be fixed by
with following:

\begin{quotation}
\setcounter{quotelem}{42} % to get 43
  \begin{quotelem}
    If \(f(x) = a_nX^n + a_{n-1}X^{n-1} + a_{n-2}X^{n-2} + \cdots\),
    then replacing $X$ by \(x - a_{n-1}/na_n\) gives a reduced polynomial
    \begin{equation*}
      \tilde{f}(x) = f(x - a_{n-1}/na_n);
    \end{equation*}
    \mldots
  \end{quotelem}
\end{quotation}


%%%%%%%%%%%%%%%%%%%%%%%%%%%%%%%%%%%%%%%%%%%%%%%%%%%%%%%%%%%%%%%%%%%%%%%%
\subsection{Avoid Division by Zero} \label{ss:avoid:zdiv}

In page 45, it may be better to skip the equation:
\[y^3 -q^3/27y^3 = -r,\]
since it may be invalid in case \(y=0\).
The equation follows:
\[y^6+ry^3-q^3/27=0\]
can be derived directly.
See also \ref{ss:avoid:zdiv2}.

%%%%%%%%%%%%%%%%%%%%%%%%%%%%%%%%%%%%%%%%%%%%%%%%%%%%%%%%%%%%%%%%%%%%%%%%
\subsection{Accurately Specify Taking Root}

Also in the last paragraph of page~45 it says:
\begin{quotation}
\ldots values for $y$; one is given by Eq.~(3);
\end{quotation}
But Eq~(3) has actually expression for \(y^3\).
So it would be better to say:
\begin{quotation}
\ldots values for $y$; one is given by taking some cubic root
of the Eq~(3)'s right side expression.
\end{quotation}

%%%%%%%%%%%%%%%%%%%%%%%%%%%%%%%%%%%%%%%%%%%%%%%%%%%%%%%%%%%%%%%%%%%%%%%%
%%%%%%%%%%%%%%%%%%%%%%%%%%%%%%%%%%%%%%%%%%%%%%%%%%%%%%%%%%%%%%%%%%%%%%%%
\section{Exercises (page 49)}

%%%%%%%%%%%%%%%%
\begin{myenumerate}

%%%%% 68
\item
\begin{excopy}
Given numbers $u$ and $v$, prove that there exist (possibly complex)
numbers $y$ and $z$ such that
\[y+z=u \quad\textrm{and}\quad yz=v.\]
\end{excopy}

Let \((u,v=(u \pm \sqrt{u^2-4v})/2\). Indeed
\begin{equation*}
y+z = (u + \sqrt{u^2-4v})/2 + (u - \sqrt{u^2-4v})/2 = u/2+ u/2 = u
\end{equation*}
and
\begin{eqnarray*}
yz
 & = & \left(\left(u + \sqrt{u^2-4v}\right)/2\right) \cdot
       \left(\left(u - \sqrt{u^2-4v}\right)/2\right) \\
 & = & (u^2 - (u^2-4v))/4 = v
\end{eqnarray*}

%%%%%
\item
\begin{excopy}
Factor \(x^3+x^2-36\) in \(\Q[x]\).
\end{excopy}

\[x^3+x^2-36 = (x-3)(x^2+4x+12).\]

%%%%%
\item
\begin{excopy}
Let \(g(x)= x^3+qx+r\) and define \(R=r^2+4q^3/27\).
Let $u$ be a root of \(g(x)\) and let \(u=y+z\),
where \(y^3 = \frac{1}{2}\left(-r+\sqrt{R}\right)\).
Prove that
\[ z^3 = \frac{1}{2}\left(-r-\sqrt{R}\right).\]
\end{excopy}

Repeating the analysis of page~45 in \cite{Rotman98}.
\begin{equation*}
u^3 = (y+z)^3 = y^3+z^3+3uyz.
\end{equation*}
and
\begin{equation} \label{eq:y3z3ur}
y^3+z^3+(3yz+q)u+r = 0
\end{equation}
Now if we \emph{further} require that \(3yz+q=0\)
we do \emph{not} get a contradiction, since
then we get the following system of equations:
\begin{eqnarray}
y^3+z^3 & = & -r \label{eq:y3z3} \\
y^3z^3  & = & -q^3/27.
\end{eqnarray}

Combining with (\ref{eq:y3z3ur})
and we get
\begin{equation*}
(y^3)^2+ry^3-q^3/27 = 0
\end{equation*}
For which the \emph{assumption} \(y^3= \frac{1}{2}\left(-r+\sqrt{R}\right)\)
is a solution. Thus from (\ref{eq:y3z3})
we have
\begin{equation*}
z^3 = -r - y^3 =
  -r - \frac{1}{2}\left(-r+\sqrt{R}\right) =
  \frac{1}{2}\left(-r-\sqrt{R}\right)
\end{equation*}


%%%%% 71
\item
\begin{excopy}
Find the roots of the following polynomials \(f(x)\in \R[x]\).
\begin{itemize}
 \item[(i)]    \(f(x) = x^3 - 3x + 1\).
 \item[(ii)]   \(f(x) = x^3 - 9x + 28\).
 \item[(iii)]  \(f(x) = x^3 - 24x^2 -24x - 25\).
 \item[(iv)]   \(f(x) = x^3 - 15x - 4\).
 \item[(v)]    \(f(x) = x^3 - 6x + 4\).
 \item[(vi)]   \(f(x) = x^4 - 15x^2 -20x - 6\).
\end{itemize}
\end{excopy}

Using the solutions for \(x^3+qx+r=0\) given by
the analysis of page~45 in \cite{Rotman98}.
\begin{itemize}
 \item[(i)]
   We have \(f(x) = x^3 - 3x + 1\) and so
        \(q=-3\), \(r=1\),  \(R=r^2+4q^3/27=1+4\cdot(-3)^3/27=1-4=-3\).
   Now let a solution be \(x=y+z\) and \(yz=-q/3=1\).
   Hence
   \begin{equation*}
   y^3 = \frac{1}{2}\left(-1+\sqrt{-3}\right) = -1/2 + i(\sqrt{3}/2) =
       e^{2\pi i/3}
   \end{equation*}
   Thus
   \[y = \{
          e^{2\pi i/9},
          e^{8\pi i/9},
          e^{11\pi i/9}
         \}\]
   and repectively
   \[z=-q/3y=1/y = \{
          e^{-2\pi i/9},
          e^{-8\pi i/9},
          e^{-11\pi i/9}
         \}\]
   and so the solutions:
   \begin{eqnarray*}
    x_1 & = & e^{2\pi i/9} + e^{-2\pi i/9} \\
    x_2 & = & e^{8\pi i/9} + e^{-8\pi i/9} \\
    x_3 & = & e^{11\pi i/9} + e^{-11\pi i/9}.
   \end{eqnarray*}
   are all real.

 \item[(ii)]
   We have \(f(x) = x^3 - 9x + 28\) and so
   \(q=-9\), \(r=28\),  \(R=784-4\cdot {9^3}/{3^3}=784-4\cdot 27=676\).
   Now let a solution be \(x=y+z\) and \(yz=-q/3=1\).
   Hence
   \begin{equation*}
   y^3 = \frac{1}{2}\left(-28+\sqrt{676}\right) = -14+13=-1
   \end{equation*}
   and so one solution is \(y=-1\) with \(z=-q/3y=-(-9/3(-1))=-3\)
   and \(x_1=y+z=-4\). Indeed \((-4)^3-9(-4)+28=-64+36+28=0\).

   Now we can simply divide \(f(x)/(x+4)=x^2-4x+7\)
   for which we have \(x_{2,3}=2\pm i\sqrt{-3}\).

 \item[(iii)]
   Using the substitution \(x=u+8\) we compute:
   \begin{eqnarray*}
   f(x)
     & = & x^3 - 24x^2 - 24x - 25 \\
     & = & u^3 + 3\cdot 8u^2 +       3\cdot 8^2 u  + 8^3 \,
                      -24u^2 - 24\cdot 2\cdot 8 u  - 24\cdot 8^2 \\
     &   &                                  -24 u  - 24\cdot 8 \, -25 \\
     & = & u^3 - 216 u - 1241
   \end{eqnarray*}
   and so we solve
   \begin{equation} \label{eq:gu}
   g(u) = u^3 - 216u - 1241
   \end{equation}
    with \(q=-216\), \(r=-1241\) and
   \(R=1241^2 - 4 \cdot 216^3/27 =  47089 = 217^2\).
   Again, put \(u=y+z\) and \(y^3=\half(1241+217)=729\) and \(y=9\)
   is a solution with \(z=-q/3y=216/(3\cdot 9)=8\)
   and indeed \(u=y+z=17\) is a solution to (\ref{eq:gu}).
   and \(x_1=u+8=25\) is a solution to \(f(x)\).
   Dividing by \(x-25\) we get
   \begin{equation*}
   f(x)= x^3 - 24x^2 - 24x - 25 = (x-25)(x^2+x+1).
   \end{equation*}
   Solving \(x^2+x+1\) gives the other solutions:
   \(x_{2,3} = \half \pm i\sqrt{3}/2\).

 \item[(iv)]
   We have \(f(x) = x^3 - 15x - 4\) thus
        \(q=-15\), \(r=-4\) and
        \[R = r^2+4q^3/27 = 16-4\cdot (3\cdot5)^3/3^3 =
          % 16 - 4\cdot 125 =
          -484 = -22^2.\]
   Now let a solution be \(x=y+z\) and \(yz=-q/3=5\).
   \(y^3=2+11i\) and by ``smart guess'' \(y=2+i\) is a solution.
   and \(z=5/y=5(2-i)/(2^2-i^2)=5(2-i)/5=2-i\)
   and a solution is \(x_1=y+z=2+i+2-i=4\).

   Now dividing by \(x-4\) we have:
   \begin{equation*}
   x^3 - 15x - 4 = (x-4)(x^2 -4x+1).
   \end{equation*}
   The remaining solutions are \(x_{2,3} = -2 \pm \sqrt{3}\).

 \item[(v)]
   We have \(f(x) = x^3 - 6x + 4\) thus
   \(q=-6\), \(r=4\) and
   \(R=4^2-4\cdot 6^3/27=-16\).
   Let \(y+z\) be a solution with \(y^3=(-4+\sqrt{-16})/2 = -2 + 2i\)
   and so \(y=1+i\) and \(z=-(-6/3y)=2/(1+i)=1-i\).
   Thus our first solution is \(x_1=y+z=2\).

   Factor \(f(x)=(x-2)(x^2+2x-2)\)
   and the other solutions are \(x_{2,3}=-1\pm i\).
\end{itemize}

For the quartic equation,
follow technique shown on page~48 in \cite{Rotman98}.
\begin{itemize}

 \item[(vi)]
  We solve \(x^4 - 15x^2 -20x - 6= 0\).
  Denote \(q=-15\), \(r=-20\), \(s=-6\).
  We look for factorization
  \begin{equation*}
    x^4+qx^2+rx+s = (x^2+kx+l)(x^2+kx+m).
  \end{equation*}
  As we saw in the reference, we first find \(\kappa=k^2\)
  by solving the cubic equation:
  \begin{equation*}
  k^6 + 2qk^4 + (q^2-4s)k^2 - r^2 = g(\kappa) =
    \kappa^3 - 30\kappa^2 + 249\kappa - 400 = 0.
  \end{equation*}
  To get rid of the quadratic coefficient, we substitute
  by \(\kappa=u+10\)
  \begin{eqnarray*}
   g(\kappa) & = & g(u+10) \\
    & = & u^3 + 30u^2 + 300u + 1000 \\
    &   &     - 30u^2 - 600u - 3000 \\
    &   &             + 249u + 2490 \, - 400 \\
    & = & u^3         -  51u + 90
  \end{eqnarray*}

  Let \(y+z\) be a solutions to \(u^3-51u+90\).
  Put \(q'=-51\), \(r'=90\), \(R=90^2-4\cdot 51^3/27=-11552=2\cdot76^2\).
  Thus \(y^3=-45+38\sqrt{2}i\).
  It follows that \(y=3+2\sqrt{2}i\),
  and \(z = -q'/3y = 17/(3+2\sqrt{2}i) = 3-2\sqrt{2}i\).
  Thus \(u=y+z=6\), \(\kappa=16\) and \(k=4\).

  From the reference analysis we solve
  \begin{eqnarray*}
  m & = & \left(k^2+q+r/k\right)/2 = (16 - 15 - 5)/2 = -2 \\
  l & = & \left(k^2+q-r/k\right)/2 = (16 - 15 + 5)/2 = 3
  \end{eqnarray*}

  Hence, to solve our quartic equation, we need to solve the following
  two quadratic equations:
  \begin{eqnarray*}
  x^2 + 4x + 3 & = & 0 \\
  x^2 - 4x - 2 & = & 0
  \end{eqnarray*}
  Thus the solutions are:
  \begin{eqnarray*}
    x_{1,2} & = & -2 \pm 1 = -1, -3 \\
    x_{3,4} & = & 2 \pm \sqrt{6}
  \end{eqnarray*}
\end{itemize}


   %%%%%%%%%%%%%
\end{myenumerate}
%%%%%%%%%%%%%%%%%


%%%%%%%%%%%%%%%%%%%%%%%%%%%%%%%%%%%%%%%%%%%%%%%%%%%%%%%%%%%%%%%%%%%%%%%%
%%%%%%%%%%%%%%%%%%%%%%%%%%%%%%%%%%%%%%%%%%%%%%%%%%%%%%%%%%%%%%%%%%%%%%%%
%%%%%%%%%%%%%%%%%%%%%%%%%%%%%%%%%%%%%%%%%%%%%%%%%%%%%%%%%%%%%%%%%%%%%%%%
\chapterTypeout{Splitting Fields}

%%%%%%%%%%%%%%%%%%%%%%%%%%%%%%%%%%%%%%%%%%%%%%%%%%%%%%%%%%%%%%%%%%%%%%%%
%%%%%%%%%%%%%%%%%%%%%%%%%%%%%%%%%%%%%%%%%%%%%%%%%%%%%%%%%%%%%%%%%%%%%%%%
\section{Notes}

On page~52, Theorem~47, items  (i) and (iii) could have been unified.
The split seems to be simply for the convenience of the proof.

On page~55 the definition of
\index{separable!polynomial}
\emph{separable polynomial} \(f(x)\in F[x]\) refers to not having
\index{repeated roots}
\emph{repeated roots}.
It would be nice to have a reminder that this property
is independent of the extension field as shown
in Exercise~44.

%%%%%%%%%%%%%%%%%%%%%%%%%%%%%%%%%%%%%%%%%%%%%%%%%%%%%%%%%%%%%%%%%%%%%%%%
%%%%%%%%%%%%%%%%%%%%%%%%%%%%%%%%%%%%%%%%%%%%%%%%%%%%%%%%%%%%%%%%%%%%%%%%
\section{Exercises (page 58)}

%%%%%%%%%%%%%%%%
\begin{myenumerate}

%%%%% 72
\item
\begin{excopy}
\begin{itemize}
 \item[(i)]
   Let \(E/F\) be an extension, and let \(\alpha, \beta\in E\) be algebraic
   elements over $F$.
   If \(\alpha \neq 0\), prove that \(\alpha+\beta\), \(\alpha\beta\)
   and \(\alpha^{-1}\) are all algebraic over $F$.
   (Hint. Use Lemma~49 to prove that \(F(\alpha,\beta)\)
   is a finite dimensional vector space over $F$.)
 \item[(ii)]
   If \(E/F\) is an extension, prove that the subset
   \[K = \{\alpha \in E:\alpha \textrm{\ is algebraic over\ } F\}\]
   is a subfield of $E$ containing $F$.
 \item[(iii)]
   Define the
   \index{algebraic numbers}
   \textbf{algebraic numbers} \A\ to be the set of
   all those complex numbers that are algebraic over \Q.
   Prove that \(\A/\Q\) is an algebraic extension that is not finite.

\end{itemize}
\end{excopy}

\begin{itemize}
 \item[(i)]
  Clearly, \(\alpha+\beta, \alpha\beta,\alpha^{-1}\in F(\alpha,\beta\)
  and by Lemma~49, \(F(\alpha,\beta\) is a finite dimensional vector space
  over $F$ and thus for any \(\gamma = \alpha+\beta, \alpha\beta,\alpha^{-1}\)
  there is some finite $m$, such that
   \(\{\gamma^i\}_{i=0}^m\) is linearly dependent
  over $F$, and such dependence gives a polynomial in \(F[x]\) for
  which \(\gamma\) is a root.
 \item[(ii)]
  This is a direct result of (i).
 \item[(iii)]
  The extension \(\A/\Q\) is algebraic by definition.
  Now for any $n$ there exists a prime \(p>n\) and
  the
  \index{cyclomtomic polynomial}
  cyclomtomic polynomial
   \(\Phi_p(x) = (x^p-1)/(x-1) = \sum_{k=0}^m x^k\)
  is irreducible over \(\Q[x]\)
  (see \cite{Rotman98} Corollary~41, page~42).
  Thus a root \(\alpha\) for \(\Phi_p(x)\) is in \A, and
  \([\A:\Q] \geq [\Q(\alpha):\Q] > p > n\).
\end{itemize}

%%%%%
\item
\begin{excopy}
Let $F$ be a field. Prove that if \(\sigma\) is an isomorphism of
\(F(\seqalphn)\) with itself such that
\(\sigma(\alpha_i)=\alpha_i\) for \(i=1,\ldots,n\), and
\(\sigma(c)=c\) for all \(c\in F\),
then \(\sigma\) is the identity. Conclude that if $E$ is a field
extension of $F$ and if
\(\sigma,\tau: F(\seqalphn) \rightarrow E\) fix $F$ pointwise
and \(\sigma(\alpha_i)=\tau(\alpha_i\) for all $i$,
then \(\sigma=\tau\).
\end{excopy}

\textbf{The first part}.

For any \(\gamma\in F(\seqalphn)\) there are polynomials
\[f(\seqxn),g(\seqxn)\in F[\seqxn]\]
 such that
\(\gamma = f(\seqalphn)/g(\seqalphn)\) since such evaluations
fo rational polynomials generate \(F(\seqalphn)\).
Now
\begin{eqnarray*}
 \sigma(\gamma)
 & = & \sigma(f(\seqalphn)) / \sigma(g(\seqalphn)) \\
 & = & \sigma(f)(\sigma(\alpha_1)\ldots\sigma(\alpha_n)) \,/\,
       \sigma(g)(\sigma(\alpha_1)\ldots\sigma(\alpha_n)) \\
 & = & f(\seqalphn) \,/\, g(\seqalphn)) = \gamma
\end{eqnarray*}
and thus \(\sigma\) is the identity.

\textbf{The second part}.

Similarly, for any \(\gamma\in F(\seqalphn)\)
\begin{eqnarray*}
 \sigma(\gamma)
 & = & \sigma(f(\seqalphn)) / \sigma(g(\seqalphn)) \\
 & = & \sigma(f)(\sigma(\alpha_1)\ldots\sigma(\alpha_n)) \,/\,
       \sigma(g)(\sigma(\alpha_1)\ldots\sigma(\alpha_n)) \\
 & = & \tau(f)(\tau(\alpha_1)\ldots\tau(\alpha_n)) \,/\,
       \tau(g)(\tau(\alpha_1)\ldots\tau(\alpha_n)) \\
 & = & \tau(f(\seqalphn)) / \tau(g(\seqalphn)) \\
 & = & \tau(\gamma).
\end{eqnarray*}
and thus \(\gamma=\tau\).

%%%%%
\item
\begin{excopy}
If \(F \subset B \subset E\) are fields  and \(E/F\) is finite,
then both \(E/B\) and \(B/F\) are finite, and
\([E:F] = [E:B][B:F]\).
\end{excopy}

Since \(E/F\) is finite, let \seqn{a} a base of $E$ over $F$.
Clearly, \(\seqn{a}\) span $E$ over $B$ and so \(E/B\) is finite.
Now if by negation \(B/F\) is inifinite, that we would have an inifinite
linearly independent set \(\{b_i: b_i\in B,\, i\in \N\}\)
which will also be a inifinite set of vectors from $E$
that are linearly independent over which is a contradiction.

Now the equality trivially follows from
Lemma~49 of \cite{Rotman98}.


%%%%%
\item
\begin{excopy}
If \(K=\Z_p(t)\), prove that \(f(x)=x^p-t\) is irreducible in \(K[x]\).
(Hint. If \(E/K\) is a splitting field of \(f(x)\), then \(x^p-t=(x-\alpha)^p\)
for some \(\alpha\in E\).)
\end{excopy}

For any polynomial \(q(t)\in K\), the polynomial \(q^p(t)\in K\)
all of its nonzero coefficient must correspond
to \(t^{mp}\) for some integer $m$, and specifically \(q^p(t)\neq t\)
and thus \(f(x)\) has no root in $K$.

Let \(\alpha\) be a root of \(f(x)\) in some extension field $E$.
In \(E[x]\) we have \((x-\alpha)^p=x^p-\alpha^p\).
The monic irreducible polynomial \(g(x)\) for which \(\alpha\)
is a root,must divide \(f(x)\) and so \(g(x)=(x-\alpha)^m\)
for some \(m\leq p\).
Now if \(m<q\) then \(\gcd(f(x),g(x))=x-\alpha\), contradiction
to the fact that \(\alpha\notin K\) and so
\[f(x) = g(x) = (x - \alpha)^p = x^p - t\]
is irreducible in \(K[x]\).

%%%%%
\item
\begin{excopy}
Show that
\label{ex:PerfProot}
a field $F$ of characteristic $p$ is perfect if and only if
every element of $F$ has a $p$th root in $F$.
\end{excopy}

Assume $F$ is perfect. Let \(a\in F\) and
\(f(x)=x^p-a\).
If be negation \(f(x)\) has no root in $F$
than it has some root \(\alpha\) in some extension field $E$,
and in \(E[x]\) we have \(f(x) = x^p - a = (x - \alpha)^p\).
But then \(f(x)\) is not separable which is
a contradiction to $F$ being perfect. Therefore,
\(f(x)\) has a root in $F$ and $a$ has a $p$th root in $F$.

Conversely, assume every element of $F$ has a $p$th root in $F$.
Let \(f(x)\in F[x]\), we will show that \(f(x)\) is separable.
\Wlogy\ we can assume that \(f(x)\) is irreducible.
If by negation \(f(x)\) has a repeated root \(\alpha\).
So we have  \(f(x)=(x-\alpha)^n g(x)\) where \(n>1\) and
\(g(\alpha)\neq 0\).

Now
\begin{equation*}
 f'(x) = n(x-a)^{n-1}g(x) + (x-a)^n g'(x)  \quad\textrm{in}\, F[x].
\end{equation*}

Now if \(f'(x)\neq 0\) and so \(\gcd(f(x),f'(x)\) has \(\alpha\)
as a root, contradicting the fact that \(f(x)\) is irreducible.
Thus \(f'(x)=0\) and so \(n=0 \bmod p\).
Again a contradiction to the assumption of \(n>1\)
and \(f(x)\) has no repeated roots and so is separable
and we have shown that $F$ is perfect.


%%%%%
\item
\begin{excopy}
Show that every finite $F$ is perfect. (Hint:
The function \(a\mapsto a^p\) is always an injection \(F\rightarrow F\).)
\end{excopy}

In Lemma~32 (\cite{Rotman98}) we saw that this mapping
is a ring homomorphism. Since \(a^p-b^p = (a-b)^p\)
and so if \(a^p=b^p\) then \((a-b)^p=0\) and so \(a=b\)
and this mapping is an injection.
Since $F$ is finite, any injection must also be surjective,
and so every element of $F$ must have a $p$-th root.
From the pervious exercise~\ref{ex:PerfProot} $F$ is perfect.


   %%%%%%%%%%%%%
\end{myenumerate}
%%%%%%%%%%%%%%%%%


%%%%%%%%%%%%%%%%%%%%%%%%%%%%%%%%%%%%%%%%%%%%%%%%%%%%%%%%%%%%%%%%%%%%%%%%
%%%%%%%%%%%%%%%%%%%%%%%%%%%%%%%%%%%%%%%%%%%%%%%%%%%%%%%%%%%%%%%%%%%%%%%%
%%%%%%%%%%%%%%%%%%%%%%%%%%%%%%%%%%%%%%%%%%%%%%%%%%%%%%%%%%%%%%%%%%%%%%%%
\chapterTypeout{The Galois Group}

%%%%%%%%%%%%%%%%%%%%%%%%%%%%%%%%%%%%%%%%%%%%%%%%%%%%%%%%%%%%%%%%%%%%%%%%
%%%%%%%%%%%%%%%%%%%%%%%%%%%%%%%%%%%%%%%%%%%%%%%%%%%%%%%%%%%%%%%%%%%%%%%%
\section{Notes}

In page~62, proof of Theorem~58 refers to
\begin{quotation}
the first isomorphism theorem for groups
\end{quotation}
It would be nice to refer to its full presentation and proof
in \textbf{Theorem~G.5} on page~114.

In Exercise~79, instead of
\begin{quotation}
Let \(f(x)\in F[x]\), let \(E/F\) be a splitting field, and \ldots
\end{quotation}
it would be more clear to explicitly say
\begin{quotation}
\ldots\ be a splitting field of \(f(x)\), and \ldots
\end{quotation}


%%%%%%%%%%%%%%%%%%%%%%%%%%%%%%%%%%%%%%%%%%%%%%%%%%%%%%%%%%%%%%%%%%%%%%%%
%%%%%%%%%%%%%%%%%%%%%%%%%%%%%%%%%%%%%%%%%%%%%%%%%%%%%%%%%%%%%%%%%%%%%%%%
\section{Exercises (page 63)}

%%%%%%%%%%%%%%%%
\begin{myenumerate}

%%%%%
\item % 78
\begin{excopy}
Let \(f(x)\in F[x]\) be an irreducible polynomial of degree $n$,
and let \(E/F\) be a splitting field of \(f(x)\).
\begin{itemize}
 \item[(i)]
   Prove that \(n \mid [E:F]\).
 \item[(ii)]
   Prove that if \(f(x)\) is separable, then \(n \mid \left|\Gal(E/F)\right|\).
\end{itemize}
\end{excopy}

\begin{itemize}
 \item[(i)]
 Let \(\alpha\in E\) be a root of \(f(x)\), then \([F(\alpha):F]=n\).
 We also have \([E:F]=[E:F(\alpha)][F(\alpha):F]\) and so \(n\mid [E:F]\).
 \item[(ii)]
    From Theorem~56 (\cite{Rotman98}), we have \(\Gal(E/F)=[E:F]\)
    and using (i) we get \(n \\mid \left|\Gal(E/F)\right|\).
\end{itemize}

%%%%%
\item % 79
\begin{excopy}
Let
\label{ex:trans}
\(f(x)\in F[x]\), let \(E/F\) be a splitting field, and let
\(G=\Gal(E/F)\) be the Galois group.
\begin{itemize}
 \item[(i)]
   If \(f(x)\) is irreducible, then $G$
   \index{transitively!act}
   acts \textbf{transitively} on the set of all roots of \(f(x)\)
   (if \(\alpha\)and \(\beta\) are any two roots of \(f(x)\) in $E$,
   there exists \(\sigma\in G\) with \(\sigma(\alpha)=\beta\).
   (Hint: Lemma~50).
 \item[(ii)]
   If \(f(x)\) has no
   \index{repeated roots}
   repeated roots and $G$ acts transitively on the roots,
   then \(f(x)\) is irreducible. Conclude, after comparing with Exercise~2,
   that irreducible polynomials are analogous to regular polygons.
   (Hint: If \(f(x)=g(x)h(x)\), then the \(\gcd(g(x),h(x))=1\);
    if \(\alpha\) is a common root of \(g(x)\) and \(h(x)\).)

\end{itemize}
\end{excopy}

\begin{itemize}
 \item[(i)]
   We will build \(\sigma\) in two stages.
   Using \(F=F'\) and \(\sigma=\id_{|F}\) in Lemma~50 (\cite{Rotman98}),
   we also have \(p^*(x)=p(x)\) and the lemma shows
   that for any root of \(p(x)\) can be mapped to any root of \(p(x)\)
   extending the identity on $F$.
   Thus, given any \(\alpha, \beta \in F\) we get
   \(\tilde{\sigma}: F(\alpha)\rightarrow F(\beta)\).

   Now using Theorem~51 (\cite{Rotman98}) with
   \(F(\alpha)\) and \(F(\beta)\) playing the role of $F$ and $F'$
   and \(E=E'\), we can extend \(\tilde{\sigma}\) into \(\sigma:E\rightarrow E\)
   with the desired requirements

 \item[(ii)]
   By negation, assume \(f(x)\) is reducible and factorize by
   \(f(x)=g(x)h(x)\).
   Now let \(\alpha\) and \(\beta\)
   be roots of \(g(x)\) and \(h(x)\) respectively.
   By $G$ acting transitively, there is \(\sigma\in G\) such that
   \(\sigma(\alpha)=\beta\). Now since \(\sigma(f(x))=f(x)\) and
   by homomorphism properties
   \begin{equation*}
   f(\beta) = \sigma(f(\alpha)) = 0.
   \end{equation*}
   Thus \(\beta\)  is a common roor of  \(g(x)\) and \(h(x)\)
   contradiction to the assumption that \(f(x)\) has no repeated roots.
\end{itemize}

%%%%%
\item
\begin{excopy}
Let $E$ be the splitting field of \(f(x)=x^4-10x^2+1\) over \Q.
Find \(\Gal(E/\Q)\).
(Hint. See Exercise~67 and Example~20. The roots of \(f(x)\) are
\[
  \sqrt{2}+\sqrt{3},\quad
  \sqrt{2}-\sqrt{3},\quad
  -\sqrt{2}+\sqrt{3},\quad
  -\sqrt{2}-\sqrt{3}.\textrm{)}\]
\end{excopy}

In Example~20 (\cite{Rotman98}) we saw that \([E:Q]=4\).
Since \(f(x)\) is separable, by Theorem~56 (\cite{Rotman98})
we have \(G=\left|\Gal(E/\Q)\right|=4\).


There are two non-isomorphic groups of order $4$.
The cylcic \Z{4} and \(\Z_2\times\Z_2\). We will show that
$G$ is isomorphic to the latter.

Clearly, \([\Q(\sqrt{2}:\Q]=2\) with a base\(\{1,\sqrt{2}\}\)
and \([E:\Q(\sqrt{2}]=2\) with a base\(\{1,\sqrt{3}\}\).
Now using Lemma~49 (\cite{Rotman98}) we have
the base \(\{1, \sqrt{2}, \sqrt{3}, \sqrt{2}\cdot\sqrt{3}\}\)
of $E$ over \Q.

For any \(\sigma\in G\) we have
  \(\left(\sigma(\sqrt{2})\right)^2 = 2\)
and
  \(\left(\sigma(\sqrt{3})\right)^2 = 3\).
Since in $E$, both the polynomials \(x^2-2\) and \(x^2-3\)
have exactly two solutions, we conclude that
\begin{eqnarray} \label{eq:sigma23}
\sigma(\sqrt{2}) & = & \pm\sqrt{2} \\
\sigma(\sqrt{3}) & = & \pm\sqrt{3}.
\end{eqnarray}
So \(\sigma\) is determined by the above which gives \(2\times 2=4\) choices.
We will see that all are valid.

For any \(u\in E\) there is a unique representation
\begin{equation*}
w = q_1 + \sqrt{2}\cdot q_2 + \sqrt{3}\cdot q_3 + \sqrt{2}\sqrt{3}\cdot q_4.
\end{equation*}
with \(q_i\in \Q\) for \(i=1,2,3,4\).
and so we define
\begin{equation*}
\sigma(w) = q_1 + \sigma(\sqrt{2}) q_2 + \sigma(\sqrt{3}) q_3 +
                  \sigma(\sqrt{2})\sigma(\sqrt{3}) q_4.
\end{equation*}

Now for any of the 4 choices for \(\sigma\) given by (\ref{eq:sigma23})
additivity is trivially maintained.
For multiplicative behaviour,
it is sufficient to check for the \(4^2-4 = 12\) unordered pairs
of base elements multiplied by scalars (of \Q).
But if $b$ is a base element and \(q\in Q\)
then \(\sigma(qb)=q\sigma(b)=\sigma(q)\sigma(b)\)
and so we can simply check the \(3^2-3=6\) unordered pairs
of non rational base elements. That is of \(\{\sqrt{2}, \sqrt{3}, \sqrt{6}\}\).

{
 \newcommand{\sigmaof}[1]{\sigma\left(#1\right)}

In the following we repeatedly use
\(\sigmaof{\sqrt{2}\sqrt{3}} =
  \sigmaof{\sqrt{6}} =
  \sigmaof{\sqrt{2}}\sigmaof{\sqrt{3}}\)
which holds by definition.


\begin{eqnarray*}
\sigmaof{\sqrt{2}\sqrt{2}}
 & = & \sigma(2) = \left(\pm\sqrt{2}\right)^2 =
     \sigmaof{\sqrt{2}} \sigmaof{\sqrt{2}} \\
%%%
\sigmaof{\sqrt{2}\sqrt{3}}
 & = &\sigma(\sqrt{2})\sigma(\sqrt{3}) \\
%%%
\sigmaof{\sqrt{2}\sqrt{6}}
 & = & \sigmaof{2\sqrt{3}} = 2 \sigmaof{\sqrt{3}} =
       \left(\pm\sqrt{2}\right)^2\sigmaof{\sqrt{3}} = \\
 & &   \sigmaof{\sqrt{2}}^2\sigmaof{\sqrt{3}} =
       \sigmaof{\sqrt{2}}\sigmaof{\sqrt{6}} \\
%%%
\sigmaof{\sqrt{3}\sqrt{3}}
 & = & \sigma(3) = 3 = \left(\pm\sqrt{3}\right)^2 =
       \sigmaof{\sqrt{3}} \sigmaof{\sqrt{3}} \\
%%%
\sigmaof{\sqrt{3}\sqrt{6}}
  & = & \sigmaof{3\sqrt{3}} = 3 \sigmaof{\sqrt{2}} =
        \left(\pm\sqrt{3}\right)^2\sigmaof{\sqrt{2}} = \\
  & &   \sigmaof{\sqrt{3}}^2\sigmaof{\sqrt{2}} =
        \sigmaof{\sqrt{3}}\sigmaof{\sqrt{6}} \\
%%%
\sigmaof{\sqrt{6}\sqrt{6}}
 & = & \sigma(6) = 6 = 2\cdot 3 =
       \left(\sigmaof{\sqrt{2}}\right)^2 \left(\sigmaof{\sqrt{3}}\right)^2 = \\
 & &  \left(\sigmaof{\sqrt{2}} \sigmaof{\sqrt{3}}\right)^2 =
     \sigmaof{\sqrt{6}} \sigmaof{\sqrt{6}}
%%%
\end{eqnarray*}
}

Thus such \(\sigma\)'s are indeed homomorphisms.
Since for every \(\sigma\in G\) we have \(\sigma^2=\id_{|E}\)
and so \(G \cong \Z_2\times\Z_2\).

%%%%%%%%%%%%%%%%%%%%%%%%%%%%% {
\iffalse
let \(w_1,w_2\in E\) have the representations:
\begin{equation*}
w_i = q_{i,1} + \sqrt{2}\cdot q_{i,2} + \sqrt{3}\cdot q_{i,3} +
      \sqrt{2}\sqrt{3}\cdot q_{i,4}.
\qquad\textrm{for}\;(i=1,2)
\end{equation*}
with \(q_{i,j}\in \Q\).

Now
\begin{eqnarray*}
w_1 w_2 & = & (q_{1,1}q_{2,1} + 2q_{1,2}q_{2,2} + 3 q_{1,3}q_{2,3} +
                                                   6q_{1,4}q_{2,4}) + \\
&&            (q_{1,2}q_{2,1} + q_{1,1}q_{2,2} +
               3q_{1,3}q_{2,4} + 3q_{1,4}q_{2,3})\sqrt{2} + \\
&&            (q_{1,3}q_{2,1} + q_{1,1}q_{2,3} +
               2q_{1,2}q_{2,4} + 2q_{1,4}q_{2,2})\sqrt{3} + \\
&&            (q_{1,4}q_{2,1} + q_{1,1}q_{2,4} +
               q_{1,2}q_{2,3} + q_{1,3}q_{2,2})\sqrt{2}\sqrt{3}.
\end{eqnarray*}
\fi
%%%%%%%%%%%%%%%%%%%%%%%%%%%%% }

   %%%%%%%%%%%%%
\end{myenumerate}
%%%%%%%%%%%%%%%%%

%%%%%%%%%%%%%%%%%%%%%%%%%%%%%%%%%%%%%%%%%%%%%%%%%%%%%%%%%%%%%%%%%%%%%%%%
%%%%%%%%%%%%%%%%%%%%%%%%%%%%%%%%%%%%%%%%%%%%%%%%%%%%%%%%%%%%%%%%%%%%%%%%
%%%%%%%%%%%%%%%%%%%%%%%%%%%%%%%%%%%%%%%%%%%%%%%%%%%%%%%%%%%%%%%%%%%%%%%%
\chapterTypeout{Roots of Unity}

%%%%%%%%%%%%%%%%%%%%%%%%%%%%%%%%%%%%%%%%%%%%%%%%%%%%%%%%%%%%%%%%%%%%%%%%
%%%%%%%%%%%%%%%%%%%%%%%%%%%%%%%%%%%%%%%%%%%%%%%%%%%%%%%%%%%%%%%%%%%%%%%%
\section{Notes}

%%%%%%%%%%%%%%%%%%%%%%%%%%%%%%%%%%%%%%%%%%%%%%%%%%%%%%%%%%%%%%%%%%%%%%%%
\subsection{Typographic Error (\ensuremath{GF})} \label{typo:GF1}

On page~67, starting from Lemma~66 till the proof of Theorem~67,
the wrong notation
\,\(GF(p^n)\)\,
is used three times instead of
\, \(\GF(p^n)\)\, (See also a note in \ref{typo:GF2}).
Note that on page~57, where the notion of this
\index{field!Galois}
\index{Galois field}
\textbf{Galois Field}
is introduced, the notation is well typeset.

%%%%%%%%%%%%%%%%%%%%%%%%%%%%%%%%%%%%%%%%%%%%%%%%%%%%%%%%%%%%%%%%%%%%%%%%
\subsection{Clarification}

On page 67, in the proof  of Theorem~67 we have:
\begin{quotation}
If \(\alpha\) is a primitive element,
\end{quotation}
It would be clearer if the reader would be reminded  that such \(\alpha\)
\emph{does exist}, as shown by Corollary~64 on page~65.

%%%%%%%%%%%%%%%%%%%%%%%%%%%%%%%%%%%%%%%%%%%%%%%%%%%%%%%%%%%%%%%%%%%%%%%%
\subsection{Notational --- pth}

In corollaries 71 and 72, there is a usage of the term
\textsl{a $p$th root}. It may be better within a slanted font context
to put a dash after the $p$ to get \textsl{a $p$-th root}.

%%%%%%%%%%%%%%%%%%%%%%%%%%%%%%%%%%%%%%%%%%%%%%%%%%%%%%%%%%%%%%%%%%%%%%%%
\subsection{Missing Trivial Argument in Proof}

In the proof of Corollary~72, one direction seems to be missing
from the proof. The following can begin the proof:

\begin{quotation}
If $c$ has a $p$th root \(\alpha\) in $F$, then obvioulsy the term
\(x-\alpha\)
divides \(x^p-c\) and thus
% the latter
\(x^p-c\)
is not irreducible.
\end{quotation}

%%%%%%%%%%%%%%%%%%%%%%%%%%%%%%%%%%%%%%%%%%%%%%%%%%%%%%%%%%%%%%%%%%%%%%%%
%%%%%%%%%%%%%%%%%%%%%%%%%%%%%%%%%%%%%%%%%%%%%%%%%%%%%%%%%%%%%%%%%%%%%%%%
\section{Exercises (page 70)}

%%%%%%%%%%%%%%%%
\begin{myenumerate}

%%%%% 81
\item
\begin{excopy}
Prove that if $F$ is an infinite field, then its
multiplicative group \(F^{\#}\) is never cyclic.
(Hint. To eliminate the possibility
 \(F^{\#} = \lrangle{u}\),  consider the cases of
characteristic $0$ and characteristic \(p>0\)
separately; the latter case should
be further subdivided into cases:
\index{transcendental}
$u$ transcendental over the prime field
\Zp\ and $u$ algebraic over \Zp.)
\end{excopy}

\textbf{Note:} Given a group $G$, if being a \emph{cyclic}
group generated by $u$
would mean that for any \(g\in G\) there exist a \textbf{positive} integer $n$
such that \(g=u^n\) then this exercise would have been much simpler.
By simply looking for a power \(j>0\) such that \(u^{-1}=u^j\),
followed by an argument of that the set \(\{u^i\}_{i\in \N}\) is finite.

\iffalse
Assume \(\fchar F = 0\) and by negation \(F^{\#} = \lrangle{u}\).
We have already seen (in Theorem~31 \cite{Rotman98}) that its prime
field is isomorphic to \Q.
Now let $n$ be the minimal absolute such that
\(u^n\in \Q\) or \(u^{-n}\in\Q\).
\fi

Let  $F$ be an infinite field, and by negation
let $u$ be a generator of \(F^{\#}\).
We will treat the cases:
\begin{itemize}
 \item \(\fchar F \neq 2\)
 \item \(\fchar F = 2\)
\end{itemize}
separately.

Assume \(\fchar F \neq 2\), then \(-1 \neq 1\). Let $n$ be an integer
with a minimal absolute value such that \(u^n=-1\).
Then \(u^{2n}=1\) and it is easy to see that
the for any integer $k$, we have \(u^k=u^{k+2n}\) and
thus $u$  generates only $2n$ elements contradiction
to assumption of generating all of $F$.

Now assume \(\fchar F = 2\) (actually any finite characteristic
will work with trivial modification). We will deal
with the case of $u$ being algebraic or not separately.

If $u$ is algebraic over
\(\{0,1\}\subset F\), then there is a polynomial \(f(x)\in \Z_2[x]\)
of degree $d$ such that \(f(u)=0\).
Now for any \(n>0\),
let \(r_n(x)\in\Z_2[x]\) be the residue
of \(x^n\) divided by \(f(x)\).
Thus \(u^n = r_n(u)\). But the set \(\{r_n(u): n\in \N\}\)
can have at most \(2^d\) elements, since that is the number of
possibile polynomials in \(\Z_2[x]\) with degree of less than $d$.
Contradiction to generation of the inifinite $F$.

Assume $u$ is transcendental (non algebraic).
Since \(u\neq 0 \neq u^2\) and so
Since \(u\neq u^2+u \neq u^2\).
So there must be some integer $n$, such that
\(u^n=u^2+u\) and surely \(0\neq n \neq 1\).
Now $u$ is a root of either the polynomial
\(x^n+x^2+x\) when \(n>1>0\)
or the polynomial \(x^{-n+2}+x^{-n+1}+1\) when \(n<0\).
A contradiction to assumption of $u$ being not algebraic.


   %%%%%%%%%%%%%
\end{myenumerate}
%%%%%%%%%%%%%%%%%


%%%%%%%%%%%%%%%%%%%%%%%%%%%%%%%%%%%%%%%%%%%%%%%%%%%%%%%%%%%%%%%%%%%%%%%%
%%%%%%%%%%%%%%%%%%%%%%%%%%%%%%%%%%%%%%%%%%%%%%%%%%%%%%%%%%%%%%%%%%%%%%%%
%%%%%%%%%%%%%%%%%%%%%%%%%%%%%%%%%%%%%%%%%%%%%%%%%%%%%%%%%%%%%%%%%%%%%%%%
\chapterTypeout{Solvability by Radicals}

%%%%%%%%%%%%%%%%%%%%%%%%%%%%%%%%%%%%%%%%%%%%%%%%%%%%%%%%%%%%%%%%%%%%%%%%
%%%%%%%%%%%%%%%%%%%%%%%%%%%%%%%%%%%%%%%%%%%%%%%%%%%%%%%%%%%%%%%%%%%%%%%%
\section{Notes}

%%%%%%%%%%%%%%%%%%%%%%%%%%%%%%%%%%%%%%%%%%%%%%%%%%%%%%%%%%%%%%%%%%%%%%%%
\subsection{Heavy Weight on Exercises}

The proof of Lemma~73 (page~72) is using \emph{all} of
that chapter's exercises (82---85).
Note the exercises dependencies: 83 \(\rightarrow\) 84 \(\rightarrow\) 85.

For now, % October 6, 2001
I think solving exercise~83(ii) actually requires
dealing with notions of
\index{normal extension}
\index{extension!normal}
\emph{normal extension}.
\iffalse
and may be even
 \index{algebraic closure}
 \index{closure!algebraic}
 \emph{algebraic closure}.
\fi

%%%%%%%%%%%%%%%%%%%%%%%%%%%%%%%%%%%%%%%%%%%%%%%%%%%%%%%%%%%%%%%%%%%%%%%%
\subsection{Generous Description}

Page 73, in the proof of Theorem~74 it says:
\begin{quotation}
\mldots, and then refined so that \mldots
\end{quotation}
may be more generous description would be to say:
\begin{quotation}
\mldots, and then this last step refined so that \mldots
\end{quotation}

%%%%%%%%%%%%%%%%%%%%%%%%%%%%%%%%%%%%%%%%%%%%%%%%%%%%%%%%%%%%%%%%%%%%%%%%
\subsection{Pure --- or Trivial? (Not Sure)}

Page 74, still the same proof of Theorem~74,
refering to the tower
\begin{equation*}
F = R_0 \subset F(\omega) \subset R_1(\omega)
    \subset \cdots \subset R_t(\omega) = R'
\end{equation*}
it says:
\begin{quotation}
Notice that each extension in this tower is pure \mldots.
\end{quotation}
Some extensions may be trivial!? and so may more accurate to say:
\begin{quotation}
Notice that each extension in this tower is trivial or pure \mldots.
\end{quotation}

%%%%%%%%%%%%%%%%%%%%%%%%%%%%%%%%%%%%%%%%%%%%%%%%%%%%%%%%%%%%%%%%%%%%%%%%
\subsection{Unclear Requirement (in Exercise)}

See solution to Exercise~\ref{ex:84ii}(ii).

%%%%%%%%%%%%%%%%%%%%%%%%%%%%%%%%%%%%%%%%%%%%%%%%%%%%%%%%%%%%%%%%%%%%%%%%
%%%%%%%%%%%%%%%%%%%%%%%%%%%%%%%%%%%%%%%%%%%%%%%%%%%%%%%%%%%%%%%%%%%%%%%%
\section{Exercises (page 75)}

%%%%%%%%%%%%%%%%
\begin{myenumerate}

%%%%% 82
\item
\begin{excopy}
If \(E/F\) is a radical extension over $F$, then there is a radical tower
\begin{equation*}
F = B_0 \subset B_1 \subset \cdots \subset B_t
\end{equation*}
\index{pure extension}
\index{extension!pure}
with each \([B_{i+1}:B_i]\) a pure extension of prime type.
(Hint. If \(\alpha^n\in F\) and \(n=pm\), then there is a tower of fields
\(F\subset F(\alpha^p)\subset F(\alpha)\).)
\end{excopy}

Since \(E/F\) is a radical extension, we have an extension tower:
\begin{equation*}
F = R_0 \subset R_1 \subset \cdots \subset R_s
\end{equation*}
Where each \(R_{i+1}/R_i\) is a pure extension.
We will show that each such extension can be refined by
a sub-tower of pure extension of prime type.

Let \(R_{i+1}=R(\alpha)\) where \(\alpha^n\in R_i\).
Let \(n=\prod_{j=1}^l p_j\)
where \(p_j\) are (possibly repeated) prime numbers.
Set \(m_k = \prod_{j=k}^l p_j\)
and \(\beta_k = \alpha^{m_k}\).
The \(m_k\)'s have the equalities:
\(m_1 = n\), \(m_l = 1\) and \(m_k = p_k m_{k+1} \).
The \(\beta_k\)'s have the equalities:
\(\beta_1 = \alpha^n\),
\(\beta_l = \alpha\) and
\[\beta_k = \alpha^{m_k} = \alpha^{p_k m_{k+1}} = \beta_{k+1}^{p_k}.\]
Construct the following tower
\begin{equation} \label{eq:primtower}
R_i = R_i(\beta_1)
   \subset R_i(\beta_2)
   \cdots \subset R_i(\beta_k) \subset R_i(\beta_{k+1}) \cdots
   \subset R_i(\beta_l) = R_{i+1}.
\end{equation}
Each step in (\ref{eq:primtower}) is a radical pure extension of prime type,
since
\[B = R_i(\beta_k)  \subset B(\beta_{k+1}) =  R_i(\beta_{k+1})\]
and \(\beta_k = \beta_{k+1}^{p_k} \in B\).


%%%%%
\item
\begin{excopy}
Let \(B/F\) be a finite extension. Prove that there is an extension \(K/B\)
so that \(K/F\) is a splitting field  of some polynomial \(f(x)\in F[x]\).
(Hint. Since \(B/F\) is finite, it is algebraic, and there are elements
\seqalphn\ with \(B=F(\seqalphn)\). If \(p_i(x)=F[x]\) is the irreducible
polynomial of \(\alpha_i\),
take $K$ to be a splitting field of \(f(x)=p_1(x)\cdots p_n(x)\).)
\end{excopy}

Following the hint, we just need to embed $B$ within
the splitting field $K$ of
\begin{equation*}
f(x)=\prod_{i=1}^n p_1(x)\in F(x).
\end{equation*}
First we esure that linearly,
\seqalphn\ spans $B$ and are independent over $F$.

Define the embedding \(\sigma : B \rightarrow K\) over $F$
by \(\sigma(\alpha_i)=\tau_i\), where
for any \(\alpha_i\) we choose a solution \(\tau_i\) for \(p_i(x)\) in $K$.

%% Is it necessary to show that \(\sigma\) preserve field multiplication?



%%%%%
\item
\begin{excopy}
\begin{itemize}
 \item[(i)]
   If $B$ and $C$ are subfields of a field $E$, then their
   \index{compositum!field}
   \index{compositum@\(\vee\)}
   \textbf{compositum} \(B\vee C\) is the intersection of all the subfields
   of $E$ containing $B$ and $C$.
   Prove that if \(\seqalphn\in E\),
   then \(F(\alpha_1)\vee \cdots \vee F(\alpha_n) = F(\seqalphn)\).
 \item[(ii)]
   Prove that any splitting field \(K/F\) containing $B$ (as in Exercise~83)
   has a form \(K = B_1 \vee \cdots \vee B_r\), where each \(B_i\)
   is isomorphic to $B$ via an isomorphism which fixes $F$.
   (Hint. If \(\Gal(K/F)=\{\seqn{\sigma}\}\),
   then define \(B_i=\sigma_i(B)\).)
\end{itemize}
\end{excopy}

\begin{itemize}
 \item[(i)]
   Since \(F(\alpha_1)\vee \cdots \vee F(\alpha_n)\) is a field
   containing $F$ and all of \seqalphn, then
   then by minimality,
   \[F(\seqalphn) \subset F(\alpha_1)\vee \cdots \vee F(\alpha_n).\]
   Now It is clear that \(F(\alpha_i)\subset F(\seqalphn)\)
   for all \(\{F(\alpha_i)_{i=1}^n\). Hence the compositum
   of these must be contained in any field containing all of them, that is
   \[F(\alpha_1)\vee \cdots \vee F(\alpha_n) \subset F(\seqalphn).\]
 \item[(ii)] \label{ex:84ii}
   The wording in parenthesis:
   \begin{quotation}
   \mldots (as in Exercise~83) \mldots
   \end{quotation}
   are not clear. Surely if this is just an illustrative remark,
   then $K$ may contain a solution to some irreducible polynomial in \(F[x]\)
   that has no solution in $B$. In such case, $K$ \emph{cannot}
   be equal to compositum of ``copies'' of $B$.

   Hence, $K$ must have specific construction definition, or
   some miminality requirements. So let us assume
   that
   \begin{itemize}
    \item
      \([B:F]<\infty\).
    \item
      $K$ \emph{is} a
      \index{split closure}
      \index{closure!split}
      {split closure} of \(B/F\). As the footnote~(12) defines:
      in page~75 of \cite{Rotman98}:
      \begin{quotation}
      A smallest such extension \(K/F\) is called \textbf{split closure}
      of \(B/F\).
      \end{quotation}
      Seems like an argument missing to show that
      such smallest extension exists.
      Basically, showing that an intersection of splitting fields is
      a splitting field.
   \end{itemize}

   First let us establish the following
   (\cite{McCarthy91} Theorem~16, page~16):
   %%%%%%%%%%%%%%%%%%%%%%%%%%%%%%%%
   \begin{llem}[Normal extension] \label{lem:normal:ext}
   If \(K/k\) is a splitting field of \(f(x)\in k[x]\)
   then any irreducible polynomial \(g(x)\in k[x]\)
   that has a root in $K$ splits in $K$.
   \end{llem}
   %%%%%%%%%%%%%%%
   \textbf{Proof.}
   Let \seqan\ be the roots of \(f(x)\in k[x]\) and
   \begin{equation} \label{eq:K:k:seqan}
   K=k(\seqan).
   \end{equation}
   Let \gx\ be an irreducible polynomial in \(k[x]\) with a root \(b\in K\).
   Let $L$ be a splitting field of \gx\ over $K$ and let \(b'\in L\)
   be any root of \gx\ in $L$.
   By Lemma~50 (\cite{Rotman98}), there is an isomorphism \(\sigma\)
   of \(k(b)\)
   onto \(k(b')\) which extends the identity on $k$, such that
   \(\sigma(b)=b'\). Under \(\sigma\), \fx\ is mapped to itself.
   It is clear that \(K=k(b)(\seqan)\) is a splitting field
   of \fx\ over \(k(b)\). Furthermore, \(K(b')= k(\seqan,b')\) is
   a splitting field of \fx\ over \(k(b')\).

   By Theorem~51-(ii) (\cite{Rotman98}) \(\sigma\) can be extended
   into isomorphism: \(tau: K \rightarrow K(b)\).
   Clearly, \(\tau\) maps any root of \fx\ to a root of \fx.
   From (\ref{eq:K:k:seqan}) and the fact that \seqan\ are all
   algebraic over $k$, there is a polynomial
    \(h(\seqxn)\in k[\seqxn]\) such that \(b = h(\seqan) \in K\).
   Then,
   \begin{equation*}
   b' = \tau(b) = \tau\left(h(\seqan)\right) =
        h(\tau(a_1),\ldots,\tau(a_n)) \in K.
   \end{equation*}
   Thus \gx\ splits in $K$.
   \proofend
   %%%%%%%%%%%%%%%%%%%%%%%%%%%%%%%%

   Notes:
   \begin{itemize}
    \item
      Theorem~81 (\cite{Rotman98}, page~80)  (iii) \(\Rightarrow\) (ii)
      could not simply be used here, because it assumes separability.
      This textbook notion of
      \index{normal extension}
      \index{extension!normal}
      \emph{normal extension} is restricted to separable extensions.
    \item
      Note that although Theorem~81 is introduced in the book later
      then this current chapter, its proof is based on results
      we already have now.
   \end{itemize}

   The next lemma will show that the field generated as in Exercise~83
   can always be embedded in an splitting field \(\hat{K}/F\) containing $B$.
   But first let us have a definition:

   \textbf{Definition.}
   (See \cite{Lang94} page~224)
   Let \(E/F\) be field extension, and let \(\alpha\in E\) be
   algebraic over $F$. The
   \index{Irreducible polynomial}
   \index{polynomial!Irreducible}
   \index{Irr@\(\Irr(a,F,x)\)}
   \textbf{irreducible polynomial}
    in \(F[x]\) with leading coefficient $1$,
   for which \(\alpha\) is root will be denoed by \(\Irr(\alpha,F,x)\).

   Note. It is easy to see that \(\Irr(\alpha,F,x)\) exists and is unique.


   \begin{llem} \label{lem:BF:splitclos}
   Let \(B/F\) be a finite extension such that
   \begin{equation} \label{eq:B:Fa}
   B=F(\seqan)
   \end{equation}
   and let \(\hat{K}/F\)
   be a splitting field containing $B$.
   Then,
   \begin{itemize}
    \item[(1)]
      The polynomials \(p_i(x) = \Irr(a_i,F,x)\) split in \(\hat{K}\)
      as \(p_i(x) = \prod_{j=1}^{n_i} (x - b_{ij})\) where
      \(n_i = \gdeg(p_i(x))\) and
      \(b_{ij}\in \hat{K}\) for \(i=1,\ldots,n\) and \(j=1,\ldots,n_i\)
      \textnormal{(we can assume  \(b_{i1}=a_i\))}.
    \item[(2)]
      The field
      \begin{equation} \label{eq:Kaibij}
      K = K_{\seqan}
        = F\bigl(\{b_{ij}: 1\leq i \leq n,\, 1\leq j \leq n_i\}\bigr)
      \end{equation}
      generated by all the roots of \(f(x)=\prod_{i=1}^n p_i(x) \in F[x]\)
      is a splitting field of \px\ and
      satisfies:  \(B\subset K \subset \hat{K}\).
      \textnormal{Note that \(N=\gdeg(f(x))=\sum_{i=1}^n n_i\)}.

    \item[(3)]
      The field $K$ in (\ref{eq:Kaibij}) is a
      \index{split closure}
      \index{closure!split}
      split closure of \(B/F\).
   \end{itemize}
   \end{llem}
   %%%%%%%%%%%%%%%
   \textbf{Proof.}
   By the above lemma~(\ref{lem:normal:ext}), since \(a_i\in B\subset K\)
   all the polynomials \(p_i(x)\) split in \(\hat{K}\).
   Thus the \(\hat{K}\) contains the splitting field $K$.
   The construction of $K$ (see (\ref{eq:B:Fa}) depends on the choice of
   of \(\seqan\).
   But since $K$ satisfies the requirements from \(\hat{K}\),
   it must contain any $K$ constructed with other
   \(\seq{a'}{n'}\) such that \(B=F(\seq{a'}{n'})\).
   In that sense, $K$ is minimal and so is the split closure of \(B/F\).
   \proofend

   Back to the exercise.

   So let \(G=\Gal(K/F)\) and denote the compositum
   \(\hat{K} = \bigvee_{\sigma\in G} \,\sigma(B)\).
   Clearly \(\hat{K}\subset K\).

   To show the opposite inclusion, let \(b\in K\).
   Since $K$ is the split closure of \(B/F\),
   we can assume --- using lemma~(\ref{lem:BF:splitclos}) % ==
   that $b$ is the evaluation \(h(b_{ij})\)
   of some polynomial \(h(\seq{x}{N})\in F[\seq{x}{N}]\)
   of $N$ variables
   where \(N=\gdeg(f(x)\) as in that lemma.

   By Exercise~79(i) (Note that \emph{there} $E$ can be ``larger'' than
   a split of ``just'' \fx)
   \index{transitively!act}
   $G$ acts transitively,
   and so for any \(b_{ij}\) where
   \(i=1,\ldots,n\) and \(j=1,\ldots,n_i\)
   there exists \(\sigma\in G\) and \(a_i\) such that
   \(b_{ij} = \sigma(a_i) \in \sigma(B_i)\) and so
   \(b \in \bigvee_{\sigma\in G} \,\sigma(B)\).
   Therefore, \(K = \bigvee_{\sigma\in G} \,\sigma(B)\).


 \iffalse
   Before focusing on the opposite inclusion, we
   will have some observation on the \(K/B\) extension.
   Let \(B=F(\seqan)\) where \(a_i\in B\setminus F\) for \(i=1,\ldots,n\).
   \index{Irreducible polynomial}
   \index{polynomial!Irreducible}
   \index{Irr@\(\Irr(a,F,x)\)}
   Let \(p_i(x)=\Irr(a_i,F,x)\in F[x]\) be the irreducible
   polynomial (see \cite{Lang94} page~224) of \(a_i\) over $F$
   (for \(i=1,\ldots,n\)).
   By the above lemma, since \(a_i\in B\subset K\)
   all the polynomials \(p_i(x)\) split in $K$.
   Let the split be: \(p_i(x) = \prod_{j=1}^{n_i} (x - b_{ij})\) where
   \(n_i = \gdeg(p_i(x))\) and  \(b_{i1}=a_i\).
   Note that \(b_{ij}\in K\) for \(i=1,\ldots,n\) and \(j=1,\ldots,n_i\).
   Thus the $K$ contains the splitting field \(K'\) of \(\prod_{i=1}^n p_i(x)\).
   The construction of \(K'\) depends on the choice
   of \(\seqan\).

   Since $K$ is a splitting field of \(\prod_{i=1}^n p_i(x)\)
   (of Exercise~83), let \(seq{\beta}{m}\) be all the roots
   of \(\{p_i(x)\}_{i=1}^n\) (of course, all of \(\{\alpha_i\}_{i=1}^n\) are
   among them).
   So there is a polynomial \(h(\seq{x}{m}) \in F(\seq{x}{m})\)
   such that \(b=h(\seq{x}{m}\). Clearly from Lemma~51 (\cite{Rotman98})
   every root \(beta_k\)
   there is \(\sigma\in \Gal(K/F)\) and \(\alpha_i\) such that
   \(\sigma(\alpha_i)=\beta_k\).

   Now the irreducible polynomial
   \(f_b(x)\in F[x]\) with root $b$ must split in $K$.


   %the above lemma, shows that
   %any polynomial in \(F[x]\) with root in $B$, splits in $K$.
   Since \(B=F(\seqalphn)\), any isomorphism \(\sigma\in \Gal(K/F)\)

 \fi

\end{itemize}

%%%%%
\item
\begin{excopy}
Using Exercise~84, prove that any splitting field \(K/F\) containing a radical
extension \(R_t/F\) (as in Exercise~83) is itself a radical extension.
Conclude that, in the definition of solvable by radicals, one can assume
that the last field \(B_t\) is a splitting field of some polynomial over $F$.
\end{excopy}

Let the radical extension \(R_t/F\) have the tower
\begin{equation*}
 F = R_0 \subset R_1 \subset \cdots \subset R_t
\end{equation*}
Now assume each step \(R_{i}/R_{i-1}\) is a pure extension of type \(m_i\).
such that \(R_{i+1}=R_i(a_i)\), where \(a_i^{m_i}\in R_i\).
Factorize \(p_i(x) = x^{m_i} - 1 =\prod_j q_{ij}(x)\)
where \(q_{ij}(x) \in R_i[x]\) be irreducible polynomials.
For any \(i=1,\ldots,t\) let \(\hat{p}_i(x)=q_{ij}(x)\) such that
\(q_{ij}(a_i) = 0\).
\index{Irr@\(\Irr(a,F,x)\)}
Actually, \(\hat{p}_i(x) = \Irr(a_i,R_i,x) \mid (x^{m_i}-1)\).
Let \(\alpha_{ij}\) where \(j=1,\ldots,\gdeg(\hat{p}_i(x)\)
be the roots of \(\hat{p}_i(x)\).

Using lemma~\ref{lem:normal:ext}, the polynomials \(\hat{p}_i(x)\)
and the polynomial
\begin{equation*}
 f(x) = \prod_{i=1}^t \hat{p}_i(x)
\end{equation*}
must split in $K$.
Assuming (``as in Exercise~83'') that $K$ is the split closure of \(R_t/F\)
we saw in lemma~\ref{lem:BF:splitclos}(3)
that
\begin{equation*}
 K = R_t\bigl(\bigl\{\alpha_{ij}: 1\leq i\leq t,
                             \, 1\leq j\leq \gdeg(\hat{p}_i(x))\bigr\}\bigr)
\end{equation*}
We can refine the step \(K/R_t\) to a tower where each step is
done by adjoining the \(b_{ij}\)'s step by step,
where each is a pure extension, since \(b_{ij}^{m_i} \in R_i \subset R_t\).

   %%%%%%%%%%%%%
\end{myenumerate}
%%%%%%%%%%%%%%%%%


%%%%%%%%%%%%%%%%%%%%%%%%%%%%%%%%%%%%%%%%%%%%%%%%%%%%%%%%%%%%%%%%%%%%%%%%
%%%%%%%%%%%%%%%%%%%%%%%%%%%%%%%%%%%%%%%%%%%%%%%%%%%%%%%%%%%%%%%%%%%%%%%%
%%%%%%%%%%%%%%%%%%%%%%%%%%%%%%%%%%%%%%%%%%%%%%%%%%%%%%%%%%%%%%%%%%%%%%%%
\chapterTypeout{Independence of Characters}

\index{character}

%%%%%%%%%%%%%%%%%%%%%%%%%%%%%%%%%%%%%%%%%%%%%%%%%%%%%%%%%%%%%%%%%%%%%%%%
%%%%%%%%%%%%%%%%%%%%%%%%%%%%%%%%%%%%%%%%%%%%%%%%%%%%%%%%%%%%%%%%%%%%%%%%
\section{Notes}

%%%%%%%%%%%%%%%%%%%%%%%%%%%%%%%%%%%%%%%%%%%%%%%%%%%%%%%%%%%%%%%%%%%%%%%%
\subsection{Not a Vector Space}

The footnote (13) on page~76 says:
\begin{quotation}
All the characters in a field $E$ form
a vector space \(V(G,E)\) over $E$ \mldots
\end{quotation}
This could be misleading, since such vector space cannot consist
of just characters, since the zero vector (\(x\mapsto 0\)),
must be in \(V(G,E)\) but is surely not a character.

%%%%%%%%%%%%%%%%%%%%%%%%%%%%%%%%%%%%%%%%%%%%%%%%%%%%%%%%%%%%%%%%%%%%%%%%
%%%%%%%%%%%%%%%%%%%%%%%%%%%%%%%%%%%%%%%%%%%%%%%%%%%%%%%%%%%%%%%%%%%%%%%%
%%%%%%%%%%%%%%%%%%%%%%%%%%%%%%%%%%%%%%%%%%%%%%%%%%%%%%%%%%%%%%%%%%%%%%%%
\chapterTypeout{Galois Extensions}

%%%%%%%%%%%%%%%%%%%%%%%%%%%%%%%%%%%%%%%%%%%%%%%%%%%%%%%%%%%%%%%%%%%%%%%%
%%%%%%%%%%%%%%%%%%%%%%%%%%%%%%%%%%%%%%%%%%%%%%%%%%%%%%%%%%%%%%%%%%%%%%%%
\section{Notes}

%%%%%%%%%%%%%%%%%%%%%%%%%%%%%%%%%%%%%%%%%%%%%%%%%%%%%%%%%%%%%%%%%%%%%%%%
\subsection{Wrong Conjugate Field}

Example~30 on page~82, defines: \(\alpha=\sqrt[3]{2}\) and
\(\omega = e^{2\pi i/3}\).
Now the last line of the example says:
\index{conjugate!field}
\begin{quotation}
\(C=\Q(\alpha)\), then \(\Q(\alpha^2)\) is a conjugate of $C$,
and \(\Q(\alpha^2)\neq C\).
\end{quotation}

But this is not true, since \(\alpha^2=2/\alpha\) and so
we have \(\Q(\alpha)=\Q(\alpha^2)\).
The actual conjugates of \(C=\Q(\alpha)\)
are  \(\Q(\alpha\omega)\)
and  \(\Q(\alpha\omega^2)\).

%%%%%%%%%%%%%%%%%%%%%%%%%%%%%%%%%%%%%%%%%%%%%%%%%%%%%%%%%%%%%%%%%%%%%%%%
%%%%%%%%%%%%%%%%%%%%%%%%%%%%%%%%%%%%%%%%%%%%%%%%%%%%%%%%%%%%%%%%%%%%%%%%
\section{Exercises (page 82)}

%%%%%%%%%%%%%%%%
\begin{myenumerate} % 86-91

%%%%% % 86
\item
\begin{excopy}
If \(E/F\) is a Galois extension and $B$ is an intermeiate field,
then \(E/B\) is a Galois extension.
\end{excopy}

From Theorem~81(iii) (\cite{Rotman98}) there exists a polynomial
\(f(x)\in F[x]\) for which $E$ is a splitting field.
Now $E$ is also a splitting field of \((f(x)\in B[x]\) and so
\(E/B\) is a Galois extension.

%%%%%
\item
\begin{excopy}
If $F$ has characteristic \(\neq 2\) and \(E/F\) is a field extension
with \([E:F]=2\), then \(E/F\) is Galois.
\end{excopy}

Let \(a\in E\setminus F\) and let \(f(x)=\Irr(a,F,x)\).
Clearly \(1<\gdeg(f(x))\leq [E:F]=2\) and so \(\gdeg(f(x))=2\).
In \(E[x]\) we have \((x-a)\mid f(x)\) and so \(f(x)=(x-a)(x-a')\)
splits in $E$ with \(a'\in E\).

% If by negation \(a=a'\) then \(f'(x)=2(x-a)\).
If by negation \(a=a'\) then the linear coefficient of \(f(x)\)
would be \(-2a \in F\) and since \(\fchar F\neq2\)
we get \(a\in F\)  contradicting the choice of $a$.
Hence \(f(x)\) is a separable polynomial for which $E$ is a splitting field
and so \(E/F\) is a Galois extension.

%%%%%
\item
\begin{excopy}
Show that being Galois need not be transitive; that is,
if \(F\subset B\subset E\) and \(E/B\) and \(B/F\) are Galois, then
\(E/F\) need not be Galois.
(Hint: Consider \(\Q \subset \Q(\alpha) \subset \Q(\beta)\),
where \(\alpha\) is a square root of $2$
and \(\beta\) is a fourth root of $2$.)
\end{excopy}

Following the hint,
let \(\alpha=\sqrt{2}\) and \(\beta=\sqrt[4]{2}=\sqrt{\alpha}\).
Looking at the polynomials
\(x^2-2\) and \(x^2-\alpha\)
we conclude that both \(\Q(\alpha)/\Q\) and \(\Q(\beta)/\Q(\alpha)\)
are extensions of order $2$.
From previous exercise, both are Galois extensions.

Now as for \(\Q(\beta)/\Q\).
By Eisenstein criterion (Theorem~40 \cite{Rotman98})
the polynomial \(x^4-2\) is irreducible in \(\Q[x]\),
it has a root in \(\Q(\beta)\), namely \(\pm\beta\)
but it does not split there, since the other roots are
of the polynomial
\begin{equation*}
 x^2+\sqrt{2} = (x^4-2)/\left((x+\beta)(x-\beta)\right) = (x^4-2)/(x^2-\sqrt{2})
\end{equation*}
are \(\pm i\beta \notin \Q(\beta)\).
Hence  \(\Q(\beta)/\Q\) is not Galois extension.

%%%%%
\item
\begin{excopy}
Let \(E=F(\seqxn)\) and let $S$ be the subfield of all symmetric functions.
Prove that \([E:S]=n!\) and \(\Gal(E/S)\cong S_n\).
(Hint: Show that \(E/S\) is a splitting field of the separable polynomial
\(f(t)=\prod(t-x_i)\)).
\end{excopy}

The polynomials \(x_i\in E\setminus S\) for \(i=1,\ldots,n\)
and \(f(t)=\prod(t-x_i)\in S\) since for any permutation \(\sigma\in S_n\)
\(\prod_{i=1}^n(t-x_i) = \prod_{i=1}^n (t-x_{\sigma(i)})\).
Hence \(E/S\) is a splitting field of \(f(t)\).
By definition, \(S = E^{S_n}\) and by Theorem~79 (\cite{Rotman98})
\begin{equation*}
[E:S] = [E:E^{S_n}] = |S_n| = n! \,.
\end{equation*}

Define a map \(\varphi:S_n = \Aut(E)\) by
\begin{equation*}
(\varphi(\sigma))(f(\seqxn) = f(x_{\sigma(1)},\ldots,x_{\sigma(n)}).
\end{equation*}
As we saw already \(\varphi(\sigma)\in \Gal(E/S)\) for every \(\sigma\in S_n\).
Let \(g \in \Gal(E/S)\). Then $g$ must permutate
the roots of \(f(t)\) and these roots,
namely the polynomials \(x_i\in F(\seqxn)\),
generate $E$ over $S$.
Thus there must be \(\sigma \in S_N\) so \(\varphi(\sigma)=g\).
Clearly \(\varphi(\sigma\tau) = \varphi(\sigma)\varphi(\tau)\),
since for any \(g(\seqxn)\in E\)
\begin{eqnarray*}
 (\varphi\bigl(\sigma\tau)\bigr)\bigl(g(\seqxn)\bigr)
 & = & g\bigl(x_{\sigma\tau(1)},\ldots,x_{\sigma\tau(n)}\bigr) \\
 & = & \bigl(\varphi(\sigma)\bigr)g\bigl(x_{\tau(1)},\ldots,x_{\tau(n)}\bigr) \\
 & = & \bigl(\varphi(\sigma)\bigr)
       \bigl(\bigl(\varphi(\tau))(g(\seqxn)\bigr)\bigr) \\
 & = & \bigl(\varphi(\sigma)\varphi(\tau)\bigr)\bigl(g(\seqxn))\bigr)
\end{eqnarray*}
and  thus \(\Gal(E/S)\cong S_n\).

%%%%%
\item
\begin{excopy}
Let \(E/F\) be a Galois extension and let \(p(x)\in F[x]\) be irreducible.
Show that all the irreducible factors of \(p(x)\) in \(E[x]\) have the same
degree.
(Hint: Use Exercise~84.)
\end{excopy}

For normal extension, there is another equivalent criterion
using the notion of field closure
(See Chapter~\textsf{V} Theorem~3.3-\textbf{NOR1.}\ \cite{Lang94}).
We will prove the following much reduced result.

\begin{llem} \label{llem:normal:auto}
If \(E/F\) is a finite normal extension of fields,
and \(K/F\) any other extension,
then \(\sigma(E)=E\) for any \(\sigma\in \Gal(K/F)\).
\end{llem}
%%%%%%%%%%
\textbf{Proof.}
Let \(\sigma\in \Gal(K/F)\).
Take any \(a\in E\) and let \(f(x)=\Irr(a,F,x)\in F[x]\).
Since \(E/F\) is normal extension, \fx\ splits
and \(f(x)=\prod_{i=1}^n (x-a_i)\) where \(a_i\in E\).
By naturally inducing \(\sigma\) on \(F[x]\) it is clear that
\(\sigma(a)\) must also be a root of \fx\ and by unique factorization,
For some $i$, \(\sigma(a) = a_i \in E\) and thus \(\sigma(E)\subset E\).
The dimensions of both $E$ and \(\sigma(E)\) over $F$ must be equal
and so \(\sigma(E)=E\). \proofend

Back to the exercise.

Let $K$ be a splitting field of \(p(x)\) over $E$.

Let \(\sigma\in \Gal(K/F)\). Then \(\sigma^*\) (See exercise \ref{ex:sigstar}
and Theorem~34 \cite{Rotman98})
can be viewed by the above Lemma~\ref{llem:normal:auto}
as a automorphism of \(E[x]\) which is the identity restricted to \(F[x]\).
It is clear that \(\sigma^*\) maps irreducible polynomials
to irreducible polynomials. Since \(\sigma^*(f(x))=f(x)\)
and \fx\ has unique factorization in \(E[x]\)
\(\sigma^*\) permutates the irreducible factors of
\(p(x)=\prod_{i=1}^n g_i(x)\) in \(E[x]\).
We will show that \(\{\sigma*: \sigma\in \Gal(E/F)\}\)
\index{transitively!act}
acts transitively on \(\{g_i(x)\}_{i=1}^n\).

If \(p(x)\) has a solution in $E$, then by being normal extension,
\(p(x)\) splits in $E$ and the irreducible factors are all of degree~1.
Needless to say that by Lemma~50 (\cite{Rotman98})
for any two roots \(r_1\), \(r_2\) of \(p(x)\)
there is a map \(\sigma \in \Gal(E/F)\) such that
\(\sigma(r_1)=r_2\) and \(\sigma^*((x-r_1))=(x-r_2)\).

Let \(g_1(x)\), \(g_2(x)\) be any irreducible factors of \(p(x)\) in \(E[x]\).
If \(g_1(x) = g_2(x)\) then the identity map, maps
\(g_1(x)\) to \(g_2(x)\).
Otherwise, \(g_1(x) \neq g_2(x)\) and \(\gcd(g_1(x), g_2(x)) = 1\)
and so differenet factors of \(p(x)\) do not have common roots
(though each factor may have repeated roots).
Pick any two roots \(r_i\in K\) of \(g_i(x)\) for \(i=1,2\).
Again by Lemma~50 \cite{Rotman98} there is
\(\sigma \in \Gal(K/E) \subset \Gal(K/F)\)
such that \(\sigma(r_1)=r_2\)
and so \(\sigma^*(g_1(x))=g_2(x)\).

Thus \(\gdeg(g_1(x)) = \gdeg(g_1(x))\) as we had to show.

\textbf{Note:} But I failed to see how to use the hint.

%%%%%
\item
\begin{excopy}
Given a field $F$ and a finite group $G$ of order $n$,
show that there is a subfield
\(K\subset E = F(\seqxn)\) with \(\Gal(E/K)\cong G\).
(Hint: Use Exercise~89 and Cayley theorem (Theorem~G.24).)
\end{excopy}

By Cayley theorem, $G$ is isomorphic to a subgroup of \(S_n\).
We will identify $G$ with such a subgroup. We view each \(\sigma\in G\)
also as \(\sigma\in \Aut(E)\) by \(\sigma(f)(\seqxn)=f(\sigma(\seqxn))\).
Now we define a generalization of symmetric functions \(K=E^G\),
that is the functions invariant to permutations of variables by $G$.
% Using Exercise~89 subfield of symmetric functions \(S=E^{S_n}\) we
Let \(S=E^{S_n}\) be the subfield of symmetric functions.)
We have the field extensions \(S\subset K \subset E\).
By Exercise~89 \(S/E\) is a Galois extension and by Exercise~86 so is \(S/K\).
Hence \(K=E^{\Gal(E/K)}\). By Corollary~80 \cite{Rotman98} \(G=\Gal(E/K)\).

\iffalse
% Using Exercise~\ref{ex:sigstar} we also identify $G$ with the subgroup
Define the subgroup
\begin{equation*}
G^* = \{\sigma*\in \Aut(E): \sigma\in G\}
\end{equation*}
where \(\sigma^*(f(\seqxn)) = f(\sigma(x_1),\ldots,\sigma(x_n))\).
It is easy to see that \(\sigma \mapsto \sigma^*\) gives
the isomorphism \(G\cong G^*\). We will identify $G$ with \(G^*\) as well.


Define (generalization of symmetric functions):
\begin{equation*}
K = \{f(x)\in E: \forall \sigma\in G,\, \sigma^*(f(x))=f(x)\}.
\end{equation*}

By definition, for any \(\sigma\in G\), the induced map \(\sigma^*\)
is the identity on $K$ and so \(\sigma\in \Gal(E/K\) and \(G^* \subset
\Gal(E/K)\).  Conversely, let \(\sigma\in \Gal(E/K)\).

\fi

   %%%%%%%%%%%%%
\end{myenumerate}
%%%%%%%%%%%%%%%%%


%%%%%%%%%%%%%%%%%%%%%%%%%%%%%%%%%%%%%%%%%%%%%%%%%%%%%%%%%%%%%%%%%%%%%%%%
%%%%%%%%%%%%%%%%%%%%%%%%%%%%%%%%%%%%%%%%%%%%%%%%%%%%%%%%%%%%%%%%%%%%%%%%
%%%%%%%%%%%%%%%%%%%%%%%%%%%%%%%%%%%%%%%%%%%%%%%%%%%%%%%%%%%%%%%%%%%%%%%%
\chapter[The Fundemental Theorem]{%
        The Fundemental Theorem \\ of Galois Theory}
\typeout{The Fundemental Theorem of Galois Theory}

%%%%%%%%%%%%%%%%%%%%%%%%%%%%%%%%%%%%%%%%%%%%%%%%%%%%%%%%%%%%%%%%%%%%%%%%
%%%%%%%%%%%%%%%%%%%%%%%%%%%%%%%%%%%%%%%%%%%%%%%%%%%%%%%%%%%%%%%%%%%%%%%%
\section{Notes}

%%%%%%%%%%%%%%%%%%%%%%%%%%%%%%%%%%%%%%%%%%%%%%%%%%%%%%%%%%%%%%%%%%%%%%%%
\subsection{Wrong Lemma Number Reference}

In page~84, the proof of Theorem~84~(iii) refers to:
\begin{quotation}
 \mldots\ from Lemma 85 \mldots
\end{quotation}
while it \emph{should} be:
\begin{quotation}
 \mldots\ from Lemma 83 \mldots
\end{quotation}

%%%%%%%%%%%%%%%%%%%%%%%%%%%%%%%%%%%%%%%%%%%%%%%%%%%%%%%%%%%%%%%%%%%%%%%%
%%%%%%%%%%%%%%%%%%%%%%%%%%%%%%%%%%%%%%%%%%%%%%%%%%%%%%%%%%%%%%%%%%%%%%%%
%%%%%%%%%%%%%%%%%%%%%%%%%%%%%%%%%%%%%%%%%%%%%%%%%%%%%%%%%%%%%%%%%%%%%%%%
\chapterTypeout{Applications}

%%%%%%%%%%%%%%%%%%%%%%%%%%%%%%%%%%%%%%%%%%%%%%%%%%%%%%%%%%%%%%%%%%%%%%%%
%%%%%%%%%%%%%%%%%%%%%%%%%%%%%%%%%%%%%%%%%%%%%%%%%%%%%%%%%%%%%%%%%%%%%%%%
\section{Notes}

%%%%%%%%%%%%%%%%%%%%%%%%%%%%%%%%%%%%%%%%%%%%%%%%%%%%%%%%%%%%%%%%%%%%%%%%
\subsection{Accuracy in Induction Step}

\index{Steinitz}
The proof of Theorem~86 (Steinitz) has:
\begin{quotation}
\mldots, it suffices to prove that \(E = F(\alpha, \beta)\)
is a simple extension.
\end{quotation}
It would be more accurate to have:
\begin{quotation}
\mldots, it suffices to prove that \(F(\alpha, \beta)\)
is a simple extension of $F$.
\end{quotation}
Since by induction one can define \(F_0 = F\) and \(F_i = F_{n-1}(\alpha_i)\)
for \(i=1,\ldots,n\). So \(E=F_n\) and each step is simple,
but requiring \(E = F_i\) can and ought to be avoided.

%%%%%%%%%%%%%%%%%%%%%%%%%%%%%%%%%%%%%%%%%%%%%%%%%%%%%%%%%%%%%%%%%%%%%%%%
\subsection{Typographic Error (\ensuremath{GF})}  \label{typo:GF2}

As in \ref{typo:GF1}, there is a typographic error
in Corollary~89 (page~86) and its proof where \, \(GF(p^n)\), is used instead
of \, \(\GF(p^n)\).


%%%%%%%%%%%%%%%%%%%%%%%%%%%%%%%%%%%%%%%%%%%%%%%%%%%%%%%%%%%%%%%%%%%%%%%%
\subsection{Typographic Error --- Extraneous Parenthesis}

On page 89, in the proof of Theorem~92 (Fundamental Theorem of Algebra),
the 3rd line has:
\begin{quotation}
\(f(x)\overline{f})(x) \in \R[x]\)
\end{quotation}
It should have:
\begin{quotation}
\(f(x)\overline{f}(x) \in \R[x]\)
\end{quotation}


%%%%%%%%%%%%%%%%%%%%%%%%%%%%%%%%%%%%%%%%%%%%%%%%%%%%%%%%%%%%%%%%%%%%%%%%
%%%%%%%%%%%%%%%%%%%%%%%%%%%%%%%%%%%%%%%%%%%%%%%%%%%%%%%%%%%%%%%%%%%%%%%%
\section{Exercises (page 90)}

%%%%%%%%%%%%%%%%
\begin{myenumerate}

%%%%% 92
\item
\begin{excopy}
Let \(E/F\) be a Galois extension with \([E:F]>1\).
\begin{itemize}
 \item[(i)]
   Must there be an intermeiate field of prime degree over $F$?
   \index{alternating!group}
   (Hint: The alternating group \(A_6\) has no subgroup of prime index
    [See Theorem~G.37].)
 \item[(ii)]
   Same quotation as in (i) with the added hypothesis that \(\Gal(E/F)\)
   \index{solvable!group}
   is a solvable group.
\end{itemize}
\end{excopy}

\begin{itemize}
 \item[(i)]

   No. Let (using Exercise~89) \(K = \Z_2(x_1,x_2,x_3,x_4,x_5,x_6)\)
   be the field of rational polynomials with $6$ variables and
   coefficients in \Zn{2} (or any other field). Let \(F \subset K\) be
   a subfield of all symmetric functions.  We have seen in Exercise~89
   that \(K/F\) is a Galois extension and that \(\Gal(K/F) \cong
   S_6\).  Now \(S_6:A_6] = 2\) and by Lemma~G.28 \cite{Rotman98}
   \(A_6 \subnormal S_6\).

   Let \(E=K^{A_6}\) and
   by Theorem~84 \cite{Rotman98} \(E/F\) is a Galois extension.
   with \(\Gal(E/F) \cong A_6\). Since by Theorem~G.37 \cite{Rotman98}
   \(A_6\) has no subgroup with prime (\(2,3,5\)) index
   and again by Theorem~84 \cite{Rotman98} there is no intermediate
   field for this \(E/F\) extension with prime degree over $F$.

 \item[(ii)]
   Yes. If \(\Gal(E/F)\) is solvable, then by Theorem~G.16 \cite{Rotman98}
   it must have a subgroup $H$ of prime index. By Theorem~84 \cite{Rotman98}
   \([E^H:F] = [G:H]\).
\end{itemize}

%%%%%
\item
\begin{excopy}
Show that \(\Z_p(x,y)\) is a finite extension
of its subfield \(\Z_p(x^p,y^p)\),
but it is not a simple extension.
\end{excopy}

% Using hint I found on the Internet.
Let \(F=\Z_p(x^p,y^p)\) and \(E=\Z_p(x,y)\).
% \(f(t) = (t^p - x^p)(t^p - y^p)\) is an irreducible polynomial in \(F[t]\).
Now \(f(t) = (t^p - x^p)\) is an irreducible polynomial in \(F[t]\),
and so the extension \(K=F(x)\) has a degree \([K:F]=p\).
Also, \(g(t) = (t^p - y^p)\) is an irreducible polynomial in \(K[t]\),
and so the extension \(E=K(y)=F(x,y)\) has a degree
\begin{equation} \label{eq:EF:p2}
[E:F] = [E:K][K:F] = p^2 < \infty.
\end{equation}

Now assume by negation that \(E/F\) is simple.
Since $F$ is inifinite then by
Theorem (primitive element)~86 \cite{Rotman98}
there must be \(u_1,u_2\in F\)
such that \(u_1 \neq u_2\) and \(D = F(u_1 x + y) = F(u_2 x + y)\).
But then \(u_1 x + y, u_2 x + y \in D\) and
so \((u_1 - u_2)x \in D\) from which \(x\in D\) and \(y\in D\) follows.
Therefore (still under negation hypothesis),
there exists \(u\in F\) such that \(E = F(ux + y)\).

For any \(n \geq 0\) let \(n=mp+r\) with \(r<p\).
\begin{eqnarray*}
 (ux+y)^n
   & = & (ux+y)^{mp+r} = (ux+y)^{mp}(ux+y)^r \\
   & = & (u^p x^p + y^p)^m (ux+y)^r.
\end{eqnarray*}
Note that \((u^p x^p + y^p)^m \in F\)
and we just showed that
\(\{(ux + y)^k:\; 0 \neq k < p\}\) spans $E$ over $F$.
Thus \([E:F]\leq p\) which contradicts (\ref{eq:EF:p2}).


%%%%%
\item
\begin{excopy} \label{ex:Zp:xpxt}
Let \(K=\Z_p(t)\) be the field of rational functions,
let \(f(x) = x^p - x - t \in K[x]\),
and let \(E/K\) be a splitting field of \fx.
Prove that \(\Gal(E/K) \cong \Z_p\) but that \fx\ is not solvable by radicals.
\end{excopy}

First we will prove that \fx\ has no root in $K$.
By negation let \(u = g(t)/h(t) \in K\) be a root of \fx,
where \(g(t),h(t)\in \Z_p[t]\) and \(\gcd(g(t),h(t)) = 1\).
We can assume that the leading coefficient of \(h(t)\) is $1$.

We will first show that \(h(t)=1\).
Let \(S/K\) be an extension where \(g(t)h(t)\) split with
\(g(t) = \mu \prod_{i=1}^m (t - \alpha_i)\)
and
\(h(t) = \prod_{i=1}^n (t - \beta_i)\).
Clearly \(\alpha_i \neq \beta_j\) for all \(i=1,\ldots,m\), \(j=1,\ldots,n\)
since otherwise \(\gcd(g(t),h(t)) \neq 1\).
Since \(f(u) = 0\) an equality in $K$, we have
\begin{equation} \label{eq:xpxt:ghp}
 \left(\mu \prod_{i=1}^m (t - \alpha_i) / \prod_{i=1}^n (t - \beta_i)\right)^p
 - \mu \prod_{i=1}^m (t - \alpha_i) / \prod_{i=1}^n (t - \beta_i) - t = 0.
\end{equation}
Multiplying by \(h^p(t)\) we get
\begin{equation*}
 \mu \prod_{i=1}^m (t - \alpha_i)^p
 - \mu \prod_{i=1}^m (t - \alpha_i) \cdot \prod_{i=1}^n (t - \beta_i)^{p-1}
 - t \prod_{i=1}^n (t - \beta_i)^p = 0.
\end{equation*}
If \(n \neq 0\) then \((t - \beta_1) \mid \prod_{i=1}^m (t - \alpha_i)\)
a contradiction, and so \(h(t) = 1\).

Now (\ref{eq:xpxt:ghp}) can be simplified to
\begin{equation*}
 \mu \prod_{i=1}^m (t - \alpha_i)^p - \mu \prod_{i=1}^m (t - \alpha_i) - t = 0.
\end{equation*}
Now we see that for all \(i = 1,\ldots,m\),
we have \((t - \alpha_i) \mid t \) and so \(\alpha_i = 0\).
Now \(f(u) = \mu t^p - \mu t - t\).
But since \(f(u) = 0\), the leading coefficient \(\mu\) must be zero
and \(t=0\) in $K$ gives the desired contradiction,
showing that \fx\ has no roots in $K$.

Since \(f'(x) = -1\),
by Exercise~44 \cite{Rotman98}
the polynomial \fx\ is separable.
By Theorem~56 \cite{Rotman98} \(|\Gal(E/K)| = [E:K]\).

Using the future Exercise~97 as a hint,
If $u$ is a root of \fx, then
\begin{equation} \label{eq:fu:fu1}
f(u+1) = (u+1)^p - (u+1) - t = f(u)
\end{equation}
and clearly \(f(u) = (u+i)\)  where \(0 \leq i < p\).
and so \(K(u)\) is a splitting field for \fx. Thus \(K(u)=E\)
and \([E:K] = \gdeg(f(x)) = p\).
Since \(|\Gal(E/K)| = p\), we have \(\Gal(E/K) \cong \Z_p\).
By Theorem~81 \cite{Rotman98},
we also see by the roots of \fx\ that \(E/K\)
is a Galois (separable) extension.

Assume by negation that \fx\ is solvable by radicals, since \([E:K]=p\)
there cannot be an intermediate field between the $E$ and $K$,
and thus \(E/K\) is a pure extension of type $m$,
say \(\alpha\in E\setminus K\)
and \(\alpha^p\in K\) and so
\(g(x)=x^p-\alpha^p\) is irreducible in \(K[x]\) with a root \(\alpha\in E\).
By being Galois extension \(g(x)\) is separable, contradiction to
\(g(x)=(x-\alpha)^p \in E[x]\).


   %%%%%%%%%%%%%
\end{myenumerate}
%%%%%%%%%%%%%%%%%


%%%%%%%%%%%%%%%%%%%%%%%%%%%%%%%%%%%%%%%%%%%%%%%%%%%%%%%%%%%%%%%%%%%%%%%%
%%%%%%%%%%%%%%%%%%%%%%%%%%%%%%%%%%%%%%%%%%%%%%%%%%%%%%%%%%%%%%%%%%%%%%%%
%%%%%%%%%%%%%%%%%%%%%%%%%%%%%%%%%%%%%%%%%%%%%%%%%%%%%%%%%%%%%%%%%%%%%%%%
\chapterTypeout{Galois's Great Theorem}

%%%%%%%%%%%%%%%%%%%%%%%%%%%%%%%%%%%%%%%%%%%%%%%%%%%%%%%%%%%%%%%%%%%%%%%%
%%%%%%%%%%%%%%%%%%%%%%%%%%%%%%%%%%%%%%%%%%%%%%%%%%%%%%%%%%%%%%%%%%%%%%%%
\section{Notes}

%%%%%%%%%%%%%%%%%%%%%%%%%%%%%%%%%%%%%%%%%%%%%%%%%%%%%%%%%%%%%%%%%%%%%%%%
\subsection{Missing Explicit Galois Requirement} \label{ss:missgal}

In exercise~\ref{ex:96} (page~\pageref{ex:96}) the hint in item (i)
says:
\begin{quotation}
Use \(E/F\) being a Galois extension \mldots
\end{quotation}
Now, the only assumptions given for this \(E/F\) extension in the exercise
are
\begin{itemize}
 \item \(E/F\) is separable.
 \item \(\Gal(E/F)\) is cyclic.
\end{itemize}
But these do \emph{not} imply that \(E/F\) is Galois.
For example
let \(\alpha = \sqrt[4]{2} > 0\) and
\(\Q(\alpha)/\Q\) is a finite algebraic extension.
The Galois group \(G = \Gal(\Q(\alpha)/\Q \cong S_2\)
since any mapping \(\sigma\in G\)
is determined by \(\sigma(\alpha) = \pm\alpha\).
But \(\alpha^2 = \sqrt{2} \in {\Q(\alpha)}^G \setminus \Q\)
and so \(\Q(\alpha)/\Q\) is \emph{not} Galois (and \emph{not} normal).

%%%%%%%%%%%%%%%%%%%%%%%%%%%%%%%%%%%%%%%%%%%%%%%%%%%%%%%%%%%%%%%%%%%%%%%%
\subsection{ker(T) and not ker(tau)} \label{ss:tau2t}

There is a mistake (or typo) in the hint of exercise~\ref{ex:96}~(i).
It says:
\begin{quotation}
\mldots; show that \(\dim(\ker \tau) = n - 1\) as well.
\end{quotation}
While it should say:
\begin{quotation}
\mldots; show that \(\dim(\ker T) = n - 1\) as well.
\end{quotation}

%%%%%%%%%%%%%%%%%%%%%%%%%%%%%%%%%%%%%%%%%%%%%%%%%%%%%%%%%%%%%%%%%%%%%%%%
\subsection{tau(alpha) NOT Necessarily Zero} \label{ss:taualphnz}

Exercise~\ref{ex:notrace} (ii) asks to show that  \(\tau(\alpha) = 0\).
It is \emph{not} necessarily true.
See analysis in the solution to the exercise.

%%%%%%%%%%%%%%%%%%%%%%%%%%%%%%%%%%%%%%%%%%%%%%%%%%%%%%%%%%%%%%%%%%%%%%%%
%%%%%%%%%%%%%%%%%%%%%%%%%%%%%%%%%%%%%%%%%%%%%%%%%%%%%%%%%%%%%%%%%%%%%%%%
\section{Exercises (pages 94--95)}

%%%%%%%%%%%%%%%%
\begin{myenumerate}

%%%%% 95
\item
\begin{excopy}
Let \label{ex:trace:def}
\(E/F\) be a finite separable extension with Galois group $G$.
Define the
\index{trace}
\textbf{trace} \(T: E \rightarrow E\) by
\(T(\alpha) =  \sum_{\sigma \in G} \sigma(\alpha)\).
\begin{itemize}
 \item[(i)]
   Prove that \(\im T \subset F\) and that
   \begin{equation*}
      T(\alpha + \beta) = T(\alpha) + T(\beta)
   \end{equation*}
   for all \(\alpha, \beta \in E\).
 \item[(ii)]
   Show that $T$ is not identically zero.  (Hint: Independence of characters.)
\end{itemize}
\end{excopy}

See notes of \ref{ss:missgal} and \ref{ss:tau2t}.

\begin{itemize}
 \item [(i)]
    Let \(\alpha\in E\). If \(\alpha\in F\) then \(\sigma(\alpha)\in F\)
    for any \(\sigma\in G\) and so \(T(\alpha)\in F\).

    The other case is \(\alpha \notin F\). Let
    \(f(x)=\Irr(\alpha,F,x)\in F[x]\).  Since \(E/F\) is separable, so
    is \fx.  By Theorem~56 (\cite{Rotman98})

    \begin{equation} \label{eq:FvF}
      |\Gal(F(\alpha)/F)| = [F(\alpha):F] = \gdeg(f).
    \end{equation}

    Using Lemma~50 (\cite{Rotman98}) we see that for
    any root $w$ of \fx\ (in \(F(\alpha)\)) there is
    \(\sigma\in  \Gal(F(\alpha)/F\)
    such that \(\sigma(\alpha)=\beta\).  By the
    (\ref{eq:FvF}) equality there is exactly one such mapping.

    It is easy to see that \(E/F(\alpha)\) is also separable. Since it
    is finite, by induction one can find a separable polynomial
    \(g(x)\in F(\alpha)[x]\) such that $E$ is its splitting field.

    Utilizing Theorem~51 (\cite{Rotman98}) we see that
    for any root \(\beta\) of \fx, the mapping \(\sigma\)
    has \([E:F(\alpha)]\) exactly extensions \(\tilde\sigma\) to $E$

    By Theorem~58 (\cite{Rotman98})
    \begin{equation*}
     |G| = |\Gal(E/F)| = |\Gal(E/F(\alpha))| \cdot |\Gal(F(\alpha)/F),
    \end{equation*}
    and hence these extensions are all of $G$.

    Now
    \begin{equation*}
      T(\alpha) = \sum_{\sigma\in G} \sigma(\alpha)
        = [E:F(\alpha)] \sum_{i=1}^{\gdeg(f)} \alpha_i
        = [E:F(\alpha)] \cdot a_{\gdeg(f)-1}
    \end{equation*}
    where \(\{\alpha_i\}_{i=1}^{\gdeg(f)}\) are the roots of \fx
    and \(a_{\gdeg(f)-1}\) is the coefficient of \(x^{\gdeg(f)-1}\) in \fx.
    Therefore, \(T(\alpha) \in F\).

 \item [(ii)]
    All \(\{\sigma_{|E}\}_{\sigma\in G}\) are distinct characters.
    By Corollary~77 (\cite{Rotman98}) they are independent, and in particular
    their sum cannot be zero.

\end{itemize}

%%%%% 96
\item
\begin{excopy}
Assume  \label{ex:96}
that \(E/F\) is a separable extension of degree $n$ and cyclic
Galois group \(G = \Gal(E/F) = \lrangle{\alpha}\).
\begin{itemize}
 \item[(i)]
   If \(\sigma \in G\), define \(\tau = \sigma - \textrm{identity}\),
   and prove that \(\ker T = \im \tau\).
   (Hint: Use \(E/F\) being a Galois extension to show that
    \(\ker \tau = F\) and hence \(\dim(\im\tau) = n - 1\);
    show that \(\dim(\ker\tau) = n - 1\) as well.)
 \item[(ii)]
   Prove the
   \index{Trace Theorem}
   \textbf{Trace Theorem}: If \(E/F\) is a Galois extension
   with cyclic Galois group \(\Gal(E/F) = \lrangle{\sigma}\), then
   \begin{equation*}
    \ker T = \{\alpha \in E: \alpha = \sigma(\beta) - \beta\;
                             \textrm{for some}\; \beta \in E\}.
   \end{equation*}
\end{itemize}
\end{excopy}

\begin{itemize}
 \item[(i)]
   First we show that
   \begin{equation} \label{eq:imtau:sub:kerT}
   \im\tau \subseteq \ker T.
   \end{equation}
   Let \(v \in \im\tau\) and so there is \(w\in E\) such that
   \(v = \sigma(w) - w\).  Now
   \begin{eqnarray*}
     T(v)
     & = & \sum_{\sigma \in G} \sigma(v) \\
     & = & \sum_{\sigma \in G} \sigma(\sigma(w) - w) \\
     & = & \sum_{\sigma \in G} \sigma(\sigma(w)) -
           \sum_{\sigma \in G} \sigma(w) \\
     & = & 0
   \end{eqnarray*}
   and thus \(v\in \ker T\).

   In view of the comments made in \ref{ss:missgal}
   we assume also that \(E/F\) is a Galois extension.

   By definition any mapping \(\sigma \in G\) fixes $F$
   and so \(F \subseteq \ker \tau\). Now if \(v \in \ker \tau\)
   then \(\sigma(v) = v\) and since \(\sigma\) generates $G$
   we have \(v\in E^G\). By being Galois extension, \(\ker \tau = E^G = F\).

   As a vector spaces over $F$, $E$ has a dimension $n$.
   Since \(\ker \tau =  F\), we have \(\dim(\ker \tau) = 1\), hence
   \(\dim(\im \tau) = n - 1\).

   By previous exercise~\ref{ex:trace:def}, we can see that
    \(T: E \rightarrow E\) is linear mapping \emph{over} $F$
   and \(\im T \subseteq F\). Since \(T \neq 0\), we must have
   \(\im T = F\) and so \(\dim(\im T) = 1\) and so
   \(\dim(\ker T) = n - 1\).

   Since  \(\dim(\im \tau) = \dim(\ker T)\)
   we have equality in (\ref{eq:imtau:sub:kerT}).

 \item[(ii)]
   This is smiply formalization of (i).
\end{itemize}

%%%%% 97
\item
\begin{excopy}
Let
\label{ex:xpxc}
$F$ be a field of characteristic \(p>0\).
\begin{itemize}
 \item[(i)]
   Let \(f(x) = x^p - x - c \in F[x]\) and let $u$ be a root of \fx\ in some
   splitting field \(E/F\). Show that every root of \fx\ has the form \(u+i\),
   where \(0 \leq i < p\).
 \item[(ii)]
   Show that \(x^p -x - c \in F[x]\) either splits or is irreducible.
\end{itemize}
\end{excopy}

There is some similarity with Exercise~\ref{ex:Zp:xpxt}.

\begin{itemize}
 \item[(i)]
   Assume \(0\leq i < p\). Since \(i^p = i\) (see equation (\ref{eq:fu:fu1}))
   we have
   \begin{equation*}
   f(u+i) = (u+i)^p - (u+i) - c = u^p - u + (i^p - i) - c = f(u) = 0.
   \end{equation*}
   These are all the roots since \(\gdeg(f) = n\).
 \item[(ii)]
   Any field of characteristic $p$, contains any $i$ such that \(0\leq i < p\).
   Hence from (i) for any such $i$ and eny extension \(E/F\),
   we have \(u \in E\) iff \(u+1 \in E\).
\end{itemize}

%%%%% 98
\item
\begin{excopy}
Let $F$ be a field of characteristic \(p>0\), and let \(E/F\) be a
Galois extension with cyclic Galois bgroup \lrangle{\sigma} of order $p$.

\begin{itemize}
 \item[(i)]
   Prove there is \(\alpha \in E\) with \(\sigma(\alpha) - \alpha = 1\).
   \index{Trace Theorem}
   (Hint. Use the trace theorem).
 \item[(ii)]
   Prove that \(E=F(\alpha)\), where \(\alpha\)
   is a root of an irreducible polynomial in \(F[x]\)
   of the form \(x^p - x - c\).
\end{itemize}
\end{excopy}

\begin{itemize}
 \item[(i)]
   Here we have \(|G|=p\) and so
   \begin{equation*}
   T(1) = \sum_{\sigma \in G} \sigma(1)  = \sum_{\sigma \in G} 1 = p = 0.
   \end{equation*}
   Thus \(1\in \ker T\) and from the trace theorem, there is
   \(\alpha \in E\) with \(\sigma(\alpha) - \alpha = 1\).

 \item[(ii)]
   Continuing (i). We see that \(\sigma(\alpha) = \alpha + 1\)
   and for \(\sigma^i(\alpha) = \alpha + i\) where \(0\leq i < p\).
   Now \(\alpha\notin F\), otherwise \(\sigma(\alpha) = \alpha\).
   Hence, \(F\subsetneq F(\alpha) \subseteq E\). Because \([E:F]=p\) is prime,
   clearly \(E=F(\alpha)\).

   \index{Artin}
   \index{Schreier}
   Now (see Theorem~6.4 [Artin-Schreier] in chapter \textsf{VI} \cite{Lang94})
   \begin{equation*}
    \sigma(\alpha^p - \alpha) = \sigma(\alpha)^p - \sigma(\alpha) =
     (\alpha+1)^p - (\alpha+1) = \alpha^p - \alpha
   \end{equation*}
   Since $G$ is generated by \(\sigma\),
   we have \(c = \alpha^p - \alpha \in E^G = F\).
   Now \(\alpha\) is a root of \(f(x) = x^p - x - c \in F[x]\).
   Clearly, \(\Irr(\alpha,F,x) \mid f(x)\) and the splitting field
        % \(F(\alpha)\)
   of \(\Irr(\alpha,F,x)\) must be contained in $E$.
   Theorem~56 \cite{Rotman98} shows that \([E:F]=p\) and so
   the splitting field is actually $E$.
   We have  \(\Irr(\alpha,F,x) \mid f(x)\) but now also
   \(\gdeg(\Irr(\alpha,F,x)) = \gdeg(f(x)) = p\) and
   so \(f(x) = \Irr(\alpha,F,x)\)  and is irreducible.
\end{itemize}

%%%%% 99
\item
\begin{excopy}
Here
\label{ex:notrace}
is a proof of Exercise~98, similar to that of Corollary~97, which
does not use the trace theorem.  Let \(E/F\) be a Galois extension,
where $F$ is a field of characteristic \(p>0\), and let
\(\sigma \in \Gal(E/F)\) have order $p$. View \(\sigma\) as a linear
transformation, and define \(\tau = \sigma - \textrm{identity}\).
\begin{itemize}
 \item[(i)] Prove that \(\tau^p = 0\).

 \item[(ii)]
   Prove that if \(\alpha \in \ker \tau + \im \tau\), then
   \(\tau^{p-1}(\alpha) = 0\).  Using the fact that $p$ and \(p-1\)
   are relatively prime, prove that \(\tau(\alpha) = 0\).

 \item[(iii)]
   Prove that \(\ker \tau = F\) and that \(\im \tau \cap \ker \tau
   \neq \{0\}\).  (Hint. Show that \(E = \im\tau + \ker\tau\) if
   \(\im\tau \cap \ker\tau = \{0\}\).)

 \item[(iv)]
   Prove that \(1\in \im\tau\).  (Hint. Prove that
   \(\im\tau \cap \ker\tau = F\), and so \(F\subset \im\tau\).)
\end{itemize}
\end{excopy}

\begin{itemize}

 \item[(i)]
    Since the identity commutes we have
    \begin{equation} \label{eq:taupzero}
     \tau^p = (\sigma - \id)^p =
     \sum_{i=0}^p (-1)^i\binom{p}{i}\sigma^i\id^{p-i} = \sigma^p - \id = 0.
    \end{equation}

 \item[(ii)]
    Let \(\alpha = \alpha_0 + \alpha_1\) such that
     \(\alpha_0\in \ker \tau\) and
     \(\alpha_1\in \im \tau\).
     Let \(\beta\in E\) be such that \(\tau(\beta)=\alpha_1\).
     Now
     \begin{equation*}
      \tau^{p-1}(\alpha) = \tau^{p-1}(\alpha_0) + \tau^{p-1}(\alpha_1) =
         \tau^{p-2}(0) + \tau^p(\beta) = 0.
     \end{equation*}

     \textbf{Note:} The following was remarked in \ref{ss:missgal}.
     It is \emph{not} necessarily that \(\tau(\alpha) = 0\).

     For example let \(F=\Z_3\) and let \(f(t)=t^3-t-1\in\Z_3[t]\)
     be an irreducible polynomial. Its splitting field is
     \(E = \Z_3[t]/[f(t)]\).
     Clearly \(|E|=3^3=27\) and (the cosets of) the polynomials:
     \(r_i(x)=x+i\) where \(i=0,1,2\) are roots of \(f(t)\).
     The group \(G = \Gal(E/F) \cong Z_3\) is generated by
     \(\sigma\in G\) that permutates the roots with the mapping
     \(x+i \mapsto x + i + 1\).
     Now look how the linear transformations \(\sigma\) and
     powers of \(\tau = \sigma - \id\)
     map a basis of $E$ over $F$.

     \begin{alignat}{3}
     \sigma(1) &= 1 & \qquad \sigma(x) &= x+1 & \qquad \sigma(x^2) & = x^2+2x+1
                                                                       \notag\\
     \tau(1)   &= 0 & \qquad \tau(x)   &= 1   & \qquad \tau(x^2) & = 2x+1
                                                            \label{eq:tausq} \\
     \tau^2(1) &= 0 & \qquad \tau^2(x) &= 0   & \qquad \tau^2(x^2) & = 2
                                                                      \notag \\
     \tau^3(1) &= 0 & \qquad \tau^3(x) &= 0   & \qquad \tau^3(x^2) & = 0 \notag
     \end{alignat}

     From (\ref{eq:tausq}) we see that \(\alpha = 2x+1 \in \im\tau\)
     and \(\tau(\alpha) = 2 \neq 0\).



 \item[(iii)]
    Similar to exercise~\ref{ex:96},
    since \(\sigma_{|F} = \id_{|F}\) we have
    \(F \subseteq \ker \tau\) trivially.
    If \(v\in \ker\tau\) then \(\sigma(v)=v\) and also
    \(\sigma^i(v)=v\) and so \(v\in E^G = F\) and so \(\ker\tau = F\).

    Assume  by negation \(\im\tau \cap \ker\tau = \{0\}\).

    Let \(\tilde{\tau} = \tau_{|\im\tau}\) be a
    restriction map. Clearly \(\im\tilde{\tau} \subseteq \im\tau\)
    and \(\ker\tilde{\tau} = \{0\}\). Considering dimension, we must have
    \(\im\tilde{\tau} = \im\tau\) and thus \(\tilde{\tau}\) is invertible
    contradiction to (\ref{eq:taupzero}).

 \item[(iv)]
    Since \(\im\tau \cap \ker\tau \neq \{0\}\) and \(\ker\tau = F\)
    there is \(\lambda\in F\setminus\{0\}\) such that \(\lambda\in \im\tau\).
    Hence, there is \(\mu\in E\) such that \(\tau(\mu) = \lambda\)
    and being linear, \(\tau(\lambda^{-1}\mu) = 1 \in \im\tau\).
\end{itemize}

   %%%%%%%%%%%%%
\end{myenumerate}
%%%%%%%%%%%%%%%%%

%%%%%%%%%%%%%%%%%%%%%%%%%%%%%%%%%%%%%%%%%%%%%%%%%%%%%%%%%%%%%%%%%%%%%%%%
%%%%%%%%%%%%%%%%%%%%%%%%%%%%%%%%%%%%%%%%%%%%%%%%%%%%%%%%%%%%%%%%%%%%%%%%
%%%%%%%%%%%%%%%%%%%%%%%%%%%%%%%%%%%%%%%%%%%%%%%%%%%%%%%%%%%%%%%%%%%%%%%%
\chapterTypeout{Discriminants}

%%%%%%%%%%%%%%%%%%%%%%%%%%%%%%%%%%%%%%%%%%%%%%%%%%%%%%%%%%%%%%%%%%%%%%%%
%%%%%%%%%%%%%%%%%%%%%%%%%%%%%%%%%%%%%%%%%%%%%%%%%%%%%%%%%%%%%%%%%%%%%%%%
\section{Notes}


%%%%%%%%%%%%%%%%%%%%%%%%%%%%%%%%%%%%%%%%%%%%%%%%%%%%%%%%%%%%%%%%%%%%%%%%
\subsection{Add Page Reference to Cubic Formula}

In the proof of Theorem~100,
the reference to the \emph{cubic formula} uses the same notations
defined in the analysis of the cubic formula
in the chapter: \textbf{Classic Formulas}.
Therefore instead of
\begin{quotation}
(ii) The cubic formula gives \mldots
\end{quotation}
it would be helpful to put
\begin{quotation}
(ii) The cubic formula (pages 45,46) gives \mldots
\end{quotation}


%%%%%%%%%%%%%%%%%%%%%%%%%%%%%%%%%%%%%%%%%%%%%%%%%%%%%%%%%%%%%%%%%%%%%%%%
\subsection{Avoiding Dividing by Zero}  \label{ss:avoid:zdiv2}

Similar problems as in \ref{ss:avoid:zdiv}. In the proof of Theorem~100
the case of \(q=0\) should be treated \emph{separately}. Then
\(D=-27r^2\) and from exercise~\ref{ex:omega} shows
one case see that put \(\rho = \sqrt[3]{r}\) we have
\begin{eqnarray*}
 \pm\Delta
  & = & \left((\rho - \omega\rho)(\rho - \omega^2\rho)(\omega\rho
                                                    - \omega^2\rho)\right)^2 \\
  & = & \rho^3(1 - \omega)(1 - \omega^2)(\omega - \omega^2) \\
  & = & \rho^3\omega(1 - \omega)^2(1 - \omega^2) \\
  & = & r\cdot 3i\sqrt{3}
\end{eqnarray*}
and so indeed \(D = 3^2\cdot(-1)\cdot 3 r^2 = -27r^2\).
and dealing with \(R=r^2\), \(y=0\) and and undefined \(z=-q/3y\) is avoided.

With this separate handling of \(q=0\) case, we can proceed assuming
\(z=-q/3y\neq 0\) thus avoiding the undefined step
of substituting $x$ with \(y/z\) in the identity
\begin{equation*}
 x^3 - 1 (x - 1)(x - \omega)(x - \omega^2).
\end{equation*}

Note that this also takes care of the case where
\begin{equation*}
y = \left[(-r+\sqrt{R})/2\right]^{1/3} = 0
\end{equation*}
Since then \(r=\sqrt{R}\) implying \(r^2 = R = r^2 + 4q^3/27\) and
so \(q=0\).


%%%%%%%%%%%%%%%%%%%%%%%%%%%%%%%%%%%%%%%%%%%%%%%%%%%%%%%%%%%%%%%%%%%%%%%%
\subsection{Irreducible --- Specify where}

\index{Casus Irreducibilis}
The way Theorem~102 (Casus Irreducibilis) is presented
is confusing.
\begin{quotation}
  \setcounter{quotethm}{101} % to get 102
  \begin{quotethm}[Casus Irreducibilis]
   Let \(f(x) = x^3+qx+r\in \R[x]\) be an irreducible cubic having real
   roots $u$, $v$ and $w$. \mldots
  \end{quotethm}
\end{quotation}

Now \fx\ is obviously reducible in \(\R[x]\) into \((x-u)(x-v)(x-w)\).
This will be better written as:

\begin{quotation}
  \setcounter{quotethm}{101} % to get 102
  \begin{quotethm}[Casus Irreducibilis]
  Let \(f(x) = x^3+qx+r\in \R[x]\) with real roots \(u,v,w\in \R\),
  let \(F=\Q(q,r)\) and assume \(f(x)\) is irreducible in \(F[x]\).
  Let \(E=F(u,v,w)\) \mldots
  \end{quotethm}
\end{quotation}



%%%%%%%%%%%%%%%%%%%%%%%%%%%%%%%%%%%%%%%%%%%%%%%%%%%%%%%%%%%%%%%%%%%%%%%%
\subsection[Missing Primitive requirement]{
            Missing Primitive requirement of \(\omega\)}

In the proof of Theorem~100 on page~97 it says:
\begin{quotation}
\mldots, \(\omega\) is a cube root of unity, \mldots
\end{quotation}
It should say:
\begin{quotation}
\mldots, \(\omega\) is a primitive cube root of unity, \mldots
\end{quotation}
or:
\begin{quotation}
\mldots, \(\omega\neq 1\) is a cube root of unity, \mldots
\end{quotation}


%%%%%%%%%%%%%%%%%%%%%%%%%%%%%%%%%%%%%%%%%%%%%%%%%%%%%%%%%%%%%%%%%%%%%%%%
\subsection{Missing ``Let'' of \ensuremath{\omega}}


Exercise~100 (page 100) should start with:
\begin{quotation}
Let \(\omega\in \C\) be a primitive cubic of unity (say \(e^{2\pi i/3}\)).
\end{quotation}


%%%%%%%%%%%%%%%%%%%%%%%%%%%%%%%%%%%%%%%%%%%%%%%%%%%%%%%%%%%%%%%%%%%%%%%%
%%%%%%%%%%%%%%%%%%%%%%%%%%%%%%%%%%%%%%%%%%%%%%%%%%%%%%%%%%%%%%%%%%%%%%%%
\section{Exercises (page 100)}

%%%%%%%%%%%%%%%%
\begin{myenumerate}

%%%%% 100
\item
\begin{excopy}
 Prove \label{ex:omega}
that \(\omega(1 - \omega^2)(1-\omega)^2 = 3i\sqrt{3}\).
\end{excopy}

This is actually the \(\Delta\) for \(f(x) = x^3 - 1\).
Let \(\omega\) be a primitive root of \fx. That is
\begin{equation}
\omega^3=1\neq\omega. \label{eq:omega:def}
\end{equation}
Now
\begin{eqnarray}
\Delta & = & (\omega^0 - \omega^1)(\omega^0 - \omega^2)(\omega^1 - \omega^2)
                                                                      \notag\\
 & = & (1 - \omega)(1 - \omega^2)(\omega - \omega^2) \notag\\
 & = & \omega(1 - \omega)^2(1 - \omega^2) \label{eq:ex:delta:omega} \\
 & = & \omega - 2\omega^2 + \omega^3 - \omega^3 + 2\omega^4 - \omega^5 \notag\\
 & = & 3\omega - 3\omega^2 = 3(\omega -  \omega^2). \label{eq:ex:3omegas}
\end{eqnarray}

Let is compute the last term in (\ref{eq:ex:3omegas})
Start with the geomtric sum
\begin{equation*}
1+\omega+\omega^2 = (\omega^3 - 1)/ (\omega - 1) = 0.
\end{equation*}
From which we get \(\omega^2+\omega = -1)\) and
\begin{equation*}
\left(\omega - \omega^2\right)^2 = \omega^2 - 2 + \omega = -3.
\end{equation*}
Taking square root, we get
\begin{equation}
\omega - \omega^2 = \pm i\sqrt{3}. \label{eq:pmi:sq3}
\end{equation}
To figure out the sign in \ref{eq:pmi:sq3}, let \(\omega = a+bi\)
where \(a,b\in\R\). The primitive root \(omega\) is chosen
as the one comlex number with minimal positive argument (degree),
and so \(b>0\).

Using (\ref{eq:omega:def}) we separate the real and imaginary part to get
\begin{equation*}
 \left\{\begin{array}{rl}
       a^3-3ab^2    & = 1 \\
       3a^2b - b3 & = 0
       \end{array}\right.
\end{equation*}
From the second equation we get \(3a^2=b^2\).
Substituting in the first equation, we get
\begin{equation*}
 a^3 - 3a\cdot3a^2 = -8a^3 = 1
\end{equation*}
and thus \(a = -1/2\) and \(b = \pm \sqrt{3}/2 = \sqrt{3}/2\)
dropping the multi-sign by the choice of primitive root.
Now we see that
\begin{equation*}
\omega^2 = (-1/2 + i\sqrt{3}/2)^2 = (-1/2 - i\sqrt{3}/2)^2 = \overline{\omega}.
\end{equation*}
Examining the imaginary part, we get
\begin{equation}
\Im(\omega - \omega^2) = \sqrt{3}/2 -(-\sqrt{3}/2) = \sqrt{3} > 0.
                                                     \label{eq:img:omega}
\end{equation}
and so the non ambiguous sign in (\ref{eq:pmi:sq3}) is positive.

Combining equations
(\ref{eq:ex:delta:omega}),
(\ref{eq:ex:3omegas}),
(\ref{eq:pmi:sq3}) and
(\ref{eq:img:omega}) we conclude that
\begin{equation*}
 \omega(1 - \omega)^2(1 - \omega^2) = 3 i\sqrt{3}.
\end{equation*}

%%%%%
\item
\begin{excopy}
\begin{itemize}
 \item[(i)]
   Prove that if \(a \neq 0\), then \(f(x)\) and \(af(x)\) have the
   same discriminant and the same Galois group. Conclude that it is no
   loss in generality to restrict attention to monic polynomials when
   computing Galois groups.
 \item[(ii)]
   Prove that a polynomial \fx\ and its associated reduced polynomial
   \(\tilde{f}(x)\) have the same Galois group.
\end{itemize}
\end{excopy}

\begin{itemize}
 \item[(i)] The discriminant is determiemd by the roots of the polynomial.
 and the roots of \(f(x)\) and \(af(x)\) for \(a\neq 0 \) are the same.
 \item[(ii)]
 We will prove a slightly more general result.

 \begin{quotation}
 Assume \(\tilde{f}(x) = f(x+s)\) where \(s\in E\).
 If $G$ and \(\tilde{G}\) are the Galois groups
 of \(f(x)\) and \(\tilde{f}(x)\)  respectively,
 then \(G = \tilde{G}\).
 \end{quotation}

 Let $E$ and \(\tilde{E}\) be the splitting fields of
 \(f(x)\) and \(\tilde{f}(x)\)  respectively. Clearly, \(\alpha\in E\)
 is a root of \(f(x)\) iff \(\alpha - s\) is a root of \(\tilde{f}(x)\).
 Now \(\alpha \in E\) iff \(\alpha - s \in E\) and so \(E = \tilde{E}\).

 % Let \seqn{b} be a basis of $E$ over $F$ with \(b_1=1_F\) (that spans $F$).
 Any \(\sigma\in G\) permutates the roots of \(f(x)\).
 We will show that such \(\sigma\)
 also permutates the root of \(\tilde{f}(x)\).
 Let \(\tilde{\beta}\) be a root of \(\tilde{f}(x)\),
 then \(\beta=\tilde{\beta}+s\)
 is a root  of \(f(x)\). By Lemma~54 \cite{Rotman98},
 \(\sigma(\beta)=\beta'\)
 is some (possibly other) root of \(f(x)\).
 % With a linear combination \(\beta'=\sum_{i=1}^n a_i b_i\) with \(a_i\in F\)
 % for \(1\leq i \leq n\).

 Now
 \begin{equation*}
 \sigma(\tilde{\beta})
   =  \sigma(\beta - s)  =  \sigma(\beta) - \sigma(s)  =  \beta' - s.
 \end{equation*}
 But we established that \(\beta' - s\) is a root of \(\tilde{f}(x)\).
 Thus \(G\in \tilde{G}\). Similarly, we can show the reverse inclusion
 and so \(G = \tilde{G}\).

\end{itemize}

%%%%% 102
\item\label{ex:det:cubic}%
\begin{excopy}
\begin{itemize}
 \item[(i)]
   If \(f(x) = x^3 + ax^2 + bx + c\), then its associated reduced polynomial
   is \(x^3+qx+r\), where
   \begin{equation*}
     q =  b - a^2/3 \qquad \textrm{and} \qquad r = 2a^3/27 - ab/3 + c.
   \end{equation*}
 \item[(ii)]
   Show that the discriminant of \fx\ is
   \begin{equation*}
     D = a^2b^2 - 4b^3 - 4a^3c - 27c^2 + 18abc.
   \end{equation*}
\end{itemize}
\end{excopy}

\begin{itemize}
 \item[(i)]
 Compute
 \begin{eqnarray*}
 \tilde{f}(x) & = & f(x - a/3) \\
  & = & (x-a/3)^3 + a(x - a/3)^2 + b(x - a/3) + c \\
  & = & x^3 - ax^2 + 3a^2x/9 - a^3/27 \; + ax^2 - 2a^2x/3 + a^3/9 \;
        + bx - ab/3 + c \\
  & = & x^3 + (a^2/3 - 2a^2/3 + b)x  + (-a^3/27 + a^3/9 - ab/3 + c) \\
  & = & x^3 + (b - a^2/3)x  + (2a^3/27 - ab/3 + c).
 \end{eqnarray*}
 \item[(ii)]
  With \(q = b - a^2/3\) and \(r = a^3/27 - ab/3 + c\), compute:
 \begin{eqnarray*}
  D & = & -4q^3 - 27r^2 \\
    & = & -4(b - a^2/3)^3 - 27(2a^3/27 - ab/3 + c)^2 \\
    & = & -4b^3 + 4a^2b^2 - 4a^4b/3 + 4a^6/27 \\
    &   & - 4a^6/27 - 3a^2b^2 - 27c^2 + 4a^4b/3 - 4a^3c + 18abc \\
    & = & a^2b^2 -4b^3 - 4a^3c  - 27c^2 + 18abc.
 \end{eqnarray*}
\end{itemize}

   %%%%%%%%%%%%%
\end{myenumerate}
%%%%%%%%%%%%%%%%%


%%%%%%%%%%%%%%%%%%%%%%%%%%%%%%%%%%%%%%%%%%%%%%%%%%%%%%%%%%%%%%%%%%%%%%%%
%%%%%%%%%%%%%%%%%%%%%%%%%%%%%%%%%%%%%%%%%%%%%%%%%%%%%%%%%%%%%%%%%%%%%%%%
%%%%%%%%%%%%%%%%%%%%%%%%%%%%%%%%%%%%%%%%%%%%%%%%%%%%%%%%%%%%%%%%%%%%%%%%
\chapter[Groups -- Quadratics, Cubics and Quartics]{%
         Galois Groups of\\ Quadratics, Cubics and Quartics}
\typeout{Galois Groups of Quadratics, Cubics and Quartics}


%%%%%%%%%%%%%%%%%%%%%%%%%%%%%%%%%%%%%%%%%%%%%%%%%%%%%%%%%%%%%%%%%%%%%%%%
%%%%%%%%%%%%%%%%%%%%%%%%%%%%%%%%%%%%%%%%%%%%%%%%%%%%%%%%%%%%%%%%%%%%%%%%
\section{Notes}

%%%%%%%%%%%%%%%%%%%%%%%%%%%%%%%%%%%%%%%%%%%%%%%%%%%%%%%%%%%%%%%%%%%%%%%%
\subsection{Implicit Assumption of Galois Extension in Lemma}

Lemma~103 (page 101) refers to \(G = \Gal(E/F)\) and
later the proof implicitly uses the fact that it is Galois
extension.

Since this chapter deals with extension of splitting fields
of irreducible polynomials over \Q, it is always the case.
But this assumption is not explicitly stated.
But even so, explicitly adding the assumption that \(E/F\)
is a Galois extension, would support the proof
and keep the generality of the lemma.


%%%%%%%%%%%%%%%%%%%%%%%%%%%%%%%%%%%%%%%%%%%%%%%%%%%%%%%%%%%%%%%%%%%%%%%%
\subsection{Missing Argument Checking \ensuremath{S_4}}

In Example~35, on page~102 After defining the four group
\begin{equation*}
 V = \{(1),(12)(34), (13)(24), (14)(23)\}
\end{equation*}
it says:
\begin{quotation}
\mldots if \(\sigma \in S_4\) fixes
 \((\alpha_i + \alpha_j)(\alpha_k+\alpha_\ell)\),
then \(\sigma \in V \cup \{(ij), (k\ell), (i k j \ell), (i\ell jk)\}\).
\end{quotation}

Now the reason \(\sigma\) cannot be, for example, one of
\(\{(i\ell),(jk),(ijl),(ij\ell k),\ldots\}\)
is that by negation we will have:
\begin{equation*}
 (\alpha_i + \alpha_j)(\alpha_k+\alpha_\ell) =
 (\alpha_i + \alpha_k)(\alpha_j+\alpha_\ell)
\end{equation*}
and this would give
\begin{equation*}
 \alpha_i\alpha_k + \alpha_j\alpha_\ell =
 \alpha_i\alpha_j + \alpha_k\alpha_\ell
\end{equation*}
or equivalently \((\alpha_i-\alpha_\ell)(\alpha_j - \alpha_k) = 0\).
A contradiction to the fact that \(\{\alpha_i: 1\leq i \leq 4\}\) are
roots of irreducible polynomial over \Q\ and must be mutually different.


%%%%%%%%%%%%%%%%%%%%%%%%%%%%%%%%%%%%%%%%%%%%%%%%%%%%%%%%%%%%%%%%%%%%%%%%
\subsection{Resolvent Preference} \label{ss:resolv:foot}

The footnote of page~103 explains the preference of the chosen
\index{resolvent cubic}
resolvent cubic \(g(x)\)
over an alternative \(h(x)\), both of \(f(x)\).
The reason given is that \(g(x)\) can be used to
compute the discriminant of \(f(x)\).
But \(h(x)\) can be used as well, as seen in
solution to exercise~\ref{ex:Kaplansky:reciprocal} 
(Lemma~\ref{lthm:quar:dick}~(ii)).


%%%%%%%%%%%%%%%%%%%%%%%%%%%%%%%%%%%%%%%%%%%%%%%%%%%%%%%%%%%%%%%%%%%%%%%%
\subsection{Negated Discriminant} \label{ss:neg:disc}

In Exersice~\ref{ex:disc:quat} the given expression
for the discriminant is wrong. It got mistakenly negated.
So instead of
\begin{equation*}
 D = -16a^4c + 4a^3b^2 + 128a^2c^2  - 144ab2^c + 27b^4 -256c^3.
\end{equation*}
It shuold be:
\begin{equation*}
 D = 16a^4c - 4a^3b^2 - 128a^2c^2 + 144ab^2c - 27b^4 + 256c^3.
\end{equation*}


%%%%%%%%%%%%%%%%%%%%%%%%%%%%%%%%%%%%%%%%%%%%%%%%%%%%%%%%%%%%%%%%%%%%%%%%
\subsection{Exercises: Declare $V$ Globally}

In exersice~106 the \emph{four group}
\(V = \{1,(12)(34),(13)(24),(14)(23)\}\)
is introduced. It is later used in Exercises: 107, 113, 114 without
refering to the definition. It may be nicer to declare it globally
for this batch of exercises.


%%%%%%%%%%%%%%%%%%%%%%%%%%%%%%%%%%%%%%%%%%%%%%%%%%%%%%%%%%%%%%%%%%%%%%%%
\subsection{Transitivity Uncredited}  \label{ss:trans:uncred}

Exercise~107 follows with the explanation:
\begin{quotation}
The added hypothesis \(|G/G\cap V|=w\) removes
the possibility \(G\cong V\) \mldots
\end{quotation}
While it should say:
\begin{quotation}
The added hypothesis that $G$ acts transitively removes
the possibility \(G\cong V\) \mldots
\end{quotation}
or (addition in bold):
\begin{quotation}
The added hypothesis \textbf{to} \(|G/G\cap V|=w\) removes
the possibility \(G\cong V\) \mldots
\end{quotation}

%%%%%%%%%%%%%%%%%%%%%%%%%%%%%%%%%%%%%%%%%%%%%%%%%%%%%%%%%%%%%%%%%%%%%%%%
\subsection{Missing formal \ensuremath{f(x)} Polynomial Naming}
            \label{ss:mis:polynam}


In Exercise~108,
\begin{quotation}
\mldots of \(x^4+x^2-6\).
\end{quotation}
would better be presented as (similar to Exercise~109)
\begin{quotation}
\mldots of \(f(x) = x^4+x^2-6\).
\end{quotation}


%%%%%%%%%%%%%%%%%%%%%%%%%%%%%%%%%%%%%%%%%%%%%%%%%%%%%%%%%%%%%%%%%%%%%%%%
\subsection{Missing Irreducible Condition} \label{ss:miss:irr}

Exercise~\ref{ex:Kaplansky:reciprocal}
should add the requirement that
\(x^4 + bx^3 + cx^2 + bx + 1\) must be irreducible.
See the solution.


%%%%%%%%%%%%%%%%%%%%%%%%%%%%%%%%%%%%%%%%%%%%%%%%%%%%%%%%%%%%%%%%%%%%%%%%
%%%%%%%%%%%%%%%%%%%%%%%%%%%%%%%%%%%%%%%%%%%%%%%%%%%%%%%%%%%%%%%%%%%%%%%%
\section{Exercises (pages 105--106)}

%%%%%%%%%%%%%%%%
\begin{myenumerate}

%%%%% 103
\item
\begin{excopy}
If
\label{ex:disc:uvw}
\fx\ is a quartic, then its discriminant is the discriminant of its
\index{resolvent cubic}
resolvent cubic. (Hint:
 \begin{eqnarray*}
  u - v & = & -(\alpha_1 - \alpha_4)(\alpha_2 - \alpha_3) \\
  u - w & = & -(\alpha_1 - \alpha_3)(\alpha_2 - \alpha_4) \\
  v - w & = & -(\alpha_1 - \alpha_2)(\alpha_3 - \alpha_4).\textrm{)}
 \end{eqnarray*}
\end{excopy}

We have defined
\begin{eqnarray}
 u & = & (\alpha_1 + \alpha_2)(\alpha_3 + \alpha_4) \notag \\
 v & = & (\alpha_1 + \alpha_3)(\alpha_2 + \alpha_4) \\
                                                    \label{eq:resolv:uvw}
 w & = & (\alpha_1 + \alpha_4)(\alpha_2 + \alpha_3) \notag
\end{eqnarray}
and so
\begin{eqnarray*}
u - v
 & = & (\alpha_1 + \alpha_2)(\alpha_3 + \alpha_4) -
       (\alpha_1 + \alpha_3)(\alpha_2 + \alpha_4) \\
 & = &   \alpha_1\alpha_3 + \alpha_2\alpha_4 \;
       - \alpha_1\alpha_2 - \alpha_3\alpha_4 \\
 & = & -(\alpha_1 - \alpha_4)(\alpha_2 - \alpha_3).
\end{eqnarray*}
Similarly, we can compute and get:
\begin{eqnarray*}
  u - w & = & -(\alpha_1 - \alpha_3)(\alpha_2 - \alpha_4) \\
  v - w & = & -(\alpha_1 - \alpha_2)(\alpha_3 - \alpha_4).
\end{eqnarray*}
Thus
\begin{eqnarray*}
D_{\alpha_1,\alpha_2,\alpha_3,\alpha_4}
 & = &
 \left((\alpha_1 - \alpha_2)(\alpha_1 - \alpha_2)
       (\alpha_1 - \alpha_4)(\alpha_2 - \alpha_3)
       (\alpha_2 - \alpha_4)(\alpha_3 - \alpha_4)\right)^2 \\
 & = &
 \left((\alpha_1 - \alpha_4)(\alpha_2 - \alpha_3)\right)^2
 \left((\alpha_1 - \alpha_3)(\alpha_2 - \alpha_4)\right)^2
 \left((\alpha_1 - \alpha_2)(\alpha_3 - \alpha_4)\right)^2 \\
 & = & \left((u-v)(u-w)(v-w)\right)^2 \\
 & = & D_{u,v,w}.
\end{eqnarray*}


%%%%%
\item
\begin{excopy}
If
\label{ex:disc:quat}
\(f(x) = x^4 + ax^2 + bx + c\), prove that the discriminant of \fx\ is
\begin{equation*}
 D = -16a^4c + 4a^3b^2 + 128a^2c^2  - 144ab2^c + 27b^4 -256c^3.
\end{equation*}
\end{excopy}

\textbf{Note:} See remark on \ref{ss:neg:disc} regarding
the need to negaate the expression above for $D$.

By Theorem~105 \cite{Rotman98},
\index{resolvent cubic}
the resolvent cubic of
\[f(x) = x^4 + ax^2 + bx + c\]
is \[g(x) = x^3 - 2ax^2 + (a^2-4c)x + b^2.\]
By the previous exercise, the discriminants of \(f(x)\) and \(g(x)\)
are equal.

Using Exercise~\ref{ex:det:cubic},
we saw that for
\(g(x) = x^3 + a'x^2 + b'x + c'\), the discriminant is
   \begin{equation*}
      D' = a'^2b'^2 - 4b'^3 - 4a'^3c - 27c'^2 + 18a'b'c'.
   \end{equation*}
Substituting
\(a' = -2a\),\quad
\(b' = a^2-4c\),\quad
\(c' = b^2\),
we compute
\begin{eqnarray*}
D
& = & (-2a)^2(a^2-4c)^2 - 4(a^2-4c)^3 - 4(-2a)^3(b^2)
      - 27(b^2)^2 + 18(-2a)(a^2-4c)(b^2) \\
& = & 4a^2(a^2-4c)^2 - 4(a^2-4c)^3 + 32a^3b^2 \;
      - 27b^4 - 36a(a^2-4c)b^2 \\
& = & 4a^6 - 32a^4c + 64a^2c^2 \\
&   &   -4a^6 + 48a^4c - 192a^2c^2 + 256c^3 \\
&   &   + 32a^3b^2 - 27b^4 \\
&   &   - 36a^3b^2 + 144ab^2c \\
& = & 16a^4c - 4a^3b^2 - 128a^2c^2 + 144ab^2c - 27b^4 + 256c^3
\end{eqnarray*}

Which is \emph{exactly the negation} of the expression
the exercise asked to get.

%%%%%
\item
\begin{excopy}Show that \(x^3 + ax + 2\in \R[x]\) has three real roots
if and only if \(a\leq -3\).
\end{excopy}

First let us have the following lemma and proof
similar to Theorem~104 \cite{Rotman98}.

% It is clear that \(D=0\) iff there are two equal roots.

\begin{llem}
Let \(f(x)\in \R[x]\) be a cubic with discriminant $D$.
Then \(f(x)\) has three real roots iff \(D\geq 0\).
\end{llem}
\textbf{Proof.}
If there are three real roots then $D$ is a square
of a product of real numbers. Obviously \(D\geq0\).

Conversely, let the roots be \(r_1,r_2,r_3\).
Clearly having real coefficients, \(\overline{f(z)} = f(\overline{z})\)
and so \(\overline{r_1},\overline{r_2},\overline{r_3}\)
are roots (actually a permutation).

We will split the case of \(D\geq 0\).

Assume \(D = 0\).
It is clear there are two equal roots, say \(r_1=r_2\)
If they are real then
\begin{equation} \label{eq:f3real}
 f(x)/\left((x-r_1)(x-r_2)\right) = (x-r_3) \in \R[x]
\end{equation}
and so \(r_3\in \R\).
If by negation \(r_1,r_2\notin \R\) then since
\(\overline{r_1}\) is a root, we must have
\(r_1 = r_2 = \overline{r_3}\) and so \(f(x)\) has no real root
which contradict
consequence (4)
on page~88 \cite{Rotman98} where it was shown
that a real polynomial of odd degree has a real root.

Now assume \(D > 0\). Since at least one of the roots must be real,
we can assume \(r_1\in \R\).
If \(r_2\in \R\) then as in (\ref{eq:f3real}), we have \(r_3\in \R\)
and we are done. If by negation \(r_2\notin \R\)
then we must have \(r_1 \neq \overline{r_2} = r_3\).
Now
\begin{eqnarray*}
\Delta & = & (r_1 - r_2)(r_1 - r_3)(r_2 - r_3)\\
 & = & (r_1 - r_2)(r_1 - \overline{r_2})(r_2 - \overline{r_2})\\
 & = & (r_1 - r_2)(\overline{r_1 - r_2})(r_2 - \overline{r_2})\\
 & = & |r_1 - r_2)|^2(2\Im(r_2),
\end{eqnarray*}
and so \(D = \Delta^2 = -4|\Im(r_2)|^2|r_1 - r_2)| < 0\) a contradiction.
\proofend

Back to the exercise. We compute the discriminant,
with \(a'=0\), \(b'=a\), \(c'=2\).
\begin{eqnarray*}
D & = & a'^2b'^2 - 4b'^3 - 4a'^3c' - 27c'^2 + 18a'b'c' \\
  & = & 4(-a)^3 - 27\cdot 4
\end{eqnarray*}
Thus \(D\geq 0\) iff \(-a \geq 3\) iff \(a \leq 3\).


%%%%%
\item
\begin{excopy}
Let $G$ be a subgroup of \(S_4\) with \(|G|\) a multiple of $4$; define
\begin{equation*}
 m = |G/G\cap V|,
\end{equation*}
where \(V = \{1,(12)(34),(13)(24),(14)(23)\}\) is the four group.
\begin{itemize}
 \item[(i)] Prove that $M$ is a divisor of $6$.
 \item[(ii)]
   If \(m = 6\), then \(G = S_4\);
   If \(m = 3\), then \(G = A_4\);
   If \(m = 1\), then \(G = V\);
   If \(m = 2\), then \(G = D_8\);
   If \(m = 2\), then \(G \cong D_8\), \(G \cong \Z_4\) or \(G \cong V\).
   (Hint: This exercise in group theory is Theorem~G.35.)
\end{itemize}
\end{excopy}

\begin{itemize}

 \item[(i)]
 (Taken from page~126 \cite{Rotman98}.)
 The subgroup $V$ is normal in \(S_4\) since it contains the identity
 and permutations with 2 2-cycles. Conjugates of such permutations
 keep this cycle patterns. Since $V$ contain \emph{all}
 such 2 2-cycles permutations, a conjugate of $V$ must be $V$ itself.
 Theorem~G.7 \cite{Rotman98} gives:
 \begin{equation*}
  G/(G\cap V) \cong GV/V.
 \end{equation*}
  Hence
 \begin{equation*}
  m = |GV/V| \mid 24/4 = 6.
 \end{equation*}

 \item[(ii)]
  As the hint said --- this is Theorem~G.35 in \cite{Rotman98}.
\end{itemize}

%%%%%
\item
\begin{excopy}
Let \label{ex:Gtrans}
$G$ be a subgroup of \(S_4\). If $G$ acts transitively on
\(X = \{1,2,3,4\}\) and \(|G/G\cap V| = 2\), then \(G \cong \Z_4\).
(If we merely assume that $G$ acts transitively on $X$, then \(|G|\) is
a multiple of $4$ (Theorem \(G.10\).
The added hypothesis \(G/G\cap V| = 2\)  removes the possibility \(G \cong V\)
when \(m = 2\) in Exercise~106.)

\end{excopy}

Note: See \ref{ss:trans:uncred}.

This is basically the result of Theorem~G.35 \cite{Rotman98}
and the discussion following.
The case of \(m = |G/G\cap V| = 2\) and \(G\cong V\)
can occur when
\begin{quotation}
G = \{1, (12)(34), (12), (34)\}.
\end{quotation}
But this cannot happen, since here it contradicts
the assumption that  $G$ acts transitively on $X$.
What is left to show is the following:

\begin{llem}
If \(G\subset S_4\) is isomorphic to
\begin{equation*}
 V = \{(1),(12)(34), (13)(24), (14)(23)\}
\end{equation*}
and \(|G/G\cap V| = 2\)
then $G$ does not act transitively on \(X=\{1,2,3,4\}\).
\end{llem}
\textbf{Proof.}
Label $G$ elements \(G=\{1,g_1,g_2,g_3\}\)
and \(G^{\#}=\{g_1,g_2,g_3\}\).
Now \(g_1^2 = g_2^2 = g_3^2 = 1\).
So elements of \(G^{\#}\)
must be a single 2-cylce, or
a two 2-cycles.
Also as in $V$ we the following equalities:
\(g_1 = g_2 g_3\),
\(g_2 = g_1 g_3\) and
\(g_3 = g_1 g_2\).

It is clear that \(G\neq V\) since otherwise  \(|G/G\cap V| = 1\).
Since in \(S_4\) there are only 3 permutations that are tw0 2-cycles,
\(G^{\#}\) must have some single 2-cycle element.

If by negation two of \(G^{\#}\) say \(g_i\), \(g_k\) are two 2-cycles
then the third \(g_{(1+2+3)-(i+j)}\) is a two 2-cycles as well
by simple computation:
\begin{equation}
((12)(34))\cdot((13)(24) = (14)(23)  \label{eq:1234}
\end{equation}
Similar equalities exist with permutations of (\ref{eq:1234}).
Now we get \(G=V\) which is a contradiction.

If \(G^{\#}\) contains two elements of single 2-cylce
with common index, say \(g_1 = (ij)\) and \(g_2 = (jk)\)
then \(g_3 = (ijk)\) and \(g_3^2 \neq 1\) is a contradiction.

Therefore, \(G^{\#}\) must contain two elements
of single cycles that are disjoint.
The third element, must be a product of the first two which is
a concatenation, and so
\begin{equation*}
G^{\#} = \{(ij), (k\ell), (ij)(k\ell)\}
  \qquad \textrm{where} \quad
 \{i,j,k,l\} = X.
\end{equation*}
Clearly, \(G = \{1\} \cup G^{\#}\) does not act transitively on $X$.
\proofend


%%%%% 108
\item
\begin{excopy}
Compute the Galois group over \Q\ of \(x^4 + x^2 - 6\).
\end{excopy}

We have the factorization
 (see \ref{ss:mis:polynam})
\begin{equation*}
f(x) = x^4 + x^2 - 6 = (x^2 - 2)(x^2 + 3)
\end{equation*}
in \(\Q[x]\).
The roots of \(f(x)\)
are \(\pm\sqrt{2}\) and \(\pm3i\) and so the splitting field
of \(f(x)\) is \(E = \Q(\sqrt{2},\sqrt{3}i\)
and $E$ is linearly independently spanned by \(\{\sqrt{2}, \sqrt{3}i\}\).
It is clear that for any \(g\in G = \Gal(E/\Q)\)
the equalities \(g(\sqrt{2}) = \pm\sqrt{2}\)
and \(g(\sqrt{3}i) = \pm\sqrt{3}i\).
Thus let for any \(w = q_1 \sqrt{2} + q_2 \sqrt{3}i \in E\)
with \(q_1,q_2\in\Q\),
define the mappings
\begin{eqnarray*}
g_1(q_1 \sqrt{2} + q_2 \sqrt{3}i) & = & -q_1 \sqrt{2} + q_2 \sqrt{3}i \\
g_2(q_1 \sqrt{2} + q_2 \sqrt{3}i) & = & +q_1 \sqrt{2} - q_2 \sqrt{3}i \\
\end{eqnarray*}
and \(G = \{1,g_1,g_2,g_1g_2\} \cong V\).

%%%%% 109
\item
\begin{excopy}
Compute the Galois group over \Q\ of \(f(x) = x^4 + x^2 + x + 1\).
\end{excopy}

Since \(f(x) = (x^5 - 1)/(x - 1)\) the roots
of \(f(x)\) are the primitive 5th roots of unity:
\(\{e^{2\pi k/5}\}_{k=1}^4\).
Thus the splitting field of \(f(x)\) is \(E = \Q(e^{2\pi i/5})\)
and so \(G = \Gal(E/\Q) \cong \Z_4\).

%%%%% 110
\item
\begin{excopy}
Compute the Galois group over \Q\ of \(f(x) = 4x^4 + 12x + 9\).
(Hint. Prove that \fx\ is irreducible in two steps:
first show it has no rational roots, and then use
\index{Descartes}
Descartes's method for the quantic formula and Exercise~64
to show that \fx\ is not the product of two quadratics over \Q.)
\end{excopy}

We will show that \(f(x)\) is irreducible in \(\Q[x]\).
By first showing there is no linear (solution) factor,
and later that there is no quadratic factor.

\index{Eisenstein criterion}
Note that the assumptions
of Eisenstein criterion would fails here with \(p=3\) since \(p^2=a_0 = 9\).
But still, \(f(x)\) is irreducible in \(\Q[x]\) since from
Exercise~\ref{ex:QZx:rs}, a rational root
in a reduced \(r/s\) form must satisfy \(r \mid 9\) and \(s\mid 4\).
So \(r\in \{0, \pm 1, \pm 3, \pm 9\}\)
and \(s\in \{\pm 1, \pm 2, \pm 4\}\).
For ``smooth'' integer computation:
\begin{eqnarray*}
f(n/d) = 4n^4/d^4 + 12n/d + 9 = (4n^4 + 12nd^3 + 9d^4)/d^4
\end{eqnarray*}
We use the following
\index{Python}
\textbf{Python} script:
{\scriptsize % q4_4_00_12_9.py
\verbatiminput{q4_4_00_12_9.py}
}

Indeed there are no roots as the following tedious computations
done by the above script show:

{\small
\begin{eqnarray*}
\input{q4_4_00_12_9.out}
\end{eqnarray*}
}

Now if by negation \(f(x)\) is reducible in \(\Q[x]\),
then it is the monic
\begin{equation}  \label{eq:fqaud:monic}
\tilde{f}(x) = (1/4)f(x) = x^4 + 3x + 9/4
\end{equation}
must be factorized into two quadratics as in the mothod
of Descartes as in page~48 \cite{Rotman98}. There we have:
\begin{equation*}
 h(x) = x^4 + qx^2 + rx + s = (x^2 + kx + \ell)(x^2 - kx + m)
\end{equation*}
In our case \(q = 0\), \(r = 3\) and \(s = 9/4\).
Thus gives a simpler equation (see the analysis there)
for \(y = k^2\):
\begin{equation*}
k^6 + 2qk^4 + (q^4 - 4s)k^2 - r^2 =
y^3 - 9y^2 - 9 = 0.
\end{equation*}
We have seen in Exercise~\ref{ex:4polys:inQ}~(iv) that
the latter cubic polynamial is irreducible in \(\Q[x]\).

Now that it is established that \(f(x)\) is irreducible in \(\Q[x]\)
We will be able to use Theorem~106 \cite{Rotman98}.

\index{resolvent cubic}
The resolvent cubic \(g(x)\) of \(f(x)\) is
the same as the resolvent of
the monic \(\tilde{f}(x)\) of (\ref{eq:fqaud:monic})
since it has the same roots as of \(f(x)\). Using
Theorem~105 \cite{Rotman98} we have:
\begin{equation*}
g(x) = x^3 - 2qx^2 + (q^2 - 4s)x + r^2 = x^3 - 9x + 9.
\end{equation*}
We will now compute the Galois group \(G_g\) of \(g(x)\).
The discriminant of \(g(x)\) is by Theorem~100 \cite{Rotman98}
\begin{equation*}
D = -4(-9)^3 - 27\cdot9^2 = 2916 - 2187 = 729 = 3^6 = 81^2.
\end{equation*}
Since \(D>0\) from Theorem~104
\(g(x)\) has three real roots:
\begin{equation*}
-3.4114,\quad 1.1848,\quad 2.2267\qquad \textnormal{(Approximately)}
\end{equation*}
and
since \(\sqrt{D} = 81 \in \Q\) this theorem shows that \(G_g \cong \Z_3\).

Finally we use Theorem~106 \cite{Rotman98} to compute
the Galois group $G$ of \(f(x)\).
Since \(m = |G_g| = |\Z_3| = 3\)
we get (case (ii)) \(G \cong A_4\).


%%%%% 111
\item
\begin{excopy}
\begin{itemize}
 \item[(i)]
   Prove that a quintic polynomial over \Q\ is solvable by radicals
   if and only if its Galois group has order \(\leq 24\).
 \item[(ii)]
   Prove that an irreducible quintic over \Q\ is solvable by radicals
   if and only if its Galois group has order \(\leq 20\).
   (Hint: A subgroup $G$ of \(S_3\) is solvable if and only if \(|G|\leq 24\);
    see Theorem~G.40)
\end{itemize}
\end{excopy}

\begin{itemize}
 \item[(i)]
 The Galois group of the quintic is a subgroup of \(S_5\).
 Hence, this exercise solution is a result of Galois's Great Theorem~98
 and Theorem~G.40 \cite{Rotman98}.

 \item[(ii)]
 We already established that a quintic in \(\Q[x]\) is solvable by radicals
 iff its Galois group $G$ has order \(|G| \leq 24\).
 Let \(f(x)\in \Q[x]\) be an irreducible quintic, let \(E/F\)
 be its splitting field
 and let \(\alpha\in E\) be a root of \(f(x)\).
 Now the subfield \(\Q(\alpha)\subset E\) has an order
 \([\Q(\alpha):\Q] = 5\).
 Since \(\Q\) has characteristic $0$, the splitting field $E$
 is separable and so is a Galois extension. From the
 \index{fundamental theorem}
 (Fundamental) Theorem~84~(iv) \cite{Rotman98}
 we have:
 \begin{equation*}
   [\Gal(E/\Q):\Gal(E/\Q(\alpha)\,)] = [\Q(\alpha) : \Q] = 5
 \end{equation*}
 Thus \(5\mid \|\, |\Gal(E/\Q)|\) and so \(|\Gal(E/\Q)|\neq 24\).
 Since \(\Gal(E/\Q)\) is isomorphic to a subgroup of \(S_5\),
 we also have \(|\Gal(E/\Q)|\,\mid 120\) and so
 \(|\Gal(E/\Q)| \notin\{21,22,23\}\). Thus the condition of
 \(|\Gal(E/\Q)| \leq 24\) is equivalent in this case (of irreducible quintic)
 to \(|\Gal(E/\Q)| \leq 20\).
\end{itemize}

%%%%% 112
\item
\begin{excopy}
(Kaplansky)
\index{Kaplansky}
Let \(f(x)\in \Q[x]\) be an irreducible quartic with
Galois group $G$. If \fx\ has exactly two real roots, then either
\(G \cong S_4\) or \(G \cong D_8\).
\end{excopy}

We will use Theorem~106 \cite{Rotman98}.
Let \(g(x)\) be the resolvent cubic of \(f(x)\)
and $m$ be the order of the Galois group of \(g(x)\).
Clearly \(m \mid 6\).
If we show that $m$ is even, then \(m=6\) or \(m=2\).

Let \(\alpha_1,\alpha_2,\alpha_3,\alpha_4\) be the roots of \(f(x)\) satisfying
\(\alpha_1,\alpha_2\in \R\), \(\alpha_1 \neq \alpha_2\).
and \(\alpha_3 = \overline{\alpha_4} \in \C \setminus \R\).
Using the definitions of (\ref{eq:resolv:uvw})
we have
\begin{equation*}
u = (\alpha_1 + \alpha_2)(\alpha_3 + \alpha_4) \in \R.
\end{equation*}
% If by negation \(v\in \R\) then we
Since
\begin{equation*}
 v =  (\alpha_1 + \alpha_3)(\alpha_2 + \alpha_4)
\end{equation*}
We look at the imaginary part, using \(\alpha_3\alpha_4 \in \R\) we get:
\begin{eqnarray*}
 \Im(v)
 & = & \Im\left(\alpha_1 + \alpha_3)(\alpha_2 + \alpha_4)\right) \\
 & = & \Im(\alpha_1\alpha_4 + \alpha_2\alpha_3) \\
 & = & (\alpha_2 - \alpha_1)\Im(\alpha_3) \neq = 0\\
\end{eqnarray*}
and so \(v \in \C\setminus \R\).
Since the \(g(x)\in \Q[x]\) the third root
\(w = \overline{v} \in \C\setminus\R\).

So the Galois group \(G_g = \Gal(\Q(u,v,w)/\Q)\) of \(g(x)\) must contain the
subgroup
\(\{\id_{\Q(u,v,w)},\gamma\}\)
(isomorphic to \(S_2 \cong \Z_2\)),
where \(\gamma\) maps by conjunction (\(z \rightarrow \overline{z}\))
and maps \((u,v,w) \rightarrow (u,w,v)\) respectively.
Thus \(2 \mid m = |G_g| \in \{3,6\}\).

From Theorem~106 \cite{Rotman98},
\(G \cong S_4\) or \(G \cong D_8\) or \(G \cong \Z_4\).
Let $E$ be the splitting field of \(f(x)\).
We have \([\Q(\alpha_1),\Q] = 4\) and
\(\alpha_3,\alpha_4 \in E \setminus \R \subset E \setminus \Q(\alpha_1)\).
Hence,
\begin{equation*}
|\Gal(E/\Q)| = |\Gal(E/\Q(\alpha_1))| \cdot |\Gal(\Q(\alpha)/\Q| > 4 = |\Z_4|
\end{equation*}
and the last option \(G \cong \Z_4\) is actually impossible.


%%%%% 113
\item
\begin{excopy}
(Kaplansky)
\index{Kaplansky}
Let \(x^4 + ax^2 + b\) be an irreducible polynomial over \Q\ having
Galois group $G$.
\begin{itemize}
 \item[(i)] If $b$ is a square in \Q, then \(G \cong V\).
 \item[(ii)]
   If $b$ is not a square in \Q\ but \(b(a^2-4b)\) is a square,
   then \(G \cong \Z_4\).
 \item[(iii)]
   If neither $b$ nor \(b(a^2-4b)\) is a square, then \(G \cong D_8\).
\end{itemize}
\end{excopy}

Call \(f(x) = x^4 + ax^2 + b\).
Its resolvent polynomial is
\begin{equation*}
g(x) = x^3 -2ax^2 + (a^2 -4b)x.
\end{equation*}
Clearly, for \(g(x)\) we have the root \(x=0\in \Q\)
and also the roots of the quadratic
\begin{equation*}
h(x) = g(x)/x = x^2 - 2ax + a^2 -4b
\end{equation*}
The discriminant of \(h(x)\) is \(D = 4a^2 - 4(a^2 -4b) = 4^2b\).

\begin{itemize}
 \item[(i)]
   If $b$ is a square in \Q\ then \(h(x)\) has $2$ rational roots
   and \(g(x)\) has $3$ rational roots. Therefore the Galois group
   \(G_g\) of \(g(x)\) is trivial and \(|G_g| = 1\).
   By Theorem~106~(iii) \cite{Rotman98}, \(G \cong V\).
\end{itemize}

If $b$ is not a square in \Q\ then \(g(x)\) has two conjugate non real
roots. Therefore, the Galois group \(G_g\) of \(g(x)\)
is isomorphic to \(S_2\), that is the identity mapping
and the conjunction mapping. We have \(m = |G_g| = 2\)
and by Theorem~106~(iv) \cite{Rotman98},
\begin{equation} \label{eq:D8Z4}
G \cong D_8 \qquad\textrm{or}\qquad G \cong \Z_4.
\end{equation}


Solving \(f(x)\) as a quadratic in \(x^2\) gives
\begin{equation*}
x^2 = \left(-a \pm \sqrt{a^2 - 4b}\right)/2.
\end{equation*}
Now since
\begin{equation*}
 \pm\sqrt{\left(-a \pm\sqrt{a^2-4b}\right)/2}
 \;=\; \pm\sqrt{\left(a \pm\sqrt{a^2-4b}\right)/2}\,\cdot i
\end{equation*}
This gives the four solutions:
\newcommand{\kapsol}[1]{\sqrt{\left(a #1\sqrt{a^2-4b}\,\right)/2}\,\cdot i}
\begin{alignat*}{2}
\alpha_1 &= \kapsol{+} \qquad  \alpha_2 &= -\kapsol{+} & \\
\alpha_3 &= \kapsol{-} \qquad  \alpha_4 &= -\kapsol{-} & \\
\end{alignat*}

Since \(f(x)\) is irreducible, we have \([\Q(\alpha_1):\Q] = 4\).
We will also use the following equality:
\begin{eqnarray}  \label{eq:kapsol13}
\alpha_1 \alpha_3
 & = & \kapsol{+} \cdot \kapsol{-} \\
 & = & (-1)\sqrt{(a^2 - (a^2 - 4b))/4} \notag \\
 & = & -\sqrt{b}. \notag
\end{eqnarray}

From the solution above we have:
\begin{equation*}
  \alpha_1^2 = \left(-a + \sqrt{a^2 - 4b}\right)/2.
\end{equation*}
Thus
\begin{equation} \label{eq:kapldisc}
\sqrt{a^2 - 4b} \in \Q(\alpha_1).
\end{equation}

Back to the exercise's last two items.

\begin{itemize}
 \item[(ii)]
  Assume \(b(a^2-4b)\) is a square in \Q.
  Clearly \(\alpha_2\in\Q(\alpha_1)\).
  We will show \(\alpha_3\in\Q(\alpha_1)\).
  If \(a^2-4b = 0\) then it is obvious, so we can now assume \(a^2-4b \neq 0\).
  From the (\ref{eq:kapldisc}) we get \(1/\sqrt{a^2 - 4b} \in \Q(\alpha_1)\).
  Since  \(\sqrt{b(a^2-4b)} \in \Q\) we have \(\sqrt{b} \in \Q(\alpha_1)\).
  We note that \(\alpha_1 \neq 0\), since otherwise \(f(0) = b = 0\),
  but \(\sqrt{b}\notin \Q\).
  Using (\ref{eq:kapsol13}) we get
  \begin{equation*}
  \alpha_3 = \sqrt{b}/\alpha_1 \in \Q(\alpha_1).
  \end{equation*}
  Since \(\alpha_4 = -\alpha_3\) we see that \(f(x)\)
  splits in \(\Q(\alpha_1\)) and therefore \(|G| = [\Q(\alpha_1):\Q] = 4\).
  Therefore $G$ cannot be isomorphic to \(D_8\) and
  so by (\ref{eq:D8Z4}) we have \(G \cong \Z_4\).

 \item[(iii)]

  Note: This exercise is similar to exercise~8 on page~322
  \cite{Lang94} chapter \textsf{VI}.

  Assume \(\sqrt{b(a^2-4b)}\notin \Q\) and by negation
  assume \(G \cong \Z_4\).
  By Exercise~\ref{ex:trans}, $G$ acts transitively
  on
  \begin{equation*}
   \{\alpha_1, \alpha_2, \alpha_3, \alpha_4\} =
   \{\alpha_1, -\alpha_1, \alpha_3, -\alpha_3\}
  \end{equation*}
  There exists  \(\iota\in G\)  such that
  \(\iota(\alpha_1) = -\alpha_1\). We let \(\alpha=\alpha_1\) and so
  \begin{equation*}
  \iota(-\alpha) =
  \iota((-1)\alpha) =
  (-1)\iota(\alpha) =
  (-1)(-\alpha) = \alpha.
  \end{equation*}
  Clearly \(\iota^2= 1_{G}\).
  Let \(g\in G\) a generator of $G$. Since \(g^4=1_{G}\) we must have
  \(g^2 = \iota\) and \(g(\alpha) = \beta = \pm\alpha_3\).
  To summerize
  \begin{equation*}
   g = \left(\begin{array}{rrrr}
             \alpha &  \beta  & -\alpha & -\beta \\
             \beta  & -\alpha & -\beta  & \alpha \\
             \end{array}\right).
  \end{equation*}
  We note that since \(\sqrt{b}\notin \Q\) we know that \(b\neq 0\)
  and by being solutions, we must have \(\alpha \neq 0 \neq \beta\).
  Computing:
  \begin{equation*}
  g\left(\frac{\alpha}{\beta} - \frac{\beta}{\alpha}\right) =
  \frac{\beta}{-\alpha} - \frac{-\alpha}{\beta} =
  \frac{\alpha}{\beta} - \frac{\beta}{\alpha}\quad.
  \end{equation*}
  Since \(G = \langle g \rangle\), we now have
  \begin{equation*}
  \frac{\alpha}{\beta} - \frac{\beta}{\alpha}
  = \frac{\alpha^2 - \beta^2}{\alpha\beta} \in \Q(\alpha,\beta)^G = \Q
  \end{equation*}
  With \(f(x) = (x-\alpha)(x+\alpha)(x-\beta)(x+\beta)\)
  we have \(b = \alpha^2\beta^2\) and \(a = -\alpha^2 - \beta^2\).
  We now get
  \begin{eqnarray*}
  b(a^2 - 4b)
  &=&  \alpha^2\beta^2\left((a^2 + \beta^2)^2 - 4\alpha^2\beta^2\right) \\
  &=&  \alpha^2\beta^2(a^2 - \beta^2)^2 \\
  &=&  \left(\alpha\beta(a^2 - \beta^2)\right)^2 \\
  \end{eqnarray*}
  Therefore we get
  \begin{equation*}
  \sqrt{b(a^2 - 4b)}
     = \alpha\beta(a^2 - \beta^2)
     = b\frac{a^2 - \beta^2}{\alpha\beta} \Q.
  \end{equation*}
  which is a contradiction to the assumption of \(G \cong \Z_4\).
  By (\ref{eq:D8Z4}) we finally get \(G \cong D_8\).
\end{itemize}

%%%%%%%%%%%%%%%%
\end{myenumerate}

Before proceeding to the final exercise, we need to 
establish some general results.


The footnote on page~103 \cite{Rotman98} mentions
an alternative resolvent cubic (see above \ref{ss:resolv:foot} note).
This is the resolvent used in \cite{Kap74}.
This resolvent is also used for analysing and solving the general
\index{quartic} equation in  \cite{Dickson1926} \S~79, pages 142--143).
We will bring here that analysis, adapted to our formulation
as the following definition and theorem.

\begin{lthm}
Let \label{lthm:quar:dick}
\(f(x) = x^4 + bx^3 + cx^2 + dx + e\) be a polynomial in \(F[x]\)
with the roots \(\alpha_1,\alpha_2,\alpha_3,\alpha_4\).
If
\begin{eqnarray}
u' &=& \alpha_1 \alpha_2 + \alpha_3 \alpha_4 \notag \\
v' &=& \alpha_1 \alpha_3 + \alpha_2 \alpha_4  \label{eq:resolv:uvw:tags} \\
w' &=& \alpha_1 \alpha_4 + \alpha_2 \alpha_3, \notag 
\end{eqnarray}
then
\begin{itemize}
 \item[\textnormal{(i)}]
  The monic cubic polynomial with roots $u'$, $v'$, $w'$
  is the
  \index{product resolvent@\(\pi\)-resolvent}
  \index{resolvent}
  \textbf{\(\pi\)-resolvent} \textnormal{(local term)} \(h(x)\)
  \begin{equation} \label{eq:prod:resolvent}
   h(x) = (x-u')(x-v')(x-w') =
     x^3 - cx^2 + (bd - 4e)x + 4ce - b^2e - d^2.
  \end{equation}

 \item[\textnormal{(ii)}]
     The following equalities hold:
     \begin{eqnarray}
      u' - v' & = & (\alpha_1 - \alpha_4)(\alpha_2 - \alpha_3) \notag \\
      u' - w' & = & (\alpha_1 - \alpha_3)(\alpha_2 - \alpha_4) 
                                         \label{eq:diff:uvw:tag} \\
      v' - w' & = & (\alpha_1 - \alpha_2)(\alpha_3 - \alpha_4). \notag
     \end{eqnarray}
     Thus the discriminants of \(h(x)\) and \(f(x)\) are equal.

 \item[\textnormal{(iii)}]
   If \(f(x)\in \Q[x]\) and
   \(G = \Gal(\Q(\alpha_1,\alpha_2,\alpha_3,\alpha_4)/\Q)\),
   then the fixed field 
   \begin{equation*}
   \Q^{V\cap G} = \Q(u',v',w').
   \end{equation*}

  \iffalse % probably not true!
   %Using the definitions in (\ref{eq:resolv:uvw}),
   %we have \(F(u,v,w) = F(u',v',w')\).
   %Therefore,
   %\begin{equation*}
   %  \Gal(F(u,v,w)/F) = \Gal(F(u',v',w')/F).
   %\end{equation*}
  \fi

\end{itemize}
\end{lthm}
\textbf{Proof.}
The original \(f(x)=0\) equation is:
\begin{equation*}
x^4 + bx^3 + cx^2 + dx + e = 0.
\end{equation*}
``Square completion'' gives:
\begin{equation*}
(x^2 + bx/2)^2 = (b^2/4-c)x^2 - dx - e.
\end{equation*}
Adding \((x^2+bx/2)y + y^2/4\) to both sides gives
\begin{equation} \label{eq:quar:ferrari}
(x^2 + bx/2+y/2)^2 = (b^2/4 -c + y)x^2 + (by/2-d)x + (y^2/4 - e).
\end{equation}
Requiring that the right hand side of the last equation
will be a square
\begin{equation*}
  \mu^2x^2 + 2\mu\nu x + \nu^2 = (\mu x + \nu)^2
\end{equation*}
of a linear function in $x$ becomes
the condition:
\begin{equation*}
(2\mu\nu)^2 =\; (by/2-d)^2 = 4(b^2/4 -c + y)(y^2/4 - e) \;= 4(\mu^2)(\nu^2)
\end{equation*}
or equivalently --- the \(\pi\)-resolvent:
\begin{equation} \label{eq:hy}
h(y) =  y^3 - cy^2 + (bd - 4e)y + 4ce - b^2e - d^2 = 0.
\end{equation}
Let $u'$ be a root of \(h(y)\)
then equation (\ref{eq:quar:ferrari}) implies
\begin{equation} \label{eq:x2u:pm}
x^2 + bx/2+u'/2 = \pm(\mu x + \nu)
\end{equation}
Let \(\alpha_1\), \(\alpha_2\) be the solution of (\ref{eq:x2u:pm}$+$)
and \(\alpha_3\), \(\alpha_4\) be the solution of (\ref{eq:x2u:pm}$-$).
Clearly
\begin{eqnarray*}
\alpha_1 \alpha_2 & = & u'/2 - \nu \\
\alpha_3 \alpha_4 & = & u'/2 + \nu \\
\end{eqnarray*}
Thus,
\begin{equation*}
 \alpha_1 \alpha_2 + \alpha_3 \alpha_4 = u'.
\end{equation*}
Similarly, if we choose the other
roots $v'$ and $w'$ of \(h(y)\) of (\ref{eq:hy})
we get similar relations with \(\alpha_1,\alpha_2,\alpha_3,\alpha_4\)
with different pairing.
Hence the three roots of \(h(y)\) are as in (\ref{eq:resolv:uvw:tags}).
The uniqeness of monic polynomial with given roots, completes
the (i) part of the theorem.

To show (ii), let us compute
\begin{eqnarray*}
u' - v'
 & = &   (\alpha_1 \alpha_2 + \alpha_3 \alpha_4)
       - (\alpha_1 \alpha_3 + \alpha_2 \alpha_4) \\
 & = &     \alpha_1 \alpha_2 - \alpha_1 \alpha_3
         + \alpha_4 \alpha_2 - \alpha_4 \alpha_3 \\
 & = &   (\alpha_1 - \alpha_4)(\alpha_2 - \alpha_3)
\end{eqnarray*}
With similar simple derivation we get  the other two equalities.
Multiplying these three equalities gives the equality 
between the discriminant of \(h(x)\)
and the discriminant of \(f(x)\).

Showing (iii) is repeating arguments used
in \cite{Rotman98} --- bottom paragraph of page~102
on equalities (\ref{eq:resolv:uvw:tags}).
That is, for every \(\sigma\in V\)
looking at the expressions with \(\alpha\)'s we see that
\(\sigma(u') = u'\),
\(\sigma(v') = v'\) and
\(\sigma(w') = w'\).
Conversely, since \(f(x)\) is irreducible in \(\Q[x]\)
and so separable, all the roots \(\alpha_i\) are different
and if \(\sigma\in S_4\supset G\) fixes $u'$, $v'$ and $w'$
then \(\sigma\in V\).
\proofend

Theorem~106 in \cite{Rotman98} deals with the resolvent \(g(x)\).
But it can apply to \(h(x)\) as well, since
the only thing the proof uses of the resolvent is that
its splitting field \(\Q(u,v,w)\) is the fixed field of \(V\cap G\).
As we have just shown, this holds also for \(\Q(u',v',w')\) of \(h(x)\).

Theorem~43 in \cite{Kap74} is very similar to Theorem~106.
The main additions are for the \(m=2\) case.
Here is an adapted summary, where 
the last two items bring the extra observations.

\begin{llem} \label{llem:quartic:m2}
Let \(f(x)=\in \Q[x]\) be an irreducible quartic.
Let $E$ be its splitting field, with Galois
group $G$. Let \(h(x)\) be the cubic \(\pi\)-resolvent
defined in (\ref{eq:prod:resolvent}).
Denote \(B = \Q(u',v',w')\) the splitting field of \(h(x)\)
and \(H = \Gal(B/\Q)\).

If \(|H| = 2\) then,
\begin{itemize}
 \item[(i)] \([E:B] = 2\).
 \item[(ii)] \(H = V \cap G\).
 \item[(iii)] \(G \cong \Z_4\) or \(G \cong D_8\).
 \item[(iv)] If \(G \cong D_8\), then \(G\cap V = V\)
       and \(f(x)\) is irreducible in \(B[x]\).
 \item[(v)] If \(G \cong \Z_4\), then \(|G\cap V| = 2\)
       and \(f(x)\) factors in \(B[x]\).
\end{itemize}
\end{llem}
\textbf{Proof.}
\begin{itemize}

 \item[(i)] 
  Since \(E/\Q\) is Galois, we have \([E:\Q] = |G|\).
  By Theorem (Fundemental of Galois Theory)~84 \cite{Rotman98},
  \(B = E^H\) and
  \begin{equation*}
  [E:B] = [E:\Q]/[B:\Q] = |G|/[G:H] = |H| = 2.
  \end{equation*}

 \item[(ii),(iii)] 
  By the remark preceding this Theorem, we can apply 
  Theorem~106 \cite{Rotman98}, with \(m=2\).

 \item[(iv),(iv)]
  The analysis preceding Theorem~106 \cite{Rotman98} shows that
  \(B = E^{V\cap G}\) and so 
  \(H = G / V \cap G\). %  and \(|G/V\cap G| = |B/
  giving the equality: 
  \begin{equation*}
   |G| / |V \cap G| = 2.
  \end{equation*}
  Therefore:
  \begin{itemize}
   \item[\(D_8\)] --- 
    If \(|G|=8\) we have \(|V \cap G| = 4\) and thus \(V \cap G = V\).
    Hence \(\Gal(E/B)\) still acts transitively on the roots
    and \(f(x)\) is irreducible in  \(B[x]\).
   \item[\(\Z_4\)] --- 
    If \(|G|=4\) we have \(|V \cap G| = 2\).
    Thus \(|\Gal(E/B)| = 2 < 4\) and \(f(x)\) 
    that splits in $E$, cannot be irreducible \(B[x]\).
  \end{itemize}
   
\end{itemize}
\proofend

We will also need the following simple square root membership rule.

\begin{llem} \label{llem:sqrts:in}
Let $K$ be a field with \(\fchar K \neq 2\).
Let \(a_i \in K\) such that \(\sqrt{a_i}\notin K\) for \(i=1,2\).
If \(\sqrt{a_1}\in K(\sqrt{a_2})\) then \(\sqrt{a_1 a_2} \in K\).
\end{llem}
\textbf{Proof.}
Viewing \(K(\sqrt{a_2})\) as a vector space over $K$, we know 
it has dimension $2$. Thus 
\begin{equation*}
\sqrt{a_1} = u + v \sqrt{a_2}
\end{equation*}
for some \(u,v\in K\). Taking squares, we get
\begin{equation*}
 \sqrt{a_1 b_1} = \left(-a_1 - v^2a_2 + u^2\right)/2v \in K.
\end{equation*}
\proofend

\iffalse
\begin{llem} \label{llem:sqrt:exts}
Let $K$ be a field with \(\fchar K \neq 2\).
If \(a,b\in K\) and \(\sqrt{a}\in K(\sqrt{b})\setminus K\) 
then \(\sqrt{ab}\in K\).
\end{llem}
\textbf{Proof.}
Viewing \(K(\sqrt{b})\) as a vector space over $K$
for some \(k_1,k_2\in K\),
we have the following linear combination 
with simple derivations:
\begin{eqnarray*}
\sqrt{a} &=& k_1+k_2\sqrt{b} \\
k_1 &=& k_2\sqrt{b} - \sqrt{a} \\
k_1^2 &=& k_2^2 b - 2k_2\sqrt{ab} + a \\
\sqrt{ab} &=& \left(a + k_2^2 b - k_1^2\right) / 2k_2 \in K.
\end{eqnarray*}
\proofend
\fi

%%%%%%%%%%%%%%%%%%%%
Back to the last exercise.

\begin{myenumerate}
%%%%% 114
\item
\begin{excopy}
(Kaplansky)
\label{ex:Kaplansky:reciprocal}
\index{Kaplansky}
Let \(x^4 + bx^3 + cx^2 + bx + 1 \in \Q[x]\)
have Galois group $G$.
\begin{itemize}
 \item[(i)]
   If \(h = c^2 + 4c + 4 - 4b^2\) is a square in \Q, then \(G \cong V\).
 \item[(ii)]
  If $h$ is not a square in \Q\ but \(h(b^2 - 4c + 8)\) is a square,
  then \(G \cong \Z_4\).
 \item[(iii)]
   If niether $h$ nor \(h(b^2 - 4c + 8)\) is a square in \Q,
   then \(G \cong D_8\).
\end{itemize}
\end{excopy}

Call \(f(x) = x^4 + bx^3 + cx^2 + bx + 1\).
It is easy to find cases where \(f(x)\) splits in \(\Q[x]\)
For example with \(b=4\) and \(c=6\),
\begin{equation*}
(x-(-1))^4 = x^4 + 4x^3 + 6x^2 + 4x + 1.
\end{equation*}
In this case, \(G \cong S_1\) is trivial, formally contradicting the exercise.
As mentioned in \ref{ss:miss:irr},
we will \emph{add} the assumption that \(f(x)\) is irreducible
and let $E$ be the splitting field of \(f(x)\).

The polynomial \(f(x) = x^4 + bx^3 + cx^2 + bx + 1\) 
\index{reciprocal}
is reciprocal. That is \(f(\alpha) = 0\) iff \(f(1/\alpha) = 0\).
To see this we note that \(f(0)\neq 0 \) and so the equation \(f(x)=0\)
is equivalent to
\begin{equation} \label{eq:recip}
 x^2 + bx + c + b(1/x) + (1/x)^2 = 0.
\end{equation}
By simplifying (\ref{eq:hy})
the \(\pi\)-resolvent in this specific case (\(d=b\), \(e=1\)) is
\begin{equation*}
\eta(x) =  x^3 - cx^2 + (b^2 - 4)x + 4c - 2b^2.
\end{equation*}
Let $m$ be the order of the Galois group of \(\eta(x)\).
Say the four roots of \(f(x)\) are \(\alpha_1,\alpha_2,\alpha_3,\alpha_4\).
We can assume 
\(\alpha_2 = 1/\alpha_1\) and
\(\alpha_4 = 1/\alpha_3\).
Thus one of the roots of \(\eta(x)\) is 
\begin{equation*}
u' = \alpha_1 \alpha_2 + \alpha_3 \alpha_4 = 1 + 1 = 2.
\end{equation*}
The other roots are $v'$ and $w'$.
By dividing, we get
\begin{eqnarray*}
\tilde{\eta}(x) 
 &=& (x-v')(x-w') \\
 &=& \eta(x)/(x-2) \\
 &=& \left(x^3 - cx^2 + (b^2 - 4)x + 4c - 2b^2\right)\,/\,(x-2)\\
 &=& x^2 + (-c + 2)x + (-2c + b^2).
\end{eqnarray*}
The discriminant 
\begin{equation*}
h = \Delta_{\tilde{\eta}} = (-c + 2)^2 - 4(-2c + b^2) = c^2 + 4c +b - 4b^2.
\end{equation*}

\begin{itemize}
 \item[(i)]
   If \(\sqrt{h}\in \Q\) then \(\eta(x)\) splits in \(\Q[x]\)
   and thus \(m=1\) and by Theorem~106(iii) \cite{Rotman98}, we have \(G=V\).
 \item[(ii)]
   If \(\sqrt{h}\notin \Q\) then \(m=2\).
   By Theorem~106(iv) \cite{Rotman98}, we have \(G=D_8\) or \(G=\Z_4\).
   Solving \ref{eq:recip} for \(x+1/x\)
   we get 
   \begin{equation*}
   (x+1/x)^2 + b(x+1/x) + c - 2 = 0
   \end{equation*}
   and \(x+1/x = (-b \pm\sqrt{b^2 -4c + 8})/2\).
   Let the last discriminant \(\zeta = b^2 -4c + 8\).
   Therefore, \(\sqrt{\zeta} = 2x + 2/x + b \in \Q(\alpha_1)\).
   Clearly \(\alpha_2 = 1/\alpha_1 \in \Q(\alpha_1)\).   
   Now if \(\sqrt{h\zeta}\in \Q\) then \(\sqrt{h}\in \Q(\alpha_1)\).
   Thus \(\eta(x)\) splits in \(\Q(\alpha_1)\) and
   so \(u',v',w'\in\Q(\alpha_1)\).
   
   Since \(\sum_{j=1}^4 \alpha_i = -b\) and with (\ref{eq:diff:uvw:tag})
   we have the linear system with \(\alpha_3\), \(\alpha_4\)
   \begin{eqnarray*}
    \alpha_3 + \alpha_4 & = & -\alpha_1 - \alpha_2 - b\\
    \alpha_3 - \alpha_4 & = & (v' - w')/(\alpha_1 - \alpha_2)
   \end{eqnarray*}
   Solving it shows that \(\alpha_3, \alpha_4\in \Q(\alpha_1)\)
   and so  \(f(x)\) splits in \(\Q(\alpha_1)\).
   Therefore \(|G| = [\Q(\alpha_1):\Q] = 4\) 
   eliminating the \(G=D_8\) possibility. Thus we have shown \(G=\Z_4\).
   
 \item[(iii)]
   As in the beginning of (ii), we have \(G=D_8\) or \(G=\Z_4\).
   Let $E$ be the splitting field of \(f(x)\) and 
   by negation assume  \(G=\Z_4\).
   By Lemma~\ref{llem:quartic:m2} above, \(f(x)\) factors
   in 
   \begin{equation*}
    K = \Q(u',v',w') = \Q(v',w') = \Q(v') = \Q(w') = \Q(\sqrt{h}).
   \end{equation*}
   The equalities are true since \(u'=2\in\Q\)
   and the fact that $u'$ and $v'$ are roots of quadratic equation in \Q\
   for which $h$ is the discriminant.
   By Theorem~56 \cite{Rotman98} \(\Gal(E/\Q) = 4\).
   Hence by degree argumentation \(E = \Q(\alpha_1)\).
   Since \(K\subset E\), we must have \([E:K] = 2\).

   In (ii), we have seen that \(\sqrt{\zeta} \in E\) and so
   we have
   \begin{equation*}
        [E:\Q(\sqrt{h})] = [E:\Q(\sqrt{\zeta})] = 
        2 =
        [\Q(\sqrt{h}):\Q] = [\Q(\sqrt{\zeta}):\Q]
   \end{equation*}
   and so
   \begin{equation*}
    \Q \subsetneq \Q(\sqrt{h}), \Q(\sqrt{\zeta}) \subsetneq E.
   \end{equation*}
   \index{fundamental theorem}
   By the Fundamental Theorem~84 \cite{Rotman98}, since there is only
   one non trivial subgroup of \Z{4}, so there is only one non trivial 
   subfield of $E$ containing \Q. Therefore 
   \begin{equation*}
      \Q(\sqrt{h}) = \Q(\sqrt{\zeta}).
   \end{equation*}
   By local lemma~\ref{llem:sqrts:in}, we get the contradiction
   \(\sqrt{h\zeta} \in \Q\).

   % That was hard, and took me more than a week to solve...

  %% Unchecked reference that may be interesting:
  %%  Galois group is found in L. E. Dickson, 
  %%  "The Galois group of a reciprocal quartic aquation"
  %%  Amer. Math. Monthly 15. 

 \iffalse
   Putting \(\alpha=\alpha_1\) and \(\beta=\alpha_3\)
   The possibilities for factorization of 
   \begin{equation*}
   f(x) = (x-\alpha)(x-1/\alpha)(x-\beta)(x-1/\beta) \in K[x]
   \end{equation*}
   are the following:
   % \newcommand{\FcLab}[1]{\textbf{F\ensuremath{\bm{#1}}.}}
   \newcommand{\FcLab}[1]{\fbox{\textbf{\ensuremath{\bm{#1}}}}}
   \newcommand{\FcLabp}[1]{\FcLab{#1}\textbf{.}}
   \begin{itemize}

    \item[\FcLabp{\alpha}]
      \(x-\alpha\in K[x]\) or \(x-\beta\in K[x]\).

    \item[\FcLabp{\alpha+}]
      \((x-\alpha)(x-1/\alpha)\in K[x]\) and
      \((x-\beta)(x-1/\beta)\in K[x]\)

    \item[\FcLabp{\alpha\beta}]
      \((x-\alpha)(x-\beta)\in K[x]\) and
      \((x-1/\alpha)(x-1/\beta)\in K[x]\).

    \item [\FcLabp{\alpha/\beta}]
      \((x-\alpha)(x-1/\beta)\in K[x]\) and
      \((x-1/\alpha)(x-\beta)\in K[x]\).

   \end{itemize}

   In \FcLab{\alpha} we


   Therefore the discriminant
   \begin{equation*}
    h = (u'-v')(u'-w')(v'-w') \in \Q(u',v',w') = K
   \end{equation*}
   since \(u'=2\in\Q\).
   
 %%%
   Let $E$ be the splitting field of \(f(x)\).
   We know \([\Q(\alpha_1):\Q] = \gdeg f(x) = 4\)
   and \(\Q(\alpha_1) \subset E\).
   Since \(f(x)\in \Q[x]\) it is separable and so 
   \(4 = [\Q(\alpha_1):\Q] \mid [E:\Q] = |G| = 4\).
   Hence \(\Q(\alpha_1) = E\). 
 \fi
   
   
\end{itemize}


%%%%% 115
\item
\begin{excopy}
If a herring and a half cost a penny and a half,
how much does a dozen herring cost?
(Answer: One shilling.)
\end{excopy}

Why no hint?



   %%%%%%%%%%%%%
\end{myenumerate}
%%%%%%%%%%%%%%%%%

\iffalse
%%%%%%%%%%%%%%%%%%%%%%%%%%%%%%%%%%%%%%%%%%%%%%%%%%%%%%%%%%%%%%%%%%%%%%%%
%%%%%%%%%%%%%%%%%%%%%%%%%%%%%%%%%%%%%%%%%%%%%%%%%%%%%%%%%%%%%%%%%%%%%%%%
%%%%%%%%%%%%%%%%%%%%%%%%%%%%%%%%%%%%%%%%%%%%%%%%%%%%%%%%%%%%%%%%%%%%%%%%
\chapterTypeout{Next Chapter}

%%%%%%%%%%%%%%%%%%%%%%%%%%%%%%%%%%%%%%%%%%%%%%%%%%%%%%%%%%%%%%%%%%%%%%%%
%%%%%%%%%%%%%%%%%%%%%%%%%%%%%%%%%%%%%%%%%%%%%%%%%%%%%%%%%%%%%%%%%%%%%%%%
\section{Exercises (pages 0--0)}

%%%%%%%%%%%%%%%%
\begin{myenumerate}

%%%%%
\item
\begin{excopy}
\end{excopy}

   %%%%%%%%%%%%%
\end{myenumerate}
%%%%%%%%%%%%%%%%%

%%%%%
\item
\begin{excopy}
\begin{itemize}
 \item[(i)]
 \item[(ii)]
\end{itemize}
\end{excopy}

\begin{itemize}
 \item[(i)]
 \item[(ii)]
\end{itemize}

%%%
\fi


%%%%%%%%%%%%%%%%%%%%%%%%%%%%%%%%%%%%%%%%%%%%%%%%%%%%%%%%%%%%%%%%%%%%%%%%
%%%%%%%%%%%%%%%%%%%%%%%%%%%%%%%%%%%%%%%%%%%%%%%%%%%%%%%%%%%%%%%%%%%%%%%%
%%%%%%%%%%%%%%%%%%%%%%%%%%%%%%%%%%%%%%%%%%%%%%%%%%%%%%%%%%%%%%%%%%%%%%%%
            %%%%%%%%%%%%%%%%%
            % \part{Appendices}



%%%%%%%%%%%%%%%%%%%%%%%%%%%%%%%%%%%%%%%%%%%%%%%%%%%%%%%%%%%%%%%%%%%%%%%%
%%%%%%%%%%%%%%%%%%%%%%%%%%%%%%%%%%%%%%%%%%%%%%%%%%%%%%%%%%%%%%%%%%%%%%%%
%%%%%%%%%%%%%%%%%%%%%%%%%%%%%%%%%%%%%%%%%%%%%%%%%%%%%%%%%%%%%%%%%%%%%%%%
\appendix

%%%%%%%%%%%%%%%%%%%%%%%%%%%%%%%%%%%%%%%%%%%%%%%%%%%%%%%%%%%%%%%%%%%%%%%%
%%%%%%%%%%%%%%%%%%%%%%%%%%%%%%%%%%%%%%%%%%%%%%%%%%%%%%%%%%%%%%%%%%%%%%%%
%%%%%%%%%%%%%%%%%%%%%%%%%%%%%%%%%%%%%%%%%%%%%%%%%%%%%%%%%%%%%%%%%%%%%%%%
\chapterTypeout{Group Theory Dictionary}

%%%%%%%%%%%%%%%%%%%%%%%%%%%%%%%%%%%%%%%%%%%%%%%%%%%%%%%%%%%%%%%%%%%%%%%%
%%%%%%%%%%%%%%%%%%%%%%%%%%%%%%%%%%%%%%%%%%%%%%%%%%%%%%%%%%%%%%%%%%%%%%%%
\section{Notes}

On page~111,
\index{normal!subgroup}
\index{normal@\(\subnormal\)}
the \emph{Normal subgroup} item should present the notation
\(H\subnormal G\). It is presented in the \emph{Kernel} item
on previous page, but here is certainly not less important place.

On page~111,
\index{p-group@$p$-group}
the \emph{$p$-group} item.
Two fixes are needed for the last sentence that reads:
\begin{quotation}
If $G$ is finite, the \(|G|\) is a power of $p$.
\end{quotation}

The fixes:
\begin{itemize}
 \item Obvious `then' instead of `the'.
 \item It ought to be explicitly required that $G$ is a $p$-group.
\end{itemize}
Thus it should be:
\begin{quotation}
If $G$ is a finite $p$-group, then \(|G|\) is a power of $p$.
\end{quotation}

%%%%%%%%%%%%%%%%%%%%%%%%%%%%%%%%%%%%%%%%%%%%%%%%%%%%%%%%%%%%%%%%%%%%%%%%
%%%%%%%%%%%%%%%%%%%%%%%%%%%%%%%%%%%%%%%%%%%%%%%%%%%%%%%%%%%%%%%%%%%%%%%%
%%%%%%%%%%%%%%%%%%%%%%%%%%%%%%%%%%%%%%%%%%%%%%%%%%%%%%%%%%%%%%%%%%%%%%%%
\chapterTypeout{Group Theory Used in the Text}

%%%%%%%%%%%%%%%%%%%%%%%%%%%%%%%%%%%%%%%%%%%%%%%%%%%%%%%%%%%%%%%%%%%%%%%%
%%%%%%%%%%%%%%%%%%%%%%%%%%%%%%%%%%%%%%%%%%%%%%%%%%%%%%%%%%%%%%%%%%%%%%%%
\section{Notes}

%%%%%%%%%%%%%%%%%%%%%%%%%%%%%%%%%%%%%%%%%%%%%%%%%%%%%%%%%%%%%%%%%%%%%%%%
\subsection*{General}

%%%%%%%%%%%%%%%%%%%%%%%%%%%%%%%%%%%%%%%%%%%%%%%%%%%%%%%%%%%%%%%%%%%%%%%%
\subsection{Even Permutation}

Both this the preceding appendix refer to \emph{even} permutations
when discussing \(A_n\). The definition of even permutation
should follow a proof that an even permutation cannot be a prdoct
of odd number of transpositions.

In \cite{Lang94} (Proposition~5.3 page~31) for example,
a homomorphism \(\epsilon: S_n \rightarrow \{\pm1\}\) is constructed.
Transposition are mapped to \(-1\) and
the kernel is used to define \(A_n\).

%%%%%%%%%%%%%%%%%%%%%%%%%%%%%%%%%%%%%%%%%%%%%%%%%%%%%%%%%%%%%%%%%%%%%%%%
\subsection*{Sylow Groups}

\index{Sylow groups}
In page 118, there is a remark that ``one knows''
that the number of Sylow $p$-subgroups is \(1 \bmod p\).
This is actually an important tool in showing solvability
of many groups. This is needed in the proof of Theorem~G.40,
the part that is left to the reader.



%%%%%%%%%%%%%%%%%%%%%%%%%%%%%%%%%%%%%%%%%%%%%%%%%%%%%%%%%%%%%%%%%%%%%%%%
\subsection*{Implicit Usage of Later Lemma}

In page~122, the proof of Lemma G.26 implicitly uses the fact
that \(S_n/A_n\) is indeed a group.
But this can be established \emph{after} Lemma~G.28.

To summerize the needed fixes:
\begin{itemize}
 \item Present Lemma~G.28 \emph{earlier} than Lemma~G.26
       (to become Lemma~G.xx).
 \item In the proof of Lemma~G.26 change:
   \begin{quotation}
      Since \(S_n/A_n\) is abelian (it has order 2),
   \end{quotation}
   into:
   \begin{quotation}
      Since \([S_n:A_n]=2\), from Lemma~G.xx
      \(A_n \subnormal S_n\) and so \(S_n/A_n\) is a group
      which is abelian (it has order 2),
   \end{quotation}
\end{itemize}

%%%%%%%%%%%%%%%%%%%%%%%%%%%%%%%%%%%%%%%%%%%%%%%%%%%%%%%%%%%%%%%%%%%%%%%%
\subsection*{Simplification}

The proof of Lemma~G.28 can be simplified as follows:

\begin{quotation}
\renewcommand{\thesection}{G}
\setcounter{quotelems}{27} % to get 28
\begin{quotelems}
If $H$ is a subgroup of a group $G$ of index 2,
then $H$ is a normal subgroup of $G$.
\end{quotelems}
\setlength{\parindent=}{0pt}
\textbf{Proof.} If \(a\in H\) then clearly \(aHa^{-1}=H\).
Assume \(a\in G\setminus H\), then \(aH\cap H = \varnothing\).
Hence, by hypothesis, \(aH=G\setminus H\).
Similarly we have \(Ha=G\setminus H\)
and therefore \(aH=Ha\).
It follows that for any \(a\in G\),
we have \(aHa^{-1}=H\) and so \(H \subnormal G\). \proofend

\end{quotation}

%%%%%%%%%%%%%%%%%%%%%%%%%%%%%%%%%%%%%%%%%%%%%%%%%%%%%%%%%%%%%%%%%%%%%%%%
\subsection*{Crediting Lemma}

The proof of Theorem G.29 should credit Lemma~G.26 as well.
This is for using \(A_n = S'_n\).


%%%%%%%%%%%%%%%%%%%%%%%%%%%%%%%%%%%%%%%%%%%%%%%%%%%%%%%%%%%%%%%%%%%%%%%%
\subsection*{Index Typo}

In page~125, in the proof of Corollary~G.33,
Change: \(h\notin A_n\)
into: \(h\notin A_5\).

%%%%%%%%%%%%%%%%%%%%%%%%%%%%%%%%%%%%%%%%%%%%%%%%%%%%%%%%%%%%%%%%%%%%%%%%
%%%%%%%%%%%%%%%%%%%%%%%%%%%%%%%%%%%%%%%%%%%%%%%%%%%%%%%%%%%%%%%%%%%%%%%%
%%%%%%%%%%%%%%%%%%%%%%%%%%%%%%%%%%%%%%%%%%%%%%%%%%%%%%%%%%%%%%%%%%%%%%%%
\chapterTypeout{Ruler-Compass Constructions}

%%%%%%%%%%%%%%%%%%%%%%%%%%%%%%%%%%%%%%%%%%%%%%%%%%%%%%%%%%%%%%%%%%%%%%%%
%%%%%%%%%%%%%%%%%%%%%%%%%%%%%%%%%%%%%%%%%%%%%%%%%%%%%%%%%%%%%%%%%%%%%%%%
\section{Notes}

%%%%%%%%%%%%%%%%%%%%%%%%%%%%%%%%%%%%%%%%%%%%%%%%%%%%%%%%%%%%%%%%%%%%%%%%
\subsection{Label Lemma Cases}

The proof of Lemma~R.6 ends with
\begin{quotation}
\ldots\  returns us to the situation fo the second paragraph.
\end{quotation}

It may be better to use more formal reference.
By labeling the three cases for the intersection complex point \(\alpha\)
and refering to the second case.
Suggestive labels for different intersection type like:
\begin{center}
(LL) \quad (LC) \quad (CC)
\end{center}
--- for line(s) and circle(s) --- could work.


%%%%%%%%%%%%%%%%%%%%%%%%%%%%%%%%%%%%%%%%%%%%%%%%%%%%%%%%%%%%%%%%%%%%%%%%
\subsection{Detail Workout in Gauss \ensuremath{p}-gon Theorem}

\index{Gauss}
In Theorem~R.9 (Gauss) here is the ``forbidden factorization''.
Say the factorization of \(x^k+1\) that the proof shows is
\(x^k+1 = (x+1)g(x)\).
So with \(p-1 = s^2\), \(s=mk\) and \(x=2^m\) we get
\begin{equation*}
p = (p - 1) + 1 = 2^s + 1 = \left(2^m\right)^k + 1 = (2^m+1)g(2^m).
\end{equation*}

%%%%%%%%%%%%%%%%%%%%%%%%%%%%%%%%%%%%%%%%%%%%%%%%%%%%%%%%%%%%%%%%%%%%%%%%
\subsection{Fermat Non Prime}

\index{Fermat}
The remark on page~138 mentions \emph{Fermat primes} of the form
\(2^{2^t}+1\). It may be worth mentioning the number equality:
\begin{equation*}
 2^{2^5} + 1 = 4294967297 = 641 \times 6700417.
\end{equation*}


%%%%%%%%%%%%%%%%%%%%%%%%%%%%%%%%%%%%%%%%%%%%%%%%%%%%%%%%%%%%%%%%%%%%%%%%
\subsection{Constructible \ensuremath{n}-gon}

Theorem~R.11 on page 138 is presented without proof.
One direction could be easily proved, namely if
\(n=2^m p\) where $p$ is a Fermat prime, then a regular
$n$-gon is constructible.

\index{Gauss}
To prove, we construct the $p$-gon using Gauss Theorem~R.9.
Then given the angle \(\theta_p = 2\pi/p\), we bisect it $m$ times
and get the anle \(\theta_n = \theta_p / 2^m\)
which can determine the vertices of the $n$-gon.


%%%%%%%%%%%%%%%%%%%%%%%%%%%%%%%%%%%%%%%%%%%%%%%%%%%%%%%%%%%%%%%%%%%%%%%%
%%%%%%%%%%%%%%%%%%%%%%%%%%%%%%%%%%%%%%%%%%%%%%%%%%%%%%%%%%%%%%%%%%%%%%%%
%%%%%%%%%%%%%%%%%%%%%%%%%%%%%%%%%%%%%%%%%%%%%%%%%%%%%%%%%%%%%%%%%%%%%%%%
\chapterTypeout{Old-Fashioned Galois Theory}

%%%%%%%%%%%%%%%%%%%%%%%%%%%%%%%%%%%%%%%%%%%%%%%%%%%%%%%%%%%%%%%%%%%%%%%%
%%%%%%%%%%%%%%%%%%%%%%%%%%%%%%%%%%%%%%%%%%%%%%%%%%%%%%%%%%%%%%%%%%%%%%%%
\section{Notes}

%%%%%%%%%%%%%%%%%%%%%%%%%%%%%%%%%%%%%%%%%%%%%%%%%%%%%%%%%%%%%%%%%%%%%%%%
\subsection{Avoid Division by Zero} \label{ofgt:avoid:zdiv}

On page~141, the classical formulas are recalled.
Again, mentioning of how division by zero (\(z=-q/3y\))
as in \ref{ss:avoid:zdiv} is needed.

\iffalse % Really untrue!
 %%%%%%%%%%%%%%%%%%%%%%%%%%%%%%%%%%%%%%%%%%%%%%%%%%%%%%%%%%%%%%%%%%%%%%%%
 \subsection{Transposition Notation}
 In page~141, the text refers to transpositions (23) and (132).
 The should be notated in ``math mode'' as \((23)\) and \((132)\).
 Without math mode, the symbol "(23)" looks like a formula reference.
\fi


%%%%%%%%%%%%%%%%%%%%%%%%%%%%%%%%%%%%%%%%%%%%%%%%%%%%%%%%%%%%%%%%%%%%%%%%
\subsection{\ensuremath{\varphi_i}-s}

\index{Lagrange}
On page~142 quoting the work of Lagrange,
the expressions \(\varphi_i(\omega)\) for \(i=1,\ldots,n\)
are introduced.
Using
  \(\alpha_0,\alpha_1,\ldots,\alpha_{n-1}\)
and
  \(\varphi_0,\varphi_1,\ldots,\varphi_{n-1}\)
instead of \seqalphn\ and \seqn{\varphi}
may simplify formulas that follow.


%%%%%%%%%%%%%%%%%%%%%%%%%%%%%%%%%%%%%%%%%%%%%%%%%%%%%%%%%%%%%%%%%%%%%%%%
\subsubsection{Numbers or Functions}

\index{Lagrange}
On page~142 quoting the work of Lagrange,
when introducing \(\varphi_i(\omega)\) for \(i=1,\ldots,n\),
the text says:
\begin{quotation}
\ldots, define numbers
\end{quotation}
But later it refers to \(\varphi(1)\) and also after
defining \(\psi(\omega) = [\varphi_1(\omega)]^n\)
the values \(\psi(\omega^k)\) are used for \(k=1,\ldots,n-1\).
So formally, it may be more accurate to define the \(\varphi_i(w)\)
as the following \emph{functions}:
\begin{eqnarray*}
\varphi_1(x) &=& \alpha_1 + \alpha_2x + \cdots + \alpha_nx^{n-1}\\
\varphi_2(x) &=& \alpha_2 + \alpha_3x + \cdots + \alpha_1x^{n-1}\\
           &\vdots& \\
\varphi_n(x) &=& \alpha_n + \alpha_1x + \cdots + \alpha_{n-1}x^{n-1}
\end{eqnarray*}

%%%%%%%%%%%%%%%%%%%%%%%%%%%%%%%%%%%%%%%%%%%%%%%%%%%%%%%%%%%%%%%%%%%%%%%%
\subsubsection{Power Miss for Unit Root} \label{sss:powermiss}

Page~142 has:
\begin{equation*}
\varphi_i(\omega) = \omega^{-i-1}\varphi_1(\omega),
\end{equation*}
But the true equality is:
\begin{equation*}
\varphi_i(\omega) = \omega^{-(i-1)}\varphi_1(\omega),
\end{equation*}
This mistake ---
\(\omega^{-i-1}\) instead of
\(\omega^{-(i-1)}\) ---
is carried over later, though it does not by itself,
effects the essential results.

%%%%%%%%%%%%%%%%%%%%%%%%%%%%%%%%%%%%%%%%%%%%%%%%%%%%%%%%%%%%%%%%%%%%%%%%
\subsubsection{Working Out Equality --- \ensuremath{\varphi_i^n}}

Let us show the equality in the bootom of page~142.
For $i$ such that \(1\leq i \leq n\) we have (see \ref{sss:powermiss}):
\begin{equation*}
\left[\varphi_{i}(\omega)\right]^n
 = \left[\omega^{-(i-1)}\varphi_1(\omega)\right]^n
 = \left[\omega^{-(i-1)}\right]^n \left[\varphi_1(\omega)\right]^n
 = \left[\varphi_1(\omega)\right]^n
\end{equation*}

\iffalse % no need for this complication
For $i$ such that \(0\leq i < n\) we have (see \ref{sss:powermiss}):
\begin{eqnarray*}
\varphi_{i+1}(\omega)
 & = & \alpha_{i+1}\omega^0 + \cdots \alpha_n\omega^{n-i-1}
       + \alpha_1\omega^{n-i} \cdots + \alpha_{i}\omega^{n-1} \\
 & = & \omega^{n-i}\left(
         \alpha_{i+1}\omega^i + \cdots \alpha_n\omega^{n-1}
         + \alpha_1\omega^{n} \cdots + \alpha_{i}\omega^{i-1}
       \right) \\
 & = & \omega^{n-i}\varphi_1(\omega).
\end{eqnarray*}

Thus,
\begin{equation*}
\left[\varphi_{i+1}(\omega)\right]^n
  = \left[\omega^{n-i}\varphi_1(\omega)\right]^n \\
  = \left[\varphi_1(\omega)\right]^n
\end{equation*}
\fi


%%%%%%%%%%%%%%%%%%%%%%%%%%%%%%%%%%%%%%%%%%%%%%%%%%%%%%%%%%%%%%%%%%%%%%%%
\subsection{Lemma H.3 --- Determining Roots by \ensuremath{\psi(\omega^k)}-s}

\textbf{Suggestion}. Since complex numbers are involved, the symbol $i$
should not be used as an index.

%%%%%%%%%%%%%%%%%%%%%%%%%%%%%%%%%%%%%%%%%%%%%%%%%%%%%%%%%%%%%%%%%%%%%%%%
\subsubsection{Dimension Disappearing}

Lemma~H.3 claims that the $n$ roots of a polynomial
of degree $n$ are determined by \(n - 1\) numbers:
\begin{equation*}
\psi(\omega),
\psi(\omega^2),\ldots,\psi(\omega^{n-1})
\end{equation*}
Of course, as later the proof that follows shows,
the determination of the solutions also depends on
\begin{equation*}
\psi(\omega^0) = [\varphi(1)]^n = (-b_{n-1})^n.
\end{equation*}

The paragraph after the proof, discusses
the idea of inductive step of solving a polynomial of degree \(n-1\).
To support this discussion, the lemma could be
presented as something like
\begin{quotation}
\ldots the roots are determined as affine combination
with coefficients in $F$,
of primitive roots of \(n-1\) numbers.
\end{quotation}

%%%%%%%%%%%%%%%%%%%%%%%%%%%%%%%%%%%%%%%%%%%%%%%%%%%%%%%%%%%%%%%%%%%%%%%%
\subsubsection{Suspicious Solutions}

The (fixed in \ref{sss:powermiss}) relation
\begin{equation*}
\varphi_i(\omega) = \omega^{-(i-1)}\varphi_1(\omega)
\end{equation*}
does \emph{not} imply
\begin{equation*}
\varphi_i(\omega^j) = \omega^{-(i-1)}\varphi_1(\omega^j).
\end{equation*}
Seems like the proof in Lemma~H.3 uses this to get
\begin{equation*}
n\alpha_i = \omega^{-(i-1)}\sum_{j=0}^{n-1} \varphi_1(\omega^j).
\end{equation*}
Had this been true it would have meant that
the ration between roots \(\alpha_j/\alpha_1 = \omega^{-j}\)
is constant and in particular, independent of the polynomial coefficients.


%%%%%%%%%%%%%%%%%%%%%%%%%%%%%%%%%%%%%%%%%%%%%%%%%%%%%%%%%%%%%%%%%%%%%%%%
\subsection{Lagrange Rational Function Theorem}

\index{Lagrange}
Theorem H.4 of Lagrange is quoted without proof.
The sufficiency direction of it is rather simple.
That is, if \(g=\theta(h)\) and the coefficients
of \(\theta(x)\in F[x]\) are symmetric in \seqxn, then
\(G(h)\subset G(g)\).

%%%%%%%%%%%%%%%%%%%%%%%%%%%%%%%%%%%%%%%%%%%%%%%%%%%%%%%%%%%%%%%%%%%%%%%%
\subsection{Theorem of Primitive Equality --- Explicit Construction}

The proof of Corollary~H.8 uses \(h(\seqxn)\in F[\seqxn]\)
as any \(n!\)-valued function. Later the primitive element
\(\eta = h(\seqalphn)\) is defined.
To have a fully explicit construction of a primitive element,
it would be nice to choose such an $h$, for example
\begin{equation*}
h(\seqxn) = \sum_{j=1}^n jx_j .
\end{equation*}

%%%%%%%%%%%%%%%%%%%%%%%%%%%%%%%%%%%%%%%%%%%%%%%%%%%%%%%%%%%%%%%%%%%%%%%%
\subsection{Repetition of Lagrange Resolvent}

\index{Lagrange}
Pages 147--149 summerize historic results.
There, the detailed explanation of Lagrange resolvent is
basically a repetition of the its formal presentation
given on pages 144-145. A shorter reference to these preceding
pages would do.

%%%%%%%%%%%%%%%%%%%%%%%%%%%%%%%%%%%%%%%%%%%%%%%%%%%%%%%%%%%%%%%%%%%%%%%%
\subsection{Last Theorem -- Beginning of Galois's 1831 Paper}

The exposition of Theorem~H.9 and the short discussion ending the book
are just beautiful.


%%%%%%%%%%%%%%%%%%%%%%%%%%%%%%%%%%%%%%%%%%%%%%%%%%%%%%%%%%%%%%%%%%%%%%%%
%%%%%%%%%%%%%%%%%%%%%%%%%%%%%%%%%%%%%%%%%%%%%%%%%%%%%%%%%%%%%%%%%%%%%%%%
%%%%%%%%%%%%%%%%%%%%%%%%%%%%%%%%%%%%%%%%%%%%%%%%%%%%%%%%%%%%%%%%%%%%%%%%
% \bibliographystyle{plain}
\bibliographystyle{alpha}
\bibliography{rotgal}

%%%%%%%%%%%%%%%%%%%%%%%%%%%%%%%%%%%%%%%%%%%%%%%%%%%%%%%%%%%%%%%%%%%%%%%%
%%%%%%%%%%%%%%%%%%%%%%%%%%%%%%%%%%%%%%%%%%%%%%%%%%%%%%%%%%%%%%%%%%%%%%%%
%%%%%%%%%%%%%%%%%%%%%%%%%%%%%%%%%%%%%%%%%%%%%%%%%%%%%%%%%%%%%%%%%%%%%%%%
\printindex


%%%%%%%%%%%%%%%%%%%%%%%%%%%%%%%%%%%%%%%%%%%%%%%%%%%%%%%%%%%%%%%%%%%%%%%%
%%%%%%%%%%%%%%%%%%%%%%%%%%%%%%%%%%%%%%%%%%%%%%%%%%%%%%%%%%%%%%%%%%%%%%%%
%%%%%%%%%%%%%%%%%%%%%%%%%%%%%%%%%%%%%%%%%%%%%%%%%%%%%%%%%%%%%%%%%%%%%%%%
\end{document}

\iffalse
%%%%%
\item
\begin{excopy}
\end{excopy}
\fi
