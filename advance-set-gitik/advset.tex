\documentclass[11pt,pdftex,twoside,a4paper]{article}
\usepackage{amsthm}
\usepackage{array}
\usepackage{enumitem}
\usepackage{graphicx}
\usepackage{relsize}

\usepackage{amsmath}
\usepackage{amssymb}
% \usepackage{eucal}
\usepackage{mathrsfs}

% \usepackage{fullpage}

\usepackage{geometry}
\geometry{a4paper, left=2cm, right=2cm, top=2cm, bottom=2cm, includeheadfoot}

\setlength{\parindent}{0pt}
\setlength{\parskip}{6pt}


% are we in pdftex ????
\ifx\pdfoutput\undefined % We're not running pdftex
\else
\RequirePackage[colorlinks,hyperindex,plainpages=false]{hyperref}
\def\pdfBorderAttrs{/Border [0 0 0] } % No border arround Links
\fi

% \usepackage{fancyheadings}
\usepackage{fancyhdr}
\usepackage{pifont}

\pagestyle{fancy}
% \addtolength{\headwidth}{\marginparsep}
% \addtolength{\headwidth}{\marginparwidth}
%  \addtolength{\textheight}{2pt}

\newcommand{\ineqjton}{\overset{1\leq i,j \leq n}{i \neq j}}
\newcommand{\srightmark}{\rightmark}
\newcommand{\sfbfpg}{\sffamily\bfseries{\thepage}}
  \newcommand{\symenvelop}{%
     {\nullfont\ }\relax\lower.2ex\hbox{\large\Pisymbol{pzd}{41}}}
% \renewcommand{\chaptermark}[1]{\markboth{\thechapter.\ #1}}

\iffalse
% \lhead[\fancyplain{}{{\sfbfpg}}]{\fancyplain{}\bfseries\srightmark}
\lhead[\fancyplain{}{{\sfbfpg}}]{\fancyplain{}\sl\srightmark}
% \rhead[\fancyplain{}\bfseries\leftmark]{\fancyplain{}{{\sfbfpg}}}
\rhead[\fancyplain{}\sl\leftmark]{\fancyplain{}{{\sfbfpg}}}
\lfoot{\today}
\cfoot{Yotam Medini \copyright}
  \newcommand{\symenvelop}{%
     {\nullfont a}\relax\lower.2ex\hbox{\large\Pisymbol{pzd}{41}}}
\rfoot{\symenvelop\ \texttt{yotam.medini@gmail.com}}

\renewcommand{\headrulewidth}{0.4pt}
\renewcommand{\footrulewidth}{0.4pt}
\fi

\setlength{\headheight}{16pt}
\fancyplain{plain}{%
 \fancyhf{}
 \fancyhead[LE,RO]{\fancyplain{}{{\sfbfpg}}}
 \fancyhead[RE,LO]{\sl\leftmark}
 \fancyfoot[L]{\today}
 \fancyfoot[C]{Yotam Medini \copyright}
 \fancyfoot[R]{\symenvelop\ \texttt{yotam.medini@gmail.com}}
 \renewcommand{\headrulewidth}{0.4pt}
 \renewcommand{\footrulewidth}{0.4pt}
}

% \usepackage{amstex}
% \usepackage{amsmath}
% \usepackage{amssymb}
\usepackage{amsthm}
\usepackage{bm}
\usepackage{makeidx}
\makeindex % enable

% 'Inspired' by:
%% This is file `uwamaths.sty',
%%%     author   = "Greg Gamble",
%%%     email     = "gregg@csee.uq.edu.au (Internet)",

\makeatletter
\def\DOTSB{\relax}
\def\dotcup{\DOTSB\mathop{\overset{\textstyle.}\cup}}
 \def\@avr#1{\vrule height #1ex width 0pt}
 \def\@dotbigcupD{\smash\bigcup\@avr{2.1}}
 \def\@dotbigcupT{\smash\bigcup\@avr{1.5}}
 \def\dotbigcupD{\DOTSB\mathop{\overset{\textstyle.}\@dotbigcupD%
                               \vphantom{\bigcup}}}

\def\dotbigcupT{\DOTSB\smash{\mathop{\overset{\textstyle.}\@dotbigcupT%
                              \vphantom{\bigcup}}}%
                       \vphantom{\bigcup}\@avr{2.0}}
\def\dotbigcup{\mathop{\mathchoice{\dotbigcupD}{\dotbigcupT}
                                  {\dotbigcupT}{\dotbigcupT}}}
\let\disjunion\dotcup
\let\Disjunion\dotbigcup
\makeatother


\newcommand{\half}{\ensuremath{\frac{1}{2}}}



\newcommand{\C}{\ensuremath{\mathbb{C}}} % The Complex set
\newcommand{\aded}{\ensuremath{\textrm{a.e.}}} % almost everyehere
\newcommand{\chhi}{\raise2pt\hbox{\ensuremath\chi}}           %raise the chi
\newcommand{\calA}{\ensuremath{\mathcal{A}}}
\newcommand{\calB}{\ensuremath{\mathcal{B}}}
\newcommand{\calE}{\ensuremath{\mathcal{E}}}
\newcommand{\calF}{\ensuremath{\mathcal{F}}}
\newcommand{\calG}{\ensuremath{\mathcal{G}}}
\newcommand{\calM}{\ensuremath{\mathcal{M}}}
\newcommand{\calR}{\ensuremath{\mathcal{R}}}
\newcommand{\eqdef}{\ensuremath{\stackrel{\mbox{\upshape\tiny def}}{=}}}
\newcommand{\frakB}{\ensuremath{\mathfrak{B}}}
\newcommand{\frakC}{\ensuremath{\mathfrak{C}}}
\newcommand{\frakF}{\ensuremath{\mathfrak{F}}}
\newcommand{\frakG}{\ensuremath{\mathfrak{G}}}
\newcommand{\frakI}{\ensuremath{\mathfrak{I}}}
\newcommand{\frakM}{\ensuremath{\mathfrak{M}}}
\newcommand{\scrA}{\ensuremath{\mathscr{A}}}
\newcommand{\scrB}{\ensuremath{\mathscr{B}}}
\newcommand{\scrD}{\ensuremath{\mathscr{D}}}
\newcommand{\scrF}{\ensuremath{\mathscr{F}}}
\newcommand{\scrN}{\ensuremath{\mathscr{N}}}
\newcommand{\scrP}{\ensuremath{\mathscr{P}}}
\newcommand{\scrQ}{\ensuremath{\mathscr{Q}}}
\newcommand{\scrR}{\ensuremath{\mathscr{R}}}
\newcommand{\scrT}{\ensuremath{\mathscr{T}}}
\newcommand{\Lp}[1]{\ensuremath{\mathbf{L}^{#1}}} % Lp space
\newcommand{\N}{\ensuremath{\mathbb{N}}} % The Natural Set
\newcommand{\bbP}{\ensuremath{\mathbb{P}}} % Some partially ordered set
\newcommand{\Q}{\ensuremath{\mathbb{Q}}} % The Rational set
\newcommand{\R}{\ensuremath{\mathbb{R}}} % The Real Set
\newcommand{\T}{\ensuremath{\mathbb{T}}} % The Thorus [-pi,\pi)
\newcommand{\Z}{\ensuremath{\mathbb{Z}}} % The Integer Set
\newcommand{\intR}{\int_{-\infty}^{\infty}} % Integral over the reals
\newcommand{\posthat}[1]{#1{\,\hat{}\,}}

% sequences
\newcommand{\seq}[2]{\ensuremath{#1_1,\ldots,#1_{#2}}}
\newcommand{\seqn}[1]{\seq{#1}{n}}
\newcommand{\seqan}{\seq{a}{n}}
\newcommand{\seqxn}{\seq{x}{n}}
\newcommand{\seqalphn}{\seq{\alpha}{n}}

\newcommand{\mset}[1]{\ensuremath{\{#1\}}}


%%%%%%%%%%%%
%% math op's
\newcommand{\Alt}{\mathop{\rm Alt}\nolimits}
\newcommand{\Ang}{\mathop{\rm Ang}\nolimits}
\newcommand{\Arg}{\mathop{\rm Arg}\nolimits}
\newcommand{\co}{\mathop{\rm co}\nolimits}
\newcommand{\conv}{\mathop{\rm conv}\nolimits}
\newcommand{\diam}{\mathop{\rm diam}\nolimits}
\newcommand{\dom}{\mathop{\rm dom}\nolimits}
% \newcommand{\dim}{\mathop{\rm dim}\nolimits}
% \newcommand{\esssup}{\mathop{\rm ess\ sup}\nolimits}
\DeclareMathOperator*{\esssup}{ess\,sup}
\newcommand{\ext}{\mathop{\rm ext}\nolimits}
\newcommand{\Id}{\mathop{\rm Id}\nolimits}
\newcommand{\Image}{\mathop{\rm Im}\nolimits}
\newcommand{\Ind}{\mathop{\rm Ind}\nolimits}
\newcommand{\Lip}{\mathop{\rm Lip}\nolimits}
\newcommand{\lip}{\mathop{\rm lip}\nolimits}
\newcommand{\percB}{
  \mathbin{\ooalign{$\hidewidth\%\hidewidth$\cr$\phantom{+}$}}}
\newcommand{\bres}[2]{\ensuremath{#1 \percB #2}}

\newcommand{\Ker}{\mathop{\rm Ker}\nolimits}
\newcommand{\rank}{\mathop{\rm rank}\nolimits}
\newcommand{\rng}{\mathop{\rm rng}\nolimits}
\newcommand{\Res}{\mathop{\rm Res}\nolimits}
\newcommand{\supp}{\mathop{\rm supp}\nolimits}
\newcommand{\vol}{\mathop{\rm vol}\nolimits}
\newcommand{\vspan}{\mathop{\rm span}\nolimits}

% I wish this was more standardized
\renewcommand{\Re}{\mathop{\bf Re}\nolimits}
\renewcommand{\Im}{\mathop{\bf Im}\nolimits}

\newcommand{\inter}[1]{\ensuremath{#1^{\circ}}}  % interior
\newcommand{\closure}[1]{\ensuremath{\overline{#1}}} % closure
\newcommand{\boundary}[1]{\ensuremath{\partial #1}} % closure


\newcommand{\ich}[1]{(\textit{#1})}
\newcommand{\itemch}[1]{\item[\ich{#1}]}
\newcommand{\itemdim}{\item[\(\diamond\)]}

% Special names
\newcommand{\Cech}{\u{C}ech}

\author{Yotam Medini}


%%%%%%%%%%%
%% Theorems
%%
\makeatletter
\@ifclassloaded{book}{
 \newtheorem{thm}{Theorem}[chapter]
 \newtheorem{cor}[thm]{Corollary}
 \newtheorem{lem}[thm]{Lemma}
 \newtheorem{llem}[thm]{Local Lemma}
 \newtheorem{lthm}[thm]{Local Theorem}
 % \newtheorem{quotecor}{Corollary}
 % \newtheorem{quotelem}{Lemma}[section]
 \newtheorem{quotethm}{Theorem}[chapter]
}{}
\makeatother
\newtheorem{Def}{Definition}

\newtheorem{manualtheoreminner}{Theorem}
\newenvironment{manualtheorem}[1]{%
  \renewcommand\themanualtheoreminner{#1}%
  \manualtheoreminner
}{\endmanualtheoreminner}

\newtheorem{manuallemmainner}{Lemma}
\newenvironment{manuallemma}[1]{%
  \renewcommand\themanuallemmainner{#1}%
  \manuallemmainner
}{\endmanuallemmainner}

\newcommand{\loclemma}{Lemma}


% \newcommand{\proofend}{\(\bullet\)}
% \newcommand{\proofend}{\hfill\(\blacksquare\)}
\newcommand{\proofend}{\hfill\(\Box\)}
\newenvironment{thmproof}
{\textbf{Proof.}}
{\proofend}

\newcommand{\chapterTypeout}[1]{\typeout{#1} \chapter{#1}}
\newcommand{\sectionTypeout}[1]{\typeout{#1} \section{#1}}

% abbreviations, ensuremath
\newcommand{\fx}{\ensuremath{f(x)}}
\newcommand{\gx}{\ensuremath{g(x)}}
\newcommand{\lrangle}[1]{\ensuremath{\left\langle #1 \right\rangle}}
\newcommand{\lrbangle}[1]{\ensuremath{\left\langle #1 \right\rangle}}
\newcommand{\M}{\ensuremath{\mathfrak{M}}}
\newcommand{\mldots}{\ensuremath{\ldots}}
\newcommand{\salgebra}{\(\sigma\)-algebra}
\newcommand{\swedge}{\;\wedge\;}
\newcommand{\wlogy}{without loss of generality}
\newcommand{\Wlogy}{Without loss of generality}
\newcommand{\twopii}{\ensuremath{2\pi i}}
\newcommand{\dtwopii}{\ensuremath{\frac{1}{\twopii}}}

% https://tex.stackexchange.com/
% questions/22252/how-to-typeset-function-restrictions
\newcommand\restr[2]{\ensuremath{% we make the whole thing an ordinary symbol
  \left.\kern-\nulldelimiterspace % automatically resize the bar with \right
  #1 % the function
  \vphantom{\big|} % pretend it's a little taller at normal size
  \right|_{#2} % this is the delimiter
  }}

\newenvironment{excopyOLD}
{\item\begin{minipage}[t]{.8\textwidth}\footnotesize}
{\smallskip\hrule\end{minipage}}

\newenvironment{excopy}
{\item % \relax
 \begin{list}{}{
 \setlength{\topsep}{0pt}
 \setlength{\partopsep}{0pt}
 \setlength{\itemsep}{0pt}
 \setlength{\parsep}{0pt}
 \setlength{\leftmargin}{0pt}
 \setlength{\rightmargin}{20pt}
 \setlength{\listparindent}{0pt}
 \setlength{\itemindent}{0pt}
 % \setlength{\labelsep}{0pt}
 \setlength{\labelwidth}{0pt}
 \footnotesize
 }
 \item
}
{\par
 % {\nullfont 0}
 \hrulefill
 \end{list}
}


\newcommand{\B}[1]{\textbf{#1}}
\newcommand{\Cohen}{\mathop{\rm Cohen}\nolimits}
\newcommand{\MA}{\ensuremath{\mathop{\rm MA}\nolimits}}
\newcommand{\ccc}{c.c.c.}
\newcommand{\crotin}{\rotatebox[origin=c]{-90}{\(\in\)}}
\newcommand{\finitepartial}{\textnormal{finite partial}}
\newcommand{\Lev}{\ensuremath{\mathop{\rm Lev}\nolimits}}
\newcommand{\otp}{\ensuremath{\mathop{\rm otpo}\nolimits}}
\newcommand{\rotin}{\rotatebox[origin=c]{90}{\(\in\)}}
\newcommand{\Suc}{\ensuremath{\mathop{\rm Suc}\nolimits}}
\newcommand{\tht}{\ensuremath{\mathop{\rm ht}\nolimits}}

%% \newtheorem{manualclaiminner}{Claim}
%% \newenvironment{manualclaim}[1]{%
%%   \renewcommand\themanualclaiminner{#1}%
%%   \manualclaiminner
%% }{\endmanualclaiminner}
%% 
%% \newtheorem{claim}{Claim}

\newtheorem{thm}{Theorem}[section]
\newtheorem{lemma}[thm]{Lemma}
\newtheorem{claim}[thm]{Claim}
\newtheorem{corollary}[thm]{Corollary}
\theoremstyle{definition} 
\newtheorem{ldef}[thm]{Definition}

\title{Advanced Set Theory
          \\
       Lecture Notes --- Professor Mordechai Gitik}

\author{Yotam Medini}

%%%%%%%%%%%%%%%%%%%%%%%%%%%%%%%%%%%%%%%%%%%%%%%%%%%%%%%%%%%%%%%%%%%%%%%%
%%%%%%%%%%%%%%%%%%%%%%%%%%%%%%%%%%%%%%%%%%%%%%%%%%%%%%%%%%%%%%%%%%%%%%%%
%%%%%%%%%%%%%%%%%%%%%%%%%%%%%%%%%%%%%%%%%%%%%%%%%%%%%%%%%%%%%%%%%%%%%%%%
\begin{document}
\maketitle
\newpage
\tableofcontents
\newpage

%%%%%%%%%%%%%%%%%%%%%%%%%%%%%%%%%%%%%%%%%%%%%%%%%%%%%%%%%%%%%%%%%%%%%%%%
%%%%%%%%%%%%%%%%%%%%%%%%%%%%%%%%%%%%%%%%%%%%%%%%%%%%%%%%%%%%%%%%%%%%%%%%
\section{2025/March/23}

In this course we will deal with infinite combinatorics.

Text book for the course:
  \B{Kunen} / \textit{Set SET THEORY, 
    An Introduction to Independence Proofs}

We will use strengthening of the continuum-hypothesis
and strengthening of negation of the continuum-hypothesis.

%%%%%%%%%%%%%%%%%%%%%%%%%%%%%%%%%%%%%%%%%%%%%%%%%%%%%%%%%%%%%%%%%%%%%%%%
\subsection{Almost Disjoint Sets}

\begin{ldef}
Let \(\kappa\) be infinite cardinal, \(\scrA \subseteq P(\kappa)\)
is a family of of \emph{almost disjoint sets} iff 
\begin{enumerate}
\item For any \(A\in \scrA\), \(|A|=\kappa\).
\item For any \(A,B \in \scrA\) if \(A\neq B\) then \(|A \cap B|<\kappa\).
\end{enumerate}
\end{ldef}

%%%%%%%%%%%%%%%%%%%%%%%%%%%%%%%%%%%%%%%%%%%%%%%%%%%%%%%%%%%%%%%%%%%%%%%%
\subsection{Maximal Families of Almost Disjoint Sets}

\begin{claim}
Let \(\kappa \geq \aleph_0\) be a regular cardinal.
Let \(\calA \subset P(\kappa)\) be a family of almost disjoint sets.
Then:
\begin{itemize}
\itemch{a} If \(|A|=\kappa\) then \calA\ is not maximal,
that is there exists \(d\leq \kappa\) and for any \(a\in \calA\),
\(d\cap a| < \kappa\).
\itemch{b} There exists a maximal family \(\calB \supset \calA\)
of almost disjoint sets, \(|B| \geq \kappa^+\).
\end{itemize}
\end{claim}
\begin{proof}
\ich{a} \(\implies\) \ich{b} by the Zorn lemma.

Prooving \ich{a}. Let \(\calA = \{A_\alpha | \alpha < \kappa\}\).
We will ``convert'' it to be ``disjoint''. We will define a family
\(\{A_\alpha'|\alpha < \kappa\}\) of disjoint sets.
\begin{align*}
 A_0' &= A_0 \\
 A_1' &= A_1 \setminus A_0 \\
 A_\alpha' &= A_\alpha \setminus 
   \mathsmaller{\bigcup_{\beta<\alpha}} (A_\alpha \setminus A_\beta).
\end{align*}
We get completely disjoint sets.

Define \(B=\{b_\alpha| \alpha<\kappa\}\) where \(b_\alpha \in A_\alpha'\).
\\
\vspace{-12pt}
\begin{center}
\begin{tabular}{llll}
\(A_0'\) & \(A_1'\) & \(\cdots\) & \(A_\alpha'\) \\
\rotatebox[origin=c]{90}{\(\in\)} &
\rotatebox[origin=c]{90}{\(\in\)} &
\(\cdots\) &
\rotatebox[origin=c]{90}{\(\in\)} \\
\(b_0\) & \(b_1\) & \(\cdots\) & \(b_\alpha\)
\end{tabular}
\end{center}
Therefore \(|A_\alpha \cap B| < \kappa\).
\end{proof}

How can we get maximal cardinal?\\
In \(\omega\) such family would have \(2^{\aleph_0}\) cardinality.
We can generalize.

\begin{claim}
Let \(\kappa \geq \aleph_0\) infinite cardinal.

\B{Assumption}: for each \(\alpha < \kappa\), \(2^\alpha \leq \kappa\).
\begin{equation*}
2^{>\kappa} = \mathsmaller{\bigcup_{\alpha < \kappa}}2^\alpha.
\end{equation*}
\(\implies 2^{<\kappa} = \kappa\).
This assumption is kind of the continuum-hypothesis.

There exists \(\calA \subset P(\kappa)\) a family of almost disjoint sets
with maximal cardinality \(2^\kappa\).
\end{claim}
Here we do not assume regularity of \(\kappa\)
\begin{proof}
(With a nice trick.)
Let
\begin{equation*}
I = \{X \subseteq \kappa | \exists \alpha < \kappa, \alpha \supseteq X\}.
\end{equation*}
Therefore \(|I| = \kappa\).
Now argument of \(\leq\) and \(\geq\).
For each \(y \supseteq \kappa\) let
\begin{equation*}
A_y = \{y \cap \alpha | \alpha < \kappa\} \subset I.
\end{equation*}
Assume \(y_1,y_2 \subset \kappa\), \(y_1 \neq y_2\).
So 
\begin{equation*}
 |A_{y_1} \cap A_{y_2}| < \kappa.
\end{equation*}
\emph{Note:} A difference in one element implies a cardinality difference.
Assume \(\delta \in y_1 \setminus y_2\). Then for each \(\alpha\)
such that \(\delta < \alpha < \kappa\) then
\begin{equation*}
 \delta \in y_1\cap \alpha \neq y_2 \cap \alpha
\end{equation*}
\end{proof}
Thus we got a maximal family of disjoint sets.

%%%%%%%%%%%%%%%%%%%%%%%%%%%%%%%%%%%%%%%%%%%%%%%%%%%%%%%%%%%%%%%%%%%%%%%%
\subsection{Topological Spaces}

\begin{ldef}
A topological space \(\lrangle{X, \tau}\) is \B{separable}
iff there exists \(D\subset X\) countable and dense.
\end{ldef}

\begin{ldef}
A topological space \(\lrangle{X, \tau}\) is called \B{\ccc} 
(Countable Chain Condition) iff there are no more than \(\aleph_0\)
disjoint open sets.
\end{ldef}

Separability implies \ccc. But the opposite direction depends
on conjectures.

Using Tychonoff product:
\begin{itemize}
\item Product of \ccc\ spaces is \ccc.
\item Product of separable spces is not necessarily separable.
\end{itemize}

\begin{thm}
Let \lrangle{X_i,\tau_i}, \(i\in I\) be \ccc\ topological spaces.
Assume that any finite product is \ccc\ then
the Tychonoff product \lrangle{\Pi X_i, =\Pi \tau_i} is \ccc.
\end{thm}
\begin{proof}
We will Specify the  topology of the product
\begin{equation*}
X = \Pi_{i\in I} = \{f | 
  f: I \to\mathsmaller{\bigcup_{i\in I}} X_i
  \quad\land\quad \forall i\in I, f(i)\in X_i\}.
\end{equation*}
The basis neighborhoods are defined, for \(\{i_1,\ldots,i_n\}\)
for any \(i \in \{i_1,\ldots,i_n\}\)
we choose \(U_i \in \tau_i\) and get
\begin{equation*}
 U = U_{i_1} \times U_{i_2} \times \cdots \times U_{i_n}.
\end{equation*}
\begin{equation*}
 V_U = \{f \in \pi_{i\in I}| \forall i \in \{i_1,\ldots,i_n\}, f(i)\in U_i\}.
\end{equation*}
By negation, assume that the product is not \ccc.
Then at least \(\aleph_1\) disjoint non-empty sets.
Then at least \(\aleph_1\) basisnon-empty open sets:
\begin{equation*}
 \{W_\beta | \beta < \aleph_1\}.
\end{equation*}
For each \(\beta\) there exists finite set
\(\{i_1^\beta, i_2^\beta, \ldots, i_{n_\beta}^\beta\}\)
and open sets \(\{U_{i_\beta}^\beta\}\)
with ??  \(\{a_\beta | \beta < \aleph_1\}\)
\(\aleph_1\) finite subsets of $I$.
Then we have \(J \subseteq \aleph_1\) with \(|J|=\aleph_1\)
and a set $a$ such that for any \(i,j\in J\), 
if \(i\neq j\) then \(a_i \cap a_j = a\).
\end{proof}

% Delta System ??
Let \(a = \{x_1,\ldots,x_k\}\). Our assumption is that
\begin{equation*}
X_{x_1} \times X_{x_2} \times \cdots X_{x_k}
\end{equation*}
is \ccc.
\begin{equation*}
U_{x_1}^\beta \times U_{x_2}^\beta \times \cdots U_{x_k}^\beta \qquad (\beta \in J)
\end{equation*}
Form \ccc\ there exist \(\beta,\gamma \in J\), \(\beta\neq\gamma\)
such that \(W = W^\beta \cap W^\gamma \neq \emptyset\)
open in the finite product.

Define \(f\in \Pi_{i\in I} X_i\)
\begin{equation*}
\begin{array}{lll}
x_1,\ldots,x_n & 
  e_1^\beta,\ldots,e_{k_\beta}^\beta & e_1^\gamma,\ldots,e_{k_\gamma}^\gamma
\\
a\;\textnormal{kernel} & 
  a_\beta\; \textnormal{kernel} & a_\gamma\;\textnormal{kernel} 
\end{array}
\end{equation*}

The ``problem'' is with the common coordinates, otherwise
it does not matter what is chosen.
\begin{equation*}
 \begin{array}{llll}
 U_{x_1}^\beta \cap U_{x_1}^\gamma, &
 U_{x_2}^\beta \cap U_{x_2}^\gamma, &
 \cdots &
 U_{x_k}^\beta \cap U_{x_k}^\gamma, \\
 x_1 & x_2 & \ldots & x_k
 \end{array}
\end{equation*}

\begin{equation*}
\begin{array}{ll}
a_\beta \setminus\{x_1,\ldots,x_k\}  & a_\gamma \setminus\{x_1,\ldots,x_k\} \\
\rotatebox[origin=c]{90}{\(=\)} & \rotatebox[origin=c]{90}{\(=\)} \\
\{i_1^\beta, \ldots, i_{k_\beta}^\beta\} & \{i_1^\gamma, \ldots, i_{k_\gamma}^\gamma \}
\end{array}
\end{equation*}
and we have a finite number of coordinates.

\(W_\beta \cap W_\gamma \supseteq\)  of basis open set, generated by
\begin{equation*}
U_{x_1}^\beta \cap \cdots \cap U_{x_k}^\beta 
  \quad \cap \quad 
  U_{i_{l_\beta}}^\beta \cap \cdots \cap  U_{i_{n_\beta}}^\beta
  \quad \cap \quad 
  U_{i_{l_\gamma}}^\gamma \cap \cdots \cap  U_{i_{n_\gamma}}^\gamma
\end{equation*}

%%%%%%%%%%%%%%%%%%%%%%%%%%%%%%%%%%%%%%%%%%%%%%%%%%%%%%%%%%%%%%%%%%%%%%%%
\subsection{Guiding Questions}

Assume \(\aleph_1 < 2^{\aleph_0}\)
\begin{itemize}
\itemch{a} Is for any \(\kappa \in 2^{\aleph_0}\), \(2^\kappa = 2^{\aleph_0}\).
\itemch{b} Is there a family \(\scrA \subset P(\omega)\)
of almost disjoint sets which is maximal such that \(|A| < 2^{\aleph_0}\).
\itemch{c} Let \(\kappa < 2^{\aleph_0}\). Is a union of \(\kappa\)
sets of Lebesgue's $0$-measure is a set of Lebesgue's $0$-measure ?
\itemch{d} Let \(\kappa < 2^{\aleph_0}\). 
Is a union of \(\kappa\) sets of first category is a set of first category.
\\
\emph{Note}: A set $A$ is meager if its closure has empty interior.
% ? what was said about empty segment ?
$B$ is of first category iff $B$ is a countable union of meager sets.
\end{itemize}

%%%%%%%%%%%%%%%%%%%%%%%%%%%%%%%%%%%%%%%%%%%%%%%%%%%%%%%%%%%%%%%%%%%%%%%%
\subsection{Developing Tools}

We will develop tools to address the questions above.
(Forcing term?)

\begin{ldef}
Let \lrangle{P,\leq} be a \B{partially ordered set},
iff for any \(x,y,z\in P\) the following hold:
\begin{enumerate}
\item \(x\leq x\).
\item \(x\leq y \; \land\; y\leq z \implies x \leq z\).
\item \(x \leq y\; \land\; y \leq z \implies x = z\).
\end{enumerate}
\end{ldef}
Frequently, we can do without the 3rd requirement, by equivalence relation.

\begin{ldef}
Let \lrangle{P,\leq} be a {partially ordered set}..
\(p,q \in P\) are \B{compatible} \(p \| q\) if there exists \(r\in P\)
such that \(p,q < r\).\\
\emph{Note}: \textnormal{In Kunen's book it is the opposite.}\\
Otherwise \(p,q\) are called \B{non-compatible}.
\end{ldef}

\begin{ldef}
\(A\subseteq P\) is called \B{antichain} iff for any
 distinct \(p,q\in A\) are non-compatible.
$A$ is a \B{maximal antichain} iff thete is no antichain $B$
such that \(A \subsetneq B\).
\end{ldef}

\begin{ldef}
\lrangle{\mathbb{P},\leq} \B{has the \ccc} iff 
for any antichain \(A \subseteq \mathbb{P}\), 
\(|A|\leq \aleph_0\).
\end{ldef}

\begin{ldef}
Let \lrangle{\mathbb{P},\leq} be partially ordered.
\(D\subseteq P\) is called \B{dense} iff fir each \(p\in P\)
the exists \(q\in D\) such that \(q\leq p\).
\end{ldef}

%%%%%%%%%%%%%%%%%%%%%%%%%%%%%%%%%%%%%%%%%%%%%%%%%%%%%%%%%%%%%%%%%%%%%%%%
\subsection{Examples}

\begin{enumerate}
\item \lrangle{L,\leq} linear ordered (any two are comparable)
is \ccc\ since there are no non-compatible.
Here, dense means not bounded.

\item \lrangle{P(\omega)\setminus\{0\},\leq}. % \supseteq ?
For \(a,b\subset \omega\), \(a\|b\) iff \(a\cap b \neq \emptyset\).

\item \lrangle{X,\tau} topological space.
Let \(P = \tau \setminus \{\emptyset\}\) and \(p,q \in P\).\\
\(p \geq q\ \;\iff\; p \subseteq q\).
\(p,q\) are comparable iff \(p\cap q\neq \emptyset\).\\
$P$ has \ccc\ iff \lrangle{X,\tau} has \ccc.

\item \(C = \{f| f: \omega \to 2 \mathsmaller=\{0,1\}\)
(finite partial).\\
$f$ \emph{continues} $g$ iff  \(f \geq g\) (as graph).\\
? If $C$ is , countable \(|C| = \aleph_1\) the \ccc.
\end{enumerate}

\emph{Note}: Cohen forcing !?!

\begin{ldef}
Let \scrF\ be a collection of dense sets of $P$.
\(G \subseteq P\) is called \scrF-\B{generic} iff 
\begin{enumerate}
\item \(p\in G \land q\leq p \implies q \in G\).
\item \(p,q \in G \implies \exists r\in G, p,q\leq r\) Like an ideal.
\item \(\forall D \in \scrF\, G\cap D \neq \emptyset\).
\end{enumerate}
\end{ldef}

%%%%%%%%%%%%%%%%%%%%%%%%%%%%%%%%%%%%%%%%%%%%%%%%%%%%%%%%%%%%%%%%%%%%%%%%
\subsection{Martin Axiom}

\begin{ldef}
Let \(\kappa \geq \aleph_0\).
\B{Martin axiom} \(\MA(\kappa)\) means:
For any partially ordered \lrangle{P,\leq} that has \ccc\
and for any collection \scrF\ of dense subsets, \(|\scrF| \leq \kappa\)
then there exists an \scrF-generic \(G\subset P\).
\end{ldef}

\begin{claim}
\(\MA(\aleph_0)\) holds.
\end{claim}
\begin{proof}
Let \lrangle{P,\leq} be partially ordered (\ccc\ not needed).
Let \scrF\ collection of dense subsets of $P$
and \(|\scrF| = \aleph_0\). Let \(\scrF = \{D_n| n < \omega\}\).
Let arbitrary \(p\in P\).  \(D_0\) is dense, so 
\begin{align*}
p \leq & p0 \in D_0 \\
p_0 \leq & p1 \in D_1 \\
\end{align*}
By induction, assume we picked \(p_n \in D_n\).
We pick \(p_{n+1}\) such that \(p_n \leq p_{n+1} \in D_{n+1}\).
We have 
\begin{equation*}
\begin{array}{lllll}
p \leq & p_0 \leq & p_1 \leq &\cdots \leq & p_n \\
       &\crotin   &\crotin   &            & \crotin \\
       & D_0      & D_1      &\cdots      & D_n
\end{array}
\end{equation*}
Now \(G = \{q \in P |\exists n, q \leq p_n\}\) is \scrF-generic.
This proves \(\MA(\aleph_0)\).
\end{proof}

\begin{claim}
\(\MA(e^{\aleph_0})\) fails.
\end{claim}
\begin{proof}
Let \(C = \{f| f: \omega \to 2 \;\textnormal{finite parial}\}\). ??
\\
\(f \geq g \iff f \supseteq g\).
\\
For any \(n\in \omega\) let
\begin{equation*}
D_n = \{f\in C| n \in\dom(f)\}
\end{equation*}
is dense.
Let \(G \subset C\)  be  \(\{D_n|n < \omega\}\)-generic.

Let \(g = \cup GL \omega \to 2\) partial function.

\begin{equation*}
 G \cap D_n \neq \emptyset \;\implies\; n \in \dom(g)
\end{equation*}
Hence  \(g:\omega \to 2\).

For any \(h:w\to 2\) let
\begin{equation*}
D^h - \{f\in C| \exists n\in \dom(f) f(n)\neq h(n)\}.
\end{equation*}
\B{Claim:} \(D^h\) is dense.
\(f\in C\) is finite partial, then \(\exists n \notin \dom(f_0)\)
\begin{equation*}
f_1 = f_0 \cup \{(n,1 - h(n)\} \qquad \textnormal{(differs from $h$)}.
\end{equation*}
Let
\begin{equation*}
\scrF = \{D_n | n < \omega\} \cup \{D^h | h : \omega \to 2\}.
\end{equation*}
$G$ is generic for \scrF.
Let
\begin{equation*}
f = \cup G_i \omega \to 2.
\end{equation*}
\begin{equation*}
 G \cap D^g \neq \emptyset
\end{equation*}
contradiction.
\end{proof}

\(\MA\)i ?
\begin{equation*}
\forall \kappa < 2^{\aleph_0}, \MA(\kappa).
\end{equation*}

%%%%%%%%%%%%%%%%%%%%%%%%%%%%%%%%%%%%%%%%%%%%%%%%%%%%%%%%%%%%%%%%%%%%%%%%
%%%%%%%%%%%%%%%%%%%%%%%%%%%%%%%%%%%%%%%%%%%%%%%%%%%%%%%%%%%%%%%%%%%%%%%%
\section{2025/March/30}

% https://tau-ac-il.zoom.us/rec/share/CvWMQAysHRKdrJNRQeh8r8b7PLGekyuIH_0z8LgoaSD5GmOv_cI9zS4e3aUNeaMh.HulZNhXY_C65aaxt

Some repeat contents.

\begin{ldef}
Let \lrangle{P,\leq} partially ordered
{\small (interesting for ``far'' from linear)}.
\(p,q\in P\) are \B{compatible}
iff \(\exists r\in P, r\geq p,q\)
\end{ldef}
We denote \(p\| q\) if compatible, \(p\perp q\) otherwise.

\begin{ldef}
Let \lrangle{P,\leq} partially ordered.
\(D\subseteq P\) is \B{dense} iff for each \(q\in P\) there exists \(p\in D\)
such that \(q\leq p\).
\end{ldef}

\begin{ldef}
Let \lrangle{P,\leq} partially ordered.
\(A\subseteq P\) is called \B{antichain} iff
\hbox{\(\forall p,q\in A, p\neq q \implies p\perp q\)}.
\end{ldef}

For each dense set, we can find antichain that generates it.

Of $A$ is a maximal antichain then
\begin{equation*}
D_n = \{ p\in P | \exists p \in A, q \geq p\}.
\end{equation*}
Such set must be dense, Why ?\\
Let \(r\in P\). There exists \(a\in A\), \(a \leq r\) (because of maximality).
Then there exists \(t\in P\), \(a,r\leq t \in D_A\).

We got some something a little stronger (open set).

\begin{ldef}
Let \lrangle{P, \leq} has \ccc\ iff for all antichain in $P$ 
is infinite countable.
\end{ldef}

If \(|P|=\aleph_0\) then any antichain, and any subset is dense countable.
But what happens if \(|P| > \aleph_0\). Two possibilities:
\begin{itemize}
\item Small antichains.
\item Dense large antichains.
\end{itemize}

\B{Example:} Let \(\kappa\) %  \geq \aleph_0\) 
be a cardinal.
\begin{equation*}
\Cohen(\omega,\kappa) 
  = \{f | f: \kappa \times \omega \; \textnormal{finite partial}\}.
\end{equation*}
If \(\kappa \geq \aleph_0\) then \(|\Cohen(\omega,\kappa)| = \kappa\).

For each \(\alpha < \kappa\), \(n < \omega\)
\begin{equation*}
D_{n,\alpha} = \{f\in \Cohen(\omega,\kappa)| (\alpha,n)\in\dom(f)\}.
\end{equation*}
This set is dense. The cardinality stays \(\kappa\)
since we remove ``just few''.
\(|D_{n,\alpha}|=\kappa\).

We can argue simpler. $P$ is dense in $P$ (in itself).
We will show that it has \ccc.

\begin{claim}
\(\Cohen(\omega,\kappa)\) has \ccc.
\end{claim}
(Does not depend on \(\kappa\)).
\begin{proof}
Let
\begin{equation*}
\{f_i| i < \omega\} \subseteq \Cohen(\omega,\kappa).
\end{equation*}
We look at \(i\neq j\).
\(f_i \perp f_j\)  \(\implies\) there exists \((n,\alpha)\)
\begin{equation*}
\exists(n,\alpha) \in \dom(f_i)\cap\dom(f_j), f_i(\alpha,n) \neq f_j(\alpha,n).
\end{equation*}
We will find distinct \(i^*,j^* < \omega_1\) such that
\(f_{i^*}\), \(f_{j^*}\) are compatible
\begin{equation*}
\{\dom(f_i) | i < \omega_1.
\end{equation*}
From the \(\Delta\)-system theorem, there exists \(S\subseteq \omega_1\),
\(|S|=\aleph_1\), there exists $a$ such that for any \(i\neq j\in S\)
\begin{equation*}
a = \dom(f_1) \cap \dom(f_2).
\end{equation*}
\({}^a 2\) is finite, thus there exists \(S'\subseteq S\), \(|S^*| = \aleph_1\)
and \(e: a\to 2\) such that for each \(i\in S^*\), \(f_i|_a = e\)
\(i,j\in S^*\), \(f_i \cup f_j\) a function, \(f_i\| f_j\).
\end{proof}

\begin{ldef}
Let \lrangle{P,\leq} partially ordered.
Let \scrF\ be a family of dense sets of $P$. \(G\subseteq P\)
is called $F$-generic iff
\begin{enumerate}
\item \(p\in G,\,p \geq q\,\implies\, q\in G\).
\item \(p,q\in G,\; \exists r\in G\, (p,q \leq r)\).
\item \(\forall D\in F\; (G\cap D\neq \emptyset)\).
\end{enumerate}
\end{ldef}
In Kunen's book, requirements \ich{1}+\ich{2} define a \B{filter}.

\begin{ldef}
Let \(\kappa\) be a cardinal.
\(\MA(\kappa)\) says: For each partially ordered \lrangle{P,\leq} that 
has \ccc\ and for all collection \scrF\ of subsets of $P$,
if \(|\scrF| \leq \kappa\) then there exists \(G\subset P\) \scrF-generic.
\end{ldef}
We saw: \(\MA(\aleph_0)\), \(\lnot\MA(2^{\aleph_0})\) and 
\(\forall \kappa < 2^{\aleph_0}\, \MA(\kappa)\).

\begin{claim}
Let \(\kappa \geq \aleph_0\). If \(\MA(\kappa)\) then \(\kappa < 2^{\aleph_0}\).
\end{claim}
\begin{proof}
By negation.
\begin{equation*}
C = \{f|: f: \omega \to \;\textnormal{finite partial}\}
\end{equation*}
We look at $C$ with inclusion order.
\begin{equation*}
D_n = \{f\in C| n \in \dom(f)\}
\end{equation*}
For \(h:\omega\to 2\),
\begin{equation*}
D^h = \{f\in C | \exists \kappa < \omega\, (f(k)\neq h(k))\}.
\end{equation*}
% lecture break 

Let \(G\subseteq C\) and 
\begin{equation*}
\{D^h | h:\omega\to t\} \cup \{D_n | n < \omega\}
\end{equation*}
-generic. We get a complete function \(g = \cup G:\omega \to 2\)
\(G \cap D^g\neq \emptyset\). 
If $f$ in this non-empty interesction, then $f$ contradicts $g$.
\end{proof}

An alternative, would be by applying Cohen's forcing.

Let \(n<\omega\), \(\alpha < \kappa\).
\begin{equation*}
D_{n,\alpha} = \{f | (\alpha,n)\in \dom(f)\}.
\end{equation*}
is dense. The quantity of the sets is \(\kappa\) ??!.

For distinct \(\alpha,\beta < \kappa\)
\begin{equation*}
D^{\alpha,\beta} = \{f | \exists n < \omega\; (\alpha,n),(\beta,n)\in \dom(f)
\;\land\; f(\alpha,n) \neq f(\beta,n)\}
\end{equation*}
is dense. Let
\begin{equation*}
G = \{D^{\alpha,\beta}| \alpha,\beta<\kappa\land \alpha\neq\beta\}
\;\cup\; \{D_{n,\alpha} | n < \omega \land \alpha < \kappa\}
\end{equation*}
--generic. We get complete function
\begin{equation*}
f = \cup G: \kappa\times\omega \to 2\}
\end{equation*}
for each \(\alpha<\kappa\) a sub-function \(g_\alpha : \omega \to 2\)
By \(g_\alpha(n) = g_{\alpha,n}\)
For distinct \(\alpha,\beta <\kappa\) then \(g_\alpha \neq g_\beta\).
We got \(|\{g_\alpha | \alpha < \kappa\}|= \kappa\).
Hence \(\kappa \leq 2^{\aleph_0}\).
This is the way Paul Cohen showed \ldots ?

General remark on \MA.
In the proof of \(\MA(\aleph_0)\) we did not use \ccc.
But for \(\MA(\aleph_1)\) we did.
For \(\MA(\kappa)\) and above we need that \lrangle{P, \leq} has \ccc.
\begin{equation*}
\textnormal{Col}(\omega,\omega_1) 
  = \{f| f: \omega\to\omega_1 \;\finitepartial\}
\end{equation*}
For all \(\alpha< \omega_1\)
\begin{equation*}
D^\alpha< = \{f| \alpha\in\rng(f)\}
\end{equation*}
\begin{equation*}
D_n = \{f| n \dom(f)\| \qquad (n < \omega).
\end{equation*}
We will tak a generic set that meets any (dense?) set.
Let \(G\subset \textnormal{Col}(\omega,\omega_1)\)
\begin{equation*}
\{D_n| n < \omega\} \cup \{D^\alpha| \alpha < \omega_1\}
\end{equation*}
and again, \(g = \cup G: \omega\to\omega_1\)
and this is a \emph{surjective} function, contradiction.

This set does not have \ccc.
\(\{(0,\alpha)| \alpha < \omega_1\}\) is antichain, 
\(\aleph_1\) functions send $0$ to different values. Hence \ccc\ is required.

\begin{ldef}
Let \(\scrA \subset P(\omega)\) and define
\begin{equation*}
P_\scrA = \{(s,F)|  s\subseteq\omega \,\land\, F\subset\scrA 
  \,\land\, |s|<\aleph_0 \,\land\, |F|<\aleph_0\}
\end{equation*}
\(F\subset P(P(\omega))\) with the order relation:
\begin{equation*}
  \lrangle{s_1,F_1} \leq \lrangle{s_2,F_2}
\end{equation*}
iff
\begin{enumerate}
\item \(s_1 \subset s_2\)
\item \(F_1 \subset F_2\)
\item \(\forall x\in F_1\, s_2\cap x \subset S_1\).
\end{enumerate}
\end{ldef}
Intuitively, can grow, but \ich{3} limits.

Assume \((s,F), (s',F') \in P_\scrA\) what can harm the compatibility ?

Without \ich{3}, we could pick \((s\cup s',F\cup F')\).
Assume \((w,T)\) that makes the assumed pair compatible. But then 
\((s\cup s',F\cup F')\in P_\scrA\) and \(F\cup F' \subseteq T\)
and
\begin{align*}
\forall x\in F\;
r \cap x \subset S   &\implies (s\cup s')\cap x\subset S \\
\forall x\in F'\;
r \cap x' \subset S' &\implies (s\cup s')\cap x'\subset S' \\
\end{align*}
Now \((s,F) \| (s', F') \; \Leftrightarrow\)
for all \(x\in T\) % ??
\begin{align*}
(S\cup S')\cap x &\subseteq X'\\
S'\cap x &\subseteq S\\
\forall x'\in F'\; s \cap x' &\subseteq S'
\end{align*}

\begin{claim}
Let \((s,F)\in P_\scrA\)
then 
\((s,F)\leq (s,F\cup\{x\})\)
for all \(x\in\scrA\).
\end{claim}

\begin{claim}
\(P_\scrA\) has \ccc.
\end{claim}
\begin{proof}
Let 
\begin{equation*}
\scrT_\scrA = \{(s_\alpha, F_\alpha)|\alpha < \omega_1\}.
\end{equation*}
and \(s_\alpha \subset \omega\) finite.
There exists \(I \subseteq \omega_1\), \(|I| = \aleph_1\)
and there exists \(S^*\) such that \(\forall \alpha\in I\, S_\alpha=S^*\)
\begin{equation*}
\{(s^*,F_\alpha)| \alpha\in I\}
\end{equation*}
\begin{equation*}
(s^*,F_\alpha),(s^*,F_\beta) \leq (S^*,F_\alpha\cup F_\beta)
   \qquad (\alpha,\beta\in I).
\end{equation*}
so we have \ccc.
\end{proof}

Consider the next
\begin{claim}
For any \(x\in \scrA\)
\begin{equation*}
D_x = \{(s, F)\in P_\scrA | x\in F\}
\end{equation*}
is dense.
\end{claim}
Let \(G\subseteq P_\scrA\) and \(\{D_x|x\in\scrA\}\)x generoc.
\begin{equation*}
 \omega \supseteq d = \cup \{s|\exists F\,(s,F)\in G\}.
\end{equation*}
For all \(x\in \scrA \; |d\cap x| < \aleph_0\)
From requirement \ich{3}: \((s_1,F_1) \in G\cap D_x\) and so \(x\in F_1\).
Assume
\begin{equation*}
 (s_1,F_1) \leq (s_2,F_2) \in G
\end{equation*}
we get \(d\cap x\subseteq S_1\)
(We need that $d$ be infinite).

Assume \scrA\ is am infinite set of almost disjoint sets of \(\omega\).
Our objective is \(|d|=\aleph_0\).
% lecture break
For all \(n < \omega\)
\begin{equation*}
D_n = \{(s,F)| \max(s)>n\} % ??
\end{equation*}
We have \((s_1,F_1)\in P_\scrA\)
and we want to add to \(s_1\).
\(\omega \setminus F_1| = \aleph_0\) and therefore
there exists \(n < m \in \omega \setminus \cup F_1\)
\begin{equation*}
(s_1,F_1) \leq (s_1\cup\{m\}, F_1)
\end{equation*}
We take \(s_1\cup\{m\}\cap x \subseteq s_1\), \(x \in F_1\).
\\
\emph{Explanation}:
\begin{equation*}
 F_1 = \{B_1,B_2,\ldots, B_L\}\\
B \in \scrA \setminus F_1 \\
\end{equation*}
\(B\cap B_1\), \(B\cap B_2\), \ldots \(B\cap B_L\) are finite,
hence \(|B \setminus \cup F_1| = \aleph_0\).

We claim: Assume for each \(n<\omega\) \(D_n\) is dense.
Let
\begin{equation*}
G = \{D_x| x\in\scrA\} \cup \{D_n | n < \omega\}
\end{equation*}
generic.
Then for all \(x\in\scrA\), \(|d\cap x|< \aleph_0\) and \(|d|=\aleph_0\)

\begin{thm}
Assum \(\MA(\kappa)\). Let \(\scrA,\scrB\in P(\omega)\) and 
\(|\scrA|,|\scrB| \leq \kappa\).
Assume for each finite \(y\in\scrB\) and for each finite \(F\subseteq \scrA\),
\(|t\setminus \cup F|=\aleph_0\).
Then there exists \(d\leq \omega\) such that for all \(b\in\scrB\),
\(|d\cap b|=\aleph_0\) 
\(\forall a\in\scrA\, |d\cap a|<\aleph_0\).
\end{thm}
\begin{proof}
Look at \(P_\scrA\). For each \(a\in\scrA\) we take \(D_a\).
for all \(n<\omega\), \(b\in\scrB\) we take
\begin{equation*}
D_n^b = \{(s,F)\in P_\scrA | \max(s\cap b) > n\}.
\end{equation*}
\((s_1,F_1)\leq(s_1\cup\{m\},F_1) \in D_n^b\).
\\
We want to show that \(D_n^b\) is dense.
Let \((s_1,F_1)\in P_\scrA\), \(n < m\in b \setminus \cup F_1\),
We define $d$ as before, ``we lose initial coordinates''.
With the same arguments \(\forall a\in\scrA\, |d\cap a|<\aleph_0\).
\end{proof}

\begin{claim}
Assume \(\MA(\kappa)\). Let \(\scrA \subseteq P(\omega)\)
infinite family of almost disjoint sets. Then \(\scrA\) is not maximal.
\end{claim}
\begin{proof}
(There are several ways.)\\
\(\scrA,\scrB=\{\omega\}\).
From the last theorem, we get infinite $d$ almost disjoint 
for all sets in \scrA. Hence \scrA\ is not maximal.
\end{proof}

\begin{corollary}
\(\MA \implies\) if \(\scrA \subseteq P(\omega\) is a family
of almost disjoint sets of \(\omega\) infinite and maximal, then
\(|\scrA| = 2^{\aleph_0}\).
\end{corollary}

\begin{claim}
Let \(\scrB \subseteq P(\omega)\) an infinite family of 
almost disjoint sets. Then for all \(scrA \subseteq \scrB\)
there exists \(d \leq \omega\) such that for all \(a\in\scrA\)
and for all \(b\in \scrB\setminus\scrA\) \(|d\cap b|=\aleph_0\).
\end{claim}
\begin{proof}
\ldots
\end{proof}

\begin{thm}
If \(\MA(\kappa)\) then \(2\kappa = 2_{\aleph_1}\).
\end{thm}
\begin{corollary}
(\(\MA\)) \(\forall \kappa < 2^{\aleph_0},\,2^\kappa = 2^{\aleph_0}\).
\end{corollary}
\begin{proof}
Take \(|\scrB|=\kappa\).
Define \(H: P(\omega) \xrightarrow{\text{onto}} P(\scrB)\).
This will be sufficient. We will use ``encryption-like mapping''.
For each \(d\leq\omega\)
\begin{equation*}
H(d) = \{x\in\scrB| |x\cap d| < \aleph_0\}
\end{equation*}
For each \(\scrA\subseteq\scrB\) there exists \(d\subseteq\omega\)
such that \(\scrA = H(d)\) by the previous claim.
\end{proof}
\begin{corollary}
 (\MA). \(2^{\aleph_0}\) is a regular cardinal.
\end{corollary}
\begin{proof}
Assume otherwise. \(\textnormal{cof}(2^{\aleph_0})=\kappa < 2^{\aleph_0}\).
But \(2^\kappa = 2^{\aleph_0}\) and \(\kappa < \textnormal{cof}(2^\kappa)\).
Contradiction.
\end{proof}

The trick that we can encrypt a set in \(omega\) by an element in \(\omega\).

\begin{thm}
Assume \(\MA(\kappa)\).
A union of \(\kappa\) sets of real numbers of first category
is of first category.
\end{thm}


%%%%%%%%%%%%%%%%%%%%%%%%%%%%%%%%%%%%%%%%%%%%%%%%%%%%%%%%%%%%%%%%%%%%%%%%
%%%%%%%%%%%%%%%%%%%%%%%%%%%%%%%%%%%%%%%%%%%%%%%%%%%%%%%%%%%%%%%%%%%%%%%%
\section{2025/April/6}


We will see how some set can separate between two families.

\begin{thm} \label{thm:MAk:AB:sepd}
\(\MA(\kappa)\).
Let \(\scrA,\scrB \subseteq P(\omega)\)) and \(|\scrA|,|\scrB| < \kappa\).
Assume for each each \(y\in\scrB\) and for each finite \(F\subseteq \scrA\),
\(|y \setminus \cup F|=\aleph_0\).
Then there exists \(d\subseteq \omega\) such that 
\(\forall y\in\scrB\,(|d\cap y|=\aleph_0\) and
\(\forall y\in\scrA\,(|d\cap y|<\aleph_0\).
\end{thm}

%%%%%%%%%%%%%%%%%%%%%%%%%%%%%%%%%%%%%%%%%%%%%%%%%%%%%%%%%%%%%%%%%%%%%%%%
\subsection{Sets of First Category}

An application:
\begin{thm}
\(\MA(\kappa)\).
A union of \(\leq \kappa\) subsets of \R\ of first category, is a set of
first category.
\end{thm}
\B{Reminder:} \(X\subset\R\) is of first category iff
it is contained in a countable union of closed sets nowhere dense
(its closure has empty interior).
(We want to show additivity above \(\aleph_0\)).
\begin{proof}
Let \(\{\mathscr{K}_\aleph: \aleph < \kappa\}\) 
a family of closed and nowhere dense subsets of \R.
\iffalse
We need to find countable family of closed nowhere dense subsets
\(\{H_n: n < \omega\}\) such that
\begin{equation*}
\mathsmaller{\bigcup_{d<\kappa}} \mathscr{K}_d \subseteq 
  \mathsmaller{\bigcup_{n<\omega}} H_n.
\end{equation*}
\fi
We will work with complementa \(\{U_\alpha: \alpha<\kappa\}\)
open dense subsets of \R. We need to find \(\{V_n: n<\omega\}\)
such that 
\begin{equation*}
\mathsmaller{\bigcap_{n<\omega}} V_n \subseteq 
  \mathsmaller{\bigcap_{\alpha<\kappa}} U_\alpha\,.
\end{equation*}
Look at all open segments with rational endpoints
\(B=\{(a_i, b_i): i < \omega \land a_i,b_i \in \Q\}\).
Assume \(d\subset \omega\).
\begin{alignat*}{2}
V_n &= \mathsmaller{\bigcup} \{B_i: i > n \land i \in d\} 
  \qquad && (\forall n < \omega) \\
c_j &= \{i < \omega: B_i \subset B_j\}
  \qquad && (\forall j < \omega) \\
\end{alignat*}
If \(|d \cap c_j| = \aleph_0\) 
then \(\forall n < \omega\, (V_n \cap B_j \neq \emptyset)\).
And if \(\forall j < \omega\, (|d \cap c_j| = \aleph_0)\) 
then \(V_n\) is dense for all \(n \leq \omega\).

Put 
\begin{equation*}
a_\alpha = \{i < \omega: B_i \not\subseteq U_\alpha\} \qquad (\alpha < \kappa).
\end{equation*}
If \(a_\alpha \cap d \subseteq n\) (finite and bounded by $n$)
then \(\forall i\in d\,(i < n \implies B_i \subseteq U_\alpha)\)
and also 
\begin{equation*}
\mathsmaller{\bigcup_{m<\omega}} V_m \subseteq V_n \subset U_\alpha.
\end{equation*}
If \(\forall \alpha<\kappa\,(|a_\alpha \cap d|<\aleph_0)\)
then 
\begin{equation*}
\mathsmaller{\bigcap_{n<\omega}} V_n \subset
  \mathsmaller{\bigcap_{\alpha<\kappa}} U_\alpha .
\end{equation*}
\end{proof}

We want to create a set $d$
(with conditions on \(c_j\) and \(a_\alpha\)).
We will want to show that the families satisfy 
the conditions of \ref{thm:MAk:AB:sepd}.
We take
\begin{align*}
A &= \{ a_\alpha: \alpha < \kappa\} \\
B &= \{c_j: j<\omega\}
\end{align*}
A finite \(F\subset  \kappa\)
\begin{align*}
c_j \setminus \mathsmaller{\bigcap_{\alpha\in F}} a_\alpha 
  &= \Pi<\omega: B_j \subseteq B_j\;\land\; 
    \forall \alpha\in F\,(B_i \subseteq U_\alpha\} \\
  &= \{i < \omega: B_j \subseteq B_j\;\land\;  
    B_i \subseteq \cup_{\alpha\in F} U_\alpha\} \\
  &= \{i < \omega: 
    B_j \subseteq B_j\cap
      \left(\mathsmaller{\bigcap_{\alpha\in F}} U_\alpha\right)\}.
\end{align*}
The latter \(\mathsmaller{\bigcap_{\alpha\in F}} U_\alpha\) is a dense open set,
therefore the difference above is infinite.

%%%%%%%%%%%%%%%%%%%%%%%%%%%%%%%%%%%%%%%%%%%%%%%%%%%%%%%%%%%%%%%%%%%%%%%%
\subsection{Sets of Measure Zero}

\begin{thm}
\(\MA(\kappa)\).
A union of \(\kappa\) subsets of \(\R\) with measure-$0$
has measure-$0$.
\end{thm}
\begin{proof}
We will use a forcing-notion in a different way.
Let \(X\subseteq\R\) and \(\mu(X)=0\).
For each \(\epsilon>0\) there exists an open \(U\supseteq X\)
such that \(\mu(U)<\epsilon\).
Assume there is a sequence \(\{M\_\alpha: \alpha<\kappa\}\)
such that \(\forall \alpha<\kappa\,(\mu(M_\alpha)=0\).
We need to show \(\mu(\cup_{\alpha<\kappa}M_\alpha) = 0\).
For any \(\epsilon > 0\) we need to find an open set 
\(U_\epsilon \supseteq \cup_{\alpha<\kappa}M_\alpha\)
such that \(\mu(U_\epsilon)<\epsilon\).

Let \(\epsilon>0\) and define
\begin{equation*}
P_\epsilon = \{p; \mu(p)<\epsilon\;\land\;p\textnormal{ is open}\}.
\end{equation*}
\end{proof}
We define \(p \geq q \; \iff \; p \supseteq q\).

Let \(G \subseteq P_\epsilon\) ``sufficiently'' generic.
and \(\cup G\) is open.
We want to show that \(\mu(\cup G) \leq \epsilon\).

Let \(p,q\in P_\epsilon\), now \(p\| q\;\iff\; \mu(p\cup q)<\epsilon\).
If \(p_1,\ldots,p_n\) are compatible, then \(\mu(\cup_{i=1}^n p_i) < \epsilon\)
Note: any two compatible elements in $G$ by element in $G$.
So \(\cup_{i=1}^n p_i \in G\).

If there exists a countable \(A \subseteq G\) such that
\(\cup G = \cup A\) then \(\mu(\cup A) = \mu(\cup G) < \epsilon\).
Thus it is sufficient to find a countable subset.
Again we will use $B$ the family of all open sets with rational endpoints,
\(|B|=\aleph_0\).

Let \(A = G \cap B\). We will show that \(\cup G = \cup A\).
Clearly \(\cup A \subseteq \cup G\).
For the other direction, let \(x\in\cup G\).
Then there exists an open set \(p\in G\) such that \(x\in p\).
So \(\exists a,b\in\Q\,(x\in (a,b)\subseteq p\),
since $G$ is ``closed'' downwards. Thus \(x\in \cup A\).

Let \(\alpha<\kappa\) and 
\begin{equation*}
D_\alpha = \{p \in P_\epsilon : p \supseteq M_\alpha\}.
\end{equation*}
We will show that \(D_\alpha\) is dense.
Let \(q\in P_\epsilon\), \(\epsilon - \mu(q) > 0\).
Find an open \(r \supseteq M_\alpha\) such that \(\mu(r) < \epsilon - \mu(q)\).
\begin{equation*}
p - e \cup q \supseteq M_\alpha\;\implies\;\mu(p)<\epsilon.
\end{equation*}
Hence \(p\in D_\alpha\).

Take generic \(G = \{D_\alpha: \alpha < \kappa\}\).
Then \(\cup_{\alpha<\kappa}M_\alpha \supseteq \cup G\)
and \(\mu(\cup G)<\epsilon\).
We still need to show that the partial order has \ccc.
Assume otherwise, let \(\{p_\alpha: \alpha < \alpha < \omega_1\}\)
be an antichain (any two are not compatible),
\(\forall \alpha<\omega_1\,(\mu(p_\alpha)<\epsilon)\).

Take an increasing sequence
 \(\{\epsilon_n: n < \omega\land\; \epsilon_n<\epsilon\}\).
For each \(\alpha<\omega_1\) there exists \(n_\alpha<\omega\) 
such that 
 \(\forall n<\omega\,(n_\alpha<n)\,\implies\,\mu(p_\alpha)<\epsilon_n)\).

By the ``pigeonhole principle'' there exist \(n^*\) and \(S\subseteq \omega_1\),
\(|S|=\aleph_1\) such that for all \(\alpha \in S\), \(n_\alpha<n^*\)
and \(\forall \alpha\in S\,(\mu(p_\alpha)<\epsilon_{n^*} < \epsilon)\).
We select \(\delta>0\)  such that 
\(\forall\alpha\in S\,(\mu(p_\alpha) < \epsilon - 3\delta)\)

Any open set can be approximated by finite unions.
Let $C$ be a family of finite unions (of open segments) from $B$.
For each open $U$ and \(\tau > 0\) there exists \(v\in C\)
such that \(\mu(U \Delta V) < \tau\).
\(|C|=\aleph_0\).
For all \(\alpha\in S\) pick \(C_\alpha\subseteq C\) such that
\(\mu(p_\alpha \Delta C_\alpha) < \delta\).
We reduce $S$ into \(\tilde{S}\), so still \(|\tilde{S}|=\alpha_1\)
and \(\forall \alpha\in\tilde{S}\,(C_\alpha=C)\).
Pick distinct \(\alpha,\beta\in \tilde{S}\).
Now \(\mu(p_\alpha\cup p_\beta) \geq \epsilon\) since \(p_\alpha\perp p_\beta\)
and we have:
\begin{align*}
 \mu(p_\alpha \cap p_\beta) &< \epsilon - 3\delta \\
  \mu(p_\alpha \Delta C) &< \delta \\
  \mu(p_\beta \Delta C) &< \delta
\end{align*}
This gives \ccc\ and we can apply \(\MA(\kappa)\).

%%%%%%%%%%%%%%%%%%%%%%%%%%%%%%%%%%%%%%%%%%%%%%%%%%%%%%%%%%%%%%%%%%%%%%%%
\subsection{Intersection in Compact Space}

\begin{thm}
Let $X$ be a topological compact space that has \ccc\
(No \(\omega_1\) disjoint open subsets), Hausdorff (\(T_2\)).
Let \(\{U_\alpha: \alpha < \kappa\}\) be a family
of open dense subsets, then
\(\mathsmaller{\bigcap_{\alpha<\kappa}}U_\alpha \neq \emptyset\).
\end{thm}
\begin{proof}
Let \(P = \{p\subseteq X: p\neq\emptyset \land\,p\;\textnormal{is open}\}\).
Define \(p\geq q \;\iff\; p \subseteq q\) (opposite to what we had before).
(stronger means smaller). % $P$ has \ccc.
\(p\| q\;\iff\; p\cap q\neq \emptyset\)
Hence $P$ has \ccc.

We will have ``sufficiently'' generic \(G \subseteq P\).
Put \(F = \{\overline{p}: p\in G\}\) (closed subsets).
By compactness \(\cap F \neq \emptyset\).
We want to show \(\cap F \subseteq \cap_{\alpha<\kappa} U_\alpha\).

Let \(\alpha<\kappa\), and
 \(D_\alpha = \{p\in P: \overline{p}\subseteq U_\alpha\}\).
We will show that \(D_\alpha\) is dense.
Let nonempty \(q\in P\). \(V=U_\alpha \cap q\) is open and nonempty.
Find nonempty open \(p\) such that \(\overline{p} \subseteq V\)

We cover the boundary of $V$ by finite collection of open sets
that separate the boundary from a neighborhood $p$ of internal \(a\in V\)
and \(\overline{p}\) does not intersect the boundary of $V$.
\end{proof}

%%%%%%%%%%%%%%%%%%%%%%%%%%%%%%%%%%%%%%%%%%%%%%%%%%%%%%%%%%%%%%%%%%%%%%%%
\subsection{Product of \ccc\ Spaces}

\begin{thm} \label{thm:prod:ccc}
\(\MA(\kappa)\).
A product of \ccc\ spaces is \ccc.
\end{thm}
It is sufficient to treat final products, and actually product of two.
\begin{claim} \label{claim:open:subset:finite:sect}
\(\MA(\omega_1)\).
Let $X$ be a topological space that has \ccc.
Let \(\{U_\alpha: U_\alpha\subseteq X \;\land\; \alpha < \aleph_1\}\) 
be a family of nonempty open subsets.
Then there exists \(A \subseteq \omega_1\), \(|A|=\aleph_1\)
such that \(\{U_\alpha: \alpha\in A\}\)
has finite intersection property, 
namely, every intersection of finite number of subsets \(\neq\emptyset\).
Formally:
\begin{equation*}
\forall F\subset A: |F|\leq \omega\;\implies\; 
  \cap_{\beta\in F} A_\beta \neq \emptyset.
\end{equation*}
\end{claim}
Before proving the claim, we will use it to prove \ref{thm:prod:ccc}.
\begin{proof}
Assume \ref{claim:open:subset:finite:sect}.
By negation, assume otherwise namely, $X$ and $Y$ are \ccc\ 
and \(X\times Y\) is not \ccc.
So we have \(\{W_\alpha \subset X\times Y:\alpha<\omega_1\}\)
nonempty disjoint open sets. For each \(\alpha<\omega_1\)
we have \(V_{1\alpha}\subset X\) and \(V_{2\alpha}\subset Y\) such that
\(V_{1\alpha}\times V_{2\alpha} \subseteq W_\alpha \).

We look at the family \(\{V_{1\alpha}:\alpha<\omega_1\}\).
From \ref{claim:open:subset:finite:sect} there exists \(A\subseteq \omega_1\),
\(|A|=\aleph_1\) such that \(\{V_{1\alpha}:\alpha\in A\}\)
(and also  \(\{V_{2\alpha}:\alpha\in A\}\))
has finite intersection property.
Pick distinct \(\alpha,\beta\in A\), \(W_\alpha\cap W_\beta=\emptyset\)
but \(V_{1\alpha}\cap V_{1\beta} \neq \emptyset\). Therefore 
\(V_{2\alpha}\cap V_{2\beta} = \emptyset\).
Hence \(\{V_{2\alpha}:\alpha\in A\}\) contradicts the \ccc\ of $Y$,
which proves the theorem.
\end{proof}
We now prove \ref{claim:open:subset:finite:sect}.
\begin{proof}
For each \(\alpha <\omega_1\) define 
\(V_\alpha = \bigcup_{\alpha<\gamma<\omega_1} U_\gamma\).
If \(\alpha<\beta<\omega_1\) then \(V_\alpha \supseteq V_\beta\),
that is non-increasing.

\B{Supporting Claim:} There exists \(\alpha^* <\omega_1\) 
such that 
\begin{equation*}
\forall \beta<\omega_1\,\left(\alpha<\beta\,\implies\, 
 \overline{V_{\alpha^*}} = \overline{V_\beta}\right)
\end{equation*}
Intuitively, the closures sequence must stabilize.
To prove by negation, define properly decreasing sequence
\(\{\overline{V_{\beta_i}}: i < \omega_1\}\), 
where \(\overline{V_{\beta_i}} \supsetneq \overline{V_{\beta_j}}\)
whenever \(i<j<\omega_1\). 
Now \(\{V_{\beta_i} \setminus \overline{V_{\beta_{i+1}}}\}\)
are open and disjoint contradicting \ccc.
This proves the supporting claim.

Let
\begin{equation*}
P = \{p \subset V_{\alpha^*}: p\neq\emptyset\;\land\; p\textnormal{ is open}\}.
\end{equation*}
Define \(p\geq q\,\iff\, p\subseteq q\).
$P$ has \ccc since $X$ has \ccc.
Let \(G\subseteq P\) ``sufficiently'' generic.
$G$ has finite intersection property because
any two members are compatible and so any finite number of members.

Put \(A=\{\alpha<\omega_1: \exists p\in G (p\subseteq U_\alpha)\}\).
If \(|A|=\aleph_1\) then \(\{U_\alpha: \alpha\in A\}\) as required.

We will need to choose open dense sets to ensure \(|A|=\aleph_1\).
For any \(\gamma\) such that \(\alpha^*<\gamma<\omega_1\) define 
\begin{equation*}
D_\gamma = \{p\in P: \exists \alpha>\gamma\,(U_\alpha \subseteq p)\}.
\end{equation*}
We want to show that \(D_\gamma\) is dense.
Let \(p\in P\), \(p\subseteq V_{\alpha^*}\), 
\(p\subseteq \overline{V_{\alpha^*}} = \overline{V_\gamma}\),
\(p \cap V_\gamma \neq \emptyset\).
(Here there was an argument looking at \(x\in \partial V_\gamma\).)
\begin{equation*}
V_\gamma = \cup_{\alpha<\gamma} U_\alpha\;\implies\;
 \exists \alpha>\gamma\,(p\cap U_\alpha \neq \emptyset).
\end{equation*}
\end{proof}

There will be an application related to Suslin conjecture.

%%%%%%%%%%%%%%%%%%%%%%%%%%%%%%%%%%%%%%%%%%%%%%%%%%%%%%%%%%%%%%%%%%%%%%%%
\subsection{Invariants of Continuum}

\begin{ldef}
Let \(f,g:\omega\to\omega\). We say that $f$ \B{dominates} $g$
written as \(f \geq^* g\) iff there exists \(n_0<\omega\)
such that \(\forall n\geq n_0\,(f(n)\geq g(n))\).
\end{ldef}
We look at \(\lrangle{\omega^\omega, \leq^*}\).
\begin{ldef}
{\nullfont a}
\begin{itemize}
\itemch{a} \(F\subseteq \omega^\omega\) is called \emph{unbounded}
if there is no \(f:\omega\to\omega\) that dominates all \(g\in F\).
\itemch{b} \(F\subseteq \omega^\omega\) is called \emph{cofinal} if
\begin{equation*}
 \forall f \in \omega^\omega\,\exists g\in F\,(g \geq^* f).
\end{equation*}
\end{itemize}
\end{ldef}

\begin{ldef}
\begin{equation*}
\underset{\sim}{b} 
  = \min\{|F|: f\subseteq \omega^\omega\;\land\; F\textnormal{ is unbounded}\}
\end{equation*}
is called \emph{bounding number}.
\begin{equation*}
\underset{\sim}{d}
  = \min\{|F|: f\subseteq \omega^\omega\;\land\; F\textnormal{ is cofinal}\}
\end{equation*}
is called \emph{dominating number}.
\end{ldef}
We always have \(\underset{\sim}{b} \leq \underset{\sim}{d}\).
intuitively, if cofinal then unbounded.

\begin{thm}
\MA. 
\begin{equation*}
\underset{\sim}{b} = \underset{\sim}{d} = 2^{\aleph_0}.
\end{equation*}
\end{thm}
\begin{proof}
We will show \(\MA(\kappa)\;\implies\; \underset{\sim}{b}>\kappa\).
For any \(\{f_\alpha: \alpha>\kappa\}\) there exists $f$
that dominates all.
(We introduce forcing notion).
Define
\begin{equation*}
P = \{(s,F): s: n\to\omega\,\land\,n<\omega\,\land\,|F|<\aleph_0\}.
\end{equation*}
and \((s_1,F_1) \geq (s_2,F_2) \;\iff\)
\begin{enumerate}
\item \(s_1\supseteq s_2\)
\item \(F_1 \supseteq F_2\)
\item \(\forall k \in \dom(s_1)\setminus\dom(s_2)\,\forall \alpha\in F_2\;
  (s_1(k) > f_\alpha(k))\).
\end{enumerate}
We want that ``the generated function will dominate \(F_2\)''.
We will have a generic \(G\subseteq P\) and \(f=\cup G\).
we want to show \(f:\omega\to\omega\).

For eack \(k\in\omega\) define
\begin{equation*}
D_k = \{(s,F): k\in \dom(s)\}
\end{equation*}
which is dense.
Assume \((s,F)\), \(k<\omega\). If \(k\in\dom(s)\) then no problem.
Otherwise, \(\dom(s^*)= \dom(s) \cup\{k\}\) with
\begin{equation*}
s^*(k) = \max(\{f_\alpha(k): \alpha\in F\}) + 1.
\end{equation*}
This \(\max\) is possibile for $F$ is finite.
Note:
\begin{equation*}
(s,F) \leq (s^*,F> \in D_k.
\end{equation*}
Put \(f=\cup G: \omega\to\omega\).
\begin{equation*}
D^{\alpha} = \{(s,F): \alpha\in F\} \qquad (\alpha<\kappa).
\end{equation*}
We assume for $G$ (condition (3.) is not involved)
\((s,F) \in G \cap D^\alpha\)
Hence \(f >^* f_\alpha\).
\end{proof}

We will discuss \(\alpha_1\) relative to \(\aleph_0\),
and Hechlen forcing.

%%%%%%%%%%%%%%%%%%%%%%%%%%%%%%%%%%%%%%%%%%%%%%%%%%%%%%%%%%%%%%%%%%%%%%%%
%%%%%%%%%%%%%%%%%%%%%%%%%%%%%%%%%%%%%%%%%%%%%%%%%%%%%%%%%%%%%%%%%%%%%%%%
\section{2025/April/27}


%%%%%%%%%%%%%%%%%%%%%%%%%%%%%%%%%%%%%%%%%%%%%%%%%%%%%%%%%%%%%%%%%%%%%%%%
\subsection{Suslin Hypothesis}

\B{Suslin Hypothesis} --- source of Martin Axiom.

The real numbers \lrangle{\R, \leq} have:
\begin{enumerate}[label=(\alph*)]
\item Linear dense order with neither first nor last element.
\item Completeness. For all bounded subset, there exists supremmum infimum.
\item Separable. The exists countable dense subset.
\end{enumerate}

\B{Question} Can Separability be replaced by \ccc.

\ccc\ says that there are no \(\aleph_1\) open (non-empty?) disjoing segments.

{Suslin Hypothesis (SH) says: ``Yes replace is possible''.
Jech, Tenenbaum, Solvay showed consistenct with negation (of SH).

\begin{ldef}
  Let \lrangle{L,\leq} be a set with linear order.
  It is called \emph{Suslin line} iff
  iff it has \ccc\ but not separable (And also dense order).
\end{ldef}

To discuss existence or non-existence we need to move from a line to a tree.
\begin{ldef}
  A poset \lrangle{T,\leq_T} is called a \emph{tree} iff
  for all \(x\in T\) the subset
  \(\hat{x} = \{t\in T: t\leq_T x\}\) is well-ordered.
\end{ldef}

The \emph{height} of $x$ is \(\tht(x) = \otp(\hat(x),<_T)\).

The \emph{height} of $T$ is \(\tht(T) = \sup_{x\in T} (\tht(x) + 1)\).

The \emph{level} ---
 \(\forall \alpha<\tht(T), \Lev_\alpha(T) = \{x\in T: \tht_T(x)=\alpha\}\).

\begin{ldef}
Let \lrangle{T,\leq_T} be a tree. \(b\subseteq T\) is called
a \emph{branch} (\emph{chain}?) iff 
\begin{enumerate}[label=(\alph*)]
\item \lrangle{b,\leq_T} is well ordered.
\item \(\forall x\in b, \forall y\leq_T x\, y\in b\).
\end{enumerate}
\end{ldef}

A branch $b$ is called \emph{maximal branch} iff there is no 
branch $c$ such that \(b\subsetneq c\).

Let \lrangle{T,\leq_T} be a tree. \(x,y\in T\) 
are \emph{compatible} iff \(x=y\) or \(x\leq_T y\) or \(y\leq_Tx\).
Existence of \(z\geq x,y\) is equivalentm since \(x,y\) are on a same branch.

\(A\subseteq T\) is \emph{antichain} iff any two distinct elements 
are not compatible.

\begin{ldef}
\lrangle{T,\leq_T} is called \(\aleph_1\)\emph{-tree} iff
\begin{enumerate}[label=(\alph*)]
\item \(\tht(T)=\aleph_1\).
\item \(\forall \alpha<\aleph_1\, |\Lev_\alpha(T)\leq\aleph_0\).
\end{enumerate}
\end{ldef}

\begin{ldef}
\lrangle{T,\leq_T}  is called \emph{Suslin tree} 
\begin{enumerate}
\item It has no branch of length \(\aleph_1\).
\item Every level is an antichain.
\end{enumerate}
\end{ldef}
Looks like the definition can be simplified.

%%%%%%%%%%%%%%%%%%%%%%%%%%%%%%%%%%%%%%%%%%%%%%%%%%%%%%%%%%%%%%%%%%%%%%%%
\subsection{Relation between Suslin Tree and Suslin Line}

\begin{claim}
Assume there exists a Suslin line. Then there exists a Suslin tree. 
\end{claim}
\begin{proof}
Let \lrangle{L,\leq_L} be a Suslin line. Let
\(C=\{c\in L: c\;\textnormal{is isolated}\}\).
By \ccc, \(|C|\leq\aleph_0\).
\end{proof}
We look at \(L\setminus C\). We will build by induction
a set of closed segments \(\{[a_\alpha,b_\alpha]: \alpha<\omega_1\}\).
We also require \((a_\alpha,b_\alpha)\neq\emptyset\).
Assume we selected  \(\{[a_\beta,b_\beta]: \beta<\alpha\}\)
for \(\alpha<\omega_1\). We will select \([a_\alpha,b_\alpha]\).
Let \(X_\alpha = \{a_\beta,b_\beta: \beta<\alpha\} \cup C\).
We have \(|X_\alpha|\leq \aleph_0\).
Since \lrangle{L,\leq_L} is not separable,
we take \([a_\alpha,b_\alpha]\) such that 
\((a_\alpha,b_\alpha)\neq \emptyset\) 
and \([a_\alpha,b_\alpha]\cap X_\alpha = \emptyset\).
We define a tree \(T=\{[a_\alpha,b_\alpha]: \alpha<\omega_1\}\) and
\begin{equation*}
[a_\alpha,b_\alpha] <_T [a_\beta,b_\beta]
\;\iff\; 
[a_\alpha,b_\alpha] \supsetneq [a_\beta,b_\beta].
\end{equation*}
(opposite Containment). We will show that $T$ is Suslin-tree.

If \(x\in T\) then there exists \(b\in\omega_1\)
such that \(x=[a_\beta,b_\beta]\).

\(\{t\in T: t <_T x\}\).
\([a_\beta,b_\beta]\subset t\) then there exists \(\gamma < \beta\)
such that \(t = [a_\gamma,b_\gamma]\).
The order between the indices respects the order of the ordinals.

On each step, we select a segment that is contained in
previous segment or disjoint to all previous segments.
Every antichain in $T$ consist of disjoint segments \([a_\alpha, b_\beta]\).
By \ccc\ of $L$ we get \ccc\ fot $T$.

Assume \(b\subseteq T\) a branch of length \(\aleph_1\).
\(e=\{[a_{\alpha_i},b_{\alpha_i}]: i < \omega_1\}\)
a decreasing sequence of segments. We will look at ``left segments'':
\(\{(a_{\alpha_i},a_{\alpha_{i+1}}): i < \omega_1\}\)
and we get \(\aleph_1\) disjoint segments, a contradiction.

Opposite direction. 
\begin{claim} \label{claim:suslin:tree:line}
If there exists Suslin tree, then there exists Suslin line.
\end{claim}
\begin{ldef}
An \(\aleph_1\)-tree \lrangle{T,\leq_T} is called (normal) 
\emph{well pruned tree} iff
\begin{enumerate}
\item \(\Lev_0(T)=1\) that is single root.
\item \(\forall x\in T\,\forall \alpha<\omega_1\, 
 \alpha > \tht_T(x) \implies \exists y\in\Lev_\alpha(T)\).
\item \(\forall x\in T\) the set 
  \(\Suc_T(x) = \{y\in T: x<_T y \land \tht_T(y)=\tht_T(x)+1\}\)
  is countable.
\item For each limit ordinal \(\alpha<\omega_1\) if a branch \(b\subseteq T\)
that consists of points of levels \(< \alpha\) and $b$ continues
to level \(\alpha\) then there exists a unique continuation 
of $b$ to level \(\alpha\).
\end{enumerate}
\end{ldef}
\begin{claim}
If there exists Suslin tree, then there exists a well pruned Suslin tree.
\end{claim}
\begin{proof}
Let \lrangle{T,\leq_T} a Suslin tree. We will well prune it.
FOr all \(x\in T\) let \(T_x = \{y\in T: x\leq_T y \lor y\leq_T x\}\).
We look at the first level \(|\Lev_0(T)|\leq \aleph_0\).
We must have at least some \(x\in \Lev_0(T)\) such that \(|T_x| = \aleph_1\).
So we pick such $x$. This is the first pruning step.
We  put \(T^1 = T_x\) (now \(|\Lev_0(T^1)|=1\)).

We now look at \(\Lev_1(T^1\). Again \(\exists x\in T^1\) 
such that \(|T^1_x| = \aleph_1\).
We leave 
\begin{equation*}
\{x\in T^1: x\in \Lev_1(T^1) \land |T^1_x| = \aleph_1\}.
\end{equation*}
So now \(\Lev_1(T^2) = \{x\in\Lev_1(T^1): |T^1_x| = \aleph_1\}\).
We similarly continue pruning for exah level \(\alpha\) (\(1<\alpha<\omega_1\)).
In each step we drop at most \(\aleph_0\) nodes.
\(\Lev_\alpha(T^2) = \{x\in \Lev_\alpha(T^1): |T^1_x| = \aleph_1\}\).
This defined \(T^2\).

We need to handle requirement (3), 
that is \(\forall x\in T^3\,|\Suc_{T^3}(x)|=\aleph_1\).

For each \(x\in T^2\) looking at successors, at some point
there will be more than single successor.

Every sub-sequence of unique successors, we collapse into single node.
Now we have a Suslin tree, but in each node we have more than a single successor.

we look at splitting steps from \(\alpha\) to \(\alpha+\omega\).
We again collapse such sequence.
We skip formalizing this step.

This is ``ending'' the proof thta there exists well pruned Suslin tree.
\end{proof}

Back to the claim \ref{claim:suslin:tree:line}
\begin{proof}
Let \lrangle{T,\leq_T} be a well pruned Suslin tree.
For each \(x\in T\) we order \(\Suc_T(x)\) in a linear dense order
without first and without last element.
We put
\begin{equation*}
L = \{b \subseteq T: b\; \textnormal{ is a maximal branch in}\; T\}.
\end{equation*}
We will define an order \(\leq_L\) on $L$.
For each two distinct branches we will look at split point
(like dictionary order).
It is easy(?) to see that this is a linear order.
We need to show it is dense.
On the successors we impose an order by a one-to-one onto mapping
\(\aleph_0\to\Q\). As in \Q\ we get a dense order.

% We will show that there is no countable dense subset.
Assume \(D\subset L\) a countable subset.
For each \(b\in D\) there exists \(\alpha_b < \omega_1\)
so $b$ does not achieve level \(\alpha_b\).
Now \(\cup_{b\in D} \alpha_b = \alpha^* < \omega_1\).
Let $c$, $e$ be distinct maximal branches that continues beyond \(\alpha^*\).
all branches ends at \(\alpha^*\) so in $D$ we have no elements in the
the segment \((c,e)\). Thus $D$ cannot be dense.

We need to show \ccc.
By negation, assume \lrangle{L,\leq_L} has
an antichain \(A=\{(a_\alpha,b_\alpha): \alpha < \omega_1\}\).
These are segments of branches.
We will need to show \(\aleph_1\)-antichain in T.
In each segment \((a_\alpha,b_\alpha)\)
We look at last common $T$-node \(x_\alpha\) of \(\{a_\alpha,b_\alpha\}\),
we look on the successors of \(x_\alpha\). By the ordered inherited from \Q.
A split can be before \(x_\alpha\) (simpler)
or after in the successors of \(x_\alpha\).

Formal argument is missing.
\end{proof}

%%%%%%%%%%%%%%%%%%%%%%%%%%%%%%%%%%%%%%%%%%%%%%%%%%%%%%%%%%%%%%%%%%%%%%%%
%%%%%%%%%%%%%%%%%%%%%%%%%%%%%%%%%%%%%%%%%%%%%%%%%%%%%%%%%%%%%%%%%%%%%%%%
%%%%%%%%%%%%%%%%%%%%%%%%%%%%%%%%%%%%%%%%%%%%%%%%%%%%%%%%%%%%%%%%%%%%%%%%
\end{document}
