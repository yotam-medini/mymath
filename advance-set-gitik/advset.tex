\documentclass[11pt,pdftex,twoside,a4paper]{article}
\usepackage{amsthm}
\usepackage{array}
\usepackage{graphicx}
\usepackage{relsize}

\usepackage{amsmath}
\usepackage{amssymb}
% \usepackage{eucal}
\usepackage{mathrsfs}

% \usepackage{fullpage}

\usepackage{geometry}
\geometry{left=1in, right=1in, top=1in, bottom=1in}

\setlength{\parindent}{0pt}
\setlength{\parskip}{6pt}


% are we in pdftex ????
\ifx\pdfoutput\undefined % We're not running pdftex
\else
\RequirePackage[colorlinks,hyperindex,plainpages=false]{hyperref}
\def\pdfBorderAttrs{/Border [0 0 0] } % No border arround Links
\fi

% \usepackage{fancyheadings}
\usepackage{fancyhdr}
\usepackage{pifont}

\pagestyle{fancy}
% \addtolength{\headwidth}{\marginparsep}
% \addtolength{\headwidth}{\marginparwidth}
%  \addtolength{\textheight}{2pt}

\newcommand{\ineqjton}{\overset{1\leq i,j \leq n}{i \neq j}}
\newcommand{\srightmark}{\rightmark}
\newcommand{\sfbfpg}{\sffamily\bfseries{\thepage}}
  \newcommand{\symenvelop}{%
     {\nullfont\ }\relax\lower.2ex\hbox{\large\Pisymbol{pzd}{41}}}
% \renewcommand{\chaptermark}[1]{\markboth{\thechapter.\ #1}}

\iffalse
% \lhead[\fancyplain{}{{\sfbfpg}}]{\fancyplain{}\bfseries\srightmark}
\lhead[\fancyplain{}{{\sfbfpg}}]{\fancyplain{}\sl\srightmark}
% \rhead[\fancyplain{}\bfseries\leftmark]{\fancyplain{}{{\sfbfpg}}}
\rhead[\fancyplain{}\sl\leftmark]{\fancyplain{}{{\sfbfpg}}}
\lfoot{\today}
\cfoot{Yotam Medini \copyright}
  \newcommand{\symenvelop}{%
     {\nullfont a}\relax\lower.2ex\hbox{\large\Pisymbol{pzd}{41}}}
\rfoot{\symenvelop\ \texttt{yotam.medini@gmail.com}}

\renewcommand{\headrulewidth}{0.4pt}
\renewcommand{\footrulewidth}{0.4pt}
\fi

\fancyplain{plain}{%
 \fancyhf{}
 \fancyhead[LE,RO]{\fancyplain{}{{\sfbfpg}}}
 \fancyhead[RE,LO]{\sl\leftmark}
 \fancyfoot[L]{\today}
 \fancyfoot[C]{Yotam Medini \copyright}
 \fancyfoot[R]{\symenvelop\ \texttt{yotam.medini@gmail.com}}
 \renewcommand{\headrulewidth}{0.4pt}
 \renewcommand{\footrulewidth}{0.4pt}
}

% \usepackage{amstex}
% \usepackage{amsmath}
% \usepackage{amssymb}
\usepackage{amsthm}
\usepackage{bm}
\usepackage{makeidx}
\makeindex % enable

% 'Inspired' by:
%% This is file `uwamaths.sty',
%%%     author   = "Greg Gamble",
%%%     email     = "gregg@csee.uq.edu.au (Internet)",

\makeatletter
\def\DOTSB{\relax}
\def\dotcup{\DOTSB\mathop{\overset{\textstyle.}\cup}}
 \def\@avr#1{\vrule height #1ex width 0pt}
 \def\@dotbigcupD{\smash\bigcup\@avr{2.1}}
 \def\@dotbigcupT{\smash\bigcup\@avr{1.5}}
 \def\dotbigcupD{\DOTSB\mathop{\overset{\textstyle.}\@dotbigcupD%
                               \vphantom{\bigcup}}}

\def\dotbigcupT{\DOTSB\smash{\mathop{\overset{\textstyle.}\@dotbigcupT%
                              \vphantom{\bigcup}}}%
                       \vphantom{\bigcup}\@avr{2.0}}
\def\dotbigcup{\mathop{\mathchoice{\dotbigcupD}{\dotbigcupT}
                                  {\dotbigcupT}{\dotbigcupT}}}
\let\disjunion\dotcup
\let\Disjunion\dotbigcup
\makeatother


\newcommand{\half}{\ensuremath{\frac{1}{2}}}



\newcommand{\C}{\ensuremath{\mathbb{C}}} % The Complex set
\newcommand{\aded}{\ensuremath{\textrm{a.e.}}} % almost everyehere
\newcommand{\chhi}{\raise2pt\hbox{\ensuremath\chi}}           %raise the chi
\newcommand{\calB}{\ensuremath{\mathcal{B}}}
\newcommand{\calE}{\ensuremath{\mathcal{E}}}
\newcommand{\calF}{\ensuremath{\mathcal{F}}}
\newcommand{\calG}{\ensuremath{\mathcal{G}}}
\newcommand{\calM}{\ensuremath{\mathcal{M}}}
\newcommand{\calR}{\ensuremath{\mathcal{R}}}
\newcommand{\eqdef}{\ensuremath{\stackrel{\mbox{\upshape\tiny def}}{=}}}
\newcommand{\frakB}{\ensuremath{\mathfrak{B}}}
\newcommand{\frakC}{\ensuremath{\mathfrak{C}}}
\newcommand{\frakF}{\ensuremath{\mathfrak{F}}}
\newcommand{\frakG}{\ensuremath{\mathfrak{G}}}
\newcommand{\frakI}{\ensuremath{\mathfrak{I}}}
\newcommand{\frakM}{\ensuremath{\mathfrak{M}}}
\newcommand{\scrB}{\ensuremath{\mathscr{B}}}
\newcommand{\scrD}{\ensuremath{\mathscr{D}}}
\newcommand{\scrN}{\ensuremath{\mathscr{N}}}
\newcommand{\scrP}{\ensuremath{\mathscr{P}}}
\newcommand{\scrQ}{\ensuremath{\mathscr{Q}}}
\newcommand{\scrR}{\ensuremath{\mathscr{R}}}
\newcommand{\Lp}[1]{\ensuremath{\mathbf{L}^{#1}}} % Lp space
\newcommand{\N}{\ensuremath{\mathbb{N}}} % The Natural Set
\newcommand{\Q}{\ensuremath{\mathbb{Q}}} % The Rational set
\newcommand{\R}{\ensuremath{\mathbb{R}}} % The Real Set
\newcommand{\T}{\ensuremath{\mathbb{T}}} % The Thorus [-pi,\pi)
\newcommand{\Z}{\ensuremath{\mathbb{Z}}} % The Integer Set
\newcommand{\intR}{\int_{-\infty}^{\infty}} % Integral over the reals
\newcommand{\posthat}[1]{#1{\,\hat{}\,}}

% sequences
\newcommand{\seq}[2]{\ensuremath{#1_1,\ldots,#1_{#2}}}
\newcommand{\seqn}[1]{\seq{#1}{n}}
\newcommand{\seqan}{\seq{a}{n}}
\newcommand{\seqxn}{\seq{x}{n}}
\newcommand{\seqalphn}{\seq{\alpha}{n}}

\newcommand{\mset}[1]{\ensuremath{\{#1\}}}


%%%%%%%%%%%%
%% math op's
\newcommand{\Ang}{\mathop{\rm Ang}\nolimits}
\newcommand{\Arg}{\mathop{\rm Arg}\nolimits}
\newcommand{\co}{\mathop{\rm co}\nolimits}
\newcommand{\conv}{\mathop{\rm conv}\nolimits}
\newcommand{\diam}{\mathop{\rm diam}\nolimits}
% \newcommand{\dim}{\mathop{\rm dim}\nolimits}
% \newcommand{\esssup}{\mathop{\rm ess\ sup}\nolimits}
\DeclareMathOperator*{\esssup}{ess\,sup}
\newcommand{\ext}{\mathop{\rm ext}\nolimits}
\newcommand{\Id}{\mathop{\rm Id}\nolimits}
\newcommand{\Image}{\mathop{\rm Im}\nolimits}
\newcommand{\Ind}{\mathop{\rm Ind}\nolimits}
\newcommand{\Lip}{\mathop{\rm Lip}\nolimits}
\newcommand{\lip}{\mathop{\rm lip}\nolimits}
\newcommand{\Ker}{\mathop{\rm Ker}\nolimits}
\newcommand{\rank}{\mathop{\rm rank}\nolimits}
\newcommand{\Res}{\mathop{\rm Res}\nolimits}
\newcommand{\supp}{\mathop{\rm supp}\nolimits}
\newcommand{\vol}{\mathop{\rm vol}\nolimits}
\newcommand{\vspan}{\mathop{\rm span}\nolimits}

% I wish this was more standardized
\renewcommand{\Re}{\mathop{\bf Re}\nolimits}
\renewcommand{\Im}{\mathop{\bf Im}\nolimits}

\newcommand{\inter}[1]{\ensuremath{#1^{\circ}}}  % interior
\newcommand{\closure}[1]{\ensuremath{\overline{#1}}} % closure
\newcommand{\boundary}[1]{\ensuremath{\partial #1}} % closure


\newcommand{\ich}[1]{(\textit{#1})}
\newcommand{\itemch}[1]{\item[\ich{#1}]}
\newcommand{\itemdim}{\item[\(\diamond\)]}

% Special names
\newcommand{\Cech}{\u{C}ech}

\author{Yotam Medini}


%%%%%%%%%%%
%% Theorems
%%
\makeatletter
\@ifclassloaded{book}{
 \newtheorem{thm}{Theorem}[chapter]
 \newtheorem{cor}[thm]{Corollary}
 \newtheorem{lem}[thm]{Lemma}
 \newtheorem{llem}[thm]{Local Lemma}
 \newtheorem{lthm}[thm]{Local Theorem}
 % \newtheorem{quotecor}{Corollary}
 % \newtheorem{quotelem}{Lemma}[section]
 \newtheorem{quotethm}{Theorem}[chapter]
}{}
\makeatother
\newtheorem{Def}{Definition}


\newcommand{\loclemma}{Lemma}


% \newcommand{\proofend}{\(\bullet\)}
% \newcommand{\proofend}{\hfill\(\blacksquare\)}
\newcommand{\proofend}{\hfill\(\Box\)}
\newenvironment{thmproof}
{\textbf{Proof.}}
{\proofend}

\newcommand{\chapterTypeout}[1]{\typeout{#1} \chapter{#1}}
\newcommand{\sectionTypeout}[1]{\typeout{#1} \section{#1}}

% abbreviations, ensuremath
\newcommand{\fx}{\ensuremath{f(x)}}
\newcommand{\gx}{\ensuremath{g(x)}}
\newcommand{\lrangle}[1]{\ensuremath{\langle #1 \rangle}}
\newcommand{\lrbangle}[1]{\ensuremath{\left\langle #1 \right\rangle}}
\newcommand{\M}{\ensuremath{\mathfrak{M}}}
\newcommand{\mldots}{\ensuremath{\ldots}}
\newcommand{\salgebra}{\(\sigma\)-algebra}
\newcommand{\swedge}{\;\wedge\;}
\newcommand{\wlogy}{without loss of generality}
\newcommand{\Wlogy}{Without loss of generality}
\newcommand{\twopii}{\ensuremath{2\pi i}}
\newcommand{\dtwopii}{\ensuremath{\frac{1}{\twopii}}}

% https://tex.stackexchange.com/
% questions/22252/how-to-typeset-function-restrictions
\newcommand\restr[2]{\ensuremath{% we make the whole thing an ordinary symbol
  \left.\kern-\nulldelimiterspace % automatically resize the bar with \right
  #1 % the function
  \vphantom{\big|} % pretend it's a little taller at normal size
  \right|_{#2} % this is the delimiter
  }}

\newenvironment{excopyOLD}
{\item\begin{minipage}[t]{.8\textwidth}\footnotesize}
{\smallskip\hrule\end{minipage}}

\newenvironment{excopy}
{\item % \relax
 \begin{list}{}{
 \setlength{\topsep}{0pt}
 \setlength{\partopsep}{0pt}
 \setlength{\itemsep}{0pt}
 \setlength{\parsep}{0pt}
 \setlength{\leftmargin}{0pt}
 \setlength{\rightmargin}{20pt}
 \setlength{\listparindent}{0pt}
 \setlength{\itemindent}{0pt}
 % \setlength{\labelsep}{0pt}
 \setlength{\labelwidth}{0pt}
 \footnotesize
 }
 \item
}
{\par
 % {\nullfont 0}
 \hrulefill
 \end{list}
}


\newcommand{\B}[1]{\textbf{#1}}
\newcommand{\ccc}{c.c.c.}

%% \newtheorem{manualclaiminner}{Claim}
%% \newenvironment{manualclaim}[1]{%
%%   \renewcommand\themanualclaiminner{#1}%
%%   \manualclaiminner
%% }{\endmanualclaiminner}
%% 
%% \newtheorem{claim}{Claim}

\newtheorem{thm}{Theorem}[section]
\newtheorem{ldef}[thm]{Definition}
\newtheorem{lemma}[thm]{Lemma}
\newtheorem{claim}[thm]{Claim}

\title{Advanced Set Theory
          \\
       Lecture Notes --- Professor Mordechai Gitik}

\author{Yotam Medini}

%%%%%%%%%%%%%%%%%%%%%%%%%%%%%%%%%%%%%%%%%%%%%%%%%%%%%%%%%%%%%%%%%%%%%%%%
%%%%%%%%%%%%%%%%%%%%%%%%%%%%%%%%%%%%%%%%%%%%%%%%%%%%%%%%%%%%%%%%%%%%%%%%
%%%%%%%%%%%%%%%%%%%%%%%%%%%%%%%%%%%%%%%%%%%%%%%%%%%%%%%%%%%%%%%%%%%%%%%%
\begin{document}
\maketitle
\newpage
\tableofcontents
\newpage

%%%%%%%%%%%%%%%%%%%%%%%%%%%%%%%%%%%%%%%%%%%%%%%%%%%%%%%%%%%%%%%%%%%%%%%%
%%%%%%%%%%%%%%%%%%%%%%%%%%%%%%%%%%%%%%%%%%%%%%%%%%%%%%%%%%%%%%%%%%%%%%%%
\section{2025/March/23}

In this course we will deal with infinite combinatorics.

Text book for the course:
  \B{Kunen} / \textit{Set SET THEORY, 
    An Introduction to Independence Proofs}

We will use strengthening of the continuum-hypothesis
and strengthening of negation of the continuum-hypothesis.

%%%%%%%%%%%%%%%%%%%%%%%%%%%%%%%%%%%%%%%%%%%%%%%%%%%%%%%%%%%%%%%%%%%%%%%%
\subsection{Almost Disjoint Sets}

\begin{ldef}
Let \(\kappa\) be infinite cardinal, \(\scrA \subseteq P(\kappa\)
is a family of of \emph{almost disjoint sets} iff 
\begin{enumerate}
\item For any \(A\in \scrA\), \(|A|=\kappa\).
\item For any \(A,B \in scrA\) if \(A\neq B\) then \(|A \cap B|<\kappa\).
\end{enumerate}
\end{ldef}

%%%%%%%%%%%%%%%%%%%%%%%%%%%%%%%%%%%%%%%%%%%%%%%%%%%%%%%%%%%%%%%%%%%%%%%%
\subsection{Maximal Families of Almost Disjoint Sets}

\begin{claim}
Let \(\kappa \geq \aleph_0\) be a regular cardinal.
Let \(\calA \subset P(\kappa)\) be a family of almost disjoint sets.
Then:
\begin{itemize}
\itemch{a} If \(|A|=\kappa\) then \calA\ is not maximal,
that is there exists \(d\leq \kappa\) and for any \(a\in \calA\),
\(d\cap a| < \kappa\).
\itemch{b} There exists a maximal family \(\calB \supset \calA\)
of almost disjoint sets, \(|B| \geq \kappa^+\).
\end{itemize}
\end{claim}
\begin{proof}
\ich{a} \(\implies\) \ich{b} by the Zorn lemma.

Prooving \ich{a}. Let \(\calA = \{A_\alpha | \alpha < \kappa\}\).
We will ``convert'' it to be ``disjoint''. We will define a family
\(\{A_\alpha'|\alpha < \kappa\}\) of disjoint sets.
\begin{align*}
 A_0' &= A_0 \\
 A_1' &= A_1 \setminus A_0 \\
 A_\alpha' &= A_\alpha \setminus 
   \mathsmaller{\bigcup_{\beta<\alpha}} (A_\alpha \setminus A_\beta).
\end{align*}
We get completely disjoint sets.

Define \(B=\{b_\alpha| \alpha<\kappa\}\) where \(b_\alpha \in A_\alpha'\).
\\
\vspace{-12pt}
\begin{center}
\begin{tabular}{llll}
\(A_0'\) & \(A_1'\) & \(\cdots\) & \(A_\alpha'\) \\
\rotatebox[origin=c]{90}{\(\in\)} &
\rotatebox[origin=c]{90}{\(\in\)} &
\(\cdots\) &
\rotatebox[origin=c]{90}{\(\in\)} \\
\(b_0\) & \(b_1\) & \(\cdots\) & \(b_\alpha\)
\end{tabular}
\end{center}
Therefore \(|A_\alpha \cap B| < \kappa\).
\end{proof}

How can we get maximal cardinal?\\
In \(\omega\) such family would have \(2^{\aleph_0}\) cardinality.
We can generalize.

\begin{claim}
Let \(\kappa \geq \aleph_0\) infinite cardinal.

\B{Assumption}: for each \(\alpha < \kappa\), \(2^\alpha \leq \kappa\).
\begin{equation*}
2^{>\kappa} = \mathsmaller{\bigcup_{\alpha < \kappa}}2^\alpha.
\end{equation*}
\(\implies 2^{<\kappa} = \kappa\).
This assumption is kind of the continuum-hypothesis.

There exists \(\calA \subset P(\kappa)\) a family of almost disjoint sets
with maximal cardinality \(2^\kappa\).
\end{claim}
Here we do not assume regularity of \(\kappa\)
\begin{proof}
(With a nice trick.)
Let
\begin{equation*}
I = \{X \subseteq \kappa | \exists \alpha < \kappa, \alpha \supseteq X\}.
\end{equation*}
Therefore \(|I| = \kappa\).
Now argument of \(\leq\) and \(\geq\).
For each \(y \supseteq \kappa\) let
\begin{equation*}
A_y = \{y \cap \alpha | \alpha < \kappa\} \subset I.
\end{equation*}
Assume \(y_1,y_2 \subset \kappa\), \(y_1 \neq y_2\).
So 
\begin{equation*}
 |A_{y_1} \cap A_{y_2}| < \kappa.
\end{equation*}
\emph{Note:} A difference in one element implies a cardinality difference.
Assume \(\delta \in y_1 \setminus y_2\). Then for each \(\alpha\)
such that \(\delta < \alpha < \kappa\) then
\begin{equation*}
 \delta \in y_1\cap \alpha \neq y_2 \cap \alpha
\end{equation*}
\end{proof}
Thus we got a maximal family of disjoint sets.

%%%%%%%%%%%%%%%%%%%%%%%%%%%%%%%%%%%%%%%%%%%%%%%%%%%%%%%%%%%%%%%%%%%%%%%%
\subsection{Topological Spaces}

\begin{ldef}
A topological space \(\lrangle{X, \tau}\) is \B{separable}
iff there exists \(D\subset X\) countable and dense.
\end{ldef}

\begin{ldef}
A topological space \(\lrangle{X, \tau}\) is called \B{c.c.c.} 
(Countable Chain Condition) iff there are no more than \(\aleph_0\)
disjoint open sets.
\end{ldef}

Separability implies c.c.c. But the opposite direction depends
on conjectures.

Using Tychonoff product:
\begin{itemize}
\item Product of c.c.c spaces is c.c.c.
\item Product of separable spces is not necessarily separable.
\end{itemize}

\begin{thm}
Let \lrangle{X_i,\tau_i}, \(i\in I\) be \ccc\ topological spaces.
Assume that any finite product is \ccc\ then
the Tychonoff product \lrangle{\Pi X_i, =\Pi \tau_i} is \ccc.
\end{thm}
\begin{proof}
We will Specify the  topology of the product
\begin{equation*}
X = \Pi_{i\in I} = \{f | 
  f: I \to\mathsmaller{\bigcup_{i\in I}} X_i
  \quad\land\quad \forall i\in I, f(i)\in X_i\}.
\end{equation*}
The base neighborhoods are defined, for \(\{i_1,\ldots,i_n\}\)
for any \(i \in \{i_1,\ldots,i_n\}\)
we choose \(U_i \in \tau_i\) and get
\begin{equation*}
 U = U_{i_1} \times U_{i_2} \times \cdots \times U_{i_n}.
\end{equation*}
\begin{equation*}
 V_U = \{f \in \pi_{i\in I}| \forall i \in \{i_1,\ldots,i_n\}, f(i)\in U_i\}.
\end{equation*}
By negation, assume that the product is not \ccc.
Then at least \(\aleph_1\) disjoint non-empty sets.
Then at least \(\aleph_1\) base non-empty open sets:
\begin{equation*}
 \{W_\beta | \beta < \aleph_1\}.
\end{equation*}
For each \(\beta\) there exists finite set
\(\{i_1^\beta, i_2^\beta, \ldots, i_{n_\beta}^\beta\}\)
and open sets \(\{U_{i_\beta}^\beta\}\)
with ??  \(\{a_\beta | \beta < \aleph_1\}\)
\(\aleph_1\) finite subsets of $I$.
Then we have \(J \subseteq \aleph_1\) with \(|J|=\aleph_1\)
and a set $a$ such that for any \(i,j\in J\), 
if \(i\neq j\) then \(a_i \cap a_j = a\).
\end{proof}

% Delta System ??
Let \(a = \{x_1,\ldots,x_k\}\). Our assumption is that
\begin{equation*}
X_{x_1} \times X_{x_2} \times \cdots X_{x_k}
\end{equation*}
is \ccc.
\begin{equation*}
U_{x_1}^\beta \times U_{x_2}^\beta \times \cdots U_{x_k}^\beta \qquad (\beta \in J)
\end{equation*}
Form \ccc\ there exist \(\beta,\gamma \in J\), \(\beta\neq\gamma\)
such that \(W = W^\beta \cap W^\gamma \neq \emptyset\)
open in the finite product.

Define \(f\in \Pi_{i\in I} X_i\)
\begin{equation*}
\begin{array}{lll}
x_1,\ldots,x_n & 
  e_1^\beta,\ldots,e_{k_\beta}^\beta & e_1^\gamma,\ldots,e_{k_\gamma}^\gamma
\\
a\;\textnormal{kernel} & 
  a_\beta\; \textnormal{kernel} & a_\gamma\;\textnormal{kernel} 
\end{array}
\end{equation*}

The ``problem'' is with the common coordinates, otherwise
it does not matter what is chosen.
\begin{equation*}
 \begin{array}{llll}
 U_{x_1}^\beta \cap U_{x_1}^\gamma, &
 U_{x_2}^\beta \cap U_{x_2}^\gamma, &
 \cdots &
 U_{x_k}^\beta \cap U_{x_k}^\gamma, \\
 x_1 & x_2 & \ldots & x_k
 \end{array}
\end{equation*}

\begin{equation*}
\begin{array}{ll}
a_\beta \setminus\{x_1,\ldots,x_k\}  & a_\gamma \setminus\{x_1,\ldots,x_k\} \\
\rotatebox[origin=c]{90}{\(=\)} & \rotatebox[origin=c]{90}{\(=\)} \\
\{i_1^\beta, \ldots, i_{k_\beta}^\beta\} & \{i_1^\gamma, \ldots, i_{k_\gamma}^\gamma \}
\end{array}
\end{equation*}

\end{document}
