% -*- latex -*-

%%%%%%%%%%%%%%%%%%%%%%%%%%%%%%%%%%%%%%%%%%%%%%%%%%%%%%%%%%%%%%%%%%%%%%%%
%%%%%%%%%%%%%%%%%%%%%%%%%%%%%%%%%%%%%%%%%%%%%%%%%%%%%%%%%%%%%%%%%%%%%%%%
%%%%%%%%%%%%%%%%%%%%%%%%%%%%%%%%%%%%%%%%%%%%%%%%%%%%%%%%%%%%%%%%%%%%%%%%
\chapterTypeout{Integration on Product Spaces} % 8


%%%%%%%%%%%%%%%%%%%%%%%%%%%%%%%%%%%%%%%%%%%%%%%%%%%%%%%%%%%%%%%%%%%%%%%%
%%%%%%%%%%%%%%%%%%%%%%%%%%%%%%%%%%%%%%%%%%%%%%%%%%%%%%%%%%%%%%%%%%%%%%%%
\section{Notes}

%%%%%%%%%%%%%%%%%%%%%%%%%%%%%%%%%%%%%%%%%%%%%%%%%%%%%%%%%%%%%%%%%%%%%%%%
\subsection{Product of Complex Measures} \label{subsec:prod:complex:measures}

Following Theorem~8.6 that assumes the measures are \(\sigma\)-finite
which also implies that thy are positive, there is 
a definition of products \(\mu\times\lambda\) with the similar assumptions.
We need to generalize the notion for complex measures as well.
Given a measure \(\mu\) we can define the following decomposition
based on 
\index{Jordan Decomposition}
Jordan Decomposition.
\begin{alignat*}{2}
\mu_{r} &= \Re(\mu) \qquad& \mu_{i} &= \Im(\mu) \\
\mu_{r+} &= (|\mu_r|+\mu_r)/2  \qquad& \mu_{i+} &= (|\mu_i|+\mu_i)/2 \\
\mu_{r-} &= (|\mu_r|-\mu_r)/2  \qquad& \mu_{i-} &= (|\mu_i|-\mu_i)/2
\end{alignat*}
Now by distributive law for measures we get the desired expected generalization
of the definition of product space to complex (finite!) measures.


%%%%%%%%%%%%%%%%%%%%%%%%%%%%%%%%%%%%%%%%%%%%%%%%%%%%%%%%%%%%%%%%%%%%%%%%
\subsection{Young's Inequality} \label{subsec:young:ineq}

We bring here a proof of 
\index{Young!inequality}
Young's inequality based on lecture notes from \cite{Viaclovsky:18125:lec20}.

\begin{lthm}
Let \(p, q, r \in[1,\infty]\) such that
\begin{equation*}
\frac{1}{r} = \frac{1}{p} + \frac{1}{q} - 1
\end{equation*}
if \(f \in L^p(\R)\) and \(g\in L^q(\R)\), 
then \(f \ast g\) exists \aded\ and \(f\ast g\in L^r(\R)\). 
Moreover,
\begin{equation*}
\|f\ast g\|_r \leq \|f\|_p \cdot \|g\|_q\,. \label{eq:young:ineq}
\end{equation*}
\end{lthm}


\begin{thmproof}
If \(f=0\,\aded\) or \(g=0\,\aded\) then the inequality is trivial.
Without loss of generality, let \(\|f\|_p = \|g\|_q = 1\)
Since otherwise we can look at 
\(f/\|f\|_p\) and \(g/\|g\|_q\) instead. The general
case follows from the nonnegative
case, so assume \(f, g \geq 0\). 
Let \(p'\) and \(q'\) be the exponential conjugate of $p$ and $q$ respectably.
Using \(1/r+1/q'+1/p'=1\) and 
applying H\"older’s inequality Theorem~3.5,
\begin{align*}
(f\ast g)(x)
&= \int_{\R}\left(f(y)^{p/r}g(x-y)^{q/r}\right)f(y)^{1-p/r}g(x-y)^{1-q/r}\,dy \\
&\leq 
  \left(\int_{\R} f(y)^pg(x-y)^q\,dy\right)^{1/r}
     \left(\int_{\R} f(y)^{(1-p/r)q'}\,dy\right)^{1/q'}
        \left(\int_{\R} f(x-y)^{(1-q/r)p'}\,dy\right)^{1/p'}\,.
\end{align*}
Since 
\begin{equation*}
(1-p/r)q' = p \qquad\textnormal{and}\qquad (1-q/r)p' =q
\end{equation*}
we have
\begin{equation*}
(f\ast g)(x) 
\leq \left(\int_{\R} f(y)^pg(x-y)^q\,dy\right)^{1/r}\cdot1\cdot1.
\end{equation*}
Hence
\begin{equation*}
(f\ast g)^r(x) \leq \int_{\R} f(y)^pg(x-y)^q\,dy
\end{equation*}
That is \((f\ast g)^r \leq f^p \ast g^q\).
Now
\begin{align}
\|f\ast g\|_r^r
&= \int_{\R} (f\ast g)^r(x)\,dx \notag \\
&\leq  \int_{\R} (f^p \ast g^q)(x)\,dx = \|(^p \ast g^q\|_1 \notag \\
&\leq \|f^p\|_1 \|g^q\|_1 \label{eq:young:conv1} \\
&=  \|f\|_p^p \cdot \|g\|_q^q = 1. \notag
\end{align}
The inequality~\eqref{eq:young:conv1} is given by Theorem~8.14.
Thus desired \eqref{eq:young:ineq} was shown.
\end{thmproof}

%%%%%%%%%%%%%%%%%%%%%%%%%%%%%%%%%%%%%%%%%%%%%%%%%%%%%%%%%%%%%%%%%%%%%%%%
%%%%%%%%%%%%%%%%%%%%%%%%%%%%%%%%%%%%%%%%%%%%%%%%%%%%%%%%%%%%%%%%%%%%%%%%
\section{Exercises} % pages 174-177

%%%%%%%%%%%%%%%%%
\begin{enumerate}
%%%%%%%%%%%%%%%%%


%%%%%%%%%%%%%% 1
\begin{excopy}
Find a monotone class \frakM\ in \(\R^1\) which is not a \salgebra,
even though \(\R^1 \in \frakM\) and \(\R^1\setminus A \in \frakM\)
for every \(A\in \frakM\).
\end{excopy}

Take
\begin{equation*}
\frakM = \bigl\{\emptyset, \R^1, \; 
                \{0\}, \R^1\setminus\{0\}, \;
                \{1\}, \R^1\setminus\{1\}  \bigr\}.
\end{equation*}
Note that it is a monotone class but \(\{0,1\}\notin \frakM\).


%%%%%%%%%%%%%% 2
\begin{excopy}
Suppose $f$ is a Lebesgue measurable nonnegative real function on \(\R^1\)
and \(A(f)\) is the 
\emph{ordinate set}
\index{ordinate set}
of $f$. This is the set of all points \((x,y)\in\R^2\) for which \(0<y<f(x)\).
\begin{itemize}
\itemch{a} Is it true that \(A(f)\) is Lebesgue measurable, 
           in the two dimensional sense?
\itemch{b} If the answer to \ich{a} is affirmative, 
           is the integration of $f$ over \(\R^1\) 
           equal to the measure of \(A(f)\)?
\itemch{c} Is the graph of $f$ a measurable subsection of \(\R^2\)?
\itemch{d} If the answer to \ich{c} is affirmative, 
           is the measurable of the graph equal to zero?
\end{itemize}
\end{excopy}

\begin{itemize}
\itemch{a}
Yes.
Define simple functions
\begin{equation*}
f_n(x) = \lfloor nf(x)\rfloor / n \qquad (n\geq 1).
\end{equation*}
Clearly \(\lim_{n\to\infty} f_n = f\).
Since \(A(f_n)\) is a countable union of rectangles, it is measurable.
So is \(A(f) = \cup_n A(f_n)\).

\itemch{b}
Yes. Using the notations if \ich{a}, we have 
\begin{equation*}
m_2\bigl(A(f_n)\bigr) 
= \sum_{k=0}^\infty (k/n)\cdot m\bigl(\{x\in\R^1: f_n(x) = k/n\}\bigr)
= \int_{\R^1} f_n(t)\,dm(t).
\end{equation*}
The desired equality is derived 
by Lebesgue monotone convergence theorem.

\itemch{c}
Yes, see \ich{d}.

\itemch{d}
Yes.
Pick \(\epsilon>0\). Define
\begin{equation*}
s(x) = \epsilon \cdot 2 ^{-\lfloor x \rfloor} > 0
\end{equation*}
Clearly 
\begin{equation*}
\int_{\R} s(x)\,dx = 2 \epsilon \sum_{n=0}^\infty 2^{-n} = 4\epsilon.
\end{equation*}
Now 
\begin{equation*}
G 
= \{(x,f(x)): x\in\R\} 
\subset \{(x,y)\in\R^2: f(x)-s(x) < y < f(x)+s(x)\}.
\end{equation*}
and 
\begin{align*}
m(G) 
&\leq m\bigl(\{(x,y)\in\R^2: f(x)-s(x) < y < f(x)+s(x)\}\bigr) \\
&=  m\bigl(\{(x,y)\in\R^2: 0 < y < 2s(x)\}\bigr)
= 2\int_R s(x)\,dx = 8\epsilon.
\end{align*}
Since \(\epsilon\) is arbitrary, we have \(m(G)=0\).
\end{itemize}


%%%%%%%%%%%%%% 3
\begin{excopy}
Find an example of a positive continuous function $f$
in the open unit square in \(\R^2\),
whose integral (relative to Lebesgue measure)
is finite but such that \(\varphi(x)\) 
(in the notation of Theorem~8.8)
is infinite for some \(x \in (0,1)\).
\end{excopy}

Let \(U = (0,1)^2 \subset \R^2\) be the open unit square
and \(C = (1/2,0)\) a point on its boundary.
For \(n\geq 1\) define the open triangle
\begin{equation*}
T_n = \triangle\bigl(C-(1/n, 0),\; C+(1/n, 0),\; C+(0, 1/n)\bigr).
\end{equation*}
Clearly \(T_n \supset T_{n+1}\) is a decreasing sequence, 
\(m(T_n) = 1/n^2\) and \(\cap_n T_n = \emptyset\).
By  Urysohn's Lemma~2.12 \cite{RudinRCA87}, 
\index{Urysohn's lemma}
there exists
\(f_n:U\to[0,1]\) such that 
\begin{equation*}
U \setminus T_n \prec f_n \prec T_{n+1}.
\end{equation*}
Let \(f = \sum_{n=1}^\infty f_n\).
Now
\begin{equation*}
\int_U f(x,y)\,dm(x,y)
= \sum_{n=1}^\infty \int_U f_n(x,y)\,dm(x,y)
\leq \sum_{n=1}^\infty \int_U \chhi_{T_n}(x,y)\,dm(x,y)
= \sum_{n=1}^\infty 1/n^2
< \infty.
\end{equation*}
But for \(x=1/2\)
\begin{align*}
\int_0^1 f_x(y)\,dy 
&= \int_0^1 f(1/2,y)\,dy
= \sum_{n=1}^\infty \int_0^1 f(1/2,y)\,dy \\
&\geq \sum_{n=1}^\infty \int_0^1 \chhi_{T_{n+1}}(1/2,y)\,dy
= \sum_{n=1}^\infty 1/(n+1)
= \infty.
\end{align*}
 

%%%%%%%%%%%%%% 4
\begin{excopy}
Suppose \(1\leq p \leq \infty\), \(f\in L^1(\R)\) and \(g\in L^p(\R)\).
\begin{itemize}
\itemch{a}
Imitate the proof of Theorem~8.14 to show that the integral defining
\((f\ast g)(x)\) exists for almost all $x$, that \(f\ast g \in L^p(\R)\),
and that 
\begin{equation*}
\| f \ast g \|_p \leq \|f\|_1 \|g\|_p.
\end{equation*}

\itemch{b}
Show that equality can hold in \ich{a} if \(p=1\) and if \(p=\infty\),
and find the conditions under which this happens.

\itemch{c}
Assume \(1 < p < \infty\) and equality holds in \ich{a}.
Show that then either 
\(f=0\;\aded\) or 
\(g=0\;\aded\)

\itemch{d}
Assume \(1\leq p \leq \infty\), \(\epsilon>0\), and show that there exists
\(f\in L^1(\R^1)\) and \(g\in L^p(\R^1)\) such that 
\begin{equation*}
\| f \ast g \|_p > (1-\epsilon) \|f\|_1 \|g\|_p.
\end{equation*}
\end{itemize}
\end{excopy}

For a generalization with 
\index{Young}
Young inequalities, see \cite{EdwFA} Theorem~9.5.1 page~655.

\begin{itemize}
\itemch{a}
Let's first assume \(p=\infty\), then
\begin{align*}
\|f \ast g\|_\infty
&= \esssup_{x\in\R} \left| \int_{\R} f(y)g(x-y)\,dy\right|
\leq \esssup_{x\in\R}  \int_{\R} |f(y)g(x-y)|\,dy \\
&\leq \|g\|_\infty \esssup_{x\in\R} \int_{\R} |f(y)|\,dy
 = \|f\|_1 \|g\|_\infty
\end{align*}

So now we may assume \(p<\infty\). We use 
% generalization of
 H\"older's inequality 
\index{Holder@H\"older}
Theorem~3.5 \cite{RudinRCA87}
% (see local lemma~\ref{llem:hlp:188}).
Let \(q = p / (p - 1)\) be the exponent conjugate.
\begin{align}
|(f \ast g)(t)|
&\leq \int_{\R} |f(t-s)g(s)|\,ds \notag \\
&= \int_{\R} 
    \left(|f(t-s)|^{1/p} |g(s)|\right) 
    \left(|f(t-s)|^{1/q}\right) 
    \,ds \notag \\
&\leq \label{eq:ex8.4a:holder}
      \left( \int_{\R} \left(|f(t-s)|^{1/p} |g(s)|\right)^p\,ds \right)^{1/p}
      \left( \int_{\R} \left(|f(t-s)|^{1/q}\right)^q \,ds \right)^{1/q} \\
&=   \left( \int_{\R} |f(t-s)| \cdot |g(s)|^p\,ds \right)^{1/p}
      \left( \int_{\R} |f(t-s)| \,ds \right)^{1/q} \notag \\
&=   \left( \int_{\R} |f(t-s)| \cdot |g(s)|^p\,ds \right)^{1/p} \|f\|_1^{1/q}
     \notag
\end{align}
The above inequality holds for almost all $t$. Hence
\begin{align}
\|f \ast g\|_p^p
&\leq \|f\|_1^{p/q} \notag
      \int_\R 
        \left(\int_{\R} |f(t-s)| \cdot |g(s)|^p\,ds \right)^{(1/p)p}\,dt \\
&= \|f\|_1^{p/q} \notag
   \int_\R \int_{\R} |f(t-s)| \cdot |g(s)|^p\,ds\,dt \\
&= \|f\|_1^{p/q} \label{eq:8.4:fub}
   \int_\R \int_{\R} |f(t-s)| \cdot |g(s)|^p\,dt\,ds \\
&= \|f\|_1^{p/q} \notag
   \int_\R |g(s)|^p \int_{\R} |f(t-s)| \,dt\,ds \\
&= \|f\|_1^{p/q} \cdot \|g\|_p^p \cdot \|f\|_1  \notag
 = \|f\|_1^{p/(p/(p-1)) + 1}  \|g\|_p^p \\
&=  \|f\|_1^p \cdot \|g\|_p^p  \notag
\end{align}
The \eqref{eq:8.4:fub} equality is by Fubini Theorem~8.8 \cite{RudinRCA87}.
Thus \(\|f \ast g\|_p =  \|f\|_1 \|g\|_p\) as desired.

\itemch{b}
\emph{Note:} This is a result of 
\index{Riesz-Thorin}
\index{Thorin}
Riesz-Thorin theorem.

% Clearly if \(f=0\,\aded\) or \(g=0\,\aded\) then equality holds.
Let \(h = f\ast g\).
By the proof of Theorem~8.14, when \(p=1\), there is an equality iff
\begin{equation*}
|h(x)| 
= \left|\int_{-\infty}^\infty f(x-y)g(y)\,dy\right|
= \int_{-\infty}^\infty |f(x-y)g(y)|\,dy
\end{equation*}
This happens iff \(f(x-y)g(y)\) 
has the same argument almost everywhere on~$x$ and~$y$.
Equivalently, since \(x-y\) covers all of \(\R\), iff
$f$ and $g$ wach has constant argument \aded.

When \(p=\infty\) then equality holds 
if for any \(\epsilon>0\) there exists some \(x\in\R\) 
and \(\theta\in[0,2\pi]\)
such that 
\begin{equation*}
\int_{\R}\left| f(y)g(x-y)- e^{i\theta}|f(y)|\cdot\|g\|_\infty\right|\,dx 
< \epsilon.
\end{equation*}


\itemch{c}
Assme \(1<p<\infty\).
If equality holds, then equality must hold for almost all \(t\in\R\) in 
\eqref{eq:ex8.4a:holder}.
By Local~Lemma~\ref{llem:hlp:188} (H\"older), both expressions:
\begin{equation*}
|f(t-s)|^{1/p} |g(s)|  \qquad  |f(t-s)|^{1/q}
\end{equation*}
must be effectively proportional for all \(t\in\R\).
Equivalently, 
\begin{equation*}
|g(s)|  \qquad  |f(t-s)|^{1/q-1/p}
\end{equation*}
must be effectively proportional. 
Hence \(f=0\,\aded\) or \(g=0\,\aded\).

\itemch{d}
Let \(f_n(x) = n\chhi_{[-1/n,+1/n]}/2\) and \(g = \chhi_{[0,1]}\).
Clearly \(\|f_n\|_1=1\) and
\(\|g\|_p = 1\). Now 
\begin{align*}
(f_n\ast g)(x) 
&= \int_{-\infty}^\infty f_n(x-t)g(t)\,dt
 = \int_0^1 f_n(x-t)\,dt
 = (n/2) \int_0^1 \chhi_{[-1/n,+1/n]}(x-t)\,dt \\
&= \left\{\begin{array}{ll}
    n(x+1/n)/2 \qquad &  -1/n \leq x \leq 1/n \\
    1    \qquad    1/n \leq x \leq 1-1/n \\
    n(x-(1-1/n))/2 \qquad &  -1/n \leq x \leq 1/n \\
    0 \qquad              & \textnormal{otherwise}
   \end{array}\right.
\end{align*}
By Lebesgue convergence theorems
\(\lim_{n\to\infty} \|f_n\ast g\|_p = 1\).
Hence for any \(\epsilon>0\) we can find some $n$ such that 
\begin{equation*}
\| f_n \ast g \|_p > 
(1-\epsilon) =
(1-\epsilon) \|f\|_1 \|g\|_p.
\end{equation*}
\end{itemize}


%%%%%%%%%%%%%% 5
\begin{excopy}
Let $M$ be the Banach space of all complex Borel measures on \(\R^1\).
The norm in $M$ is \(\|\mu\| = |\mu|(\R^1)\).
Associate to each Borel set \(E \subset \R^1\) the set
\begin{equation*}
E_2 = \{ (x,y): x+y \in E\} \subset \R^2.
\end{equation*}
If \(\mu\) and \(\lambda \in M\) define their convolution \(\mu \ast \lambda\)
to be the set function given by
\begin{equation*}
(\mu \ast \lambda)(E) = (\mu \times \lambda)(E_2)
\end{equation*}
for every Borel set \(E\subset \R^1\);
\(\mu\times\lambda\) is as in Definition~8.7.
\begin{itemize}
\itemch{a}
Prove that \(\mu \ast \lambda \in M\) and that 
\(\|\mu \ast \lambda\| \leq \|\mu\| \, \|\lambda \|\).

\itemch{b}
Prove that \(\mu \ast \lambda \) is the unique \(\nu\in M\) such that 
\begin{equation*}
\int f\,d\nu = \int\int f(x+y)\,d\mu(x)\,d\nu(y)
\end{equation*}
for every \(f\in C_0(\R^1)\). (all integrals extend over \(\R^1\).)

\itemch{c}
Prove that convolution in $M$ is commutative, associative, and distributive
with respect to addition.

\itemch{d}
Prove the formula
\begin{equation*}
(\mu \ast \lambda)(E) = \int \mu(E-t)\,d\lambda(t)
\end{equation*}
for every \(\mu\) and \(\lambda \in M\) and every Borel set $E$. Here
\begin{equation*}
E - t = \{x - t: x\in E\}.
\end{equation*}

\itemch{e}
Define \(\mu\) to be 
\emph{discrete} 
\index{discrete!measure} 
if \(\mu\) is concentrated on a countable set;
define \(\mu\) to be 
\emph{continuous}
\index{continuous!measure}
if \(\mu(\{x\}) = 0\) for every \(x\in\R^1\);
let $m$ be Lebesgue measure on \(\R^1\)
(note that \(m\notin M\)).
Prove that  \(\mu \ast \lambda\) is discrete if both \(\mu\) and \(\lambda\)
are discrete, that \(\mu \ast \lambda\) is continuous if \(\mu\) is continuous
and \(\lambda\in M\), and that \(\mu \ast \lambda \ll m \) if \(\mu \ll m\).

\itemch{f}
Assume
\(d\mu = f\,dm\), 
\(d\lambda = g\,dm\),
\(f\in L^1(\R^1)\), 
\(g\in L^1(\R^1)\), 
and prove that \\
\(d(\mu\ast\lambda) = (f\ast g)\,dm\).

\itemch{g}
Properties \ich{a} and \ich{c} show that the Banach space $M$ is what one calls
\index{commutative!Banach algebra}
\emph{commutative Banach algebra}.
Show that \ich{e} and \ich{f} imply that the set of all discrete measures in $M$
is a subalgebra of $M$,
that the continuous measures form an ideal in $M$, and that the absolutely
continuous measures (relative to $M$) form an ideal in $M$ which is isomorphic 
(as an algebra) to \(L^1(\R^1)\).

\itemch{h}
Show that $M$ has a unit, i.e., show that there exists a \(\delta\in M\)
such that \(\delta \ast \mu = \mu\) for all \(\mu \in M\).

\itemch{i}
Only two properties of \(\R^1\) have been used in this discussion: \(\R^1\)
is a commutative group (under addition), and there exists a translation
invariant Borel measure $M$ on \(\R^1\) which is not identically $0$ and 
which is finite on all compact subsets of \(\R^1\).
Show that the same results hold if \(\R^1\)
is replaced by \(\R^k\) or by $T$ (the unit circle) or by \(T^k\)
(the $K$-dimensional torus, the cartesian product of $k$
copies of $T$),
as soon as the definitions are properly formulated.
\end{itemize}
\end{excopy}

\emph{Note:} The definition of \(\mu\times\lambda\) where 
\(\mu\) and \(\lambda\) are complex measures requires a generalization
of the definition done in section~8.7 of \cite{RudinRCA87},
see \ref{subsec:prod:complex:measures} above.

Utilizing the notations of Theorem~8.6 gives:
\begin{align*}
(E_2)_t &= \{u\in\R: (t,u)\in E_2\} = \{u\in\R: t+u\in E\} 
    = \{x-t: x\in E\} = E-t \\
(E_2)^u &= \{t\in\R: (t,u)\in E_2\} = \{t\in\R: t+u\in E\} 
    = \{y-u: y\in E\} = E-u.
\end{align*}

\begin{itemize}
\itemch{a}
Let \(\frakM\) be the set of all Borel sets \(E\subset \R\)
such that \(E_2\) is a Borel set in \(\R^2\).

% Assume \(E\subset\R\) is a Borel set. We need to whow that 
% \(E_2\) is a Borel set in \(\R^2\).x
% We will do this in steps.

% First we assume that 
If $E$ is open, then it is trivial to see that~\(E_2\) is open. 
If \(E\in\frakM\) then 
\begin{equation*}
C_2 
= \{(x,y)\in\R^2: x+y \in \R\setminus E\}
= \R^2 \setminus \{(x,y)\in\R^2: x+y \in E\}
= \R^2 \setminus E_2\,,
\end{equation*}
hence \((\R\setminus E)\in\frakM\).

Let \(E = \cup_{n\in\N}B_n\) a Partition of Borel sets
and assume \(B_n\in\frakM\) for all \(n\in\N\), then
\begin{equation*}
E_2 
= \{(x,y)\in\R^2: x+y \in \cup_{n\in\N}B_n\}
= \bigcup_{n\in\N}B_n \{(x,y)\in\R^2: x+y \in B_n\}
\end{equation*}
hence \(E\in\frakM\).
Therefore \frakM\ is a \salgebra\ that contains the open sets,
and so it contains all the Borel sets.


Now we have to show that \(\mu\ast\lambda\) is a measure.
Let \(E=\disjunion_{n\in\N} B_n\) a partition of Borel sets.
\begin{align*}
(\mu\ast\lambda)(E)
&= (\mu\times\lambda)\left(\{(x,y)\in\R^2: x+y\in\disjunion_{n\in\N} B_n\}\right)
  \\
&= \sum_{n\in\N}(\mu\times\lambda)\left(\{(x,y)\in\R^2: x+y \in B_n\}\right) \\
&= \sum_{n\in\N}(\mu\ast\lambda)(B_n)
\end{align*}

We note that if \(A,B\subset\R\) are disjoint, then
\(A_2,B_2\subset\R^2\) are disjoint as well. Hence
\begin{align*}
\left|\mu\ast\lambda\right|
&= |\mu\ast\lambda|(\R)
 = \sup_{\R=\disjunion_{n\in\N} F_n} 
   \left(\sum_{n\in\N} \left| (\mu\ast\lambda) (F_n)\right|\right)
 = \sup_{\R=\disjunion_{n\in\N} F_n} 
   \left(\sum_{n\in\N} \left| (\mu\times\lambda) (F_n)_2\right|\right)
   \\
&\leq (|\mu|\times|\lambda|)(\R^2) 
 = \|\mu\|\cdot\|\lambda\|.
\end{align*}
The last equality is a trivial applying of 
\index{Fubini}
Fubini's Theorem~8.8.

\itemch{b}
The Borel measure is determined by its value on open sets.
In the case of \(\R\) therefore, it is determined by the value on intervals.
But for any interval $I$ we can aproximate \(\chhi_I\) by functions
from \(C_0(\R)\). Thus the uniqueness follows.

\itemch{d}
\emph{Note:} before \ich{c} since we will need this one there.
First we have the set manipulation
\begin{equation*}
(E_2)^y 
= \{x\in\R: (x,y)\in E_2\}
= \{x\in\R: x+y\in E\}
= \{t-y\in\R: t\in E\}
= E - y.
\end{equation*}
Now
\begin{equation*}
(\mu \ast \lambda)(E) 
= (\mu\times\lambda)(E_2) 
= \int_{\R} \mu\left(E_2)^t\right)\,d\lambda(t)
= \int_{\R} \mu(E-t)\,d\lambda(t)
\end{equation*}

\itemch{c}
Let \(\lambda,\mu,\nu\in M\) and a Borel set \(E\subset\R\),
and \(E_2\) defined as above.

\paragraph{Convolution is commutative.}
\begin{equation*}
(\lambda\ast \mu)(E)
= (\lambda\times\mu)(E)
= (\mu\times\lambda)(E)
= (\mu\ast\lambda)(E)
\end{equation*}

We note the set equality:
\begin{gather*}
(E^y)_2 = (E-y)_2 
 = \{(t,u)\in\R^2: t+u\in E-y\}
 = \{(t,u)\in\R^2: t+u+y\in E\} \\
%
\begin{align*}
\bigl((E-y)_2\bigr)^s 
&= \{r\in\R: (r,s)\in (E-y)_2\}
 = \{r\in\R: r+s\in E-y\} \\
&= \{r\in\R: r+s+y\in E\}
 = E-(y+s)
\end{align*}
\end{gather*}

\paragraph{Convolution is associative.}
Using Theorem~8.6 and its notations
\begin{align*}
\bigl((\lambda\ast \mu)\ast \nu\bigr)(E)
&=\bigl((\lambda\ast \mu)\times \nu\bigr)(E_2)
 % = \int_{\R} \nu\bigl((E_2)_x\bigr)\,d(\lambda\ast \mu)(x) \\
 = \int_{\R} (\lambda\ast \mu)\bigl((E_2)^y\bigr)\,d\nu(y) \\
&= \int_{\R} (\lambda\ast \mu)(E-y)\,d\nu(y) 
 = \int_{\R} (\lambda\times \mu)\bigl((E-y)_2\bigr)\,d\nu(y) \\
&= \int_{\R} 
   \left(
      \int_{\R}\lambda\left(\bigl((E-y)_2\bigr)^s\right)\,d\mu(s)
   \right)\,d\nu(y) \\
&= \int_{\R} \left(\int_{\R}\lambda(E-y-s)\,d\mu(s)\right)\,d\nu(y)
\end{align*}

Summerizing the above
\begin{equation} \label{eq:measure:conv:assoc}
\bigl((\lambda\ast \mu)\ast \nu\bigr)(E)
= \int_{\R} \left(\int_{\R}\lambda(E-y-s)\,d\mu(s)\right)\,d\nu(y)
\end{equation}

We now use commutativity and \eqref{eq:measure:conv:assoc} twice:
\begin{equation*}
\bigl((\lambda\ast \mu)\ast \nu\bigr)(E)
\bigl((\mu\ast \lambda)\ast \nu\bigr)(E)
= \int_{\R} \left(\int_{\R}\mu(E-y-s)\,d\lambda(s)\right)\,d\nu(y)
\end{equation*}
and
\begin{equation*}
\bigl(\lambda\ast (\mu\ast \nu)\bigr)(E)
\bigl((\mu\ast \nu)\ast \lambda\bigr)(E)
= \int_{\R} \left(\int_{\R}\mu(E-y-s)\,d\nu(s)\right)\,d\lambda(y)
\end{equation*}
By Fubini's Theorem~8.8 the last integrals of the last two equalities
are equal
\begin{equation*}
  \int_{\R} \left(\int_{\R}\mu(E-y-s)\,d\lambda(s)\right)\,d\nu(y)
= \int_{\R} \left(\int_{\R}\mu(E-y-s)\,d\nu(s)\right)\,d\lambda(y)
\end{equation*}
hence,
\begin{equation*}
  \bigl((\lambda\ast \mu)\ast \nu\bigr)(E) 
= \bigl(\lambda\ast (\mu\ast \nu)\bigr)(E).
\end{equation*}

\paragraph{Convolution is distributive.}
\begin{align*}
\bigl((\lambda+\mu)\ast \nu)(E)
&= \int_{\R} (\lambda+\mu)(E-y)\,d\nu(y)
     \int_{\R} \lambda(E-y)\,d\nu(y)
   + \int_{\R} \mu(E-y)\,d\nu(y) \\
&=  (\lambda+\mu)\ast \nu)(E)
  + (\mu+\mu)\ast \nu)(E)
\end{align*}
The distribution of other side follows by commutativity.

\itemch{e}
\textbf{Discreteness.}
Assume both \(\mu\) and \(\lambda\) are discrete.
Let \(A,B\subset\R\) the countable sets on which the measures are concentrated
respectably. We freely use the notations like \(\mu(\{x\})=\mu(x)\).
Now
\begin{align*}
(\lambda\ast\mu)(E)
&= \int_{\R} \lambda(E-y)\,d\mu(y)
 = \sum_{y\in B} \lambda(E-y)\mu(y)
 = \sum_{y\in B} \left(\sum_{x\in A\cap (E-y)}\lambda(x)\right)\mu(y) \\
&= \sum{x\in A}\sum_{y\in B} \chhi_E(x+y)\lambda(x)\cdot\mu(y)
\end{align*}
Therefore \(\lambda\ast\mu\) is concentrated in \(\{x+y: (x,y)\in A\times B\}\).
and 
\begin{equation*}
(\lambda\ast\mu)(w) = \sum_{\stackrel{(x,y)\in  A\times B}{x+y=w}} \lambda(x)\cdot\mu(y).
\end{equation*}


\textbf{Continuity.}
Assume \(\mu\) is continuous. Pick arbitrary \(x\in\R\).
\begin{equation*}
(\mu\ast\lambda)(x) 
= \int_{\R} \mu(x-y)\,d\lambda(y) = 
= \int_{\R} 0\,d\lambda(y) = 0.
\end{equation*}
Hence \(\mu\ast\lambda\) is continuous.

\textbf{Absolute Continuity.}
Assume \(\mu \ll m\) and \(m(E)=0\). 
Clearly \(\mu(E-y)=0\) for each \(y\in\R\) since \(m(E-y)=0\)
by construction of the Lebesgue measure. Now
\begin{equation*}
(\mu\ast \lambda)(E) 
= \int_{\R} \lambda(E-y)\,d\mu(y)
= \int_{\R} 0\,d\mu(y) = 0
\end{equation*}

\itemch{f}
Take a Borel set $E$. Using the fact that $m$ is a measure that 
is invariant under translation and 
\index{Fubini}
Fubini's Theorem~8.8
\begin{align*}
(\mu\ast\lambda)(E)
&= \int_{\R} \chhi_E\,d(\mu\ast\lambda)
 = \int_{\R} \mu(E-y)\,d\lambda(y)
 = \int_{\R} \left(\int_{E-y}\!f(x)\,dm(x)\right)g(y)\,dm(y) \\
&= \int_{\R} \left(\int_E f(x-y)\,dm(x)\right)g(y)\,dm(y) \\
&= \int_{\R} \left(\int_{\R} \chhi_E(x) f(x-y)g(y)\,dm(x)\right)\,dm(y) \\
&= \int_{\R} \left(\int_{\R} \chhi_E(x) f(x-y)g(y)\,dm(y)\right)\,dm(x) \\
&= \int_{\R} \chhi_E(x) \left(\int_{\R} f(x-y)g(y)\,dm(y)\right)\,dm(x) 
 = \int_{\R} \chhi_E(x) (f\ast g)(x)\,dm(x) \\
&= \int_{\R} \chhi_E(x)\,d(f\ast g)(x)m(x)
\end{align*}


\itemch{g}
Discrete measure are subalgebra by \ich{e} and \ich{f}.
Closure under addition and scalar multiplication is trivial.

Let $C$ be the family of continuous measures.
Again, closure under addition and scalar multiplication is trivial.
By \ich{e} and \ich{f} we have shown that $C$ is an ideal.

Let $A$ be the family of absolutely continuous measures with respect to $m$.
Again, closure under addition and scalar multiplication is trivial.
If \(\mu\ll m\) then by 
\index{Radon-Nikodym}
Radon-Nikodym Theorem~6.10 trivially extended to complex measures,
there exists $m$-measurable $f$
such that \(f\,dm = d\mu\). With this,\ich{e} and \ich{f} $A$ is an ideal
and the mapping \(\mu \to f\) is an isomorphism.


\itemch{h}
Put \(\delta(E) = \chhi_E(0)\). Verification is trivial.

\itemch{i}
For \(E\subset \T^k\) we need to modify the definition 
\begin{equation*}
E_2 = \left\{(u_1,u_2)\in \left(\T^k\right)^2: \exists u\in E,\;
         \forall j\in\N_k,\; (z_1)j\cdot(z_2)_j = u_j\right\}
\end{equation*}
Now all arguments can be easily applied to \(\R^k\) and \(\T^k\).
\end{itemize}


%%%%%%%%%%%%%% 6
\begin{excopy}
(Polar coordinates in \(\R^k\).)
Let \(S_{k-1}\) be the unit sphere in \(\R^k\), 
i.e., the set of all \(u \in \R^k\) whose distance from the origin $0$ is $1$.
Show that every \(x\in \R^k\), except for \(x=0\), has a unique representation
of the form \(x=ru\), where $r$ is a positive real number and \(u\in S_{k-1}\).
Thus \(\R^k\setminus\{0\}\) may be regarded as the cartesian product
\((0,\infty)\times S_{k-1}\).

Let \(m_k\) be the Lebesgue measure on \(\R^k\), and define a measure
\(\sigma_{k-1}\) on \(S_{k-1}\) as follows:
If \(A \subset S_{k-1}\) and $A$ is a Borel set, let \(\overline{A}\)
be the set of all points \(ru\), where \(0 < r < 1\)
and \(u\in A\), and define
\begin{equation*}
\sigma_{k-1}(A) = k \cdot m_k(\overline{A}).
\end{equation*}
Prove that the formula
\begin{equation*}
\int_{R^k} f\,dm_k 
= \int_0^\infty r^{k-1}\,dr \int_{S_{k-1}} f(ru)\,d\sigma_{k-1}(u)
\end{equation*}
is valid for every nonnegative Borel function $f$ on \(\R^{k}\).
Check that this coincides with familiar results 
when \(k=2\)
and when \(k=3\).

\emph{Suggestion}: If \(0 < r_1 < r_2\) and if $A$ 
is an open subset of \(S_{k-1}\), let $E$ be the set of all \(ru\) with
\(r_1 < r < r_2\), \(u\in A\),
and verify that the formula holds for the characteristic function of $E$.
Pass from these to characteristic functions of Borel sets in \(\R^k\).
\end{excopy}

Pick $E$ as suggested. Now
\begin{align*}
\int_{R^k} \chhi_E\,dm_k 
&= m_k(E) 
 = m_k\left(\{ru\in \R^k: u\in A\;\wedge\; r_1<r<r_2\}\right) \\
&=   m_k\left(\{ru\in \R^k: u\in A\;\wedge\; r<r_2\}\right)
   - m_k\left(\{ru\in \R^k: u\in A\;\wedge\; r\leq r_1\}\right) \\
&= m_k(\overline{A})\left(r_2^k - r_1^k\right)
 = k\int_{r_1}^{r_2} r^{k-1}m(\overline{A})\,dr \\
&= \int_{r_1}^{r_2} r^{k-1}\sigma_{k-1}(A)\,dr
 = \int_{r_1}^{r_2} r^{k-1}\left(\int_A 1\,d\sigma_{k-1}(u)\right)\,dr \\
&= \int_0^\infty r^{k-1}\,dr \int_{S_{k-1}} \chhi_E(ru)\,d\sigma_{k-1}(u)
\end{align*}
Thus the desired equality holds for \(f=\chhi_E\).

Now every Borel function \(f\geq 0\) can be monotonically approximated
by simple functions of the forum
\begin{equation*}
s(x) = \sum_{j\in\N} a_j\chhi_{E_j}
\end{equation*}
where \(a_j\geq 0\) and \(E_j\) are sets of the suggested type.
Note: For such sets, \(A_j\) sets can be picked so they are intersection
of \(B_k(u;\epsilon)\) with \(u\in S_{k-1}\) and \(\epsilon>0\).
By Lebesgue's monotone~ convergence Theorem~1.34
the equality holds for non-negative Borel functions.


%%%%%%%%%%%%%% 7
\begin{excopy}
Suppose \((X,\calG, \mu)\)
and \((Y,\calF, \lambda)\)
are finite measure spaces, and suppose \(\psi\) is a measure on
\(\calG\times\calF\) such that 
\begin{equation*}
\psi(A\times B) = \mu(A) \mu(B)
\end{equation*}
whenever \(A\in\calG\) and \(B\in\calF\).
Prove that then \(\psi(E) = (\mu\times\lambda)(E)\) 
for every \(E\in \calG\times\calF\).
\end{excopy}

Let \calE\ be the family if sets $E$ for which the desired equality holds.
Clearly all the elementary sets are in \calE\ and it is monotonic.
By Theorem~8.3  \(\calG\times\calF \subset \calE\).


%%%%%%%%%%%%%% 8
\begin{excopy}
\begin{itemize}
\itemch{a}
Suppose $g$ is a real function on \(\R^k\) such that each section \(f_x\)
is a Borel measurable and each section \(f^y\) is continuous.
Prove that $f$ is Borel measurable on \(\R^2\).
Note the contrast between this and Example~8.9(c).
\itemch{b}
Suppose $g$ is a real function on \(\R^k\) which is continuous in each 
of the $k$ variables separately.
More explicitly, for every choice of \(x_2,\ldots\,x_n\), the mapping
\(x_1 \to g(\seqn{x})\) is continuous, etc.
Prove that $g$ is a Borel function.
\end{itemize}
\emph{Hint}: If \((i-1)/n = a_{i-1} \leq x \leq a_i = i/n\), put
\begin{equation*}
f_n(x,y) = 
\frac{a_i - x}{a_i - a_{i-1}} f(a_{i-1}, y) 
+
\frac{x - a_{i-1}}{a_i - a_{i-1}} f(a_i, y).
\end{equation*}
\end{excopy}

\begin{itemize}
\itemch{a}
We first look at the function \(g_2:\R^2\to\R^2\) defined by
\(g_2(x,y) = (x,f_a(y))\).
Pick a base open set 
\begin{equation*}
G = (u-\delta_x,u+\delta_x)\times(v-\delta_y,u+\delta_y) \subset \R^2.
\end{equation*}
Now
\begin{equation*}
g_2^{-1}(G) = (u-\delta_x,u+\delta_x)\times f_a^{-1}(v-\delta_y,u+\delta_y).
\end{equation*}
which is a cartesian product of two Borel sets, hence
\(g_2^{-1}(G)\) is a Borel set and \(g_2\) is a Borel function.
By Theorem~1.12\ich{d} \(xg_2(x,y) = x\cdot f_a(x,y)\) are Borel
functions for all \(a\in\R\).
Hence the functions \(f_n\) (defined in the hint) are Borel functions.

Clearly for \(\lim_n f_n(x,y) = f(x,y)\) for all \((x,y)\in\R^2\).
If \(G\subset \R^2\) is any open set then
\begin{equation*}
f^{-1}(G) = \bigcup_{m\in\N} \left(\bigcap_{n\geq m} f_n^{-1}(G)\right)
\end{equation*}
Since each \(f_n^{-1}(G)\) is a Borel set, 
by being a \salgebra\ \(f^{-1}(G)\) is a Borel set. 
Therefore $f$ is Borel measurable.

\itemch{b}

Just for comparison with continuity, recall the following example:
\begin{equation*}
f(x,y) = \left\{
\begin{array}{ll}
0 & \textnormal{if}\; x= y = 0 \\
\frac{xy}{x^2+y^2}  \quad & \textnormal{otherwise}
\end{array}\right.
\end{equation*}
Now 
\begin{align*}
\lim_{x\to 0} f(x,0) &= 0 \\
\lim_{y\to 0} f(0,y) &= 0 \\
\lim_{x\to 0} f(x,ax) &= \frac{a}{1+a^2} 
   = \left(\frac{1}{a}\right)\,
     \bigm/\,
     \left(1+\left(\frac{1}{a}\right)^2\right)
     \qquad (a\neq 0)
\end{align*}

By induction. If \(k=1\) then trivially $g$ is continuous,
and every continuous function is also a Borel function.
Now assume that \(k=n\) and that the claim holds for all \(k<n\).
For each fixed \(x_n\) the function \(\tilde{g}:\R^{n-1}\to\R\) defined by 
\begin{equation*}
\tilde{g}(x_1,x_2,\ldots,x_{n-1}) = g(x_1,x_2,\ldots,x_{n-1},x_n)
\end{equation*}
is Borel.
Similar to previous item, 
for each \((x_1,x_2,\ldots,x_{n-1})\in\R^{n-1}\) and 
\begin{equation*}
x_n \in [a_{i-1},a_i] = [(i-1)/n, i/n]
\end{equation*}
we define
\begin{equation*}
f_n(x_1,\ldots,x_n) = 
\frac{a_i - x_n}{a_i - a_{i-1}} g(x_1,\ldots,x_{n-1},a_{i-1}) 
+
\frac{x_n - a_{i-1}}{a_i - a_{i-1}} f(x_1,\ldots,x_{n-1},a_i).
\end{equation*}
Clearly \(\lim_{n\to\infty} f_n = f\) pointwise, and by similar argumnets 
the limit of Borel function is Borel as well. Thus $f$ is a Borel function.
\end{itemize}


%%%%%%%%%%%%%% 9
\begin{excopy}
Suppose $E$ is a dense set in \(\R^1\) and $f$ is a real function on \(\R^2\)
such that 
\ich{a} \(f_x\) is Lebesgue measurable for each \(x\in E\) and 
\ich{b} \(f^y\) is continuous for almost all  \(y\in \R^1\).
Prove that $f$ is a Lebesgue measurable on \(\R^2\).
\end{excopy}

Enumerate $E$ as \(\{a_j\}_{j\in\N}\) such that for each $n$ 
there exists some $n$ such that 
for each \(x\in[-n,n]\) we can there exists \(j,k\in\N_m\)
such that \(a_j\leq x \leq a_k\) and \(a_k-a_j < 1/n\).
(Note that \(2n^2 \leq m < \infty)\).) Let
\(K_n = \left[\min(\{a_j: j\leq n\}), \max(\{a_j: j\leq n\})\right]\)

Let $D$ be the set of all \(y\in\R\) such that \(f^y\) is not continuous.
Clearly \(m(D)=0\) and also 
\(m_2\left(\{(x,y)\in\R^2: y\in D\}\right) = m_2(\R\times D) = 0\).

Now define 
\begin{equation*}
f_n(x,y) = \left\{
\begin{array}{ll}
0 & y \in D \;\vee\; x \notin K_n \\
\frac{a_i - x}{a_i - a_{i-1}} f(a_{i-1}, y) 
+
\frac{x - a_{i-1}}{a_i - a_{i-1}} f(a_i, y) & \textnormal{otherwise}.
\end{array}
\right.
\end{equation*}
Clearly \(f_n\) are Lebesgue measurable, and
\(\lim_{n\to\infty} f_n = f \aded\) Thus $f$ is Lebesgue measurable.


%%%%%%%%%%%%%% 10
\begin{excopy}
Suppose $f$ is a real function on \(\R^2\),
\(f_x\) is a Lebesgue measurable for each $x$,
and \(f^y\) is continuous for each $y$.
Suppose \(g: \R^1\to \R^1\) is Lebesgue measurable,
and put \mbox{\(h(y) = f(g(y),y)\)},
Prove that $h$ is Lebesgue measurable on \(\R^1\).

\emph{Hint}: Define \(f_n\) as in Exercise~8, put \(h_n(y) = f_n(g(y),y)\).
show that each \(h_n\) is measurable and that \(h_n(y) \to h(y)\).
\end{excopy}

Following the hint. For each \(y\in\R\) and \(n\in\N\)
find 
\begin{equation*}
a_{i-1} = (i-1)/n \leq g(y) \leq i/n = a_i
\end{equation*}
and define
\begin{equation*}
h_n(y) 
= f_n(g(y),y) 
= n\left(a_i-g(y)\right)\cdot f(a_{i-1},y) + n(g(y)-a_{i-1})\cdot f(a_i,y).
\end{equation*}
By Theorem~1.9\ich{c} \(h_n\) is Lebesgue measurable.
Fix \(y\in\R\) since \(a_{i}-a_{i-1} = 1/n\) and \(f^y\) is a continuous section
we have \(\lim_{n\to\infty}h_n(y) = h(y)\).
By Corollary~\ich{a} of Theorem~1.14 $h$ is Lebesgue measurable.


%%%%%%%%%%%%%% 11
\begin{excopy}
Let \(\calB_k\) be the \salgebra\ of all Borel sets in \(\R^k\).
Prove that \(\calB_{m+n} = \calB_m \times \calB_n\).
This is relevant in Theorem~8.14.
\end{excopy}

% It is trivial to see that \(\calB_{m+n} \supseteq \calB_m \times \calB_n\).
We start with a topological lemma.
\begin{llem} \label{lem:open-countable}
If \(G\in\R^n\) is open, then it is a union of a countable collection
of $n$-cubes.
\end{llem}
\begin{thmproof}
Let \(D=\Q^n\cap G\) a countable dense set in $G$.
For each \(x\in D\) let \(C_x\) be the maximal open cube whose center is $x$
and \(C_w\subset G\). Clearly \(\cup_{w\in D} C_w \subset G\).
Assume by negation \(y\in G \setminus \cup_{w\in D} C_w\).
Let \(\delta>0\) be such that 
\begin{equation*}
C_y(\delta) := \{x\in \R^n: \|x-y\|_1<\delta\} \subset G.
\end{equation*}
Pick some \(x\in C_y(\delta/2)\cap G\), then 
\(y\in C_x\), with the contradiction \(y\in \cup_{w\in D} C_w\).
\end{thmproof}

By the Lemma every open set \(G\in \R^{m+n}\) is a countable union
of open ``rectangles'' \(X\times Y\subset \R^m\times\R^n\) 
where \(X\in\R^m\) and \(Y\in\R^n\) are open sets.
Hence \(G\in \calB_m \times \calB_n\).
But \(\calB_{m+n}\) is a minimal \salgebra\ that contains all open sets
in \(\R^{m+n}\) and so \(\calB_{m+n} \subset \calB_m \times \calB_n\).

To show the opposite inclusion it suffices to show that if
an arbitrary elementary set
\(E = E_m\times E_n \in \calB_m \times \calB_n\)
then \(E\in \calB_{m+n}\).
But since \(E_m\in \calB_m\) then by minimality of the 
Borel \salgebra\ \(\calB_m\) we also have (its inverse projection)
\(E_m\times\R^n\in\calB_{m+n}\).
Similarly \(R^m\times E_n\in\calB_{m+n}\). Now
\begin{equation*}
E = E_m\times\R^n \;\cap\;R^m\times E_n \in \calB_{m+n}.
\end{equation*}


%%%%%%%%%%%%%% 12
\begin{excopy}
Use Fubini's
\index{Fubini}
Theorem~8.8 and the relation
\begin{equation*}
\frac{1}{x} = \int_0^\infty e^{-xt}\,dt \qquad (x>0)
\end{equation*}
to prove that 
\begin{equation*}
\lim_{A\to\infty} \int_0^A \frac{\sin x}{x}\,dx = \frac{\pi}{2}.
\end{equation*}
\end{excopy}

\begin{align*}
\lim_{A\to\infty}\int_0^A \frac{\sin x}{x}\,dx 
&= \int_0^\infty \sin x\left( \int_0^\infty e^{-xt}\,dt\right)\,dx 
 = \int_0^\infty \left(\int_0^\infty e^{-xt}\frac{e^{ix}-e^{-ix}}{2i}\,dt\right)dx \\
&= \frac{1}{2i}\int_0^\infty \left(\int_0^\infty e^{-xt}(e^{ix}-e^{-ix})\,dt\right)dx
   \\
&= \frac{1}{2i}\int_0^\infty 
        \left(\int_0^\infty e^{(-t+i)x}-e^{-(t+i)x}\,dx\right)dt 
   \qquad \textnormal{(Fubini)}
   \\
&= \frac{1}{2i}\int_0^\infty \left(
       \frac{e^{(-t+i)x}}{-t+i} - \frac{e^{-(t+i)x}}{-t-i}
    \right)\biggm|_{x=0}^\infty \,dt \\
&= \frac{1}{2i}\int_0^\infty 
        \left(0 - \frac{1}{-t + i}\right)
        - \left(0 - \frac{1}{-t - i}\right)\,dt \\
&= \frac{1}{2i}\int_0^\infty 
        \frac{1}{-t - i} - \frac{1}{-t + i}\,dt
 = \frac{1}{2i}\int_0^\infty \frac{2i}{t^2 + 1}\,dt
 = \int_0^\infty \frac{1}{t^2 + 1}\,dt \\
&= \left(\tan^{-1}(t)\right)\bigm|_{t=0}^\infty
 = \tan^{-1}(\infty) - \tan^{-1}(0) = \pi/2
\end{align*}


%%%%%%%%%%%%%% 13
\begin{excopy}
If \(\mu\) is a complex measure on a \salgebra\ \frakM, show that every 
set \(E\in\frakM\) has a subset $A$ for which 
\begin{equation*}
|\mu(A)| \geq \frac{1}{\pi} |\mu|(E).
\end{equation*}
\emph{Suggestion}: There is a measurable real function \(\theta\) so that
\(d\mu = e^{i\theta}\,d|\mu|\).
Let \(A_\alpha\) be the subset of $E$ where \(\cos(\theta - \alpha)>0\), 
show that
\begin{equation*}
\Re[e^{-i\alpha}\mu(A_\alpha)] = \int_E \cos^+(\theta - \alpha)\,d|\mu|.
\end{equation*}
and integrate with respect to \(\alpha\) (as in Lemma~6.3).

Show by an example, that \(1/\pi\) is the best constant in this inequality.
\end{excopy}

Compute
\begin{align*}
e^{-i\alpha} \mu(A_\alpha)
&= e^{-i\alpha}\int_{A_\alpha} 1\,d\mu
 = \int_{A_\alpha} e^{i(\theta - \alpha)}\,d|\mu|
 = \int_{A_\alpha} \cos(\theta-\alpha)\,d|\mu| 
   + i\int_{A_\alpha} \sin(\theta - \alpha)\,d|\mu|.
\end{align*}
Hence
\begin{equation*}
\Re[e^{-i\alpha}\mu(A_\alpha)]
= \int_{A_\alpha} \cos(\theta-\alpha)\,d|\mu| 
= \int_E \cos^+(\theta-\alpha)\,d|\mu|.
\end{equation*}

Integrate by \(\alpha\)
\begin{align*}
\int_0^{2\pi} \Re[e^{-i\alpha}\mu(A_\alpha)]\,d\alpha
&= \int_0^{2\pi} 
   \left(\int_E \cos^+(\theta(x)-\alpha)\,d|\mu|(x)\right)\,d\alpha  \\
&= \int_E \left(
      \int_0^{2\pi} \cos^+(\theta(x)-\alpha)\,d\alpha
          \right)\,d|\mu|(x)  \qquad\textnormal{(Fubini)} \\
&= \int_E \left(
      \int_0^{2\pi} \cos^+(-\alpha)\,d\alpha \right)\,d|\mu|(x) 
 = \int_E 2\,d|\mu|(x) 
 = 2|\mu|(E).
\end{align*}

Assume by negation
\(|\mu(A_\alpha)| <  |\mu|(E)/\pi\) 
for all \(\alpha\in[0,2\pi]\). Then
\begin{align*}
\left|\int_0^{2\pi} \Re[e^{-i\alpha}\mu(A_\alpha)]\,d\alpha\right|
&\leq \int_0^{2\pi} \left|\Re[e^{-i\alpha}\mu(A_\alpha)]\right|\,d\alpha
 \leq \int_0^{2\pi} |\mu(A_\alpha)|\,d\alpha \\
&< \int_0^{2\pi} |\mu|(E)/\pi\,d\alpha
  = 2|\mu|(E)
\end{align*}
that contradicts the previous established equality.
Hence there exists some \(\alpha\)
such that \(|\mu(A_\alpha)| \geq |\mu|(E)/\pi\).

\paragraph{Best Constant.}
Consider the unit circle \(\gamma:[0,2\pi]\to\C\) 
defined by \(\gamma(t) = e^{it}\) with the complex measure
\(d\mu = \lambda'(t)\,dm(t)\).
Clearly \(|\mu|(\gamma^*) = 2\pi\).

For any \(E\subset \gamma^*\) and any \(u\in\C\) such that \(|u|=1\)
we put \(E_u = \{uz: z\in E\}\).
It is easy to see that \(E_u\subset \gamma^*\) and that 
\(|\mu(E)| = |\mu(E_u)|\).

Take 
\begin{equation*}
A = \{\gamma(t): t\in[0,2\pi] \wedge\;\Re(\gamma'(t)) \geq 0\}
  = \{z\in\gamma^{*}: \Im(z) \leq 0\}
\end{equation*}
Now
\begin{equation*}
\mu(A) = \int_{\pi}^{2\pi} = 1\cdot(\gamma^{it})'\,dt
 = \gamma(2\pi) - \gamma(\pi) = 2 = |\mu|(\gamma^*)/\pi.
\end{equation*}

We will show that $A$ consist of the positive part of \(\mu\)
and \(\gamma^*\setminus A\) of the negative.
Let \(P \subset A\) and \(N \subset \gamma^*\setminus A\), then
using the definition of $A$ we have
\begin{equation*}
\Re(\mu(P)) 
= \int_P \Re\left((\gamma^{it})'\right)\,dt
= \int_P \left|\Re\left((\gamma^{it})'\right)\right|\,dt
\geq 0.
\end{equation*}
and
\begin{equation*}
\Re(\mu(N)) 
= \int_N \Re\left((\gamma^{it})'\right)\,dt
= \int_N -\left|\Re\left((\gamma^{it})'\right)\right|\,dt
\leq 0.
\end{equation*}

Assume by negation there exists some \(E\subset\gamma^*\) such that 
\(|\mu(E)| > 2\). \Wlogy\ (or by replacing $E$ by some \(E_u\)
we may assume that \(\mu(E) > 0\) is real.
Now we get the following contradiction
\begin{align*}
\mu(E) 
&= \Re(\mu(E\cap A)) + \Re(\mu(E\setminus A))
 \leq \Re(\mu(E\cap A))
 = \Re(\mu(E\cap A)) \\
&\leq \Re(\mu(E\cap A)) + \Re(\mu(A\cap E))
 = \Re(\mu(A)) 
 = \mu(A)
 = 2.
\end{align*}

%%%%%%%%%%%%%% 14
\begin{excopy}
Complete the following proof of Hardy's
\index{Hardy} 
inequality
(Chap.~3 Exercise~14).
Suppose \(f\geq 0\) on \((0,\infty)\), \(f\in L^p\), \(1<p<\infty\), and
\begin{equation*}
F(x) = \frac{1}{x}\int_0^x f(t)\,dt.
\end{equation*}
Write \(xF(x) = \int_0^x f(t)t^\alpha t^{-\alpha}\,dt\),
where \(0<\alpha<1/q\), use H\"older's inequality 
to get an upper bound for
\(F(x)^p\), and integrate to obtain
\begin{equation*}
\int_0^\infty F^p(x)\,dx 
\leq 
(1-\alpha q)^{1-p} (\alpha p)^{-1} \int_0^\infty f^p(t)\,dt.
\end{equation*}
Show that the best choice of \(\alpha\) yields
\begin{equation*}
\int_0^\infty F^p(x)\,dx 
\leq 
\left(\frac{p}{p-1}\right)^{p} \int_0^\infty f^p(t)\,dt.
\end{equation*}
\end{excopy}

In \cite{Garling2007} Section~7.3,
given \(\mu\)-measurable \(f:\Omega\to\C\)
\index{maximal function!Muirhead}
\index{Muirhead!maximal function}
Muirhead's maximal function is introduced, 
\begin{equation*}
f^\dagger(t) = \sup\left\{\frac{1}{t}\int_E|f|\,d\mu: \mu(E)\leq t\right\}
\qquad \textnormal{where}\; 0 < t < \mu(\Omega).
\end{equation*}

The treatment there (\cite{Garling2007}) uses
the following notation
\begin{equation*}
\lambda_f(t) = \mu(f > t) \qquad (t\geq 0)
\end{equation*}
for the distribution function.


See \cite{Garling2007} Theorem~8.1.1 and its Corollary.

Using H\"older's inequality:
\begin{equation*}
\int_0^x f(t)\,dt
= \int_0^x f(t)t^\alpha t^{-\alpha}\,dt
\leq \left(\int_0^x \left(f(t)t^\alpha\right)^p\,dt\right)^{1/p}
      \cdot \left(\int_0^x t^{-\alpha q}\,dt\right)^{1/q}
\end{equation*}
Simplifying the last term
\begin{equation*}
\left(\int_0^x t^{-\alpha q}\,dt\right)^{1/q}
= \left(\frac{x^{1-\alpha q}}{1-\alpha q}\right)^{1/q}.
\end{equation*}
Combining with the inequality gives
\begin{align*}
F(x)^p
&\leq \left(x^{-1} \left(\frac{x^{1-\alpha q}}{1-\alpha q}\right)^{1/q}\right)^p
      \left(\int_0^x \left(f(t)t^\alpha\right)^p\,dt\right)
 = (1-\alpha q)^{-p/q} x^{-\alpha p - 1}
   \left(\int_0^x \left(f(t)t^\alpha\right)^p\,dt\right) \\
&= (1-\alpha q)^{1-p} x^{-\alpha p - 1}
   \left(\int_0^x \left(f(t)t^\alpha\right)^p\,dt\right).
\end{align*}
The power of $x$ was simplified
by \((-1 + (1-\alpha q)\frac{1}{q})p = -p + p/q - \alpha p = -\alpha p - 1\).

We now integrate using 
\index{Fubini}
Fubini's Theorem~8.8 on \(\{(x,t)\in\R^2: 0\leq t \leq x\}\)
\begin{align*}
\int_0^\infty F^p(x)\,dx 
&\leq \int_0^\infty
  \left((1-\alpha q)^{1-p} x^{-\alpha p - 1}
   \left(\int_0^x \left(f(t)t^\alpha\right)^p\,dt\right)\right)\,dx
   \\
&= (1-\alpha q)^{1-p}
   \int_0^\infty
   \left( x^{-\alpha p - 1}
   \left(\int_0^x \left(f(t)t^\alpha\right)^p\,dt\right)\right)\,dx
   \\
&= (1-\alpha q)^{1-p}
   \int_0^\infty 
     \left(\int_t^\infty x^{-\alpha p - 1}
                 \left(f(t)t^\alpha\right)^p\,dx\right)\,dt
   \\
&= (1-\alpha q)^{1-p}
   \int_0^\infty 
     \left(\left(f(t)t^\alpha\right)^p
       \int_t^\infty x^{-\alpha p - 1}
                 \,dx\right)\,dt.
\end{align*}
The inner integral is simplified as follows:
\begin{equation*}
\int_t^\infty x^{-\alpha p - 1}
= -\frac{1}{\alpha p}\left(x^{-\alpha p}\right)\biggm|_{x=t}^\infty
= -(\alpha p)^{-1}\left(0 - t^{-\alpha p}\right)/p = (\alpha p)^{-1} t^{-\alpha p}.
\end{equation*}
Back to previous integration
\begin{align*}
\int_0^\infty F^p(x)\,dx 
 &\leq (1-\alpha q)^{1-p}
   \int_0^\infty \left(f(t)t^\alpha\right)^p (\alpha p)^{-1} t^{-\alpha p}/p\,dt
 \\
 &= (1-\alpha q)^{1-p} (\alpha p)^{-1} \int_0^\infty \bigl(f(t)\bigr)^p\,dt
\end{align*}

\paragraph{Best Constant.}
We want to minimize
\begin{equation*}
b(\alpha) = (1-\alpha q)^{1-p} (\alpha p)^{-1}
\end{equation*}
where \(0 < \alpha < 1/q\)
This is equivalent to maximize  
\begin{equation*}
c(\alpha) = (1-\alpha q)^{p-1} \alpha p.
\end{equation*}
where \(0 < \alpha q < 1\).
Clearly in this domain \(c(\alpha) > 0\). We differentiate and equate to zero
\begin{gather*}
c'(\alpha) =  (1 - \alpha q)^{p-1} - \alpha q (p-1)(1 - \alpha q)^{p-2} = 0 \\
1 - \alpha q = \alpha q(p-1) \\
\alpha = \frac{1}{pq}\,.
\end{gather*}
Hence the best constant is
\begin{align*}
b(\alpha) 
&= (1-\alpha q)^{1-p} (\alpha p)^{-1}
= \left(1-\frac{1}{p}\right)^{1-p} \left(\frac{1}{q}\right)^{-1}
= \left(\frac{1}{q}\right)^{1-p} \left(\frac{1}{q}\right)^{-1}
= \left(\frac{1}{q}\right)^{-p}
\\
&= \left(1-\frac{1}{p}\right)^{-p}
= \left(\frac{p-1}{p}\right)^{-p}
= \left(\frac{p}{p-1}\right)^{p}\,.
\end{align*}

%%%%%%%%%%%%%% 15
\begin{excopy}
Put \(\varphi(t) = 1 - \cos t\) if \(0\leq t \leq 2\pi\), 
\(\varphi(t) = 0\) for all other real $t$.
For \(-\infty < x < \infty\), define
\begin{equation*}
f(x) = 1, 
\qquad g(x) = \varphi'(x),
\qquad h(x) = \int_{-\infty}^x \varphi(t)\,dt.
\end{equation*}
Verify the following statements about convolutions of these functions:
\begin{itemize}
\item[(i)] \((f\ast g)(x) = 0\) for all $x$.
\item[(ii)] \((g\ast h)(x) = (\varphi \ast \varphi)(x) > 0\) on \((0,4\pi)\).
\item[(iii)] Therefore \((f\ast g)\ast h = 0\), 
             whereas \(f\ast (g\ast h)\) is a positive constant.
\end{itemize}
But convolution is supposedly associative, by Fubini's Theorem~8.8
\index{Fubini}
(Exercise~5(c)). What went wrong?
\end{excopy}

First compute the functions
\begin{equation*}
g(x) = \left\{\begin{array}{ll}
0            & x < 0 \\
\sin x \quad & 0 < x < 2\pi \\
0            & x > 2\pi
\end{array}
\right.
\qquad
h(x) = \left\{\begin{array}{ll}
0                & x \leq 0 \\
x - \sin x \quad & 0 \leq x \leq 2\pi \\
2\pi             & x \geq 2\pi
\end{array}
\right.
\end{equation*}

\begin{itemize}
\item[(i)]
Compute:
\begin{equation*}
(f\ast g)(x) 
= \int_\R f(x-t)g(t)\,dt
= \int_\R g(t)\,dt
= \int_0^{2\pi} \sin(t)\,dt 
= 0
\end{equation*}

\item[(ii)]
We note that \(0 < x-t < 2\pi\) iff  \(x-2\pi < t < x\).
Compute:
\begin{align*}
(g\ast h)(x)
&= \int_\R g(x-t)h(t)\,dt
 = \int_0^{2\pi} g(x-t)\cdot (t - \sin t)\,dt
% = \int_\R g(t)h(x-t)\,dt
% = \int_0^{2\pi} \sin(t)\cdot h(x-t)\,dt
 \\
&= \left\{\begin{array}{ll}
   \int_{\max(x-2\pi,0)}^{\min(x,2\pi)} \sin (x-t)\cdot (t - \sin t)\,dt 
        \quad & x \in [0,4\pi] \\
   0  & x \notin [0,4\pi]
   \end{array}\right.
\end{align*}
We have 
\begin{itemize}
\item[\(\circ\)] \(\max(x-2\pi,0) < \min(x,2\pi)\) when \(0<x<4\pi\),
\item[\(\circ\)] \(t-\sin t > 0\) for all \(t>0\),
\item[\(\circ\)] \(\sin(x-t)>0\) when \(0 < x-t < 2\pi\).
\end{itemize}
Hence \((g\ast h)(x)>0\) when \(0<x<4\pi\).

\item[(iii)]
Clearly by linearity of convolution.

\end{itemize}
The associativity holds in Exercise~5\ich{c} for complex measures, 
meaning finite measures on~\R. Viewing these as functions, 
the associativity holds for functions in \(L^1(\R)\) but clearly
here \(f\notin L^1(\R)\).

%%%%%%%%%%%%%% 16
\begin{excopy}
Prove the following analogue of Minkowski's inequality
\index{Minkowski}
, for \(f\geq 0\):
\begin{equation*}
\left\{\int \left[\int f(x,y)\,d\lambda(y)\right]^p\,d\mu(x)\right\}^{1/p}
\leq
\int \left[\int f^p(x,y)\,d\mu(x) \right]^{1/p}\,d\lambda(y).
\end{equation*}
Supply the required hypothesis.
(Many further developments of this theme may be found in [9].)
\end{excopy}

See also Theorem~2.4 in \cite{LiebLoss200104}.

Let \((X,\mu)\) and \(Y,\lambda\) be \(\sigma\)-finite measurable spaces
and \(f:X\times Y\to  \R^\oplus\) measurable function.

\emph{Note:} If \(Y = \{0,1\}\) and \(\lambda\) is the counting measure,
then we get the known Minkowski's inequality
of Theorem~3.5(2). We follow similar idea as in the proof there.

We may assume that $f$ and \(\supp f\) are bounded
and generalize the result by applying
Lebesgue's monotone convergence Theorem~1.34.

Denote the two sides
and the inner integral of the left side of the desired inequality as
\begin{align*}
L &= \left(\int_X 
        \left(\int_Y f(x,y)\,d\lambda(y)\right)^p d\mu(x)\right)^{1/p} \\
R &= \int_Y\left(\int_X \left(f(x,y)\right)^pd\mu(x) \right)^{1/p} d\lambda(y) \\
\psi(x) &= \int_Y f(x,y)\,d\lambda(y).
\end{align*}
Fix \(v\in Y\), the 
\index{H\"older}
H\"older inequality of of Theorem~3.5(1) gives
\begin{equation*}
\int_X f(x,v)\left(\psi(x)\right)^{p-1} d\mu(x)
\leq \left(\int_X \left(f(x,v)\right)^p\,d\mu(x)\right)^{1/p} 
     \left(\int_X \left(\psi(x)\right)^{(p-1)q} d\mu(x)\right)^{1/q}
\end{equation*}
Using Fubini's Theorem~8.8 and \(p = (p-1)q\), we integrate both sides
\begin{align}
A
&= \int_Y 
    \left(\int_X f(x,v)\left(\psi(x)\right)^{p-1} d\mu(x)\right)\,d\lambda(v)
 = \int_X 
    \left(\int_Y f(x,v)\left(\psi(x)\right)^{p-1} d\lambda(v)\right)\,d\mu(x)
   \notag \\
&= \int_X 
    \left(\psi(x)\right)^{p-1} \left(\int_Y f(x,v)\,d\lambda(v)\right)\,d\mu(x)
 = \int_X \left(\psi(x)\right)^p d\mu(x) = L^p
   \notag \\
&\leq  \label{eq:minkowski:gen1}
    \left(
      \int_Y
        \left(\int_X \left(f(x,v)\right)^p\,d\mu(x)\right)^{1/p} 
      d\lambda(v)
    \right)
    \left(\int_X \left(\psi(x)\right)^p d\mu(x)\right)^{1/q} 
  \\
&= \left(\int_X \left(\psi(x)\right)^p d\mu(x)\right)^{1/q} \cdot R.
  \notag
\end{align}

We now have \(L = A^{1/p}\) and establish an inequality
\(A \leq A^{1/q} R\).
Three cases regarding $A$ are to be considered.

\paragraph{Case 1 (Zero).} 
If \(A=0\) then \(L = A^{1/p}=0\) and the desired inequality is trivial.  

\paragraph{Case 2 (Infinity).}
The case of \(A=\infty\) is impossible because of
our assumption that \(\|f\|_\infty<\infty\) 
and \((\mu\times\lambda)(\supp f) < \infty\).

\paragraph{Case 3 (Normal).}
If \(0 < A < \infty\) then we can safely divide the inequality
we got by \(A^{1/q}\) and since \(1-1/q=1/p\) we get the desired inequality,
\(A^{1/p} \leq R\) or explicitly
\begin{equation*}
 \left(\int_X 
    \left(\int_Y f(x,y)\,d\lambda(y)\right)^p d\mu(x)\right)^{1/p} 
\leq
 \int_Y\left(\int_X \left(f(x,y)\right)^pd\mu(x) \right)^{1/p} d\lambda(y).
\end{equation*}


%%%%%%%%%%%%%%%%%
\end{enumerate}
