%%%%%%%%%%%%%%%%%%%%%%%%%%%%%%%%%%%%%%%%%%%%%%%%%%%%%%%%%%%%%%%%%%%%%%%%
%%%%%%%%%%%%%%%%%%%%%%%%%%%%%%%%%%%%%%%%%%%%%%%%%%%%%%%%%%%%%%%%%%%%%%%%
%%%%%%%%%%%%%%%%%%%%%%%%%%%%%%%%%%%%%%%%%%%%%%%%%%%%%%%%%%%%%%%%%%%%%%%%
%chapter 10
\chapterTypeout{Elementary Properties of Holomorphic Functions}

\newcommand{\itwopi}{\frac{1}{2\pi}}
\newcommand{\itwopii}{\frac{1}{2\pi i}}

%%%%%%%%%%%%%%%%%%%%%%%%%%%%%%%%%%%%%%%%%%%%%%%%%%%%%%%%%%%%%%%%%%%%%%%%
%%%%%%%%%%%%%%%%%%%%%%%%%%%%%%%%%%%%%%%%%%%%%%%%%%%%%%%%%%%%%%%%%%%%%%%%
\section{Notes}

%%%%%%%%%%%%%%%%%%%%%%%%%%%%%%%%%%%%%%%%%%%%%%%%%%%%%%%%%%%%%%%%%%%%%%%%
\subsection{Theoerm 10.6 --- Power Series is Holomorphic}

In the proof of theorem~10.6 the following equality
\begin{equation} \label{eq:thm:10.6}
\left[ \frac{z^n - w^n}{z-w} - nw^{n-1} \right]
= (z-w)\sum_{k=1}^{n-1} kw^{k-1} z^{n-k-1}
\end{equation}
is used for \(n\geq 2\) and \(z\neq w\). Let's work it out.
In order to compute the fraction we start with
\begin{align*}
(z-w)\sum_{k=0}^{n-1} w^k z^{n-k-1}
&=  \left(\sum_{k=0}^{n-1} w^k z^{n-k}\right)
  - \left(\sum_{k=0}^{n-1} w^{k+1} z^{n-k-1}\right) \\
&=  \left(\sum_{k=0}^{n-1} w^k z^{n-k}\right)
  - \left(\sum_{k=1}^{n} w^{k} z^{n-k}\right) \\
&= z^n + \left(\sum_{k=0}^{n-1} \left(w^k z^{n-k} - w^k z^{n-k}\right)\right) - w^n\\
&= z^n - w^n
\end{align*}
Hence
\begin{equation} \label{eq:thm:10.6:frac}
\frac{z^n - w^n}{z-w} = \sum_{k=0}^{n-1} w^k z^{n-k-1}
\end{equation}

Finally,
\begin{align*}
(z-w)\sum_{k=1}^{n-1} kw^{k-1} z^{n-k-1}
&= \left(\sum_{k=1}^{n-1} kw^{k-1} z^{n-k}\right) -
   \left(\sum_{k=1}^{n-1} kw^{k} z^{n-k-1}\right) \\
&= \left(\sum_{k=0}^{n-2} (k+1)w^{k} z^{n-k-1}\right) -
   \left(\sum_{k=1}^{n-1} kw^{k} z^{n-k-1}\right) \\
&= (0-1)w^0z^{n-1} +
   \left(\sum_{k=1}^{n-2} (k_1-k)w^{k} z^{n-k-1}\right) -
   (n-1)w^{n-1}z^0 \\
&= z^{n-1} + \left(\sum_{k=1}^{n-2} w^kz^{n-k-1}\right) + w^{n-1} - nw^{n-1} \\
&= \left(\sum_{k=0}^{n-1} w^kz^{n-k-1}\right) - nw^{n-1} \\
&= \frac{z^n - w^n}{z-w} - nw^{n-1}
\end{align*}
We used \eqref{eq:thm:10.6:frac} in the last equality,
hence \eqref{eq:thm:10.6} holds.


%%%%%%%%%%%%%%%%%%%%%%%%%%%%%%%%%%%%%%%%%%%%%%%%%%%%%%%%%%%%%%%%%%%%%%%%
\subsection{Theoerm 10.15 --- Cauchy Fromula in Convex Region}

Let us work out the last derivation in the proof of theoerm~10.15.
\begin{equation*}
0 = \itwopii\int_\gamma g(\xi)\,d\xi
= \itwopii\int_\gamma \frac{f(\xi)-f(z)}{\xi - z}\,d\xi
\end{equation*}
Hence
\begin{equation*}
\itwopii\int_\gamma \frac{f(\xi)}{\xi - z}\,d\xi
= \frac{f(z)}{2\pi i}\int_\gamma \frac{d\xi}{\xi - z}\,d\xi
= f(z)\cdot\Ind_\gamma(z)
\end{equation*}


%%%%%%%%%%%%%%%%%%%%%%%%%%%%%%%%%%%%%%%%%%%%%%%%%%%%%%%%%%%%%%%%%%%%%%%%
\subsection{Theoerm 10.25 --- Inequality}

Let us show the inequality used in the proof of theoerm~10.25.

Given a polynomial \(P(x) = \sum_{j=0}^n a_nz^n\) where \(a_n=1\)
and
\begin{equation} \label{eq:thm10.25:r}
r > 1 + 2|a_0| + \sum_{j=1}^{n-1} |a_j|
\end{equation}
we need to show that
\begin{equation} \label{eq:thm10.25}
\left| P(re^{i\theta})\right| > |P(0)| = |a_0|.
\end{equation}
By \eqref{eq:thm10.25:r} we have
\begin{equation*}
r^n
> r^n + 2|a_0|r^n + \sum_{j=1}^{n-1} |a_j|r^n
> 2|a_0| + \sum_{j=1}^{n-1} |a_j|r^j
\end{equation*}
In the last inequality we used the fact that \(r>1\).
Hence
\begin{equation*}
r^n - \sum_{j=0}^{n-1} |a_j|r^n > |a_0|.
\end{equation*}
Now clearly
\begin{equation*}
\left| P(re^{i\theta})\right|
= \left| r^ne^{ni\theta} + \sum{j=1}^{n-1} a_j r^j e^{ji\theta}\right|
\geq r^n - \sum{j=1}^{n-1} |a_j| r^j > |a_0| = |P(0)|.
\end{equation*}
Hence \eqref{eq:thm10.25} is true.

%%%%%%%%%%%%%%%%%%%%%%%%%%%%%%%%%%%%%%%%%%%%%%%%%%%%%%%%%%%%%%%%%%%%%%%%
\subsection{Theoerm 10.26 --- Deriving the Estimation}

Let's finalize the derivation in the proof of theoerm~10.26.
By Theoerm~10.22 for
\begin{equation*}
f(z) = \sum_{n=0}^\infty c_n(z-a)^n
\end{equation*}
we have
\begin{equation*}
\sum_{n=0}^\infty |c_n|^2 r^{2n}
= \frac{1}{2\pi}\int_0^{2\pi} \left|f(a+re^{it})\right|^2\,d\theta \leq M^2
\end{equation*}
for any \(r<R\). Hence \(|c_n|r^n < M\) for all \(r<R\) and for all $n$.
Thus \(|c_n| < M/R^n\) for all $n$. Now clearly
\begin{equation*}
f^{(n)}(a) = n!c_n
\end{equation*}
and so
\begin{equation*}
\left|f^{(n)}(a)\right| = n!|c_n| \leq n!M/R^n.
\end{equation*}


%%%%%%%%%%%%%%%%%%%%%%%%%%%%%%%%%%%%%%%%%%%%%%%%%%%%%%%%%%%%%%%%%%%%%%%%
\subsection{Theoerm 10.28 --- Compact Argument}

In the proof of theoerm~10.28, we are given a compact \(K \subset \Omega\)
and the proof claims that there exists \(r>0\) such that
\begin{equation*}
E := \bigcup_{z\in K} \overline{D}(z;r)
\end{equation*}
is a compact subset of \(\Omega\).
Here is the justification.

Take the distance 
\begin{equation*}
d=d(K,\Omega^c)=\inf\{d(z,w):z\in K, w\in \C\setminus\Omega\}.
\end{equation*}
By compactness argument and the fat that \(\C\) is a Hausdorff space, \(d>0\).
Put \(r=d/2\) and
for each \(z\in K\) we pick \(r_z>0\) such that \(D(z,r) \subset \Omega\).
By compactness of $K$, there is a finite subcover
\begin{equation*}
E = \subset \bigcup_{z\in I} D(z,r) \subset \Omega.
\end{equation*}
where \(I\subset K\) is finite.


%%%%%%%%%%%%%%%%%%%%%%%%%%%%%%%%%%%%%%%%%%%%%%%%%%%%%%%%%%%%%%%%%%%%%%%%
\subsection{Lemma 10.29 --- Working out an Integral}

In the proof of Lemma~10.29.
we use the formula for integration oevr path devleopped
in section~10.8, namely for a smooth \(\gamma:[\alpha,\beta]\to\C\) path
\begin{equation*}
\int_\gamma f(x)\,dz = \int_\alpha^\beta f(\bigl(\gamma(t)\bigr)\gamma'(t)\,dt.
\end{equation*}

with \(\zeta(t) = (1-t)z+tw\) we can compute
\begin{align*}
f(w) - f(z)
&= f(\zeta(1)) - f(\zeta(0)) \\
&= \int_{\zeta([0,1])} f'(\xi)\,d\xi
 = \int_{\zeta([0,1])} f'(\zeta(t))\zeta'(t)\,dt
 = \int_{\zeta([0,1])} f'(\zeta(t))(w-z)\,dt \\
&= (w-z)\int_{\zeta([0,1])} f'(\zeta(t))\,dt
 = (w-z)\int_{\zeta([0,1])} f'(\zeta(t))\,dt
\end{align*}
Hence
\begin{equation*}
\int_0^1 \left[f'(\zeta(t)) - f'(a)\right]\,dt
= \int_0^1 f'(\zeta(t))\,dt - f'(a)
= \bigl((f(w) - f(z)\bigr)/(w-z) - f'(a).
\end{equation*}

It is tempting to try prroving this lemma directly without integration.
For this we need to have a local bound on \(f'\) which is given
by the fact that \(f^{(2)}\) exists and is continuous.
The idea is to show continuity at \((0,0)\) and then for sufficiently small
\(r>0\) and \(z,w\in D'(0,r)\) estimate
\begin{equation*}
\left|\frac{f(z)-f(w)}{z-w} - f'(0)\right|
\leq \left|\frac{f(z)-f(w)}{z-w} - f'(w)\right| + |f'(w) - f(0)|
\end{equation*}
and show that it can be small as desired.

%%%%%%%%%%%%%%%%%%%%%%%%%%%%%%%%%%%%%%%%%%%%%%%%%%%%%%%%%%%%%%%%%%%%%%%%
\subsection{Lemma 10.30 --- Adding Details}

The proof of theoerm~10.30 has several arguments that may use
clarifying details.

\paragraph{Setting $c$.} We can set
\begin{equation*}
c := \left|r\varphi'(z_0)\right|/5
\end{equation*}

\paragraph{Onto neighborhood.}
We pick \(\lambda\) such that
\begin{equation} \label{eq:thm10.30:lambda}
|\lambda - \varphi(a)| < c.
\end{equation}
Then after we show
\begin{equation*}
\min_\theta \left| \lambda - \varphi(a + re^{i\theta})\right|
\geq
  \left(\min_\theta \left| \varphi(a + re^{i\theta}) - \varphi(a)\right|\right)
  - |\varphi(a) - \lambda|
> 2c - c = c
\end{equation*}
the proof utilizes corollary of theoerm~10.24. Let's show how.
Set \(\psi(z) = \lambda - \varphi(z)\).
The corollary says that if
\(\psi(z)\) has no zeros in \(D(a;r)\) then
\begin{equation*}
|\psi(a)| = |\lambda - \varphi(a)|
\geq \min_\theta \left| \lambda - \varphi(a + re^{i\theta})\right|.
\end{equation*}
But the last inequality contradicts \eqref{eq:thm10.30:lambda}, hence
\(\psi\) has a zero in \(D(a;r)\).


%%%%%%%%%%%%%%%%%%%%%%%%%%%%%%%%%%%%%%%%%%%%%%%%%%%%%%%%%%%%%%%%%%%%%%%%
\subsection{Lemma 10.32 --- Differentiation}

Given \(g'/g = h'\) for some \(h\in H(\Omega)\).
Put \(\eta = g\cdot\exp(-h)\). Now
\begin{equation*}
\eta'
= g'\cdot\exp(-h) + g(-h)'\cdot\exp(-h)
= g'\cdot\exp(-h) + g(-g'/g)'\cdot\exp(-h)
= (g'-g')\cdot\exp(-h)
= 0.
\end{equation*}

%%%%%%%%%%%%%%%%%%%%%%%%%%%%%%%%%%%%%%%%%%%%%%%%%%%%%%%%%%%%%%%%%%%%%%%%
\subsection{Lemma 10.32 --- Details}

The term
\index{deleted neighborhood}
\index{deleted!neighborhood}
\emph{deleted neighborhood} of \(z_0\) means \(V\setminus\{z_0\}\)
where $V$ is a neighborhood of \(z_0\).

Tracing the proof. Assuming by negation \(f'(z_0) = 0\).
By theoerm~10.32 for a neighborhood $V$ of \(z_0\) we have
\begin{equation*}
f(z) = f(z_0) + \bigl(\varphi(z)\bigr)^m
\qquad \textnormal{where}\;\varphi\in H(V),\; m > 1.
\end{equation*}
and \(\varphi\) is \emph{onto} some \(D(0;r)\).
So if we look at inverse image
\begin{equation*}
A := \{\varphi^{-1}(\rho e^{2\pi k/m}\} \qquad \textnormal{for}\; \rho < r.
\end{equation*}
Now \(|A|=m\) and \(f(A)\) has one value, thus $f$ is $m$-to-$1$.
This contradiction shows that \(f'(z_0) \neq 0\).

%%%%%%%%%%%%%%%%%%%%%%%%%%%%%%%%%%%%%%%%%%%%%%%%%%%%%%%%%%%%%%%%%%%%%%%%
\subsection{Theoerm 10.37 --- Details}

There is a typo in the end of the stated Theorem.
Instead of
\begin{quote}
\textsl{\ldots\ if \(x\in D_-\) \ldots}
\end{quote}
It should be
\begin{quote}
\textsl{\ldots\ if \(z\in D_-\) \ldots}
\end{quote}

There are illustrations of the paths defined in the proof 
in \figurename{\ref{fig:10-37}}.
\begin{figure}[ht]
 % \centering
 % \captionsetup[subfloat]{nearskip=-3pt}
%
\subfloat[\(C(s)\) and \(\gamma(s)\)]{%
\begin{minipage}[b]{0.4\textwidth}
\centering
\input{10-37-a}
\end{minipage}
}
%
\hspace{0.1\textwidth}
%
\subfloat[\(f(s)\)]{%
\begin{minipage}[b]{0.4\textwidth}
\centering
\input{10-37-b}
\end{minipage}
}
%
\\[20pt]%
%
\subfloat[\(g(s)\)]{%
\begin{minipage}[b]{0.4\textwidth}
\centering
\input{10-37-c}
\end{minipage}
}
%
\hspace{0.1\textwidth}
%
\subfloat[\(h(s)\)]{%
\begin{minipage}[b]{0.4\textwidth}
\centering
\input{10-37-d}
\end{minipage}
}
%
\caption{Paths in the Proof of Theorem 10.37}
\label{fig:10-37}
\end{figure}


%%%%%%%%%%%%%%%%%%%%%%%%%%%%%%%%%%%%%%%%%%%%%%%%%%%%%%%%%%%%%%%%%%%%%%%%
\subsection{Lemma 10.39 --- Details}

Given \(\gamma = (\gamma_1 - \alpha) / (\gamma_0 - \alpha)\)
perfrom the computation:
\begin{equation*}
\frac{\gamma'}{\gamma}
= \frac{\gamma_1'(\gamma_0-\alpha) - (\gamma_1-\alpha)\gamma_0'}{
               (\gamma_0 - \alpha)^2}
   \cdot \frac{\gamma_0 - \alpha}{\gamma_1 - \alpha}
= \frac{\gamma_1'(\gamma_0-\alpha) - (\gamma_1-\alpha)\gamma_0'}{
               (\gamma_1 - \alpha)(\gamma_0 - \alpha)}
= \frac{\gamma_1'}{\gamma_1-\alpha} - \frac{\gamma_0'}{\gamma_0-\alpha}.
\end{equation*}

%%%%%%%%%%%%%%%%%%%%%%%%%%%%%%%%%%%%%%%%%%%%%%%%%%%%%%%%%%%%%%%%%%%%%%%%
%%%%%%%%%%%%%%%%%%%%%%%%%%%%%%%%%%%%%%%%%%%%%%%%%%%%%%%%%%%%%%%%%%%%%%%%
\section{Edition 2 (Old) Exercises} % pages 193-195

Some exercises in edition~2 do not appear or are different than in edition~3.
We bring some of them here

%%%%%%%%%%%%%%%%%
\begin{enumerate}
%%%%%%%%%%%%%%%%%

\iffalse
%%%%%%%%%%%%%% 1
\begin{excopy}
   If $A$ and $B$ are disjoint subsets of the plane, if $A$ is compact,
   and if $B$ is closed, then there exists
   a~\(\delta > 0\) such that \(|\alpha-\beta| \geq 0\) for all
   \(\alpha \in A\) and \(\beta \in B\). Prove this with an arbitrary
   metric space in place of the plane.
\end{excopy}
  Let $d$ be the metric. Define
  \[G_n = \{x \in A: d(x,B) > 1/n\}\]
  for all \(n>0\). It is clear that for all \(x\in A\) \(d(x,B)>0\)
  and so \(\cup G_n = A\). Since A is compact there exists $m$ such that
  \(\cup_{n=1}^m G_n = A\).
  Hence, for all \(x\in A\) \(d(x,B)>1/m\)
  Put \(\delta=1/m\) that satsify the requirement.
\fi

\setcounter{enumi}{1}
%%%%%%%%%%%%%% 2
\begin{excopy}
    At the end of section~10.8 occurs the definition of the length of a path
    \(\gamma\) as
    \[\int_\alpha^\beta|{\gamma^\prime}(t)|dt.\]
    Does this agree with the definitions given in Exercise~10, Chapter 8?
    Length of a graph of~$f$ is
       \[f_s(1) + \int_0^1 \sqrt{1 + |f^\prime(t)|^2}dt.\]
\end{excopy}

\text{Note:} This is of course referring to exercise in
edition~2 (\cite{RudinRCA80}).
In Edition~3, it appears in chapter~7, exercise~21.
Actually in this exercise we have shown in \eqref{eq:ex7.21:Sdif}
what we need here. There we used the result for a specific case
of \(x(t)=t\).

%%%%%%%%%%%%%%
\end{enumerate}


%%%%%%%%%%%%%%%%%%%%%%%%%%%%%%%%%%%%%%%%%%%%%%%%%%%%%%%%%%%%%%%%%%%%%%%%
%%%%%%%%%%%%%%%%%%%%%%%%%%%%%%%%%%%%%%%%%%%%%%%%%%%%%%%%%%%%%%%%%%%%%%%%
\section{The Exercises} % pages 227-230

Exercises of \emph{current} 3rd edition.

%%%%%%%%%%%%%%%%%
\begin{enumerate}
%%%%%%%%%%%%%%%%%

%%%%%%%%%%%%%% 01
\begin{excopy}
The following fact was tacitly used in this chapter:
If $A$ and $B$ are disjoint subsets of the plane,
if $A$ is compat and $B$ is closed, then there exists a \(\lambda > 0\)
such that \(|\alpha - \beta| \geq \lambda\) for all
\(\alpha \in A\) and \(\beta \in B\).
Prove this, with an arbitrary metric space in place of the plane.
\end{excopy}

Let \((X,d)\) be a matric space and $A$ and $B$ as described in $X$.
For each \(x\in A\) pick some \(\lambda_x>0\) such that
\(D(x;\lambda_x) \cap B = \emptyset\). Such \(\delta_x\) always exists
since otherwise \(x \in \overline{B}=B\).

Clearly \(\{D(x;\lambda_x)\}_{x\in A}\) is an open covering of the compact $A$,
hence there is a finite set \(F\subset X\) such that
\begin{equation*}
 X \subset \bigcup_{x\in F} D(x;\lambda_x).
\end{equation*}
Hence we can pick a \(\delta = \min(\{\delta_x: x\in F\}) > 0\)
that satisfies the requirement.


%%%%%%%%%%%%%% 02
\begin{excopy}
Suppose that $f$ is an entire function,
and that in every  power series
\begin{equation*}
f(z) = \sum_{n=0}^\infty c_n(z-a)^n
\end{equation*}
at least one coefficient is $0$. Prove that $f$ is a polynomial.
\\ \emph{Hint:} \(n!\,c_n = f^{(n)}(a)\).
\end{excopy}

For each \(a\in D(0,1)\) let \(n_a\) be the minimal \(n\in\Z^+\) such that
\(c_n=0\) for the power series representation around $a$.
By cardinality argument \(\|D(0,1)|=2^{\aleph_0} > \aleph_0 = \Z^+\)
there is some \(m\in\Z^+\) and infinite subset \(A\subset D(0,1)\)
such that \(m=n_a\) for each \(a\in A\).
But then \(f^{(m)}\) has infinite zeros in \(D(0,1)\)
and hence \(f^{(m)}(z) = 0\) for all \(z\in\C\),
hence $f$ is a polynomial of degree  \(<m\).

%%%%%%%%%%%%%% 03
\begin{excopy}
If $f$ and $g$ are entire function, and \(|f(z)| \leq |g(z)|\)
for every $z$. What conclusions can you draw?
\end{excopy}

\textbf{Claim:}
Under these conditions, \(f(z) = ag(z)\) for all \(z\in\C\) for some
\(a\in\C\) such that \(a\leq 1\).

If \(g = 0\) then so is \(f = 0\) and we are done.

We claim that \(q(z) = f(z)/g(z)\) for \(z\notin Z(g)\)
can be always be extended to an entire function.

So we may now assume that $g$ is not constantly zero, and by theoerm~10.18
\(Z(g)\) has no limit point.
Assume \(g(w) = 0\)
then bt theoerm~10.18 we have the following representations
\begin{equation*}
f(z) = (z-w)^m\tilde{f}(z)
\qquad
g(z) = (z-w)^n\tilde{g}(z)
\end{equation*}
and \(\tilde{f}(w) \neq 0 \neq \tilde{f}(w)\).
If by negation \(m < n\) then for suffuciently small \(\delta>0\)
\begin{equation*}
\delta^{n-m} < \bigl|\ \tilde{f}(z+\delta) / \tilde{g}(z+\delta)\bigr|
\end{equation*}
but then \(|f(z+\delta)| > |g(z+\delta)|\) which is a contradiction.
Hence by theoerm~10.20 \(q=f/g\) has a removable singularity at $z$
and
\begin{equation*}
q(z) = \left\{\begin{array}{ll}
\tilde{f}(z)/\tilde{g}(z) \qquad& m=n \\
0 & m > n
\end{array}\right.
\end{equation*}

Similarly $q$ can be defined for all \(Z(f)\),
hence we can view it as an entire function. By the given condition,
$q$ is bounded. By
\index{Lioville}
Lioville's theoerm~10.23 $q$ is constant. Hence \(f(z)= q(0) g(z)\).


%%%%%%%%%%%%%% 04
\begin{excopy}
Suppose that $f$ is an entire function, and
\begin{equation*}
|f(x)| \leq A + B|z|^k
\end{equation*}
for all $z$, where $A$, $B$, and $k$ are positive numbers.
Prove that $f$ must be a polynomial.
\end{excopy}

Let \(f(z) = \sum c_nz^n\). By
\index{Lioville}
theoerm~10.22 for any \(r\geq 0\) we have
\begin{equation*}
\sum_{n=0}^\infty |c_n|^2 r^{2n}
= \frac{1}{2\pi}\int_{-\pi}^\pi \left| f(re^{i\theta})\right|^2\,d\theta
\leq \frac{1}{2\pi} (2\pi)\cdot(A+Br^k)^2
= A^2+2ABr^k + B^2r^{2k}
\end{equation*}
Clearly \(c_n = 0\) whenever \(n > 2k\), otherwise the above inequality
would fail. Hence $f$ is a polynomial.

%%%%%%%%%%%%%% 05
\begin{excopy} 
Suppose 
\label{ex:fn:uniform}
\(\{f_n\}\) is a uniformly bounded sequence of holomorphic function
in \(\Omega\) such that  \(\{f_n(z)\}\) converges for every \(z\in \Omega\).
Prove that the convergence is uniform on every compact subset of \(\Omega\).
\\ \emph{Hint:} Apply the dominated convergence theorem to the Cauchy formula
for \(f_n - f_m\).
\end{excopy}

Define the pointwise limit \(f(z) = \lim_{n\to\infty} f_n(z)\).
Since \(\{f_n\}\) is a uniformly bounded, we have
\begin{equation*}
M := \sup_n \|f_n\|_\infty < \infty.
\end{equation*}

Let us temporarily restrict the function to some disc
\(\overline{D}(a;r/2)\subset D(a;r) \subset\Omega\).
Let \(\gamma(t) = a+re^{it}\), hence \(\gamma^* = \partial(D(a;r))\).
By the Cauchy formula
\begin{equation*}
\left| \frac{f_n(w)}{w - z}\right| \leq 2M/r
\qquad (z\in \overline{D}(a;r/2),\, w \in \gamma^*)
\end{equation*}
For abbreviation, let \(w_t = a+re^{it}\).
Now for \(z\in \overline{D}(a;r/2)\) we have
\begin{align*}
|f_m(z) - f_n(z) 
&\leq \frac{1}{2\pi} \int_{\gamma^*} \frac{|f_m(w)-f_n(w)}{|w-z|}\,d|w|
 \leq \frac{1}{\pi r}  \int_0^{2\pi} |f_m(w_t)-f_n(w_t)|\,dt \\
& \leq \frac{1}{\pi r}  
       \left(\int_0^{2\pi} |f_m(w_t)-f(w_t)|\,dt +
             \int_0^{2\pi} |f_n(w_t)-f(w_t)|\,dt\right)
\end{align*}
The last integrands are each dominated by \(2M\).
By Lebesgue's dominated convergence theorem~1.34,
for each \(\epsilon>0\) there exists $N$ such that 
the last 2-intergrals expression is \(<\epsilon\)
if \(m,n\geq N\), for \emph{any} \(z\in \overline{D}(a;r/2)\).
Hence \(\{f_n(z)\}\) converges uniformly on \(\overline{D}(a;r/2)\).

Any compact \(K\subset\Omega\) can be covered by finite set
of such \(\{\overline{D}(a_j;r/2): j\in J\}\) closed discs
with \(|J|<\infty\) and \(D(a_j;r)\subset \Omega\).

Returning to the original definitions of \(\{f_n\}\) on \(\Omega\),
the  \(\{f_n(z)\}\) converges uniformly on each closed disc, 
and therefore the sequence converges uniformly on a finite union of them.
Hence  \(\{f_n(z)\}\) converges uniformly on $K$.

%%%%%%%%%%%%%% 06
\begin{excopy}
There is a region \(\Omega\) that \(\exp(\Omega) = D(1;1)\).
Show that \(\exp\) is one-to-one in \(\Omega\),
but that there are many such \(\Omega\). Fix one, and define
\(\log z\), for \(|z-1|<1\), to be that \(w\in\Omega\) for which \(e^w = z\).
Prove that \(\log'(z) = 1/z\). Find the coefficients \(a_n\) in
\begin{equation*}
\frac{1}{z} = \sum_{n=0}^\infty a_n(z-1)^n
\end{equation*}
and hence the coefficients \(c_n\) in the expansion
\begin{equation*}
\log z = \sum_{n=0}^\infty c_n(z-1)^n.
\end{equation*}
In what other discs can this be done?
\end{excopy}

For each \(z\in D(1;1)\) there exists a unique \(\theta\in (-\pi,\pi)\) 
and \(r>0\) such that \(z=re^{i\theta}\).
Let 
\begin{equation*}
\Omega = \left\{\log(r) + i\theta:  
           re^{i\theta} \in D(1;1) \;\wedge\; \theta\in (-\pi,\pi)\right\}.
\end{equation*}
Clearly \(\Omega_n = \{z + 2\pi ni: z\in \Omega\}\)
can be a region as required for all \(n\in\Z\).

Now with the above definion of \(\log\), we have
\begin{equation*}
(\exp\circ\log)(z) = \Id_\Omega(z) \qquad (z\in\Omega).
\end{equation*}
Hence for \(z\in\Omega\) we have 
\begin{equation*}
1 
= \log'(z)\cdot (\exp'\circ \log)(z)
= \log'(z)\cdot (\exp\circ \log)(z) 
= \log'(z)\cdot z.
\end{equation*}
Thus \(\log'(z) = 1/z\) for \(z\in \Omega\).

\paragraph{Computing coefficients.}
Differentiation of \(1/z\) gives
\begin{equation*}
\frac{d^n(z^{-1})}{dz^n} = 
% z^{-1}, -z^{-2}, -2z^{-3}, 6z^{-4}, -24z^{-5}
 (-1)^n \cdot n!\cdot z^{-n-1}
\end{equation*}
Hence, the coefficients of the power series of \(1/z\) around $1$ are
\begin{equation*}
a_n = \frac{d^n(z^{-1})}{dz^n}(z=1)/n!
=  (-1)^n \cdot n!\cdot 1^{-n-1} / n! = (-1)^n
\end{equation*}

The coefficients of the ``primitive'' \(\log(z)\) are:

This can be done on any open disc that does contains the zero.
\begin{align*}
c_0 &= 0 \\
c_n &= -(-1)^n/n \qquad (n > 0)
\end{align*}
That is \((c_n)=0,1,-1/2,1/3,-1/4,\ldots\).

%%%%%%%%%%%%%% 07
\begin{excopy}
If \(f\in H(\Omega)\), the Cauchy formula for the derivatives of $f$,
\begin{equation*}
f^{(n)}(z)
= \frac{n!}{2\pi i} \int_\Gamma \frac{f(\xi)}{(\xi - z)^{n+1}}\,d\xi
\qquad (n=1,2,3,\ldots)
\end{equation*}
is valid under conditions on $z$ and \(\Gamma\).
State these and prove the formula.
\end{excopy}

Sufficent conditions are:
\begin{itemize}
\item \(\Gamma\) is a cycle in \(\Omega\).
\item \(\Ind_\Gamma(\alpha)=0\) for every \(\alpha\in \C\setminus\Omega\).
\item \(z\notin \Gamma^*\).
\item \(\Ind_\Gamma(z)=1\).
\end{itemize}
By Cauchy's formula theorem~10.15 we have
\begin{equation*}
f(z) = \itwopii\int_\Gamma \frac{f(w)}{w-z}\,dw.
\end{equation*}
This shows the desired formula holds for \(n=0\).
By induction assumes that it holds for \(n=k\), that is
\begin{equation*}
f^{(k)}(z)
= \frac{k!}{2\pi i} \int_\Gamma \frac{f(w)}{(w - z)^{k+1}}\,dw
\end{equation*}
Now
\begin{align}
f^{(k+1)}(z)
&= \lim_{h\to 0} \left(f^{(k)}(z+h)- f^{(k)}(z+h)\right)/h \notag \\
&= \lim_{h\to 0} \frac{k!}{2\pi i} 
   \left(
    \int_\Gamma 
    f(w)
    \left(
       \frac{1}{(w - (z+h))^{k+1}} - \frac{1}{(w - z)^{k+1}}
    \right)\,dw  
    \right) \bigm/ h
    \notag \\
&= \frac{k!}{2\pi i} \int_\Gamma \lim_{h\to 0} 
    f(w)
    \left(
       \frac{1}{(w - (z+h))^{k+1}} - \frac{1}{(w - z)^{k+1}}
    \right)\bigm/h\,dw  \label{eq:cauchy:diflim} \\
&= \frac{k!}{2\pi i} \int_\Gamma 
    f(w) \frac{d\,\left((w-z)^{-(k+1)}\right)}{dz}\,dw  \notag \\
&= \frac{k!}{2\pi i} \int_\Gamma f(w)(k+1)(w-z)^{-(k+2)}\,dw \notag \\
&= \frac{(k+1)!}{2\pi i} \int_\Gamma \frac{f(w)}{(w - z)^{k+2}}\,dw \notag
\end{align}
The justification of \eqref{eq:cauchy:diflim} is based on the fact
that \(\Gamma^*\) is compact it is it sufficient to use
sequences as limits and fintally utilize exercise~\ref{ex:fn:uniform} above.

%%%%%%%%%%%%%% 08
\begin{excopy}
Suppose $P$ and $Q$ are polynomials, the  degree of $Q$ exceeds that of $P$
by at least $2$,
and the rational function \(R = P/Q\) has no pole on the real axis.
Prove that the integral of $R$ over \((-\infty,\infty)\)
is \(2\pi i\) times the sum of the residues of $R$ in the upper half plane.
[Replace the integral over \((-A,A)\)  by one over a suitable semicircle,
and apply the residue theorem.] What is the analogous statement for the lower
half plane? Use this method to compute
\begin{equation*}
 \int_{-\infty}^\infty \frac{x^2}{1+x^4}\,dx.
\end{equation*}
\end{excopy}

Let \(a>0\) be such that all the roots of $Q$ are in \(D(0;a)\).
For each \(A>a\) consider the closed path 
\(\gamma_A\: [-A, A+\pi \to\C\) that is defined as follows
\begin{equation*}
\Gamma_A(t) = \left\{%
\begin{array}{ll}
t & t \leq A \\
Ae^{i(t-A)} \quad & t \geq A
\end{array}\right.
\end{equation*}
Now define the upper arc \(U_A=\{z\in\C: |z|=A \wedge \Im(z)>0\}\) and
\begin{equation*}
I(A) 
= \int_U P(z)/Q(z)\,dz 
= \int_0^\pi P(e^{iAt})/Q(e^{iAt})\frac{dz}{dt}\,dt
= iA\int_0^\pi P(e^{iAt})e^{iAt}/Q(e^{iAt})\frac{dz}{dt}\,dt
\end{equation*}
Hence \(|I(A)| \leq \pi A \sup_{z\in A} |P(e^{iAt})/Q(e^{iAt})|\).
Since \(\deg(P) \leq \deg(Q)+2\) we have
\begin{equation*}
\lim_{A\to\infty} I(A) = 0.
\end{equation*}
\index{residue theorem}
The set of poles of $R$ is exactly the set \(Z_Q\) of zeros of $Q$.
We define 
\begin{equation*}
Z_{Q^+} = \{z\in Z_Q: \Im(z)>0\}
\qquad
Z_{Q^-} = \{z\in Z_Q: \Im(z)<0\}
\end{equation*}
By the residue theorem~10.42.
\begin{align*}
\int_{-\infty}^\infty R(x)\,dx 
&= \lim_{A\to\infty} 
   \left(\int_{\Gamma_A}  R(z)\,dz - \int_{U_A} P(z)/Q(z)\,dz\right) \\
&= \lim_{A\to\infty} \int_{\Gamma_A}  R(z)\,dz 
   - \lim_{A\to\infty} \int_{U_A} P(z)/Q(z)\,dz 
 = \lim_{A\to\infty} \int_{\Gamma_A}  R(z)\,dz \\
&= 2\pi i \sum_{z\in Z_{Q^+}} \Res(R;z).
\end{align*}

The analogous statement for \(Z_{Q^-}\) is
\begin{equation*}
\int_{-\infty}^{\infty} R(x)\,dx 
= -\int_{\infty}^{-\infty} R(x)\,dx 
= 2\pi i \sum_{z\in Z_{Q^-}} \Res(R;z).
\end{equation*}
Hence if all the zeros if \(Q(z)\) are on one side, then the integral
over the real line is zero. 
Note that if all the coefficients of \(Q(z)\) are real,
then \(Q(z)=0\) iff \(Q(\overline{z})=0\).

The set of zeros of \(x^4+1\) is \(\{e^{(2k+1)\pi i/4}: k=0,1,2,3\}\).
Hence the zeros above the real line are \(\{e^{\pi i/4}, e^{3\pi i/4}\}\)
or \(\{q,qi\}\) where \(q=e^{\pi i/4}=(1+i)\sqrt{2}/2\).
Now \(Q(z)=\prod_{k=0^3} (z-qi^k)\) hence
\begin{equation*}
\Res(R;q_j) = q^2/ \prod_{0\leq k\leq3 \wedge k\neq j} (z-qi^k) \qquad (j=0,1,2,3)
\end{equation*}
Applying it gives
\begin{align*}
 \int_{-\infty}^\infty \frac{x^2}{1+x^4}\,dx
 &= 2\pi i \sum_{z\in Z_Q} \Res(R;z)
  = 2\pi i \left( \Res(R; e^{\pi i/4}) + \Res(R; e^{\pi i/4})\right) \\
 &= 2\pi i \left( q^2/\bigl((q-qi)(q+q)(q+qi)\bigr) 
                 -q^2/\bigl((qi-q)(qi+q)(qi+qi)\bigr)\right) \\
 &= (2\pi i/q) \cdot
    \left(\frac{1}{(1-i)2(1+i)} - \frac{1}{(-1+i)(1+i)2i}\right) \\
 &= (\pi i/q)  \left(\frac{1}{2}  - \frac{1}{-2i}\right) 
  = \pi q (1-i)/2 
  = \sqrt{2}\pi(1+i)(1-i)/4 \\
 &= \sqrt{2}\pi/2 \simeq 2.2214414690791831
\end{align*}


%%%%%%%%%%%%%% 09
\begin{excopy}
Compute \(\int_{-\infty}^\infty e^{-itx}/(1+x^2)\,dx\) for real $t$,
by methods described on Exercise~8.
Check your answer against the inversion theorem for Fourier transforms.
\end{excopy}

Put \(f_t(z) = e^{-itz}/(1+z^2)\). 
Assume first that \(t\leq 0\). 
If \(\Im(z)\geq 0\) then \(\Re(-itz) \leq 0\) and so \(|e^{-itz}|\geq 1\).
% Thus \(|e^{-itz}/(1+z^2)| \leq 1/z^2\).
Consider the 
 upper plane half circle with radius \(r>1\) defined by:
\begin{equation*}
D_r = \{z\in\C: |z|\geq r \wedge \Im z \geq 0\}.
\end{equation*}
Integration over its boundary gives:
\begin{equation*}
\int_{\partial D_r} e^{-itz}/(1+z^2)\,dz 
= (2\pi i) \Res(e^{-itz}/(1+z^2); i)
= (2\pi i) e^{-iti}/(i-(-i)) = \pi e^t
\end{equation*}
On the arc part of \(D_r\) 
we have \(|e^{-itz}/(1+z^2)| \leq |1/z^2| = 1/r^2\).
Hence
\begin{equation*}
\int_{-\infty}^\infty f_t(x)\,dx
= \lim_{r\to\infty} \int_{\partial D_r} f_t(z)\,dz 
= \Res(e^{-itz}/(1+z^2); i) 
= \pi e^t
\end{equation*}
Since \(f_t(x) = f_{-t}(-x)\) for any \(t\in\R\) we have
\(\int_{-\infty}^\infty f_t(x)\,dx = \int_{-\infty}^\infty f_{-t}(x)\,dx\).
Therefore
\begin{equation*}
\int_{-\infty}^\infty e^{-itx}/(1+x^2)\,dx = \pi e^{-|t|} \qquad(t\in\R).
\end{equation*}

\paragraph{Fourier Inversion.}
\begin{align*}
\int_{-\infty}^\infty e^{itx} e^{-|t|}\,dt
&= \int_{-\infty}^0 e^{itx} e^{t}\,dt  + \int_0^\infty e^{itx} e^{-t}\,dt
 = \int_{-\infty}^0 e^{(1+ix)t}\,dt  + \int_0^\infty e^{(-1+ix)t}\,dt \\
&=   \left(e^{(1+ix)t}/(1+ix)\right)\biggm|_{t=-\infty}^0
   + \left(e^{(-1+ix)t}/(-1+ix)\right)\biggm|_{t=0}^\infty \\
&= e^{(1+ix)0}/(1+ix) - e^{(-1+ix)0}/(-1+ix) 
 = 1/(1+ix) - 1/(-1+ix) \\
&= 2/(x^2+1)
\end{align*}
Using \(\pi e^{-|t|}\) instead of \(e^{-|t|}\) 
and factoring with \(1/\sqrt{2\pi}\) provides 
the expected equality with the original \(1/(1+x^2)\) function.



%%%%%%%%%%%%%% 10
\begin{excopy}
Let \(\gamma\) be a poitively oriented unit circle, and compute
\begin{equation*}
\itwopii \int_\gamma \frac{e^z - e^{-z}}{z^4}\,dz
\end{equation*}
\end{excopy}

Using the residue theorem~10.42 and Taylor expansion
\begin{align*}
\itwopii \int_\gamma \frac{e^z - e^{-z}}{z^4}\,dz
&= \itwopii \int_\gamma z^{-4}\left(\sum_{n=0}^\infty (z^n - (-z)^n)/n!\right)\,dz \\
&= \itwopii \int_\gamma 2z^{-4}\left(\sum_{k=0}^\infty z^{2k+1}/(2k+1)!\right)\,dz \\
&= 2\cdot\Res\left(\sum_{k=0}^\infty z^{2k-3}/(2k+1)!;\,0\right)
 = 2\cdot\Res\left(z^{2\cdot 1-3}/(2\cdot 1+1)!;\,0\right) = \\
&= 2/3! = 1/3
\end{align*}


%%%%%%%%%%%%%% 11
\begin{excopy}
Suppose \(\alpha\) is a complex number, \(|\alpha|\neq 1\), and compute
\begin{equation*}
\int_0^{2\pi} \frac{d\theta}{1 - 2\alpha \cos\theta + \alpha^2}
\end{equation*}
by integrating \((z-\alpha)^{-1}(z-1/\alpha)^{-1}\) over the unit circle.
\end{excopy}

Putting \(z=e^{i\theta}\) (or \(\theta = -i\log(z)\))
then \(dz/d\theta = iz\) and \(d\theta/dz = -i/z\)
and when \(|z|=1\)
then \(\bar{z}=1/z\) and \(\cos \theta = (z+1/z)/2\).
Hence
\begin{align*}
\int_0^{2\pi} \frac{d\theta}{1 - 2\alpha \cos\theta + \alpha^2}
&= -i \int_{|z|=1} \bigl(1 - \alpha(z+1/z) + \alpha^2\bigr)^{-1} \bigm/ z\,dz \\
&= -i \int_{|z|=1} \bigl(z - \alpha(z^2+1) + z\alpha^2\bigr)^{-1}\,dz \\
&= (i/\alpha)  \int_{|z|=1} \left(z^2 - (\alpha+1/\alpha)z+1\right)^{-1}\,dz \\
&= (i/\alpha)  \int_{|z|=1} (z-\alpha)^{-1}(z-1/\alpha)^{-1}\,dz \\
&= (2\pi i^2/\alpha) \cdot \left\{%
   \begin{array}{ll}
   (1/\alpha - \alpha) \quad & \alpha > 1 \\
   (\alpha - 1/\alpha) \quad & \alpha < 1 \\
   \end{array}\right. \\
&= 2\pi / |\alpha^2 - 1|
\end{align*}


%%%%%%%%%%%%%% 12
\begin{excopy}
Compute
\begin{equation*}
\int_{-\infty}^\infty \left(\frac{\sin x}{x}\right)^2 e^{itx}\,dx
\qquad (\textnormal{for real }\; t).
\end{equation*}
\end{excopy}

Using Python and MatPlotLib, I have a conjecture, that the integral equals:
\begin{equation*}
\left\{\begin{array}{ll} %
\pi(2-|t|)/4 \qquad & |t| \leq 2 \\
0 & |t| \geq 2
\end{array}\right.
\end{equation*}

We follow the ideas of section~10.44.

The function \((\sin z/z)^2 e^{itz}\) has a removable singularity at \(z=0\)
and can be defined there as $1$ and be considered an entire function.
Since \(2i \sin z = e^{iz} - e^{-iz}\) we have
\begin{equation*}
-4\sin^2 z = \left(e^{iz} - e^{-iz}\right)^2 = e^{2iz} + e^{-2iz} - 2\,.
\end{equation*}
Thus
\begin{equation*}
(\sin z/z)^2 e^{itz} = -e^{i(t+2)z}/4z^2 - e^{i(t-2)z}/4z^2 + e^{itz}/2z^2\,.
\end{equation*}

Consider the path \(\Gamma_A\) from \(-A\) to \(-1\)
then lower unit circle and finally from $1$ to $A$.
Put
\begin{align*}
% \frac{1}{\pi}\varphi_A(s) = \itwopii \int_{\Gamma_A} \frac{e^{isz}}{z}\,dz\,.
\psi_s(z) &= \frac{e^{isz}}{z^2} \\
\varphi_A(s) 
  &= \int_{\Gamma_A} \frac{e^{isz}}{z^2}\,dz
  = \int_{\Gamma_A} \psi_s(z)\,dz\,.
  % = \int_{-A}^A \psi_s(x)\,dx\,. Not equal 
\end{align*}
Now
\begin{equation} \label{eq:10.12:sin2phis}
\int_{-A}^A \left(\frac{\sin x}{x}\right)^2 e^{itx}\,dx
% = -\varphi_A(t+2)/4 - \varphi_A(t-2)/4 + (1/2)\int_{\Gamma_A} z^{-2}\,dz
= \varphi_A(t)/2 - \bigl(\varphi_A(t+2)/4 - \varphi_A(t-2)\bigr)/4 
\end{equation}

We complete \(\Gamma_A\) into a closed path in two ways.
\begin{itemize}
\item With the lower half circle of radius $A$. Inside this closed path, 
      \(\psi_s(z)\) is analytic.
\item With the upper half circle of radius $A$. Inside this closed path, 
      \(\psi_A(z)\) has one pole at \(z=0\)
      with residue \(is\) (consider \(((isz)^1/1!)/z^2 = is/z\)).
\end{itemize}

For both path closing options, we use the substitution \(z = Ae^{i\theta}\)
and \(dz/d\theta = iAe^{i\theta}\)

Using the first choice of path closing gives:
\begin{align}
\varphi_A(s) 
&= \int_{|z|=A \wedge \Im(z)<0} \psi_s(z)\,dz
 = \int_{-\pi}^0 \psi_s(Ae^{i\theta}) \cdot iAe^{i\theta} \,d\theta \notag \\
&= (i/A) \int_{-\pi}^0 \exp(isAe^{i\theta})/e^{i\theta}\,d\theta.
   \label{eq:ex10.12:lowcirc}
\end{align}
Using the second choice with the residue in the pole gives:
\begin{align}
\varphi_A(s) 
&= 2\pi i\cdot \Res(\psi_s;0) - \int_{|z|=A \wedge \Im(z)>0} \psi_s(z)\,dz
 = -2\pi s - \int_0^\pi iAe^{i\theta} \psi_s(Ae^{i\theta})\,d\theta \notag \\
&= -2\pi s - (i/A) \int_{-\pi}^0 \exp(isAe^{i\theta})/e^{i\theta}\,d\theta.
   \label{eq:ex10.12:upcirc}
\end{align}
Since
\begin{equation*}
\left|\exp(isAe^{i\theta})\right| = \exp\bigl(-As\sin(\theta)\bigr),
\end{equation*}
and that is \(<1\) if $s$ and \(\sin(\theta)\) of the same sign.
Hence as \(A\to\infty\)
the integral of \eqref{eq:ex10.12:lowcirc} when \(s<0\)
and the integral of \eqref{eq:ex10.12:upcirc} when \(s>0\)
converges to~$0$.
Therefore
\begin{equation}
\lim_{A\to\infty} \varphi_A(s) = 
 \left\{\begin{array}{ll}%
 -2\pi s \qquad & \textnormal{if\ }\; s > 0\\
 0       \qquad & \textnormal{if\ }\; s \leq 0
 \end{array}
 \right. \label{eq:10.12:varphiA}
\end{equation}

\iffalse
We compute the rational integral term of \eqref{eq:10.12:sin2phis}.
Within the real axis
\begin{equation*}
\int_{\R\setminus(-1,1)} \frac{dx}{x^2} 
 = 2\int_1^{\infty} x^{-2}\,dx 
 = 2\left(-x^{-1}\right)\bigm|_1^\infty = 2\bigl(0-(-1)\bigr)=2
 \end{equation*}
and within the arc part
\begin{align*}
\int_{-\pi}^0 e^{-2i\theta}\frac{dz}{d\theta}\,d\theta 
&= i \int_{-\pi}^0 e^{-2i\theta}\cdot e^{i\theta}\,d\theta 
 = i \int_{-\pi}^0 e^{-i\theta}\,d\theta
 = i\left.\left(\frac{1}{-i}e^{-i\theta}\right)\right|_{-\pi}^0 \\
&= -\left.e^{-i\theta}\right|_{-\pi}^0 = % BADBAD 1 - \bigl(-(-1)\bigr) = 0.
   -\bigl(1 - (-1)\bigr) = -2
\end{align*}
Adding to:
\begin{equation}
\int_{\Gamma_A} \frac{dz}{z^2}
 = \int_{\R\setminus(-1,1)} \frac{dx}{x^2} 
   + \int_{-\pi}^0 \exp(isAe^{i\theta})e^{i\theta}\,d\theta \\
 = 2 - 2 = 0 \label{eq:10.12:intrat}
\end{equation}
\fi % false

Applying \eqref{eq:10.12:varphiA} % and \eqref{eq:10.12:intrat} 
to \eqref{eq:10.12:sin2phis} gives
\begin{align*}
\int_{-\infty}^\infty (\sin x/x)^2 e^{itx}\,dx
&= \lim_{A\to\infty} 
   \left(\varphi_A(t) -\bigl(\varphi_A(t+2) + \varphi_A(t-2)\bigr)/4)
   \right)\\
&= \left\{\begin{array}{ll}%
   0 + 0 + 0  \qquad & t \leq -2 \\
   0 + 2\pi(t+2)/4 + 0 \qquad & -2 \leq t \leq 0 \\
   -2\pi t/2 + 2\pi(t+2)/4 + 0 \qquad & 0 \leq t \leq 2 \\
   -2\pi t/2 + 2\pi(t+2)/4 + 2\pi(t-2)/4 \qquad & t \geq 2
   \end{array}\right. \\
&= \left\{\begin{array}{ll}%
   0  \qquad & t \leq -2 \\
   \pi t/2+\pi \qquad & -2 \leq t \leq 0 \\
   -\pi t/2 + \pi \qquad & 0 \leq t \leq 2 \\
   0 \qquad & t \geq 2
   \end{array}\right. \\
&= \max\left(0, \pi-|\pi t/2|\right)
\end{align*}

%%%%%%%%%%%%%% 13
\begin{excopy}
Compute
\begin{equation*}
\int_0^\infty \frac{dx}{1 + x^n} \qquad (n=2,3,4,\ldots).
\end{equation*}
[For even $n$, the method of Exercise~8 can be used.
However, a different path can be chosen, which simplifies the computation
and which also works for odd $n$:
from $0$ to $R$ to \(R = \exp(2\pi i/n)\) to $0$.]
\\
\phantom{AAAA}\emph{Answer:} \((\pi/n)\sin(\pi/n)\).
\end{excopy}

Put 
\begin{equation*}
f(z) = \frac{1}{1 + z^n}.
\end{equation*}
Assume first that n is even (\(n/2\in\Z\)) and \(n\geq 2\).
The roots of \(z^n+1\) are \mbox{\(\{e^{\pi i(2k+1)/n}\!: k\in \Z_n\}\)}.
Put \(\alpha_k = e^{\pi i(2k+1)/n}\) for \(k\in\Z_n\).
The roots in the upper place are
 \(\{\alpha_k: k\in \Z_{n/2}\}\).
Using the result of exercise~8 above and the fact that the integrand is 
an even function, we have
\begin{align*}
\int_0^\infty \frac{dx}{1 + x^n}
&= \half\cdot \int_{-\infty}^\infty \left(1 + x^n\right)^{-1}\,dx
 = \pi i \cdot \sum_{k\in\Z_{n/2}} 
        \Res\left( (1 + z^n)^{-1}; \alpha_k \right) \\
&= \pi i \cdot \sum_{k\in\Z_{n/2}}\; 
      \prod_{j\in \Z_n\setminus\{k\}} \left(\alpha_k - \alpha_j)\right)^{-1}
\end{align*}

\iffalse
Let's compute the residue of the ``first'' pole.
\begin{align*}
\Res(f, \alpha_0) 
&= \prod_{j\in \Z_n\setminus\{0\}} (\alpha_0 - \alpha_j)^{-1}
 = \lim_{z\to\alpha_0} \frac{z-\alpha_0}{\prod_{j\in \Z_n} (z - \alpha_j)}
 = \lim_{z\to\alpha_0} \frac{z-\alpha_0}{z^n+1} 
 = \lim_{z\to\alpha_0}\frac{1}{nz^{n-1}} \\
&= e^{\pi i/n}/n
\end{align*}
\fi 

Let's compute the residue of a pole.
\begin{align*}
\Res(f, \alpha_k) 
&= \prod_{j\in \Z_n\setminus\{k\}} (\alpha_k - \alpha_j)^{-1}
 = \lim_{z\to\alpha_k} \frac{z-\alpha_k}{\prod_{j\in \Z_n} (z - \alpha_j)}
 = \lim_{z\to\alpha_k} \frac{z-\alpha_k}{z^n+1} 
 = \lim_{z\to\alpha_k}\frac{1}{nz^{n-1}} \\
&= \alpha_k^{-(n-1)}/n
 %= \left(\alpha_k/\alpha_k^n\right)\bigm/n
 = \alpha_k^{-n}\alpha_k/n = -e^{(2k+1)\pi i/n}/n
% = e^{((2k+1)\pi i/n)(1-n))}/n
% = e^{(2k+1)\pi i(1-n)/n)}/n
\end{align*}
In particular 
\begin{equation*}
\Res(f, \alpha_0) = -\alpha_0/n0) = -e^{\pi i /n}/n.
\end{equation*}

Using the hint, We consider the closed path \(\Gamma_R\)
which is the sum of
\begin{equation*}
\begin{array}{ll}
\gamma_1:[0,R]\to\C & \gamma_1(t) = t \\
\gamma_2:[0,2\pi/n]\to\C \qquad& \gamma_2(t) = e^{it} \\
\gamma_3:[0,R]\to\C & \gamma_3(t) = e^{2\pi i/n}(R-t) 
\end{array}
\end{equation*}

Note that if \(z\in \gamma_3^*\) then 
\(z^n = |z|^n\) and so \((1+z^n)^{-1} = (1+|z|^n)^{-1}\).
For any \(R>1\) we have
\begin{equation*} 
\int_{\Gamma_R} \frac{dz}{1+z^n} = 2\pi i \Res(f;\alpha_0) 
= - 2\pi i e^{-\pi i/n}/n.
\end{equation*}
Now
\begin{align*}
\lim_{R\to\infty} \int_{\Gamma_R} \frac{dz}{1+z^n}
&= \lim_{R\to\infty} 
   \left( \int_{\gamma_1}\cdots +\int_{\gamma_2}\cdots +\int_{\gamma_3}\cdots \right)
 = \lim_{R\to\infty} \left( \int_{\gamma_1}\cdots +\int_{\gamma_3}\cdots \right) \\
&= \lim_{R\to\infty} \left( 
        \int_0^R (x^n+1)^{-1}\,dx +
        \int_0^R \left(\bigl((R-x)e^{2\pi i /n}\bigr)^n+1\right)^{-1}
                 \frac{dz}{dx}
                 \,dx\right) \\
&= (1 - e^{2\pi i /n}) \lim_{R\to\infty}  \int_0^R (x^n+1)^{-1}\,dx
\end{align*}
Hence 
\begin{align*}
 \int_0^\infty (x^n+1)^{-1}\,dx 
 &= \frac{2\pi i \left(-e^{\pi i/n}/n\right)}{1 - e^{2\pi i /n}}
  = \frac{-2\pi i }{ne^{-\pi i/n}(1 - e^{2\pi i /n})}
  = \frac{2\pi i }{n(e^{\pi i/n} - e^{-\pi i /n})} \\
 &= \frac{\pi}{n(e^{\pi i/n} - e^{-\pi i /n})/(2i)}
  = \frac{\pi}{n\cdot\sin(\pi/n)} \\
 &= (\pi/n)\bigm/\sin(\pi/n).
\end{align*}


%%%%%%%%%%%%%% 14
\begin{excopy}
Suppose \(\Omega_1\) and \(\Omega_2\) are plane regions, $f$ and $g$ are
nonconstant functions defined on
\(\Omega_1\) and \(\Omega_2\), respectively,
and \(f(\Omega_1) \subset \Omega_2\).
Put \(h = g\circ f\).
If $f$ and $g$ are holomorphic, we know that $h$ is  holomorphic.
Suppose we know that $f$ and $h$ are holomorphic.
Can we conclude anything about $g$?
What if we know that $g$ and $h$ are holomorphic?
\end{excopy}

In the second case we \emph{cannot} know much about $g$.
Take \(f(z)=0\) and \(g(z)=|z|\).
Clearly \(h=0\) and $g$ is not holomorphic.
A More interesting question would be if we also assume that 
$F$ is \emph{onto} \(\Omega_2\).


In the second case we \emph{cannot} know much about $f$.
For example, take \(g=0\) and \(h=0\) constant functions, 
and the equality holds for any \(f \in \Omega_2^{\Omega_1}\).

%%%%%%%%%%%%%% 15
\begin{excopy}
Suppose \(\Omega\) is a region, \(\varphi \in H(\Omega)\),
\(\varphi'\) has no zero in \(\Omega\),
\(f\in H(\varphi(\Omega))\),
\(g = f\circ \varphi\), \(z_0\in\Omega\), and
\(w_0 = \varphi(z_0)\).
Prove that if $f$ has a zero of order $m$ at \(w_0\),
then $g$ also has a zero of order $m$ at \(z_0\).
How is this modified if \(\varphi'\) has a zero of order $k$ at \(z_0\)?
\end{excopy}

Clearly \(g'(z_0) = f'(w_0)\varphi'(z_0)\).
Then  both \(g'\) and \(f'\) 
has zero of order \(m-1\) in \(w_0\) and \(z_0\) respectively.
Hence  $f$ has a zero of order~$m$ at~\(w_0\).
It is now easy to see that the zero order of~\(\varphi\) at~\(z_0\)
adds to the order $g$ at~\(z_0\).

%%%%%%%%%%%%%% 16
\begin{excopy}
Suppose \(\mu\) is a complex measure on a measure space~$X$,
\(\Omega\)~is an open set in the plane,
\(\varphi\)~is a bounded function on \(\Omega\times X\)
such that \(\varphi(z,t)\) is a measurable function of $t$,
for each \(z \in \Omega\), and \(\varphi(z,t)\) is holomorphic in \(\Omega\),
for each \(t\in X\). Define
\begin{equation*}
f(x) = \int_X \varphi(z,t)\,d\mu(t)
\end{equation*}
for \(z\in\Omega\). Prove that \(f\in H(\Omega)\).
\emph{Hint:} Show that to every compact \(K \subset \Omega\) there corresponds
a constant \(M<\infty\) such that
\begin{equation*}
\left| \frac{\varphi(z,t) - \varphi(z_0,t)}{z - z_0}\right| < M
\qquad (z \;\textnormal{ and }\; z_0\in K,\; t\in X).
\end{equation*}
\end{excopy}

Obviously, in the above formula we assume \(z\neq z_0\).
Let $U$ be some upper bound of \(|\varphi|\).
For all~\(t\in X\),
denote the holomorphic functions \(\varphi_t(z) = \varphi(z,t)\)
and the ratios
\begin{equation*}
\Delta_t(z_1,z_0) = \frac{\varphi_t(z_1) - \varphi_t(z_0)}{z_1 - z_0}
\qquad (\textnormal{for all distinct}\; z_0,z_1\in\Omega).
\end{equation*}
Recall that \(D'(a;r) = \{z: 0 < |z-a|<r\}\).

Let \(K\subset \Omega\) be a compact set.
Pick some arbitrary \(w\in K\).
Then we can find some \(r>0\) such that \(\overline{D(w;r)}\subset \Omega\).
We pick \(v_0\in D(w,r/2)\) and
we will estimate \(|\Delta_t(z,v_0)|\) for all \(z\in\Omega\). 

\paragraph{Local case.}
Clearly \(D(v_0;r/2)\subset D(w;r)\subset\Omega\) and
\(|{\varphi_t}'(v_0)| \leq 2U/r\) for all \(t\in X\) by Theorem~10.26.
% This shows that \(|\Delta_t(w,v_0)| \leq 2U/r\) 
for all \(t\in X\).

Pick \(v_1\in D'(v_0;r/2)\). 
Use the path segment 
\(\gamma(\tau) = v_0 + \tau(v_1-v_0)\) with \(\tau\in[0,1]\).
Then
\begin{equation*}
\varphi_t(v_1) 
= \varphi_t(v_0) + \int_\gamma {\varphi_t}'(z)\,dz
= \varphi_t(v_0) + \int_0^1 {\varphi_t}'(\gamma(\tau))\gamma'(\tau)\,d\tau
\end{equation*}
Hence
\begin{equation*}
\left|\varphi_t(v_1) - \varphi_t(v_0)\right|
\leq \int_0^1 \left|{\varphi_t}'(\gamma(\tau))\gamma'(\tau)\right|\,d\tau
\leq 2|v_1-v_0|U/r.
\end{equation*}
Subsequently
\begin{equation} \label{eq:ex10.16:estimate:in}
|\Delta_t(v_1,v_0)| \leq 2U/r 
\qquad (\textnormal{for all} v_0\in D(w;r/2),\;v_1 \in D(v_0;r/2)\,).
\end{equation}

\paragraph{Outside case.}
When \(v_1\in\Omega \setminus D(v_0;r/2)\)
\begin{equation} \label{eq:ex10.16:estimate:out}
|\Delta_t(v_1,v_0)| 
\leq \frac{|\varphi_t(w) - \varphi_t(v)|}{|r/2|} \leq 4U/r.
\end{equation}

To combine the case, define
\begin{equation*}
\Omega_M := 
\left\{z\in\Omega: 
  \forall t\in X, \forall \zeta\in\Omega\setminus\{z\},\; |\Delta_t(z,\zeta)|<M
\right\}.
\end{equation*}

From \eqref{eq:ex10.16:estimate:in} and \eqref{eq:ex10.16:estimate:out}
we know that there exists \(M = M_w\) (for example \(M_w = 4U/r + 1\))
such that 
\begin{equation*}
D(w;r/2) \subset \Omega_M
\end{equation*}

Since $w$ was arbitrarily chosen, \(\Omega \subset \cup_{M\in\N}\Omega_M\)
and since $K$ is compact we can find some \(M<0\) such that 
\(|\Delta_t(z_1,z_0)|<M\) for all \(t\in X\) and all distinct \(z_0,z_1\in K\).
Thus establishing the hint.

In order for $f$ to be differentiable at $z$, it is sufficient that
for any sequence \(\{z_n\}_{n\in\N}\) there exists a limit 
\begin{equation*}
f'(z) = \lim_{n\to\infty} \frac{f(z_n) - f(z)}{z_n - z} 
 \qquad (\text{where}\; z_n \neq z)
\end{equation*}
Now
\begin{equation*}
\frac{f(z_n) - f(z)}{z_n - z} 
= \frac{\int_X \varphi(z_n,t)\,d\mu - \int_X \varphi(z,t)\,d\mu}{z_n - z} \\
= \int_X \frac{\varphi(z_n,t) -  \varphi(z,t)}{z_n - z}\,d\mu
\end{equation*}

There exists some \(\delta>0\) 
such that \({D(z,\delta)}\subset\Omega\).
For sufficient large $m$, for all \(n>m\) 
we have \(z_n\in \overline{D(z,\delta/2)}\) and by its compatcness
and the established hint, \(\Delta_t(z_n,z)\) are bounded
and we can apply 
\index{Lebesgue}
Lebesgue's Dominated Theorem~1.34 that here gives 
\begin{equation*}
f'(z) 
= \lim_{n\to\infty} \int_X \Delta_t(z_n,t)\,d\mu
= \int_X \lim_{n\to\infty} \Delta_t(z_n,t)\,d\mu
= \int_X \varphi_t'(z)\,d\mu.
\end{equation*}

%%%%%%%%%%%%%% 17
\begin{excopy}
Determine the regions in which the following functions are defined
and holomorphic:
\begin{equation*}
f(z) = \int_0^1 \frac{dt}{1+tz}, \qquad
g(z) = \int_0^\infty \frac{e^{tz}}{1+t^2}\,dt, \qquad
h(z) = \int_{-1}^1 \frac{e^{tz}}{1+t^2}\,dt\,.
\end{equation*}
\emph{Hint}: either use Exercise~16, or combine
\index{Morera}
Morera's theorem with
\index{Fubini}
Fubini's.
\end{excopy}

\begin{itemize}

\item The function \(f(z)\) is holomorphic on any region \(\Omega\) for which 
\begin{equation*}
\{1/(1+tz): t\in[0,1]\;\wedge\; z\in\Omega\}
\end{equation*}
is bounded. This happens when there exists \(r > 0\) such that 
\begin{equation*}
\Omega \subset \{w\in\C: \forall z\in (-\infty,-1],\, |w-z|>r\}.
\end{equation*}

\item
The function \(g(z)\) is holomorphic on any region \(\Omega\) 
such that \(\Re(z)\leq 0\) for all \(z\in\Omega\).
Assume \(\Omega\) satisfies this condition.
Define
\begin{equation*}
g_n(z) = \int_0^n \frac{e^{tz}}{1+t^2}\,dt, \qquad
\end{equation*}
By previous exercise, \(g_n\) are holomorphic on \(\Omega\).
\index{Lebesgue}
By Lebesgue's Dominated Theorem~1.34 
\(\lim_{n\to\infty} g_n(z) = g(z)\) and it is easy to see that 
the convergence is uniform on any bounded subset of \(\Omega\),
in particular on triangle boudnaries. Thus the condition
\index{Morera}
of Morera's Theorem~10.17 holds, hence \(g\in H(\Omega)\). 

\item The function \(h(z)\) is holomorphic on any region \(\Omega\) for which 
\begin{equation*}
\{\exp(tz): t\in[-1,1] \;\wedge\; z\in\Omega\}
\end{equation*}
is bounded. This happens when there exists some \(M<\infty\) such that 
\begin{equation*}
\forall w\in\Omega,\;\Re(w) \leq M.
\end{equation*}

\end{itemize}

%%%%%%%%%%%%%% 18
\begin{excopy}
Suppose \(f\in H(\Omega)\),
\(\overline{D}(a;r)\subset \Omega\),
\(\gamma\) is a positively oriented circle with center at $a$ and radius $r$,
and $f$ has no zero on \(\gamma^*\). For \(p=0\), the integral
\begin{equation*}
\itwopii \int_\gamma \frac{f'(z)}{f(z)} z^p\,dz
\end{equation*}
is equal to the number of zeros of $f$ in \(D(a;r)\).
What is the value of this integral (in terms of zeros of $f$)
for \(p=1,2,3,\ldots\)?
What is the answer if \(z^p\) is replaced by any \(\varphi\in H(\Omega)\)?
\end{excopy}

For \(p = 0\) the claim is shown in Theorem~10.43\ich{a}.
Otherwise the number it still gives the number of zeros,
except that if $f$ has zero of order $k$ in \(z=0\) then 
the number is reduced by \(\min(p, k)\).
Similarly of \(\varphi\) has zeros of order \(p_j\) on \(z_j\)
where $f$ has zeros of order \(k_j\)
then the number is reduced by \(\sum_j \min(p_j,k_j)\).


%%%%%%%%%%%%%% 19
\begin{excopy}
Suppose \(f\in H(U)\), \(g\in H(U)\),
and neither $f$ nor $g$ has a zero in $U$. If
\begin{equation*}
\frac{f'}{f}\left(\frac{1}{n}\right) =
\frac{g'}{g}\left(\frac{1}{n}\right)
\qquad (n=1,2,3,\ldots)
\end{equation*}
find another simple relation between $f$ and $g$.
\end{excopy}

Both \((f'/f), (g'/g)\in H(U)\).
By the Corollary to Theorem~10.18 we have \(f'/f = g'/g\) in \(H(U)\).
Using line segments from $0$ to \(z=te^{i\theta}\in U\) 
as the path: \(\gamma(\tau) = \tau\cdot e^{i\theta}\) with \(0\leq \tau\leq t\), 
Define
\begin{equation*}
F(z) 
= \int_0^t (f'/f)(\tau e^{i\theta})\cdot e^{i\theta}\,d\tau
= e^{i\theta} \int_0^t \frac{d(\log\circ f)}{d\tau}(\tau e^{i\theta})\,d\tau
\end{equation*}
Similarly, we define \(G(z)\) by integrating \(g/g'\).
Clearly $F$ and $G$ differ by a constant, say \(c=F(z)-G(z)\). Hence
\begin{equation*}
(\log\circ f)(z) = (\log\circ g)(z) + c
\end{equation*}
By taking exponents, we get the relation
\begin{equation*}
f(z) = e^c \cdot g(z).
\end{equation*}

%%%%%%%%%%%%%% 20
\begin{excopy}
Suppose \(\Omega\) is a region,
\(f_n\in H(\Omega)\) for \(n=1,2,3,\ldots\),
none of the functions \(f_n\) has a zero in \(\Omega\),
and \(f_n\) converges to $f$ uniformly on compact subsets of \(\Omega\).
Prove that either $f$ has no zero in \(\Omega\) or \(f(z)=0\)
for all \(z\in\Omega\).
\end{excopy}

Consider \(N = f^{-1}(0) \subset\Omega\).
If \(N=\emptyset\) or \(N = \Omega\) we are done.

By negation assume \(\emptyset \subsetneq N \subsetneq \Omega\).
By Theorem~10.18 there exists an isolated zero.
Hence we have 
\(B(z;r) \subset \Omega\setminus N\) for some \(z\in N\) and \(r>0\).
We look at the circle path \(\gamma\) whose image \(\gamma^*\) 
is the boundary of \(\overline{B(z;r/2)}\). 
By Theorem~10.28 the derivatives 
\({f_n}\) converge uniformly to \(f'\) on \(\gamma*\).
But now Theorem~10.43\ich{a} gives the following
\begin{equation*}
0 = \lim_{n\to\infty} N_{f_n} = N_f = 1
\end{equation*}
contradiction.


%%%%%%%%%%%%%% 21
\begin{excopy}
Suppose \(f\in H(\Omega)\), \(\Omega\) contains the closed unit disc, and
\(|f(z)| < 1\) if \(|z|=1\).
How many fixed points must $f$ have in the disc?
That is, how many solutions does the equation \(f(z)=z\) have there?
\end{excopy}

By looking at \(h(z) = z/2\) we can see that the minimal number of
solutions for such $f$ cannot exceed $1$.

Let \(U_0 = \{z\in\C: |z|\leq 1\}\) be the unit circle.
Define by induction \(U_k = f(U_{k-1})\) for \(k\geq 1\).
Also by induction we can see that \(U_k \subset U_{k-1}\).
These are compact sets and thus have non empty intersection
\(X = \cap_{k\in\Z^+} U_k\). Clearly \(f(X) = X\).

\index{maximum modulus}
By the maximum modulus Theorem~10.24
\index{Cauchy's estimates}
and Cauchy's estimates Theorem~10.26, we have \(|f'(z)|<1\) for all \(z\in U\).
Moreover, \(s = \max_{z\in U}|f'(z)|<1\).
Hence the diameter of the compact sets \(\{U_k\}_{k\in\Z^+}\) is decreasing,
such that \(\diam(U_k) \leq s^k\). Hence $X$ must be a singleton.

%%%%%%%%%%%%%% 22
\begin{excopy}
Suppose \(f\in H(\Omega)\), \(\Omega\) contains the closed unit disc,
\(|f(z)| > 2\) if \(|z|=1\), and \(f(0)=1\).
Must $f$ have a zero in the unit disc?
\end{excopy}

If by negation there was no zero, then \(g = 1/f \in H(\Omega)\)
and 
\begin{equation*}
|g(0)| = 1 > 1/2 > \max\{|f(z)|: |z|=1\}
\end{equation*}
that contradicts the
\index{Maximum Modulus}
Maximum Modulus Theorem~10.24.

%%%%%%%%%%%%%% 23
\begin{excopy}
Suppose \(P_n(z) = 1 + z/1! + \cdots + z^n/n!\),
\(Q_n(z) = P_n(z) - 1\), where \(n=1,2,3,\ldots\),
% none of the functions
What can you say about the location of the zeros
of \(P_n\) and \(Q_n\) for large $n$?
Be as specific as you can.
\end{excopy}

The polynomials \(P_n\) have zeros that converge to infinity.
More accurately, for each \(R<0\) there exists some \(m<\infty\)
such that \(P_n\) has \emph{no} zeros in \(B(0;R)\) for all \(n>m\).
Otherwise, there would be an increasing sub-sequence of indices 
\(\{s_n\}_{n\in\N}\) and zeros \(\{z_n\}_{n\in\N}\) in \(B(0;R)\) 
such that \(P_{s_n}(z_n) = 0\) and \(\lim_{n\to\infty} z_n = z\in B(0;R)\).
Now \(\lim_{n\to\infty} P_{s_n}(z) = \exp(z)\) uniformly on \(B(0;R)\)
which leads to the \(\exp(z) = 0\) contradiction.

In addition to \(Q_n(0)=0\) for all \(n\in\N\),
the polynomials \(Q_n\) have zeros that converge to \(\{2\pi i k: \;k\in\Z\}\).
But the other roots diverge. More accurately,
for all \(R<0\) and \(r>0\)  there exists some \(m<\infty\)
such that \(Q_n\) has \emph{no} zeros in
\begin{equation*}
\{z\in\C:\; |z|\leq R \;\wedge\; \Re(z)\geq r\}
\end{equation*}
for all \(n>m\).


%%%%%%%%%%%%%% 24
\begin{excopy}
Prove the following form of
\index{Rouche}
Rouche's theorem:
Let \(\Omega\) be the interior of a compact set $K$ in the plane.
Suppose $f$ and $g$ are continuous on $K$ and holomorphic in \(\Omega\),
and \(|f(z)-g(z)|<|f(z)|\) for all \(z\in K\setminus \Omega\).
Then $f$ and $g$ have the same number of zeros in \(\Omega\).
\end{excopy}

Let $D$ be a connected component of  \(\Omega\).
We will show that $f$ and $g$ have the same number of zeros in $D$.
If by negation $f$ has have infinite number of zeros in any such $D$,
then then \(f_{\restriction D}=0\) and by continuity
 \(f_{\restriction \partial D}\) but this contradicts the assumption 
that \(|f(z)-g(z)|<|f(z)|\) for all \(z\in K\setminus \Omega\).
Thus the number of zeros of $f$ in $D$ is finite.
Once we establish that this number of zeros of $g$ in $D$ is the same,
we can sum up the zeros in all such components and get the desired
result. Let \(N = \{z\in D:\; f(z) = 0\}\).

The boundary \(\partial D \subset K\). For any \(\epsilon>0\),
by introducing sufficiently small squares,
we can find a path \(\gamma_\epsilon\) that is ``\(\epsilon\)-near'' the boundary.
More accurately, \(d(z,K)<\epsilon\) for all \(z\in \gamma_\epsilon^*\).

Since the number of zeros of $f$ in $D$ is finite, we can pick 
\(\epsilon < d(N,\partial D)\).
Hence \(f(z)\neq 0\) for all \(z\in \gamma_\epsilon^*\).
Since $f$ and $g$ are continuous, we can further require \(\epsilon\)
to sufficiently small such that the inequality \(|f(z)-g(z)|<|f(z)|\)
also holds for all \(z\in\gamma_\epsilon^*\).
The needed step to complete is to apply Theorem~10.43\ich{b}
which shows that the numbers of zeros of $f$ and $g$ in $D$ are the same.


%%%%%%%%%%%%%% 25
\begin{excopy}
Let $A$ be the
\index{annulus}
annulus \(\{x: r_1 < |z| < r_2\}\), where \(r_1\) and \(r_2\) are given
positive numbers.
\begin{itemize}

\itemch{a} Show that the Cauchy formula
\begin{equation*}
f(z) = \itwopii \left(\int_{\gamma_1} + \int_{\gamma_2}\right)
       \frac{f(\zeta)}{\zeta - z}\,d\zeta
\end{equation*}
is valid under the following conditions: \(f\in H(A)\),
\begin{equation*}
r_1 + \epsilon < |z| < r_2 - \epsilon,
\end{equation*}
and
\begin{equation*}
\gamma_1(t) = (r_1+\epsilon)e^{-it},
\qquad
\gamma_2(t) = (r_2-\epsilon)e^{it},
\qquad
(0\leq t \leq 2\pi).
\end{equation*}

\itemch{b} Show by means of \ich{a} that every \(f\in H(A)\)
can be decmposed info a sum \(f=f_1+f_2\), when \(f_1\) is holomorphic
outside \(\overline{D}(0;r_1)\)
and  \(f_2 \in H(D(0;r_1))\).
The decomposition is unique if we require that 
\(f_1(z)\to 0\) as \(|z|\to\infty\).

\itemch{c} Use this decomposition to associate with each \(f\in H(A)\) its
so-called
\index{Laurent series}
``Laurent series''
\begin{equation*}
\sum_{-\infty}^\infty c_n z^n
\end{equation*}
which converges to $f$ in $A$. Show that there is only one such series for 
each $f$. Show that it converges to $f$ uniformly on compact subsets of $A$.

\itemch{d} If \(f\in H(A)\) and $f$ is bounded in $A$, show that the components
\(f_1\) and \(f_2\) are also bounded.

\itemch{e} How much of the foregoing can you extend to the case \(r_1=0\)
(or \(r_2=\infty\), or both)?

\itemch{f} How much of the foregoing can you extend  to region bounded 
by finitely many (more than two) cycles?
\end{itemize}
\end{excopy}

See \cite{Gamelin2003} Chapter~VI, Section~1.
\begin{itemize}

\itemch{a}
This result follows by applying \index{Cauchy} Cauchy's Theorem~10.35
for \(\Gamma = \gamma_1 + \gamma_2\).

\itemch{b}
We define
\begin{align*}
f_1(z) &= \itwopii \int_{\gamma_1} \frac{f(\zeta)}{\zeta - z}\,d\zeta \\
f_2(z) &= \itwopii \int_{\gamma_2} \frac{f(\zeta)}{\zeta - z}\,d\zeta\,.
\end{align*}
Clearly \(f = f_1 +f_2\) and \(\lim_{z\to\infty} f_1(z) = 0\).
Say  \(f = g_1 +g_2\) and \(\lim_{z\to\infty} g_1(z) = 0\).
Then \(f_1 - g_1 = g_2 - f_2\) in $A$.
The function \(f_1 - g_1\) is holomorphic outside of \(\gamma_1\)
and the function \(g_2 - f_2\) is holomorphic inside \(\gamma_2\).
Since they agree on the intersection, we have an entire function $h$
\begin{align*}
h(z) &= g_2(z) - f_2(z) \qquad (|z| < r_1) \\
h(z) &= f_1(z) - g_1(z) \qquad (|z| < r_2)
\end{align*}
But \(\lim_{z\to\infty} h(z) = 0\), hence \(h(z)=0\) for all \(z\in\C\)
and the uniqueness follows.

\itemch{c}
We have
\begin{equation*}
f_2(z) = \sum_{n=0}^\infty c_n z^n
\end{equation*}
We can compute the coefficients \(c_n\) using Theorem~10.7
\begin{equation*}
c_n = \itwopii \int_{\gamma_2} \frac{f(\zeta)}{\zeta^{n+1}} \qquad (n\geq 0).
\end{equation*}

Since \(\lim_{z\to\infty} f_1(z) = 0\), we can put \(z=1/w\) and define
\begin{equation*}
g_1(w) = f(1/w) = f(z) \qquad (|z|>r  \Leftrightarrow |w|<1/r)
\end{equation*}
and with \(g_1(0)=0\), we have \(g_1 \in H(\{w\in\C: |w|<1/r\})\).
Hence it as a power series 
\begin{align*}
g_1(w) &= \sum_{n=1}^\infty b_n w^n \\
f_1(z) &= g_1(1/z) = \sum_{n-1}^\infty b_n z^{-n}
\end{align*}
By putting \(c_{-n} = b_n\) we get the desired representation for $f$.

For each \(n\in\Z\), we consider \(f(z)/z^{n+1}\). Now for any sub-annulus of $A$
we can find sufficiently small \(\epsilon>0\) such that it is contained
between the circles of \(\gamma_1^*\) and  \(\gamma_2^*\) and we can compute
\begin{equation*}
\int_{\gamma_1^*} \frac{f(z)\,dz}{z^{n+1}}
 = \int_{\gamma_1^*} \frac{1}{z^{n+1}}\sum_{n\in\Z} c_k z^k\,dz
 = \sum_{n\in\Z} c_k \int_{\gamma_1^*} z^{k-n-1}\,dz
 = 2\pi i c_{n}
\end{equation*}
Since the last integral vanishes whenever \(k-n-1\neq -1\).
Hence
\begin{equation*}
c_n = \itwopii \int_{\gamma_1^*} \frac{f(z)\,dz}{z^{n+1}}.
\end{equation*}

By the arguments of section~10.5, 
the power series of
\begin{equation*}
f_2(z) = \sum_{n=0}^\infty c_n z^n
\end{equation*}
converges absolutely in \(\overline{D(0;\rho)}\) for every \(\rho<r_2\).
Similarly by looking on \(g_1\), the power series of 
\begin{equation*}
f_1(z) = \sum_{n=-1}^{-\infty} c_n z^n
\end{equation*}
converges absolutely in \(\{z\in\C: |z|\geq \rho\}\) for every \(\rho> r_1\).
Now for every compact \(K\subset A\) we can find 
\(\rho_1\) and \(\rho_2\) such that
\begin{equation*}
K \subset \{z\in\C: \rho_1 \leq |z| \leq \rho_2\}
\qquad \textnormal{and}\qquad 
r_1 < \rho_1 < \rho_2 < r_2\
\end{equation*}
and clearlt the whole power series converges absolutely in $K$.x

If by negation there was another power series, 
then by  positive and negative powers split, we would get 
another decomposition for $f$ that contradicts the uniqueness established 
in~\ich{b}.

\itemch{d}
Put \(r_3 = (r_1+r_2)/2\). 
Clearly \(f_2\) is bounded in~\(B_2 = \overline{D(0;r_3)}\)
and  \(f_1\) is bounded in~\(B_1 = \C\setminus D(0;r_3)\).
If \(f = f_1+f_f2\) is bounded in $A$, then 
\(f_1\) must be bounded in \(A\setminus B_2\) and
\(f_2\) must be bounded in \(A\setminus B_1\).
Consequently, \(f_1\) and \(f_2\) are bounded on~$A$.

\itemch{e}
The case \(r_1=0\) and \(r_2=1\) applies for the function 
\begin{equation*}
f(z) = \sum_{n\geq -1} z^n = 1/z + 1/(1-z)
\end{equation*}

The cases of \(r_2=\infty\) requires different interpration of ``\(\epsilon\)''
for selecting \(\gamma_2\). We can use the following condition;
for any \(M<\infty\), let \(\gamma_2(t) = Me^{it}\) for \(0\leq t\leq 2\pi\).
Function that can be applied for such case have essential singularity 
in infinity, for example \(f(z) = \sin(z)\).
This can be combined with \(r_1=0\) as well (add \(1/z\)).

\itemch{f}
With $n$ circles all centered at the origin, there are \(n-1\) annulus regions.
All the above applies to each such annulus.

\end{itemize}

%%%%%%%%%%%%%% 26
\begin{excopy}
It is required to extend the function 
\begin{equation*}
\frac{1}{1-z^2} + \frac{1}{3-z}
\end{equation*}
in the series of the form \(\sum_{-\infty}^\infty c_n z^n\).

How many such expansions are there?
In which region is each of them valid?
Find the coefficients \(c_n\) explicitly for each of these expansions.
\end{excopy}

We use the identity:
\begin{equation*}
f(z) = \frac{1}{1-z^2} + \frac{1}{3-z}
= \frac{1}{2}\left(\frac{1}{z-1} - \frac{1}{z+1}\right) + \frac{1}{3-z}
\end{equation*}

Consider the following case:
\begin{alignat*}{2}
\frac{1}{z-1} &= \sum_{n=0}^\infty z^n && \qquad (|z|<1) \\
\frac{1}{z+1} &= \sum_{n=0}^\infty (-1)^n z^n  && \qquad (|z|<1) \\
\frac{1}{3-z} &= \sum_{n=0}^\infty 3^{-(n+1)} z^n  && \qquad (|z|<3) \\
\frac{1}{z-1} 
 &= \frac{1}{z(1-(1/z))} = 
  \sum_{n<0} z^n && \qquad (|z|>1) \\
\frac{1}{z+1} 
 &= \frac{1}{z(1-(-1/z))} 
 = \sum_{n<0} (-1)^{n+1} z^n && \qquad (|z|>1) \\
\frac{1}{3-z} 
  &= \frac{-1}{z} \cdot \frac{1}{1-3/z} 
  = \sum_{n<0}^\infty (-1)^{n+1}\cdot 3^{n} z^n  && \qquad (|z|>3)
\end{alignat*}

Gathering the results:
\begin{alignat*}{2}
f(z) &= \sum_{n=-1}^{-\infty} 3^{-(n+1)} z^n + (2/3)z^0 + \sum_{n=1}^\infty z^n  
  & \qquad & (|z|<1) \\
f(z) &= \sum_{n=-1}^{-\infty} \left((-1)^{n+1} + 1\right)z^n + 
        \sum_{n=0}^\infty 3^{-(n+1)} z^n 
     & \qquad & (1<|z|<3) \\
f(z) &= \sum_{n=-1}^{-\infty} \left(1 - (-1)^{n}(1+3^n) \right)z^n
     & \qquad & (|z|>3)
\end{alignat*}


%%%%%%%%%%%%%% 27
\begin{excopy}
Suppose \(\Omega\) is a horizontal strip determined by the inequalities
\(a<y<b\), say.
Suppose \(f\in H(\Omega)\) and \(f(z) = f(z+1)\) for all \(z\in\Omega\).
Prove that $f$ has a Fourier expansion in \(\Omega\),
\begin{equation*}
f(z) = \sum_{-\infty}^\infty c_n e^{2\pi inz},
\end{equation*}
which converges uniformly in \(\{z: a+\epsilon \leq y \leq b - \epsilon\}\),
for every \(\epsilon > 0\).
\emph{Hint:} The map \(z\to e^{2\pi i z}\) converts $f$ to a function in an 
annulus.

Find the integral formula by means on which the coefficients \(c_n\) can be 
computed from $f$.
\end{excopy}

Denote \(z=x+iy\) with \(x,y\in\R\). The map
\begin{equation*}
\varphi(z) = e^{2\pi i z} = e^{2\pi i (x+iy)} = e^{-2\pi y}\cdot e^{2\pi i x}
\end{equation*}
maps the stripe \(\Omega = \{x+iy\in\C: x\in\R \wedge a<y<b\}\)
onto the annulus
\begin{equation*}
A = \{z\in\C: e^{-2\pi b} < |z| < e^{-2\pi a}\}.
\end{equation*}
The mapping is not injective, but if \(\varphi(x_1+iy_1) = \varphi(x_2+iy_2)\)
where \(x_i,y_i\in\R\) then \(y_1=y_2\) and \(x_2-x_1\in\Z\).
Hence we can define \(g:A\to\C\) by \(g(w) = f(z)\) where \(w=\varphi(z)\)
which is well defined since \(f(z)=f(z+1)\) for all \(z\in\Omega\).

But the mapping holomorphic and 
is locally injective, that is for each \(z\in\Omega\)
there is a neighborhood \(V_z\subset\Omega\) such that \(\varphi\)
is one-to-one in \(V_z\).
Hence \(g\in H(A)\) by Theorem~1.30.

By previous exercise there exists a sequence \((c_n)_{n\in\Z}\)
of complex numbers such that 
\begin{equation*}
g(w) = \sum_{n\in\Z} c_n w^n \qquad (w\in A).
\end{equation*}
But
\begin{equation*}
c_nw^n = c_n  \left(e^{2\pi i z}\right)^n = e^{2\pi i nz}
\end{equation*}
substituting the last expression in the previous power-series gives
the desired \index{Fourier} Fourier expansion.


%%%%%%%%%%%%%% 28
\begin{excopy}
Suppose \(\Gamma\) is a closed curve in the plane, with parameter interval
\([0,2\pi]\).
Take \mbox{\(\alpha \notin \Gamma^*\).}
Approximate \(\Gamma\) uniformly by trigonometric polynomials \(\Gamma_n\).
Show that \(\Ind_{\Gamma_m}(\alpha) = \Ind_{\Gamma_n}(\alpha)\) if $m$ and $n$
are sufficiently large. Define this common value to be \(\Ind_{\Gamma}(\alpha)\).
Prove that the result does not depend on the choice of \(\{\Gamma_n\}\);
prove that Lemma~10.39 is now true for closed curves, and use this to give 
a~different proof of Theorem~1.40.
\end{excopy}

By Theorem~4.25 we for each $n$, we can find a trigonometric polynomial
\(P_n:\T\to\C\) such that \(\max\{|\Gamma(t)-P_n(t)|: t\in\T\}< 1/n\).
The continuous image \(\Gamma^*\) is compact, hence 
if \(\alpha\notin \Gamma^*\) then \(d(\alpha,\Gamma^*)>0\).
Hence, we can find some \(k<\infty\) such that \(d(\alpha,\Gamma^*)>1/k\).
Now by Lemma~10.39
\begin{equation} \label{eq:ex10.29:Pmn}
\Ind_{P_m}(\alpha) = \Ind_{P_n}(\alpha) 
\end{equation}
for all \(m, n \geq 2k\).
Hence 
\begin{equation*}
\Ind_\Gamma(\alpha) := \lim_{n\to\infty}{P_m}(\alpha)
\end{equation*}
is well defined and is independent of the choice 
of the trigonometric polynomials, since with any replacment of them,
Lemma~10.39 could still be applied and \eqref{eq:ex10.29:Pmn} will hold.

Now we can simplify the proof of Theorem~10.40.
Instead of introducing the paths 
\begin{equation*}
\gamma_k(s) 
 = H\left(\frac{i}{n},\frac{k}{n}\right)(ns + 1 -i)
 + H\left(\frac{i-1}{n},\frac{k}{n}\right)(i - ns)
\qquad (i\in\N_n, \; k\in \Z_n, \; i-1\leq ns\leq i)
\end{equation*}
we can use the curves (now, not necessarily paths)
\begin{equation*}
\gamma_k(s) = H(s,k/n) \qquad (k\in \Z_n)
\end{equation*}
and proceed with the original proof.


%%%%%%%%%%%%%% 29
\begin{excopy}
Define 
\begin{equation*}
f(z) = \frac{1}{\pi} \int_0^1 r\,dr \int_{-\pi}^\pi \frac{d\theta}{re^{i\theta}+z}.
\end{equation*}
Show that \(f(z) = \overline{z}\) if \(|z|<1\) and that 
\(f(z) = 1/z\) if \(|z|\geq 1\).

Thus $f$ is not holomorphic in the unit disc, 
although the integrand is a holomorphic function of~$z$. 
Note the contrast between this, on the one hand, and Theorem~10.7 
and Exercise~16 on the other.

\emph{Suggestion:} Compute the inner integral separately 
for \(r<|z|\) and for \(r>|z|\).
\end{excopy}

We first compute the inner intergral. We define
\begin{equation*}
h(r,z) = \int_{-\pi}^\pi \frac{d\theta}{re^{i\theta}+z}.
\end{equation*}
\paragraph{Assuming \(|r|<z\).} In this case \(z\neq 0\) and we have
\begin{equation*}
\frac{1}{re^{i\theta}+z}
= \frac{1}{z}\cdot\frac{1}{1-((-r/z)e^{i\theta})}
= \frac{1}{z}\sum_{n=0}^\infty \left(-\frac{r}{z}\right)^n e^{in\theta}.
\end{equation*}
Hence
\begin{align*}
% \int_{-\pi}^\pi \frac{d\theta}{re^{i\theta}+z}
h(r,z)
&= \frac{1}{z} \int_{-\pi}^\pi
    \left( \sum_{n=0}^\infty \left(-\frac{r}{z}\right)^n e^{in\theta} 
    \right)\,d\theta
 = \frac{1}{z} \sum_{n=0}^\infty \left(-\frac{r}{z}\right)^n 
     \int_{-\pi}^\pi e^{in\theta} \,d\theta
 = \frac{1}{z} \left(-\frac{r}{z}\right)^0 \cdot 2\pi \\
&= 2\pi/z
\end{align*}


\paragraph{Assuming \(|r|>z\).} In this case we have
\begin{equation*}
\frac{1}{re^{i\theta}+z}
= \frac{1}{re^{i\theta}}\cdot\frac{1}{1-(-z/r)e^{-i\theta}}
= \frac{1}{re^{i\theta}}\sum_{n=0}^\infty (-z/r)^n e^{-in\theta}.
\end{equation*}
Hence
\begin{equation*}
% \int_{-\pi}^\pi \frac{d\theta}{re^{i\theta}+z}
h(r,z)
 = \int_{-\pi}^\pi \frac{1}{re^{i\theta}}
    \left(\sum_{n=0}^\infty \left(\frac{-z}{r}\right)^n e^{-in\theta}\right) d\theta 
 = \sum_{n=0}^\infty \frac{(-z)^n}{r^{n+1}}
    \int_{-\pi}^\pi e^{-i(n+1)\theta}\,d\theta 
 = 0.
\end{equation*}

Back to the outer integral, we again have two cases.
\paragraph{Assume \(|z|\leq 1\).}
\begin{align*}
f(z) 
&= \frac{1}{\pi} \left(
   \int_0^{|z|} r\cdot h(r,z)\,dr + \int_{|z|}^1 r\cdot h(r,z)\,dr \right) 
 = \frac{1}{\pi} \left(
      \int_0^{|z|} \frac{2\pi r}{z}\,dr +  0\right) \\
&= (2/z) \left.\left(r^2/2\right)\right|_0^{|z|}
 = |z|^2/z
 = \overline{z}\,.
\end{align*}

\paragraph{Assume \(|z|\geq 1\).}
\begin{align*}
f(z) 
&= \frac{1}{\pi} \int_0^1 r\cdot h(r,z)\,dr 
 = \frac{1}{\pi} \int_0^1 (2\pi r/z) \,dr 
 = (1/z)\left.\left(r^2\right)\right|_{0}^1
 = 1/z\,.
\end{align*}


%%%%%%%%%%%%%% 30
\begin{excopy}
Let \(\Omega\) be the plane minus two points, and show that some closed paths
\(\Gamma\) in \(\Omega\) satisfy assumtion~(1) of Theorem~10.35 without being
\index{null-homotopic}
\index{null!homotopic}
null-homotopic in~\(\Omega\).
\end{excopy}

Intuitively, make a double ``eight''-pattern, where each ``eight''
is in different direction.
Say the two deleted points are $1$ and $3$.
Let's for the path as the a polyline passing thru the following
points in the given order:
{\newcommand{\qtoq}{\;\to\;}
\begin{align*}
2 \qtoq 3-i \qtoq 4 \qtoq 3+i &\qtoq 2 \qtoq 1-i \qtoq 0 \qtoq 1+i \qtoq \\
2 \qtoq 3+i \qtoq 4 \qtoq 3-i &\qtoq 2 \qtoq 1+i \qtoq 0 \qtoq 1-i \qtoq 2.
\end{align*}
}
Of course, there are similar paths that are nicer
in the sense that they cross themselves only 
in a finite number of points~($5$?).

%%%%%%%%%%%%%%%%%
\end{enumerate}


