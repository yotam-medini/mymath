% -*- latex -*-
% $Id: rudinrca7.tex,v 1.15 2006/09/05 20:39:02 yotam Exp $


%%%%%%%%%%%%%%%%%%%%%%%%%%%%%%%%%%%%%%%%%%%%%%%%%%%%%%%%%%%%%%%%%%%%%%%%
%%%%%%%%%%%%%%%%%%%%%%%%%%%%%%%%%%%%%%%%%%%%%%%%%%%%%%%%%%%%%%%%%%%%%%%%
%%%%%%%%%%%%%%%%%%%%%%%%%%%%%%%%%%%%%%%%%%%%%%%%%%%%%%%%%%%%%%%%%%%%%%%%
\chapterTypeout{Differentiation} % 7

%%%%%%%%%%%%%%%%%%%%%%%%%%%%%%%%%%%%%%%%%%%%%%%%%%%%%%%%%%%%%%%%%%%%%%%%
%%%%%%%%%%%%%%%%%%%%%%%%%%%%%%%%%%%%%%%%%%%%%%%%%%%%%%%%%%%%%%%%%%%%%%%%
\section{Notes}

%%%%%%%%%%%%%%%%%%%%%%%%%%%%%%%%%%%%%%%%%%%%%%%%%%%%%%%%%%%%%%%%%%%%%%%%
\subsection{Notations}

\paragraph{Null Set.} We use the term
\emph{nullset}
\index{nullset}
of \cite{Oxtoby1980}
for a set of measure zero.

We denote the \emph{length} of an interval $I$ by \(\ell(I)\).

%%%%%%%%%%%%%%%%%%%%%%%%%%%%%%%%%%%%%%%%%%%%%%%%%%%%%%%%%%%%%%%%%%%%%%%%
\subsection{Outer Measure}

In Rudin's treatment \cite{RudinRCA87}, measure theory is developed
without relying in the notion of outer measure.
In contrast, Royden gives a the following definition
(\cite{Royden} Chapter~3, Section~2 page~56).

\paragraph{Definition.}
\index{outer measure}
\index{measure!outer}
The \index{outer measure} of a set \(A\subset \R\) is
\begin{equation*}
m^*(A) = \inf_{A\subset \cup I_n} \ell(I_n).
\end{equation*}
When can generalize this to sets in \(\R^n\) and also show that
\begin{equation*}
m^*(A) = \inf \{m(G): A\subset G \wedge G \;\mathrm{is\ open}\}.
\end{equation*}

%%%%%%%%%%%%%%%%%%%%%%%%%%%%%%%%%%%%%%%%%%%%%%%%%%%%%%%%%%%%%%%%%%%%%%%%
\subsection{Lemma of Vitali}

We bring the 
Lemma of Vitali
\index{Vitali!of Vitali}
from \cite{Royden}.

\begin{llem} \label{lem:vitali}
Let $E$ be a set of finite measure and \frakI\ a set of intervals
that covers $E$ in the sense of Vitali, that is
\begin{equation*}
\forall x\in E\, \forall \epsilon>0\;
 \exists I\in\frakI\, x\in I \wedge \ell(I) < \epsilon.
\end{equation*}
Then given \(\epsilon>0\) there exists a finite subset of intervals 
\(\{I_n\}_{n=1}^N \subset \frakI\) such that 
\begin{equation}
m\left(E \setminus \cup_{n=1}^N I_n\right) < \epsilon>0.
\end{equation}
\end{llem}

\textbf{Note} Originally, the lemma deals with \emph{outer} measure,
but here we use less generalized version.

\begin{thmproof}
\Wlogy\ we may assume that \(\{\ell(I): I\in\frakI\}\) is bounded,
otherwise we pick some finite measure open set \(O\supset E\) 
and intersect each interval with this opens set, producing new interval(s)
that still satisfy the requirements.
Form a sequence \(\{I_j\}_{j\in\N}\) by induction. Say 
\(\{I_j\}_{j=1}^n\) were picked. Define
\begin{align*}
U_n &= \cup_{j=1}^n \overline{I_j} \\
k_n &= \sup \{\ell(I): I\in\frakI \wedge I \cap U_n = \emptyset\}.
\end{align*}
If \(E\subset U_n\) we are done, otherwise \(0< k_n < \infty\).
Pick \(I_{n+1}\) such that \(\ell(I_{n+1}) > k_n/2\).
Clearly \(\lim_{n\to\infty} \ell(I_n) = 0\), hence there exists $N$
such that
\begin{equation*}
\sum_{n=N+1}^\infty \ell(I_n) < \epsilon/5.
\end{equation*}
Let \(x\in R\) and pick \(I\in\frakI\) such that \(I\cap U_N=\emptyset\).
Let $n$ be the smallest integer such that \(I\cap I_n \neq \emptyset\).
Clearly \(n>N\) 
and by our method of forming the sequences 
\(\ell(I) \leq k_{n-1} \leq 2(I_n)\). If \(c_n\) is the center of \(I_n\) then
\begin{equation*}
|x-c_n| \leq \ell(I) + \ell(I_n)/2 \leq \frac{5}{2} \ell(I_n).
\end{equation*}
Define  $5$-time expansions of \(I_n\)
\begin{equation*}
J_n = \left[c_n - \frac{5}{2} \ell(I_n),\, c_n + \frac{5}{2} \ell(I_n)\right]
\end{equation*}
and now  \(x\in J_n\) and so
\begin{equation*}
R \subset \cup_{n=N+1}^\infty J_n
\end{equation*}
Hence
\begin{equation*}
m(R) 
\leq \sum_{n=N+1}^\infty \ell(J_n)
= 5\sum_{n=N+1}^\infty \ell(I_n)
< \epsilon.
\end{equation*}
\end{thmproof}


%%%%%%%%%%%%%%%%%%%%%%%%%%%%%%%%%%%%%%%%%%%%%%%%%%%%%%%%%%%%%%%%%%%%%%%%
\subsection{Lebesgue Differentiation Theorem}

Here is a theorem given in \cite{Royden} Chapter~5, Section~1, page~100,
Theorem~3.

\begin{llem} \label{lem:leb:diff}
If \(f:[a,b]\to\R\) be a non decreasing function
then $f$ is differentiable almost everywhere.
The derivative \(f'\) is measurable and 
\begin{equation*}
\int_a^b f'(x)\,dx \leq f(b) - f(a). \label{eq:lem:leb:diff}
\end{equation*}
In particular, \(f'\in L^1([a,b],m)\).
\end{llem}
\begin{thmproof}
Let \(D\subset [a,b]\) be the set of points where $f$ is discontinuous.
Since $f$ is non decreasing we now that $D$ is countable.

Define
\begin{equation} \label{eq:7:DfMDfm}
D^+f(x) = \varlimsup_{h\to 0} \frac{f(x+h)-f(x)}{h}
\qquad
D^-f(x) = \varliminf_{h\to 0} \frac{f(x+h)-f(x)}{h}
\end{equation}
where \(x,x+h \in [a,b]\) of course.

We would like to be able to use limits such that that $h$ run over
some countable set for all \(x\in[a,b]\).
\begin{align} 
\Delta^+f(x) &\eqdef \label{eq:7:DfMQ} 
   \varlimsup_{\stackrel{h\to 0}{h\in\Q}} \frac{f(x+h)-f(x)}{h} \\
\Delta^-f(x) &\eqdef \label{eq:7:DfmQ}
   \varliminf_{\stackrel{h\to 0}{h\in\Q}} \frac{f(x+h)-f(x)}{h}
\end{align}

\paragraph{Claim:} 
\begin{align}
D^+f(x) &= \Delta^+f(x)  \label{eq:DfM:eqQ} \\
D^-f(x) &= \Delta^-f(x)  \label{eq:Dfm:eqQ}
\end{align}
We will show \eqref{eq:DfM:eqQ} and \eqref{eq:Dfm:eqQ} 
will follow from \(\varliminf f = -\varlimsup -f\).
Clearly \(\Delta^+f(x) \leq D^+f(x) \) since \(\Q\subset\R\).
For this claim, fix \(x\in[a,b]\) and let \(u \eqdef D^+f(x)\).
Pick an arbitrarily small \(\delta>0\).
There is \(h\neq 0\) such that \(|h|<\delta\)
\begin{equation*}
\bigl(f(x+h) - f(x)\bigr)/h > u - \epsilon \qquad x+h\in(a,b).
\end{equation*}
Let us assume that \(h>0\). The negative case can be similarly treated.
Put
\begin{equation*}
Q(x,h) = \{q\in\Q: q\geq h \wedge x+q\in[a,b]\}
\end{equation*}
Clearly \(Q(x,h)\neq \emptyset\) and 
 \(f(x+h)\leq f(x+q)\) for each \(q\in Q(x,h)\).
Thus
\begin{equation*}
\varlimsup_{\stackrel{q\to h}{q\in\Q(x,h)}}  \frac{f(x+q) - f(x)}{q}
\geq \varlimsup_{\stackrel{q\to h}{q\in\Q(x,h)}}  \frac{f(x+h) - f(x)}{q}
= \varlimsup_{\stackrel{q\to h}{q\in\Q(x,h)}}  \frac{f(x+h) - f(x)}{h}
= u.
\end{equation*}
Hence \(\Delta^+f(x) \geq D^+f(x)\) for any \(x\in[a,b]\)
and the claim \eqref{eq:DfM:eqQ} is shown.

Define the ``bad'' sets
\begin{align}
E       &\eqdef \left\{x\in[a,b]: D^-f(x) < D^+f(x)\right\} \notag \\
E_{u,v} &\eqdef \left\{x\in[a,b]: D^-f(x) < v < u < D^+f(x)\right\} 
        \label{eq:7:Euv}
\end{align}
and clearly
\begin{equation} \label{eq:ex7.14:Eu}
E = \bigcup_{u,v\in\Q} E_{u,v}
\end{equation}
Since 
\begin{align*}
\Delta^+f(x) &= \inf_{r\in\Q^+} 
    \sup\left\{ \frac{f(x+q)-f(x)}{q} :
                q\in\Q \swedge 0<|q|<r \swedge x+q\in[a,b] \right\}
\\
\Delta^-f(x) &= \sup_{r\in\Q^+} 
    \inf\left\{ \frac{f(x+q)-f(x)}{q} :
                q\in\Q \swedge 0<|q|<r \swedge x+q\in[a,b] \right\}
\end{align*}
the sets $E$ and \(E_{u,v}\) are measurable! They could be defined
by countable intersection and unions of open sets.
Noting that inverse image of intervals via $f$ are intervals since
$f$ is non decreasing.
This was 
the whole reason for the claim, thus avoiding the usage of outer measure
and using a simplified lemma (\ref{lem:vitali}) of Vitali.

We note that 
$f$ is differentiable on \([a,b]\setminus E\) 
and obviously \(E_{u,v} = \emptyset\) if \(u\geq v\).
Since the above \eqref{eq:ex7.14:Eu} is a countable union 
it will suffice to show that \(m(E_{u,v}) = 0\) whenever \(u < v\).
Fix some \(u,v\in\R\) such that \(v<u\).
Assume by negation 
\begin{equation} \label{eq:7:vit:sgt0}
s \eqdef m(E_{u,v}) > 0.
\end{equation}
Pick arbitrary \(\epsilon>0\) and take an open \(G \supset E_{u,v}\)
such that \(m(G) < s + \epsilon\)
For each \(x\in E_{u,v}\) we can find some $h$ such that
the interval \([x,x+h]\) or  \([x+h,x]\) is contained in $G$ and 
\begin{equation} \label{eq:7:vit:ltv}
\bigl(f(x+h) - f(x)\bigr)/h < v.
\end{equation}
By local lemma (Vitali)~\ref{lem:vitali} 
there is a finite set of disjoint open intervals 
\(\{I_j\}_{j=1}^N = \{(a_j,b_j)\}_{j=1}^N\) (where \(a_j<b_j\))
that 
cover a subset \(A = \cap_{j=1}^N I_j \cap  E_{u,v}\) 
and \(m(A) > s - \epsilon\). Summing \eqref{eq:7:vit:ltv} we get
\begin{equation} \label{eq:7:vit:ltvs}
\sum_{j=1}^N f(b_j) - f(a_j) 
< v\sum_{j=1}^N b_j - a_j < v\,m(G)  <  v(s+\epsilon).
\end{equation}
Now for each \(y\in A\) there is an interval
 \([y,y+k]\) or  \([y+k,y]\) is contained in some \(I_j\) such that 
\begin{equation} \label{eq:7:vit:gtu}
\bigl(f(y+k) - f(y)\bigr)/k > u.
\end{equation}
Applying the same lemma again, there are disjoint intervals
\(\{J_j\}_{j=1}^M = \{(c_j,d_j)\}_{j=1}^M\) (where \(c_j<d_j\))
that 
cover a subset \(B = \cap_{j=1}^M J_j \cap  A\) 
such that \(m(B) > s-2\epsilon\). 
Similar summation of~\eqref{eq:7:vit:gtu} gives
\begin{equation} \label{eq:7:vit:gtus}
\sum_{j=1}^M f(d_j) - f(c_j) > u \sum_{j=1}^M d_j - c_j >  u(s - 2\epsilon).
\end{equation}
For the segment \(I_n\), since $f$ is nondecreasing we have
\begin{equation*}
\sum_{\stackrel{1\leq j \leq M}{J_j \subset I_n}} f(d_j) - f(c_j) 
\leq f(b_n) - f(a_n).
\end{equation*}
Summing over all \(\{I_j\}_{j=1}^N\) gives
\begin{equation} \label{eq:vit:dcltba}
\sum_{j=1}^M f(d_j) - f(c_j) \leq \sum_{j=1}^N f(b_j) - f(a_j) 
\end{equation}
Combining 
\eqref{eq:7:vit:ltvs},
\eqref{eq:7:vit:gtus} and
\eqref{eq:vit:dcltba}
gives
\begin{equation*}
u(s - 2\epsilon) \leq v(s+\epsilon).
\end{equation*}
Since \(\epsilon\) is arbitrarily small, we have
\(us \leq vs\) and by \eqref{eq:7:Euv} \(s=0\) contradiction 
to~\eqref{eq:7:vit:sgt0}. Thus \(m(E)=0\) and $f$ is differentiable
almost everywhere.

To show the final part of this differentiation theorem, let 
\begin{equation*}
g_n(x) \eqdef n\bigl(f(x+1/n) - f(x)\bigr) 
                        \qquad \forall x>b,\; f(x)\eqdef f(b).
\end{equation*}
Clearly \(\lim_{n\to\infty} g_n(x) = f'(x)\) hence it is measurable.
Now by Fatou's lemma
\index{Fatou}
\begin{align*}
\int_a^b f'(x)\,dx
&\leq \varliminf_{n\to\infty} \int_a^b g_n(x)\,dx  
= \varliminf_{n\to\infty} n \int_a^b \bigl(f(x+1/n) - f(x)\bigr) \,dx  \\
&= \varliminf_{n\to\infty} 
   \left(n \int_b^{b+1/n} f(x)\,dx  - \int_a^{a+1/n} f(x)\,dx  \right)
 = \varliminf_{n\to\infty} 
   \left(f(b)  - n\int_a^{a+1/n} f(x)\,dx  \right) \\
& \leq f(b) - f(a).
\end{align*}
Hence \eqref{eq:lem:leb:diff} holds.
\end{thmproof}



%%%%%%%%%%%%%%%%%%%%%%%%%%%%%%%%%%%%%%%%%%%%%%%%%%%%%%%%%%%%%%%%%%%%%%%%
\subsection{Absolute Continuity and Bounded Variation.}

Trvial lemma showing that absolute continuity is
stronger than having bounded variation.

\begin{llem}
If \(f:[a,b]\to\C\) is an absolute continuous function,
then it has bounded variation.
\end{llem}
\begin{thmproof}
Pick \(\epsilon=1\) then there exists \(\delta>0\)
such that 
\(\sum_{j=1}^n |f(b_k) - f(a_k)| < \epsilon=1\)
whenever
\(\sum_{j=1}^n b_k - a_k| < \delta\)
where \(a \leq a_k \leq b_k \leq b\).

Now given an arbitrary partition \(a = t_0 < t_1 < \cdots < t_m = b\)
we can extended it to 
Now given a partition \(a = u_0 < u_1 < \cdots < u_n = b\)
such that 
\(u_k - u_{k-1} < \delta/2\) for all \(k\in\N_n\) and 
for any \(j\in\N_m\) there exists \(k\in\N_n\) 
such that \(t_j = u_k\).
We split the partitions to chunks bounded by \(\delta/2\) by defining
\begin{equation*}
c(k) = \min \{k\in\Z^+: u_k - a \geq n\delta/2\}.
\qquad 0\leq n \leq \lceil 2(b-a)/\delta \rceil.
\end{equation*}
Clearly \(u_{c(n)} - u_{c(n-1)} < \delta/2\).
Now
\begin{equation*}
\sum_{j=1}^m |f(t_j) - f(t_{j-1})|
\leq \sum_{j=1}^n |f(u_j) - f(u_{j-1})|
= \sum_k \sum_{j=c(k) + 1}^{c(k+1)} |f(u_j) - f(u_{j-1})|
\leq \lceil 2(b-a)/\delta \rceil\cdot 1.
\end{equation*}
Thus $f$ has bounded variation.
\end{thmproof}



%%%%%%%%%%%%%%%%%%%%%%%%%%%%%%%%%%%%%%%%%%%%%%%%%%%%%%%%%%%%%%%%%%%%%%%%
\subsection{Detailed Reference.}

In the \textsc{Notes and Comments} appendix for this chapter,
there is a reference for elementary proof of almost everywhere
differentiability of monotone function.
The full reference is:\\
\emph{A Geometric Proof of the Lebesgue Differentiation Theorem},
\textbf{Donald Austin},
Proceedings of the American Mathematical Society, 
Vol.~16, No.~2 (Apr.,~1965), pp.~ 220--221.


%%%%%%%%%%%%%%%%%%%%%%%%%%%%%%%%%%%%%%%%%%%%%%%%%%%%%%%%%%%%%%%%%%%%%%%%
%%%%%%%%%%%%%%%%%%%%%%%%%%%%%%%%%%%%%%%%%%%%%%%%%%%%%%%%%%%%%%%%%%%%%%%%
\section{Exercises} % pages 156-159

%%%%%%%%%%%%%%%%%
\begin{enumerate}
%%%%%%%%%%%%%%%%%


%%%%%%%%%%%%%% 1
\begin{excopy}
Show that \(|f(x)| \leq (Mf)(x)\) at every Lebesgue point of $f$ if 
\(f\in L^1(\R^k)\).
\end{excopy}

Say $x$ is a Lebesgue point
\index{Lebesgue point}
then
\begin{equation*}
\lim_{r\to\infty} \frac{1}{m(B_r)} \int_{B(x,r)} |f(y)-f(x)|\,dm(y) = 0.
\end{equation*}
Pick \(\epsilon>0\) and \(r>0\) such that
\begin{equation*}
D = \frac{1}{m(B_r)} \int_{B(x,r)} |f(y)-f(x)|\,dm(y) < \epsilon.
\end{equation*}
Hence
\begin{align*}
D 
&\geq \frac{1}{m(B_r)} \left( \int_{B(x,r)} |f(x)|\,dm(y) - 
                             \int_{B(x,r)} |f(y)|\,dm(y) \right) \\
&=  |f(x)| - \frac{1}{m(B_r)} \int_{B(x,r)} |f(y)|\,dm(y) 
\end{align*}
Combing the above, we get
\begin{equation*}
|f(x)| \leq \frac{1}{m(B_r)} \int_{B(x,r)} |f(y)|\,dm(y) + \epsilon.
\end{equation*}
Since \(\epsilon>0\) was arbitrary, we actually have 
the desired
\begin{equation*}
|f(x)| \leq \sup_{r>0} \frac{1}{m(B_r)} \int_{B(x,r)} |f|\,dm
\end{equation*}
inequality.


%%%%%%%%%%%%%%
\begin{excopy}
For \(\delta>0\), let \(I(\delta)\) be the 
segment \((-\delta,\delta) \subset \R^1\). Given \(\alpha\) and \(\beta\),
\(0\leq\alpha\leq \beta\leq 1\),
construct a measurable set \(E\subset \R^1\) so that the upper and lower
limits of 
\begin{equation*}
\frac{m\bigl(E\cap I(\delta)\bigr)}{2\delta}
\end{equation*}
are \(\beta\) and \(\alpha\) respectively, as \(\delta\rightarrow 0\).

(Compare this with Section~7.12.)
\end{excopy}

Let \(0< s,u < 1\). We will soon determine their actual values.
Define % the decreasing sequence \(s_n = s^n\) for \(n\geq 0\) and
the sequence of clopen intervals 
\begin{equation*}
I_n = (s^n,s^{n-1}] \qquad n \geq 0.
\end{equation*}
Clearly \((0,1] = \disjunion_{n\geq 0} I_n\).
Define ``inter-points'' and open sub-intervals.
\begin{align*}
t_n &= s^n + u\cdot m(I_n) \\
U_n &= (s^n, t_n) \subsetneq I_n \qquad n \geq 0.
\end{align*}
and let \(U = \cup_n U_n\), put \(U^- = \{x\in\R: -x \in U\)
and finally \(E = U \disjunion U^-\).
Clearly 
\begin{equation*}
\frac{m\bigl(E\cap I(\delta)\bigr)}{2\delta} 
= m\bigl(U\cap I(\delta)\bigr) / \delta.
\end{equation*}
Thus it is sufficiently to look at the limits of 
\(m(U\cap [0,\delta]) / \delta\).
For any \(\delta\) looking at the 
the subsequence of \(\{U_n\}_{n\geq 0}\) below \(\delta\)
we can see that
\begin{equation*}
\liminf_{\delta\to 0} m(U\cap [0,\delta]) / \delta = u.
\end{equation*}
Similarly, if we look at the ``shifted segments'' \(S_n = (t_{n+1},t_n)\)
we can see that
\begin{equation*}
\limsup_{\delta\to 0} m(U\cap [0,\delta]) / \delta 
= u / (s(1-u) + u) = u / (s + (1-s)u) > u.
\end{equation*}
Trivial solving give the desired values \(u=\alpha\)
and for
\begin{equation*}
u / (s(1-u) + u) = \beta
\end{equation*}
we have
\begin{equation*}
s 
= (u / \beta - u) / (1-u)
= (\alpha / \beta - \alpha) / (1-\alpha).
\end{equation*}

In Section~7.12 we see that the limit exists and must be $0$ or $1$
almost everywhere. 
So a case like constructed above may happen only on a nullset (measure zero). 

%%%%%%%%%%%%%%
\begin{excopy}
Suppose that $E$ is a measurable set of real numbers with arbitrarily
small periods. Explicitly,
this means that there are positive numbers \(p_i\), converging to $0$
as \(i\to \infty\), so that 
\begin{equation*}
E + p_i = E \qquad (i=1,2,3,\ldots).
\end{equation*}

Prove that then either $E$ or its complement has measure $0$.

\emph{Hint:} Pick \(\alpha\in\R^1\), 
put \(F(x) = m(E\cap[\alpha,x])\) for \(x>\alpha\), show that
\begin{equation*}
F(x + p_i) - F(x - p_i) = F(y + p_i) - F(y - p_i)
\end{equation*}
if \(\alpha + p_i < x < y\). 
What does this imply about \(F(x)\) if \(m(E)>0\)?
\end{excopy}

Let \(n = \lfloor (y-x)/2p_i \rfloor\).
Then \(x - p_i < (y-2np_i) - p_i < (y-2np_i) + p_i\).
Now
\begin{eqnarray*}
F(x + p_i) - F(x - p_i) 
&=& m(E \cap [x-p_i, x+p_i]) \\
&=& m(E \cap ([x-p_i, (y-2np_i) - p_i] \cup [(y-2np_i) - p_i, x+p_i]) \\
&=& m(E \cap ([x-p_i, (y-2np_i) - p_i]) \\
& & + m(E \cap [(y-2np_i) - p_i, x+p_i]) \\
&=& m((E - 2p_i) \cap ([x-p_i, (y-2np_i) - p_i]) \\
& & + m(E \cup [(y-2np_i) - p_i, x+p_i]) \\
&=& m(E \cap [(y-2np_i) - p_i, x+p_i]) \\
& & + m(E \cap ([x+p_i, (y-2np_i) + p_i]) \\
&=& m(E \cap [(y-2np_i) - p_i, (y-2np_i) + p_i]) \\
&=& m(E \cap [y - p_i, y + p_i]) \\
&=& F(y + p_i) - F(y - p_i)
\end{eqnarray*}

By Theorem~7.18 \cite{RudinRCA87}, \(F(x)\) is differentiable
almost everywhere. By the above equality, its value is constant
simply by looking at 
\begin{equation*}
F'(x) = \lim_{j\to\infty} \frac{F(x+p_j) - F(x-p_j)}{2p_j}.
\end{equation*}
This value is actually the 
metric density
\index{metric density}
of $E$ and by the discussion in Section~7.12 \cite{RudinRCA87}
this constant must be $0$ or $1$. If \(m(E)>0\) it must be $1$
and thus \(F(x) = x-\alpha\) for \(x>\alpha\).
Since \(\alpha\) was arbitrary, for any interval $I$ we have
\(m(E\cap I) = m(I)\) and so \(m(\R\setminus E) = 0\).


%%%%%%%%%%%%%%
\begin{excopy}
Call $t$ a 
\emph{period}
\index{period}
of the function $f$ on \(\R^1\) if \(f(x+t)=f(x)\) for all \(x\in\R^1\).
Suppose $f$ is a real Lebesgue measurable function with periods $s$ and $t$
whose quotient is irrational.
Prove that there is a constant $c$ such that \(f(x)=c\;\aded\),
but that $f$ need not be constant.

\emph{Hint}: Apply Exercise~3 to the sets \(\{f>\lambda\}\).
\end{excopy}

We construct a sequence \(\{p_j\}_{j\in\N}\) such that 
\(\lim_{j\to\infty} p_j = 0\) and \(p_j > 0\) are periods of $f$.
Assume \(s>t\). Let (Similar to Euclid GCD algorithm)
\begin{align*}
p_1 &= s \\
p_2 &= t \\
p_n &= p_{n-2} - p_{n-1}\left\lfloor \frac{p_{n-2}}{p_{n-1}}\right\rfloor
       < p_{n-1}.
\end{align*}
Assume by negation that  $f$ is not constant almost everywhere.
We can find some \(\lambda\) such that 
\(\{f\leq\lambda\}\) and its complement \(\{f>\lambda\}\)
each is not a nullset.
The sequence \(\{p_j\}_{j\in\N}\) is also periods of these sets.
But by previous exercise, one of these two sets must be a nullset.


%%%%%%%%%%%%%% 5
\begin{excopy}
If \(A \subset \R^1\) and \(B \subset \R^1\), 
define \(A+B = \{a+b: a\in A, b\in B\}\).
Suppose \(m(A)>0\) and \(m(B)>0\).
Prove that \(A+B\) contains a segment, by completing the following outline:

There are points \(a_0\) and \(b_0\) where $A$ and $B$ have 
\index{metric density}
metric density~$1$.
Choose a small \(\delta>0\).
Put \(c_0 = a_0 + b_0\).
For each \(\epsilon\), positive or negative, define \(B_\epsilon\) 
to be the set of all \(c_0 + \epsilon - b\) for which 
\(b\in B\) and \(|b-b_0| < \delta\).
Then \(B_\epsilon \subset  (a_0 + \epsilon - \delta, a_0 + \epsilon + \delta)\).
If \(\delta\) was well chosen and \(|\epsilon|\) is sufficiently small,
it follows that $A$ intersects \(B_\epsilon\), so that 
\(A+B \supset (c_0 - \epsilon_0, c_0 + \epsilon_0)\) 
for some \(\epsilon_0 > 0\).

Let $C$ be 
\index{Cantor}
Cantor's ``middle thirds'' set and show that \(C+C\) is an interval,
although \(m(C)=0\).

(See also Exercise~19, Chap. 9.)
\end{excopy}

In this exercise solution, whenever we use $X$, we mean that
the expression or statement holds both for $A$ and for $B$.
\Wlogy, we may assume \(a_0=b_0=0\),
thus we have \(c_0=0\).
Take \(\delta>0\) such that 
\begin{align*}
m\bigl(X\cap (-h,h)\bigr)/2h &> 2/3 \\ 
\end{align*}
for any \(0<h<\delta\).
Pick \(\epsilon_0 = \delta/3\).
For \(\epsilon < \epsilon_0\), we have
\begin{equation*}
X_\epsilon = \epsilon - \bigl(X \cap (-\delta,+\delta)\bigr)
\subset (\epsilon - \delta, \epsilon + \delta).
\end{equation*}
Clearly \(m(X_\epsilon) = m(X\cap(-\delta,\delta))\).
Now
\begin{equation*}
\bigl(X\cap(\epsilon-\delta,\epsilon+\delta)\bigr)
\setminus
\bigl(X\cap(-\delta,\delta)\bigr)
\supset 
(\epsilon-\delta,\epsilon+\delta) \setminus  (-\delta,\delta)
\supset (\delta,\delta + \epsilon)
\end{equation*}
therefore,
\begin{align*}
   m\bigl(X \cap (\epsilon-\delta, \epsilon+\delta)\bigr)
&\geq  m\bigl(X \cap (-\delta, +\delta)\bigr) - \epsilon.
\end{align*}
Thus
\begin{align*}
m\bigl(A \cap (\epsilon-\delta, \epsilon+\delta)\bigr) + m(B_\epsilon)
&\geq m\bigl(A \cap (-\delta, +\delta)\bigr) - \epsilon 
      + m\bigl(B\cap(-\delta,\delta)\bigr) \\
&\geq \bigl((2/3)\cdot2\delta - \epsilon\bigr) + (2/3)\cdot2\delta \\
&> 2\bigl((2/3)\cdot2\delta - \epsilon\bigr)
 > 2(4\delta/3 - \epsilon_0) = 2\delta.
\end{align*}
and so \(A\cap B_\epsilon \neq \emptyset\).

Equivalently, for all \(\epsilon\in(-\epsilon_0,+\epsilon_0)\), 
there are \(a\in A\) and \(b\in B\) such that \(a=\epsilon -b\),
that shows that \(\epsilon\in A+B\) and actually
\((-\epsilon,\epsilon) \subset A+B\).
 
\paragraph{Cantor Set.} 
We will show that \(C+C = [0,2]\).
Any arbitrary \(x\in[0,2]\) can be represented as
\begin{equation*}
x = \sum_{n=0}^\infty t_n 3^{-n}
\end{equation*}
where \(t_0\in\{0,1\}\) and \(t_n\in\{0,1,2\}\) for \(n\geq 1\).
We will construct \(a,b\in C\) such that \(x=a+b\).
Put \(c_{-1} = 0\), using
\begin{equation*}
v_n = t_n + 3c_{n-1}
\end{equation*}
we define \(a_n\),\(b_n\) and \(c_n\) 
by induction for \(n\geq 0\) by the following 6 cases
\begin{equation*}
(a_n,b_n,c_n) = 
\left
\{\begin{array}{ll}
(0,0,0) \quad & v_n = 0 \\
(0,0,1) \quad & v_n = 1 \\
(2,0,0) \quad & v_n = 2 \\
(2,0,1) \quad & v_n = 3 \\
(2,2,0) \quad & v_n = 4 \\
(2,2,1) \quad & v_n = 5
\end{array}\right.
\end{equation*}
Note that \(a_0 = b_0 = 0\), so  if we put
\begin{equation*}
a = \sum_{n=0}^\infty a_n 3^{-n} \in C
\qquad
b = \sum_{n=0}^\infty b_n 3^{-n} \in C
\end{equation*}
we get the desired \(x = a + b \in C + C\) .
 

%%%%%%%%%%%%%%
\begin{excopy}
Suppose $G$ is a subgroup of \(\R^1\) (relative to addition),
\(G\neq \R^1\), and $G$ is Lebesgue measurable. Prove that then \(m(G)=0\).

\emph{Hint}: Use Exercise~5.
\end{excopy}

Assume by negation \(m(G)>0\). By previous exercise \(G+G\) contains a segment.
But \(G+G=G\) since it is a group. Thus $G$ contains an interval 
\([a,b]\in G\) such that \(a<b\).
Pick arbitrary \(x\in \R\). Define
\begin{align*}
d &= b - a > 0\\
n &= \lfloor x / d \rfloor \in \Z \\
r &= x - nd \in [a,b) \subset G
\end{align*}
Now 
\begin{align*}
x = n(b-a) + r \in G
\end{align*}
and so we get the \(\R \subset G\) contradiction.


%%%%%%%%%%%%%%
\begin{excopy}
Construct a continuous monotonic function $f$ on \(\R^1\) so that $f$ is not
constant on any segment although \(f'(x)=0\;\aded\)
\end{excopy}

The function constructed in Section~7.16 Example~\ich{b} satisfies
the requirements.

%%%%%%%%%%%%%% 8
\begin{excopy}
Let \(V=(a,b)\) be a bounded segment in \(\R^1\).
Choose segments \(W_n\subset V\) in  such a way that their union $W$ is dense 
in $V$ and the set \(K = V \setminus W\) has \(m(K)>0\).
Choose continuous function \(\varphi_n\) so that 
\(\varphi_n(x)=0\) outside \(W_n\), \(0<\varphi_n(x)<2^{-n}\) in \(W_n\).
Put \(\varphi = \sum \varphi_n\) and define
\begin{equation*}
T(x) = \int_a^x \varphi(t)\,dt \qquad (a<x<b).
\end{equation*}
Prove the following statements:
\begin{itemize}
\itemch{a} $T$ satisfies the hypothesis of Theorem~7.26, with \(X=V\).
\itemch{b} $T$ is continuous, \(T'(x)=0\) on $K$, \(m(T(K)) = 0\).
\itemch{c} If $E$ is a nonmeasurable subset of $K$ 
           (See Theorem~2.22) and \(A=T(E)\), then \(\chhi_A\) 
           is Lebesgue measurable but \(\chhi_A \circ T\) is not.
\itemch{d} The functions \(\varphi_n\) can be so chosen that the resulting $T$
           is an \emph{infinitely differentiable} homeomorphism of $V$ onto some
           segment in \(\R^1\) and \ich{c} still holds.

\end{itemize}
\end{excopy}

Let \(\{q_j\}_{j\in\N}\) be a sequence consisting of all of \(\Q\cap V\).
Put \(d=b-a\) and let \(W_1 = (a_1,b_1)\) be an open segment such that
\(q_1 \in W_1\subset V\) and \(d_1 = b_1 - a_1 < 2^{-1}d\).
By induction, assume \(W_j\) where chosen for \(j<n\).
Pick the first \(q_k\) in our sequence such that 
\(q_k \notin \cup_{j<n} W_j\) and choose \(W_n = (a_n,b_n)\) such that
\begin{align*}
q_n &\in W_n \subset V \setminus \left(\cup_{j<n} W_j\right) \\
d_n &= b_n - a_n < 2^{-n} d
\end{align*}
Clearly the chosen intervals \(\{W_j\}_{j\in\N}\) are disjoint
and satisfy the requirements.

We now use Exercise~7.1 in \cite{RudinPMA85}
and  Example~3.11 (page 40) in \cite{Gelb1996}.
For each interval \(W_n = (a_n,b_n)\) we define 
\begin{equation*}
g_n(x) = \left\{\begin{array}{ll}
c_n\exp\left((a-b) \,/\, (x-a_n)^2(b_n-x)^2\right) \qquad & a_n < x < b_n \\
0  & \mathrm{otherwise}.
\end{array}
\right.
\end{equation*}
and choose \(c_n\) such that \(\varphi_n(x) \leq  2^{-n+1 } < 2^{-n}\).
Actually 
\begin{equation*}
c_n =   2^{-n+1 } \exp\left((b-a) \,/\, (x-a_n)^2(b_n-x)^2\right).
\end{equation*}

Note that if
\begin{equation*}
g_a(x) = 
\left\{
\begin{array}{ll}
e^{-1/(x-a_n)^2} \quad & a_n < x  \\
0  & \mathrm{otherwise}.
\end{array}\right.
\qquad
g_b(x) = 
\left\{
\begin{array}{ll}
e^{-1/(b_n-x)^2} \quad &  x < b_n \\
0  & \mathrm{otherwise}.
\end{array}\right.
\end{equation*}
Then \(g_a^{(k)}(a_n) = g_b^{(k)}(b_n) = 0\) for all \(k\in\N\)
and for \(x\in (a_n,b_n)\) we have
\begin{align*}
g_a(x)g_b(x) 
&= e^{-1/(x-a)^2} e^{-1/(b-x)^2} 
 = \exp\left(- 1/(x-a)^2 - 1/(b-x)^2\right) \\
&= \exp\left(\frac{(x-b) - (x-a)}{(x-a)^2 (b-x)^2}\right)
 = \exp\left(\frac{a-b}{(x-a)^2 (b-x)^2}\right) \\
&= \varphi_n(x)
\end{align*}
Each \(\varphi_n\) is continuous, and since their support sets
are disjoint, \(\varphi\) is also continuous.
Now that $T$ is defined we prove the above.
\begin{itemize}

\itemch{a}

Now we show that $T$
satisfies Theorem~7.26's requirements.
\begin{itemize}

\itemch{i} 
\(X=V\) is a union of open intervals and so $V$ is open.
The continuity of $T$ follows from continuity of \(\varphi\).

\itemch{ii}
$X$ is open and so measurable.
Let \(x_1,x_2\in V=(a,b)\) and  \(x_1 < x_2\).
There must be some \(W_n = (a_n,b_n) \subsetneq (x_1,x_2)\).
Since 
\begin{equation*}
\int_{W_n} \varphi_n(t)\,dt > 0
\end{equation*}
We have \(T(x_1) < T(x_2)\) and so $T$ is injective.
The fact that \(\varphi\) is continuous
also implies that $T$ is differentiable on $X$.
and \(T'(x)=\varphi(x)\) for all \(x\in X\).

\itemch{iii}
Trivially, \(m(T(V-X))=0\) since \(V-X=\emptyset\).
\end{itemize}

\itemch{b}
As previously shown in \ich{a}-\ich{ii}, 
\(T'(x)=\varphi(x)\) for all \(x\in X\),
in particular \(T(x)=0\) for \(x\in K\).

\itemch{c}
Let \(E\subset K\) be a non measurable set.
then \(A=T(E)\subset T(K)\) and clearly Lebesgue measurable
since \(m(T(K)) = 0\).

\itemch{d}
We already took this infinitely differentiability into account
in our construction of \(\{\varphi_n\}_{n\in\N}\).
\end{itemize}

%%%%%%%%%%%%%%
\begin{excopy}
Suppose \(0<\alpha < 1\). Pick $t$ so that \(t^\alpha = 2\). Then
\(t>2\) and the construction of Example~\ich{b} in Sec.~7.16 can be
carried out with \(\delta_n = (2/t)^n\). Show that the resulting function $f$
\index{Lip@\(\Lip\)}
\index{Lipschitz condition}
belongs to \(\Lip \alpha\) on \([0,1]\).
\end{excopy}

For each \(n\in\N\) it is easy to see that
\begin{equation*}
\sup_{0\leq x <y\leq1} \frac{|f(y)-f(x)|}{y-x}
= \frac{f\bigl((2/t)^n/2^n\bigr) - f(0)}{
        \left((2/t)^n/2^n - 0\right)^\alpha}
= \frac{2^{-n}}{\left((2/t)^n/2^n\right)^\alpha}
= \frac{2^{-n}}{t^{-n\alpha}}
= (t^\alpha/2)^n.
\end{equation*}
Since \((t^\alpha/2)=1\) we have \(f_n\in\Lip\alpha\)
which is a Banach space and closed. Hence \(f\in\Lip\alpha\).


%%%%%%%%%%%%%%
\begin{excopy}
If 
\index{Lip@\(\Lip\)}
\index{Lipschitz condition}
\(f\in \Lip 1\) on \([a,b]\), prove that $f$ is absolutely continuous
and that \(f'\in L^\infty\).
\end{excopy}

Pick arbitrary \(\epsilon>0\), choose some \(\delta < \epsilon\).
For any set of disjoint intervals \(\{(\alpha_j,\beta_j)\}_{j\in\N}\)
such that \(\sum_j (\beta_j - \alpha_j) < \delta\) we have
\begin{equation*}
\sum_j |f(\beta_j) - f(\alpha_j)|
\leq \sum_j 1\cdot|\beta_j - \alpha_j|
= \sum_j \beta_j - \alpha_j < \delta < \epsilon
\end{equation*}
Thus $f$ is absolutely continuous.
By Theorem~7.18 $f$ is differentiable \aded\ and 
by \(f\in\Lip 1\) we have \(\|f'\|_{\infty} \leq 1\) and thus
\(f'\in L^\infty\).


%%%%%%%%%%%%%% 11
\begin{excopy}
Assume that \(1<p<\infty\), $f$ is absolutely continuous on \([a,b]\),
\(f'\in L^p\), and \(\alpha = 1/q\) where $q$ is the exponent conjugate to $p$.
Prove that \(f\in \Lip\alpha\).
\end{excopy}

{\small (From Curtis T.~McMullen's\\
\texttt{\scriptsize
www.math.harvard.edu/{\~{}}ctm/home/text/class/harvard/212a/03/html/home/course/course.pdf})}

If \(a\leq x<y\leq b\) then
\begin{equation*}
|f(y)-f(x)| 
\leq \int_a^b |f'(t)| \cdot \chhi_{[x,y]}\,dt
\leq \|f'\|_p \cdot \|\chhi_{[x,y]}\|_q 
=    \|f'\|_p \cdot(y-x)^{1/q}.
\end{equation*}
Hence 
\begin{equation*}
\frac{|f(y)-f(x)|}{(y-x)^\alpha} < \|f'\|_p < \infty.
\end{equation*}
and \(f\in \Lip_\alpha\).





%%%%%%%%%%%%%% 12
\begin{excopy}
Suppose 
\label{ex:7.12}
\(\varphi: [a,b]\to\R^1\) is nondecreasing.
\begin{itemize}

\itemch{a} 
Show that there is a left-continuous nondecreasing $f$ on \([a,b]\),
so that \(\{f\neq\varphi\}\) is at most countable. 
[Left-continuous means: if \(a<x\leq b\) and \(\epsilon > 0\), then there is 
a \(\delta>0\) so that \(|f(x) - f(x-t)|<\epsilon\) whenever \(0<t<\delta\).]

\itemch{b}
Imitate the proof of Theorem~7.18 to show that there is a positive Borel measure
\(\mu\) on \([a,b]\) for which
\begin{equation*}
f(x) - f(a) = \mu([a,x)) \qquad (a\leq x \leq b).
\end{equation*}

\itemch{c}
Deduce from \ich{b} that \(f'(x)\) exists \aded \([m]\), 
that \(f'\in L^1(m)\), and that
\begin{equation*}
f(x) - f(a) = \int_a^b f'(t)\,dt + s(x) \qquad (a\leq x \leq b)
\end{equation*}
where $s$ is a nondecreasing and \(s'(x)=0\;\aded[m]\).

\itemch{d}
Show that \(\mu\perp m\) if and only if \(f'(x)=0\;\aded[m]\),
and that \(\mu \ll m\) if and only if $f$ is AC on \([a,b]\).

\itemch{e}
Prove that \(\varphi'(x) = f'(x)\;\aded[m]\).
\end{itemize}
\end{excopy}

\begin{itemize}
\itemch{a}
If \(x\in(a,b)\) is a point of discontinuity than
\begin{equation*}
\sup_{t<x} \varphi(t) < \inf_{t>x} \varphi(t) 
\end{equation*}
There could be at most countable number of such discontinuity points.
Otherwise we have an uncountable sum of positive numbers
which must be less than \(b-a\).
Now we define \(f(x) = \varphi(a)\) and for \(x\in(a,b)\)
\begin{equation} \label{eq:ex7.12e:f}
f(x) = \sup_{a\leq t < x} \varphi(t).
\end{equation}
which is clearly left-continuous and differs from \(\varphi\)
only where the latter is discontinuous.

\itemch{b}
Define \((g(x) = f(x) + x\), clearly $g$ is monotonic, one to one and 
left-continuous. Let \(V\subset (a,b)\) some open set.
We will show that \(g(V)\) is a Borel set.
Recalling that a monotonic function can have at most countable
number of points of discontinuity.
We can represent $V$ is a countable disjoint union of intervals
\(V = \disjunion_{j\in\N} I_j\) such that each \(I_j\)
is open, closed or half open, and $g$ is continuous on it.
Clearly, \(g(I_j)\) is an interval --- open, closed or half open.
Thus \(g(V)\) is a Borel set. By being one-to-one, 
\begin{equation*}
g\bigl((a,b)\setminus V\bigr) = \bigl(g(a),g(b)\bigr) \setminus g(V) 
\end{equation*}
and so \(g(C)\) is a Borel set for any closed set $C$
and consequently 
\(g(E)\) is a Borel set for any Borel set $E$.

Now we can define for each Borel set $E$
\begin{align*}
\nu(E) &= m\bigl(g(E)\bigr) \\
\mu(E) &= \nu(E) - m(E).
\end{align*}
If follows that for each \(x\in(a,b)\)
\begin{align*}
g(x) - g(a) &= \nu([a,x]) \\
f(x) - f(a) &= \nu([a,x]) - (x-a) = \mu([a,x)).
\end{align*}

\itemch{c}
By Lebesgue differentiation theorem (local lemma~\ref{lem:leb:diff})
\(f'\) exists almost everywhere and \(f'\in L^1([a,b],m)\). Put 
\begin{equation} \label{eq:ex7.12:sx}
s(x) = f(x) - f(a) - \int_a^x f'(t)\,dm(t).
\end{equation}
By Theorem~7.11 \cite{RudinPMA85} 
\begin{equation*}
\frac{d}{dx} \int_a^x f'(t)\,dm(t) = f'(x) \quad\aded
\end{equation*}
Hence \(s'(x) = 0\; \aded\).


\itemch{d}
Let 
\begin{equation*}
\mu = \mu_a + mu_s \qquad  d\mu = h\,dm + d\mu_s
\end{equation*}
be the unique and positive Lebesgue decomposition 
of Radon-Nikodym Theorem~6.10 \cite{RudinRCA87}. 
By Theorem~7.14 \cite{RudinRCA87}.
\begin{equation*}
\lim_{t\to 0+} \frac{\mu\bigl([x+t)\bigr)}{m\bigl([x+t)\bigr)}
= \lim_{t\to 0+} \frac{\mu\bigl([x+t)\bigr)}{t}
= h(x) \quad \aded[m].
\end{equation*}
But
\begin{equation*}
\lim_{t\to 0+} \frac{\mu\bigl([x+t)\bigr)}{t} = f'(x) \quad \aded[m].
\end{equation*}
So by the the uniqueness of $h$, we have \(f' = h\;\aded[m]\).
Hence, by \eqref{eq:ex7.12:sx} we have \(\mu_s([a,x)] = s(x)\).

\paragraph{Equivalence 1.}
Assume \(\mu\perp m\), then \(\mu_a = h\,dm \perp m\).
Since $h$ is positive \(h = 0\;\aded[m]\) and so \(f'=0\;\aded[m]\).
Conversely, assume  \(f'=0\;\aded[m]\). Then \(\mu_a = h\,dm = 0\)
and thus \(\mu_a \perp m\) trivially.
Since \(\mu_s \perp m\) so is \(\mu\).

\paragraph{Equivalence 2.}
Assume  \(\mu \ll m\). Then $f$ maps nullsets to nullsets and 
by Theorem~7.18 \cite{RudinRCA87} $f$ is absolutely continuous.
Conversely, assume $f$ is absolutely continuous.
then by Theorem~7.18 \cite{RudinRCA87} 
\begin{equation*}
f(x) - f(a) = \int_a^x f'(t)\,dt  \qquad a\leq x \leq b
\end{equation*}
Thus \(\mu_s = 0\) and so \(\mu = \mu_a\) and \(\mu \ll m\).

\itemch{e}
We know that by construction \eqref{eq:ex7.12e:f}
 \(f \leq \varphi\) 
and \(f = \varphi\;\aded[m]\) and both are nondecreasing.
Pick \(x\in[a,b]\) such that \(f(x) = \varphi(x)\) and \(f'(x)\) exists.
Hence if \(h>0\) and \(x\pm h \in [a,b]\) then
\begin{equation*}
     \frac{\varphi(x-h) - \varphi(x)}{-h} 
\leq f'(x) 
\leq \frac{\varphi(x+h) - \varphi(x)}{h}.
\end{equation*}
Therefore
\begin{align*}
     \varliminf_{h\to x^-} \frac{\varphi(x+h) - \varphi(x)}{h}
&\leq \varlimsup_{h\to x^-} \frac{\varphi(x+h) - \varphi(x)}{h} \\
&\leq f'(x) \\
&\leq \varliminf_{h\to x^+} \frac{\varphi(x+h) - \varphi(x)}{h}
\leq \varlimsup_{h\to x^+} \frac{\varphi(x+h) - \varphi(x)}{h}.
\end{align*}

We will show 
\begin{equation} \label{eq:ex7.12e:desired}
\varlimsup_{h\to x^+} \frac{\varphi(x+h) - \varphi(x)}{h} = f'(x).
\end{equation}
The analog equality
\begin{equation*}
\varliminf_{h\to x^-} \frac{\varphi(x+h) - \varphi(x)}{h}  = f'(x)
\end{equation*}
can be shown with similar arguments. Thus the above inequalities
are actually equalities. Hence \(\varphi'(x) = f'(x)\).

By definition of \(\varliminf\), we have
a strictly decreasing sequence 
\(\{h_j\}_{j\in\N}\) 
of positive numbers
such that
\begin{align*}
\lim_{j\to\infty} h_j &= 0 \\
\lim_{j\to\infty} \frac{\varphi(x+h_j) - \varphi(x)}{h_j}
&=\varlimsup_{h\to x^+} \frac{\varphi(x+h) - \varphi(x)}{h}
\end{align*}
By the construction of $f$ in \eqref{eq:ex7.12e:f} 
the ``modified'' set \(S=\{x\in[a,b]: f(x)\neq \varphi(x)\}\)
is countable.
We also note that for \(\varphi(x_1) \leq f(x_2)\)
whenever \(a\leq x_1\leq x_2 \leq b\).
Thus in each segment \(H_n = [x + h_{n}, x + h_{n-1}]\)
we can find a sequence \(G_n = \{g_j\}_{j\in\N}\) such that
\begin{align*}
f(g_j) &= \varphi(g_j) \qquad (\mathrm{since}\;g_j\notin S) \\
\lim_{j\to\infty} g_j &= x + h_{n} 
\end{align*}
Hence
\begin{equation*}
\lim_{j\to\infty} f(g_j) \geq \varphi(x + h_{n})
\end{equation*}
and therefore
\begin{equation*}
\lim_{j\to\infty} \frac{\varphi(g_j) - \varphi(x)}{g_j - x}
 \geq \lim_{j\to\infty} \frac{f(g_j) - f(x)}{g_j - x}
 \geq \frac{\varphi(x+h_{n}) - \varphi(x)}{h_{n}}.
\end{equation*}
By a diagonal process we can pick 
a sequence \(\{x_j\}_{j\in\N}\) such that for all $n$
\begin{align*}
x_n &\in G_n \subset H_n \\
\lim_{n\to\infty} \frac{f(x_n) - f(x)}{x_n - x}
&\geq  \frac{\varphi(x+h_{n}) - \varphi(x)}{h_{n}} + \frac{1}{n}
\end{align*}
Clearly 
\begin{align*}
\lim_{n\to\infty} x_n &= x \\
f'(x) &= \lim_{n\to\infty} \frac{f(x_n) - f(x)}{x_n - x}
 \geq  \lim_{n\to\infty} \frac{\varphi(x+h_{n}) - \varphi(x)}{h_{n}}.
\end{align*}
Hence \eqref{eq:ex7.12e:desired} is true.

\end{itemize}


%%%%%%%%%%%%%% 13
\begin{excopy}
Let \(BV\) be the class of all $f$ on \([a,b]\) that have bounded variation
on \([a,b]\), as defined after Theorem~7.19. Prove the following statements.
\begin{itemize}
\itemch{a} 
Every monotonic bounded function on \([a,b]\) is in \(BV\).

\itemch{b}
If \(f \in BV\) is real, there exists bounded monotonic functions
\(f_1\) and \(f_2\) so that \(f=f_1-f_2\).\\
\emph{Hint}: Imitate the proof of Theorem~7.19.

\itemch{c}
If \(f\in BV\) is left-continuous then \(f_1\) and \(f_2\) 
can be chosen in \ich{b} so as to be also left-continuous.

\itemch{d}
If \(f\in BV\) is left-continuous then there is a Borel measure \(\mu\) 
on \([a,b]\) that satisfies
\begin{equation*}
f(x) - f(a) = \mu([a,x)) \qquad (a\leq x \leq b);
\end{equation*}
\(\mu\ll m\) if and only if $f$ is AC on \([a,b]\).

\itemch{e}
Every \(f\in BV\) is differentiable \(\aded[m]\), and \(f'\in L^1(m)\).
\end{itemize}
\end{excopy}

\begin{itemize}
\itemch{a}
Let $f$ be monotonic on \([a,b]\).
If 
\begin{equation*}
a\leq t_0 < t_1 < \cdots < t_N\leq b
\end{equation*}
then clearly 
\begin{equation*}
\sum_{j=1}^N |f(t_j) - f(t_{j-1})|
= \left| \sum_{j=1}^N \bigl(f(t_j) - f(t_{j-1})\bigr) \,\right|
= |f(t_N)-f(t_0)|.
\end{equation*}
Therefore,
\begin{equation*}
\sup_{a=t_0 < t_1 < \cdots < t_N=b} \sum_{j=1}^N |f(t_j) - f(t_{j-1})|
= |f(b)-f(a)|
\end{equation*}

\itemch{b}
By Theorem~7.19 \cite{RudinRCA87}
if
\begin{equation}  \label{eq:Fx:ex:7.13b}
F(x) = 
\sup_{a=t_0 < t_1 < \cdots < t_N=x} \sum_{j=1}^N |f(t_j) - f(t_{j-1})|
= |f(b)-f(a)|
\end{equation}
then \(F\pm f\) are nondecreasing and continuous.
Now
\begin{equation*}
f_1 = (F+f)/2 \qquad f_2 = -(F-f)/2
\end{equation*}
satisfy the requirements.

\itemch{c}
Assume $f$ is left-continuous, then $F$ defined in \eqref{eq:Fx:ex:7.13b}
is left-continuous as well. To show this pick \(x\in[a,b]\) and \(\epsilon>0\).
Take \(\delta>0\) such that \((a\leq x-\delta\) and 
\begin{equation*}
\sum_{j=1}^n |f(b_j) - f(a_j)| < \epsilon
\end{equation*}
whenever 
\begin{equation*}
\sum_{j=1}^n b_j - a_j < \delta \qquad  \forall j\, a\leq a_j\leq b_j\leq b
\end{equation*}
Now we can see that 
\begin{equation*}
F(x-\delta) + \epsilon \geq F(x).
\end{equation*}
Thus our construction satisfy the left-continuous requirement.

\itemch{d}
Assume such required Borel measure \(\mu\) exists.
Now $f$ maps sets of measure $0$ to sets of measure $0$ since 
\(\mu \ll m\). By Theorem~7.18 \cite{RudinRCA87} $f$ is absolutely continuous.

Conversely, assume $f$ is absolutely continuous.
By Theorem~7.20 \cite{RudinRCA87} \(f'\in L^1(m)\) and we can define
\begin{equation*}
\mu(E) = \int_E f'(x)\,dm(x)
\end{equation*}
for each Borel set $E$. Now \(\mu\) is a Borel measure and \(\mu \ll m\).

\itemch{e}
By \ich{b} we can represent \(f=f_1-f_2\) where \(f_1,f_2\) are monotonic.
As we saw in Exercise~12\ich{a} these functions of at most countable
number of points where they are not continuous.
By Exercise~\ref{ex:7.12}\ich{c}
\(f_j'\) exist \aded\ and \(f_j'\in L^1(m)\) for \(j=1,2\)
and so is \(f' = f_1' - f_2'\).

\paragraph{Note:} Here we \emph{cannot} show that 
\(f(x)-f(a) = \int_a^x f't)\,dt\). Absolute continuity is missing for this.
\end{itemize}

%%%%%%%%%%%%%% 
\begin{excopy}
Show that the product of two absolutely continuous functions on \([a,b]\)
is absolutely continuous. Use this to derive a theorem about integration
by parts.
\end{excopy}

Let \(f_1,f_2\) be two absolutely continuous functions and \(\epsilon>0\).
Thus there is a \(\delta>0\) such that 
\begin{equation*}
\sum_{j=1}^N |f(t_j) - f(t_{j-1})|<\epsilon
\end{equation*}
if 
\begin{equation} \label{eq:ex7.14}
\sum_{j=1}^N |t_j - t_{j-1}|<\delta.
\end{equation}
Put \(f = f_1\cdot f_2\) and 
for abbreviation put \(a_k = f_1(a_k)\) and \(b_k = f_2(a_k)\). 
Using:
\begin{align*}
|f(t_j) - f(t_{j-1})|
&= |(f_1\cdot f_2)(t_j) - (f_1\cdot f_2)(t_{j-1})|
 = |a_j b_j - a_{j-1}b_{j-1}| \\
&= \left|
    \bigl(a_{j-1} + (a_j - a_{j-1})\bigr)
    \bigl(b_{j-1} + (b_j - b_{j-1})\bigr) - a_{j-1}b_{j-1}
   \right| \\
&= \bigl|
      (a_j - a_{j-1})b_{j-1} 
    + (b_j - b_{j-1})a_{j-1} 
    + (a_j - a_{j-1}) (b_j - b_{j-1}) 
   \bigr| \\
&\leq 
    |a_j - a_{j-1}|\cdot \|f_2\|_\infty
    + |b_j - b_{j-1}|\cdot \|f_1\|_\infty
    + |a_j - a_{j-1}|\cdot |b_j - b_{j-1}| \\
&\leq 
    |a_j - a_{j-1}|\cdot 3\|f_2\|_\infty
    + |b_j - b_{j-1}|\cdot \|f_1\|_\infty
\end{align*}
Assuming \eqref{eq:ex7.14} we have:
\begin{align*}
\sum_{j=1}^N |f(t_j) - f(t_{j-1})|
&\leq
    3\|f_2\|_\infty \sum_{j=1}^N  |a_j - a_{j-1}|
  + \|f_1\|_\infty \sum_{j=1}^N  |b_j - b_{j-1}|  \\
&\leq \left(\|f_1\|_\infty + 3\|f_2\|_\infty \right) \epsilon
\end{align*}
Thus $f$ is absolutely continuous as well.

Now assume \(u,v: [a,b]\to\C\) are absolutely continuous functions.
By the above result, they are simultaneously differentiable almost everywhere.
Hence by Leibnitz's rule
\begin{equation*}
\frac{d\bigl(f(x)g(x)\bigr)}{dx} = f'(x)g(x) + f(x)g'(x) \; \aded
\end{equation*}
Consequently
\begin{equation*}
f(b) - f(a) =
\int_a^b \frac{d\bigl(f(x)g(x)\bigr)}{dx}\,dx 
= \int_a^b f'(x)g(x)\,dx + \int_a^b f(x)g'(x)\,dx.
\end{equation*}



%%%%%%%%%%%%%% 15
\begin{excopy}
Construct a monotonic functions $f$ on \(\R^1\) so that \(f'(x)\) exists
(finitely) for every \(x\in \R^1\), but \(f'\) is not a continuous function.
\end{excopy}

Based on \cite{Gelb1996} Chapter~3 counterexample~2 page~36. Let
\begin{equation*}
f(x) = \left\{
\begin{array}{ll}
x^2\sin(1/x) - 2x^2 + x \quad & \mathrm{if}\; x < 0 \\
0            \quad & \mathrm{if}\; x =  0 \\
x^2\sin(1/x) + 2x^2 + x \quad & \mathrm{if}\; x > 0
\end{array}
\right.
\end{equation*}
Now
\begin{equation*}
f'(x) = \left\{
\begin{array}{ll}
2\bigl(x\sin(1/x) + 2|x|\bigr) + \bigl(2-\cos(1/x)\bigr)
    \quad & \mathrm{if}\; x\neq 0 \\
1   \quad & \mathrm{if}\; x=0 \\
\end{array}
\right.
\end{equation*}
Clearly \(f'>0\) but is not continuous in $0$ since \(\cos(1/x)\) is not.


%%%%%%%%%%%%%%
\begin{excopy}
Suppose \(E\subset [a,b]\), \(m(E)=0\).
Construct an absolutely continuous monotonic function $f$ on \([a,b]\)
so that \(f'(x)=\infty\) at every \(x\in E\).

\emph{Hint}: \(E \subset \cap V_n\), \(V_n\) open, \(m(V_n)<2^{-n}\).
Consider the sum of characteristic functions of these sets.
\end{excopy}

Following the hint.
Let \(V_n\) as suggested, but also require \(V_{n+1}\subset V_n\).
This could be easily done, by possibly replacing
\begin{equation*}
V_n \leftarrow \cap_{k\leq n} V_n
\end{equation*}
Define
\begin{equation*}
g(x) = \sum_{n\in\N} \chhi_{V_n}
\end{equation*}
Thus \(g(x)=0\) iff \(x\in E\) and \(g\in L^1([a,b])\).
Now define
\begin{equation*}
f(x) = \int_a^x g(t)\,dt.
\end{equation*}
By Exercise~10\ich{a} of Chapter~6 \cite{RudinRCA87}
$g$ is uniformly integrable (by itself).
Thus $f$ is absolutely continuous.
Now let \(x\in E\) and \(M\in\N\).
There is some \(\delta>0\) such that
\((x-\delta,x+\delta) \subset V_M\) and 
\begin{equation*}
\frac{f(x+h) - f(x)}{h} 
= \frac{1}{h}\int_x^{x+h} g(t)\,dt
\geq \frac{1}{h}\int_x^{x+h} \sum_{n=1}^M \chhi_{V_n}\,dt
= Mh/h = M.
\end{equation*}
Hence \(f'(x)=\infty\).



 By Theorem~7.19
$f$ is differentiable \aded\ and
\begin{equation*}
f(x) = \int_a^x f'(t)\,dt \qquad \forall x\in[a,b]
\end{equation*}


%%%%%%%%%%%%%% 17
\begin{excopy}
Suppose \(\{\mu_n\}\) is a sequence of positive Borel measures on \(\R^k\) and
\begin{equation*}
\mu(E) = \sum_{n=1}^\infty \mu_n(E).
\end{equation*}
Assume \(\mu(\R^k)<\infty\). Show that \(\mu\) is a Borel measure.
What is the relation between Lebesgue decomposition of the \(\mu_n\) 
and that of \(\mu\)?

Prove that 
\begin{equation*}
(D\mu)(x) = \sum_{n=1}^\infty (D\mu_n)(x) \quad \aded[m].
\end{equation*}
Derive corresponding theorems for sequences \(\{f_n\}\) 
of positive nondecreasing functions on \(\R^1\) and their sums \(f=\sum f_n\).
\end{excopy}

Let \(\mu_n = \mu_{n,a} + \mu_{n,s}\) be the Lebesgue decomposition.
Clearly 
\(\sum_{n=1}^\infty \mu_{n,a}\)
and
\(\sum_{n=1}^\infty \mu_{n,s}\)
are positive Borel measures and
\begin{equation*}
\sum_{n=1}^\infty \mu_{n,a} \ll m 
\qquad
\sum_{n=1}^\infty \mu_{n,s} \perp m.
\end{equation*}
By uniqueness, these make the Lebesgue decomposition of \(\mu\). 

By Theorem~7.14 \cite{RudinRCA87} and the above observation, if 
\(d\mu = f\,dm + d\mu_s\) and
\(d\mu_n = f_n\,dm + d\mu_{n,s}\)
then 
\begin{equation*}
(D\mu)(x) = f(x) = \sum_{n=1}^\infty f_n(x) = \sum_{n=1}^\infty (D\mu_n)(x) 
\quad\aded[m].
\end{equation*}

Say \(\{f_n\}_n\in\N\) are positive nondecreasing functions on \(\R^1\)
such that 
\begin{equation*}
f = \sum_{n=1}^\infty f_n \in L^\infty(\R,m).
\end{equation*}
Then
\begin{equation*}
f'(x) = \sum_{n=1}^\infty f_n'(x) \qquad \aded[m].
\end{equation*}
This can be shown by using Exercise~12 and replacing $a$ and \(f(a)\)
by \(-\infty\) and 
\begin{equation*}
\lim_{t\to-\infty} f(t) = \inf_{t\in\R} f(t).
\end{equation*}


%%%%%%%%%%%%%% 18
\begin{excopy}
Let \(\varphi_0(t) = 1\) on \([0,1)\), \(\varphi_0(t) = -1\)
on \([1,2)\), extended \(\varphi_0\) to \(\R^1\) so as to have a period $2$,
and define \(\varphi_n(t) = \varphi_0(2^n t)\), \(n=1,2,3,\ldots\)

\textsl{
Assume that \(\sum |c_n|^2 < \infty\) and prove that the series
\begin{equation} \label{eq:ex7.18}
\sum_{n=1}^\infty c_n\varphi_n(t)
\end{equation}
converges then for almost every $t$.
}

Probabilistic Interpretation: The series \(\sum(\pm c_n)\) converges
with probability $1$.\\
\emph{Suggestion}: \(\{\varphi_n\}\) is orthonormal on \([0,1]\), 
hence \eqref{eq:ex7.18} is the Fourier series of some \(f\in L^2\).
If \(a = j\cdot 2^{-N}\), \(b = (j+1)\cdot 2^{-N}\), \(a<t<b\),
and \(s_N = c_1\varphi_1 + \cdots + c_N\varphi_N\), then, for \(n>N\),
\begin{equation*}
s_N(t) = \frac{1}{b-a} \int_a^b s_N\,dm = \frac{1}{b-a} \int_a^b s_n\,dm,
\end{equation*}
and the last integral converges to \(\int_a^b f\,dm\), as \(n\to\infty\).
Show that \eqref{eq:ex7.18} at almost every Lebesgue point of $f$.
\end{excopy}

If \(m<n\) then within any segment \([j 2^{-n},  (j+1) 2^{-n}]\) clearly
\(\varphi_m\) is constant, while \(\varphi_n\) equally alternates signs
in \(2^{n-m}\) subsegments. Thus \(\varphi_m \perp \varphi_n\).
This argument also shows that the above equality of the integrals holds.

The existence of $f$ is implied by the completeness of 
the Hilbert space \(L^2([0,1],m)\).

Let $x$ be a Lebesgue point
\index{Lebesgue point}
of $f$ such that \(x\in B\) --- the countable \(B = \{j2^{-n}: j,n\in\Z^+\}\). 
We will define
nicely shrinking sets
\index{nicely shrinking sets}
for $x$ with \(\alpha = 1/2\). 
Let \(r_n=2^{-n}\) 
and pick \(E_n = [jr_n/2, (j+1)r_n/2]\subset [0,1]\)
such that \(x\in E_n\). 
We have 
\(E_n \subset B(x,r_n)\) and
\begin{equation*}
m(E_n) = r_n/2 = \alpha r_n = m\bigl(B(x,r_n)\bigr).
\end{equation*}
Now by the above observation and Theorem~7.10 \cite{RudinRCA87}
\begin{equation*}
f(x) 
= \lim_{n\to\infty} \frac{1}{m(E_n)} \int_{E_n} f\,dm
= \lim_{n\to\infty} s_n(x).
\end{equation*}
Thus the series converges for all Lebesgue points except for at most 
a countable set.


%%%%%%%%%%%%%% 19
\begin{excopy}
Suppose $f$ is continuous on \(\R^1\), \(f(x)>0\) if \(0 < x < 1\), \(f(x)= 0\)
otherwise. Define
\begin{equation*}
  h_c(x) = \sup \{n^c f(nx): n = 1,2,3,\ldots\}.
\end{equation*}
Prove that
\begin{itemize}
\itemch{a} \(h_c\) is in \(L^1(\R^1)\) if \(0<c<1\),
\itemch{b} \(h_1\) is in weak \(L^1\) but not in \(L^1(\R^1)\),
\itemch{c}  \(h_c\) is not weak \(L^1\) if \(c>1\).
\end{itemize}
\end{excopy}

\textbf{Remimder:} A function $f$ is 
\emph{weak} \(L^1\)\
\index{weak L1@weak \(L^1\)}
if 
\begin{equation*}
\lambda\cdot m\left( \{x: |f(x)| > \lambda\}\right)
\end{equation*}
is a bounded function of \(\lambda\).

For abbreviation we put \(M = \|f\|_\infty\).
For any $c$ we have.
\begin{align*}
\int_{\R^1} |h_c(t)|\,dt
&=    \int_0^1 h_c(t)\,dt
 =    \sum_{m=1}^\infty \int_{1/(m+1)}^{1/m} h_c(t)\,dt \\
&=    \sum_{m=1}^\infty \int_{1/(m+1)}^{1/m} 
      \bigl(\sup_{n\in\N} n^c f(nt)\bigr)\,dt 
 =    \sum_{m=1}^\infty \int_{1/(m+1)}^{1/m} 
      \bigl(\sup_{1\leq n\leq m+1} n^c f(nt)\bigr)\,dt
\end{align*}
and also put \(I_m=[1/(m+1),1/m]\)
the mid-third \(K= [1/2,2/3]\) and using continuity, let 
\begin{equation*}
T = \inf\{f(x): x\in K\} = \min\{f(x): x\in K\} > 0.
\end{equation*}

\begin{itemize} 
\itemch{a}
If \(0<c<1\) then
\begin{align*}
\int_{\R^1} |h_c(t)|\,dt
&\leq \sum_{m=1}^\infty \int_{1/(m+1)}^{1/m} (m+1)^c M\,dt \\
&=    M \sum_{m=1}^\infty (m+1)^c \left(\frac{1}{m+1} - \frac{1}{m}\right) 
 =    M \sum_{m=1}^\infty (m+1)^c \frac{1}{m(m+1)} \\
&=    M \sum_{m=1}^\infty m^{-1}(m+1)^{c-1} \\
&\leq M \sum_{m=1}^\infty (m+1)^{c-2} \\
&\leq M \int_1^\infty x^{c-2}\,dx 
=     \frac{M}{c-1}x^{c-1}\bigm|_1^\infty = M(1-c) < \infty.
\end{align*}
Hence \(h_c\in L^1(\R)\).

\itemch{b}
clearly \(m\cdot I_{2m} \subset K\) for 
any \(m\geq 1\). Thus we have the following  ``generous'' estimation:
%  (m/2) [1/(m+1),1/m]  \subset [1/3,2/3]
\begin{align*}
\int_{\R^1} |h_1(t)|\,dt
&\geq \sum_{m=1}^\infty \int_{I_m} h_1(t)\,dt 
 \geq \sum_{m=1}^\infty \int_{I_{2m}} h_1(t)\,dt 
 \geq \sum_{m=1}^\infty \int_{I_{2m}} 
      \bigl(\sup_{1\leq n\leq 2m+1} n^1 f(nt)\bigr)\,dt \\
&\geq \sum_{m=1}^\infty \int_{I_{2m}} m T 
 =    T \sum_{m=1}^\infty m \ell(I_{2m})  
 =    T \sum_{m=1}^\infty m \frac{1}{2m(2m+1)} \\
&=    (T/2) \sum_{m=1}^\infty  \frac{1}{2m+1}
 \geq (T/2) \sum_{m=1}^\infty  \frac{1}{2m+2}
 =    (T/4) \sum_{m=2}^\infty  \frac{1}{m} = \infty.
\end{align*}
Hence \(h_1 \notin L^1(\R)\).

To show that \(h_1\) is weak \(L^1\) we first assume that \(\lambda < M\).
Then
\begin{equation*}
\lambda \cdot m\bigl( \{x\in\R: |h_1(x)| > \lambda\}\bigl)
\leq M \cdot \ell([0,1]) = M.
\end{equation*}
Now assume that \(\lambda \geq M\). If \(|h_1(x)|> \lambda\)
then \(n\cdot f(nx) > \lambda\) for some $n$. 
Hence \(nM > \lambda\) and  \(x < 1/n\).
Thus \(n \geq \lfloor \lambda/M \rfloor \geq 1\) and 
\begin{equation*}
0 < x < 1/\lfloor \lambda/M \rfloor.
\end{equation*}
Since
\begin{equation*}
\lambda < \left(\lfloor \lambda/M\rfloor + 1\right)M 
        \leq 2 \lfloor \lambda/M\rfloor M 
\end{equation*}
we can estimate
\begin{align*}
\lambda \cdot m\bigl( \{x\in\R: |h_1(x)| > \lambda\}\bigl) 
&<  \lambda /\lfloor \lambda/M \rfloor < 2M.
\end{align*}
In both cases the expression is bounded for any \(\lambda\) and 
\(h_1\) is weak \(L^1\).

\itemch{c}
By checking odd and even cases \(\lfloor n/2\rfloor I_n \subset K\)
for all \(n\geq 2\). 
It is sufficient to consider \(\lambda \geq T\).
If 
\begin{equation*}
n \geq 2\left\lceil \left(\frac{\lambda}{T}\right)^{1/c} \right\rceil
  \geq        \left(\frac{\lambda}{T}\right)^{1/c} 
\end{equation*}
then
\begin{equation*}
n^c T > \lambda.
\end{equation*}
Therefore if \(x \in I_{2n+1} \cup I_{2n}\) then \(nx \in K\) 
and \(h_c(x) \geq \lambda\), consequently
\begin{equation*}
(0,1/2n] \subset \{x\in\R: h_c(x)> \lambda\}.
\end{equation*}
Now since \(\lambda/T \geq 1\) we see that
\begin{align*}
\lambda \cdot m\bigl( \{x\in\R: |h_1(x)| > \lambda\}\bigl) 
&\geq \lambda \cdot \ell\bigl((0,1/n]\bigr) \\
&\geq \lambda \bigm/ 
      \left\lceil (\lambda/T)^{1/c} \right\rceil 
 \geq \lambda \bigm/ 2 (\lambda/T)^{1/c} \\
&= \left(T^{-c}/2\right) \lambda^{1-1/c}
\end{align*}
is clearly unbounded for \(\lambda\).
Hence \(h_c\) is not weak \(L^1(\R)\).
\end{itemize}


%%%%%%%%%%%%%% 20
\begin{excopy}
\begin{itemize}
\itemch{a}
For any set \(E\subset \R^2\), the boundary \(\partial E\) of $E$ is,
by definition the closure of $E$ minus the interior of $E$. Show that
$E$ is Lebesgue measurable whenever \(m(\partial E) = 0\).

\itemch{b}
Suppose that $E$ is the union of a (\emph{possibly uncountable}) collection
of \emph{closed} discs in \(\R^2\) whose radii are at least $1$.
Use \ich{a} to show that $E$ is Lebesgue measurable.

\itemch{c} 
Show that the conclusion of \ich{b} is true even when the
radii are unrestricted.

\itemch{d}
Show that some unions of closed discs of radius $1$
are not Borel sets (See Sec.~2.21.)

\itemch{e}
Can discs be replaced by triangles, rectangles, arbitrary polygons, etc.,
in all this?
What is the relevant geometric property?
\end{itemize}
\end{excopy}

\begin{itemize}

\itemch{a}
For every set $E$ we have
\begin{equation*}
E = \inter{A} \disjunion (E \cap \partial E).
\end{equation*}
Now if \(m(\partial E) = 0\)
then also \(m(E \cap \partial E) = 0\) and thus $E$ is measurable
as a union of two measurable sets.


\itemch{b}

{\small Using a hint of Dmitry Ryabogin
see 
\linebreak[1]
\texttt{http://www.math.ksu.edu/\~{}ryabs/tar13.pdf}
\linebreak[1]
or
\texttt{http://www.math.ksu.edu/\~{}ryabs/tar13.dvi}{}.}

Let \(\{D_j\}_{j\in J}\) be a family of closed discs
with radii \(\geq 1\). 
Let \(U = \cup_{j\in J} D_j\) and \(V = \cup_{j\in J} \inter{D_j}\).
Let $Y$ be the boundary of the $U$. Since $Y$ is closed, it is measurable.
Assume by negation that \(m(Y)>0\). 
By Lebesgue's density theorem~7.7 \cite{RudinRCA87}
there are points (actually almost all)
\(b\in Y\) such that their denisty is $1$.
Pick arbitrary \(b\in Y\) and let \(\iota = 1/4\)
(Actually and value \(0 < \iota < 1/2\) will do).
We will show that the density 
\begin{equation*}
\lim_{r\to 0^+} \frac{m\bigl(Y\cap B(b,r)\bigr)}{m(B_r)} \leq 1 - \iota < 1
\end{equation*}
which will provide the desired contradiction.
Let \(0<r<\iota\) and \(0<\epsilon<r\).
Since $b$ is a boundary point,
there exist some disc with radius \(\rho\geq 1\)
whose center $c$ such that \(\|b-c\| < \rho + \epsilon\).
\Wlogy, assume \(b=(0,0)\) and \(c=(\rho + \epsilon, 0)\).
Let $S$ be the intersection
of the discs \(B(b,r)\) and \(B(c,\rho + \epsilon)\).
These boundaries of these discs intersects in two points
\((r\cos\alpha, \pm r \sin\alpha)\). Denote the following points
\begin{alignat*}{2}
Q &= (r\cos\alpha, r \sin\alpha) & & \qquad\textrm{Positive intersection} \\
P &= (r\cos\alpha, 0)            & & \qquad\textrm{Projection of\ }\; Q \\
A &= (\epsilon, 0)               & & \qquad\textrm{Nearest point of}\; 
                                     B(c,\rho+\epsilon)\;\textrm{to}\; b \\
R &= (r, 0)
                                 & & \qquad\textrm{Nearest point of}\; 
                                     B(b,r)\;\textrm{to}\; c \\
\end{alignat*} 
Now we estimate the area which is the measure of the intersection
\begin{align*}
m\bigl(B(b,r) \cap B(c,\rho + \epsilon)\bigr)
&= 2m\bigl(B(b,r) \cap B(c,\rho + \epsilon) 
           \cap \{(x,y)\in\R^2: y\geq 0\} \bigr) \\
&\geq 2 m\bigl(\vartriangle(A,R,Q)\bigr) 
 \geq 2 m\bigl(\vartriangle(b,R,Q)\bigr) 
 \geq 2 m\bigl(\vartriangle(P,R,Q)\bigr) \\
&= 2 \frac{\|P-R\|\cdot \|Q-P\|}{2}
 = r(1-\cos\alpha)\cdot r\sin\alpha.
\end{align*}

To use this estimatation, we need to see how \(\alpha\)
that clearly satisfies \(0<\alpha < \pi/2\)
depends on $r$, \(\rho\) and \(\epsilon\).
Equating the square distance of \(\overline{Qc}\) by Pythagoras
\begin{equation*}
\rho^2 = \|P-c\|^2 + \|Q-P\|^2 
= \bigl((\rho+\epsilon)-r\cos\alpha\bigr)^2 +(r\sin\alpha)^2
= (\rho+\epsilon)^2-2r(\rho+\epsilon)\cos + r^2
\end{equation*}
hence
\begin{equation*}
\cos\alpha = \frac{r^2 + (\rho+\epsilon)^2 - \rho^2}{2r(\rho+\epsilon)} > 0
\end{equation*}
and so
\begin{equation*}
\lim_{\epsilon\to 0^+} \cos\alpha = r/2\rho \leq r/2.
\end{equation*}
For each (sufficiently small) \(r>0\), We can pick 
a disc \(D_{j(r)} = D_j\) of the family with distance of 
\(\epsilon\geq 0\) to $x$ 
(radius \(\rho\) and center $c$ such that \(\|c-x\| = \rho+\epsilon\))
such that \(\alpha_r = \alpha > \pi/2 - r\).

We start estimating the density of $Y$ in $x$. 
We note that \(U \subset \inter{V} \subset Y\),
hence \(U \cap Y = \emptyset\).
\begin{equation} \label{eq:ex7.20:DYx}
D_Y(x) = 
\lim_{r\to0+} \frac{m\bigl( B(x,r) \cap Y\bigr)}{m(B_r)}
\leq
1 - \lim_{r\to0+} \frac{m\bigl( B(x,r) V\bigr)}{m(B_r)}\;.
\end{equation}
Focusing on the last limit 
\begin{align*}
\lim_{r\to 0+} \frac{m\bigl( B(x,r) \cap V\bigr)}{m(B_r)}
&\geq \lim_{r\to 0+} \frac{m\bigl( B(x,r) \cap D_{j(r)}\bigr)}{m(B_r)} 
 \geq \lim_{r\to 0+} \frac{r^2(1-\cos\alpha_r)\sin\alpha_r}{\pi r^2} \\
&\geq \lim_{r\to 0+} (1-\cos\alpha_r)\sin\alpha_r / \pi = 1/\pi > 0.
\end{align*}
Returning to \eqref{eq:ex7.20:DYx} we see that 
\begin{equation*}
D_Y(x) \leq 1 - 1/\pi < 1.
\end{equation*}
which is the the desired contradiction.


\itemch{c}
The statement of \ich{b} can 
be trivially to any lower bound \(r\geq 0\) instead of $1$ for the radii.
So now if \calF\ is any collection if closed circles,
let \(\calF(r)\) be the sub-collection consisting of closed circles 
of \calF\ whose radii is at least $r$.
Now since
\begin{equation*}
\calF = \cup_{n=1}^\infty \calF(1 - 1/n)
\end{equation*}
we have
\begin{equation*}
\bigcup_{C\in\calF} C
= \cup_{n=1}^\infty \left(\bigcup_{C\in \calF(1 - 1/n)} C\right).
\end{equation*}
Thus any union of closed circles, is a countable union of 
Lebesgue measurable sets
and so the union is Lebesgue measurable as well.

\itemch{d}
Pick a non measurable set \(A\subset \R^1\). Now let
\begin{equation*}
U 
= \cup_{a\in A} B\bigl((a,1),1\bigr)
= \cup_{a\in A} \{(x,y)\in\R^2: (x-a)^2+(y-1)^2 \leq 1\}.
\end{equation*}
Assume by negation that $U$ is a Borel set. Then so would be
\begin{equation*}
A \times \{0\} = U \cap \left(\R\times \{0\}\right).
\end{equation*}
Now $A$ is generated by a countable sequence of countable unions
and countable intersections of the 
\(\sigma\)-algebra base of open sets. Restricting these open sets to \(\R^1\)
results with open sets in \(\R^1\). Producing the same countable processes
in \(\R^1\) now generates $A$ that must be Borel, which is a contradiction.


\itemch{e}
Instead of disc, the required property is that the geometric shape
would have a lower bound on the internal angles. Not ``too accute'' angle.
Thus perfect polygons would work.
\end{itemize}

%%%%%%%%%%%%%% 21
\begin{excopy}
  If $f$ is a real function on \([0,1]\) and
\begin{equation*}
  \gamma(t) = t + if(t)
\end{equation*}
the length of the graph of $f$ is, by definition, the total variation
of \(\gamma\) on \([0,1]\). Show that this length is finite if and
only if \(f\in BV\).  (See Exercise~13.)  Show that it is equal to
\begin{equation*}
  \int_0^1 \sqrt{1 + [f'(t)]^2}\,dt
\end{equation*}
if $f$ is absolutely continuous.
\end{excopy}


We will deal with more general case of a curve, 
following Theorem~(8.4)(iii) of \cite{Saks37}.

\begin{llem}
Let \(\phi:[0,1]\to \R^2\) be a continuous curve given by 
\(\phi(t) = \bigl(x(t),y(t)\bigr)\).
Let
\begin{equation*}
d(t_0,t_1) = \|\phi(t_1) - \phi(t_0)\| = 
\sqrt{\bigl(x(t_1) - x(t_0)\bigr)^2 + \bigl(y(t_1) - y(t_0)\bigr)^2}
\qquad (0\leq t_0 \leq t_1 \leq 1)
\end{equation*}
the distance of the curve \(\phi\) on the parameter range \([t_0,t_1]\).
Let 
\begin{equation*}
S(\phi,\alpha;t) = \sup_{\alpha = a_0 < a_1 < \cdots < a_n = t}
                   \sum_{j=1}^n d(a_j,a_{j-1})
\qquad (0\leq \alpha \leq t \leq 1)
\end{equation*}
be the length of the curve on the parameter range \([\alpha,t]\).
\begin{itemize}
\itemch{i} The length $S$ of the curve is finite iff \(x(t)\) and \(y(t)\)
           have bounded variation.
\itemch{ii} If  \(x(t)\) and \(y(t)\) are absolutely continuous, then
\begin{equation} \label{eq:7.20:Seq}
S(\phi,0;1) = \int_0^1 \sqrt{\bigl(x'(t)\bigr)^2 + \bigl(y'(t)\bigr)^2}.
\end{equation}
\end{itemize}
\end{llem}
\begin{thmproof}
Assume the length is finite.
Let \(0=a_0 < a_1 < \cdots a_n = 1\) be any partition, then
\begin{equation*}
\sum_{j=1}^n |x(a_j) - x(a_{j-1})|
\leq \sum_{j=1}^n d(a_j,a_{j-1}) \leq S(\phi,0;1) < \infty
\end{equation*}
and so \(x(t)\) has bounded variation.
Simiar estimation can be done with \(y(t)\) that  has bounded variation as well.

Conversely, assume  \(x(t)\) and \(y(t)\) have  bounded variation.
Again \(0=a_0 < a_1 < \cdots a_n = 1\) be any partition, then
\begin{equation*}
\sum_{j=1}^n d(a_j,a_{j-1}) 
\leq \sum_{j=1}^n |x(a_j) - x(a_{j-1})| + |y(a_j) - y(a_{j-1})| < \infty
\end{equation*}
and the curve has finite length and \ich{i} is proved.

Now assume that  \(x(t)\) and \(y(t)\) are absolutely continuous.


We first show that so is \(S(\phi,0,t)\).
Let \(\epsilon>0\) and \(\delta\) be such that 
\(\sum_{k=1}^n |x(b_j) - x(a_j)| < \epsilon\) 
and
\(\sum_{k=1}^n |y(b_j) - y(a_j)| < \epsilon\) 
whenever \(\sum_{k=1}^n b_j - a_j < \delta\) where \(0 \leq a_j < b_j \leq 1\).
Now \(S(t) = S(\phi,0;t)\) is clearly monotonically increasing, 
and we have
\begin{eqnarray*}
\sum_{j=1}^n S(b_j) - S(a_j) 
&=&
\sum_{j=1}^n \sup_{a_j = t_0 < t_1 < \cdots < t_n = b_j} 
             \sum_{k=1}^n  d(t_{k-1},t_k) \\
&\leq&
\sum_{j=1}^n \sup_{a_j = t_0 < t_1 < \cdots < t_n = b_j} 
             \sum_{k=1}^n  |x(k) - x(t_{k-1})| + |y(k) - y(t_{k-1})|   \\
&\leq&
\sum_{j=1}^n 
\left(\sup_{a_j = t_0 < t_1 < \cdots < t_n = b_j} 
             \sum_{k=1}^n  |x(k) - x(t_{k-1})| \right) \\
&& +\;
\left(\sup_{a_j = t_0 < t_1 < \cdots < t_n = b_j} 
             \sum_{k=1}^n  |y(k) - y(t_{k-1})| \right)
\\
&<& 2\epsilon
\end{eqnarray*}

Thus \(S(t)\) is absolutely continuous as well.
By Theorem~7.18 \cite{RudinRCA87} \(x(t)\), \(y(t)\) and \(S(\phi,0;t)\)
are differentiable almost everywhere and consequently
are differentiable \emph{simultaneously} almost everywhere.
Let \(J\subset[0,1]\)
the set of points all three functions are differentiable
and \(t\in J\).
We have
\begin{equation*}
S(t+h) -S(t) 
\geq d(t+h,t) 
= \sqrt{\bigl(x(t+h) - x(t)\bigr)^2 + \bigl(y(t+h) - y(t)\bigr)^2}
\end{equation*}
for any \(h>0\). Hence
\begin{align*}
S'(t)
&\geq \lim_{h\to 0^+}  
   \sqrt{\bigl(x(t+h) - x(t)\bigr)^2 + \bigl(y(t+h) - y(t)\bigr)^2} \bigm/ h \\
&= \lim_{h\to 0^+}  
   \sqrt{\left(\bigl(x(t+h) - x(t)\bigr)/h\right)^2 + 
         \left(\bigl(y(t+h) - y(t)\bigr)/h\right)^2}  \\
&= \sqrt{\bigl(x'(t)\bigr)^2 + \bigl(y'(t)\bigr)^2}.
\end{align*}
We will show the reversed inequality holds almost everywhere.
For any interval \(I=(\alpha,\beta)\subset[0,1]\) we use the notations
\begin{align*}
x(I) &= |x(\beta) - x(\alpha)| \\
y(I) &= |y(\beta) - y(\alpha)| \\
S(I) &= S(\phi,\alpha;\beta). 
\end{align*}
Define the ``bad set''
\begin{equation*}
A = \left\{t\in J: 
        S'(t) > \left(\bigl(x'(t)\bigr)^2 + \bigl(y'(t)\bigr)^2\right)^{1/2}
        \right\}. 
\end{equation*}
and its subsets
\begin{align}
A_n = \bigl\{ & t\in J:   \notag \\
      & \forall I=(\alpha,\beta)\ni t,\; 0<m(I) < 1/n  \notag \\
      & \Rightarrow  \label{eq:7.21:An}
        S(I)/m(I)
        \geq \left(
                \bigl(x(I)/m(I)\bigr)^2 + \bigl(y(I)/m(I)\bigr)^2
             \right)^{1/2} + 1/n
        \bigr\}
\qquad (n\in\N).
\end{align}
(Notice that \(g'(t) = \lim_{m(I)\to 0+} g(I)/m(I)\)
if $g$ is differentiable at $t$.)
Clearly \(A = \cup_{n\in\N} A_n\).

Fix $n$ and pick arbitrary \(\epsilon>0\).
There exist a ``sufficiently rich'' partition 
\begin{equation*}
0 = t_0 < t_1 < \cdots t_p = 1
\end{equation*}
putting \(J_k = [t_{k-1},t_k]\),
such that 
\(m(J_k) = t_k - t_{k-1} < 1/n\) for \(k\in\N_p\) and 
\begin{equation} \label{eq:7.21:Sleq}
S([0,1]) = \sum_{k=1}^p S(J_k) \leq \sum_{k=1}^p d(t_{k-1},t_k) + \epsilon
\end{equation}
On the other hand, by \eqref{eq:7.21:An} we have
\begin{equation} \label{eq:7.21:Sgeq}
S(J_k) \geq d(t_{k-1},t_k) + m(J_k)/n
\end{equation}
whenever \(j_k \cap A_n \neq \emptyset\). 
By \eqref{eq:7.21:Sleq} \eqref{eq:7.21:Sgeq} we have
\newcommand{\sumJkAn}{
            \sum_{\stackrel{1\leq k \leq p}{J_k\cap A_n \neq \emptyset}}}
\begin{equation*}
m(A_n) 
\leq \sumJkAn m(J_k) 
\leq n \sumJkAn S(J_k) - d(t_{k-1},t_k) 
\leq n\epsilon.
\end{equation*}
Since \(\epsilon\) was arbitrarily picked, \(m(A_n) = 0\) 
hence \(m(A) = 0\) and the desired reversed inequality holds.
Therefore
\begin{equation} \label{eq:ex7.21:Sdif}
S'(t) = \sqrt{\bigl(x'(t)\bigr)^2 + \bigl(y'(t)\bigr)^2}
\end{equation}
which completes the proof.
\end{thmproof}

For this specifc exercise, we note that 
in a curve of a function graph
\(x(t)=t\) and \(x'(t)=1\).
Now the lemma provides the solution.


%%%%%%%%%%%%%% 22
\begin{excopy}
\begin{itemize}
\itemch{a}
Assume that both $f$ and  its maximal function \(Mf\) are in \(L^1(\R^k)\).
Prove that then \(f(x)=0\;\aded[m]\).
\emph{Hint}: To every other \(f\in L^1(\R^k)\) corresponds a constant
\(c=c(f)>0\) such that
\begin{equation*}
  (Mf)(x) \geq |x|^{-k}
\end{equation*}
whenever \(|x|\) is sufficiently large.

\itemch{b}
Define \(f(x) = x^{-1}(\log x)^{-2}\) if \(0<x<\frac{1}{2}\),
\(f(x)=0\) on the rest of \(\R^1\).
Then \(f\in L^1(\R^1)\).
Show that
\begin{equation*}
  (Mf)(x) \geq |2x \log(2x)|^{-1} \qquad (0<x<1/4)
\end{equation*}
so that \(\int_0^1 (Mf)(x)\,dx = \infty\).
\end{itemize}
\end{excopy}

\begin{itemize}

\itemch{a}
Assume by negation \(\|f\|_1\neq 0\).
Then there exist and integer \(a\in\N\) such that 
\[\alpha = \int_{B(0,a)} |f(x)|\,dm(x) > 0.\]
Assume \(x\in\R^k\) and \(\|x\|\geq a\). 
We note that \(m(B_r) = c_k \pi r^k\) 
(where \(c_k = \pi^{k/2}/\Gamma(n/2+1)\)).
Now
\begin{align*}
(Mf)(x)
&= \sup_{r>0} \frac{1}{m(B_r)} \int_{B(x,r)} |f(t)|\,dt \\
&\geq \frac{1}{m(B_{|x|+a})} \int_{B(x,|x|+a)} |f(t)|\,dt 
 \geq \frac{1}{m(B_{|x|+a})} \int_{B(0,a)} |f(t)|\,dt \\
&= \alpha / m(B_{|x|+a}).
\end{align*}
Define and estimate for each integer \(n>a\)
\begin{align*}
S_n
&= \int_{B(0,n)} (Mf)(x)\,dm(x) \\
&\geq \sum_{r=a+1}^n \int_{B(0,r)\setminus B(0,r-1) } (Mf)(x)\,dm(x) 
 \geq \sum_{r=a+1}^n 
     \int_{B(0,r)\setminus B(0,r-1) } \alpha / m(B_{|x|+a}) \,dm(x) \\
&\geq \sum_{r=a+1}^n 
     m\bigl(B(0,r)\setminus B(0,r-1)\bigr) \alpha / m(B_{2r}) \\
&= \sum_{r=a+1}^n \alpha c_k\bigl(r^k - (r-1)^k\bigr) 
                  \bigm/ \bigl( c_k (r+a)^k \bigr) 
 = \frac{\alpha}{2^k} 
   \sum_{r=a+1}^n \bigl(r^k - (r-1)^k\bigr) \bigm/ r^k \\
&\geq \frac{\alpha}{2^k} 
      \sum_{r=a+1}^n \frac{1}{r}.
\end{align*}
Since \(\lim_{n\to\infty} S_n = \infty\) we get the
contradiction \(Mf \notin L^1(\R^k)\).

\itemch{b}
Put \(F(x) = -1/\log(x)\). Now
\begin{equation*}
F'(x) = (-1/x) \cdot \left(-\bigl(\log(x)\bigr)^{-2}\right) = f(x)
\end{equation*}
when \(0<x<1/2\).
Hence
\begin{equation*}
\|f\|_1 
= \lim_{h\to 0^+} \int_h^{1/2} f(x)\,dx
= -1/\log(1/2) - \lim_{h\to 0^+}  1/\log(h)
= -1/\log(1/2) 
< \infty.
\end{equation*}
Thus \(f\in L^1(\R^1)\).
But if  \(0<x<1/4\) then
\begin{equation*}
(Mf)(x) 
= \sup_{r} \frac{1}{2r} \int_{x-r}^{x+r} f(t)\,dt
\geq \frac{1}{2x} \int_0^{2x} f(t)\,dt
= \bigl(F(2x) - F(0)\bigr)/2x 
= |2x\log(2x)|^{-1} 
\end{equation*}
Hence
\begin{align*}
\int_0^1 (Mf)(x)\,dx
&\geq \int_0^{1/4} (Mf)(x)\,dx \\
&\geq \int_0^{1/4} \frac{1}{2x} \cdot \frac{-1}{\log(2x)} \,dx \\
&= \frac{-1}{2}\left.\left(\log(|\log(2x)|)\right)\right|_0^{1/4} 
 = \frac{-1}{2} \bigl(\log(1/2) - \lim_{x\to +\infty}\log(x)\bigr) \\
&= \infty.
\end{align*}
Indeed \(Mf \notin L^1([0,1])\).
\end{itemize}


%%%%%%%%%%%%%% 23
\begin{excopy}
The definition of Lebesgue points, as made in Sec.~7.6, applies to individual
integrable functions, not to the equivalence classes discussed in Sec.~3.10.
However, if \(F\in L^1(\R^k)\) is one of these equivalence classes, one may call
a point \(x\in \R^k\)
a \emph{Lebesgue point of}
\index{Lebesgue point}
$F$ if there is a complex number, let us call it \((SF)(x)\) to be $0$
at those points \(x\in \R^k\) that are not Lebesgue points of $F$.

Prove the following statement: If \(f\in F\), and $x$ is a Lebesgue
points of $f$, then $x$ is also a Lebesgue point of $F$, 
and \(f(x)(SF)(x)\).  Hence \(SF\in F\).

Thus $S$ ``selects'' a member of $F$ that has a \emph{maximal} set
of Lebesgue points.
\end{excopy}

Assume $x$ is
\index{Lebesgue point}
a~Lebesgue point of $f$.
The set $E$ of non Lebesgue points of $f$ has measure zero by 
Theorem~7.7 \cite{RudinRCA87}. Thus
changing $f$ on $E$ does not effect the values of 
\begin{equation*}
A(x) \frac{1}{m(B_r)} \int_{B(x,r} |f(y) - f(x)|\,dm(y)
\end{equation*}
except possibly for \(x\in E\).
Hence Lebesgue points of $f$ are Lebesgue points of $F$ as well.

%%%%%%%%%%%%%%%%%
\end{enumerate}
