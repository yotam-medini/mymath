% -*- latex -*-
% $Id: rudinrca6.tex,v 1.5 2008/07/19 08:56:55 yotam Exp $


%%%%%%%%%%%%%%%%%%%%%%%%%%%%%%%%%%%%%%%%%%%%%%%%%%%%%%%%%%%%%%%%%%%%%%%%
%%%%%%%%%%%%%%%%%%%%%%%%%%%%%%%%%%%%%%%%%%%%%%%%%%%%%%%%%%%%%%%%%%%%%%%%
%%%%%%%%%%%%%%%%%%%%%%%%%%%%%%%%%%%%%%%%%%%%%%%%%%%%%%%%%%%%%%%%%%%%%%%%
\chapterTypeout{Complex Measures} % 6

%%%%%%%%%%%%%%%%%%%%%%%%%%%%%%%%%%%%%%%%%%%%%%%%%%%%%%%%%%%%%%%%%%%%%%%%
%%%%%%%%%%%%%%%%%%%%%%%%%%%%%%%%%%%%%%%%%%%%%%%%%%%%%%%%%%%%%%%%%%%%%%%%
\section{Notes}

Intuitively the following lemma shows that \(f\,d\mu\) is regular.
\begin{llem} \label{llem:fdmu:abscont}
Let $f$ be a \(\mu\)-measurable function on $X$
such that \(\int|f|\,d\mu<\infty\).
For any \(\epsilon>0\) there exists \(\delta>0\) such that
\(\int_E |f|\,d\mu < \epsilon\) whenever \(\mu(E)<\delta\).
\end{llem}
\begin{thmproof}
Define the complex measure \(\lambda(E) = \int_E f\,d\mu\).
Clearly \(\lambda \ll \mu\). Theorem~6.11 gives the desired conclusion.
\end{thmproof}

%%%%%%%%%%%%%%%%%%%%%%%%%%%%%%%%%%%%%%%%%%%%%%%%%%%%%%%%%%%%%%%%%%%%%%%%
%%%%%%%%%%%%%%%%%%%%%%%%%%%%%%%%%%%%%%%%%%%%%%%%%%%%%%%%%%%%%%%%%%%%%%%%
\section{Exercises} % pages 132-134

%%%%%%%%%%%%%%%%%
\begin{enumerate}
%%%%%%%%%%%%%%%%%

%%%%%%%%%%%%%% 01
\begin{excopy}
If \(\mu\) is a complex measure on a \(\sigma\)-algebra \frakM,
and if \(E \in \frakM\), define
\begin{equation*}
\lambda(E) = \sup \sum |\mu(E_i)|,
\end{equation*}
the supremum being taken over all \emph{finite} partitions \(\{E_i\}\) of $E$.
Does it follow that \(\lambda = |\mu|\)?
\end{excopy}

Yes.

Clearly \(\lambda(E) \leq |\mu|(E)\). To show the opposite inequality,
let \(\epsilon > 0\). We can find some countable
partition \(\{E_i\}\) of $E$
such that
\begin{equation*}
 \sum_{i\in\N} |\mu(E_i)| > |\mu|(E) - \epsilon/2.
\end{equation*}
So  we can find some integer \(m<\infty\) such that
\begin{equation*}
 \sum_{i=1}^m |\mu(E_i)| > |\mu|(E) - \epsilon.
\end{equation*}
Hence \(\lambda(E) \geq |\mu|(E)\) and thus
\(\lambda(E) = |\mu|(E)\).


%%%%%%%%%%%%%%
\begin{excopy}
Prove that the example given at the end of Sec.~6.10 has the stated properties.
\end{excopy}

\paragraph{Property 1.}
On \(I=(0,1)\), let \(\mu\) be the Lebesgue measure
and \(\lambda\) the counting measure.
By negation, let \(\lambda = \lambda_s + \lambda_a\) be the
Lebesgue decomposition.
For any \(x\in I\), we have
 \(\lambda(\{x\}) = 1\) and  \(\mu(\{x\}) = 0\)
Hence
\(\lambda_s(\{x\}) = 0\) and  \(\lambda_a(\{x\}) = 1\).
Since \(\mu\perp \lambda_a\) we have \(I = A \disjunion B\)
such that for any Lebesgue measurable $E$, we have
\begin{equation*}
\mu(E) = \mu(E\cap A) \qquad \textrm{and} \qquad
\lambda_a(E) = \lambda_a(E\cap B).
\end{equation*}
Since
\(\lambda_a(\{x\}) = 1\)
for any \(x\in I\), we must have \(B=I\), but then \(\mu=0\)
which is a contradiction.

\paragraph{Property 2.}
Assume by negation \(h\in L^1(\lambda)\) such that \(d\mu = h\,d\lambda\).
If \(h=0\) then \(\mu=0\) so we may assume that \(h(x)\neq 0\)
for some \(x\in I\). But then
\begin{equation*}
0 = \mu(\{x\}) = h(x)\lambda(\{x\}) = h(x) \neq 0.
\end{equation*}
which is again a contradiction.


%%%%%%%%%%%%%%
\begin{excopy}
Prove that the vector space \(M(X)\) of  all complex regular Borel measures
on a locally compact Hausdorff space $X$ is a Banach space if
\(\|\mu\| = |\mu|(X)\).
\emph{Hint}: Compare Exercise~8, Chap.~5.
[That the difference of any two members of \(M(X)\) is in \(M(X)\)
was used in the first
paragraph of the proof of Theorem~6.19; supply a proof of this fact.]
\end{excopy}

Let \(\lambda,\mu \in M(X)\).
For \(E = \disjunion E_j\) use definition to compute
\begin{align}
\lambda(E) - \mu(E) \notag \\
&= \left(\sum_{j\in\N} \lambda(E_j)\right) -
   \left(\sum_{j\in\N} \mu(E_j)\right) \notag \\
&= \sum_{j\in\N} \lambda(E_j) - \mu(E_j) \label{eq:ex6.3} \\
&= \sum_{j\in\N} (\lambda - \mu)(E_j) \notag \\
&= (\lambda - \mu)(E).
\end{align}
The equality in \eqref{eq:ex6.3} holds since the series converge absolutely.

If we define \((a\mu)(E) = a(\mu(E)\) for \(a\in\C\), all vector space
equalities trivally hold. Thus we need only to show completeness.
Let \(\{\mu_j\}_{j\in\N}\) be a Cauchy sequence of measures.
That is for each \(\epsilon>0\) there exists some $N$ such that
\(|\mu_j - \mu_k|(X) < \epsilon\) if \(j,k>N\).
Since
\(|\mu_j - \mu_k|(E) \leq |\mu_j - \mu_k|(X)|\)
for any \(E\in\frakM\), the sequence
\(\{\mu_j(E)\}\) is a Cauchy sequence, and we can define
its limit \(\mu(E) = \lim_{j\to\infty} \mu_j(E)\).
It is easy to see that with \(\mu\), the vector space equalities still hold.
We need to show that \(\mu\in M(X)\).

We have the unique decompositions
\begin{align*}
\mu_j &= \lambda_j^+ - \lambda_j^- + i(\nu_j^+ - \nu_j^-) \\
\mu   &= \lambda^+ - \lambda^- + i(\nu^+ - \nu^-)
\end{align*}
where
\(\lambda_j^+\), \(\lambda_j^-\), \(\nu_j^+\) and \(\nu_j^-\)
are positive measures since
the mappings\(x\mapsto|x|\), \(\Re\) \(\Im\) are continuous,
and
\(\lambda^+\), \(\lambda^-\), \(\nu^+\) and \(\nu^-\)
are non-negative functions on \frakM.

For temporary abberviation, we use
\(\lambda=\lambda^+\) and \(\lambda_j = \lambda_j^+\).

Pick arbitrary \(K<\infty\) so we can estimate
\begin{align*}
\sum_{k\leq K} \lambda(E_k)
&= \sum_{k\leq K} \lim_{j\to\infty} \lambda_j(E_k)
 = \lim_{j\to\infty} \sum_{k\leq K} \lambda_j(E_k)
 = \lim_{j\to\infty} \lambda_j\left(\Disjunion_{k\leq K} E_k\right) \\
&\leq \lim_{j\to\infty} \lambda_j(E) \\
&= \lambda(E).
\end{align*}
Hence
\begin{equation*}
\sum_{k\in\N} \lambda(E_k) \leq \lambda(E).
\end{equation*}
Looking at \(f_j = \lambda_j(E_k)\) as function on \(k\in\N\)
with at the coundting measure on \N, we utilize
Lebesgue's dominated convergence theorem
in the following \eqref{eq:ex:6.3}.

Now
If \(E = \disjunion_{j\in\N} E_i\) all in \frakM, then
\begin{align}
\lambda(E)
&= \lim_{j\to\infty} \lambda_j(E)
 = \lim_{j\to\infty} \lambda_j\left(\Disjunion_{K\in\N} E_k\right) \notag \\
&= \lim_{j\to\infty} \sum_{k\in\N} \lambda_j(E_k) \notag \\
&= \sum_{k\in\N} \lim_{j\to\infty} \lambda_j(E_k)  \label{eq:ex:6.3} \\
&= \sum_{k\in\N} \lambda(E_k) \notag
\end{align}

Similarly, we can derive the equlaities for
\(\lambda^-\), \(\nu^+\) and \(\nu^-\). Then we sum them to get
\begin{equation*}
\mu(E) = \sum_{k\in\N}  \mu(E_k).
\end{equation*}


%%%%%%%%%%%%%%
\begin{excopy}
Suppose \(1 \leq p \leq \infty\), and $q$  is the exponent conjugate to $p$.
Suppose \(\mu\) is a positive \(\sigma\)-finite measure and $g$ is a measurable
function such that \(fg\in L^1(\mu)\) for every
\(f\in L^p(\mu)\). Prove that \(g\in L^q(\mu)\).
\end{excopy}

\iffalse
% By H\"older's inequality \(\|fg\|_1 \leq \|f\|_p \|g\|_q\).
Clearly \(f \mapsto \int fg\;d\mu\) is a linear functional on \(L^p(\mu)\).
Let \(X=\disjunion_{j\in\N} X_j\) a decomposition of the space
such that \(\mu(X_j)<\infty\). We set \(S_n = \disjunion_{j\leq n} X_j\)
and let \(g_n = g_{|S_n}\).
Using Exercise~2.5\ich{a} we have
\begin{equation*}
\|g_n\|_q = \|g_n\|_\infty \leq \|g\|_\infty.
\end{equation*}

Clearly \(g\in L^q(\mu)\) iff \(|g| L^q(\mu)\).
Since \(|\mu|(X)<\infty\), we have the constant \(1\in L^q\)
for all \(q\in[1,\infty]\). Hence also
\(g\in L^q(\mu)\) iff \(g\pm 1\in L^q(\mu)\).
Let \(X_n=\{x\in X: n-1 \leq |g(x)| < n\}\).

Assume first \(1<p,q<\infty\), and by negation, assume
\begin{equation} \label{eq:ex6.4:neg}
g\notin L^q(\mu).
\end{equation}
Hence we can have the estimatation.
\begin{equation*}
\sum_{n\in \N} n^q \mu(X_n)
 \geq \sum_{n\in \N} \int_{X_n} |g|^q\,d\mu \\
 = \int_X |g|^q\,d\mu
 = \|g\|q^q = \infty.
\infty
\end{equation*}

Define the function
\begin{equation*}
h(x) =
\left\{
 \begin{array}{ll}
 0 & \quad x\in X_0\\
 n^{q/p} & \quad x\in X_n \;\textrm{and}\; n>0
 \end{array}
\right.
\end{equation*}

Now
\begin{equation*}
\|h\|_p^p
= \int_X h^p\,d\mu
= \sum_{n=0}^\infty \int_{X_n} h^p\,d\mu
= \sum_{n\in\N}  \left(n^(q/p)\right)^p\,d\mu
= \sum_{n\in\N}  n^q\,d\mu
\end{equation*}
Hence \(\||g|-1\|_q = \infty\),

\fi

For each $n$ define
\begin{equation*}
g_n(x) =
\left\{
 \begin{array}{ll}
 g(x) & \quad |g(x)| \leq n\\
 ng(x)/|g(x)| & \quad |g(x)| > n
 \end{array}
\right.
\end{equation*}
Clearly \(\lim_{n\to\infty}g_n(x) = g(x)\;\aded\)\,.
Now we can define functionals \(\Lambda_n\in (L^p(\mu))^*\) by
\begin{equation*}
\Lambda_n(f) = \int_X fg_n\,d\mu.
\end{equation*}
With the estimatation
\begin{equation*}
|\lambda_n{f}|
\leq \int_X |fg_n|\,d\mu
\leq \int_X |fg|\,d\mu
< \infty
\end{equation*}
we use the
\index{Banach-Steinhaus}
\index{Steinhaus}
Banach-Steinhaus Theorem~5.8 (\cite{RudinRCA87})
to get a unifrom bound \(\|\Lambda_n\|\leq M\).
Now for each \(f\in L^p(\mu)\) we have
\begin{equation*}
\lim_{n\to\infty} \int_X fg_n\,d\mu = \int_X fg\,d\mu
\end{equation*}
and if we define \(\Lambda(f) = \int_X fg\,d\mu\) we
have \(\|\Lambda\| \leq M\).


Applying the uniquness part of Theorem~6.16 \cite{RudinRCA87}
we get the desired result for \(p<\infty\).

Now assume \(p=\infty\). Obviously the constant function
\(1\in L^\infty(\mu)\) and by assumption
\(|\int_X 1\cdot g\,d\mu|<\infty\). Hence \(f\in L^1{\mu}\).


%%%%%%%%%%%%%% 05
\begin{excopy}
Suppose $X$ consists of two points $a$ and $b$; define
\(\mu(\{a\}) = 1\),
\(\mu(\{b\}) = \mu(X) = \infty\), and
\(\mu(\emptyset) = 0\).
Is it true for this \(\mu\) ,
that \(L^\infty(\mu)\) is the dual space of \(L^1(\mu)\)?
\end{excopy}

No. Clearly
\begin{equation*}
L^1(\mu) = \{f\in \C^X: f(b)=0\}.
\end{equation*}
Now consider \(g_1,g_2\in L^\infty(\mu)\) defined as
\(g_1=1\), \(g_2(a)=1\) and \(g_2(b)=0\). Both induces the same
functional in \(\left(L^1(\mu)\right)^*\).


%%%%%%%%%%%%%%
\begin{excopy}
Suppose \(1 < p < \infty\), and prove that
\(L^q(\mu)\) is the dual space of \(L^p(\mu)\)
even if \(\mu\) is not \(\sigma\)-finite.
(as usual \(1/p+1/q=1\).)
\end{excopy}

For positive  \(\sigma\)-finite measure the claim has been proved.
Assume now that \((X,\frakM,\mu)\) is a complex measure space.
By Theorem~6.12 (\cite{RudinRCA87}) there exists a measurable
function $h$ such that \(\forall x\in X, |h(x)|=1\) and
\(d\mu = h\,d|\mu|\).
We note, that as \emph{sets of functions} we trivially have
\(L^p(\mu)=L^p(|\mu|)\) and \(L^q(\mu)=L^q(|\mu|)\).
The vector space operations are identical.
Also the topolgies are the same, since the norms are invariant,
that is
\begin{equation*}
\|f\|_{p,\mu}
= \left(\int |f|^p\,d\mu\right)^{1/p}
= \left(\int |f|^p\,d|\mu|\right)^{1/p}
\|f\|_{p,|\mu|}.
\end{equation*}

Hence, given a \(\Lambda\in (L^p(\mu))^*\)
it may be viewed as a functional on \(L^p(|\mu|)\).
From results on positive measures we have \(g\in L^q(|\mu|)\)
such that
\begin{equation*}
\Lambda(f) = \int_X fg\,d|\mu|
\end{equation*}
for all \(f\in L^p(\mu)\).
But then
\begin{equation*}
\Lambda(f) = \int_X f(g/h)\,d\mu
\end{equation*}
and clearly \(g/h\in L^q(\mu)\).
The uniqness of \(g/h)\) is proved in same way
as in the proof of Theorem~6.16 (\cite{RudinRCA87}).


%%%%%%%%%%%%%%
\begin{excopy}
Suppose \(\mu\) is a complex Borel measure on \([0,2\pi]\)
(or on the unit circle \T),
and define the Fourier coefficients of \(\mu\) by
\begin{equation*}
\widehat{\mu}(n) = \int e^{-int}\,d\mu(t) \qquad (n=0,\pm 1,\pm 2, \ldots).
\end{equation*}
Assume that \(\widehat{\mu}(n) \to 0 \) as \(n\to +\infty\)
and prove that then
\(\widehat{\mu}(n) \to 0 \) as \(n\to -\infty\).

\emph{Hint}: The assumption also holds with \(f\,d\mu\) in  place of \(d\mu\)
if $f$ is any trigonometric polynomial, hence if $f$ is continuous,
hence if $f$ is any bounded Borel function,
hence if \(d\mu\) is replace by \(d|\mu|\).
\end{excopy}

Following the hint. For each \(m\in\Z\) we have
\begin{equation*}
\lim_{n\to+\infty} \int e^{-int}e^{imt}\,d\mu(t)
= \lim_{n\to+\infty} \int e^{-i(n-m)t}\,d\mu(t)
= \lim_{n\to+\infty} \int e^{-int}\,d\mu(t)
= 0.
\end{equation*}
If for \(j=1,2\) we have
\begin{equation*}
\lim_{n\to+\infty} \int e^{-int}f_j(t)\,d\mu(t) = 0
\end{equation*}
Then
\begin{equation*}
\lim_{n\to+\infty} \int e^{-int}(f_1+f_2)(t)\,d\mu(t) = 0.
\end{equation*}
Thus we have similar convergence \(\lim_{n\to+\infty}\widehat{(f\mu)}(n)\)
for trigonometric polynomials $f$.
Similarly this holds for all continuous $f$, in particular for
the unique $h$ such that \(|\mu| = h\mu\) and \(|h(t)|=1\)
Now
\begin{align}
\widehat{\mu}(-n)
&= \int e^{int}\,d\mu(t)
 = \int e^{int}/h(t)\,d|\mu(t)|
 = \overline{\int \overline{e^{int}}/\overline{h(t)}\,d|\mu(t)|}
 = \overline{\int e^{-int}/\overline{h(t)}\,d|\mu(t)|} \notag \\
&= \overline{\int e^{-int}\left(h(t)/\overline{h(t)}\right)\,d\mu(t)}
   \label{eq:ex6.7}.
\end{align}
As before, the convergence to zero also holds for
\(h(t)/\overline{h(t)}\) and thus the values \eqref{eq:ex6.7}
converge to zero.


%%%%%%%%%%%%%%
\begin{excopy}
In the terminology of Exercise~7, find all \(\mu\) such that
\(\hat{\mu}\) is periodic, with respect to $k$
[This means that \(\hat{\mu}(n+k) = \hat{\mu}(n)\) for all integers $n$;
of course $k$ is also assumed to be an integer.]
\end{excopy}

Equivalently, we can look for all measurable functions $f$, such that
\begin{equation*}
\int_\T e^{-int}f(t)\,dm(t)
\end{equation*}
is periodic in $n$.


\paragraph{Example:} Given $k$, for each \(j\in\N_k\) let
\(K = \{e^{2j\pi i/k}: 0\leq j < k\}\) and set
\(\mu(\{P\}) = 1/k\) for each \(P\in K\)
and \(\mu(\T\setminus K) = 0\). Complete \(\mu\) to a measure,
then
\begin{equation*}
\widehat{\mu}(n) = \left\{
 \begin{array}{ll}
 1 & \quad \textrm{if}\; n = 0 \bmod k \\
 0 & \quad \textrm{if}\; n \neq 0 \bmod k
 \end{array}
 \right.
\end{equation*}
If \(\mu\ll m\) then \(d\mu = h\,dm\) for some \(h\in L^1(\T)\)
but \(\widehat{h}(n)\) cannot be periodic (unless \(h=0\)),
since \(\lim_{n\to\infty}\widehat{h}(n) = 0\).


%%%%%%%%%%%%%%
\begin{excopy}
Suppose that \(\{g_n\}\) is a sequence of positive continuous functions on
\(I=[0,1]\), that \(\mu\) is a positive Borel measure  on $I$, and that
\begin{itemize}
\itemch{i} \(\lim_{n\to\infty} g_n(x) = 0\quad \aded [m]\).
\itemch{ii} \(\int_I g_n\,dm = 1\) for all $n$,
\itemch{iii} \(\lim_{n\to\infty} \int_I fg_n\,dm = \int_I f\,d\mu\)
              for every \(f\in C(I)\).
\end{itemize}
Does it follow that \(\mu\perp m\)?
\end{excopy}

Yes. We will show that there exists a Borel subset \(D\subset I\) such that
\(m(D)=0\) and \(\mu(E) = \mu(E\cap D)\) for each Borel set $E$.

Consider the ``bad set''
\begin{equation*}
 B = \{x\in I: \limsup g_n(x) > 0\}.
\end{equation*}
Since \(m(B) = 0\) by \ich{i} we can redefine \(g_n(x)=0\)
for each \(x\in B\).
This redefinition removes the \aded\ restriction from \ich{i}, but
does not effect
the other assumptions \ich{ii}, \ich{iii} and not the desired conclusion.

Pick some \(k\in\N\).
\index{Egoroff}
By Egoroff's Theorem (Exercise~3.16 \cite{RudinRCA87})
there exists a measurable \(D_n \subset I\) such that
\(m(D_k) < 1/k\) and such that
\(\{g_n\}\) converges \emph{uniformly} to $0$ on \(C_k\)
where \(E_k = I \setminus D_k\).
We will show that \(\mu(E_k) = 0\).
\iffalse
Hence
\begin{equation*}
\lim_{n\to\infty} \int_{E_k} g_n\,dm
\leq \bigl(1 - m(D_k)\bigr) \lim_{n\to\infty} \sup_{x\in E_k} g_n(x)
= 0.
\end{equation*}
\fi
Given \(\epsilon>0\), we
apply Lusin Theorem~2.24 (\cite{RudinRCA87}) to the function
\(\chhi_{E_k}\) to get a function \(f_k\in C(I)\)
such that \(\|f_k\|_\infty = 1\) and
\begin{equation*}
m\bigl(\{x\in I: f(x) \neq \chhi_{E_k}(x)\}\bigr) < \epsilon.
\end{equation*}
\begin{align*}
\mu(E_k)
&= \int_I \chhi{E_k}\,d\mu \\
&\leq \int_I f_k\,d\mu + \epsilon
\end{align*}




%%%%%%%%%%%%%% 10
\begin{excopy}
Let \((X,\frakM,\mu)\) be a positive measure space. Call a set
\(\Phi \subset L^1(\mu)\)
\index{uniformly  integrable}
\emph{uniformly  integrable}
if to each \(\epsilon < 0\) corresponds \(\delta>0\) such that
\begin{equation*}
\left|\int_E f\,f\mu\right| < \epsilon
\end{equation*}
whenever \(f\in\Phi\) and \(\mu(E) < \delta\).
\begin{itemize}

\itemch{a}
Prove that every finite subset of \(L^1(\mu)\) is uniformly integrable.

\itemch{b}
Prove the following convergence theorem of \index{Vitali} Vitali:

\textsl{
If (i) \(\mu(X)<\infty\), (ii) \(\{f_n\}\) is uniformly integrable,
(iii) \(f_n(x)\to f(x)\;\aded\) as \(n\to\infty\),
and (iv) \(|f(x)|<\infty\;\aded\), then \(f\in L^1(\mu)\) and
\begin{equation*}
\lim_{n\to\infty} \int_X|f_n - f|\,d\mu = 0.
\end{equation*}
}
\emph{Suggestion}: Use \index{Egoroff} Egoroff's theorem

\itemch{c}
Show that \ich{b} fails if \(\mu\) is a Lebesgue measure on
\((-\infty,\infty)\), even if \(\{\|f_n\|_1\}\) is assumed to be bounded.
Hypothesis (i) can therefore  not be omitted in \ich{b}.

\itemch{d}
Show that the hypothesis (iv)  is redundant in \ich{b} for some \(\mu\)
(for instance, for Lebesgue measure on a bounded interval), but that there are
finite measures for which the omission of~(iv)
would make \ich{b} false.

\itemch{e}
Show that Vitali's theorem implies Lebesgue's dominated convergence theorem,
for finite measure space. Construct an example in which Vitali's theorem
applies although the hypothesis of
Lebesgue's theorem does not hold.

\itemch{f}
Construct a sequence \(\{f_n\}\), say on \([0,1]\), so that
\(f_n(x)\to 0\) for every $x$,
\(\int f_n\to 0\) but \(\{f_n\}\) 
is not uniformly integrable (with respect to Lebesgue's measure).

\itemch{g}
However, the following converse of Vitali's theorem is true:

\textsl{
If \(\mu(X)<\infty\), \(f_n\in L^1(\mu)\), and
\begin{equation*}
\lim_{n\to\infty} \int_E f_n\,d\mu
\end{equation*}
exists for every \(E\in \frakM\), then \(\{f_n\}\) is uniformly integrable.
}

Prove this by completing the following outline.

Define \(\rho(A,B) = \int |\chhi_A - \chhi_B|\,d\mu\).
Then \((\frakM,\rho)\) is a complete metric space
(modulo sets of measure $0$), and \(E\mapsto \int_E f_n\,d\mu\) is
a continuous for each $n$.
If \(\epsilon > 0\), there exist \(E_0\), \(\delta\), $N$
(Exercise~13, Chap~5) so that
\begin{equation} \label{eq:ex:6.10}
\left| \int_E (f_n - f_N)\,d\mu\right| < \epsilon
\qquad \textrm{if}\qquad
\rho(E, E_0) < \delta, \qquad n > N.
\end{equation}
If \(\mu(A)<\delta\), \eqref{eq:ex:6.10} holds with \(B = E_0 - A\)
and \(C = E_0 \cup A\)
in place of $E$.
Thus \eqref{eq:ex:6.10} holds with $A$ in place of $E$ and \(2\epsilon\)
in place of \(\epsilon\).
Now apply \ich{a} to \(\{\seq{f}{N}\}\): There exists \(\delta'>0\)
such that
\begin{equation*}
\left| \int_A f_n\,d\mu \right| < 3\epsilon
\qquad \textrm{if}\qquad
\mu(A) < \delta', \qquad n=1,2,3,\ldots.
\end{equation*}
\end{itemize}
\end{excopy}

We will need the following result.
\begin{llem}
If
\(\Phi=\{f_j: j\in J\}\) is uniformly  integrable,
then
\(\{|f_j|: j\in J\}\) is uniformly  integrable.
\end{llem} \label{llem:abs:unifinteg}
\begin{thmproof}
Define the complex plane quartans
\begin{align*}
\C_0 &= \{z\in\C: \Re(z)>0 \;\wedge\; \Im(z)\geq 0\} \\
\C_j &= \{e^{j\pi i/2}z: z\in \C_0 \} \quad j=1,2,3.
\end{align*}
Now
\begin{equation*}
\C = \{0\} \disjunion \Disjunion_{j=0}^3 \C_j.
\end{equation*}
Pick arbitrary \(\epsilon> 0\). Let \(\delta>0\) such that 
\(|\int_E f\,d\mu|<\epsilon\) for all \(f\in\Phi\) 
whenever \(\mu(E)<\delta\).
Take such $E$, and define \(E_j = E\cap \C_j\) for \(j=0,1,2,3\).
For each such $j$
\begin{align*}
\int_{E_j} |f|\,d\mu
&= \int_{E_j} |\Re(f) + \Im(f)|\,d\mu \\
&\leq  \int_{E_j} |\Re(f)|\,d\mu + \int_{E_j} |\Im(f)|\,d\mu \\
&=  \left|\int_{E_j} \Re(f)\,d\mu\right| 
  + \left|\int_{E_j} \Im(f)\,d\mu\right| 
 =  \left|\Re\left(\int_{E_j} f\,d\mu\right)\right| 
  + \left|\Im\left(\int_{E_j} f\,d\mu\right)\right| \\
&\leq  2\max\left(\left|\Re\left(\int_{E_j} f\,d\mu\right)\right|,
                  \left|\Im\left(\int_{E_j} f\,d\mu\right)\right|\right) \\
&\leq  \sqrt{2}\left|\int_{E_j} f\,d\mu\right|.
\end{align*}

Hence
\begin{equation*}
\int_E |f|\,d\mu
=    \sum_{j=0}^3 \int_{E_j} |f|\,d\mu
\leq \sqrt{2}\left|\sum_{j=0}^3 \int_{E_j} f\,d\mu\right| 
<    \sqrt{2}\epsilon
\end{equation*}
and thus \(\Phi\) is uniformly integrable.
\end{thmproof}



\begin{itemize}

\itemch{a}
By Theorems~1.29 and~6.11 \cite{RudinRCA87} $f$ is uniformly integrable
(by itself) if \(f\in L^1(\mu)\).
For finite set \(\{f_j\}_{j=1}^n\), for each \(\epsilon>0\)
we pick \(\delta = \min\{\delta_j: 1\leq j \leq n\}\)
where \(\delta_j\) corresponds to \(f_j\)


\itemch{b}
By removing a subset of measure zero, we may assume
\(f_n(x) \to f(x)\) for all \(x\in X\).

Pick arbitrary \(\epsilon>0\).
By Egoroff's Theorem (Exercise~3.16)  
and by being uniform integrable, 
we can find some some \(\delta>0\) and a set \(E\subset X\) such that 
\begin{enumerate}
\itemch{i} \(\mu(X\setminus E)<\delta\) .
\itemch{ii} On \(X\setminus E\) the convergence \(f_n \to f\) is uniform.
\itemch{iii} \(\int_E |f_n|\,d\mu < \epsilon\) for all $n$, 
      see local lemma~\ref{llem:abs:unifinteg}.
\end{enumerate}
By Fatou's Lemma (Theorem~1.28 \cite{RudinRCA87}) we have
\begin{equation*}
\int_E |f|\,d\mu 
= \int_E |\lim_{n\to\infty}f_n|\,d\mu 
= \int_E |\liminf_{n\to\infty}f_n|\,d\mu 
\leq \liminf_{n\to\infty} \int_E |f_n|\,d\mu < \sqrt{2}\epsilon
\end{equation*}
Pick $m$ such that \(|f_n(x)-f(x)| < \epsilon\)
for all \(n\geq m\) and all \(x\in X\setminus E\).
Thus
\begin{equation*}
\left|\int_X f\,d\mu\right|
\leq
  \left|\int_{X\setminus E} f\,d\mu\right|
+ \left|\int_E f\,d\mu\right|
\leq
  \int_{X\setminus E} (|f_n| + \epsilon)\,d\mu + \sqrt{2}\epsilon < \infty.
\end{equation*}
Thus \(f\in L^1(\mu)\).

Now
\begin{equation*}
\int_X |f-f_n|\,d\mu
= \int_{X\setminus E} |f-f_n|\,d\mu + \int_E |f-f_n|\,d\mu 
% \leq \mu(X)\epsilon + 2\sqrt{2}\epsilon
\leq (\mu(X) + 2\sqrt{2})\epsilon
\end{equation*}
and the desired convergence holds.

\itemch{c}
For a counterexample, simple take \(f_n(x) = \chhi_{[0,n]}\)
that are uniformly integrable.
Clearly \(\lim_{n\to\infty} f_n = 1 \notin L^1(\R,m)\).

\itemch{d}
For Lebesgue measure $m$ on \([0,1]\), say we dropy the 
\ich{iv} requirement.

If we take a singleton space \(X=\{x\}\) with \(\mu(\{x\}=1\)
and \(f_n(x)=n\), then the set is trivially uniformly integrable,
by ``suggesting'' \(\delta=1/2<1\).
Clearly \(f(x)=\infty\) and \(\int_X|f_n-f|\,d\mu=\infty\).

\itemch{e}
Given the assumptions of Lebesgue's dominated convergence theorem ---
\(f_n\to f\) and \(|f_n|\leq g\in L^1(\mu)\).
we Actually need to show that \(\Phi=\{f_n\}_{n\in\N}\) 
is uniformly integrable.
By \ich{a}, the function \(\{g\}\) is uniformly integrable.
Given \(\epsilon>0\) there exists some \(\delta>0\) 
such that \(|\int_E g\,d\mu|<\epsilon\) whenever \(\mu(E)<\delta\).
For such $E$ and for any $n$, by the dominating condition, 
\(|\int_E |f_n|\,d\mu|<\epsilon\) as well, thus
\(\Phi\) is uniformly integrable.

We now show an
example that satisfies Vitali's lemma assumptions but not
Lebesgue's theorem. For all \(n\in\N\)
let \(f_n: \R\to\R\) defined by 
\begin{align*}
a_0 &= 0 \\
a_n &= \sum_{j=1}^n 1/n \\
f_n &= \chhi_{[a_n,a_{n+1}]}.
\end{align*}
Clearly \(\lim_{n\to\infty}=0\) 
and also \(\lim_{n\to\infty} \|f_n\|_1 = 0\)
but there is no dominating function for \(\{f_n\}_{n\in\N}\)
in \(L^1(\R,m)\).


\itemch{f}
% Define \(f_n = n\chhi_{(0,1/n^2]}\).
Define
\begin{equation*}
f_n(x) = \left\{
\begin{array}{ll}
n^2        & \qquad 0 < x \leq 1/2n \\
-n^2       & \qquad 1/2n < x < 1/n \\
0          & \qquad x=0\;\vee\; x \geq 1/n
\end{array}\right.
\end{equation*}
Clearly \(\lim_{n\to\infty} f_n(x) = 0\) for all \(x\in[0,1]\)
and \(\int_{[0,1]} f_n\,dm = 0\) but
\begin{equation*}
\int_{[0,1/2n]} f_n\,dm = n/2.
\end{equation*}




\itemch{g}
The triangle inequality \(\rho(A,B)\leq \rho(A,C) + \rho(B,C)\)
is trivial. To show completeness we follow \cite{Oxtoby1980} chapter~10.
Say \(\{E_j\}_{j\in\N}\) a Cauchy sequence.
For each $j$ there is \(n_j > n_{j-1}\) such that \(\rho(E_m,E_n)<2^{-j}\)
whenever \(m,n\geq j\). Let \(F_j=E_{n_j}\), 
we have \(\rho(F_j,F_k)<2^{-j}\) for all \(k>j\). Define
\begin{equation*}
H_j = \bigcap_{k=j}^\infty F_k \qquad 
E = \bigcup_{j=1}^\infty H_j.
\end{equation*}
The set $E$ consists of points that belong to all 
but a finite number of sets \(\{f_j\}_{j\in\N}\).
If \(x\in E\vartriangle  F_j\) then there are two cases
\begin{itemize}
\itemch{i} 
\(x\in  E\setminus F_j\) then we look for the first $k$
such that \(x\in F_{j+k}\).
\itemch{ii} 
\(x\in  F_j\setminus E\) then we look for the first $k$
such that \(x\notin F_{j+k}\).
\end{itemize}
In both cases \(x\in F_{j+k-1} \vartriangle F_{j+k}\).

For each  \(x\in H_j\vartriangle F_j = F_j \setminus H_j\)
by looking at the first $k$ such that \(x\notin F_{j+k}\)
we see that
\begin{equation*}
x \in F_{j+k-1} \vartriangle F_{j+k}.
\end{equation*}

Using the simple rule
\((A\vartriangle B) \cup (A\vartriangle B) \subset B\vartriangle C \),
now clearly
\begin{equation*}
E\vartriangle  F_j
\subset (E\vartriangle H_j)\cup (H_j\vartriangle F_j)
\subset \bigcup_{k=1}^\infty F_{j+k-1} \vartriangle F_{j+k}.
\end{equation*}
Consequently
\begin{equation*}
m(E\vartriangle  F_j)
\leq \sum_{k=1}^\infty m(F_{j+k-1} \vartriangle F_{j+k})
\leq \sum_{k=1}^\infty 2^{-(j+k)} = 2^{1-j}.
\end{equation*}
For any \(n\geq n_j\) we have
\begin{align*}
\rho(E,E_n) 
&= m\bigl((E\vartriangle F_j) \vartriangle
          (E_{n_j}\vartriangle E_n)\bigr) \\
&\leq m(E\vartriangle F_j) + m(E_{n_j}\vartriangle E_n) 
< 2^{1-j}+2^{-j}.
\end{align*}

Thus completeness of the metric space $S$ of measurable sets 
modulu sets of zero-measure with the metric \(\rho\) was shown.

The continuity of \(E\mapsto \int_E f_n\,d\mu\) is immediate
by \(f_n\in L^1(\mu)\).
By Exercise~5.13(b) there exists $N$ and  an open set 
\(V = \{E\in S: \rho(E_0,E) < \delta)\}\)
of \(E_0\) such that \eqref{eq:ex:6.10} holds for any \(E\in V\) and \(n>N\).
If \(\mu(A)<\delta\) then \(B = E_0 \setminus A\in V\)
and \(C = E_0 \cup A\in V\).
Thus
\begin{equation*}
\left|\int_A f_n\right| 
= \left|\int_{E_0\setminus A} f_n + \int_{E_0\cup A} f_n\right| 
\leq \left|\int_{E_0\setminus A} f_n\right| + \left|\int_{E_0\cup A} f_n\right| 
< 2\epsilon.
\end{equation*}

We pick some \(0<\delta'<\delta\) such that
\(|\int_E f_j\,d\mu| < \epsilon\) whenever \(1\leq j \leq N\).
Now if \(\mu(A)<\delta'\) then
for any \(n>N\) we have
\begin{equation*}
\left|\int_A f_n\,d\mu\right|
\leq \left|\int_A f_N\,d\mu\right| + \left|\int_A (f_n - f_N)\,d\mu\right|
\leq \epsilon + 2\epsilon < 3\epsilon
\end{equation*}
If \(n\leq N\) then \(|\int_A f_n\,d\mu| < \epsilon < 3\epsilon\) trivially.
Thus
\begin{equation*}
\left|\int_A f_n\,d\mu\right| < 3\epsilon
\end{equation*}
for any \(n\in\N\) and $A$ such that \(\mu(A) < \delta'\).
Thus \(\{f_n\}_{n\in\N}\) are uniformly integrable, and by Vitali's lemma
the desired result follows.
\end{itemize}


%%%%%%%%%%%%%%
\begin{excopy}
Suppose \(\mu\) is a positive measure on $X$, \(\mu(X)<\infty\),
\(f_n \in L^1(\mu)\) for \(n=1,2,3,\ldots\),
\(f_n(x)\to f(x)\;\aded\),
and there exists \(p>1\) and \(C<\infty\) such that
\(\int_X |f_n|^p\,d\mu<C\) for all $n$. Prove that
\begin{equation*}
\lim_{n\to\infty} \int_X |f - f_n|\,d\mu = 0.
\end{equation*}
\emph{Hint}: \(\{f_n\}\) is uniformly integrable.
\end{excopy}

By negation there exists \(\epsilon>0\) and 
a sequences \(\{\delta_n\}_{n\in\N}\) and sets \(\{A_n\}_{n\in\N}\)
such that 
\begin{align*}
\mu(A_n) &< \delta_n = \epsilon/n \\
\left|\int_{A_n} f_n\,d\mu \right| &\geq \epsilon.
\end{align*}

Let $q$ be the exponent conjugate of $q$.
By H\"older inequality
\begin{equation*}
\left|\int_{A_n} f_n\,d\mu \right|
\leq \int_{A_n} |f_n|\,d\mu 
\leq 
\iffalse
     \left(\int_{A_n} |f_n|^p\,d\mu\right)^{1/p}
     \left(\int_{A_n} 1^q\,d\mu\right)^{1/q}
= 
\fi
\left(\int_{A_n} |f_n|^p\,d\mu\right)^{1/p} \bigl(\mu(A_n)\bigr)^{1/q}
\end{equation*}
Hence
\begin{equation*}
\int_{A_n} |f_n|^p\,d\mu
\geq \left( 
      \left|\int_{A_n} f_n\,d\mu \right| \bigm/ \bigl(\mu(A_n)\bigr)^{1/q}
     \right)^p
\geq \bigl( \epsilon / (\epsilon/n)^{1/q} \bigr)^p 
= (n\epsilon^{1-1/q})^p
= n^p\epsilon.
\end{equation*}
Which gives the contradiction
\begin{equation*}
\int_X |f_n|^p\,d\mu \geq \int_{A_n} |f_n|^p\,d\mu \geq n^p\epsilon > C
\end{equation*}
For sufficiently large $n$.


%%%%%%%%%%%%%%
\begin{excopy}
Let \frakM\ be the collection of all sets $E$ in the unit interval \([0,1]\) such that either $E$ or its complement is at most countable.
Let \(\mu\) be the counting measure on this \(\sigma\)-algebra \frakM.
If \(g(x) = x\) for \(0\leq x \leq 1\),
show that $g$ is not \frakM-measurable, although the mapping
\begin{equation*}
f \to \sum xf(x) = \int fg\,d\mu
\end{equation*}
makes sense for every \(f\in L^1(\mu)\) and defines a bounded linear functional
on \(L^1(\mu)\). Thus \((L^1)^* \neq L^\infty\) in this situation.
\end{excopy}

The set \(g^{-1}([0,1/2]) = [0,1/2]\) is not measurable, hence $g$ is not.
Indeed this measurable space is not \(\sigma\)-finite.

%%%%%%%%%%%%%%
\begin{excopy}
Let \(L^\infty = L^\infty(m) \), where $m$ is a Lebesgue measure on
\(I=[0,1]\). Show that there is a bounded linear functional \(\Lambda \neq 0\)
on \(L^\infty\) that is $0$ on \(C(I)\), and therefore there is no
\(g\in L^1(m)\) that satisfies
\(\Lambda f = \int_I fg\,dm\) for every \(f\in L^\infty\).
Thus \((L^\infty)^* \neq L^1\).
\end{excopy}

\(C(I)\) is a closed subspace of \((L^\infty)\).
By Hahn Banach Theorem
\index{Hahn Banach}
Theorem~5.16 \cite{RudinRCA87} 
and its consequence Theorem~5.19 there is a functional 
\(\Lambda\in (L^\infty)^*\) such that 
\(\Lambda f = 0\) for all \(f\in C(I)\) but 
\(\Lambda(\chhi_{[0,1/2]}) \neq 0\). Viewing \(C(I)\subset L^2\)
as a subspace of Hilbert space, if \(\Lambda\) may be represented
by inner multiplication by the zero function, but it surely cannot
represent \(\Lambda\) for the \(\chhi{[0,1/2]}\) case.


%%%%%%%%%%%%%%%%%
\end{enumerate}
