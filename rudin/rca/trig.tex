% $Id: trig.tex,v 1.6 2008/07/19 08:56:55 yotam Exp $

%%%%%%%%%%%%%%%%%%%%%%%%%%%%%%%%%%%%%%%%%%%%%%%%%%%%%%%%%%%%%%%%%%%%%%%%
%%%%%%%%%%%%%%%%%%%%%%%%%%%%%%%%%%%%%%%%%%%%%%%%%%%%%%%%%%%%%%%%%%%%%%%%
%%%%%%%%%%%%%%%%%%%%%%%%%%%%%%%%%%%%%%%%%%%%%%%%%%%%%%%%%%%%%%%%%%%%%%%%
\section{Trigonometry}

We will need workout several results dealing with 
eqalities and inequalities of complex numbers.
Argumentation wil be based both on the cartesian and polar representaions.

%%%%%%%%%%%%%%%%%%%%%%%%%%%%%%%%%%%%%%%%%%%%%%%%%%%%%%%%%%%%%%%%%%%%%%%%
\subsection{Angle Argument}

For each non zero \(z\in\C\) there is unique polar representaion
\(z=re^{i\theta}\) where \(r=|z|\) and \(\theta\in[0,2\pi) \subset \R\).
We define the \emph{argument}
\begin{equation} \label{eq:arg}
\Arg(z) = \Arg(|z|e^{i\theta}) \eqdef \theta \in [0,2\pi).
\end{equation}


%%%%%%%%%%%%%%%%%%%%%%%%%%%%%%%%%%%%%%%%%%%%%%%%%%%%%%%%%%%%%%%%%%%%%%%%
\subsection{Cosines and Sines}

In addition to the Prologue of \cite{RudinRCA80}, we want to establish
the following trigonometric equalities.

%%%%%%%%%%%%%%%%%%%%%%%%%%%%%%%%
\begin{llem} \label{llem:trig:cos:sin}
\begin{alignat}{2}
 \cos(\pi/3)  & =  1/2   &\qquad \sin(\pi/3) & = \sqrt{3}/2
    \label{eq:cosin:pi3} \\
 \cos(2\pi/3) & = -1/2   &\qquad \sin(\pi/3) & = \sqrt{3}/2
    \label{eq:cosin:2pi3}
\end{alignat}
\end{llem}
\begin{thmproof}
Let \(a = c+si = e^{\pi i/3}\), with \(c,s\in\R\).
From \((c+si)^3 = e^{\pi i} = -1\)
we get the following equations
\begin{eqnarray*}
|z|^2 =  c^2 + s^2   &=& 1 \\
\Re(z) = c^3 - 3cs^2 &=& -1 \\
\Im(z) = 3c^2s - s^3  &=& 0
\end{eqnarray*}
By last equation, 
\(c=0\) iff \(s=0\). But if this happens it contradicts the first equation.
Thus \(c\neq 0 \neq s\).
We divide the last equation by $s$ and add the first.
We get \(4c^2 = 1\) and so \(s^2= 3/4\) and \(c=\pm1/2\).
But if \(c = -1/2\) then by the section equation
\(-1/8 - 3\cdot(-1/2)\cdot(3/4) = 1 \neq -1\) contradicting the second equation.
Hence \(c=1/2\) and the cosine equality of \eqref{eq:cosin:pi3} is true.


Now let \(c+si = e^{\pi 2i/3}\), with \(c,s\in\R\). 
From \((c+si)^3 = e^{2\pi i} =-1\)
we get the following equations
\begin{eqnarray*}
c^2 + s^2   &=& 1 \\
c^3 - 3cs^2 &=& 1 \\
3c^s - s^3  &=& 0
\end{eqnarray*}
By similar arguments we again get \(c=\pm1/2\), but this time
the second equations elimiates the \(c=1/2\) possibility and
so the cosine equality of \eqref{eq:cosin:2pi3} is true.

By the the \(\cos^2(x) + \sin^2(z) = 1\) equality
\begin{equation*}
 \sin(\pi/3) = \sin(2\pi/3) = \pm\sqrt{3}/2.
\end{equation*}
But we know that \(\sin(0) = \sin(\pi) = 0\)
and \(\sin(\pi/2) = 1\). Also \(sin(t)\) is increasing 
in \([0,\pi/2]\) and decreasing in \([\pi/2,\pi]\)
and thus the desired equalities for sine are forced.
\end{thmproof}


%%%%%%%%%%%%%%%%%%%%%%%%%%%%%%%%%%%%%%%%%%%%%%%%%%%%%%%%%%%%%%%%%%%%%%%%
\subsubsection{Double angle}

\begin{llem}
The following identities hold.
\begin{equation*}
\cos^2(z) = (\cos(2z) + 1)/2
\end{equation*}
\end{llem}
\begin{thmproof}
\begin{equation*}
\cos^2(z) 
= \left((e^{iz} + e^{-iz})/2\right)^2
= (e^{2iz} + + 2\cdot 0 + e^{-2iz})/4
= \cos(2z)/2
\end{equation*}
\end{thmproof}




%%%%%%%%%%%%%%%%%%%%%%%%%%%%%%%%%%%%%%%%%%%%%%%%%%%%%%%%%%%%%%%%%%%%%%%%
\subsection{Tangent}

The \emph{tangent} function is defined for every complex number \(z\in\C\)
such that \(\cos(z) \neq 0\) by
\begin{equation} \label{eq:tan}
\tan(z) \eqdef \frac{\sin(z)}{\cos(z)}.
\end{equation}

Viewed as a real function, \(\tan: \R\to\R\) is increasing where it's defined.
%%%%%%%%%%%%%%%%%%%%%%%%%%%%%%%%
\begin{llem} \label{llem:tan}
Suppose \(\alpha,\beta \in (n\pi/2, (n+1)\pi/2)\) where \(n=0,1,2,3\).
If \(\alpha < \beta\) then \(\tan(\alpha) < \tan(\beta)\).
\end{llem}
\begin{thmproof}
We simply verify each \((n\pi/2, (n+1)\pi/2)\) quadrant case. 
\begin{itemize}
\item[\(n=0\)] 
  Both \(\sin(t)\) and \(\cos(t)\) are positive, 
  \(\sin(t)\) increases and \(\cos(t)\) decreases. 
\item[\(n=1\)] Consider \(-\tan(t) = \sin(t)/(-\cos(t))\).
  Both \(\sin(t)\) and \(-\cos(t)\) are positive, 
  \(\sin(t)\) decreases and \(-\cos(t)\) increases, 
  thus \(-\tan(t)\) decreases.
\item[\(n=2\)] Consider \(\tan(t) = (-\sin(t))/(-\cos(t))\).
  Both \(-\sin(t)\) and \(-\cos(t)\) are positive, 
  \(-\sin(t)\) increases and \(-\cos(t)\) decreases.
\item[\(n=3\)] % Similar arguments as the \(n=1\) case.
  Consider \(-\tan(t) = (-\sin(t))/\cos(t)\).
  Both \(-\sin(t)\) and \(\cos(t)\) are positive, 
  \(-\sin(t)\) decreases and \(\cos(t)\) increases,
  thus \(-\tan(t)\) decreases.
\end{itemize}
Clearly \(\tan(t)\) increases in all the above cases.
\end{thmproof}


%%%%%%%%%%%%%%%%%%%%%%%%%%%%%%%%%%%%%%%%%%%%%%%%%%%%%%%%%%%%%%%%%%%%%%%%
%%%%%%%%%%%%%%%%%%%%%%%%%%%%%%%%%%%%%%%%%%%%%%%%%%%%%%%%%%%%%%%%%%%%%%%%
\section{Sums}

\paragraph{Definition.} 
Given non zero complex numbers \(z_1\) and \(z_2\), 
we define the \emph{angle} \(\Ang(z_1,z_2)\)
to be the unique \(\theta \in [0,\pi]\) such that
\begin{equation*}
\frac{z_1/ z_2}{|z_1/ z_2|} = \pm e^{i\theta}.
\end{equation*}

We start by showing a case where the absolute value must grow.

%%%%%%%%%%%%%%%%%%%%%%%%%%%%%%%%
\begin{llem} \label{llem:z1z2:grow}
Let non zero \(z_1,z_2\in \C\) satisfy \(\Ang(z_1,z_2) \leq \pi/2\)
then \(|z_1 + z_2| > \max(|z_1|,|z_2|)\). 
\end{llem}
\begin{thmproof}
By symmetry, suffice to show \(|z_1 + z_2| > |z_1|\).
Let \(z_1 = re^{i\theta}\). If we multiply both \(z_1\) and \(z_2\)
by \(e^{-i\theta}\) both the assumptions and desired inequalities remain.
Thus we can assume that \(0 < z_1 \in\R\).
With the representaion, \(z_k = x_k + iy_k\) for \(k=1,2\),
we have \(y_1 = 0\)
and since the \(\Ang(z_1,z_2) \leq \pi/2\), 
we also have \(x_1,x_2 \geq 0\).
Thus
\begin{equation*}
|z_1 + z_2|^2 = (x_1 + x_2)^2 + y_2^2 \geq x_1^2 + y_2^2 = |z_1|^2
\end{equation*}
and the desired inequality follows.
\end{thmproof}

Now we see a case where the absolute value cannot grow.
%%%%%%%%%%%%%%%%%%%%%%%%%%%%%%%%
\begin{llem} \label{llem:z12:2pi3}
Let non zero \(z_k = r_k e^{i\theta_k}\) for \(k=1,2\).
If \(\Ang(z_1,z_2) \geq 2\pi/3\) then
\begin{equation*}
|z_1 + z_2| \leq \min(|z_1|,|z_2|).
\end{equation*}
\end{llem}
\begin{thmproof}
By symmetry and by multiplying both \(z_1\) and \(z_2\) by 
\(e^{-i\theta_1}/\max(|z_1|,|z_2|)\)
we may assume 
\begin{gather*}
|z_1| \leq 1 \qquad |z_2| < 1 \\
0=\theta_1 \leq 2\pi/3 \leq \theta_2 \leq \pi
\end{gather*}
and we need to show that \(|z_1 + z_2| \leq 1\).
We look at the representaions \(z_1=x_1\) and \(z_2 = x_2 + iy_2\)
where \(x_1,x_2,y_2\in\R\).
Since \(\cos(t)\) and \(\sin(t)\) decrease on \(t\in[\pi/2,\pi]\)
we have 
\(-1 \leq x_2 \leq 0 < x_1 \leq 1\).
and
\(0\leq y_2 \leq -(\sqrt{3}/2)x_2\).
There are two cases.

\textbf{Case 1.} Assume \(x_2\geq -x_1/2\).
\begin{equation*}
|z_1 + z_2|^2
 = (x_1 + x_2)^2 + y_2^2 
 \leq (x_1 + x_2)^2 + (3/4)x_2^2 
 = (7/4)x_2^2 + 2x_1x_2 + x_1^2
\end{equation*}
Looking at the last expression as a function of \(x_2\in\R\),
it attains its minimum at \(x_2 = -2x_1\)
Hence the maximum for this case, is achieved at \(x_2 = 0\)
hence \(|z_1 + z_2|^2 \leq x_1^2 \leq 1\).

\textbf{Case 2.} Assume \(x_2\leq -x_1/2\).
Hence \(|x_1 + x_2| \leq |x_2|\) and thus
\begin{equation*}
|z_1 + z_2|^2
 = (x_1 + x_2)^2 + y_2^2 
 \leq x_2^2 + y_2^2 \leq 1
\end{equation*}

Thus in both cases \(|z_1 + z_2| \leq 1\).
\end{thmproof}

Our next lemma shows that for any small total sum of small numbers,
we can find a small pair.
%%%%%%%%%%%%%%%%%%%%%%%%%%%%%%%%
\begin{llem} \label{llem:zpairlt1}
Suppose  \(n\geq 2\) and \(z_k\in\C\) for \(k\in\N_n\).
If \(|z_k| \leq 1\) for each \(k\in \N_n\) and 
\(|\sum_{k=1}^n z_k| \leq 1\)
then there exists a pair \(j,k\in\N_n\) such that \(j\neq k\)
and 
\begin{equation} \label{eq:zpairlt1}
|z_j + z_k| \leq 1.
\end{equation}
\end{llem}
\begin{thmproof}
By induction on $n$. For \(n=1\) there is nothing to show.
For \(n=2\), we simply take \(j=1\) and \(k=2\).
Now assume the lemma is true for all \(n<n'\) for some \(n'>2\).
By negation, assume that for any \(j,k\in\N_n\) pair 
\begin{equation} \label{eq:zpairlt1:zjzk:gt1}
|z_j + z_k| > 1.
\end{equation}

We may also assume that 
\begin{equation} \label{eq:zpairlt1:Azjzk}
\Ang(z_j,z_k) < 2\pi/3
\end{equation}
for any two indices \(j,k\in\N_{n'}\), since otherwise,
\loclemma~\ref{llem:z12:2pi3} contradicts \eqref{eq:zpairlt1:zjzk:gt1}.

We want to have \(\mathbf{z} = (z_k)_{k=1}^{n'}\) be such that 
their arguments will be increasing in \([0,2\pi/3)\).
\Wlogy\ we may assume 
\(\mathbf{z}\) is sorted so \(\theta_k \leq \theta_{k+1}\)
for each \(k\in\N_{n'-1}\). 
We define the ``preceding'' angles
\(\alpha_1 = \theta_1 + 2\pi - \theta_{n'}\) 
and
\(\alpha_{k+1} = \theta_{k+1} - \theta_k\) for \(k\in\N_{n'-1}\).
Let \(K\in\N_{n'}\) be such that \(\alpha_K = \max\{\alpha_k:k\in N_{n'}\}\).
\Wlogy, we may assume \(K=1\) and \(\theta_1 = 0\). Otherwise, 
we can multiply \(\mathbf{z}\) by \(e^{-i\theta_K}\) and resort by
their (new) arguments.
To show now that the arguments of \(\mathbf{z}\) are in \([0,2\pi/3)\),
assume by negation that there is some \(k'\in\N_{n'}\) such that
\(\theta_k \in [2\pi/3,2\pi)\).
\newline
\textbf{Case \ich{i}}.
If \(\theta_{k'}\in [2\pi/3,4\pi/3]\) then \(\Arg(z_1,z_{k'})\geq 2\pi/3\)
contradiction to \eqref{eq:zpairlt1:Azjzk}.
\newline
\textbf{Case \ich{ii}}.
Let \(k''\) be the minimal such that
\(\theta_{k''}\in (4\pi/3,2\pi)\) then put 
\(\beta = \theta_{k''} - 4\pi/3 \in (0,2\pi/3)\) and 
again because of \eqref{eq:zpairlt1:Azjzk} we deduce that
\(\theta_k \notin (\alpha, \alpha + 2\pi/3)\) 
for \(k\in\N_{n'}\).
Combining with Case~\ich{i} deduction
\(\theta_k \notin (\alpha, \alpha + 4\pi/3)\)
for \(k\in\N_{n'}\). 
Hence \(\alpha_{k''} > 4\pi/3\), but
\begin{equation*}
\alpha_1 \leq \Ang(z_1,z_{k''}) = 2\pi - \theta_{k''} 
 \leq 2\pi/3 - \alpha < 2\pi/3
\end{equation*}
which contradict the assumption of 
\(K=1\), that is \(\alpha_1\) being the largest preceding angle.

We look at the ``middle sum''
\begin{equation*}
U \eqdef \sum_{k=2}^{n'-1} z_k = |U|e^{i\alpha}.
\end{equation*}
We rotate all of \(\mathbf{z}\) by \(e^{-i\alpha}\), this time allowing
for negative angle arguments. For convenience we rename back to \(\mathbf{z}\).
Now we have the following new situation.
% \begin{gather}
% \begin{alignedat}{3}
\begin{alignat*}{3}
z_k &= x_k + iy_k = |z_k|e^{i\theta_k} 
  &\qquad
  |z_k| &\leq 1 
  & \qquad (k\in\N_{n'})  \\
S &\eqdef \sum_{k=1}^{n'} z_k 
  &\qquad
|S| &\leq 1 
  &&  \notag
\end{alignat*} % \\[1pt]
\begin{gather}
 R \eqdef \sum_{k=2}^{n'-1} z_k = S - z_1 - z_{n'} \in \R. \notag \\
 \theta_k < \theta_{k+1} \qquad (k\in\N_{n'-1}) \notag \\
 \theta_1 \leq 0 \leq \theta_{n'} < \theta_1 + 2\pi/3 
    \label{eq:zpairlt1:thetan1}
\end{gather}
% \end{gather}


We may also assume \wlogy\ that 
\begin{equation}  \label{eq:zpairlt1:theta1n}
-\theta_1 < \theta_{n'}. 
\end{equation}
Otherwise we can substitue all 
\((z_k)_{k=1}^{n'}\)
with 
\((\overline{z_k})_{k=1}^{n'}\).

% Given the cartesian representation \(z_k = x_k + iy_k\) for \(k\in\N_{n'}\),
We look at the signs of \(x_1\) and \(x_{n'}\).
Clearly
\begin{alignat*}{3}
 x_1 & \leq 0     &\qquad &\textrm{iff}\quad  &\theta_1    & \leq -\pi/2 \\
 x_{n'} & \leq 0  &\qquad &\textrm{iff}\quad  &\theta_{n'} & \geq  \pi/2 
\end{alignat*}
We examine the following cases. 
\paragraph{Case 1.} Assume \(x_1 + x_{n'} \geq 0\)
(covering the case of \(x_1\geq 0\) and \(x_{n'}\geq 0\)).
Compute
\begin{eqnarray*}
|S|^2 - |z_1 + z_{n'}|^2 
&=& |(U + z_1 + z_{n'})|^2 - |z_1 + z_{n'}|^2 \\
&=& |U + (x_1 + x_{n'}) + i(y_1 + y_{n'})|^2 - 
    |(x_1 + x_{n'}) + i(y_1 + y_{n'})|^2 \\
&=& U^2 + 2U(x_1 + x_{n'})
\end{eqnarray*}
Since \(x_1 + x_{n'} \geq 0\), by the above equality
\(|S|^2 \geq |z_1 + z_{n'}|^2 > 1\) contradiction to the lemma's assumption.
\paragraph{Case 2.} Assume \(x_1\leq 0\) and \(x_{n'}\leq 0\).
Then
\(\theta_1 \leq -\pi/2\) and \(\theta_{n'} \geq \pi/2\)
contradiction to \eqref{eq:zpairlt1:thetan1}.
\paragraph{Case 3.} Assume \(x_1\leq 0\) and \(x_{n'}\geq 0\).
This contradicts the ``conjugate choice'' 
assumption of \eqref{eq:zpairlt1:theta1n}
\paragraph{Case 4.} Assume \(x_1\geq 0\), and \(x_{n'}\leq 0\),
and \(x_1 + x_{n'} < 0\).
Then
\(1/x_{n'} > -1/x_1\), and so
\begin{equation*}
\tan(z_{n'}) =  y_{n'}/x_{n'} > -y_{n'}/x_1 > y_1 / x_1 = \tan(z_1) = \tan(-z_1).
\end{equation*}
which is a contadiction, since 
\begin{equation*}
\pi/2 < \theta_{n'} < \theta_1 + \pi < \pi
\end{equation*}
and applying \loclemma~\ref{llem:tan} gives \(y_{n'}/x_{n'} \leq y_1/x_1\).

Since all cases were refuted a pair \(j,k\in\N_n\)
satisfying \eqref{eq:zpairlt1} must exist.
\end{thmproof}



%The following lemma shows that in certain conditions 
%we can order small numbers so their partial sums is stays small.
Intuitively, we now show that for numbers within the unit circle,
if their sum is inside, then they can be ordered so partial sums are inside.
%%%%%%%%%%%%%%%%%%%%%%%%%%%%%%%%
\begin{llem} \label{llem:zsubsum}
Suppose \(z_k\in\C\) for \(k\in\N_n\). % and \(S\eqdef \sum_{k=1}^n z_k\).
If \(|z_k| \leq 1\) for each \(k\in \N_n\) and 
\begin{equation} \label{eq:zsubsum:assert}
\left|\sum_{k=1}^n z_k\right| \leq 1
\end{equation}
then \((z_k)\) can be reordered such that \(|S_m| \leq 1\), where
\begin{equation*}
S_m = \sum_{k=1}^m z_k
\end{equation*}
for each \(m\in\N_n\).
\end{llem}

\paragraph{Notes.} 
\begin{itemize}
\item There could be repetitions of values in \(z_k\).
\item
We use the term ``can be reordered''.
It would be more rigorous to say that
there exist a permutation \(\sigma\) of \((1,\ldots,n)\) 
and sum \(z_{\sigma(k)}\). But we prefer to simplify notations.
\end{itemize}

\begin{thmproof}
By the lemma above there exist  \(j,k\in\N_{n'}\) such that \(j\neq k\) and
\begin{equation} \label{eq:zsubsum:zjzk}
 |z_j + z_k| \leq 1 
\end{equation}
We can look at a sequence
\(\tilde{\mathbf{z}}\) 
of \(n'-1\) numbers
made by the original \(\mathbf{z}=(z_k)_{k=1}^{n'}\) 
with \(z_j\) and \(z_k\) dropped
and \(z_j + z_k\) added. Applying the induction hypothesis on
\(\tilde{\mathbf{z}}\) we get an~\((n'-1)\)-ordering
such that the analogue of \eqref{eq:zsubsum:assert} holds.
To get reordering for \(\mathbf{z}\)
we look at the following cases:
\begin{itemize}
 \itemch{a} If \(z_j+z_k\) is first in the \((n'-1)\)-ordering,
 then we simply split it back
 \itemch{b} If \(z_j+z_k\) is last in the \((n'-1)\)-ordering,
 then we simply split it back
 \itemch{c} Assume \(z_j+z_k\) is in the $p$ place of the \((n'-1)\)-ordering.
 Since \(p<n'-1\) we can apply a sub-\((p+1)\)-ordering,
 thus changing the the initial $p$ places with new \(p+1\).
 The final \(n'-p-1\) places are just shifted (and remain the tail).
\end{itemize}
Thus, for each possible case, we have shown how a desired reordering
canbe constructed.
\end{thmproof}

Here is another view of the above.
%%%%%%%%%%%%%%%%%%%%%%%%%%%%%%%%
\begin{llem} \label{llem:zsub:block}
Suppose \(z_k\in\C\) for \(k\in\N_n\). % and \(S\eqdef \sum_{k=1}^n z_k\).
and \(|z_k| \leq 1\) for each \(k\in \N_n\).
If there exists \(m\in\N_n\) such that each subset \(A\subset\N_n\)
with $m$ elements (\(|A|=m\)), satisfy
\begin{equation*}
\left|\sum_{k\in A} z_k\right| > 1
\end{equation*}
then 
\begin{equation*}
\left|\sum_{k=1}^n z_k\right| > 1.
\end{equation*}
\end{llem}
\begin{thmproof}
By negation, applying the above \loclemma~\ref{llem:zsubsum} gives
a contradiction.
\end{thmproof}
