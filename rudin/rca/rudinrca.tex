% -*- latex -*-

\usepackage{amsmath}
\usepackage{amssymb}
% \usepackage{eucal}
\usepackage{mathrsfs}

% \usepackage{fullpage}

\usepackage{geometry}
\geometry{a4paper, left=2cm, right=2cm, top=2cm, bottom=2cm, includeheadfoot}

\setlength{\parindent}{0pt}
\setlength{\parskip}{6pt}


% are we in pdftex ????
\ifx\pdfoutput\undefined % We're not running pdftex
\else
\RequirePackage[colorlinks,hyperindex,plainpages=false]{hyperref}
\def\pdfBorderAttrs{/Border [0 0 0] } % No border arround Links
\fi

% \usepackage{fancyheadings}
\usepackage{fancyhdr}
\usepackage{pifont}

\pagestyle{fancy}
% \addtolength{\headwidth}{\marginparsep}
% \addtolength{\headwidth}{\marginparwidth}
%  \addtolength{\textheight}{2pt}

\newcommand{\ineqjton}{\overset{1\leq i,j \leq n}{i \neq j}}
\newcommand{\srightmark}{\rightmark}
\newcommand{\sfbfpg}{\sffamily\bfseries{\thepage}}
  \newcommand{\symenvelop}{%
     {\nullfont\ }\relax\lower.2ex\hbox{\large\Pisymbol{pzd}{41}}}
% \renewcommand{\chaptermark}[1]{\markboth{\thechapter.\ #1}}

\iffalse
% \lhead[\fancyplain{}{{\sfbfpg}}]{\fancyplain{}\bfseries\srightmark}
\lhead[\fancyplain{}{{\sfbfpg}}]{\fancyplain{}\sl\srightmark}
% \rhead[\fancyplain{}\bfseries\leftmark]{\fancyplain{}{{\sfbfpg}}}
\rhead[\fancyplain{}\sl\leftmark]{\fancyplain{}{{\sfbfpg}}}
\lfoot{\today}
\cfoot{Yotam Medini \copyright}
  \newcommand{\symenvelop}{%
     {\nullfont a}\relax\lower.2ex\hbox{\large\Pisymbol{pzd}{41}}}
\rfoot{\symenvelop\ \texttt{yotam.medini@gmail.com}}

\renewcommand{\headrulewidth}{0.4pt}
\renewcommand{\footrulewidth}{0.4pt}
\fi

\setlength{\headheight}{16pt}
\fancyplain{plain}{%
 \fancyhf{}
 \fancyhead[LE,RO]{\fancyplain{}{{\sfbfpg}}}
 \fancyhead[RE,LO]{\sl\leftmark}
 \fancyfoot[L]{\today}
 \fancyfoot[C]{Yotam Medini \copyright}
 \fancyfoot[R]{\symenvelop\ \texttt{yotam.medini@gmail.com}}
 \renewcommand{\headrulewidth}{0.4pt}
 \renewcommand{\footrulewidth}{0.4pt}
}

% \usepackage{amstex}
% \usepackage{amsmath}
% \usepackage{amssymb}
\usepackage{amsthm}
\usepackage{bm}
\usepackage{makeidx}
\makeindex % enable

% 'Inspired' by:
%% This is file `uwamaths.sty',
%%%     author   = "Greg Gamble",
%%%     email     = "gregg@csee.uq.edu.au (Internet)",

\makeatletter
\def\DOTSB{\relax}
\def\dotcup{\DOTSB\mathop{\overset{\textstyle.}\cup}}
 \def\@avr#1{\vrule height #1ex width 0pt}
 \def\@dotbigcupD{\smash\bigcup\@avr{2.1}}
 \def\@dotbigcupT{\smash\bigcup\@avr{1.5}}
 \def\dotbigcupD{\DOTSB\mathop{\overset{\textstyle.}\@dotbigcupD%
                               \vphantom{\bigcup}}}

\def\dotbigcupT{\DOTSB\smash{\mathop{\overset{\textstyle.}\@dotbigcupT%
                              \vphantom{\bigcup}}}%
                       \vphantom{\bigcup}\@avr{2.0}}
\def\dotbigcup{\mathop{\mathchoice{\dotbigcupD}{\dotbigcupT}
                                  {\dotbigcupT}{\dotbigcupT}}}
\let\disjunion\dotcup
\let\Disjunion\dotbigcup
\makeatother


\newcommand{\half}{\ensuremath{\frac{1}{2}}}



\newcommand{\C}{\ensuremath{\mathbb{C}}} % The Complex set
\newcommand{\aded}{\ensuremath{\textrm{a.e.}}} % almost everyehere
\newcommand{\chhi}{\raise2pt\hbox{\ensuremath\chi}}           %raise the chi
\newcommand{\calA}{\ensuremath{\mathcal{A}}}
\newcommand{\calB}{\ensuremath{\mathcal{B}}}
\newcommand{\calE}{\ensuremath{\mathcal{E}}}
\newcommand{\calF}{\ensuremath{\mathcal{F}}}
\newcommand{\calG}{\ensuremath{\mathcal{G}}}
\newcommand{\calM}{\ensuremath{\mathcal{M}}}
\newcommand{\calR}{\ensuremath{\mathcal{R}}}
\newcommand{\eqdef}{\ensuremath{\stackrel{\mbox{\upshape\tiny def}}{=}}}
\newcommand{\frakB}{\ensuremath{\mathfrak{B}}}
\newcommand{\frakC}{\ensuremath{\mathfrak{C}}}
\newcommand{\frakF}{\ensuremath{\mathfrak{F}}}
\newcommand{\frakG}{\ensuremath{\mathfrak{G}}}
\newcommand{\frakI}{\ensuremath{\mathfrak{I}}}
\newcommand{\frakM}{\ensuremath{\mathfrak{M}}}
\newcommand{\scrA}{\ensuremath{\mathscr{A}}}
\newcommand{\scrB}{\ensuremath{\mathscr{B}}}
\newcommand{\scrD}{\ensuremath{\mathscr{D}}}
\newcommand{\scrF}{\ensuremath{\mathscr{F}}}
\newcommand{\scrN}{\ensuremath{\mathscr{N}}}
\newcommand{\scrP}{\ensuremath{\mathscr{P}}}
\newcommand{\scrQ}{\ensuremath{\mathscr{Q}}}
\newcommand{\scrR}{\ensuremath{\mathscr{R}}}
\newcommand{\scrT}{\ensuremath{\mathscr{T}}}
\newcommand{\Lp}[1]{\ensuremath{\mathbf{L}^{#1}}} % Lp space
\newcommand{\N}{\ensuremath{\mathbb{N}}} % The Natural Set
\newcommand{\bbP}{\ensuremath{\mathbb{P}}} % Some partially ordered set
\newcommand{\Q}{\ensuremath{\mathbb{Q}}} % The Rational set
\newcommand{\R}{\ensuremath{\mathbb{R}}} % The Real Set
\newcommand{\T}{\ensuremath{\mathbb{T}}} % The Thorus [-pi,\pi)
\newcommand{\Z}{\ensuremath{\mathbb{Z}}} % The Integer Set
\newcommand{\intR}{\int_{-\infty}^{\infty}} % Integral over the reals
\newcommand{\posthat}[1]{#1{\,\hat{}\,}}

% sequences
\newcommand{\seq}[2]{\ensuremath{#1_1,\ldots,#1_{#2}}}
\newcommand{\seqn}[1]{\seq{#1}{n}}
\newcommand{\seqan}{\seq{a}{n}}
\newcommand{\seqxn}{\seq{x}{n}}
\newcommand{\seqalphn}{\seq{\alpha}{n}}

\newcommand{\mset}[1]{\ensuremath{\{#1\}}}


%%%%%%%%%%%%
%% math op's
\newcommand{\Alt}{\mathop{\rm Alt}\nolimits}
\newcommand{\Ang}{\mathop{\rm Ang}\nolimits}
\newcommand{\Arg}{\mathop{\rm Arg}\nolimits}
\newcommand{\co}{\mathop{\rm co}\nolimits}
\newcommand{\conv}{\mathop{\rm conv}\nolimits}
\newcommand{\diam}{\mathop{\rm diam}\nolimits}
\newcommand{\dom}{\mathop{\rm dom}\nolimits}
% \newcommand{\dim}{\mathop{\rm dim}\nolimits}
% \newcommand{\esssup}{\mathop{\rm ess\ sup}\nolimits}
\DeclareMathOperator*{\esssup}{ess\,sup}
\newcommand{\ext}{\mathop{\rm ext}\nolimits}
\newcommand{\Id}{\mathop{\rm Id}\nolimits}
\newcommand{\Image}{\mathop{\rm Im}\nolimits}
\newcommand{\Ind}{\mathop{\rm Ind}\nolimits}
\newcommand{\Lip}{\mathop{\rm Lip}\nolimits}
\newcommand{\lip}{\mathop{\rm lip}\nolimits}
\newcommand{\percB}{
  \mathbin{\ooalign{$\hidewidth\%\hidewidth$\cr$\phantom{+}$}}}
\newcommand{\bres}[2]{\ensuremath{#1 \percB #2}}

\newcommand{\Ker}{\mathop{\rm Ker}\nolimits}
\newcommand{\rank}{\mathop{\rm rank}\nolimits}
\newcommand{\rng}{\mathop{\rm rng}\nolimits}
\newcommand{\Res}{\mathop{\rm Res}\nolimits}
\newcommand{\supp}{\mathop{\rm supp}\nolimits}
\newcommand{\vol}{\mathop{\rm vol}\nolimits}
\newcommand{\vspan}{\mathop{\rm span}\nolimits}

% I wish this was more standardized
\renewcommand{\Re}{\mathop{\bf Re}\nolimits}
\renewcommand{\Im}{\mathop{\bf Im}\nolimits}

\newcommand{\inter}[1]{\ensuremath{#1^{\circ}}}  % interior
\newcommand{\closure}[1]{\ensuremath{\overline{#1}}} % closure
\newcommand{\boundary}[1]{\ensuremath{\partial #1}} % closure


\newcommand{\ich}[1]{(\textit{#1})}
\newcommand{\itemch}[1]{\item[\ich{#1}]}
\newcommand{\itemdim}{\item[\(\diamond\)]}

% Special names
\newcommand{\Cech}{\u{C}ech}

\author{Yotam Medini}


%%%%%%%%%%%
%% Theorems
%%
\makeatletter
\@ifclassloaded{book}{
 \newtheorem{thm}{Theorem}[chapter]
 \newtheorem{cor}[thm]{Corollary}
 \newtheorem{lem}[thm]{Lemma}
 \newtheorem{llem}[thm]{Local Lemma}
 \newtheorem{lthm}[thm]{Local Theorem}
 % \newtheorem{quotecor}{Corollary}
 % \newtheorem{quotelem}{Lemma}[section]
 \newtheorem{quotethm}{Theorem}[chapter]
}{}
\makeatother
\newtheorem{Def}{Definition}

\newtheorem{manualtheoreminner}{Theorem}
\newenvironment{manualtheorem}[1]{%
  \renewcommand\themanualtheoreminner{#1}%
  \manualtheoreminner
}{\endmanualtheoreminner}

\newtheorem{manuallemmainner}{Lemma}
\newenvironment{manuallemma}[1]{%
  \renewcommand\themanuallemmainner{#1}%
  \manuallemmainner
}{\endmanuallemmainner}

\newcommand{\loclemma}{Lemma}


% \newcommand{\proofend}{\(\bullet\)}
% \newcommand{\proofend}{\hfill\(\blacksquare\)}
\newcommand{\proofend}{\hfill\(\Box\)}
\newenvironment{thmproof}
{\textbf{Proof.}}
{\proofend}

\newcommand{\chapterTypeout}[1]{\typeout{#1} \chapter{#1}}
\newcommand{\sectionTypeout}[1]{\typeout{#1} \section{#1}}

% abbreviations, ensuremath
\newcommand{\fx}{\ensuremath{f(x)}}
\newcommand{\gx}{\ensuremath{g(x)}}
\newcommand{\lrangle}[1]{\ensuremath{\left\langle #1 \right\rangle}}
\newcommand{\lrbangle}[1]{\ensuremath{\left\langle #1 \right\rangle}}
\newcommand{\M}{\ensuremath{\mathfrak{M}}}
\newcommand{\mldots}{\ensuremath{\ldots}}
\newcommand{\salgebra}{\(\sigma\)-algebra}
\newcommand{\swedge}{\;\wedge\;}
\newcommand{\wlogy}{without loss of generality}
\newcommand{\Wlogy}{Without loss of generality}
\newcommand{\twopii}{\ensuremath{2\pi i}}
\newcommand{\dtwopii}{\ensuremath{\frac{1}{\twopii}}}

% https://tex.stackexchange.com/
% questions/22252/how-to-typeset-function-restrictions
\newcommand\restr[2]{\ensuremath{% we make the whole thing an ordinary symbol
  \left.\kern-\nulldelimiterspace % automatically resize the bar with \right
  #1 % the function
  \vphantom{\big|} % pretend it's a little taller at normal size
  \right|_{#2} % this is the delimiter
  }}

\newenvironment{excopyOLD}
{\item\begin{minipage}[t]{.8\textwidth}\footnotesize}
{\smallskip\hrule\end{minipage}}

\newenvironment{excopy}
{\item % \relax
 \begin{list}{}{
 \setlength{\topsep}{0pt}
 \setlength{\partopsep}{0pt}
 \setlength{\itemsep}{0pt}
 \setlength{\parsep}{0pt}
 \setlength{\leftmargin}{0pt}
 \setlength{\rightmargin}{20pt}
 \setlength{\listparindent}{0pt}
 \setlength{\itemindent}{0pt}
 % \setlength{\labelsep}{0pt}
 \setlength{\labelwidth}{0pt}
 \footnotesize
 }
 \item
}
{\par
 % {\nullfont 0}
 \hrulefill
 \end{list}
}

\newcommand{\unfinished}{\par\textbf{Unfinished !!!!!!!!!!!!!}}

\usepackage{subfig}
\usepackage[metapost]{mfpic}
\opengraphsfile{myfigs}


\title{Notes and Solutions to Exercises \\
          from \\
       Real and Complex Analysis / Walter Rudin}


%%%%%%%%%%%%%%%%%%%%%%%%%%%%%%%%%%%%%%%%%%%%%%%%%%%%%%%%%%%%%%%%%%%%%%%%
%%%%%%%%%%%%%%%%%%%%%%%%%%%%%%%%%%%%%%%%%%%%%%%%%%%%%%%%%%%%%%%%%%%%%%%%
%%%%%%%%%%%%%%%%%%%%%%%%%%%%%%%%%%%%%%%%%%%%%%%%%%%%%%%%%%%%%%%%%%%%%%%%
\begin{document}
\maketitle
\newpage
\tableofcontents
\newpage


\maketitle

%%%%%%%%%%%%%%%%%%%%%%%%%%%%%%%%%%%%%%%%%%%%%%%%%%%%%%%%%%%%%%%%%%%%%%%%
%%%%%%%%%%%%%%%%%%%%%%%%%%%%%%%%%%%%%%%%%%%%%%%%%%%%%%%%%%%%%%%%%%%%%%%%
%%%%%%%%%%%%%%%%%%%%%%%%%%%%%%%%%%%%%%%%%%%%%%%%%%%%%%%%%%%%%%%%%%%%%%%%
\setcounter{chapter}{-2}
\chapterTypeout{About this Document}

Here I try to solve problems
from the book \cite{RudinRCA80}
\begin{center}
\textbf{Real and Complex Analysis}\\
by
\textbf{Walter Rudin}

Whenever there is reference to theorem or lemma, it implicitly
targets this text, unless otherwise specified.
Furthermore, we have in this document lemmas or theorems
which are referred to as \emph{local} lemma
or \emph{local} theorem.
\end{center}

%%%%%%%%%%%%%%%%%%%%%%%%%%%%%%%%%%%%%%%%%%%%%%%%%%%%%%%%%%%%%%%%%%%%%%%%
%%%%%%%%%%%%%%%%%%%%%%%%%%%%%%%%%%%%%%%%%%%%%%%%%%%%%%%%%%%%%%%%%%%%%%%%
%%%%%%%%%%%%%%%%%%%%%%%%%%%%%%%%%%%%%%%%%%%%%%%%%%%%%%%%%%%%%%%%%%%%%%%%
\section*{Notation}

% Common notations

For each natural \(n\in\N\) we define
\begin{equation*}
\N_n \eqdef \{m\in\N: 1\leq m \leq n\} \qquad
\Z_n \eqdef \{m\in\Z: 0\leq m < n\}.
\end{equation*}

We define the positive subsets
\begin{alignat*}{2}
\Q^+ &\eqdef \{q\in\Q: q>0\}
 &\qquad
 \Q^\oplus &\eqdef \{q\in\Q: q\geq 0\} \\
\R^+ &\eqdef \{r\in\R: r>0\}
 &\qquad
 \R^\oplus &\eqdef \{r\in\R: r\geq 0\}.
\end{alignat*}
The non-negative and negation integers
\begin{align*}
\Z^+ = \{n\in\Z: n\geq 0\} = \{0\} \cup \N
\qquad
\Z^- =  \{n\in\Z: n < 0\} = \Z \setminus \Z^+.
\end{align*}


We use the following single side limit notations

\begin{alignat*}{2}
\lim_{t\to a^+} f(x) &= \lim_{\stackrel{h\to a}{h>a}} f(x)
 &\qquad
  \lim_{t\to a^-} f(x) &= \lim_{\stackrel{h\to a}{h<a}} f(x) \\
\varlimsup_{t\to a^+} f(x) &= \varlimsup_{\stackrel{h\to a}{h>a}} f(x)
 &\qquad
  \varlimsup_{t\to a^-} f(x) &= \varlimsup_{\stackrel{h\to a}{h<a}} f(x) \\
\varliminf_{t\to a^+} f(x) &= \varliminf_{\stackrel{h\to a}{h>a}} f(x)
 &\qquad
  \varliminf_{t\to a^-} f(x) &= \varliminf_{\stackrel{h\to a}{h<a}} f(x)
\end{alignat*}

For topolgical notions we use
\begin{center}
\begin{tabular}{ll}
\(\inter{A}\) & Interior of $A$ \\
\(\closure{A}\) & Closure of $A$ \\
\(\boundary{A}\) & Boundary of $A$ \\
\end{tabular}
\end{center}



%%%%%%%%%%%%%%%%%%%%%%%%%%%%%%%%%%%%%%%%%%%%%%%%%%%%%%%%%%%%%%%%%%%%%%%%
%%%%%%%%%%%%%%%%%%%%%%%%%%%%%%%%%%%%%%%%%%%%%%%%%%%%%%%%%%%%%%%%%%%%%%%%
%%%%%%%%%%%%%%%%%%%%%%%%%%%%%%%%%%%%%%%%%%%%%%%%%%%%%%%%%%%%%%%%%%%%%%%%
\chapterTypeout{Prologue: The Exponential Function}

% $Id: trig.tex,v 1.6 2008/07/19 08:56:55 yotam Exp $

%%%%%%%%%%%%%%%%%%%%%%%%%%%%%%%%%%%%%%%%%%%%%%%%%%%%%%%%%%%%%%%%%%%%%%%%
%%%%%%%%%%%%%%%%%%%%%%%%%%%%%%%%%%%%%%%%%%%%%%%%%%%%%%%%%%%%%%%%%%%%%%%%
%%%%%%%%%%%%%%%%%%%%%%%%%%%%%%%%%%%%%%%%%%%%%%%%%%%%%%%%%%%%%%%%%%%%%%%%
\section{Trigonometry}

We will need workout several results dealing with 
eqalities and inequalities of complex numbers.
Argumentation wil be based both on the cartesian and polar representaions.

%%%%%%%%%%%%%%%%%%%%%%%%%%%%%%%%%%%%%%%%%%%%%%%%%%%%%%%%%%%%%%%%%%%%%%%%
\subsection{Angle Argument}

For each non zero \(z\in\C\) there is unique polar representaion
\(z=re^{i\theta}\) where \(r=|z|\) and \(\theta\in[0,2\pi) \subset \R\).
We define the \emph{argument}
\begin{equation} \label{eq:arg}
\Arg(z) = \Arg(|z|e^{i\theta}) \eqdef \theta \in [0,2\pi).
\end{equation}


%%%%%%%%%%%%%%%%%%%%%%%%%%%%%%%%%%%%%%%%%%%%%%%%%%%%%%%%%%%%%%%%%%%%%%%%
\subsection{Cosines and Sines}

In addition to the Prologue of \cite{RudinRCA80}, we want to establish
the following trigonometric equalities.

%%%%%%%%%%%%%%%%%%%%%%%%%%%%%%%%
\begin{llem} \label{llem:trig:cos:sin}
\begin{alignat}{2}
 \cos(\pi/3)  & =  1/2   &\qquad \sin(\pi/3) & = \sqrt{3}/2
    \label{eq:cosin:pi3} \\
 \cos(2\pi/3) & = -1/2   &\qquad \sin(\pi/3) & = \sqrt{3}/2
    \label{eq:cosin:2pi3}
\end{alignat}
\end{llem}
\begin{thmproof}
Let \(a = c+si = e^{\pi i/3}\), with \(c,s\in\R\).
From \((c+si)^3 = e^{\pi i} = -1\)
we get the following equations
\begin{eqnarray*}
|z|^2 =  c^2 + s^2   &=& 1 \\
\Re(z) = c^3 - 3cs^2 &=& -1 \\
\Im(z) = 3c^2s - s^3  &=& 0
\end{eqnarray*}
By last equation, 
\(c=0\) iff \(s=0\). But if this happens it contradicts the first equation.
Thus \(c\neq 0 \neq s\).
We divide the last equation by $s$ and add the first.
We get \(4c^2 = 1\) and so \(s^2= 3/4\) and \(c=\pm1/2\).
But if \(c = -1/2\) then by the section equation
\(-1/8 - 3\cdot(-1/2)\cdot(3/4) = 1 \neq -1\) contradicting the second equation.
Hence \(c=1/2\) and the cosine equality of \eqref{eq:cosin:pi3} is true.


Now let \(c+si = e^{\pi 2i/3}\), with \(c,s\in\R\). 
From \((c+si)^3 = e^{2\pi i} =-1\)
we get the following equations
\begin{eqnarray*}
c^2 + s^2   &=& 1 \\
c^3 - 3cs^2 &=& 1 \\
3c^s - s^3  &=& 0
\end{eqnarray*}
By similar arguments we again get \(c=\pm1/2\), but this time
the second equations elimiates the \(c=1/2\) possibility and
so the cosine equality of \eqref{eq:cosin:2pi3} is true.

By the the \(\cos^2(x) + \sin^2(z) = 1\) equality
\begin{equation*}
 \sin(\pi/3) = \sin(2\pi/3) = \pm\sqrt{3}/2.
\end{equation*}
But we know that \(\sin(0) = \sin(\pi) = 0\)
and \(\sin(\pi/2) = 1\). Also \(sin(t)\) is increasing 
in \([0,\pi/2]\) and decreasing in \([\pi/2,\pi]\)
and thus the desired equalities for sine are forced.
\end{thmproof}


%%%%%%%%%%%%%%%%%%%%%%%%%%%%%%%%%%%%%%%%%%%%%%%%%%%%%%%%%%%%%%%%%%%%%%%%
\subsubsection{Double angle}

\begin{llem}
The following identities hold.
\begin{equation*}
\cos^2(z) = (\cos(2z) + 1)/2
\end{equation*}
\end{llem}
\begin{thmproof}
\begin{equation*}
\cos^2(z) 
= \left((e^{iz} + e^{-iz})/2\right)^2
= (e^{2iz} + + 2\cdot 0 + e^{-2iz})/4
= \cos(2z)/2
\end{equation*}
\end{thmproof}




%%%%%%%%%%%%%%%%%%%%%%%%%%%%%%%%%%%%%%%%%%%%%%%%%%%%%%%%%%%%%%%%%%%%%%%%
\subsection{Tangent}

The \emph{tangent} function is defined for every complex number \(z\in\C\)
such that \(\cos(z) \neq 0\) by
\begin{equation} \label{eq:tan}
\tan(z) \eqdef \frac{\sin(z)}{\cos(z)}.
\end{equation}

Viewed as a real function, \(\tan: \R\to\R\) is increasing where it's defined.
%%%%%%%%%%%%%%%%%%%%%%%%%%%%%%%%
\begin{llem} \label{llem:tan}
Suppose \(\alpha,\beta \in (n\pi/2, (n+1)\pi/2)\) where \(n=0,1,2,3\).
If \(\alpha < \beta\) then \(\tan(\alpha) < \tan(\beta)\).
\end{llem}
\begin{thmproof}
We simply verify each \((n\pi/2, (n+1)\pi/2)\) quadrant case. 
\begin{itemize}
\item[\(n=0\)] 
  Both \(\sin(t)\) and \(\cos(t)\) are positive, 
  \(\sin(t)\) increases and \(\cos(t)\) decreases. 
\item[\(n=1\)] Consider \(-\tan(t) = \sin(t)/(-\cos(t))\).
  Both \(\sin(t)\) and \(-\cos(t)\) are positive, 
  \(\sin(t)\) decreases and \(-\cos(t)\) increases, 
  thus \(-\tan(t)\) decreases.
\item[\(n=2\)] Consider \(\tan(t) = (-\sin(t))/(-\cos(t))\).
  Both \(-\sin(t)\) and \(-\cos(t)\) are positive, 
  \(-\sin(t)\) increases and \(-\cos(t)\) decreases.
\item[\(n=3\)] % Similar arguments as the \(n=1\) case.
  Consider \(-\tan(t) = (-\sin(t))/\cos(t)\).
  Both \(-\sin(t)\) and \(\cos(t)\) are positive, 
  \(-\sin(t)\) decreases and \(\cos(t)\) increases,
  thus \(-\tan(t)\) decreases.
\end{itemize}
Clearly \(\tan(t)\) increases in all the above cases.
\end{thmproof}


%%%%%%%%%%%%%%%%%%%%%%%%%%%%%%%%%%%%%%%%%%%%%%%%%%%%%%%%%%%%%%%%%%%%%%%%
%%%%%%%%%%%%%%%%%%%%%%%%%%%%%%%%%%%%%%%%%%%%%%%%%%%%%%%%%%%%%%%%%%%%%%%%
\section{Sums}

\paragraph{Definition.} 
Given non zero complex numbers \(z_1\) and \(z_2\), 
we define the \emph{angle} \(\Ang(z_1,z_2)\)
to be the unique \(\theta \in [0,\pi]\) such that
\begin{equation*}
\frac{z_1/ z_2}{|z_1/ z_2|} = \pm e^{i\theta}.
\end{equation*}

We start by showing a case where the absolute value must grow.

%%%%%%%%%%%%%%%%%%%%%%%%%%%%%%%%
\begin{llem} \label{llem:z1z2:grow}
Let non zero \(z_1,z_2\in \C\) satisfy \(\Ang(z_1,z_2) \leq \pi/2\)
then \(|z_1 + z_2| > \max(|z_1|,|z_2|)\). 
\end{llem}
\begin{thmproof}
By symmetry, suffice to show \(|z_1 + z_2| > |z_1|\).
Let \(z_1 = re^{i\theta}\). If we multiply both \(z_1\) and \(z_2\)
by \(e^{-i\theta}\) both the assumptions and desired inequalities remain.
Thus we can assume that \(0 < z_1 \in\R\).
With the representaion, \(z_k = x_k + iy_k\) for \(k=1,2\),
we have \(y_1 = 0\)
and since the \(\Ang(z_1,z_2) \leq \pi/2\), 
we also have \(x_1,x_2 \geq 0\).
Thus
\begin{equation*}
|z_1 + z_2|^2 = (x_1 + x_2)^2 + y_2^2 \geq x_1^2 + y_2^2 = |z_1|^2
\end{equation*}
and the desired inequality follows.
\end{thmproof}

Now we see a case where the absolute value cannot grow.
%%%%%%%%%%%%%%%%%%%%%%%%%%%%%%%%
\begin{llem} \label{llem:z12:2pi3}
Let non zero \(z_k = r_k e^{i\theta_k}\) for \(k=1,2\).
If \(\Ang(z_1,z_2) \geq 2\pi/3\) then
\begin{equation*}
|z_1 + z_2| \leq \min(|z_1|,|z_2|).
\end{equation*}
\end{llem}
\begin{thmproof}
By symmetry and by multiplying both \(z_1\) and \(z_2\) by 
\(e^{-i\theta_1}/\max(|z_1|,|z_2|)\)
we may assume 
\begin{gather*}
|z_1| \leq 1 \qquad |z_2| < 1 \\
0=\theta_1 \leq 2\pi/3 \leq \theta_2 \leq \pi
\end{gather*}
and we need to show that \(|z_1 + z_2| \leq 1\).
We look at the representaions \(z_1=x_1\) and \(z_2 = x_2 + iy_2\)
where \(x_1,x_2,y_2\in\R\).
Since \(\cos(t)\) and \(\sin(t)\) decrease on \(t\in[\pi/2,\pi]\)
we have 
\(-1 \leq x_2 \leq 0 < x_1 \leq 1\).
and
\(0\leq y_2 \leq -(\sqrt{3}/2)x_2\).
There are two cases.

\textbf{Case 1.} Assume \(x_2\geq -x_1/2\).
\begin{equation*}
|z_1 + z_2|^2
 = (x_1 + x_2)^2 + y_2^2 
 \leq (x_1 + x_2)^2 + (3/4)x_2^2 
 = (7/4)x_2^2 + 2x_1x_2 + x_1^2
\end{equation*}
Looking at the last expression as a function of \(x_2\in\R\),
it attains its minimum at \(x_2 = -2x_1\)
Hence the maximum for this case, is achieved at \(x_2 = 0\)
hence \(|z_1 + z_2|^2 \leq x_1^2 \leq 1\).

\textbf{Case 2.} Assume \(x_2\leq -x_1/2\).
Hence \(|x_1 + x_2| \leq |x_2|\) and thus
\begin{equation*}
|z_1 + z_2|^2
 = (x_1 + x_2)^2 + y_2^2 
 \leq x_2^2 + y_2^2 \leq 1
\end{equation*}

Thus in both cases \(|z_1 + z_2| \leq 1\).
\end{thmproof}

Our next lemma shows that for any small total sum of small numbers,
we can find a small pair.
%%%%%%%%%%%%%%%%%%%%%%%%%%%%%%%%
\begin{llem} \label{llem:zpairlt1}
Suppose  \(n\geq 2\) and \(z_k\in\C\) for \(k\in\N_n\).
If \(|z_k| \leq 1\) for each \(k\in \N_n\) and 
\(|\sum_{k=1}^n z_k| \leq 1\)
then there exists a pair \(j,k\in\N_n\) such that \(j\neq k\)
and 
\begin{equation} \label{eq:zpairlt1}
|z_j + z_k| \leq 1.
\end{equation}
\end{llem}
\begin{thmproof}
By induction on $n$. For \(n=1\) there is nothing to show.
For \(n=2\), we simply take \(j=1\) and \(k=2\).
Now assume the lemma is true for all \(n<n'\) for some \(n'>2\).
By negation, assume that for any \(j,k\in\N_n\) pair 
\begin{equation} \label{eq:zpairlt1:zjzk:gt1}
|z_j + z_k| > 1.
\end{equation}

We may also assume that 
\begin{equation} \label{eq:zpairlt1:Azjzk}
\Ang(z_j,z_k) < 2\pi/3
\end{equation}
for any two indices \(j,k\in\N_{n'}\), since otherwise,
\loclemma~\ref{llem:z12:2pi3} contradicts \eqref{eq:zpairlt1:zjzk:gt1}.

We want to have \(\mathbf{z} = (z_k)_{k=1}^{n'}\) be such that 
their arguments will be increasing in \([0,2\pi/3)\).
\Wlogy\ we may assume 
\(\mathbf{z}\) is sorted so \(\theta_k \leq \theta_{k+1}\)
for each \(k\in\N_{n'-1}\). 
We define the ``preceding'' angles
\(\alpha_1 = \theta_1 + 2\pi - \theta_{n'}\) 
and
\(\alpha_{k+1} = \theta_{k+1} - \theta_k\) for \(k\in\N_{n'-1}\).
Let \(K\in\N_{n'}\) be such that \(\alpha_K = \max\{\alpha_k:k\in N_{n'}\}\).
\Wlogy, we may assume \(K=1\) and \(\theta_1 = 0\). Otherwise, 
we can multiply \(\mathbf{z}\) by \(e^{-i\theta_K}\) and resort by
their (new) arguments.
To show now that the arguments of \(\mathbf{z}\) are in \([0,2\pi/3)\),
assume by negation that there is some \(k'\in\N_{n'}\) such that
\(\theta_k \in [2\pi/3,2\pi)\).
\newline
\textbf{Case \ich{i}}.
If \(\theta_{k'}\in [2\pi/3,4\pi/3]\) then \(\Arg(z_1,z_{k'})\geq 2\pi/3\)
contradiction to \eqref{eq:zpairlt1:Azjzk}.
\newline
\textbf{Case \ich{ii}}.
Let \(k''\) be the minimal such that
\(\theta_{k''}\in (4\pi/3,2\pi)\) then put 
\(\beta = \theta_{k''} - 4\pi/3 \in (0,2\pi/3)\) and 
again because of \eqref{eq:zpairlt1:Azjzk} we deduce that
\(\theta_k \notin (\alpha, \alpha + 2\pi/3)\) 
for \(k\in\N_{n'}\).
Combining with Case~\ich{i} deduction
\(\theta_k \notin (\alpha, \alpha + 4\pi/3)\)
for \(k\in\N_{n'}\). 
Hence \(\alpha_{k''} > 4\pi/3\), but
\begin{equation*}
\alpha_1 \leq \Ang(z_1,z_{k''}) = 2\pi - \theta_{k''} 
 \leq 2\pi/3 - \alpha < 2\pi/3
\end{equation*}
which contradict the assumption of 
\(K=1\), that is \(\alpha_1\) being the largest preceding angle.

We look at the ``middle sum''
\begin{equation*}
U \eqdef \sum_{k=2}^{n'-1} z_k = |U|e^{i\alpha}.
\end{equation*}
We rotate all of \(\mathbf{z}\) by \(e^{-i\alpha}\), this time allowing
for negative angle arguments. For convenience we rename back to \(\mathbf{z}\).
Now we have the following new situation.
% \begin{gather}
% \begin{alignedat}{3}
\begin{alignat*}{3}
z_k &= x_k + iy_k = |z_k|e^{i\theta_k} 
  &\qquad
  |z_k| &\leq 1 
  & \qquad (k\in\N_{n'})  \\
S &\eqdef \sum_{k=1}^{n'} z_k 
  &\qquad
|S| &\leq 1 
  &&  \notag
\end{alignat*} % \\[1pt]
\begin{gather}
 R \eqdef \sum_{k=2}^{n'-1} z_k = S - z_1 - z_{n'} \in \R. \notag \\
 \theta_k < \theta_{k+1} \qquad (k\in\N_{n'-1}) \notag \\
 \theta_1 \leq 0 \leq \theta_{n'} < \theta_1 + 2\pi/3 
    \label{eq:zpairlt1:thetan1}
\end{gather}
% \end{gather}


We may also assume \wlogy\ that 
\begin{equation}  \label{eq:zpairlt1:theta1n}
-\theta_1 < \theta_{n'}. 
\end{equation}
Otherwise we can substitue all 
\((z_k)_{k=1}^{n'}\)
with 
\((\overline{z_k})_{k=1}^{n'}\).

% Given the cartesian representation \(z_k = x_k + iy_k\) for \(k\in\N_{n'}\),
We look at the signs of \(x_1\) and \(x_{n'}\).
Clearly
\begin{alignat*}{3}
 x_1 & \leq 0     &\qquad &\textrm{iff}\quad  &\theta_1    & \leq -\pi/2 \\
 x_{n'} & \leq 0  &\qquad &\textrm{iff}\quad  &\theta_{n'} & \geq  \pi/2 
\end{alignat*}
We examine the following cases. 
\paragraph{Case 1.} Assume \(x_1 + x_{n'} \geq 0\)
(covering the case of \(x_1\geq 0\) and \(x_{n'}\geq 0\)).
Compute
\begin{eqnarray*}
|S|^2 - |z_1 + z_{n'}|^2 
&=& |(U + z_1 + z_{n'})|^2 - |z_1 + z_{n'}|^2 \\
&=& |U + (x_1 + x_{n'}) + i(y_1 + y_{n'})|^2 - 
    |(x_1 + x_{n'}) + i(y_1 + y_{n'})|^2 \\
&=& U^2 + 2U(x_1 + x_{n'})
\end{eqnarray*}
Since \(x_1 + x_{n'} \geq 0\), by the above equality
\(|S|^2 \geq |z_1 + z_{n'}|^2 > 1\) contradiction to the lemma's assumption.
\paragraph{Case 2.} Assume \(x_1\leq 0\) and \(x_{n'}\leq 0\).
Then
\(\theta_1 \leq -\pi/2\) and \(\theta_{n'} \geq \pi/2\)
contradiction to \eqref{eq:zpairlt1:thetan1}.
\paragraph{Case 3.} Assume \(x_1\leq 0\) and \(x_{n'}\geq 0\).
This contradicts the ``conjugate choice'' 
assumption of \eqref{eq:zpairlt1:theta1n}
\paragraph{Case 4.} Assume \(x_1\geq 0\), and \(x_{n'}\leq 0\),
and \(x_1 + x_{n'} < 0\).
Then
\(1/x_{n'} > -1/x_1\), and so
\begin{equation*}
\tan(z_{n'}) =  y_{n'}/x_{n'} > -y_{n'}/x_1 > y_1 / x_1 = \tan(z_1) = \tan(-z_1).
\end{equation*}
which is a contadiction, since 
\begin{equation*}
\pi/2 < \theta_{n'} < \theta_1 + \pi < \pi
\end{equation*}
and applying \loclemma~\ref{llem:tan} gives \(y_{n'}/x_{n'} \leq y_1/x_1\).

Since all cases were refuted a pair \(j,k\in\N_n\)
satisfying \eqref{eq:zpairlt1} must exist.
\end{thmproof}



%The following lemma shows that in certain conditions 
%we can order small numbers so their partial sums is stays small.
Intuitively, we now show that for numbers within the unit circle,
if their sum is inside, then they can be ordered so partial sums are inside.
%%%%%%%%%%%%%%%%%%%%%%%%%%%%%%%%
\begin{llem} \label{llem:zsubsum}
Suppose \(z_k\in\C\) for \(k\in\N_n\). % and \(S\eqdef \sum_{k=1}^n z_k\).
If \(|z_k| \leq 1\) for each \(k\in \N_n\) and 
\begin{equation} \label{eq:zsubsum:assert}
\left|\sum_{k=1}^n z_k\right| \leq 1
\end{equation}
then \((z_k)\) can be reordered such that \(|S_m| \leq 1\), where
\begin{equation*}
S_m = \sum_{k=1}^m z_k
\end{equation*}
for each \(m\in\N_n\).
\end{llem}

\paragraph{Notes.} 
\begin{itemize}
\item There could be repetitions of values in \(z_k\).
\item
We use the term ``can be reordered''.
It would be more rigorous to say that
there exist a permutation \(\sigma\) of \((1,\ldots,n)\) 
and sum \(z_{\sigma(k)}\). But we prefer to simplify notations.
\end{itemize}

\begin{thmproof}
By the lemma above there exist  \(j,k\in\N_{n'}\) such that \(j\neq k\) and
\begin{equation} \label{eq:zsubsum:zjzk}
 |z_j + z_k| \leq 1 
\end{equation}
We can look at a sequence
\(\tilde{\mathbf{z}}\) 
of \(n'-1\) numbers
made by the original \(\mathbf{z}=(z_k)_{k=1}^{n'}\) 
with \(z_j\) and \(z_k\) dropped
and \(z_j + z_k\) added. Applying the induction hypothesis on
\(\tilde{\mathbf{z}}\) we get an~\((n'-1)\)-ordering
such that the analogue of \eqref{eq:zsubsum:assert} holds.
To get reordering for \(\mathbf{z}\)
we look at the following cases:
\begin{itemize}
 \itemch{a} If \(z_j+z_k\) is first in the \((n'-1)\)-ordering,
 then we simply split it back
 \itemch{b} If \(z_j+z_k\) is last in the \((n'-1)\)-ordering,
 then we simply split it back
 \itemch{c} Assume \(z_j+z_k\) is in the $p$ place of the \((n'-1)\)-ordering.
 Since \(p<n'-1\) we can apply a sub-\((p+1)\)-ordering,
 thus changing the the initial $p$ places with new \(p+1\).
 The final \(n'-p-1\) places are just shifted (and remain the tail).
\end{itemize}
Thus, for each possible case, we have shown how a desired reordering
canbe constructed.
\end{thmproof}

Here is another view of the above.
%%%%%%%%%%%%%%%%%%%%%%%%%%%%%%%%
\begin{llem} \label{llem:zsub:block}
Suppose \(z_k\in\C\) for \(k\in\N_n\). % and \(S\eqdef \sum_{k=1}^n z_k\).
and \(|z_k| \leq 1\) for each \(k\in \N_n\).
If there exists \(m\in\N_n\) such that each subset \(A\subset\N_n\)
with $m$ elements (\(|A|=m\)), satisfy
\begin{equation*}
\left|\sum_{k\in A} z_k\right| > 1
\end{equation*}
then 
\begin{equation*}
\left|\sum_{k=1}^n z_k\right| > 1.
\end{equation*}
\end{llem}
\begin{thmproof}
By negation, applying the above \loclemma~\ref{llem:zsubsum} gives
a contradiction.
\end{thmproof}


\iftrue
 % -*- latex -*-
% $Id: rudinrca1.tex,v 1.2 2006/09/08 07:29:12 yotam Exp $

%%%%%%%%%%%%%%%%%%%%%%%%%%%%%%%%%%%%%%%%%%%%%%%%%%%%%%%%%%%%%%%%%%%%%%%%
%%%%%%%%%%%%%%%%%%%%%%%%%%%%%%%%%%%%%%%%%%%%%%%%%%%%%%%%%%%%%%%%%%%%%%%%
%%%%%%%%%%%%%%%%%%%%%%%%%%%%%%%%%%%%%%%%%%%%%%%%%%%%%%%%%%%%%%%%%%%%%%%%
\chapterTypeout{Abstract Integration}

%%%%%%%%%%%%%%%%%%%%%%%%%%%%%%%%%%%%%%%%%%%%%%%%%%%%%%%%%%%%%%%%%%%%%%%%
%%%%%%%%%%%%%%%%%%%%%%%%%%%%%%%%%%%%%%%%%%%%%%%%%%%%%%%%%%%%%%%%%%%%%%%%
\section{Notes}

%%%%%%%%%%%%%%%%%%%%%%%%%%%%%%%%%%%%%%%%%%%%%%%%%%%%%%%%%%%%%%%%%%%%%%%%
\subsection{Lebesgue's Monotone Convergence --- Proof Fix}

\index{Lebesgue}
In Theorem~1.26 (page~22), it says:
\begin{quotation}
 \mldots, there exists \(\alpha \in [0,\infty)\) such that
\end{quotation}
It should be:
\begin{quotation}
 \mldots, there exists \(\alpha \in [0,\infty]\) such that
\end{quotation}

%%%%%%%%%%%%%%%%%%%%%%%%%%%%%%%%%%%%%%%%%%%%%%%%%%%%%%%%%%%%%%%%%%%%%%%%
\subsection{Lebesgue's Dominated Convergence --- Variant}

\index{Lebesgue}
Theorem~1.34 on page 27 requires
\begin{equation*}
|f_n(x)| \leq g(x) \qquad \textrm{(}n=1,2,3,\ldots; x\in X\textrm{),}
\end{equation*}
Instead, it could require:
\begin{eqnarray*}
g_n &\to& g \\
\int_X g_n d\mu  &\to& \int_X g d\mu  \\
|f_n(x)| &\leq& g(x).
\end{eqnarray*}
Let's have it explicitly.
%%%%%%%%%%%%%%%%
\begin{llem} \label{lem:Lebesgue:domvar}
Suppose \(\{f_n\}\) is a sequence of complex measurable functions on $X$ such that
\begin{equation*}
 f(x) = \lim_{n\to\infty} f_n(x)
\end{equation*}
exists for every \(x\in X\). If there is a a sequence of functions \(\{g_n\}\)
in \(L^1(\mu)\) and a function \(g\in L^1(\mu)\)
such that
\begin{eqnarray*}
   |f_n(x)| &\leq& g_n(x) \qquad (\forall n\in\N,\;\forall x\in X) \\
  \lim_{n\to\infty} g_n(x) &=& g(x) \qquad (\forall x\in X) \\
  \lim_{n\to\infty} \int_X g_n\,d\mu  &=& \int_X g\,d\mu
\end{eqnarray*}
then
\begin{eqnarray}
  f &\in& L^1(\mu) \notag \\
  \lim_{n\to\infty} \int_X |f_n - f|\,d\mu  & = & 0 \label{eq:leb:domv1} \\
  \lim_{n\to\infty} \int_X f_n\,d\mu  & = & \int_X f\,d\mu \label{eq:leb:domv2}
\end{eqnarray}
\end{llem}
Note: The orginal theorem~1.34 (\cite{RudinRCA80}) follows using \(g_n=g\).

\begin{thmproof}
Since
\begin{equation*}
 |f| = \lim_{n\to\infty} |f_n| \leq \lim_{n\to\infty} |g_n| = |g|
\end{equation*}
and $f$ is measurable, \(f\in L^1(\mu)\).
Since \(|f-f_n|\leq g_n + g\),
Fatou's lemma applies to \(g + g_n - |f_n - f|\) and yields
\begin{eqnarray*}
 2 \int_X g\,d\mu
 &=& \int_X g\,d\mu
     + \int_X \lim_{n\to\infty}g_n
     - \int_X \lim_{n\to\infty}|f_n - f|\,d\mu \\
 &=& \int_X \left(\lim_{n\to\infty}(g + g_n  - |f_n - f|\right)\,d\mu \\
 &=& \int_X \liminf_{n\to\infty} g + g_n  - |f_n - f|\,d\mu \\
 &\leq& \liminf_{n\to\infty}\int_X g + g_n  - |f_n - f|\,d\mu \\
 &=& \int_X g\,d\mu
     + \lim_{n\to\infty} \int_X g_n\,d\mu
     \liminf_{n\to\infty} \left(-\int_X |f_n - f|\,d\mu\right) \\
 &=& 2\int_X g\,d\mu - \limsup_{n\to\infty} \int_X |f_n - f|\,d\mu .
\end{eqnarray*}
Since \(2\int_X g\,d\mu\) is finite
\begin{equation*}
 \liminf_{n\to\infty} \int_X |f_n - f|\,d\mu \leq 0.
\end{equation*}
which shows (\ref{eq:leb:domv1}).
The last assertion (\ref{eq:leb:domv2}) follows immediately from
\begin{equation*}
 \left| \int_X f_n\,d\mu - \int_X f\,d\mu\right|  \leq \int_X |f_n - f|\,d\mu.
\end{equation*}
\end{thmproof}





%%%%%%%%%%%%%%%%%%%%%%%%%%%%%%%%%%%%%%%%%%%%%%%%%%%%%%%%%%%%%%%%%%%%%%%%
%%%%%%%%%%%%%%%%%%%%%%%%%%%%%%%%%%%%%%%%%%%%%%%%%%%%%%%%%%%%%%%%%%%%%%%%
\section{Exercises} % pages 32-33

%%%%%%%%%%%%%%%%%
\begin{enumerate}
%%%%%%%%%%%%%%%%%

%%%%%%%%%%%%%%
\begin{excopy}
Does there exist an infinite \salgebra\ %
which has only countably many members?
\end{excopy}

No.

Say be negation a \salgebra\ \M\ in $X$ %
has countably many members \(\{A_i\}_{i=1}^\infty\).
We will build an infinite countable family out of \M,
mutually disjoint, that will be a base for \M.

For each \(x\in X\), let
\begin{equation} \label{eq:cntcap}
G_x = \bigcap_{x\in A_i} A_i.
\end{equation}
Clearly, \(x\in G_x\in \M\). The latter membership relation
by the face that the intersection in (\ref{eq:cntcap}) is
of at most countable number of sets.
We observe that if \(G_x \cap G_y \neq \emptyset\)
then \(G_x = G_y\).

By negation, if \(G_x \neq G_y\), then
\(G_x\setminus G_y \neq \emptyset\)
or
\(G_y\setminus G_x \neq \emptyset\).
\(G_x\cap G_y \subsetneq G_y\)
Put \(G = G_x\cap G_y \in \M\)
and so
\(G\subsetneq G_x\)
or
\(G \subsetneq G_y\).
\Wlogy, \(G \subsetneq G_x\).
If \(x\in G\) then $G$ participates in the intersection of (\ref{eq:cntcap})
and thus \(G_x\subset G\) leading to the \(G_x \subsetneq G_x\) contradiction.
Otherwise, \(x\neq G\) and so \(x \in G_y^c \in \M\)
and so \(G_y^c\) participates in (\ref{eq:cntcap}) and so
\(x\in G_x \subset G_y^c = G_x^c\) which is also a contradiction.

Thus the family \(\calB = \{G_x\}_{x\in X}\) is a subset of \M\
of disjoint (ignoring repetitions) subsets of $X$.
For each \(A\in \M\) we define
\begin{equation*}
A' = \bigcup_{x\in A} G_x.
\end{equation*}
Clearly, \(A\subset A'\). To show the reverse inclusion,
let \(w\in A'\).
Then for some \(x\in A\) we have \(w \in G_X\).
But this means that for every set \(S\in M\),
if \(x\in S\) then \(w\in S\). In particular, \(x\in A\) and so
\(A'\subset A\), hence \(A=A'\).

We showed that every set in \M\ is a disjoint union
of some subset of \calB.
Thus, \calB\ cannot be finite and by being subset of \M\
is countable. Therefore, the union of every subset of
\calB\ is in \M, and by being disjoint
\(|\M| = 2^{|\calB|} > \aleph_0\).


%%%%%%%%%%%%%%
\begin{excopy}
Prove and analogue of Theorem~1.8 for $n$ functions.
\end{excopy}

\begin{lthm}
Let \(\{u_i\}_{i=1}^n\) be real measurable functions on a measurable
space $X$, let
\(\Phi\)\ be a continuous mapped of \(\R^n\) into a topological space $Y$,
and define
\begin{equation*}
h(x) = \Phi(u_1(x),u_2(x),\ldots,u_n(x))
\end{equation*}
for \(x\in X\). Then \(h: X\to Y\) is measurable.
\end{lthm}
\begin{thmproof}
Put \(f(x) = (u_1(x),u_2(x),\ldots,u_n(x))\).

Then $f$ maps $X$ into \(\R^n\). Since
\(h = \Phi\circ f\), Theorem~1.7 shows that it is enough to prove
the measurability of $f$.

If $B$ is any open box in \(\R^n\), with sides parallel to the axes,
then $B$ is a cartesian product of $n$ segments \(\seqn{I}\), and
\begin{equation*}
f^{-1}(B) = \bigcap_{j=1}^n u_j^{-1}(I_j),
\end{equation*}
which is measurable, by our assertions on \(u_j\).
Every open set $V$ in \(\R^n\) is a countable union of such boxes \(B_i\),
and since
\begin{equation*}
f^{-1}(V) = f^{-1}\left(\bigcup_{i=1}^\infty B_i\right)
          = \bigcup_{i=1}^\infty f^{-1}(B_i),
\end{equation*}
\(f^{-1}\) is measurable.
\end{thmproof}

%%%%%%%%%%%%%%
\begin{excopy}
Prove that if $f$ is a real function on a measurable space $X$
such that \(\{x:f(x)\geq r\}\) is measurable for every rational $r$,
then $f$ is measurable.
\end{excopy}

Let \(\alpha\in \R\). Choose some decreasing sequence \(\{q_i\}_{i\in\N}\)
such that \(q_i\in\Q\) and \(q_i\to\alpha\). Now
\begin{equation*}
f^{-1}((\alpha,\infty]) = \bigcup_{i} f^{-1}((q_i,\infty]).
\end{equation*}
By Theorem~1.12(c), $f$ is measurable.

%%%%%%%%%%%%%%
\begin{excopy}
Let \(\{a_n\}\) and \(\{b_n\}\) be sequences in \([-\infty,\infty]\),
and prove the following assertions:

\begin{itemize}

\item[(a)] \qquad
   \(\displaystyle
      \limsup_{n\to \infty} (-a_n) =
     -\liminf_{n\to \infty} a_n
   \).

\item[(b)] \qquad
   \(\displaystyle
      \limsup_{n\to \infty} (a_n + b_n) \leq
      \limsup_{n\to \infty} a_n +
      \limsup_{n\to \infty} b_n\)

 provided none of the sums is of the form \(\infty - \infty\).

\item[(c)] If \(a_n\leq b_n\) for all $n$, then
 \[\liminf_{n\to\infty} a_n \leq \liminf_{n\to\infty} b_n\,.\]
\end{itemize}
Show by an example that strict inequality can hold in (b).
\end{excopy}

Let us have formalized definitions:
\begin{eqnarray*}
 \limsup_{n\to \infty} a_n  &=& \inf_{n\in\N}\, \sup_{i\geq n} a_i \\
 \liminf_{n\to \infty} a_n  &=& \sup_{n\in\N}\, \inf_{i\geq n} a_i
\end{eqnarray*}

\begin{itemize}
 \item[(a)]
  \begin{eqnarray*}
    \limsup_{n\to \infty} (-a_n)
     &=& \inf_{n\in\N}\, \sup_{i\geq n} -a_i \\
     &=& \inf_{n\in\N}\, -\inf_{i\geq n} a_i \\
     &=& -\sup_{n\in\N}\, \inf_{i\geq n} a_i \\
     &=& -\liminf_{n\to \infty} a_n
  \end{eqnarray*}

 \item[(b)]

 Let
 \begin{eqnarray}
  A_k &=& \sup_{n\geq k} a_n       \label{eq:Ak:limsup} \\
  B_k &=& \sup_{n\geq k} b_n       \label{eq:Bk:limsup} \\
  S_k &=& \sup_{n\geq k} a_n+b_n   \notag \\
 \end{eqnarray}
 If by negation, \(S_k > A_k + b_k\) then put
 \(\epsilon = S_k - (A_k+b_k)\).
 By definition, there exists \(m\geq k\) such that
 \begin{equation*}
 a_m+b_m > S_k - \epsilon \geq A_k + b_k
 \end{equation*}
 and so \(a_m > A_k\) or \(b_m > B_k\), but each is a contradiction.
 Hence,
 \begin{equation*}
  \sup_{n\geq k} a_n+b_n \leq \sup_{n\geq k} a_n + \sup_{n\geq k} b_n
 \end{equation*}
 for any \(k\in\N\).
 Clearly  \(\{A_k\}_{k\in\N}\) and \(\{A_k\}_{k\in\N}\)
 are decreasing and their infimum is their existing limit. Thus

 \begin{eqnarray*}
   \limsup_{n\to \infty} (a_n + b_n)
   &=& \inf_{n\in\N}\, \sup_{i\geq n} a_i+b_i \\
   &\leq& \inf_{k\in\N}\,   \left(\sup_{n\geq k} a_i
                          +       \sup_{n\geq k} b_i\right) \\
   &=& \inf_{k\in\N} A_k + B_k \\
   &=& \lim_{k\in\N} A_k + B_k \\
   &=& \lim_{k\in\N} A_k + \lim_{k\in\N} B_k \\
   &=& \inf_{k\in\N} A_k + \inf_{k\in\N} B_k \\
   &=& \limsup_{n\in\N} a_n + \limsup_{n\in\N} a_n
 \end{eqnarray*}

  Let \(a_n = (-1)^n n\) and \(b_n = (-1)^{n+1} n\) and
  so
  \(a_n+b_n = ((-1)^n + (-1)^{n+1}) n = 0\)
  But clearly \(\limsup a_n = \limsup b_n = \infty\).

 \item[(c)]
  Define the monotonic increasing sequences:
  \begin{eqnarray*}
  A_k &=& \inf_{n\geq k} a_n  \\
  B_k &=& \inf_{n\geq k} b_n  \\
  \end{eqnarray*}
  and so
  \begin{equation*}
  \liminf_{n\to\infty} a_n = \sup A_k = \lim A_k
      \leq \lim B_k = \sup B_k = \liminf_{n\to\infty} b_n\,.
  \end{equation*}

\end{itemize}

%%%%%%%%%%%%%%
\begin{excopy}
Prove that the set of points at which a sequence of measurable
real functions converges is a measurable set.
\end{excopy}

Let \(\{f_n\}_{n\in\N}\) be a sequence of measurable real functions on $X$.
Let \(\overline{f} = \limsup f_n\)
and \(\underline{f} = \liminf f_n\).
For any \(r\in \R\),
\begin{equation*}
E_r = \overline{f}^{-1}([r,\infty])
 = \bigcap_{n\in\N} \bigcup_{k\geq n} f_k^{-1}([r,\infty]).
\end{equation*}
Thus \(\overline{f}\) is measurable and similarly, so is \(\underline{f}\).
Now \(d = \overline{f} - \underline{f}\) is also measurable.
Surely
\begin{equation*}
E_0 = X \setminus d^{-1}\left([-\infty,0)\cup(0,\infty]\right) = d^{-1}(0)
\end{equation*}
is measurable. But \(E_0\) is exactly the set where
\(\overline{f}\) and \(\underline{f}\) agree
which is the set of point where \(\{f_n\}_{n\in\N}\) converge.


%%%%%%%%%%%%%%
\begin{excopy}
Let $X$ be an uncountable set, let \M\ be the collection of
all sets \(E\subset X\) such that either $E$ or \(E^c\)
is at most countable,
and define \(\mu(E)=0\) in the first case,
\(\mu(E)=1\) in the second.
Prove that \(\mu\) is a \salgebra\ in $X$ and that \(\mu\) is a measure on \M.
\end{excopy}

Clearly \(\emptyset, X\in \M\) and
\(\mu(\emptyset) = 0\)
and \(\mu(X) = 1\).
By definition, \M\ is closed under complement operation.
Now let \(\{A_n\}_{n\in\N}\) be a countable family in \M.
If some \(A_i\) is uncountable, then so is \(U = \cup A_i\in \M\)
and the \(\mu(U) = 1\).
Otherwise, since \(\aleph_0 \times \aleph_0 = \aleph_0\),
then \(|U| = |\cup A_i| = \aleph_0\) and so \(U\in M\) with \(\mu(U) = 0\).
Thus \M\ is a \salgebra.


%%%%%%%%%%%%%%
\begin{excopy}
Suppose \(f_n:X\to[0,\infty]\) is measurable for \(n=1,2,3,\ldots\),
\(f_1 \geq f_2 \geq f_3 \geq \cdots \geq 0\),
\(f_n(x)\to f(x)\) as \(n\to\infty\), for every \(x\in X\),
and \(f_1\in L^1(\mu)\).
Prove that then
\begin{equation*}
 \lim_{n\to\infty}\int_X f_n d\mu = \int_X fd\mu
\end{equation*}
and show that this conclusion does \emph{not} follow if the condition
``\(f_1\in L^1(\mu)\)'' is omitted.
\end{excopy}

\index{Lebesgue}
This is an application of Lebesgue's Dominated Convergence Theorem,
with \(g(x) = f_1(x)\) serving as the dominating function.

Say the condition \(f_1\in L^1(\mu)\)'' is omitted.
Let, \(X = [0,1]\) with the natural measure $m$, and for \(n>=1\), let
\begin{equation*}
f_n(x) = \left\{\begin{array}{l@{\qquad}l}
                \infty &  0\leq x<1/n \\
                0      &  1/n \leq x \leq 1
                \end{array}\right.
\end{equation*}
Now we easily see that \(f_n\to 0\) and so
\begin{equation*}
\int_{[0,1]} \lim_{n\to\infty} f_n(x)dm
= 0 < \infty
= \lim_{n\to\infty} \int_{[0,1]} f_n(x)dm.
\end{equation*}

%%%%%%%%%%%%%%
\begin{excopy}
Put \(f_n = \chhi_E\) if $N$ is odd, \(f_n = 1 - \chhi_E\) if $n$ is even.
What is the relevance of this example
\index{Fatou's lemma}
to Fatou's lemma.
\end{excopy}

In this case, the inequality of Fatou's Lemma becomes strict.
Say the measure is $m$ on $X$.
\begin{equation*}
\int_X \liminf_{n\to\infty} f_n(x)dm
= 0 < \min(\mu(E),\mu(X\setminus E))
= \liminf_{n\to\infty} \int_X f_n(x)dm.
\end{equation*}

%%%%%%%%%%%%%%
\begin{excopy}
Suppose \(\mu\) is a positive measure on $X$, \(f:X\to[0,\infty]\)
is measurable, \(\int_X fd\mu = c\), where \(0<c<\infty\),
and \(\alpha\) is a constant. Prove that
\begin{equation*}
\lim_{n\to\infty} \int_X n\log \left[ 1 + (f/n)^\alpha\right]d\mu
 =\left\{\begin{array}{l@{\qquad}l}
         \infty & \textrm{if }\; 0<\alpha<1,\\
         c      & \textrm{if }\; \alpha = 1,\\
         0      & \textrm{if }\; 1 < \alpha < \infty.
         \end{array}
         \right.
\end{equation*}


\emph{Hint}: If \(\alpha\geq 1\), then the integrands are dominated by
\(\alpha f\).
If \(\alpha<1\),
\index{Fatou's lemma}
Fatou's lemma  can be applied.
\end{excopy}

First we need to show some inequalities that
will establish the hint.
\begin{llem} \label{llem:apbp:geq}
If \(a \geq b \geq 0\) and \(a,\alpha\geq 1\), then
\(a^\alpha - b^\alpha \geq a - b\).
\end{llem}

\begin{thmproof}
Since \(a\geq b\) and \(\alpha-1\geq 0\) we have
\(a^{\alpha-1} \geq b^{\alpha-1}\) and so
\begin{equation*}
a^{\alpha-1} - 1 \geq b^{\alpha-1} - 1
\end{equation*}
The left hand side must be non negative.
We look at the sign of the right side of the last inequality.
In either case, we can deduce:
\begin{equation*}
a (a^{\alpha-1} - 1) \geq b(b^{\alpha-1} - 1).
\end{equation*}
The above is equivalent to
\begin{equation*}
a^\alpha - b^\alpha \geq a - b.
\end{equation*}
\end{thmproof}


\begin{llem} \label{llem:nlogx:leq}
Let \(n\in\N\)
and let \(x,\alpha\in\R\) such that \(x\geq 0\) and \(\alpha\geq 1\)
then
\begin{equation}
n\log\left(1 + \left(\frac{x}{n}\right)^\alpha\right) \leq \alpha x.
\end{equation}
\end{llem}

\begin{thmproof}
We assume \(n\in\N\) and the requirements of the lemma hold for
\(x,\alpha\in\R\).
Let
\begin{displaymath}
g(t) = e^t - t.
\end{displaymath}
Clearly, \(g(0)=1\) and for \(t\geq 0\)
we have \(g'(t) = e^t - 1 \geq 0\) and thus
\begin{equation} \label{eq:etmtg1}
e^t - t \geq 1 \qquad \textrm{for}\, t\geq 0.
\end{equation}
\begin{displaymath}
f(x) = e^{\alpha x/n} - x/n.
\end{displaymath}
If \(\alpha=1\) then from (\ref{eq:etmtg1}, we have \(f(x)\geq 1\)
for \(x\geq 0\).

Using Lemma~\ref{llem:apbp:geq}
with \(a=e^{x/n}\) and \(b=x/n\),
we see that
\begin{equation*}
e^{\alpha x/n} - (x/n)^\alpha \geq e^{x/n} - x/n \geq 1.
\end{equation*}

Hence
\begin{equation*}
1 + (x/n)^\alpha \leq e^{\alpha x/n}.
\end{equation*}
Equivalently,
\begin{equation*}
\log\left(1 + (x/n)^\alpha\right) \leq \alpha x/n,
\end{equation*}
that is,
\begin{equation*}
n\log\left(1 + (x/n)^\alpha\right) \leq \alpha x.
\end{equation*}
\end{thmproof}

% Hey, I found a hand-written (Hebrew) workout I made about 20 years ago!
% Here is a sort of edited copy.

Put
\begin{equation*}
f_n = n\log\left[ 1 + (f/n)^\alpha\right]
    = n^{1-\alpha}
      \log\left[\left(1 + f^\alpha/n^\alpha\right)^{n^\alpha}\right]
\end{equation*}
Note that if \(f(x)=0\) then \(f_n(x)=0\) and thus
\begin{equation*}
X' = f^{-1}\{(0,\infty]\}\subset X
\end{equation*}
and for every $n$ we have \(\int_X f_n = \int_{X'} f_n\)
and we may assume that \(f,f_n > 0\).

Now
\begin{eqnarray*}
\lim_{n\to\infty} f_n
 &=& \left(\lim_{n\to\infty} n^{1-\alpha}\right) \cdot
     \log\left(\lim_{n\to\infty}
         \left(1 + f^\alpha/n^\alpha\right)^{n^\alpha}\right) \\
 &=& f^\alpha  \lim_{n\to\infty} n^{1-\alpha} \\
 &=&  \left\{\begin{array}{l@{\qquad}l}
         \infty & \textrm{if }\; 0<\alpha<1,\\
         f      & \textrm{if }\; \alpha = 1,\\
         0      & \textrm{if }\; 1 < \alpha < \infty.
         \end{array}
      \right.
\end{eqnarray*}

Assume \(\alpha<1\).
\index{Fatou's lemma}
From Fatou's lemma,
\begin{equation*}
\lim_n \int f_n \geq \int \lim_n f_n = \infty.
\end{equation*}

Assume \(\alpha\geq 1\).
From local lemma~\ref{llem:nlogx:leq} we have \(f_n\leq \alpha f\).
\index{Lebesgue's!Dominated Convergence Theorem}
\index{Dominated Convergence Theorem}
Using Lebesgue's Dominated Convergence Theorem,
\begin{equation*}
\lim_n\int_X f_n\, d\mu = \int \lim_n f_n\,d\mu
 = \left\{\begin{array}{l@{\qquad}l}
         \int_X f = c     & \textrm{if }\; \alpha = 1,\\
         \int_X 0 = 0     & \textrm{if }\; 1 < \alpha
         \end{array}
  \right.
\end{equation*}




%% This is a newer trial, not remembering  the good old
\iffalse
First we need to show some inequalities that
will establish the hint.
\begin{llem} \label{llem:apbp:geq}
If \(a \geq b \geq 0\) and \(a,\alpha\geq 1\), then
\(a^\alpha - b^\alpha \geq a - b\).
\end{llem}

\begin{thmproof}
Since \(a\geq b\) and \(\alpha-1\geq 0\) we have
\(a^{\alpha-1} \geq b^{\alpha-1}\) and so
\begin{equation*}
a^{\alpha-1} - 1 \geq b^{\alpha-1} - 1
\end{equation*}
The left hand side must be non negative.
We look at the sign of the right side of the last inequality.
In either case, we can deduce:
\begin{equation*}
a (a^{\alpha-1} - 1) \geq b(b^{\alpha-1} - 1).
\end{equation*}
The above is equivalent to
\begin{equation*}
a^\alpha - b^\alpha \geq a - b.
\end{equation*}
\end{thmproof}


\begin{llem} \label{llem:nlogx:leq}
Let \(n\in\N\)
and let \(x,\alpha\in\R\) such that \(x\geq 0\) and \(\alpha\geq 1\)
then
\begin{equation}
n\log\left(1 + \left(\frac{x}{n}\right)^\alpha\right) \leq \alpha x.
\end{equation}
\end{llem}

\begin{thmproof}
We assume \(n\in\N\) and the requirements of the lemma hold for
\(x,\alpha\in\R\).
Let
\begin{displaymath}
g(t) = e^t - t.
\end{displaymath}
Clearly, \(g(0)=1\) and for \(t\geq 0\)
we have \(g'(t) = e^t - 1 \geq 0\) and thus
\begin{equation} \label{eq:etmtg1}
e^t - t \geq 1 \qquad \textrm{for}\, t\geq 0.
\end{equation}
\begin{displaymath}
f(x) = e^{\alpha x/n} - x/n.
\end{displaymath}
If \(\alpha=1\) then from (\ref{eq:etmtg1}, we have \(f(x)\geq 1\)
for \(x\geq 0\).

Using Lemma~\ref{llem:apbp:geq}
with \(a=e^{x/n}\) and \(b=x/n\),
we see that
\begin{equation*}
e^{\alpha x/n} - (x/n)^\alpha \geq e^{x/n} - x/n \geq 1.
\end{equation*}

Hence
\begin{equation*}
1 + (x/n)^\alpha \leq e^{\alpha x/n}.
\end{equation*}
Equivalently,
\begin{equation*}
\log\left(1 + (x/n)^\alpha\right) \leq \alpha x/n,
\end{equation*}
that is,
\begin{equation*}
n\log\left(1 + (x/n)^\alpha\right) \leq \alpha x.
\end{equation*}
\end{thmproof}


Back to the exercise.
We have three cases:
\begin{itemize}

 \item \(\alpha=1\)

 \begin{eqnarray*}
  \lim_{n\to\infty} n\log (1 + f/n)
   &=& \lim_{n\to\infty} \log \left((1 + f/n)^n\right) \\
   &=& \log \left(\lim_{n\to\infty} (1 + f/n)^n\right) \\
   &=& \log e^f = f
 \end{eqnarray*}
 Thus, using Lebesgue's dominated convergence theorem
 we have
 \begin{equation*}
 \lim_{n\to\infty} \int_X n\log \left(1 + f/n\right)d\mu
  = \int_X f\,d\mu = c.
 \end{equation*}

 \item \(\alpha<1\)

 \begin{eqnarray*}
      \left(1 + (x/n)^\alpha\right)^n
 &=& (1 + n^{1-\alpha}x^\alpha/n)^n \\
 &=& \sum_{k=0}^n \binom{n}{k}\left(n^{1-\alpha}x^\alpha/n\right)^k \\
 &\geq& \sum_{k=1} \cdots \\
 &\geq& n(n^{1-\alpha}x^\alpha/n) \\
 &=& n^{1-\alpha}x^\alpha.
 \end{eqnarray*}

 Thus
 \begin{equation*}
 \lim_{n\to\infty} n\log(1 + (x/n)^\alpha)
 = \lim_{n\to\infty} \log\left((1 + (x/n)^\alpha)^n\right) = \infty.
 \end{equation*}

 Again, using Lebesgue's dominated convergence theorem
 we have:
 \begin{equation*}
 \lim_{n\to\infty} \int_X n\log \left(1 + (f/n)^\alpha\right)d\mu \\
 = \int_X
      \left(\lim_{n\to\infty} n\log \left(1+(f/n)^\alpha\right)\right)d\mu \\
 = \infty.
 \end{equation*}

 \item \(\alpha>1\)

\end{itemize}
\fi % and of clumsy new trial


%%%%%%%%%%%%%%
\begin{excopy} % 10
Suppose \(\mu(X)<\infty\), \(\{f_n\}\) is a sequence of bounded complex
measurable functions on $X$, and \(f_n\to f\) uniformly on $X$. Prove that
\begin{equation*}
\lim_{n\to\infty} \int_X f_n d\mu = \int_X fd\mu,
\end{equation*}
and show that the hypothesis ``\(\mu(X)<\infty\)'' cannot be omitted.
\end{excopy}

Taking \(\epsilon = 1\), there is \(M_1>0\) such that
for any \(n\geq M_1\), we have \(|f_n(x) - f(x)| < M_1\) for all \(x\in X\).
Thus \(|f| < |f_n| + M_1\). In particular, $f$ is bounded.
Also the constant function \(g(x) = \|f\| + M_1\) dominates \(\{f_n\}\).
Applying Lebesgue's dominated convergence theorem gives the desired result

%%%%%%%%%%%%%%
\begin{excopy} % 11
Show that
\begin{equation*}
 A = \bigcap_{n=1}^\infty \bigcup_{k=n}^\infty E_k
\end{equation*}
in Theorem~1.41, and hence provided th theorem without any reference
to integration.
\end{excopy}

The set $A$ is defined as the set of all \(x\in X\)
that belong to infinitely many \(E_k\).
If \(x\in X\) then \(x \in U_k = \cup_{k=n}^\infty E_k\) for any \(n>0\).
Conversely, if \(x\notin A\) then there must be some \(N>0\)
such that \(x\notin E_k\) for all \(k\geq N\).
Clearly, \(x\notin U_N\) and the set equality is shown.

Now, let's quote the Theorem:
\begin{quotation}
\setcounter{quotethm}{40} % to get 41
  \begin{quotethm}
   Let \(\{E_k\}\) be a sequence of measurable sets in $X$, such that
   \begin{equation} \label{eq:thm41}
        \sum_{k=1}^\infty \mu(E_k) < \infty.
   \end{equation}
   Then almost all \(x\in X\) lie in at most finitely many of the sets \(E_k\).
  \end{quotethm}
  In view of this exercise, we have to show that \(\mu(A) = 0\).
  But the fact that the series in (\ref{eq:thm41}) conversges
  means that for any \(\epsilon>0\) there is an $N$ such that
  \(\sum_{k\geq N} \mu(E_k) < \epsilon\) and so
  \begin{equation*}
    \mu(A)
    = \mu \left(\bigcap_{n=1}^\infty \bigcup_{k=n}^\infty E_k\right)
    \leq \mu \left(\bigcup_{k=N}^\infty E_k\right) < \epsilon.
  \end{equation*}
  Thus \(\mu(A) = 0\).
\end{quotation}



%%%%%%%%%%%%%%
\begin{excopy}
Suppose \(f\in L^1(\mu)\). Prove that to each \(\epsilon\)
there exists a \(\delta > 0\) such that
\(\int_E |f|d\mu < \epsilon\) whenever \(\mu(E) < \delta\).
\end{excopy}

Let $X$ be the space on which \(\mu\) is defined.
\begin{equation*}
E_n = \{x\in X: n - 1 \leq |f(x)| < n
 \qquad \textrm{for } n\geq 1.
\end{equation*}
Clearly
\begin{equation*}
X = \Disjunion_{n=1}^\infty E_n\,.
\end{equation*}
Now,
\begin{equation} \label{eq:intf:sum:Ek}
\int_X |f|d\mu = \sum_{n=1}^\infty \int_{E_n} |f|d\mu < \infty
\end{equation}

Let \(\epsilon > 0\). By (\ref{eq:intf:sum:Ek}), there is
\(N>0\) such that
\begin{equation}
\sum_{n=N}^\infty \int_{E_n} |f|d\mu < \epsilon/2.
\end{equation}
Put \(H = \cup_{i=1}^{N-1} E_i\)
and \(T = \cup_{i=N}^\infty E_i\).
Note that \(X = H \disjunion T\).

Take \(\delta = \epsilon/2N\).
Now if $E$ is \(\mu\) measurable, and \(\mu(E)<\delta\) then
\begin{eqnarray*}
\int_E |f|d\mu
&=& \int_{E\cap H} |f|d\mu + \int_{E\cap T} |f|d\mu \\
&\leq& N\mu(E\cap H) + \int_{T} |f|d\mu \\
&\leq& N\mu(E) + \int_T |f|d\mu \\
&=& \epsilon/2  + \epsilon/2 = \epsilon.
\end{eqnarray*}


%%%%%%%%%%%%%%%
\end{enumerate}
%%%%%%%%%%%%%%%

 % -*- latex -*-

%%%%%%%%%%%%%%%%%%%%%%%%%%%%%%%%%%%%%%%%%%%%%%%%%%%%%%%%%%%%%%%%%%%%%%%%
%%%%%%%%%%%%%%%%%%%%%%%%%%%%%%%%%%%%%%%%%%%%%%%%%%%%%%%%%%%%%%%%%%%%%%%%
%%%%%%%%%%%%%%%%%%%%%%%%%%%%%%%%%%%%%%%%%%%%%%%%%%%%%%%%%%%%%%%%%%%%%%%%
\chapterTypeout{Positive Borel Measures}

%%%%%%%%%%%%%%%%%%%%%%%%%%%%%%%%%%%%%%%%%%%%%%%%%%%%%%%%%%%%%%%%%%%%%%%%
%%%%%%%%%%%%%%%%%%%%%%%%%%%%%%%%%%%%%%%%%%%%%%%%%%%%%%%%%%%%%%%%%%%%%%%%
\section{Notes}

While working on Exercise~\ref{ex:2:10}, I worked out the following
lemma, that eventually were not used. Here they are, so they
will not be lost.

\begin{llem} \label{llem:interval:subsum}
Assume the disjoint union
\begin{equation*}
 \Disjunion_{i\in\N} I_i \subset [0,1]
\end{equation*}
where \(I_i\) are intervals of the form
\((a_i,b_i)\),
\((a_i,b_i]\),
\([a_i,b_i)\) or
\([a_i,b_i]\). If
\begin{equation*}
 \sum_{i\in\N} m(I_i) =  \sum_{i\in\N} (b_i - a_i) < 1
\end{equation*}
then there exists \(u\in[0,1]\) such that for any open interval $J$,
where \(u\in J\subset [0,1]\), the following inequality
\begin{equation*}
 \sum_{i\in\N} m(I_i \cap J) < m(J)
\end{equation*}
holds.
\end{llem}

\begin{thmproof}
Let \(U = \disjunion_{i\in\N}\).
We will construct a decreasing sequence of clsoed intervals \(\{K_i\}_{i\in\N}\)
such that for all \(i\in\N\)
\begin{itemize}
 \item \(K_{i+i} \subset K_i\)
 \item \(m(K_i) = 2^{-i}\)
 \item \(\sum_{i\in\N} m(I_i\cap K_i) < m(K_i) = 2^{-i}\).
\end{itemize}
Put \(K_0 = [0,1]\). Now by induction, assume \(K_i = [\alpha,\beta]\)
is defined and for which satisfies the above requirements.
Put \(\gamma = (\alpha+\beta)/2\), consider the two subintervals
\(L=[\alpha,\gamma]\) and
\(R=[\gamma,\beta]\). The first two requirements hold for \(K_{i+1}\)
and at least one of them satisfies the third, and we pick it as \(K_{i+1}\).
Now let \(u = \cap_{i\in\N} K_i\) (identifying a singleton with the element),
such $u$ exists as an intersection of non empty compact sets (and also is
unique).

Now assume \(u\in J = (a,b)\) an open interval.
Then there is some (sufficiently small) \(K_j \subset J\).
Using \(m(J) = m(K_j) + m(J\setminus K_j)\) we have
\begin{eqnarray*}
 \sum_{i\in\N} m(I_i \cap J)
 &=& \sum_{i\in\N} m\left(I_i \cap (K_j \disjunion (J\setminus K_j)\right) \\
 &=& \sum_{i\in\N} {   m(I_i \cap K_j)
                     + m\left(I_i \cap (J\setminus K_j)\right)} \\
 &=& \sum_{i\in\N} m(I_i \cap K_j) +
     \sum_{i\in\N} m\left(I_i \cap (J\setminus K_j)\right) \\
 &<&     m(K_j) + \sum_{i\in\N} m\left(I_i \cap (J\setminus K_j)\right) \\
 &\leq&  m(K_j) + m(J\setminus K_j) \\
 &=& m(J).
\end{eqnarray*}
\end{thmproof}


\begin{llem} \label{llem:sumintervals:}
Assume the unit interval is a disjoint union
\begin{equation*}
 [0,1] = \Disjunion_{i\in\N} I_i
\end{equation*}
where \(I_i\) are intervals of the form
\((a_i,b_i)\),
\((a_i,b_i]\),
\([a_i,b_i)\) or
\([a_i,b_i]\). Then
\begin{equation*}
 \sum_{i\in\N} m(I_i) =  \sum_{i\in\N} (b_i - a_i)= 1.
\end{equation*}
\end{llem}

\begin{thmproof}
For any finite sub-sum \(\sum_{i=1}^N m(I_i) \leq 1\), hence
\(\sum_{i\in\N} m(I_i)\leq 1\).
By negation, we assume \(\sum_{i\in\N} m(I_i)< 1\).
Now the assumptions of the previous lemma~\ref{llem:interval:subsum} hold
giving \(u\in[0,1]\) such that for any open interval \(J\subset[0,1]\) we have
\(\sum_{i\in\N} m(I_i \cap J) < m(J)\).
By assumptions there must be (a unique) $j$ such that \(u\in I_j\).
Now there are two cases.
\begin{itemize}
 \item[(\emph{i})]
   (Internal) $u$ is inetrnal point of \(I_j\), then
   we can pick an open interval $J$ such that \(u\in J \subset I_j\).
   Note, that here we use the induced topology, so $u$ may
   also be $0$ or $1$.
   For the sake of unifying with the proof continuation, we denote
   a dummy \(j' = j + 1\).
 \item[(\emph{ii})]
   (Boundary) \(u\in\partial I_j\) that is (\(I_j\) is not open and)
   \(u=a_j\in I_j\) or \(u=b_j\in I_j\).  In that case there must be
   \(j'\), the index of the neighbor interval, such that
   \(u\in\partial I_{j'}\).  \(L=I_j\disjunion I_{j'}\) is an
   interval.  Now we can pick an open interval $J$ such that
   \(u\in J\subset L=I_j\disjunion I_{j'}\) (where $L$ is an interval).
\end{itemize}
In both cases \(J \subset I_j \disjunion I_{j'}\).
Now by lemma~\ref{llem:interval:subsum} \(\sum_{i\in\N} m(I_i\cap J) < m(J)\),
but
\begin{equation*}
 \sum_{i\in\N} m(I_i\cap J)
 \geq   m(I_j \cap J) +  m(I_{j'} \cap J) \\
 =     m\left( (I_j \disjunion I_{j'}) \cap J\right)  = m(J)
\end{equation*}
which is a contradiction.
\end{thmproof}


Let us generalize 
\index{Lusin}
Lusin theorem~2.24.
\begin{llem}
Let \(f:\R\to\C\) be a measurable function.
For each \(\epsilon>0\) there exists a continuous function \(g:\R\to\C\)
such that 
\begin{align*}
m\bigl(\{x\in\R: f(x)\neq g(x)\}\bigr) &< \epsilon\\
\forall x\in\R,\quad |g(x)| &\leq |f(x)|\,.
\end{align*}
\end{llem}
Note that the lemma does \emph{not} imply \emph{uniform} continuity of $g$.
\\
\begin{thmproof}
Pick an \(\epsilon>0\), \wlogy\ \(\epsilon<1/2\).
For each \(n\in\Z\) consider the restriction
\(f_n:[n,n+2]\to\C\) of $f$ to \([n,n+2]\) 
(actually, \(f_n = f_{\restriction[n,n+2]}\)).
By Lusin theorem~2.24 we can find \(g_n:[n,n+2]\to\C\)
such that 
\begin{align*}
m\bigl(\{x\in[n,n+2]: f(x)\neq g_n(x)\}\bigr) &< \epsilon_n = 2^{|n|+2}\epsilon\\
\forall x\in[n.n+2],\quad |g_n(x)| &\leq |f(x)|\,.
\end{align*}
We will connect \(g_n\) to define $g$.
For every $n$ pick \(t_n\in(n, n+1)\) such that 
\begin{equation*}
g_{n-1}(t_n) = g_n(t_n) = f_n(t_n) = f(t_n).
\end{equation*}
The existence of \(t_n\) is ensured by \(\epsilon_n < 1/2\)
and thus \(g_n\) and \(g_{n+1}\) could differ from $f$ in \([n,n+1]\)
ona set of measure  at most less than $1$.
For each \(x\in\R\) there is a unique \(n\in\Z\) such that 
\(t_n \leq t_{n+1}\),  and we define \(g(x) = g_n(x)\).
Clearly $g$ is continuous and differs from $f$ on a set whose
measure is at most
\begin{equation*}
\sum_{n\in\Z}\epsilon_n = \epsilon \sum_{n\in\Z}  2^{|n|+2} < \epsilon\,.
\end{equation*}
\end{thmproof}

%%%%%%%%%%%%%%%%%%%%%%%%%%%%%%%%%%%%%%%%%%%%%%%%%%%%%%%%%%%%%%%%%%%%%%%%
%%%%%%%%%%%%%%%%%%%%%%%%%%%%%%%%%%%%%%%%%%%%%%%%%%%%%%%%%%%%%%%%%%%%%%%%
\section{Exercises Support}

%%%%%%%%%%%%%%%%%%%%%%%%%%%%%%%%%%%%%%%%%%%%%%%%%%%%%%%%%%%%%%%%%%%%%%%%
\subsection{Topology}

We need to establish some more set-theoretic topological results.
Using the
\index{Stone-Cech compactification@Stone-\Cech\ compactification}
Stone-\Cech\ compactification (\cite{Dug1966}, \textsf{XI 8.3}).

\paragraph{Definition} (\cite{Dug1966}, \textsf{VII 7.1}).
A Hausdorff space $X$ is
\index{completely regular}
\emph{completely regular}
\index{Tychonoff}
(or Tychonoff)
if for each point \(p\in X\) and a closed \(A\subset X\) such that \(p\notin A\)
there is a continuous \(f:X\rightarrow I=[0,1]\) such that \(f(p) = 1\)
and \(\forall x\in A, f(x)=0\).

\paragraph{Definition} (\cite{Dug1966}, \textsf{VII~7.1}).
Let $X$ be a completely regular topological space.
Let \(P=C(X,I)\) be the set of continuous functions \(f:X\rightarrow I\)
where \(I=[0,1]\) the unit interval. The product space \(I^P\)
is compact by the Tychonoff theorem  (\cite{Dug1966}, \textsf{XI~1.4}).
We define the map
\begin{eqnarray} \label{eq:stonecech:rho}
 \rho: X & \rightarrow & I^P \\
 \rho(x) &=& (\lambda(x))_{\lambda \in C(X,I)} \notag
\end{eqnarray}
The Stone-\Cech\ compactification is
\begin{equation*}
 \beta(X) = \overline{\rho(X)},
\end{equation*}
where the closure is on \(I^P\).

\begin{llem} \label{llem:stonecech:omega1}
The Stone-\Cech\ compactification of \([0,\omega_1)\) is homeomorphic
to \([0,\omega_1]\).
\end{llem}
\begin{thmproof}
For each \(f\in C([0,\omega_1],I\), its restriction
\begin{equation*}
 f_{|[0,\omega_1)} \in C([0,\omega_1),I).
\end{equation*}
From lemma~\ref{llem:Vickery} every \(g \in C([0,\omega_1),I)\),
can be extended to \(\overline{f}\in C([0,\omega_1],I)\)
by defining \(\overline{f}(\omega_1)\) as the tail value.
Thus,
\begin{equation*}
 C([0,\omega_1),I) \cong C([0,\omega_1],I)
\end{equation*}
By using the notations of (\ref{eq:stonecech:rho}),
\begin{equation*}
 \rho\left(C([0,\omega_1),I)\right) \subset
 \rho\left(C([0,\omega_1],I)\right)
\end{equation*}
Since \([0,\omega_1]\) is compact, it image is compact
and so equals to its closure. We now have:
\begin{equation*}
 \rho\left(C([0,\omega_1),I)\right) \subset
 \overline{\rho\left(C([0,\omega_1),I)\right)} \subset
 \overline{\rho\left(C([0,\omega_1],I)\right)} =
 \rho\left(C([0,\omega_1],I)\right)
\end{equation*}
\end{thmproof}


\begin{llem}
Let \(K_1,K_2\subset [0,\omega_1]\) be uncountable compact.
Then \(K_1\cap K_2\) is uncountable compact.
\end{llem}
\begin{thmproof}
Clearly \(K = K_1\cap K_2\) is compact.
By negation assume $K$ it is countable.
For \(i=1,2\), let \(H_i = K_i\cap [0,\omega_1)\)
(note that \(\omega_1\in K_i\)),
% both \(H_i\) are uncountable and compact in the inherited topology of
and \(H=K\setminus \{\omega_1\}\).
For $H$, there exists by lemma~\ref{llem:countable:ub}
an upper bound \(b\in[0,\omega_1)\).
Since \(H_i\cap[0,b)\) is countable, we define
in the space \([0,\omega_1)\)
new compact (in the inherited topology) subsets
\begin{equation*}
 L+i = H_i \cap [b,\omega_1) = K_i \cap [b,\omega_1)
 \qquad \textrm{for}\; i=1,2.
\end{equation*}

\end{thmproof}


\iffalse
%%% NOT TRUE!!  only if Y closed.
\begin{llem}
Let $X$ be a topological space, \(Y\subset X\) a subspace
with the topology inherited from $X$.
If \(K\subset X\) compact then \(K\cap Y\) is compact in $Y$
\end{llem}
\begin{thmproof}
Let \(\{V_i\}_{i\in I}\) be an open cover in $Y$ of \(K\cap Y\).
By the definition of the inherited topology, there are open sets
 \(\{V_i\}_{i\in I}\) in $X$ such that \(U_i = V_i \cap Y\) (for \(i\in I\).
For each \(i\in I\) we define \(W_i = V_i
\end{thmproof}
\fi


%%%%%%%%%%%%%%%%%%%%%%%%%%%%%%%%%%%%%%%%%%%%%%%%%%%%%%%%%%%%%%%%%%%%%%%%
%%%%%%%%%%%%%%%%%%%%%%%%%%%%%%%%%%%%%%%%%%%%%%%%%%%%%%%%%%%%%%%%%%%%%%%%
\section{Exercises} % pages 58-61

%%%%%%%%%%%%%%%%%
\begin{enumerate}
%%%%%%%%%%%%%%%%%

%%%%%%%%%%%%%%
\begin{excopy}
Let \(\{f_n\}\) be a sequence of real nonnegative functions on \(\R^1\),
and consider the following four statements:
\begin{itemize}
 \itemch{a}
   If \(f_1\) and \(f_2\) are upper semicontinuous,
   then \(f_1 + f_2\) is upper semicontinuous.
 \itemch{b}
   If \(f_1\) and \(f_2\) are lower semicontinuous,
   then \(f_1 + f_2\) is lower semicontinuous.
 \itemch{c}
   If each \(f_n\) is upper semicontinuous, then \(\sum_1^\infty f_n\)
   is upper semicontinuous.
 \itemch{d}
   If each \(f_n\) is lower semicontinuous, then \(\sum_1^\infty f_n\)
   is lower semicontinuous.
\end{itemize}
Show that three of these atr true and one is false.
What happens if the word ``nonnegative'' is omitted?
Is the truth of the statements affected if \(\R^1\) is replaced
by a general topological space?
\end{excopy}

We will show that only \ich{c} is false.

\begin{itemize}
 \itemch{a}
  True, for
  \begin{equation*}
  \{x\in\R: f_1(x)+f_2(x) < a\}
  = \bigcup_{\alpha\in\R} \left(\{x\in\R: f_1(x)< \alpha\} \cap
                             \{x\in\R: f_2(x) < a-\alpha\} \right).
  \end{equation*}
  and since the intersection of two open sets is open, the above
  set is open as an infinite union of open sets.

 \itemch{b}
  True, for
  \begin{equation*}
  \{x\in\R: f_1(x)+f_2(x) > a\}
  = \bigcup_{\alpha\in\R} \left(\{x\in\R: f_1(x)> \alpha\} \cap
                             \{x\in\R: f_2(x) > a-\alpha\} \right).
  \end{equation*}
  and since the intersection of two open sets is open, the above
  set is open as an infinite union of open sets.
 \itemch{c}
 False. Let's construct the following sequence of upper semicontinuous
 functions. For \(n\geq 1\), let
 \begin{equation*}
  f_n(x) = \left\{\begin{array}{lc}
        1 & \qquad\textrm{if }\  1/(n+1) \leq x \leq 1/n \\
        0 & \textrm{Otherwise}
                  \end{array}\right..
 \end{equation*}
 Putting \(F = \sum f_n\).
 Now clearly \(\{x\in R: x < 1/2\} = \{0\}\) a singleton which is clearly
 \emph{not} open.

 \itemch{d}
 True.
 Using \ich{b} and induction, we see that  \(F_n = \sum_{k=1}^n f_k\)
 is lower semicontinuous. Now for any \(x\in\R\) such that
 \(\sum_{k=1}^\infty f_k(x) > a\) there exists $n$ such that \(F_n(x) > a\),
 and so
 \begin{equation*}
  \{x\in\R: \sum_{k=1}^\infty f_k(x)\ > a\} =
  \bigcup_n \{x\in\R: F_n(x)\ > a\}.
 \end{equation*}
 is open.
\end{itemize}


%%%%%%%%%%%%%%
\begin{excopy}
Let $f$ be an arbitrary complex function on \(\R^1\), end define
\begin{eqnarray*}
 \varphi(x,\delta) & = & \sup\{|f(s) - f(t)|: s,t\in (x-\delta, x+\delta)\}\\
 \varphi(x)        & = & \inf\{\varphi(x,\delta): \delta > 0\}.
\end{eqnarray*}
Prove that \(\varphi\) is upper semicontinuous, that $f$ is continuous
at a point $x$ if and only if \(\varphi(x)=0\), and hence that the set of
points of continuity of an arbitrary complex function is a \(G_\delta\).

Formulate and prove an analogous statement for general topological
space in place of \(\R^1\).
\end{excopy}

Assume \(\alpha \in \R\) and \(G=\varphi^{-1}(-\infty,\alpha)\).
Let \(b\in G\setminus\inter{G} = \) a boundary point.
Now for any \(\delta>0\), there exists
\(w\in (b-\delta/2,b+\delta/2)\) such that \(w\notin G\).
Hence there are \(x,y\in(w-\delta/2,w+\delta/2)\) such that
\(|f(x)-f(y)| \geq \alpha\). But \(x,y\in (b-\delta,b+\delta)\) as well,
and since \(\delta\) is arbitrary, \(\varphi(b)\geq \alpha\) and
so \(b\notin G\) hence $G$ is open showing that \(\varphi\) is
upper semicontinuous.

If $f$ is continuous at $w$, then for any
% (replacing classic roles of \(\epsilon\) and \(\delta\))
\(\epsilon>0\) there exists \(\delta>0\) such that
\(|f(w+h)-f(w)| < \epsilon/2\) whenever \(|h|<\delta\).
In such case, clearly for any \(x,y\in (w-\delta,w+\delta)\)
we have
\begin{equation*}
|f(x) - f(y)| \leq
|f(x) - f(w)| + |f(w) - f(y)| < \epsilon/2 + \epsilon/2 = \epsilon.
\end{equation*}
That \(\varphi(w) < \epsilon\) and since \(epsilon>0\) was arbitrary,
\(varphi(w)=0\).
Conversely, if \(varphi(w)=0\) then for any \(\epsilon>0\)
there exists \(\delta>0\) such that for any \(x,y\in (w-\delta,w+\delta)\)
we have \(|f(x)-f(y)| < \epsilon\). In particular,
 \(|f(x)-f(w)| < \epsilon\) for any \(x \in (w-\delta,w+\delta)\),
hence $f$ is continuous at $w$.

The set of points where an arbitrary function $f$ is continuous, is
 \(\cap_n \varphi^{-1}(-\infty, 1/n)\) and by what was just shown
this set is an intersection of countably many open sets, that is
a \(G_\delta\) type of set.

\paragraph{Generalization} Let \(X,T\) be a topological space
(points are in $X$ and $T$ is the family of opens sets in $X$).
Define
\begin{eqnarray*}
 \varphi(x,V) & = & \sup\{|f(s) - f(t)|: s,t\in V\}
                             \qquad \textrm{where}\; x\in V\in T\\
 \varphi(x)        & = & \inf\{\varphi(x,V): V\in T\}.
\end{eqnarray*}
The proof goes similarly, just replacing \(\delta\) and \(delta/2\)
with sub-neighborhoods.


%%%%%%%%%%%%%%
\begin{excopy}
Let $X$ be a metric space, with metric \(\rho\).
For any non empty \(E\subset X\), define
\begin{equation*}
 \rho_E(x) = \inf\{\rho(x,y): y\in E\}.
\end{equation*}
Show that \(\rho_E\) is a uniformly continuous function on $X$.
If $A$ and $B$ are disjoint nonempty closed subsets of $X$, examine
the relevance of the function
\begin{equation*}
  f(x) = \frac{\rho_A(x)}{\rho_A(x) +  \rho_B(x)}
\end{equation*}
\index{Urysohn's lemma}
to Urysohn's lemma.
\end{excopy}

Let \(x\in X\) and \(\emptyset \neq E \subset X\).
For any \(\epsilon>0\), there exists \(u\in E\) such that
\(\rho(x,u) < \rho_E(x) + \epsilon\).
Now
\begin{equation*}
\rho_E(y) \leq \rho(y,u)
          \leq \rho(x,y) + \rho(x,u)
             < \rho(x,y) + \rho_E(x) + \epsilon.
\end{equation*}
Hence \(\rho_E(y) - \rho_E(x) \leq \rho(x,y)\) and by symmetry,
\begin{equation*}
|\rho_E(y) - \rho_E(x)| \leq \rho(x,y).
\end{equation*}
From which continuity of \(\rho_E\) follows.

If a non empty \(A\subset X\) is closed and \(x\in X\setminus A\)
then \(\rho_A(x) > 0\), since otherwise \(x\in \overline{A}\).
Hence by \(A\cap B = \emptyset\), the denominator of $f$ satisfies
\(\rho_A(x)+\rho_B(x) > 0\) for any \(x\in X\) and so $f$ is
well defined and continuous.

Now this gives a simple algabraic construction of a function
as desired in Urysohn's Lemma. We can easily see
that \(f(x)=0\) for \(x\in A\) and
that \(f(x)=1\) for \(x\in B\).

%%%%%%%%%%%%%% 4
\begin{excopy}
Examine the proof of the
\index{Riesz theorem}
Riesz theorem and prove the following two statements:
\begin{itemize}
 \itemch{a}
   If \(E_1 \subset V_1\) and \(E_2 \subset V_2\), where \(V_1\) and \(V_2\)
   are disjoint open sets, then \(\mu(E_1\cup E_2) = \mu(E_1) + \mu(E_2)\),
   even if \(E_1\) and \(E_2\) are not in \frakM.
 \itemch{b}
   If \(E\in \frakM_F\) then
   \(E = N\cup K_1\cup K_2 \cup \cdots\), where \(\{K_i\}\)
   is a disjoint countable collection of compact sets and \(\mu(N) = 0\).
\end{itemize}
\end{excopy}


\begin{itemize}
 \itemch{a}
   By \textsc{step~i} of the proof of Riesz Theorem (\cite{RudinRCA80} page~44),
   we know that
   \begin{equation*}
   \mu(E_1\cup E_2) \leq \mu(E_1) + \mu(E_2).
   \end{equation*}
   For the opposite inequality, let \(\epsilon>0\) and pick
   some open set $V$ such that \(E_1\cup E_2\subset V\) and
   \(\mu(V) \leq \mu(E_1\cup E_2) + \epsilon\).
   We now have:
   \begin{eqnarray*}
   \mu(E_1) + \mu(E_2)
    & \leq & \mu(V\cap V_1) + \mu(V\cap V_2) \\
    & = & \mu\left(V\cap (V_1 \cup V_2)\right) \\
    & \leq & \mu(V) \leq \mu(E_1\cup E_2) + \epsilon.
   \end{eqnarray*}
   Since \(\epsilon\) was arbitrary, we have
   \begin{equation*}
   \mu(E_1) + \mu(E_2) \leq \mu(E_).
   \end{equation*}

 \itemch{b}
   By \textsc{step~viii} of the proof of Riesz Theorem
   (\cite{RudinRCA80} page~47), we know that \frakM\ contains all
   Borel sets with finite \(\mu\) measure.
   Now define by induction, \(K_1\) a compact set such that
   \(K_1\subset E\) and \(\mu(K_1) > \mu(E)/2\).

   If \(\{K_i\}_{1\leq i<n}\) are defined, we know that
   \(D_{n-1} = E\setminus \cup_{i<n} K_i \in \frakM_F\) as such
   we can define \(K_n\)
   as a compact set, such that \(K_n \subset D_{n-1}\)
   and \(\mu(K_n) > \mu(D_{n-1}E)/2\).
   We can easily see that
   \begin{equation*}
    \sum_{i=1}^n \mu(K_i) \geq \sum_{i=1}^n 1/2^n = 1 - 1/2^{n+1}.
   \end{equation*}
   Put \(N = E \setminus \cup_{i=1}^\infty K_i\)
   and since \(\mu(E) = \sum_{i=1}^\infty \mu(K_i)\),
   we conclude that \(\mu(N) = 0\).
\end{itemize}

\end{enumerate}

\index{Lebesgue}
In Exercises 5 to 8, $m$ stands for Lebesgue's measure on \(R^1\).
\nobreak
\begin{enumerate}

\setcounter{enumi}{4}

%%%%%%%%%%%%%% 5
\begin{excopy}
Let $E$ be
\index{Cantor}
Cantor's familiar ``middle thirds'' set.
Show that \(m(E) = 0\), even though $E$ and \(\R^1\) have the same cardinality.
\end{excopy}

After removing the thirds on the $n$ step, the set \(C_n\) ``remain'' with
\(m(C_N) = (2/3)^n\). By being in the \(\sigma\)-algebra,
the Cantor set $C$ has a measure
\begin{equation*}
m(C) =
\lim_{n\rightarrow\infty}m(C_n) = \lim_{n\rightarrow\infty}(2/3)^n = 0.
\end{equation*}

For compuing the cardinality, let's consider the ternary representation
of each \(\alpha\in [0,1]\),
\begin{equation*}
 \alpha = \sum_{i=1}^\infty t_i(\alpha) 3^{-i}
    \qquad \textrm{where}\; t_i(\alpha) \in \{0,1,2\}
\end{equation*}
Numbers, except for $0$,  with finite representation
(with \(t_i(\alpha)=0\) for all \(i > N\) for some $N$)
such as
\begin{equation*}
\alpha = \sum_{i=1}^N t_i(\alpha) 3^{-i}
\end{equation*}
where \(0\neq t_N(\alpha) \in \{1,2\}\) ---
also have an infinite representation as in:
\begin{equation*}
\sum_{i=1}^N t_i(\alpha) 3^{-i} =
\sum_{i=1}^{N-1} t_i(\alpha) 3^{-i}
+ (t_N(\alpha)-1) 3^{-N}
+ \sum_{i=N+1}^\infty 2\cdot 3^{-i}
\end{equation*}

In such cases,
we resolve the ambiguity, by choosing:
\begin{itemize}
 \item the infinite representation if \(t_N = 1\).
 \item the finite representation if \(t_N = 2\).
\end{itemize}

Now the Cantor set $C$, is exactly the numbers in \([0,1]\)
whose ternary representation (with the above choice made)
does \emph{not} contain any factor (digit) $1$, that is
\(1 \neq t_i\in \{0,2\}\) for all $i$.

Using \(b_i=t_i/2\in \{0,1\}\), we build
 the following map \(f:C\rightarrow [0,1]\)
\begin{equation*}
f(\alpha)
  = f\left(\sum_{i=1}^\infty t_i(\alpha) 3^{-i} \right)
  = \sum_{i=1}^\infty (t_i(\alpha)/2) 2^{-i}.
\end{equation*}
Maps the Cantor set \emph{onto} the unit interval, using \emph{binary}
representation. This shows directly that the cardinality of $C$ is
the same as that of \([0,1]\) which is known to be \(|\R^1|\).

%%%%%%%%%%%%%% 6
\begin{excopy}
Construct \label{ex:disc:K}
a totally disconnected compact set \(K\subset \R^1\) such that
\(m(K) > 0\).
($K$ is to have no connected subset consisting of more than one point.)

If $v$ is lower semicontinuous  and \(v\leq \chhi_K\), show that actually
\(v \leq 0\). Hence \(\chhi_K\) cannot be approximated by lower semicontinuous
function, in the sense of
\index{Vitaly}
\index{Carath\'eodory}
Vitaly-Carath\'eodory Theorem.
\end{excopy}

Construct the following similar to Cantor set.
Start from the unit interval,
in each step we ``break'' any segment of the previous step,
but (contrary to Cantor's) ensuring that the sum of open segment
taken out is ``much'' less than $1$.

More formally, we start (after step-0), with \(D_0 = [0,1]\).
After step $n$, \(D_n\) is a union of \(2^n\) closed intervals.
In step $n$, from each interval \([a,b]\) of \(D_{n-1}\)
we substract from its center an open sub-interval with length of \(2^{-(2n+1)}\)
Hence in this step we substract a total of
\begin{equation*}
 2^n \cdot 2^{-(2n+1)} = 2^{-n-1}.
\end{equation*}
The total length of open intervals removed by all steps upto step $n$ is
\begin{equation*}
 \sum_{k=1}^n 2^{-k-1} = (1/2) \sum_{k=1}^n 2^{-k} < 1/2
\end{equation*}
Thus \(K = \cap_n D_n\) is compact, \(m(D) \geq 1/2\)
and it is totally disconnected, since any interval eventually gets cut.

Now if $v$ satisfies the assumptions of the exercise, and by negation
\(v(w) > 0\) for some \(w\in \R\) then for \(a = v(w)/2\) then
\(V = v^{-1}(a,\infty)\) is open and \(w\in V\) thus there is an open
interval $I$ for which \(f(x)>a\) for all \(x\in I\),
but since $K$ is totally disconnected, there must be some \(r\in I\setminus K\)
giving a the contradiction \(\chhi(r) = 0\).

%%%%%%%%%%%%%% 7
\begin{excopy}
Given \(\epsilon > 0\), construct an open set \(E\subset [0,1]\) which is dense
in \([0,1]\), such that \(m(E)=\epsilon\).
(To say that $A$ is dense in $B$ means that the closure of $A$ contains $B$.)
\end{excopy}

Of course we should assume \(\epsilon \leq 1\).
Put \(\Q\cap[0,1]\) into q sequence \((q_i)_{i=1}^\infty\).
Now define the following sequence open sets \((G_k)_{k=1}^\infty\).
Given \(k\leq 1\), let \(H_k =  \cup_{j<k} \overline{G_j}\)
(closed, where \(H_1 = \emptyset\)),
let $i$ be the minimal such that \(q_i\notin H_k\).
Pick an open interval \(I_k\) of \(q_i\) with length \(l_k\)
such that \(l_k \leq 2^{-k}\epsilon\)
(Note: by ``open'' we mean here --- open \emph{in} \([0,1]\).
Thus \([0,a)\) and \((b,1]\) are considered open).
If \(l_k < 2^{-k} \epsilon\), also pick some finite number
of open intervals \(\{J_{k,m}\}\) such that
\begin{equation*}
G_k = I_k \cup\,\bigcup_{m=1}^{N_k} J_{k,m}
\end{equation*}
satisfies
\begin{equation*}
m(G_k) = 2^{-k} \qquad \textrm{and} \qquad
  G_k \cap \left(\cup_{j<k} G_j\right) = \emptyset.
\end{equation*}
Let \(G = \cup G_k\), surely \(m(G) = \epsilon\) and \(\Q\cap [0,1] \subset G\),
and so $G$ is dense in \([0,1]\).


%%%%%%%%%%%%%% 8
\begin{excopy}
Construct a Borel set \(E\subset \R^1\) such that
\begin{equation*}
 0 < m(E\cap I) < m(I)
\end{equation*}
for every nonempty segment $I$. Is it possible to have \(m(E) < \infty\)
for such a set?
\end{excopy}


Given \(0<a<1\) we will first build a set \(F\subset [0,1]\) such that
\(m(F) = a\) and
\begin{equation*}
0 < m(F\cap I) < m(I)
\end{equation*}
We will build $F$, and its complement \(G = [0,1] \setminus F\)
as an countable infinite union of Borel sets,
\(F = \cup_n F_n\)
and
\(G = \cup_n G_n\)
with measures \(m(F_i) = 2^{-i}a\) and \(m(G_i) = 2^{-i}(1-a)\).
At each stage,
\begin{equation*}
 R_n = [0,1] \setminus
       \left( \bigcup_{i=1}^n F_i \cup \bigcup_{i=1}^n G_i\right)
\end{equation*}
will be a countable family of intervals ---
close, open are half closed and open. That is, of
any of the forms:
\([a,b]\),
\((a,b)\),
\([a,b)\) or
\((a,b]\).

\paragraph{Building \(F_1\).}
For any interval \(I\subset[0,1]\).
Similar to Exercise~\ref{ex:disc:K} above, we construct a totally disconnected
subset \(F_1\) of [0,1]
whose complement is a countable family of open intervals.
We carefully choose the sizes of the open intervals, so
that \(m(F_1) = a/2\).

\paragraph{Building \(G_1\).}
The set \(F_1^c = [0,1]\setminus G_1\) consists of countably many \(\cal{N}\)
intervals (where \(\cal{N}<\infty\) or \({\cal{N}} = \aleph_0\)).
Within each interval \(I_i\) of \(F_1^c\) we build a totally disconnected
subset \(T_i\), such that the sum of measures of these sets equals \((1-a)/2\).
This is done either by choosing a measure of
\((1-a)/\cal{N}\) if  \(\cal{N}<\infty\),
or a measure of \(2^{i+1}(1-a)\) otherwise.
We let \(G_1= \cup_i T_i\).


\paragraph{Building \(F_n\).}
Assume \(\{F_i\}_{i=1}^{n-1}\)
and \(\{G_i\}_{i=1}^{n-1}\) are built,
The set
\begin{equation*}
 R_{n-1} = [0,1] \setminus
  \left( \bigcup_{i=1}^{n-1} F_i \cup \bigcup_{i=1}^{n-1} G_i\right)
\end{equation*}
consists of countably many intervals.
Within each of these intervals we build a totally disconnected set \(U_{n,i}\)
such that when letting \(F_n = \cup_i U_{n,i}\) be a disjoint union, we have
\begin{equation*}
m(F_n) = m\left(\cup_i U_{n,i}\right) = \sum_i m(U_{n,i}) = 2^{-(n+1)} a
\end{equation*}

\paragraph{Building \(G_n\).}
Assume \(\{F_i\}_{i=1}^{n}\)
and \(\{G_i\}_{i=1}^{n-1}\) are built,
The set
\begin{equation*}
 S_{n-1} = [0,1] \setminus
  \left( \bigcup_{i=1}^n F_i \cup \bigcup_{i=1}^{n-1} G_i\right)
\end{equation*}
consists of countably many intervals.
Within each of these intervals we build a totally disconnected set \(V_{n,i}\)
such that when letting \(G_n = \cup_i V_{n,i}\) be a disjoint union, we have
\begin{equation*}
m(G_n) = m\left(\cup_i V_{n,i}\right) = \sum_i m(U_{n,i}) = 2^{-(n+1)} (1-a).
\end{equation*}

Now $F$ and $G$ are well defined, and \(F\disjunion G = [0,1]\).

Note that the way \(\{F_i\}\) and \(\{G_i\}\) were build,
for each \(x,y\in F\), where \(x<y\), there is some $n$, such that
\(x,y\in \cup_{i\leq n} F_i\) and \(m((x,y) \cap G_i) > 0\).

Similarly,
for each \(x,y\in G\), where \(x<y\), there is some $n$, such that
\(x,y\in \cup_{i\leq n} G_i\) and \(m((x,y) \cap  F_{i+1}) > 0\).

The last observations show
the inequalities \(0 < m(F\cap\, [x,y]\,) < m([x,y])\).
To generalize it, we repeat the process of building $F_i$ for
any interval \(\{[n,n+1]\}_{n\in\Z}\), with \(a = 2^{-|n|}\).
Put \(E = \cup F_i\). Now \(m(E) = \sum_{n\in\Z} = 3 < \infty\)
and the desired inequalities \(0 < m(E\cap\, I\,) < m(I)\)
for and interval $I$ in \R\ is established.




%%%%%%%%%%%%%%
\begin{excopy}
Construct a sequence of continuous function \(f_n\) on \([0,1]\) such that
\(0\leq f_n \leq 1\), such that
\begin{equation*}
 \lim_{n\rightarrow \infty} \int_0^1 f_n(x)dx = 0,
\end{equation*}
but such that the sequence \(\{f_n(x)\}\) converges for no \(x\in[0,1]\).
\end{excopy}

Put \(N_0 =0\) and \(N_k = \sum_{i=1}^{k+1} i = (k+1)(k+2)/2\).
We will define the \(f_n\) functions in $k$-batches.
For each \(k>0\), we  define \(\{f_n:  N_{k=1} < n \leq N_k\}\):
\begin{equation*}
f_n(x) = f_{N_{k-1}+i}(x) = \left\{
 \begin{array}{l@{\quad}c}
   1 - k|x - (i-1)/k| & \textrm{if}\; |x - (i-1)/k|<1/k \\
   0                  & \textrm{Otherwise}
 \end{array}\right.
\end{equation*}
for \(1\leq i \leq N_k\).
We see that \(\int_0^1 f(n) = 1/k\) and so \(\int_0^1 f(n)\to 0\)
but for every $k$-batch, for any \(x\in[0,1]\)
there is some $n$ such that \(N_{k-1}< n \leq N_k\)
and \(f_n(x) \geq 1/2\).


%%%%%%%%%%%%%%
\begin{excopy} % 10
If \label{ex:2:10}
\(\{f_n\}\) is a sequence of continuous functions on \([0,1]\) such that
\(0\leq f_n \leq 1\), and such that
\(f_n(x)\to 0\) as \(n \to \infty\),
for every \(x\in[0,1]\), then
\begin{equation*}
 \lim_{n\rightarrow \infty} \int_0^1 f_n(x)dx = 0,
\end{equation*}
Try to prove this without using any measure theory or any theorem
about Lebesgue integration. (This is to impress you with the power of
the Lebesgue integral. A nice proof was given
\index{Eberlein}
by W.F.~Eberlein in \emph{communications on Pure and Applied Mathematics},
vol.~X, pp.~357-360, 1957.)
\end{excopy}

We will allow ourselves to use some trivial measure concepts
such as the sum of lengths of finite union of intervals.
That is if \(\{I_i\}_{i=1}^n\),
where \(I_i\) are intervals of the form
\((a_i,b_i)\),
\((a_i,b_i]\),
\([a_i,b_i)\) or
\([a_i,b_i]\), and \(I_i \cap I_j = \emptyset\) whenever \(i\neq j\) then
we use
\begin{equation*}
m\left(\Disjunion_{i=1}^n I_i\right) =
\sum_{i=1}^n m(I_i) = \sum_{i=1}^n b_i - a_i.
\end{equation*}

\iffalse
Back to the exercise. Let us define some sequences of subsets of \([0,1]\)
(with nicknames) based on \(\{f_n\}_{i\in\N}\).
\begin{equation*}
\begin{array}{lcl@{\qquad}r}
 U_{n,k} &=&
    \{x\in [0,1]: |f_n(x)| \geq 1/k\}    & \textrm{(Upper)} \\ \\
 T_{n,k} &=&
   \bigcup\limits_{i=n}^\infty U_{i,k}   & \textrm{(Tail)} \\ \\
 R_k     &=&
    \bigcap\limits_{i=1}^\infty T_{i,k}  & \textrm{(Resistance)} \\
\end{array}
\end{equation*}

When looked closely, we see that \(R_k\) consists of all \(x\in[0,1]\)
such that \(|f_n(x)| \geq 1/k\)  for infinitely many $n$.
Thus the assumption that \(f_n(x)\to 0\) for all \(x\in[0,1]\)
means that
\begin{equation*}
\cap_{k\in\N} R_k = \emptyset.
\end{equation*}
\fi

Back to the exercise. Let
\begin{equation*}
 g_n(x) = \sup_{k\geq n} |f_k(x)|
\end{equation*}
Clearly \(\{g_n\}_{n\in\N}\) is a decreasing sequence,
converges pointwise \(g_n(x)\xrightarrow{n\to\infty} 0\) and
\begin{equation*}
 \left|\int_0^1 f_n(x)dx\right|
 \leq  \int_0^1 |f_n(x)|dx \leq  \int_0^1 g_n(x)dx.
\end{equation*}
Hence it is sufficient to show that
\begin{equation} \label{eq:integ:gto0}
\lim_{n\to\infty} \int_0^1 g_n(x)dx = 0.
\end{equation}

Let \(\epsilon>0\) be arbitrary, and define
\begin{equation*}
 L_n = L_{n,\epsilon} = \{x\in[0,1]: |f_n(x) < \epsilon\}.
\end{equation*}
By definition and assumptions, \(\{L_n\}_{n\in\N}\) is an increasing
sequence of open sets with union
\begin{equation} \label{eq:ULn}
\bigcup_{n\in\N} L_n = [0,1].
\end{equation}
Each \(L_n\) is a union of countably many open intervals,
\begin{equation*}
 L_n = \bigcup_{k\in \N} I_{n,k} = \bigcup_{k\in \N} (a_{n,k}, b_{n,k})
 \qquad \textrm{for each}\; n\in\N.
\end{equation*}

The (still limited to intervals) total measure
\begin{equation}
 M_n = M(L_n) =\sum_{k\in \N} m(I_{n,k})
\end{equation}
is clearly an increasing sequence (by simply looking at sub-finite sums).
We will now show that it
converges to the measure of the unit \(m([0,1])\), that is:
\begin{equation} \label{eq:Lsubint:to1}
 \lim_{n\to\infty} \sum_{k\in \N} m(I_{n,k}) = 1.
\end{equation}
By negation, assume
\begin{equation*}
 \lim_{n\to\infty} M_n = \beta < 1.
\end{equation*}

For any sub-intervals \(I,[a,b]\subset [0,1]\)
trivially
\begin{equation*}
m(I\cap[a,b]) = m(I\cap[a,(a+b)/2]) + m(I\cap[(a+b)/2, b]).
\end{equation*}
Similarly
\begin{equation} \label{eq:MLn}
 M(L_n \cap [a,b]) =  M(L_n \cap [a,(a+b)/2]) + M(L_n\cap[(a+b)/2, b])
\end{equation}
Starting with \(a_0=0\), \(b_0=1\), and recursively bisect \([a_i,b_i]\),
we get a decreasing sequence of closed sub-segments with sizes
\(m([a_i,b_i]) = 2^{-i}\),
such that
\begin{equation*}
 M(L_n \cap [a_i,b_i]) \leq 2^{-i}\beta < 2^{-i}
\end{equation*}
for all \(n\in\N\).
This can be done, by choosing the smaller of the two choices
in each bisection step, and using (\ref{eq:MLn}).
Let $b$ be the intersection point \(\{b\} = \cap_{i\in\N} [a_i,b_i]\).
But by (\ref{eq:ULn}), there exists $n$, such that \(b\in L_n\).
Since \(L_n\) is open, there is some $i$ such that \(b\in[a_i,b_i]\subset L_n\).
Now
\begin{equation*}
M(L_n \cap [a_i,b_i]) = M([a_i,b_i]) = 2^{-i}
\end{equation*}
gives a contradiction and thus (\ref{eq:Lsubint:to1}) is true.

Let \(N<\infty\) be such that
\begin{equation*}
M_N = \sum{k\in \N} m(I_{N,k}) > 1 - \epsilon
\end{equation*}
and \(K<\infty\) be such that
\begin{equation*}
M_N = \sum{k=1}^K m(I_{N,k}) > 1 - 2\epsilon.
\end{equation*}
Set
\begin{eqnarray*}
  D & = & \cup{k=1}^K I_{N,k} \\
  U & = & [0,1] \setminus D
\end{eqnarray*}
Both $D$ and $U$ are unions of finite number of intervals.
Since \(m(D) > 1 - 2\epsilon\), the complement has \(m(U) < 2\epsilon\).
Now we compute:
\begin{equation*}
\int_0^1 g_N(x)dx = \int_D g_N(x)dx + \int_U g_N(x)dx
 \leq (1-2\epsilon)\epsilon + 2\epsilon\cdot 1 < 3\epsilon.
\end{equation*}
Since \(\epsilon>0\) was arbitrary and
\(\{g_n\}_{n\in\N}\) is a decreasing sequence,
the convergence of (\ref{eq:integ:gto0}) is shown.


%%%%%%%%%%%%%% 11
\begin{excopy}
Let \(\mu\) be a regular Borel measure on a compact Hausdorff space $X$:
assume \(\mu(X) = 1\). Prove that there is a compact set \(K\subset X\)
\index{carrier}
\index{support!measure}
(the \emph{carrier} or \emph{support} of \(\mu\))
such that \(\mu(K) = 1\) but \(\mu(H)<1\)
for every proper compact subset $H$ of $K$.
\emph{Hint}: Let $K$ be the intersection of all compact \(K_\alpha\) with
\(\mu(K_\alpha)=1\);
show that every open set $V$ which contains $K$ also contains some \(K_\alpha\).
Regularity of \(\mu\) is needed; compare Exercise~18.
Show that \(K^c\) is the largset open set in $X$ whose measure is $0$.
\end{excopy}

If both \(K_1\) and \(K_2\) are compact then
\(K_1 \cap K_2\) is compact. If
\(\mu(K_1) = \mu(K_2) = 1\), then
\(mu(X\setminus K_1) = \mu(X\setminus K_2) = 0\) and so
\begin{equation*}
\mu(K_1 \cap K_2) \geq \mu(X) - (\mu(X\setminus K_1) + \mu(X\setminus K_2) =
        1 - 0 = 1.
\end{equation*}
By induction \(\mu(\cap_{i=1}^n K_i = 1\) for finite intersection.

Let $K$ be the intersection as in the hint.
Let $V$ be an open set such that \(\cap_\alpha K_\alpha \subset V \subset X\).
Now \(V^c \subset \cup_\alpha K_\alpha^c\), this is an open covering
of the compact \(V^c\). Hence there is a sub-finite covering,
that is we have \(\{\alpha_i\}_{i=1}^m\) such that
\(V^c \subset \cup_{i=1}^m K_{\alpha_i}^c\), equivalently
\(L = \cap_{i=1}^m K_{\alpha_i} \subset V\). By our initial remarks,
$L$ is compact and \(\mu(L)=1\) thus \(L=K_\alpha\) for some \(\alpha\).

By regulartiy,
\begin{equation*}
\mu(K) = \inf \{\mu(V): V\;\textrm{is open, and }\; K\subset V\}
       \geq \mu(K_\alpha) = 1.
\end{equation*}
Thus \(\mu(K) = 1\) and by construction, $K$ is minimal compact with
such measure.


%%%%%%%%%%%%%% 12
\begin{excopy}
Show \label{ex:2:12}
that every compact set of \(\R^1\) is the support of a Borel measure.
\end{excopy}

Let \(K\subset\R^1\) be a compact set, with the inherited topology.
For any function \(f\in C_c(K) = C(K)\) we will define an extension
\(\tilde{f} \in C_c(\R)\). By $K$ being bounded, we can pick
\(b_0 < \min(K)\) and \(b_1 > \max(K)\).
Denote the compact \(\tilde{K} = K \cup \{b_0,b_1\}\)
and the complement open set \(G=\R\setminus \tilde{K}\).
For any \(x\in G\) there is a maximal open interval \((a_x,b_x) \subset $G$\)
whose endpoints \(a_x,b_x\in \tilde{K}\). Define:
\begin{equation*}
 \tilde{f}(x) = \left\{\begin{array}{l@{\qquad}l}
                  f(x) & x \in K \\
                  0    & x \leq b_0 \quad \textrm{or} \quad b_1 \leq x \\
                  \left(\tilde{f}(b_x) - \tilde{f}(a_x)\right)
                  \frac{x - a_x}{b_x - a_x}
                  + \tilde{f}(a_x)
                    & x \in (a_x,b_x)\subset G \quad\textrm{and}\quad
                      a_x,b_x \in \tilde{K}
                  \end{array}\right.
\end{equation*}
Note the ``pseudo'' recursive definition is fine, since
only \(\tilde{f}(b_x)\) and  \(\tilde{f}(a_x)\) are used
and for them \(\tilde{f}\) is well defined on previous cases.
The mapping \(f\to\tilde{f}\) is clearly linear mapping of \(C(K) \to C_c(\R)\).
We now define a positive functional on \(C(K)\)
\begin{equation*}
 \Lambda(f) = \int_{\R} \tilde{f}\,dm =  \int_{b_0}^{b_1} \tilde{f}\,dm
\end{equation*}
where $m$ is the regular Lebesgue's measure on \(\R\).

\index{Riesz}
From Riesz Theorem~2.14, there is Borel measure \(\mu\) on $K$
such that for all \(f\in C(K)\)
\begin{equation*}
 \int_{\R^1} f\,d\mu = \Lambda f.
\end{equation*}

We will now show that $K$ is the support of \(\mu\).
By negation, let \(H\subsetneq K\) be a compact such that
 \(\mu(H) = \mu(K)=1\), hence \(\mu(K\setminus H) = 0\).
Pick \(x\in K\setminus H\). By being a compact Hausdorff space,
there exists an open set $V$ such that \(x\notin V\) and \(H\subset V\).
\index{Urysohn's lemma}
By Urysohn's Lemma~2.12, we can find a continuous function \(g\in C(K)\)
such that \(\{x\} \prec g \prec V\).
Now
\begin{equation} \label{eq:Ksupp:mu}
 \Lambda g = \int_K g\,d\mu = \int_H g\,d\mu + \int_{K\setminus H} g\,d\mu
           = \int_H 0\,d\mu + \int_{\emptyset} g\,d\mu = 0+0 = 0.
\end{equation}
But \(\tilde{g} \geq 0\) is continuous, with \(\tilde{g}(x) = g(x) > 0\),
hence
\(\Lambda g = \int_{\R} \tilde{g}dm > 0\) a contradiction
to (\ref{eq:Ksupp:mu}).


%%%%%%%%%%%%%% 13
\begin{excopy}
Is it true that every compact subset of \(\R^1\) is a support of a continuous
functions? If not, can you describe the class of all compact sets in
\(\R^1\) whoch are supports of continuous functions?
Is your description valid in other topological spaces?
\end{excopy}

The first claim is false. Take a any singleton, say \(\supp f = \{a\}\)
then \(f(a) \neq 0\), by continuity, using a positive \(\epsilon < |f(a)|\)
we can find a \(\delta>0\) such that
\begin{equation*}
 \supp f \supset (a-\delta, a + \delta)
         \subset \{x\in\R: |f(x)| > |f(a)| - \epsilon > 0\}.
\end{equation*}


Here is the classification:
\begin{llem}
A compact \(K\in\R\) is a support of some \(f\in C(\R)\)
iff \(K = \overline{\inter{K}}\) (equals the closure of its interior).
\end{llem}
\begin{thmproof}
Let $K$ be  a compact $K$ set.

Assume \(K = \supp f\), for some \(f\in C(\R)\).
For any closed set $K$, (in particular compact),
\(K \subset \overline{\inter{K}}\).
If by negation \(K \subsetneq \overline{\inter{K}}\), then we can find
\(x \in K \setminus \overline{\inter{K}}\).
If \(f(x) \neq 0\) then there is a neighborhood $V$ of $x$
such that \(x\in V \subset \supp f\) which gives the contradiction
\(x\in \inter{K}\). Thus \(f(x) = 0\) but then
\(K = \supp f \subset \overline{\inter{K}}\) gives a contradiction.

Conversely, assume \(K = \overline{\inter{K}}\).
Since \(G = \inter{K} = \cup_{i\in\N} I_i\), is a countable union
of open  intervals \(I_i = (a_i, b_i)\), we denote their
length  \(l_i = b_i - a_i\) and
center
\(c_i = (a_i + b_i)/2\) to define
\begin{equation*}
f(x) = \left\{\begin{array}{l@{\qquad}l}
               l_i/2 - |x - c_i| &  a_i \leq x \leq b_i \\
               0 & x \notin K
              \end{array}\right.
\end{equation*}
Now clearly $f$ is continuous and \(\supp f = K\).
\end{thmproof}

From the proof, we can see that the classification
can extend to any space $X$ where any open set $G$ is a union
of open sets \(\{V_\alpha\}_{\alpha\in I}\) with disjoint closures
such that for each \(V_\alpha\) there is a function \(f\in C(X)\)
such that \(K \prec f \prec V\) for any compact \(K\subset V\).
This is true for \(\R^n\).

%%%%%%%%%%%%%% 14
\begin{excopy}
If $f$ is a Lebesgue measurable complex function on \(\R^1\),
prove that there is a Borel function $g$ on \(\R^1\) such that
\(f=g\)~a.e.[$m$].
\end{excopy}

With the understanding (similar to the definition
\index{Borel}
of \emph{Borel function} --- see:
\cite{RudinRCA80} page~13)
that
\index{Lebesgue!measurable function}
\(f:X\to \C\) is a \emph{Lebesgue measurable function} iff
\(f^{-1}(V)\) is Lebesgue measurable for any open set \(V\subset X\).

We will build a subset \(M \subset \R\) with \(m(M) = 0\)
and define
\begin{equation} \label{eq:g:Borel}
g(x) = \left\{\begin{array}{l@{\qquad}l}
             f(x) & x\notin M \\
             0    & x\in M
             \end{array}\right..
\end{equation}

By Theorem~2.20 for any Lebesgue measurable set $L$, we can find
Borel (\(F_{\sigma}\) set \(B\subset L\) such that \(m(L\setminus B) = 0\).

Enumerate all open rectangles with rational boundaries
\({\cal{R}} = \{R_i\}_{i\in\N}\)
where
\begin{equation*}
 R_i = \{z\in\C : r_0\leq \Re(z) \leq r_1 \;\wedge\;
                  q_0\leq \Im(z) \leq q_1 \;\wedge\;r_0,r_1,q_0,q_1\in\Q\}.
\end{equation*}

Now \(L_i = f^{-1}(R_i)\) is a Lebesgue measurable function.
Let \(B_i\subset L_i\) be a Borel measurable set
(as on the preceding remark) such that and \(m(L_i\setminus B_i) = 0\).
We define \(M = \cup_{i\in\N} (L_i\setminus B_i)\).
By \(\sigma\)-additivity, \(m(M)=0\) as requried.
Now, it is left to show
that $g$ defined in (\ref{eq:g:Borel}) is a Borel measurable function.

Any (origin-less) open set \(G\subset \C\setminus\{0\}\)
can be represented as a countable union of intervals from \(\cal{R}\),
say
\begin{equation*}
G = \cup_{i\in\N} R_{k_i}
\end{equation*}
Now computing the inverse images
\begin{eqnarray*}
 g^{-1}(G)
 &=&  f^{-1}(G) \setminus M \\
 &=& f^{-1}\left(\cup_{i\in\N} R_{k_i}\right) \;\setminus M
     = \bigcup_{i\in\N} f^{-1}( R_{k_i}) \;\setminus M
     = \bigcup_{i\in\N} L_{k_i} \;\setminus M \\
 &=& \left(\bigcup_{i\in\N} B_{k_i}
     \disjunion \left(L_{k_i}\setminus B_{k_i}\right)\right) \;\setminus M \\
 &=& \left(\bigcup_{i\in\N} B_{k_i}
     \,\cup\, \bigcup_{i\in\N}  \left(L_{k_i}\setminus B_{k_i}\right)
     \right) \;\setminus M \\
 &\subset&  \left(\bigcup_{i\in\N} B_{k_i} \cup M'\right) \;\setminus M \\
 &=& \bigcup_{i\in\N} B_{k_i}.
\end{eqnarray*}
Where
\begin{equation*}
 M' = \bigcup_{i\in\N}  \left(L_{k_i}\setminus B_{k_i}\right) \subset M.
\end{equation*}
So \(g^{-1}(G)\) is a Borel set. In particular \(g^{-1}(\C\setminus\{0\})\)
is Borel set, and by \calB\ being \(\sigma\)-algebra,
the complement \(g^{-1}(\{0\})\) is a Borel set as well.


%%%%%%%%%%%%%% 15
\begin{excopy}
It is easy to gueass the limits of
\begin{equation*}
 \int_0^\pi \left(1 - \frac{x}{n}\right)^n e^{x/2}\, dx
 \quad\textrm{and}\quad
 \int_0^\pi \left(1 + \frac{x}{n}\right)^n e^{-2x}\, dx.
\end{equation*}
as \(n\to\infty\). Prove that your guesses are correct.
\end{excopy}

\begin{itemize}

\item
We compute the limit
\begin{equation*}
\lim_{n\to\infty} \left(1 - \frac{x}{n}\right)^n e^{x/2}
=  e^{x/2} \lim_{n\to\infty} \left(1 - \frac{x}{n}\right)^n \\
=  e^{x/2} e^{-x} = e^{-x/2}.
\end{equation*}

The integrand \(\left(1 - \frac{x}{n}\right)^n e^{x/2}\)
is bounded on \([0,\pi]\) by  \(1\cdot e^{\pi/2}\).
Using \(\int e^{-x/2}\,dx = -2e^{-x/2} + c\) and
Lebesgue Dominated Theorem
\begin{equation*}
 \lim_{n\to\infty} \int_0^\pi \left(1 - \frac{x}{n}\right)^n e^{x/2}\, dx =
 \int_0^\pi e^{-x/2}\,dx = -2e^{-\pi/2} - (-2e^{-0/2}) = -2e^{-\pi/2} - 2.
\end{equation*}

\item
We compute the limit
\begin{equation*}
\lim_{n\to\infty} \left(1 + \frac{x}{n}\right)^n e^{-2x}
=  e^{-2x} \lim_{n\to\infty} \left(1 + \frac{x}{n}\right)^n \\
=  e^{-2x} e^x = e^{-x}.
\end{equation*}

The integrand \(\left(1 + \frac{x}{n}\right)^n e^{-2x}\)
is bounded on \([0,\pi]\) by  $2$.
Using \(\int e^{-x}\,dx = -e^{-x} + c\) and
Lebesgue Dominated Theorem
\begin{equation*}
 \lim_{n\to\infty} \int_0^\pi \left(1 + \frac{x}{n}\right)^n e^{-2x}\, dx =
 \int_0^\pi e^{-x}\,dx = -e^{\pi} - (-e^0) = -e^{-\pi/2} + 1.
\end{equation*}

\end{itemize}


%%%%%%%%%%%%%%
\begin{excopy}
If $m$ is a Lebesgue measurable on \(\R^k\), prove that \(m(-E) = m(E)\),
where
\begin{equation*}
 -E  = \{-x: x\in E\},
\end{equation*}
and hence that
\begin{equation*}
 \int_{\R^k} f(x)dx = \int _{\R^k} f(-x)dx =
\end{equation*}
for all \(f\in L^1(\R^k)\).
\end{excopy}

For any $k$-cell $W$, we have
\begin{equation*}
m(W) = \vol(W) = \prod_{i=1}^k (\beta_i - \alpha_i) =
\prod_{i=1}^k ((-\alpha_i) - (-\beta_i)) = \vol(-W) = m(W).
\end{equation*}
Since $E$ is a limit of countable union of $k$-cells
\(m(E) = m(-E)\) follows.
The equality of integration is now simply matter of mapping \(\R^k \to \R^k\)
and a change of variable.

%%%%%%%%%%%%%% 17
\begin{excopy}
Let $X$ be the plane, with the following topology: A set is open
if and only if its intersection with every vertical line is an open subset
of that line,
with respect to the usual topology of \(\R^1\).
Show that this $X$ s a locally compact Hausdorff space.
If \(f\in C_c(X)\), let \seqxn\ be those values of $x$
for which \(f(x,y)\neq 0\) for at least one $y$
(there are only finitely many such $x$!), and define
\begin{equation*}
\Lambda f = \sum_{j=1}^n \int_{-\infty}^\infty f(x_j,y) dy.
\end{equation*}
Let \(\mu\) be the measure associated with this \(\Lambda\) by Theorem~2.14.
If $E$ is the $x$-axis, show that \(\mu(E) = \infty\) although
\(\mu(K) = 0\) for every compact \(K\subset E\).
\end{excopy}

For any two distinct point \(P_i = (x_i,y_i)\in \R^2 = X\) where \(i=1,2\)
we present open sets \(V_1\) and \(V_2\) that separate the points.
If \(x_1\neq x_2\), then we take  \(V_i = \{x_1\}\times\R\),
otherwise \(y_1\neq y_2\) and we take
\begin{equation*}
V_i = \{(x_i,y)\in\R^2: |y-y_i| < |y_1-y_2|/2\}.
\end{equation*}

For Any point \(P=(x,y)\in\R^2=X\) and any neighborhood $V$ of $P$,
we can find \(\epsilon>0\) such that
\begin{equation*}
 G = \{(x,v)\in\R^2: |v-y| < \epsilon\} \subset
\overline{G} = \{(x,v)\in\R^2: |v-y| \leq \epsilon\}  \subset V
\end{equation*}
and clearly \(\overline{G}\) is compact,

Now that we have shown that $X$ is locally compact Hausdorff space,
we proceed. Let \(f\in C_c(X)\).
Let \(W\subset \R\)  be of all
\(x\in W\) such that there exist $y$ such that \(f(x,y)\neq 0\).
If by negation $W$ is inifinite, then
the family of open sets
\begin{equation*}
 \{\,\{x\}\times\R\, : x\in W\}
\end{equation*}
is a covering of \(\supp f\) which has no finite sub-covering.
This contradicts the assumption that \(f\in C_c(\R)\).

By looking at vertical finite segments
\(S = \{(x,y)\in\R^2: x=x_0 \wedge y_0\leq y\leq y_1\}\),
clearly \(\Lambda \chhi_S = y_1 - y_0\) hence \(\mu(S) = y_1-y_0\).
Let $E$ is the $x$-axis. The last assertion is also true for any
non empty sub interval \([a,b]\subset E\).
Any open super set \(V \supset [a,b]\) contains a continuum
of vertical segments and hence \(\mu(V) = \infty\).
By Theorem~2.14(c) \(\mu([a,b] = \infty\).

The inifinite subsets of $E$
are covered by inifinitely many vertical open segments
with no sub-finite sub-covering.
Hence, the compact subsets of $E$ are exactly the finite subsets.
Clearly the measure of a singleton \(\mu(\{(x,y)\}) = 0\),
and thus \(\mu(K)=0\) for any compact \(K\subset E\).



%%%%%%%%%%%%%% 18
\begin{excopy}
This exercise requires more set-theoretic skill than the preceding
ones.  Let $X$ be a well-ordered uncountable set which has a last
element \(\omega_1\), such that every predecessor of \(\omega_1\) has
at most countably many predecessors.
(``Construction'': Take any well-ordered set which has elements with uncountably
many predecessors, and let \(\omega_1\) be the first of these;
\(\omega_1\) is called the first uncountable ordinal.)
For \(\alpha\in X\), let \(P_\alpha\) [\(S_\alpha\)] be the set of all
predecessors (successors) of \(\alpha\),
and call a subset of $X$ open if it is
a \(P_\alpha\) or an \( S_\beta\)
or an  \(P_\alpha \cap S_\beta\)
or a union of such sets.
Prove that $X$ is then a compact Hausdorff space.
(\emph{Hint}: No well-ordered set contains an infinite decreasing sequence.)

Prove that the complement of the point \(\omega_1\) is an open set
\index{sigma-compact@\(\sigma\)-compact}
which is not \(\sigma\)-compact.

Prove that to every \(f\in C(X)\)  there corresponds an \(\alpha \neq \omega_1\)
such that $f$ is constant on \(S_\alpha\).

Prove that the intersection of every countable collection \(\{K_\alpha\}\)
of uncountable compact subsets of $X$ is uncountable.
(\emph{Hint}: Consider limits of increasing countable sequences in $X$ which
intersects each \(K_n\) in infinitely many points.)

Let \frakM\ be the collection of all \(E\subset X\) such that either
\(E\cup \{\omega_1\}\) or
\(E^c\cup \{\omega_1\}\) contains an uncountable compact set; in the first case,
define \(\lambda(E)=1\); in the second case, define \(\lambda(E) = 0\).
Prove that \frakM\ is a \(\sigma\)-algebra which contains all Borel sets
in $X$, that \(\lambda\) is a measure on \frakM\ which is \emph{not}
regular (every neighborhood of \(\omega_1\) has measure $1$), and that
\begin{equation*}
 f(\omega_1) = \int_X f\, d\lambda
\end{equation*}
for every \(f\in C(X)\). Describe the regular \(\mu\) which Theorem~2.14
associates with this linear functional.
\end{excopy}


\paragraph{Hausdorff.} Let \(\alpha<\beta\) be any two points in $X$.
Clearly \(\alpha+1\leq \beta\) and then  \(P_{\alpha+1}\) and \(S_{\alpha}\)
are open sets that separate the two points.

\paragraph{Compactness.} Let \(\{G_i\}_{i\in I}\) be a family of open
sets such that \(X=\cup_{i\in I}G_i\). By negation, assume that
there is no finite sub-covering. We build
\begin{itemize}
\item A sequence of sets
     \(\{G_{i_j}: j\in\{0\}\cup \N\;\textrm{and}\; i_j \in I\}\).
\item A strictly decreasing sequence of points \(\{\alpha_j\}_{i\in\N}\)
\end{itemize}
by induction.
For notational convenience we use ``\(G_j=G_{i_j}\)''
and we denote \(U_n = \cup_{j=0}^n G_j\).
Pick \(G_0\) such that \(\omega_1\in G_0\).

By looking at base neighborhoods, we see that
each non empty open set (except for \(\{1\}\)).
contains a
non empty open set of the form \(S_\alpha\) or \(S_\alpha \cap P_\beta\)

In step $n$, let \(\alpha_n\) be the least point such that
\(S_{\alpha_j} \subset U_{n-1}\).
or \(S_{\alpha_j}\cap P_\beta \subset U_{n-1}\) for some
\(\beta \geq \alpha+1\).
Note that \(\alpha_j \notin \cup_{j<n} G_j\). Now we pick \(G_n\) such that
\(\alpha_n \in G_n\).
By construction, there is a base neighborhood of \(\alpha_1\)
that is contained in \(U_n\). Thus \(\{\alpha_j\}_{i\in\N}\) is strictly
decreasing. Hence has no minimal element in contradiction to $X$ being
well ordered.


\paragraph{Non \(\sigma\)-compactness of \(X\setminus\{\omega_1\}\).}
Let \(Y= X\setminus\{\omega_1\}\). Now
\(\{P_y: y\in Y\}\) is an open covering of $Y$.
By exercise's, assumption \(|P_y| \leq \aleph_0\) for all \(y\in Y\).
If by negation $X$ is \(\sigma\)-compact, then be looking at the sub-compacts
\(X=\cup_{i\in\N} K_n\), the finite sub-covering of each and finally
joining the sub-coverings to a countable sub-covering
\begin{equation*}
 Y = \bigcup_{i\in N} P_{y_i}.
\end{equation*}
But then \(|Y| \leq |\N|\cdot\aleph_0 = \aleph_0\)
contradiction to the fact \(|Y|=|X|>\aleph_0\).

\paragraph{Constant Tail.}
Indeed some ``set-theoretic skill'' is needed.
Here we use some terms and results from \cite{Dug1966}.
\paragraph{Definitions} (\cite{Dug1966} \textsf{II 3.1}).
Let $W$ be a well ordered set
\index{ideal!set}
\index{initial interval}
The set of ideals \(I(W)\),
an initial interval \(W(a)\) for each \(a\in W\))
and the set of initial interval \(J(W)\)
are defined as follows:
\begin{eqnarray*}
I(W) &=& \{S\subset W: \forall x\in S\wedge y < x \Rightarrow y\in S \\
W(a) &=& \{x\in W: x < a \wedge x\neq a\}  \\
J(W) &=& \{W(a): a\in W\}
\end{eqnarray*}

%%%%%%%%%%%%%%%%%%%%%%%%%%%%%%%%
\begin{llem} \label{llem:set:ideals}
\textnormal{(\cite{Dug1966} \textsf{II 3.2(a)})}
Let $W$ be a well ordered set, then \(J(W) = I(W) \setminus \{W\}\).
\end{llem}
\begin{thmproof}
\(J(W) \subset I(W) - \{W\}\): Each initial interval is obviously an ideal
but not the whole $W$.
Conversely, let \(S\in I(W) \setminus \{W\}\) Then
\(U = W\setminus S\neq \emptyset\), so $U$ has a first element $a$.

We now show that \(W(a)=S\).
If \(x\in W(a)\), and if by negation \(x notin S\) then \(x\in U\)
contradiction to \(a=\min(U)\) since \(x<a\).
Otherwise \(x\notin W(a)\), but then \(a\leq x\) and so \(x\notin S\)
for if \(x\in S\) by being ideal we would have \(a\in S\).
\end{thmproof}

%%%%%%%%%%%%%%%%%%%%%%%%%%%%%%%%
\begin{llem} \label{llem:countable:ub}
\textnormal{(\cite{Dug1966} \textsf{II 9.1})}
Each countable subset of the ordinals \([0,\omega_1)\) has an upper bound in
 \([0,\omega_1)\).
\end{llem}
\begin{thmproof}
Let \(L = [0,\omega_1) \) the set of all ordinals less than \(\omega_1\).
Let \(A\subset L\) and let $S$ be the ideal
\begin{equation} \label{eq:UWa}
  S =
  \bigcup_{\alpha\in A} W(\alpha) = \bigcup_{\alpha\in A} \{x\in L: x< \alpha\}
  \subset L.
\end{equation}
Since the cardinality of the ordinal \(\omega_1\) is not equal
to any smaller ordinal, \(|W(\alpha)| \leq \aleph_0\)
for each \(\alpha < \omega_1\).
So the above \(\ref{eq:UWa}\) is a countable union of countable sets.
% Since \(\aleph_0\cdot \aleph_0 = \aleph_0\) using
\begin{equation*}
 |S| = \aleph_0 \cdot  \aleph_0 = \aleph_0 < \aleph_1 = |L|.
\end{equation*}
Consequently $S$ cannot be isomorphic to $L$. By lemma~\ref{llem:set:ideals},
\(S=[0,\beta)\) for some \(\beta < \omega_1\) and \(\beta\) is
the least upper bound of $A$.
\end{thmproof}

The following lemma solves the ``constant tail'' statement.
Note that in \cite{Dug1966} the underlying  space does \emph{not}
contain (``the last'') \(\omega_1\).

\index{Vickery}
\begin{llem} \textnormal{(\cite{Dug1966} \textsf{III 8.4 Ex.7}
                         \textrm{Vickery})}
 \label{llem:Vickery}
With the above interval topology defined above on \([0,\omega_1]\) and
induced on \([0,\omega_1)\),
If \(f\in C([0,\omega_1])\)
then $f$ is continuous on a tail \([\beta,\omega_1]\).
\end{llem}
\begin{thmproof}
We first assert that
\begin{equation} \label{eq:f:omega:Cauchy}
 \forall n\in\N,
 \exists \alpha_n < \omega_1\,
 \forall \xi \in (\alpha_n,\omega_1)\;
         |f(\xi) - f(\alpha_n)| < 1/n.
\end{equation}
Intuitively --- $f$ behaves like a Cauchy sequence.
By negation, assume
\begin{equation*}
 \exists n_0\in\N,
 \forall \alpha < \omega_1\,
 \exists \xi \in (\alpha,\omega_1)\;
         |f(\xi) - f(\alpha)| \geq 1/n_0.
\end{equation*}
Now we build by induction an increasing sequence \(\{\xi_i\}_{i\in \N}\)
such that \(|f(\xi) - f(\alpha)| \geq 1/n_0\).
In the  \((i+1)\)-th step of the induction we look for the first \(\xi_{i+1}\)
within \((\xi_i,\omega_1)\) that satisfies the hypothesis.
By lemma~\ref{llem:countable:ub}, \(\xi_{i+1}\) have a least upper bound
\(\gamma < \omega_1\). But then $f$ would not be continuous at \(\gamma\).
This proves (\ref{eq:f:omega:Cauchy}).


Now let \(\beta\) be an upper bound of \(\{\alpha_i\}_{i\in \N}\).
Again by lemma~\ref{llem:countable:ub} \(\beta<\omega_1\).
Now if \(\zeta\in(\beta,\omega_1)\), then
\begin{equation*}
|f(\zeta) - f(\beta)| \leq
|f(\zeta) - f(\alpha_n)| + |f(f(\beta) - f(\alpha_n)| = 2/n
\end{equation*}
for every $n$, thus \(f(\zeta) = f(\beta)\), for every \(\zeta\) such that
\(\beta < \zeta < \omega_1\).
As for last point, If by negation \(f(\omega_1)\neq f(\beta)\) then we could
find a base neighborhood interval \(I=S_{\delta}\) of \(\omega_1\)
such that for every \(x\in I\), we have
\begin{equation*}
|f(x)-f(\omega_1)| < |f(\omega_1) - f(\beta)| / 2 > 0
\end{equation*}
But then there must be an \(x\in I\setminus\{\omega_1\}\)
for which we know that \(f(x)=f(\beta)\).
Thus $f$ is constant on \(S_\beta\).
\end{thmproof}

An immediate consequence is:
\begin{llem}
With the above interval topology defined above on \([0,\omega_1]\),
If \(f\in C([0,\omega_1])\)
then $f$ is continuous on a tail \([\beta,\omega_1]\).
\end{llem}
\begin{thmproof}
With such \(f\in C([0,\omega_1])\),
put \(\tilde{f} = f_{|[0,\omega_1)}\in C([0,\omega_1))\).
Applying lemma~\ref{llem:Vickery}, we see that \(\tilde{f}\)
is constant on a tail. By continuity \(f(\omega_1)\) has this tail value,
and so $f$ is constant on a tail.
\end{thmproof}


\paragraph{Intersection of Compacts.}
Let \(\{K_n\}_{n\in\N}\) a countable family of uncountable compact sets
in \([0,\omega_1]\). We note that \(\omega_1\in K_n\) for all \(n\in\N\)
since otherwise each of their upper bound would imply countable cardinality.
Let \(K = \bigcap_{n\in\N} K_n\) It is clealy compact.
Let \(K'=K\setminus \{\omega_1\}\).
Assume by negation $K$ that is countable, and so is \(K'\).
Let $b$ be a an upper bound for \(K'\) whose existence
is implied by lemma~\ref{llem:countable:ub}.

We will build an strictly increasing sequence \(\{a_n\}_{n\in\N}\)
in \([0,\omega_1)\). We look at blocks \(B_k\) of indices of increasing size.
Let
\begin{eqnarray*}
b_0 &=& 0 \\
b_k &=& \sum_{i=1}^k i = k(k+1)/2 \qquad \textrm{for}\; k\in\N \\
B_k &=& \{m\in\N: b_{k-1} < m \leq b_k\}
\end{eqnarray*}
Note that \(\N = \disjunion_{k\in\N} B_k\). Hence
for each \(n\in\N\) there are unique \(k(n)\in\N\) such that \(n\in B_{k(n)}\).
We denote the position of $n$ within \(B_k(n)\)
by \(j(n) = n - b_{k(n)-1}\).

We now define the desired sequence by induction.
\begin{eqnarray}
a_1 &=& \min(K_1) \cap [b+1,\omega_1)  \label{eq:a1K1} \\
a_n &=& \min\left(K_{j(n)} \cap [a_{n-1},\omega_1)\right). \notag
\end{eqnarray}

The sequence \(\{a_n\}_{n\in\N}\) intersects each of the
compact sets \(\{K_n\}_{n\in\N}\) in infinitely many points.
Again by lemma~\ref{llem:countable:ub},
The limit \(u = \lim_{n\in\N} a_n < \omega_1\).
This $u$, must be the same limit of all sub-sequences, in particular
intersection of \(\{a_n\}_{n\in\N}\) with \(K_i\). Thus \(u\in K_i\),
hence \(u\in K'\), but by construction (\ref{eq:a1K1})
we have a contradiction \(u>b\).

\paragraph{The \(\lambda\) measure}
Let \frakM\ be defined as in the exercise. Clearly by definition
it is closed under completion, and it contains $X$ and \(\emptyset\).
To show it is a \(\sigma\)-algebra we need to show it is closed
under countable union. Let \(A_n\in\frakM\) for \(n\in\N\) and
let \(A=\cup_{n\in\N}A_n\). If there exists \(n\in\N\) such that
\(A_n\cup\{\omega_1\}\) contains an uncountable compact subset,
then so does $A$.  Otherwise, for every \(n\in\N\) the sets
\(A_n^c\cup\{\omega_1\}\) each contains an uncountable compact subset.
By previous result so is their intersection
\begin{equation*}
 \bigcap_{n\in\N} A_n^c\cup\{\omega_1\} =
 \left(\bigcap_{n\in\N} A_n^c\right)\cup\{\omega_1\} =
 A^c\cup\{\omega_1\}.
\end{equation*}
Hence, in both cases \(A\in\frakM\).

A closed interval \[a,b\] is uncountable iff \(a<b=\omega_1\). So clearly
\frakM\ contains all closed intervals. By completion closure, \frakM\ contains
all open intervals, hence all Borel sets.

\paragraph{The \(\mu\) measure}
Let \(\mu\) be the measure on \([0,\omega_1]\) provided by
\index{Riesz}
Riesz representation theorem. In the constructive proof
(\cite{RudinRCA80} Theorem 2.14, formula (3)) we have the definition
\begin{equation*}
 \mu(E) = \sup\{\mu(K): K\subset E,\, K\, \textrm{compact}\}.
\end{equation*}
Since the singleton \(\{\omega_1\}\) is compact, and for all \(f\in C(X)\)
we actually have \(f(\omega_1)\) as the value of the functional,
we have the following unique measure
\begin{equation*}
 \lambda(E) = \left\{\begin{array}{l@{\qquad}l}
                      1 & \omega_1 \in E \\
                      0 & \omega_1 \notin E
                     \end{array}\right.
\end{equation*}
for all Borael sets $E$.

%%%%%%%%%%%%%% 19
\begin{excopy}
If \label{ex:2:19}
\(\mu\) is an arbitrary possible measure and \(f\in L^1(\mu)\),
prove that  \(\{x: f(x)\neq 0\}\) has \(\sigma\)-finite measure.
\end{excopy}

Say the measure \(\mu\) is over $X$. Let
\begin{eqnarray*}
A_0 &=& \emptyset  \\
A_n &=& \{x\in X: |f(x)| > 1/n\} \setminus A_{n-1}
\end{eqnarray*}
Now
\begin{equation*}
\int |f|\,d\mu
 =    \cup_{n\in\N} \int_{A_n} |f|\,d\mu
 \geq \cup_{n\in\N} \mu(A_n)/n.
\end{equation*}
If \(mu\) were not \(\sigma\)-finite measure, then there would exist
\(n\in\N\) such that \(\mu(A_n)=\infty\).
But then \(\int |f|\,d\mu = \infty\) contrsicting the
assumption \(f\in L^1(\mu)\).


%%%%%%%%%%%%%%
\begin{excopy}
A positive measure \(\mu\) on a set $X$ is called
\index{sigma-finite@\(\sigma\)-finite}
\emph{\(\sigma\)-finite} if $X$ is a countable union of sets \(X_i\)
with \(\mu(X_i) < \infty\). Prove that \(\mu\) is \(\sigma\)-finite
if and only if there exists  \(f\in L^1(\mu)\)
such that \(f(x)>0\) for every \(x\in X\).
\end{excopy}

One direction was shown in the above exercise~\ref{ex:2:19}.
Conversely, say \(\mu\) is a \(\sigma\)-finite measure over $X$.
Hence we have \(X = \disjunion_{n\in\N} X_i\) and \(\mu(X_n)<\infty\).
Define
\begin{equation*}
f(x) = 2^{-n}\,/\,\max(\mu(X_n),1) \qquad \textrm{if}\; x \in X_i.
% \left\{ \begin{array}{l@{\qquad}l}
\end{equation*}
Now clearly
\begin{equation*}
\int_X |f|\,d\mu
 = \sum_{n\in\N} \int_{X_n} |f|\,d\mu
 \leq \sum_{n\in\N} 2^{-n} = 1.
\end{equation*}


%%%%%%%%%%%%%%%
\end{enumerate}
%%%%%%%%%%%%%%%

 % -*- latex -*-
% $Id: rudinrca3.tex,v 1.4 2008/07/19 08:56:55 yotam Exp $

%%%%%%%%%%%%%%%%%%%%%%%%%%%%%%%%%%%%%%%%%%%%%%%%%%%%%%%%%%%%%%%%%%%%%%%%
%%%%%%%%%%%%%%%%%%%%%%%%%%%%%%%%%%%%%%%%%%%%%%%%%%%%%%%%%%%%%%%%%%%%%%%%
%%%%%%%%%%%%%%%%%%%%%%%%%%%%%%%%%%%%%%%%%%%%%%%%%%%%%%%%%%%%%%%%%%%%%%%%
\chapterTypeout{\ensuremath{L^p}-Spaces} % Chapter 3


%%%%%%%%%%%%%%%%%%%%%%%%%%%%%%%%%%%%%%%%%%%%%%%%%%%%%%%%%%%%%%%%%%%%%%%%
%%%%%%%%%%%%%%%%%%%%%%%%%%%%%%%%%%%%%%%%%%%%%%%%%%%%%%%%%%%%%%%%%%%%%%%%
\section{Notes}

\index{Jensen}
In 3.3~Theorem (Jensen's Inequality) to derive the (2) inequality:
\begin{equation*}
 \varphi(s) \geq \varphi(t) + \beta\varphi(s-t) \qquad (a<s<b)
\end{equation*}
one should check the cases: \(s<t\) and \(t<s\) separately.


%%%%%%%%%%%%%%%%%%%%%%%%%%%%%%%%%%%%%%%%%%%%%%%%%%%%%%%%%%%%%%%%%%%%%%%%
%%%%%%%%%%%%%%%%%%%%%%%%%%%%%%%%%%%%%%%%%%%%%%%%%%%%%%%%%%%%%%%%%%%%%%%%
\section{Inequalities}

Here we bring and use results from \cite{Hardy:1952:I}.

%%%%%%%%%%%%%%%%%%%%%%%%%%%%%%%%%%%%%%%%%%%%%%%%%%%%%%%%%%%%%%%%%%%%%%%%
\subsection{Proportional Vectors}

Two vectors \(\mathbf{a} = (a_i)_{i=1}^n\)
and \(\mathbf{b} = (b_i)_{i=1}^n\) are said to be \emph{proportional}
iff there exist a scalar \(\mu\) such that
\(\mu a = b\)  or \( a = \mu b\).

The following trivial lemmas do not need a detailed proof.

\begin{lem}
Any vector is proportional to the zero vector of same dimensionality.
\end{lem}

\begin{lem}
For non zero vectors, the proportionality
is an equivalence relation.
\end{lem}


\begin{lem} \label{lem:prop:det}
The vectors \(\mathbf{a} = (a_i)_{i=1}^n\)
and \(\mathbf{b} = (b_i)_{i=1}^n\) are proportional
iff \(a_i b_j - a_j b_i = 0\) for all \(1\leq i,j \leq n\).
\end{lem}


\begin{lem}
Given a \(m\times n\) matrix \(A = (a)_{ij}\) where \(1\leq i \leq m\)
and \(1\leq j \leq n\).
The $m$ row vectors are proportional to each other
iff
the $n$ column vectors are proportional to each other.
\end{lem}
\begin{thmproof}
Applying Lemma~\ref{lem:prop:det}.
\end{thmproof}



%%%%%%%%%%%%%%%%%%%%%%%%%%%%%%%%%%%%%%%%%%%%%%%%%%%%%%%%%%%%%%%%%%%%%%%%
\subsection{Cauchy Inequality}

Let's prove the Cauchy inequality.
\begin{thm} \label{thm:cauchy}
\begin{equation} \label{eq:cos}
 \left(\sum_{i=1}^n a_i b_i\right)^2 \leq
 \left(\sum_{i=1}^n a_i^2\right)
 \left(\sum_{i=1}^n b_i^2\right)
\end{equation}
and equality happens iff
\(\mathbf{a}=(a_i)_{1\leq i\leq n}\)
and
\(\mathbf{b}=(b_i)_{1\leq i\leq n}\)
are proportional.
\end{thm}
\begin{thmproof}
Looking at the \(n^2\) square terms sum
\begin{equation}
D = \sum_{1\leq i,j \leq n} (a_i b_j - a_j b_i)^2 \geq 0
\end{equation}
carefully, we see that \(D=0\) iff
\(\mathbf{a}\) and \(\mathbf{b}\) are proportional.
Evaluate
\begin{eqnarray*}
D
  &=& \sum_{1\leq i,j \leq n} (a_i b_j - a_j b_i)^2 \\
  &=& \sum_{\ineqjton} (a_i b_j - a_j b_i)^2 \\
  &=& \sum_{\ineqjton} a_i^2 b_j^2 + a_j^2 b_i^2
                             - 2 a_i a_j b_i b_j \\
  &=& 2 \sum_{\ineqjton} a_i^2 b_j^2 - a_i a_j b_i b_j \\
\end{eqnarray*}

With the following definition of \(\Delta\), the inequality (\ref{eq:cos})
is equivalent to \(\Delta\geq 0\), which we will now show.
\begin{eqnarray*}
 \Delta
 &=& \left(\sum_{i=1}^n a_i^2\right) \left(\sum_{i=1}^n b_i^2\right) -
     \left(\sum_{i=1}^n a_i b_i\right)^2 \\
 &=& \left(\sum_{\ineqjton}{a_i^2 b_j^2} + \sum_{i=1}^n{a_i^2 b_i^2} \right)
     -
     \left(
       \sum_{i=1}^n{a_i^2 b_i^2} +
       \sum_{\ineqjton}{a_i a_j b_i b_j}
     \right) \\
 &=& \sum_{\ineqjton}{a_i^2 b_j^2} - \sum_{\ineqjton}{a_i a_j b_i b_j} \\
 &=& D/2
\end{eqnarray*}
\end{thmproof}


%%%%%%%%%%%%%%%%%%%%%%%%%%%%%%%%%%%%%%%%%%%%%%%%%%%%%%%%%%%%%%%%%%%%%%%%
\subsection{Arithmetic and Geometric Means}

Our main result here is showing that the geometric mean is bounded
by the arithmetic mean. The book \cite{Hardy:1952:I} gives several proofs,
here we follow the shortest (but less intuitive).

\paragraph{Definitions:}
Let \seqn{q} satisfy \(0\leq q_i\leq 1\) and
\(\sum_{i=1}^n q_i = 1\) and let \(a=\seqn{a}\) be non negative scalars.
\begin{itemize}
 \item For \(0<r<\infty\) the $r$-mean is
   \begin{equation}
     \frakM_r(a) = \left(\sum_{i=1}^n q_i a_i^r\right)^{1/r}
   \end{equation}
 \item We call \(\frakM_1(a)\) the \emph{arithmetic mean}.
 \item We define the \emph{geometric mean} by
        \(\frakG(a) = \prod_{i=1}^n a_i^{q_i}\).
\end{itemize}

\begin{lem} \label{lem:Mr:Mr2}
Given the assumption of the above definitions,
\begin{equation} \label{eq:Mr:Mr2}
\frakM_r(a) \leq \frakM_{2r}(a)
\end{equation}
where equality holds
iff \(a_i=a_1\) for all \(1\leq i \leq n\).
\end{lem}
\begin{thmproof}
The (\ref{eq:Mr:Mr2}) inequality
is equivalent (after taking power of \(2r\)) to
\begin{equation*}
\left(\sum_{i=1}^n q_i a_i^r\right)^2  \leq \sum_{i=1}^n q_i a_i^{2r}
\end{equation*}
Appling Cauchy's Theorem~\ref{thm:cauchy} substituting \(a_i\) and \(b_i\)
by \(\sqrt{q_i}\) and \(a_i^r\sqrt{q_i}\) gives
\begin{eqnarray*}
\left(\sum_{i=1}^n q_i a_i^r\right)^2
&=& \left(\sum_{i=1}^n \sqrt{q_i} \cdot (a_i^r\sqrt{q_i})\right)^2 \\
&\leq & (\sum_{i=1}^n \sqrt{q_i}^2)
        (\sum_{i=1}^n (a_i^r\sqrt{q_i})^2) \\
&=& 1 \cdot \sum_{i=1}^n q_i a_i^{2r}
\end{eqnarray*}
Now that inequality was shown, note that equality holds iff
\((\sqrt{q_i})_{i=1}^n\) and
\((a_i^r\sqrt{q_i}_{i=1}^n)\)  are proportional which is equivalent
to \(a_i^r\) all be equal. Since \(a_i\geq 0\), this
is equivalent to all \(a_i\) being equal.
\end{thmproof}


Now we will justify the notation \(\frakM_0 = \frakG\).
\begin{lem} \label{lem:meanr0:g}
Given the assumption of the above definitions,
\begin{equation*}
\lim_{r\to 0} \frakM(a) = \frakG(a)\,.
\end{equation*}
\end{lem}
\begin{thmproof}
Given \(r>0\), we compute
\begin{eqnarray}
\frakM_r(a)
 &=& \notag
  \exp\left(\log\left(\biggl(\sum_{i=1}^n q_i a_i^r\biggr)^{1/r}\right)\right) \\
 &=& \notag
  \exp\left(\frac{1}{r}\log\biggl(\sum_{i=1}^n q_i a_i^r\biggr)\right) \\
 &=& \notag
  \exp\left(\frac{1}{r}\log\biggl(\sum_{i=1}^n q_i a_i^r\biggr)\right) \\
 &=& \label{eq:Mr0G:taylor}
  \exp\biggl(\log\bigl(1 + r\sum_{i=1}^n q_i \log(a_i) + O(r^2)\bigr)\big/r\biggr)
\end{eqnarray}

Where the equality in (\ref{eq:Mr0G:taylor}) is the Taylor expansion
of \(\sum_{i=1}^n q_i a_i^r\)
at \(r=0\). Note \(a_i^r = e^{r\log(a_i)}\) and so
\(\frac{d}{dr} q_i a_i^r = q_i r \log(a_i)a_i^r\)

Before taking limit of the above, we concentrate first on subexpression.
Put
\begin{equation*}
 % L_r = \log\bigl(1 + r\sum_{i=1}^n q_i \log(a_i) + O(r^r)\bigr) \bigm/ r
 U(r) = 1 + r\sum_{i=1}^n q_i \log(a_i) + O(r^r)
\end{equation*}
and by l'Hospital's rule
\begin{eqnarray*}
\lim_{r\to 0} \log(U(r))/r
&=&
 \lim_{r\to 0} \frac{d}{dr} \log(U(r)) \\
&=&
 \lim_{r\to 0} \left(\frac{d}{dr} U(r) \right) \bigm/ U(r)  \\
&=&
 \lim_{r\to 0} \left(\sum_{i=1}^n q_i \log(a_i) + \frac{d}{dr}O(r^r)\right) / 1
 \\
&=& \lim_{r\to 0} \left(\sum_{i=1}^n q_i \log(a_i) + O(r)\right) \\
&=& \sum_{i=1}^n q_i \log(a_i)
\end{eqnarray*}


Finally the computation of the desired limit
\begin{eqnarray*}
\lim_{r\to 0} \frakM_r(a)
&=&
 \lim_{r\to 0}
 \exp\biggl(\log\bigl(1 + r\sum_{i=1}^n q_i \log(a_i) +
                      O(r^2)\bigr)\big/r\biggr) \\
&=&  \lim_{r\to 0} \exp(\log(U(r))/r) \\
&=&  \exp\left(\sum_{i=1}^n q_i \log(a_i)\right) \\
&=&  \prod_{i=1}^n a_i^{q_i} \\
&=&  \frakG(a)
\end{eqnarray*}
\end{thmproof}

The goal of the this section is to compare the
geometric mean \(\frakG(a)\) with the arithmetic mean \(\frakM_1(a)\)
\begin{thm} \label{thm:geo:arith}
Given the assumption of the above definitions,
\begin{equation*}
\frakG(a) \leq \frakM_1(a)
\end{equation*}
where equality holds iff
iff \(a_i=a_1\) for all \(1\leq i \leq n\).
\end{thm}
\begin{thmproof}
Utilizing lemma~\ref{lem:Mr:Mr2} and lemma~\ref{lem:meanr0:g},
we compute
\begin{equation*}
 \frakM_1(a) \geq \frakM_{\frac{1}{2}}(a)
 \cdots     \geq \frakM_{2^{-k}}(a)
 \cdots     \geq \lim_{m\to\infty} \frakM_{2^{-m}}(a) = \frakG(a).
\end{equation*}
The inequalities are equalities, as was shown in lemma~\ref{lem:Mr:Mr2},
iff \(a_i\)'s are constant.
\end{thmproof}



%%%%%%%%%%%%%%%%%%%%%%%%%%%%%%%%%%%%%%%%%%%%%%%%%%%%%%%%%%%%%%%%%%%%%%%%
\subsection{Generalizing Cauchy Inequality}

%%%%%%%%%%%%%%%%%%%%%%%%%%%%%%%%
\begin{lem} \label{lem:eqgeom}
Let $n$ and $g=2^m$ be positive integers,
if \(a_{ij}\geq 0\)
for all \(1\leq i \leq n\), \(1\leq j \leq g\),
then
\begin{equation} \label{eq:eqgeom}
 \left(\sum_{i=1}^n \prod_{j=1}^g a_{ij}\right)^g
 \leq
 \prod_{j=1}^g \sum_{i=1}^n a_{ij}^g
\end{equation}
Equality happens iff (the columns)
\(\mathbf{a}_j = (a_{ij})_{i=1}^n\) are proportional, or at least
one of them is all zero.
\end{lem}
\begin{thmproof}
The case where some \(\mathbf{a}_j\) is all zeros is trivial,
so we exclude it from the rest of the proof.
By induction on $m$.
Assume \(m=0\), then \(g=1\) and (\ref{eq:eqgeom}) is a trivial equality.
Also being single dimensional vectors, they are proportional.
The case of \(m=1\), that is \(g=2\) is proved in Theorem~\ref{thm:cauchy}.
Now assume that the lemma holds for \(m=k\geq 1\).
Consider the case of \(m=k+1\)
\begin{eqnarray}
 \left(\sum_{i=1}^{n} \prod_{j=1}^{2^{k+1}} a_{ij}\right)^{2^{k+1}}
 &=& \notag
 \left(\sum_{i=1}^{n}
       \Biggl(
       \biggl(\prod_{j=1}^{2^k} a_{ij}\biggr)
       \biggl(\prod_{j=2^k + 1}^{2^{k+1}} a_{ij}\biggr)
       \Biggr)
 \right)^{2\cdot 2^k} \\
 &\leq& \label{eq:eqgeom:induc1}
 \left(
 \Biggl(\sum_{i=1}^{n}
       \biggl(\prod_{j=1}^{2^k} a_{ij}\biggr)^2
 \Biggr)
 \cdot
 \Biggl(\sum_{i=1}^{n}
       \biggl(\prod_{j=2^k+1}^{2^{k+1}} a_{ij}\biggr)^2
 \Biggr)
 \right)^{2^k} \\
 &=& \notag
 \left(\sum_{i=1}^{n}
       \prod_{j=1}^{2^k} a_{ij}^2
 \right)^{2^k}
 \cdot
 \left(\sum_{i=1}^{n}
       \prod_{j=2^k+1}^{2^{k+1}} a_{ij}^2
 \right) ^{2^k}
 \\
 &\leq& \label{eq:eqgeom:induc2}
 \left(\prod_{j=1}^{2^k} \,\sum_{i=1}^n a_{ij}^{2^{k+1}}\right)
 \cdot
 \left(\prod_{j=2^k+1}^{2^{k+1}} \;\sum_{i=1}^n a_{ij}^{2^{k+1}}\right) \\
 &=& \notag
 \prod_{j=1}^{2^{k+1}} \,\sum_{i=1}^n a_{ij}^{2^{k+1}}
\end{eqnarray}
The inequality (\ref{eq:eqgeom:induc1}) is from Cauchy's Theorem~\ref{thm:cauchy}
and the inequality (\ref{eq:eqgeom:induc2}) by induction.
The inequalities are equality iff \(\mathbf{a}_i\) are proportional.
\end{thmproof}

Now we remove the restriction of $g$ being a power of $2$.
%%%%%%%%%%%%%%%%%%%%%%%%%%%%%%%%
\begin{lem} \label{lem:cauchy:ng}
Let $n$ and $g$ be positive integers,
if \(a_{ij}\geq 0\)
for all \(1\leq i \leq n\), \(1\leq j \leq g\),
then
\begin{equation} \label{eq:eqgeom:n}
 \left(\sum_{i=1}^n \prod_{j=1}^g a_{ij}\right)^g
 \leq
 \prod_{j=1}^g \sum_{i=1}^n a_{ij}^g
\end{equation}
Equality happens iff (the columns)
\(\mathbf{a}_j = (a_{ij})_{i=1}^n\) are proportional, or at least
one of them is all zero.
\end{lem}
\begin{thmproof}
The case where some \(\mathbf{a}_i\) is all zeros is trivial,
so we exclude it from the rest of the proof.

If the vectors are proportional, then there are \((c_i)_{i=1}^n\)
such that \(a_{ij} = c_i a_{1j}\)
for all \(1\leq i \leq n\), \(1\leq j \leq g\). Inthis case
\begin{equation*}
\left(\sum_{i=1}^n \prod_{j=1}^g a_{ij}\right)^g =
\left(\sum_{i=1}^n c_i \prod_{j=1}^g a_{1j}\right)^g =
\left(\left(\sum_{i=1}^n c_i\right) \prod_{j=1}^g a_{1j}\right)^g =
\left(\prod_{j=1}^g \sum_{i=1}^n c_i a_{1j}\right)^g
\end{equation*}


The case where \(g=2^m\) for some integer $m$ was proved
in the previous Lemma.
Let $m$ be a positive integer such that \(2^{m-1} < g < 2^m\).

Put
\begin{equation*}
T_i = \prod_{j=1}^g a_{ij}^{1/{2^m}}
\end{equation*}
and define \(2^m\)-dimensional vectors
\begin{equation*}
b_{ij} = \left\{\begin{array}{ll}
               a_{ij}^{g/{2^m}} \quad & 1 \leq j \leq g\\
               T_i              \quad & g <    j \leq 2^m
               \end{array}\right.
               \qquad \textrm{(for}\quad 1\leq i \leq n, \;
                                         1 \leq j \leq 2^m \textrm{)}
\end{equation*}
Note that
\begin{equation*}
\prod_{j=1}^{2^m} b_{ij}
= \left(\prod_{j=1}^{g} a_{ij}^{g/{2^m}} \right)
  \left(\prod_{i=g+1}^{2^m} T_i \right)
= T_i^g T_i^{2^m-g} = T_i^{2^m}
= \prod_{j=1}^g a_{ij}
\end{equation*}

Applying Lemma~\ref{lem:eqgeom} we have
\begin{equation*}
 \left(\sum_{i=1}^n \prod_{j=1}^{2^m} b_{ij}\right)^{2^m}
 \leq
 \prod_{j=1}^{2^m} \sum_{i=1}^n b_{ij}^{2^m}
\end{equation*}
With this we can compute
\begin{eqnarray}
\left(\sum_{i=1}^n \prod_{j=1}^g a_{ij}\right)^{2^m}
&=&  \left(\sum_{i=1}^n \prod_{j=1}^{2^m} b_{ij}\right)^{2^m} \notag \\
&\leq& \label{eq:eqgeom:useinduc}
    \prod_{j=1}^{2^m} \sum_{i=1}^n b_{ij}^{2^m} \\
&=& \notag
        \left(\prod_{j=1}^{g} \sum_{i=1}^n b_{ij}^{2^m}\right)
        \left(\prod_{j=g+1}^{2^m} \sum_{i=1}^n b_{ij}^{2^m}\right) \\
&=&     \notag
        \left(\prod_{j=1}^{g} \sum_{i=1}^n a_{ij}^g\right)
        \left(\sum_{i=1}^n \prod_{j=1}^g a_{ij} \right)^{2^m - g}
\end{eqnarray}

The inequality (\ref{eq:eqgeom:useinduc})
is by previous lemma and it is equality iff the columns vectors
\(\mathbf{b}_j = (b_{ij})_{i=1}^n\) are proportional.
It is easy to see that the latter condition is equivalent
to the columns vectors \(\mathbf{a}_j\) being proportional.

Now \(F=\sum_{i=1}^n \prod_{j=1}^g a_{ij} = 0\) iff
\(\prod_{j=1}^g a_{ij} = 0\) for all \(1\leq i \leq n\).
In this case (\ref{eq:eqgeom:n}) holds with strict inequality,
since all the columns \(\mathbf{a}_j=0\), as the other case was excluded.

So we now may assume \(\sum_{i=1}^n \prod_{j=1}^g a_{ij} \neq 0\).
Thus from the recent inequality, we can derive (dividing by \(F^{2^m-g}\))
\begin{equation*}
\left(\sum_{i=1}^n \prod_{j=1}^g a_{ij}\right)^g
\leq \left(\prod_{j=1}^{g} \sum_{i=1}^n a_{ij}^g\right)
\end{equation*}
\end{thmproof}

%%%%%%%%%%%%%%%%%%%%%%%%%%%%%%%%%%%%%%%%%%%%%%%%%%%%%%%%%%%%%%%%%%%%%%%%
\subsection{Hold\"er Inequality}

We begin with special cases that will support
the proof of the general Holder Inequality.

\begin{lem} \label{lem:holder:eq}
Let $n$ and $g$ be positive integers,
if \(a_{ij}\geq 0\)
for all \(1\leq i \leq n\), \(1\leq j \leq g\),
then
\begin{equation}
 \sum_{i=1}^n \prod_{j=1}^g a_{ij}^{1/g}
 \leq
 \prod_{j=1}^g \left(\sum_{i=1}^n a_{ij}\right)^{1/g}
\end{equation}
Equality happens iff (the columns)
\(\mathbf{a}_j = (a_{ij})_{i=1}^n\) are proportional, or at least
one of them is all zero.
\end{lem}
\begin{thmproof}
Using Lemma~\ref{lem:cauchy:ng},
but substituting \(a_{ij}\) with \(a_{ij}^{1/g}\)
we have
\begin{equation*}
 \left(\sum_{i=1}^n \prod_{j=1}^g a_{ij}^{1/g}\right)^g
 \leq
 \prod_{j=1}^g \sum_{i=1}^n (a_{ij}^{1/g})^g
\end{equation*}
Equivelantly
\begin{equation*}
 \sum_{i=1}^n \prod_{j=1}^g a_{ij}^{1/g}
 \leq
 \prod_{j=1}^g \left(\sum_{i=1}^n a_{ij}\right)^{1/g}
\end{equation*}
and equality happens as was needed to show, since \(\mathbf{a}_j\)
are proportional iff \(\overline{\mathbf{a}}_j = (a_{ij}^{1/g})_{i=1}^n\)
that were used here are proportional.
\end{thmproof}


Next generalization would be with varying powers.
\begin{lem} \label{lem:holder:rat}
Let $n$ and $g$ be positive integers,
if \(a_{ij}\geq 0\)
for all \(1\leq i \leq n\), \(1\leq j \leq g\),
then
if \((\alpha_j)_{j=1}^g\) are rationals satisfying \(0\leq \alpha_j \leq 1\)
and \(\sum_{j=1}^g \alpha_j = 1\) then
\begin{equation}
 \sum_{i=1}^n \prod_{j=1}^g a_{ij}^{\alpha_j}
 \leq
 \prod_{j=1}^g \left(\sum_{i=1}^n a_{ij}\right)^{\alpha_j}
\end{equation}
Equality happens iff (the columns)
\(\mathbf{a}_j = (a_{ij})_{i=1}^n\) are proportional, or at least
one of them is all zero.
\end{lem}
\begin{thmproof}
Being rationals, we can find integers $M$ and \(p_j\) such that
\(\alpha_j = p_j/M\) for \(1\leq j \leq g\).
By viewing
\begin{equation*}
a_{ij}^{p_j/M} = \left(a_{ij}^{1/M}\right)^{p_j}
\end{equation*}
noting that \(\sum_{j=1}^g p_j = M\),
by converting to equal power (\(1/M\)),
we use Leamm~\ref{lem:holder:eq}
\begin{equation*}
     \sum_{i=1}^n \prod_{j=1}^g a_{ij}^{\alpha_j}
 =   \sum_{i=1}^n \prod_{j=1}^g \left(a_{ij}^{1/M}\right)^{p_j}
\leq \prod_{j=1}^g \left(\left(\sum_{i=1}^n a_{ij}\right)^{1/M}\right)^{p_j}
=    \prod_{j=1}^g \left(\sum_{i=1}^n a_{ij}\right)^{p_j/M}
=    \prod_{j=1}^g \left(\sum_{i=1}^n a_{ij}\right)^{\alpha_j}
\end{equation*}
Again the inequality is an equality, when the same conditions
required in Lemma~\ref{lem:holder:eq} hold.
\end{thmproof}

Finally removing the restriction for \(\alpha_i\in\Q\),
\textbf{Hold\"er}'s inequality,
\begin{llem} \label{lem:holder}
Let $n$ and $g$ be positive integers,
if \(a_{ij}\geq 0\)
for all \(1\leq i \leq n\), \(1\leq j \leq g\),
then
if \((\alpha_j)_{j=1}^g\) are reals satisfying \(0\leq \alpha_j \leq 1\)
and \(\sum_{j=1}^g \alpha_j = 1\) then
\begin{equation} \label{eq:holder}
 \sum_{i=1}^n \prod_{j=1}^g a_{ij}^{\alpha_j}
 \leq
 \prod_{j=1}^g \left(\sum_{i=1}^n a_{ij}\right)^{\alpha_j}
\end{equation}
Equality happens iff (the columns)
\(\mathbf{a}_j = (a_{ij})_{i=1}^n\) are proportional, or at least
one of them is all zero.
\end{llem}
\begin{thmproof}
By applying limit process with rationals converting to reals to
Lemma~\ref{lem:holder:rat} we get the desired result, except for the
(temporary) loss of strict inequality (for the non proportional case).
So for now we know that
\begin{itemize}
 \item  If the vectors \(\mathbf{a}_j\) are proportional
        or at least one of the is zero,
        then equality holds in (\ref{eq:holder}).
 \item  Otherwise, (\ref{eq:holder}) holds.
\end{itemize}
We would now show that in the second case, it is indeed strict inequality.

We now assume that  \(\alpha_j \neq 0\)
for all \(1\leq j \leq g\). Otherwise,
we can simply drop the corresponding $j$-th columns,
without effecting the expression values in (\ref{eq:holder}).
For \(1\leq j \leq g\) we
partition \(\alpha_j = q_j + \beta_j\)
such that both \(q_j, \beta_j > 0\) and \(q_j \in \Q\).
Put,
\(q = \sum_{j=1}^g q_j\) and \(\beta = \sum_{j=1}^g \beta_j\).
Clearly \(q+\beta=1\) and \(q,\beta \in \Q\).
Now we define
\begin{eqnarray*}
Q_i &=& \prod_{j=1}^g a_{ij}^{q_j/q} \\
B_i &=& \prod_{j=1}^g a_{ij}^{\beta_j/\beta}.
\end{eqnarray*}
We know
\begin{eqnarray}
\sum_{i=1}^n Q_i = \sum_{i=1}^n \prod_{j=1}^g a_{ij}^{q_j/q}
  &<& \label{eq:holder:Qlt}
      \prod_{j=1}^g \left(\sum_{i=1}^n  a_{ij}\right)^{q_j/q} \\
\sum_{i=1}^n B_i = \sum_{i=1}^n \prod_{j=1}^g a_{ij}^{\beta_j/\beta}
  &\leq& \label{eq:holder:Bleq}
      \prod_{j=1}^g \left(\sum_{i=1}^n  a_{ij}\right)^{\beta_j/\beta}
\end{eqnarray}
where (\ref{eq:holder:Qlt}) resulted by Lemma~\ref{lem:holder:rat}
while (\ref{eq:holder:Bleq}) was shown in the beginning of the proof.
Finally
\begin{eqnarray*}
 \sum_{i=1}^n \prod_{j=1}^g a_{ij}^{\alpha_j}
 =    \sum_{i=1}^n Q_i^q B_i^\beta
 &\leq& \left(\sum_{i=1}^n Q_i\right)^q \left(\sum_{i=1}^n  B_i\right)^\beta \\
 &<&
    \prod_{j=1}^g
          \left(\sum_{i=1}^n  a_{ij}\right)^{q_j}
          \left(\sum_{i=1}^n  a_{ij}\right)^{\beta_j}
 =   \prod_{j=1}^g
          \left(\sum_{i=1}^n  a_{ij}\right)^{\alpha_j}
\end{eqnarray*}
\end{thmproof}


%%%%%%%%%%%%%%%%%%%%%%%%%%%%%%%%%%%%%%%%%%%%%%%%%%%%%%%%%%%%%%%%%%%%%%%%
\subsection{H\"older's Inequality for Integrals}

\paragraph{Definition.} Two functions $f$, $g$
are said to be \emph{effectively proportional} if
iff there exist a scalar \(\mu\) such that
\(\mu f = g \;\aded\)  or \(f = \mu g\;\aded\).


\begin{llem} \textnormal{(\cite{Hardy:1952:I} \textbf{188})}
\label{llem:hlp:188}
If $k$ is an integer and
\begin{itemize}
 \item \seq{q}{k} positive such that \(\sum_{i=1}^k q_i = 1\).
 \item \seq{f}{k} are measurable functions on \(X\to [0,\infty]\).
 \item \(\mu\) a positive measure on $X$.
\end{itemize}
Then
\begin{equation} \label{eq:HLP:188}
% \left\int_X \prod_{i=1}^k f_i^{q_i}\,d\mu\right. \leq
\int_X \prod_{i=1}^k f_i^{q_i}\,d\mu \leq
\prod_{i=1}^k \left(\int_X f_i\,d\mu\right)^{q_i}
\end{equation}
where equality holds iff \seqn{f} are effectively proportional.
\end{llem}

\begin{thmproof}
Geometric means is less or equation arithmetic mean as shown
in \cite{RudinRCA80} [page 64, (8)]
and also ``here'' in Theorem~\ref{thm:geo:arith}.
Thus
\begin{eqnarray*}
\frac{\int_X \prod_{i=1}^k f_i^{q_i}\,d\mu}{
 \prod_{i=1}^k \left(\int_X f_i\,d\mu\right)^{q_i}}
 &=& \int_X \left(\frac{f_i}{\int_X f_i\,d\mu}\right)^{q_i}\,d\mu \\
 &\leq& \int_X \sum_{i=1}^k \frac{q_i f_i}{\int_X f_i\,d\mu}\,d\mu \\
 &=& 1.
\end{eqnarray*}
and (\ref{eq:HLP:188}) is clear. By local lemma~\ref{lem:holder}
an equality happens iff
the functions \(f_i/{\int_X f_i\,d\mu}\) are effectively proportional,
or equivalently \(f_i\) are.
\end{thmproof}


%%%%%%%%%%%%%%%%%%%%%%%%%%%%%%%%%%%%%%%%%%%%%%%%%%%%%%%%%%%%%%%%%%%%%%%%
\subsection{Jensen's Strict Inequality}

We would now give a variation of Theorem~3.3 (Jensen's Inequality)
\cite{RudinRCA80}. We start with some definition and trivial results.

\textbf{Definition} A real functions \(\varphi\) defined on
a segment \((a,b)\) where \(-\infty\leq a < b \leq \infty\)
is called \emph{strictly convex} if the inequality
\begin{equation} \label{eq:convex:def}
 \varphi\bigl((1-\lambda)x + \lambda y\bigr) <
 (1-\lambda)\varphi(x) + \lambda\varphi(y)
\end{equation}
holds whenever \(a<x<b\), \(a<y<b\),  and \(0<\lambda<1\).

Note the differences with the Definition~3.1 of \emph{convex} function,
both with the strict inequality and avoiding \(\lambda=0,1\) cases.

\iffalse
\begin{llem} \label{lem:convex:stu}
If \(\varphi\) is strictly convex on \((a,b)\)
then for any $s$,$t$,$u$ such that \(s<t<u\)
\begin{equation} \label{eq:convex:stu}
\frac{\varphi(t) - \varphi(s)}{t-s} < \frac{\varphi(u) - \varphi(t)}{u-t}
\end{equation}
\end{llem}
\begin{thmproof}
Using \(\lambda = (t-s)(u-s)\),
we make several simple derivations from (\ref{eq:convex:def})
\begin{eqnarray}
 \varphi(t) &<& \frac{u-t}{u-s}\varphi(s) + \frac{t-s}{u-s}\varphi(t) \notag \\
  (u-t \;+\; t-s)\varphi(t) &<& (u-t)\varphi(s) + (t-s)\varphi(t) \notag \\
  (u-t)\bigl((\varphi(t) - \varphi(s)\bigr)
 &<& \notag
  (t-s)\bigl(\varphi(u) - \varphi(t)\bigr) \\
%%%
  u\varphi(t) - u\varphi(s) - t\varphi(t) + t\varphi(s)
 &<& \notag
  t\varphi(u) - t\varphi(s) - s\varphi(u) + s\varphi(t) \\
%%%
  (u-t)\bigl(\varphi(t) - \varphi(s)\bigr)
&<& \notag
    (t-s)\bigl(\varphi(u) - \varphi(t)\bigr) \\
%%%
\frac{\varphi(t) - \varphi(s)}{t-s} &<& \frac{\varphi(u) - \varphi(t)}{u-t}
\end{eqnarray}
\end{thmproof}
\fi

The following lemma shows that the divided differences of
a convex function on \((a,b))\) viewed as a function of
two variables on \(\{(x,y): a<x<y,\; a<y<b,\; x\neq y\}\)
is strictly increasing in each variable.

\begin{llem} \label{lem:convex:stu}
If \(\varphi\) is strictly convex on \((a,b)\)
then for any $s$,$t$,$u$ such that \(s<t<u\)
\begin{equation} \label{eq:convex:stu}
\frac{\varphi(t) - \varphi(s)}{t-s}
 < \frac{\varphi(u) - \varphi(s)}{u-s}
 < \frac{\varphi(u) - \varphi(t)}{u-t}
\end{equation}
\end{llem}
\begin{thmproof}
To express $t$ as a convex combination
\((1-\lambda)s + \lambda u\)
of $s$ and $u$, we
use \(\lambda = (t-s)(u-s)\) and
 \(1 - \lambda = (u-t)(u-s)\).

We put
\begin{equation*}
T = (1-\lambda)\varphi(s) + \lambda\varphi(u)
  = \frac{u-t}{u-s} \varphi(s) + \frac{t-s}{u-s} \varphi(u).
\end{equation*}
Let's first establish the trivial slope equalities
\begin{equation} \label{eq:convex:slope}
 \frac{T-\varphi(s)}{t-s}
 = \frac{\varphi(u)-\varphi(s)}{u-s}
 = \frac{\varphi(u) - T}{u-t}.
\end{equation}
Left equality:
\begin{eqnarray*}
 \frac{T-\varphi(s)}{t-s}
 &=& \frac{(u-s)(T-\varphi(s))}{(u-s)(t-s)}
 \;=\; \frac{(u-s)T - (u-s)\varphi(s)}{(u-s)(t-s)} \\
 &=& \frac{(u-t)\varphi(s) + (t-s)\varphi(u) - (u-s)\varphi(s)}{
           (u-s)(t-s)} \\
 &=& \frac{(t-s)\varphi(u) + (s-t)\varphi(s)}{(u-s)(t-s)} \\
 &=& \frac{\varphi(u)-\varphi(s)}{u-s}
\end{eqnarray*}
Similarly, the right equality:
\begin{eqnarray*}
 \frac{\varphi(u) - T}{u-t}
 &=& \frac{(u-s)(\varphi(u)-T)}{(u-s)(u-t)}
 \;=\; \frac{(u-s)\varphi(u) - (u-s)T)}{(u-s)(u-t)} \\
 &=& \frac{(u-s)\varphi(u) - (u-t)\varphi(s) - (t-s)\varphi(u)}{
           (u-s)(u-t)} \\
 &=& \frac{(u-t)\varphi(u) - (u-t)\varphi(s)}{(u-s)(u-t)} \\
 &=& \frac{\varphi(u)-\varphi(s)}{u-s}
\end{eqnarray*}
By strict convexirty, \(\varphi(t)<T\) and so we get
\begin{eqnarray*}
\frac{\varphi(t) - \varphi(s)}{t-s}
 &<& \frac{T - \varphi(s)}{t-s} \\
 &=& \frac{\varphi(u) - \varphi(s)}{u-s} \\
 &=& \frac{\varphi(u) - T}{u-t} \\
 &<& \frac{\varphi(u) - \varphi(t)}{u-t}
\end{eqnarray*}
that contains the desired double inequalitiy.
\end{thmproof}

\begin{llem}
If \(\varphi\) is strictly convex, then any value is assumed mostly twice.
\end{llem}
\begin{thmproof}
For any \(x<y<z\) in the domain of $f$, if \(f(x)=f(z)\)
then by strictly convexity \(f(y) < f(x)\).
\end{thmproof}

Here is general simple result (not related to convexity)
\begin{llem} \label{lem:fgz:igz}
If \(f>0\) is a \(\mu\)-measaurable function, on $X$ and \(\mu(X) > 0\)
then
\begin{equation*}
\int_X f\,d\mu > 0.
\end{equation*}
\end{llem}
\begin{thmproof}
Define measurable subsets: \(U_0 = \{x:X: f(x)>1\}\) and for \(j>0\) define
\(U_j = \{x:X: 1/(j+1) < f(x) \leq 1/j\}\).
Clearly \(X = \disjunion_{j=0}^\infty U_j\), and for at least some $j$
we have \(\mu(U_j) > 0\), thus \(\int_X f\,d\mu \geq \mu(U_j)/(j+1) > 0\).
\end{thmproof}

%%%%%%%%%%%%%%%%%%%%%%%%%%%%%%%%%%%%%%%%%%%%%%%%%%%%%%%%%%%%%%%%%%%%%%%%
Now for this section main result. The proof is also a variant
of that of Theorem~3.3 (\cite{RudinRCA80}).
\begin{llem} \label{lem:jensen:strict}
Let \(\mu\) be a positive measure on 
a~\(\sigma\)-algebra \frakM in a set \(\Omega\), so that \(\mu(\Omega)=1\).
If $f$ is a real function in \(L^1(\mu)\),
which is not constant \aded, satisfying
 \(a<f(x)<b\) for all \(x\in \Omega\)
and
if \(\varphi\) is strictly convex on \((a,b)\), then
\begin{equation} \label{eq:jensen:strict}
 \varphi\left(\int_\Omega f\,d\mu\right) <  \int_\Omega (\varphi\circ f)\,d\mu
\end{equation}
\end{llem}
\begin{thmproof}


Put \(t=\int_\Omega f,d\mu\). Then \(a<t<b\).

We split \(\Omega\) to a disjoint union of the following measurable subsets.
\begin{eqnarray*}
\Omega^{-} &=& \{x\in\Omega: f(x) < t\} \\
\Omega^{=} &=& \{x\in\Omega: f(x) = t\} \\
\Omega^{+} &=& \{x\in\Omega: f(x) > t\} \\
\end{eqnarray*}
By the assumption,
\(\mu(\Omega^{-}) > 0\)
or
\(\mu(\Omega^{+}) > 0\).


We look at the inequality of local lemma~\ref{lem:convex:stu}.
Let
\begin{eqnarray*}
 \beta  &\eqdef& \sup_{a<s<t} \frac{\varphi(t) - \varphi(s)}{t-s}\\
 \gamma &\eqdef& \inf_{t<u<b} \frac{\varphi(u) - \varphi(t)}{u-t}.
\end{eqnarray*}
By local lemma~\ref{lem:convex:stu},
For any $s$, $t$ such that \(a<s<t<u<b\) we have
\begin{equation*}
\frac{\varphi(t)-\varphi(s)}{t-s}
 < \beta \leq \gamma < \frac{\varphi(u)-\varphi(t)}{u-t}
\end{equation*}

{\small (Now we could arbitrary proceed using either \(\beta\) or \(\gamma\),
we choose \(\beta\) similar to the proof mention above).}
Thus we have two inequalities
\begin{eqnarray*}
  \varphi(s) &<& \varphi(t) + \beta(t-s)\\
  \varphi(u) &>& \varphi(t) + \beta(u-t)
\end{eqnarray*}
combined to
\begin{equation*}
  \varphi(r) \,>\, \varphi(t) + \beta(r-t)
\end{equation*}
for any $r$ such that \(a<r<b\) and \(r\neq t \).
Hence
\begin{eqnarray*}
  \varphi(f(x)) - \varphi(t) > \beta(f(x)-t)
       & \qquad \textrm{for}\; x\in\Omega^{-}\cup\Omega^{+} \\
  \varphi(f(x)) - \varphi(t) = \beta(f(x)-t)
       & \qquad \textrm{for}\; x\in\Omega^{=}
\end{eqnarray*}
Since \(\varphi\) is continuous, \(\varphi\circ f\) is measurable.
We integrate the above expression by $x$. Using local lemma~\ref{lem:fgz:igz}
gives
% \int_\Omega  \varphi(f(x)) - \varphi(t) + \beta(f(x)-t)\;d\mu
%% \begin{equation*}
%% \int_\Omega (\varphi\circ f)\,d\mu - \varphi\left(\int_\Omega f\,d\mu\right)
%%  > \int_\Omega \beta(f(x)-t)\,d\mu = 0.
%% \end{equation*}
\begin{eqnarray*}
\int_\Omega (\varphi\circ f)\,d\mu - \varphi\left(\int_\Omega f\,d\mu\right)
&=& \int_\Omega (\varphi\circ f)\,d\mu - \varphi(t) \\
&=& \int_\Omega \bigl(\varphi\circ f - \varphi(t)\bigr) \,d\mu \\
&=& \int_{\Omega^=} \bigl(\varphi\circ f- \varphi(t)\bigr) \,d\mu
 +  \int_{\Omega^-\cup\Omega^+} \bigl(\varphi\circ f- \varphi(t)\bigr) \,d\mu \\
&=& \int_{\Omega^=} \beta(f(x) - t)\,d\mu
 +  \int_{\Omega^-\cup\Omega^+} \bigl(\varphi\circ f- \varphi(t)\bigr) \,d\mu \\
&>& \int_{\Omega^=} \beta(f(x) - t)\,d\mu
   +  \int_{\Omega^-\cup\Omega^+} \beta(f(x) - t)\,d\mu \\
&=&  \int_\Omega \beta(f(x) - t)\,d\mu \\
&=& 0.
\end{eqnarray*}

Which gives the desired inequality (\ref{eq:jensen:strict}).
\end{thmproof}




%%%%%%%%%%%%%%%%%%%%%%%%%%%%%%%%%%%%%%%%%%%%%%%%%%%%%%%%%%%%%%%%%%%%%%%%
\subsection{Technical Calculus Results}

Here is a simple technical lemma, using basic calculus.
\begin{llem} \label{lem:fg:bnless}
Let \(f,g:[0,\infty)\to[0,\infty)\) functions, satsisfying
\begin{itemize}
 \item \emph{Boundness}: \(\int_0^M g(x)\,dx < \infty\)  for any \(M<\infty\).
 \item \emph{Boundless}: \(\int_0^\infty g(x)\,dx = \infty\).
 \item \emph{Dominance}:
   For every \(x\in[0,\infty)\), there exists \(\epsilon_x > 0\),
   such that
   \begin{equation*}
    f(x) > (1-\epsilon_x)g(x)
   \end{equation*}
   and \(\lim_{x\to\infty} \epsilon_x = 0\).
\end{itemize}
Then for every \(\eta>0\), there exists \(A<\infty\) such that
\begin{equation*}
 \int_0^a f(x)\,dx > (1 - \eta)\int_0^a g(x)\,dx
\end{equation*}
\end{llem}
\begin{thmproof}
Given (small) \(\eta>0\), let $h$ be such that \(\epsilon_x < \eta/2\)
for every \(x\geq h\). By the first two ssumptions,
there exists \(a<\infty\) such that
\begin{equation*}
 (1-\eta/2)\int_h^a g(x)\,dx > (1-\eta) \int_0^h g(x)\,dx.
\end{equation*}
Now we can derive the desired estimate
\begin{eqnarray*}
 \int_0^a f(x)\,dx
 &>& \int_0^a (1-\epsilon_x)g(x)\,dx  \\
 &=&   \int_0^h (1-\epsilon_x)g(x)\,dx  + \int_h^a (1-\epsilon_x)g(x)\,dx  \\
 &\geq& (1-\epsilon_h)\int_h^a g(x)\,dx  \\
 &\geq& (1-\eta)\int_h^a g(x)\,dx + (\eta/2))\int_h^a g(x)\,dx  \\
 &>& (1 - \eta)\int_0^a g(x)\,dx
\end{eqnarray*}

\end{thmproof}


%%%%%%%%%%%%%%%%%%%%%%%%%%%%%%%%%%%%%%%%%%%%%%%%%%%%%%%%%%%%%%%%%%%%%%%%
\subsection{Partial Sum Near Fraction}

Given sufficiently ``fine'' partition of a number,
we can find a sub-partition of a fraction of the number.
Let's be less general, but more precise.
\begin{llem} \label{lem:near:frac}
% Put \(N=\{k\in \N: 1\leq k \leq n\}\).
Assume \((a_k)_{k\in{\N_n}}\) be a finite sequence of complex numbers
and \(\alpha\in[0,1]\).
For each subset \(G\subset \N_n\) define
\begin{eqnarray*}
S(G) &\eqdef& \sum_{k\in G} a_k \\
D(G) &\eqdef& |S(G) - \alpha S(\N_n)|.
\end{eqnarray*}
If \(|a_k| \leq r > 0\) for all \(k\in \N_n\)
then there exists a subset \(H \subset \N_n\)
such that \(D(H) \leq \sqrt{2}r/2\).
\end{llem}
\textbf{Note.} There is no claim here
that \(\sqrt{2}/2\) is necessarily the best constant.
\newline
\begin{thmproof}
If \(n\leq 1\) or \(S=0\), then we take \(H=\N_n\) and we are done.
So we may assume \(n>1\) and \(S\neq 0\).
There exists \(\theta\in[0,2\pi]\) such that \(S= e^{i\theta}|S|\).
By multiplying all of the \(a_k\) by \(e^{-i\theta}\),
the above assumptions do not change.
Thus, \wlogy\ we may assume that \(0< S \in\R\).
To construct $H$, we define \((h_k)_{k\in\N_n}\),
a permutation of \(\N_n\) by induction. Intuitively,
we keep the partial sums close to the real line.

Let \(h_1=1\) (arbitrary) and assume \(h_k\) were defined for \(k\leq m\).
For \(h_{m+1}\) we need to pick from the remaining
\(R = \N_n\setminus \{h_k: 1\leq k \leq m\}\) indices.
We look at
\begin{eqnarray*}
T(m) &=& \sum_{k\leq m} a_{h_i} \\
I(m) &=& \Im\bigl(T(m)\bigr).
\end{eqnarray*}

Since $S$ is real, $R$ must contain some \(h_{m+1}=k\) for which
\(\Im(a_{h_k})\) and $I$ have opposite signs or both are zero.

By the way \(h_k\) were picked,
clearly \(|I(m)| \leq r\) for all \(1\leq m \leq n\).
Since
\begin{equation*}
\sum_{k=1}^n a_{h_k} = \sum_{k=1}^n a_k = S > 0,
\end{equation*}
there exist a first $j$, such that \(\Re(T(j)) \geq \alpha S\).
We will now show that \(z_0 =  T(j-1)\) or \(z_1 = T(j)\)
satisfy the desired requirements for $H$.
We put \(z_k = v_k + iw_k\) for \(k=0,1\) and we have the following
inequalities:
\begin{gather}
% \begin{align}
v_0 < \alpha S \leq v_1 \notag \\
w_0  \leq 0 \leq w_1 \label{eq:lem:subaver} \\
% \end{align} \\
 |z_1 - z_0| \leq r \notag
\end{gather}
\paragraph{Note.} By the choice of \(h_j\),
the (\ref{eq:lem:subaver}) inequality may actually be reversed.
But the treatment of such case is the same as with the following:
\begin{eqnarray*}
 |\alpha S - z_0|^2 +  |\alpha S - z_1|^2
 &=&
  (\alpha S - v_0)^2 + (\alpha S - v_1)^2  + w_0^2 + w_1^2 \\
 &\leq&
  (v_1 - v_0)^2 + (w_1 - w_0)^2 \\
 &=& |z_1 - z_0|^2 \\
 &\leq& r^2.
\end{eqnarray*}
Thus \(|\alpha S - z_k|^2 \leq r^2/2\) for \(k=0\) or \(k=1\), and so
\(|\alpha S - z_k| = \sqrt{2}r/2\).
\end{thmproof}


%%%%%%%%%%%%%%%%%%%%%%%%%%%%%%%%%%%%%%%%%%%%%%%%%%%%%%%%%%%%%%%%%%%%%%%%
%%%%%%%%%%%%%%%%%%%%%%%%%%%%%%%%%%%%%%%%%%%%%%%%%%%%%%%%%%%%%%%%%%%%%%%%
\section{Equalities}

%%%%%%%%%%%%%%%%%%%%%%%%%%%%%%%%%%%%%%%%%%%%%%%%%%%%%%%%%%%%%%%%%%%%%%%%
\subsection{Sequences Equalities}

\begin{llem} \label{lem:limsup:liminf}
Let \((a_i)_{i\in\N}\) and \((b_i)_{i\in\N}\) be sequences of real numbers.
Put
\begin{equation*}
a^{*} \eqdef \limsup_{n\to\infty} a_n \qquad
b_{*} \eqdef \liminf_{n\to\infty} b_n\,.
\end{equation*}

If
\begin{equation*}
 c \eqdef \lim_{n\to\infty} a_n + b_n
\end{equation*}
exists and \(-\infty < c < \infty\) and \(a^{*} < \infty\)
then
\begin{equation} \label{eq:limsup:liminf}
 c = a^{*} + b_{*}\,.
\end{equation}
\end{llem}
\begin{thmproof}
Take a subsequence \((a_{k(i)})_{i\in\N}\), such that
\begin{equation*}
a^{*} = \lim_{n\to\infty} a_{k(n)}.
\end{equation*}
Now clearly
\begin{equation*}
 c = \lim_{n\to\infty} a_{k(n)} + b_{k(n)} = a^{*} + \lim_{n\to\infty} b_{k(n)}.
\end{equation*}
But since
\begin{equation*}
 b_{*} \leq \lim_{n\to\infty} b_{k(n)}
\end{equation*}
we have
\begin{equation}
 c \geq a^{*} + b_{*}.
\end{equation}
Similarly, we can derive the reversed inequality.
Thus (\ref{eq:limsup:liminf}) follows.
\end{thmproof}


%%%%%%%%%%%%%%%%%%%%%%%%%%%%%%%%%%%%%%%%%%%%%%%%%%%%%%%%%%%%%%%%%%%%%%%%
\subsection{Equality in Minkowski's Inequality}

\index{Minkowski's inequality}

The discussion following the proof of Theorem~3.5 \cite{RudinRCA87}
show the condition under which H\"older's inequality becomes an inequality.
The case for Minkowski's  inequality is left in the etxt as an~excercise.

The next lemma intuitively says that if function
do not share arguments, together they lose their norm.
\begin{llem} \label{eq:fgp:leq:afagp}
Let $f$ and $g$ be complex measurable functions on $X$ and \(1\leq p < \infty\).
Then
\begin{equation} \label{eq:fg:absfg}
\|f+g\|_p = \||f|+|g|\|_p
\end{equation}
iff
\begin{equation} \label{eq:argf:argg}
f(x)g(x) = 0 \qquad\textrm{or}\qquad \Arg(f(x)) = \Arg(g(x))\;\aded
\end{equation}
\end{llem}
\emph{Note:} We assume \(\Arg(z) = 0\) when \(z=0\).\\
\begin{thmproof}
Assume \eqref{eq:argf:argg} holds.
Then there the function \(\theta(x) \eqdef \Arg(f(x))\)
on $X$, satisfies
\begin{eqnarray*}
f(x) &=& e^{i\theta(x)}|f(X)| \;\aded \\
g(x) &=& e^{i\theta(x)}|g(X)| \;\aded.
\end{eqnarray*}
Now
% \begin{eqnarray*}
\[
\|f+g\|_p^p
= \int_X |f(x)+g(x)|^p\,d\mu(x)
= \int_X |e^{i\theta(x)}|\cdot\bigl(|f(x)|+|g(x)|\bigr)^p\,d\mu(x)
= \||f|+|g|\|_p^p.
\]
% \end{eqnarray*}
and so \eqref{eq:fg:absfg} follows.

Conversely, assume \eqref{eq:fg:absfg} and by negation
there is \(E\subset X\) such that \(\mu(E) > 0\) where
\(f(x)g(x)\neq 0\) and
\(\Arg(f(x)) \neq  \Arg(g(x))\) for all \(x\in E\).
Let \(\delta(x) = |\Arg(f(x)) - \Arg(g(x))|\).
Using Pythagoras theorem, for \(x\in E_n\) we have
\begin{eqnarray*}
|f(x) + g(x)|^2
&=&   \bigl(|f(x)| + |g(x)|\cos(\delta(x))\bigr)^2
  + \bigl(|g(x)|\sin(\delta(x)\bigr)^2 \\
&=& (|f(x)| + |g(x)|)^2 + 2\bigl(1-\cos(\delta(x))\bigr)|f(x)g(x)|
\end{eqnarray*}
Define
\[
% E_n = \{x\in E: |f(x)| > 1/n\;\vee\; |g(x)| > 1/n\}.
E_n = \left\{x\in E: |f(x)g(x)|\bigl(1-\cos(\delta(x))\bigr) > 1/n\right\}.
\]
Clearly \(E=\cup_n E_n\), hence there exist some $n$ such that \(\mu(E_n) > 0\).
Hence for \(x\in E_n\) we have
\[
(|f(x)| + |g(x)|)^2 - |f(x) + g(x)|^2  > 2/n.
\]
Hence, dividing by \((|f(x)| + |g(x)|) + |f(x) + g(x)|\) which must be \(>0\)
\[
(|f(x)| + |g(x)|)^2 - |f(x) + g(x)|
  > 2\,\bigm/\,n\bigl((|f(x)| + |g(x)|) + |f(x) + g(x)|\bigr) > 0.
\]
Hence \(\int_{E_n} |f+g|^p < \int_{E_n} (|f|+|g|)^p\).
Finally
\begin{eqnarray*}
\|f+g\|_p^p
&=& \int_X |f+g|^p \\
&=& \int_{X\setminus E_n} |f+g|^p     + \int_{E_n} |f+g|^p \\
&<& \int_{X\setminus E_n} (|f|+|g|)^p + \int_{E_n} (|f|+|g|)^p \\
&=& \| |f| + |g| \|_p^p
\end{eqnarray*}
which contradicts the assumption \eqref{eq:fg:absfg}.
\end{thmproof}


We start with a weak
real version for condition on equality in Minkowski's inequality.
\begin{llem} \label{llem:mink:real:eq}
Let \(f,g\) be real non negative measurable functions on $X$ and \(1<p<\infty\).
If
\begin{equation}
 \| f + g \|_p = \|f\|_p + \|g\|_p
\end{equation}
then there exist
real non-negative constants \(a,b\) not both $0$, such that \(af=bg \;\aded\).
\end{llem}
\begin{thmproof}
If \(\|f\|_p = 0\) or \(\|g\|_p = 0\) the result is trivial.
Say  \(\|f\|_p = 0\), then \(f=0\,\aded\), and we take \(a=1\), \(b=0\).

We now may assume \(\|f\|_p \neq 0\) and \(\|g\|_p \neq 0\).
Denote
\[ a(x) \eqdef f(x) + g(x)\]
and the conjugate exponent $q$ such that \(1/p+1/q=1\).
Now
\begin{eqnarray}
\|f+g\|_p^p
&=& \int_X a^p
= \int_X a\cdot a^{p-1}
=    \int_X f\cdot a^{p-1} +\int_X g\cdot a^{p-1} \notag \\
&\leq& \|f\|_p \|a^{p-1}\|_q + \|g\|_p \|a^{p-1}\|_q
       \label{eq:mink:real:2holder} \\
       % \qquad \textrm{H\"older} \\
&=& \bigl(\|f\|_p + \|g\|_p\bigr)\cdot\|a^{p-1}\|_q \notag
= \bigl(\|f\|_p + \|g\|_p\bigr)\cdot\left(\int_X a^{(p-1)q}\right)^{1/q}
   \notag \\
&=& \bigl(\|f\|_p + \|g\|_p\bigr)\cdot \|a\|_p^{p/q} \notag
\end{eqnarray}
The inequality in \eqref{eq:mink:real:2holder} is by applying H\"older
inequality twice.
Since \(p-p/q=1\) we get, dividing by \(\|a\|_p^{p/q}\) the following
\[
\|a\|_p \leq \|f\|_p + \|g\|_p.
\]
Since it is actually an \emph{equality} by the lemma condition,
the inequality in \eqref{eq:mink:real:2holder}
becomes equalities that the discussion quoted from the text,
shows that the must be real constants \(\alpha_i\) and \(\beta_i\)
for \(i=1,2\) such that
\begin{eqnarray*}
\alpha_1 f^p = \beta_1 a^q \\
\alpha_2 g^p = \beta_2 a^q.
\end{eqnarray*}
These constants are all non zero, and we have
\[( \alpha_1/\beta_1) f^p = \alpha_2/\beta_2) g^p\]
or setting
 \(a = (\alpha_1/\beta_1)^{1/p}\)
and
 \(b_i = (\alpha_2/\beta_2)^{1/p}\)
we get equivalently:
\( a f = bg\) and clearly \(a,b>0\).
\end{thmproof}



Now, the desired generalization.
\begin{llem} \label{llem:minkowski:eq}
Let \(f,g\) be measurable functions on $X$ and \(1<p<\infty\).
Then
\begin{equation} \label{eq:mink:equal}
 \left\| |f| + |g| \right\|_p = \|f\|_p + \|g\|_p
\end{equation}
iff there exist
real non-negative constants \(a,b\) not both $0$, such that \(af=bg \;\aded\).
\end{llem}
\begin{thmproof}
If either \(\|f\|_p=0\) or \(\|g\|_p=0\) the result is trivial.
This we may assume that both \(\|f|_p,\,\|g\|_p>0\).

Assume there exist such constants. \Wlogy, \(a>0\), put \(c=b/a\) and we have
\(f=cg\;\aded\).
Now both sides of \eqref{eq:mink:equal}
equal \((1+c)\|g\|_p\).

Conversely, assume \eqref{eq:mink:equal}.
By Minkowski's inequality
% and local lemma~\ref{eq:fgp:leq:afagp}
we have
\begin{equation} \label{eq:fgp:afagp:apgp}
\|f + g\|_p \leq \||f|+|g|\|_p \leq \|f\|_p + \|g\|_p.
\end{equation}
Our assumption, force the  inequalities in \eqref{eq:fgp:afagp:apgp}
to be equalities. But then
\begin{itemize}
\item
By local lemma~\ref{llem:mink:real:eq}
there exists real non negative constants $a$ and $b$ such that
\(a|f| = b|g|\;\aded\).
\item
By local lemma~\ref{eq:fgp:leq:afagp} \(\Arg(f(x)) = \Arg(g(x))\;\aded\)
\end{itemize}
Combing these conclusion,
show that there exists real non negative constants $a$ and $b$ such that
\(af = bg\;\aded\).
\end{thmproof}


%%%%%%%%%%%%%%%%%%%%%%%%%%%%%%%%%%%%%%%%%%%%%%%%%%%%%%%%%%%%%%%%%%%%%%%%
%%%%%%%%%%%%%%%%%%%%%%%%%%%%%%%%%%%%%%%%%%%%%%%%%%%%%%%%%%%%%%%%%%%%%%%%
\section{Integration Range}

The following result about convexity of the avareges set
will be used in Exercise~19.

\begin{llem} \label{llem:averages:convex}
Let   \((X,\frakM,m)\)
be a measurable space, where \(X\subset \R^n\) and $m$
is the restriction of Lebesgue's measure.
If \(f:X\to\C\) is a $m$-measurable function,
denote the average
\begin{equation*}
a(E) = a_f(E) \eqdef \frac{1}{m(E)} \int_E f\,d\mu
 \qquad \textnormal{where}\quad E\in\frakM \quad\textnormal{and}\quad m(E)>0.
\end{equation*}
Then the set of averages
\begin{equation*}
 % f \eqdef \left\{\, \frac{1}{\mu(E)} \int_E f\,d\mu\,:
 A_f \eqdef \left\{a_f(E): E\in\frakM \wedge \mu(E)>0 \right\}
\end{equation*}
is convex.
\end{llem}
\begin{thmproof}
Let \(\|\cdot\|\) some norm in \(\R^n\) inducing the Euclidean topology
(\(\|\cdot\|_2\) will do). For any generalized scalar \(r\in[0,\infty]\)
let \(B(r) = \{x\in X: \|x\|\leq r\}\).
For any measurable \(E\in\frakM\)
\iffalse
and any scalar \(\alpha\in[0,+\infty]\), let
\begin{equation*}
 E_\alpha \eqdef \{x\in E: \|x\|\leq \alpha\}.
\end{equation*}
\fi
by regularity of $m$, the real function
\(\nu(r) = m(E\cap B(r))\) is continuous. Similarly,
\begin{equation*}
\sigma(r) = \int_{E\cap B(r)} f\,dm
\end{equation*}
is continuous.

Pick arbitrary \(E_0,E_1\in\frakM\) such that \(m(E_0)\neq 0 \neq m(E_1)\).
We will show that the \(\C\)-segment
\begin{equation} \label{eq:llem:av:cvx:0}
[a(E_0),a(E_1)] \subset A_f,
\end{equation}
thus proving the convexity of \(A_f\).
We assume \(a(E_0) \neq a(E_1)\), otherwise the case is trivial.
\Wlogy, we may also assume that
\begin{equation} \label{eq:llem:av:cvx:1}
 a(E_0) = 0 \qquad a(E_1) = 1
\end{equation}
otherwise we may replace $f$ by
\(\bigl(f - a(E_0)\bigr)/\bigl(a(E_1) - a(E_0)\bigr)\).
Thus \eqref{eq:llem:av:cvx:0} is simplified to
\begin{equation} \label{eq:llem:av:cvx:2}
[0,1] \subset A_f.
\end{equation}
\iffalse
\(a(E_1)-a(E_0) = |a(E_1)-a(E_0)|e^{i\theta}\)
for some \(\theta\in[0,2\pi)\)
and we may substitue $f$ with \(e^{-i\theta}f\) and possibly
exchange \(E_0\) with \(E_1\).
\fi

Put \(D_j = E_j \setminus E_{1-j}\) for \(j=0,1\) and \(F=E_0\cap E_1\).
\iffalse
Now
\begin{eqnarray*}
a(E_1)-a(E_0)
 &=&
  \frac{1}{m(D_1) + m(F)} \left(\int_{D_1} f\,dm + \int_F f\,dm\right)
  -
  \frac{1}{m(D_0) + m(F)} \left(\int_{D_0} f\,dm + \int_F f\,dm\right)
\end{eqnarray*}
\fi
For \(j=0,1\) let
\begin{eqnarray*}
 D_j^- &\eqdef& \{x\in D_j: \Im(f(x)) < 0\} \\
 D_j^+ &\eqdef& \{x\in D_j: \Im(f(x)) \leq 0\} \\
\end{eqnarray*}
Note that any of the above 4 mutually disjoint sets may be empty.

We will now ``morph'' \(D_0^- \to D_1^+\) and \(D_0^+ \to D_1^-\).
We pick arbitrary strictly monotonic increasing and decreasing functions
(using \(1/0=+\infty\))
\begin{alignat*}{2}
u : [0,1]  &\nearrow [0,\infty]   &\qquad  v : [0,1] &\searrow [0,\infty] \\
u(\lambda) &= 1/(1-\lambda) - 1   &\qquad  v(\lambda)  &= 1/\lambda - 1
\end{alignat*}
to define the morphing functions:
\begin{alignat*}{2}
\Phi: [0,1] &\to P(D_0^-\cup D_1^+)
          &\quad \Psi: [0,1] &\to P(D_0^+\cup D_1^-) \\
\Phi(s) &=
  \left(D_0^- \cap B(v(s))\right)
  \cup
  \left(D_1^+ \cap B(u(s))\right)
  &\quad
\Psi(s) &=
  \left(D_0^+ \cap B(v(s))\right)
  \cup
  \left(D_1^- \cap B(u(s))\right).
\end{alignat*}
Note that
\begin{alignat*}{2}
\Phi(0) &= D_0^-  &\qquad  \Phi(1) = D_1^+ \\
\Psi(0) &= D_0^+  &\qquad  \Psi(1) = D_1^-
\end{alignat*}

Now define
\begin{eqnarray*}
\varphi : [0,1]\times[0,1] \to &P(E_0\cup E_1) \\
\varphi(s,t) = &\Phi(s) \disjunion F \disjunion \Psi(t).
\end{eqnarray*}
Note
\begin{equation} \label{eq:llem:av:cvx:E01}
\varphi(0,0) = E_0 \qquad \varphi(1,1) = E_1.
\end{equation}

Now we look at the ``morphed average'' continuous function
and its imaginary part.
\begin{gather*}
\Xi : [0,1]\times[0,1] \to \C \\
\Xi(s,t) = a(\varphi(s,t)) \\
\iota(s,t) = \Im\bigl(\Xi(s,t)\bigr).
\end{gather*}

Now \(\iota(s,t)\) is increasing in $s$ and decreasing in $t$
and
\(\iota(0,0) = \iota(1,1) = 0\)
by \eqref{eq:llem:av:cvx:E01} and \eqref{eq:llem:av:cvx:1}.
Thus for any \(s\in[0,1]\) we have
\begin{equation*}
\iota(s,1) \leq 0 \leq \iota(s,0)
\end{equation*}
so by continuity of \(\iota\), we can define
\begin{equation*}
\tau(s) \eqdef \sup\{t \in[0,1]: \iota(s,t) = 0\}
\end{equation*}
which is continuous, again by continuity of \(\iota\).
Now \(\Xi(s,\tau(s))\) is a continuous function of \([0,1]\) on itself
with fixed endpoints.
Thus for any \(\lambda\in[0,1]\)
there exists \(s\in[0,1]\) such that \(\Xi(s,\tau(s)) = \lambda\).
Hence \(a(\varphi(s,\tau(s)) = \lambda\) and \eqref{eq:llem:av:cvx:2} holds.
\end{thmproof}


%%%%%%%%%%%%%%%%%%%%%%%%%%%%%%%%%%%%%%%%%%%%%%%%%%%%%%%%%%%%%%%%%%%%%%%%
%%%%%%%%%%%%%%%%%%%%%%%%%%%%%%%%%%%%%%%%%%%%%%%%%%%%%%%%%%%%%%%%%%%%%%%%
\section{Exercises} % pages 73-78

%%%%%%%%%%%%%%%%%
\begin{enumerate}
%%%%%%%%%%%%%%%%%

%%%%%%%%%%%%%%
\begin{excopy}
Prove that the supremum of any collection of convex functions on \((a,b)\)
\index{convex function}
is convex on \((a,b)\) and that pointwise limit of sequence of convex
functions are convex.
What can you say about upper and lower limits of sequence of convex functions?
\end{excopy}

Let \(\{f_i(x) \}_{i\in I}\) be a collection of convex functions on \((a,b)\).
From now on, for any given \(x_0,x_1\in(a,b)\) and \(\lambda\in([0,1]\)
we define the \(\lambda\)-convex combination
as \(x_\lambda = (1-\lambda)x_0 + \lambda x_1\).


\paragraph{Supremum.}
Let \(s(x) = \sup_{i\in I}f_i(x)\).
By negation let \(x_0,x_1\in(a,b)\) and \(\lambda\in[0,1]\) be such that
\begin{equation*}
s(x_\lambda) >  (1-\lambda)s(x_0) + \lambda s(x_1)
\end{equation*}
Let
\begin{equation*}
 \epsilon =
   s(x_\lambda) -
   \bigl((1-\lambda)s(x_0) + \lambda s(x_1)\bigl) > 0.
\end{equation*}
By definition of \(s(x)\) there exists \(i\in I\) such that
\begin{equation*}
 0 < s(x_\lambda) - f_i(x_\lambda) < \epsilon\,.
\end{equation*}
Therefore,
\begin{equation*}
f_i(x_\lambda)
> s(x_\lambda)
> (1-\lambda)s(x_0) + \lambda s(x_1)
\geq  (1-\lambda)f_i(x_0) + \lambda f_i(x_1)
\end{equation*}
contradiction to \(f_i\) being convex, hence \(s(x)\) is convex.

\paragraph{Pointwise Limit.}
Let \(l(x) = \lim_{i\in \N}f_i(x)\).
By negation let \(x_0,x_1\in(a,b)\) and \(\lambda\in[0,1]\) be such that
\begin{equation*}
l(x_\lambda) >  (1-\lambda)s(x_0) + \lambda s(x_1)
\end{equation*}
Let
\begin{equation*}
 \epsilon =
   l(x_\lambda) -
   \bigl((1-\lambda)l(x_0) + \lambda l(x_1)\bigr) > 0.
\end{equation*}
There exists a sufficient large \(n\in\N\) (maximum of three)
such that for all \(i\geq n\)
\begin{equation*}
 \bigl|l(x_t) - f_i(x_t)\bigr| < \epsilon/2 \qquad
 \textrm{where}\; t\in\{0,\lambda,1\}
\end{equation*}
Now
\begin{eqnarray*}
\Delta(f_n,\lambda)
&=& f_n(x_\lambda) - \bigl((1-\lambda)f_i(x_0) + \lambda f_i(x_1)\bigr) \\
&\geq&
\left(l(x_\lambda) - \bigl((1-\lambda)l(x_0) + \lambda l(x_1)\bigr)\right)
 \; - \\
&&
  \bigl(
     |l(x_\lambda) - f_i(x_\lambda)| +
     (1-\lambda)|l(x_0) - f_i(x_0)|
     \lambda)|l(x_1) - f_i(x_1)|\bigr)\\
&>&
\left(l(x_\lambda) - \bigl((1-\lambda)l(x_0) + \lambda l(x_1)\bigr)\right)
 - (\epsilon/2 + (1-\lambda)\epsilon/2 + \lambda\epsilon/2) \\
&=&
\left(l(x_\lambda) - \bigl((1-\lambda)l(x_0) + \lambda l(x_1)\bigr)\right)
 - \epsilon \\
&=& 0
\end{eqnarray*}
Hence
\begin{equation*}
f_n(x_\lambda) > (1-\lambda)f_i(x_0) + \lambda f_i(x_1)
\end{equation*}
contradiction to \(f_i\) being convex, hence \(l(x)\) is convex.


\paragraph{Upper Limit.} Let \(u(x)\) be the upper limit of
\(\{f_i(x) \}_{i\in \N}\). By definition
\begin{equation*}
u(x) = \limsup_{n\to \infty} f_n(x)
     = \lim_{n\to \infty} \sup_{m\geq n} f_m(x)\,.
\end{equation*}
From the previous (supremum, and pointwise limit) results \(u(x)\) is convex.


\paragraph{Lower Limit.} A different behavior here.
The functions \(f_n(x) = (-1)^n x\) are convex.
But
\begin{equation*}
w(x)
= \liminf_{n\to \infty} f_n(x)
= \min_{n\to \infty}f_n(x)
= \min\bigl(f_0(x),f_1(x)\bigr) = -|x|
\end{equation*}
which is clearly \emph{not} convex.


%%%%%%%%%%%%%%
\begin{excopy}
If \(\varphi\) is a convex on \((a,b)\) and if \(\psi\) is convex and
nondecreasing on the range of \(\varphi\), prove that \(\psi\circ\varphi\)
is convex on \((a,b)\).
For \(\varphi>0\), show that the convexity of \(\log\varphi\) implies
the convexity of \(\varphi\), but not vice versa.
\end{excopy}

Let \(\varphi\) be convex on \((a,b)\) and \(\psi\)  convex and
nondecreasing on the range of \(\varphi\).
For any \(x,y\in(a,b)\) and \(\lambda\in[0,1]\) we have:
\begin{eqnarray}
\psi\circ\varphi(\lambda x + (1-\lambda)y)
 &=& \psi\bigl(\varphi(\lambda x + (1-\lambda)y)\bigr) \notag \\
 &\leq& \psi\bigl(\lambda\varphi(x) + (1-\lambda)\varphi(y)\bigr)
        \label{eq:phiconv:psiinc} \\
 &\leq& \lambda \psi\bigl(\varphi(x)\bigr) +
        (1-\lambda)\psi\bigl(\varphi(y)\bigr)
        \label{eq:psiconv} \\
 &=& \lambda \psi\circ\varphi(x) +  (1-\lambda)\psi\circ\varphi(y). \notag
\end{eqnarray}
The inequality (\ref{eq:phiconv:psiinc}) holds because \(\varphi\) is
convex and \(\psi\) is increasing.
The inequality (\ref{eq:psiconv}) holds because \(\psi\) is convex.

If \(\varphi>0\), and \(\log\varphi\) is convex, then by using \(\psi\)
as the exponential (\(\exp(x)\) in the result just shown, we get
that \(\exp(\log(\varphi)) = \varphi\) is convex.

The converse does not hold. Take the identity \(f(x)=x\), which is
clearly convex, but \(\log = \log\circ f\) is not.


%%%%%%%%%%%%%%
\begin{excopy}
Assume that \(\varphi\) is a continuous real function on \((a,b)\)
such that
\begin{equation*}
 \varphi\left(\frac{x+y}{2}\right)
 \leq \frac{1}{2}\varphi(x) + \frac{1}{2}\varphi(y)
\end{equation*}
for all $x$ and $y$ \(\in (a,b)\). Prove that \(\varphi\) is convex.
(The conclude does \emph{not} follow if continuity is omitted from the
hypotheses.)
\end{excopy}


We first show that for every integers \(n\geq 0\) and
\(0\leq m \leq 2^n > 0\)
and any \(x,y\in (a,b)\)
\begin{equation} \label{eq:ex3.3:m:n}
 \varphi\bigl((2^n-m)x/2^n + my/2^n\bigr) \leq
  \left((2^n-m)\varphi(x) + m\varphi(x)\right)\bigm/2^n
\end{equation}

By denoting
\begin{equation}
C(m,N,x,y) = (N-m)x/N + my/N.
\end{equation}
we actually need to show in (\ref{eq:ex3.3:m:n})
\begin{equation}  \label{eq:ex3.3:m:n:C}
  \varphi(C(m,2^n,x,y)) \leq C(m,2^n,\varphi(x), \varphi(y)).
\end{equation}



By induction on $n$. For \(n=0\), we must have \(m=0\) or \(m=1\)
and the inequality becomes trivial true equality.
Assume that (\ref{eq:ex3.3:m:n}) holds for \(n\leq k\geq 0\), we will show
that it holds also for \(n=k\).
Let $m$ be an integer such that \(0\leq m \leq 2^{k+1}\).
We use the trivial equality of common \(2^x\) denominator.
There are two cases:

 \textbf{Case (i)}:
   If \(2\mid m\), then \(m/2\) is an integer, hence, by induction hypotheses
 \begin{eqnarray*}
  C(m,2^{k+1},x,y)
  &=&     \varphi\bigl(((2^{k+1}-m)x + my)/2^{k+1}\bigr) \\
  &=&     \varphi\bigl(((2^k-m/2)x + my)/2^k\bigr) \\
  &\leq&  \left((2^k-m/2)\varphi(x) + (m/2)\varphi(x)\right)\bigm/2^k \\
  &=&     \left((2^{k+1}-m)\varphi(x) + m\varphi(x)\right)\bigm/2^{k+1} \\
  &=&     C(m,2^{k+1},\varphi(x)) + C(m,2^{k+1},\varphi(y)).
 \end{eqnarray*}

\textbf{Case (ii)}:
Otherwise,
\(2\nmid m\), then \(0<m<2^n\) and \((m\pm 1)/2\) are two integers.
  we use a trivial mid-point equality:
  \begin{equation*}
  C(m,2^{k+1},x,y) = \left( C(m-1,2^{k+1},x,y) + C(m-1,2^{k+1},x,y) \right) / 2
  \end{equation*}
  By initial assumption
  \begin{equation} \label{eq:ex3.3:m:mid}
  \varphi\left(C(m,2^{k+1},x,y)\right)
   \leq
    \left(\varphi\left(C(m-1,2^{k+1},x,y)\right) +
          \varphi\left(C(m+1,2^{k+1},x,y)\right) \right) \bigm / 2.
  \end{equation}

  Now
  \begin{equation*}
   C(m\pm 1,2^{k+1},x,y) =  C((m\pm 1)/2,2^k,x,y).
  \end{equation*}

  Thus (supporting arguments follows),
  \begin{eqnarray}
     \varphi\bigl(C(m,2^{k+1},x,y)\bigr)
    &\leq& \label{eq:ex3.3:m:midC}
          \Bigl(
          \varphi\bigl(C(m-1,2^{k+1},x,y)\bigr) + \\
    && \phantom{\Bigl(}
          \varphi\bigl(C(m+1,2^{k+1},x,y)\bigr) \Bigr) \Bigm/ 2 \notag \\
    &=&
          \Bigl(
          \varphi\bigl(C((m-1)/2,2^k,x,y)\bigr) + \notag \\
    && \phantom{\Bigl(}
          \varphi\bigl(C((m+1)/2,2^k,x,y)\bigr) \Bigr) \Bigm/ 2
          \notag \\
    &\leq & \label{eq:ex3.3:induct:step}
     \Bigl(
             C\bigl((m-1)/2,2^k,\varphi(x),\varphi(y)\bigr) + \\
    &&  \phantom{\Bigl(}
             C\bigl((m+1)/2,2^k,\varphi(x),\varphi(y)\bigr) \Bigr) \Bigm/ 2
             \notag \\
    &\leq &  C\bigl((m-1)/2,2^{k+1},\varphi(x),\varphi(y)\bigr) + \notag \\
    &&       C\bigl((m+1)/2,2^{k+1},\varphi(x),\varphi(y)\bigr) \notag \\
    &= &     \label{eq:ex3.3:endinuct}
             C\bigl(m,2^{k+1},\varphi(x),\varphi(y)\bigr).
  \end{eqnarray}

The (\ref{eq:ex3.3:m:midC}) inequality follows from (\ref{eq:ex3.3:m:mid}).
The (\ref{eq:ex3.3:induct:step}) inequality follows by induction.
The last equality (\ref{eq:ex3.3:endinuct}) follows from the following
two equalities (\ref{eq:ex3.3:m}) and (\ref{eq:ex3.3:m2})
\begin{equation} \label{eq:ex3.3:m}
 \bigl((m-1)/2\bigr)/ 2^{k+1} +
 \bigl((m+1)/2\bigr)/ 2^{k+1} =
  m / 2^{k+1}
\end{equation}

\begin{equation}\label{eq:ex3.3:m2}
 \bigl(2^{k+1}-(m-1)/2\bigr)/ 2^{k+1} +
 \bigl(2^{k+1}-(m+1)/2\bigr)/ 2^{k+1} =
      (2^{k+1}-m) / 2^{k+1}
\end{equation}

Now that (\ref{eq:ex3.3:m:n}) is established, let \(\lambda\in[0,1]\)
and by negation assume that
\begin{equation*}
 \epsilon = \varphi(\lambda x + (1-\lambda)y) -
 \lambda \varphi(x) + (1-\lambda)\varphi(y) > 0.
\end{equation*}
Since \(\varphi\) is continuous on \([a,b]\) it is uniformly continuous
on \([a,b]\). So there exists \(\delta>0\) such that
\(|\varphi(t_1) - \varphi(t_2)| < \epsilon/4\)
whenever \(|t_1-t_2|<\delta\).
Since the set \(\{m/2^n: m,n\in\N\wedge 0\leq m\leq 2^n\}\)
is dense in \([0,1]\), we can find \(n>0\) and $m$ such that
(the first two inequalities are actually equivalent):
\begin{eqnarray*}
|(2^n - m)/2^n - \lambda)| &<& \delta \\
|m/2^n - (1-\lambda)| &<& \delta \\
|\bigl((2^n - m)/2^n - \lambda)\bigr)\varphi(x)| &<& \epsilon/4 \\
|\bigl(m/2^n - (1-\lambda)\bigr)\varphi(y)| &<& \epsilon/4
\end{eqnarray*}

We now get the following contradiction
\begin{eqnarray*}
\varphi(\lambda x + (1-\lambda)y)
&\leq& \varphi(C(m,n,x,y)) + \epsilon/4 + \epsilon/4 \\
&\leq& C(m,n,\varphi(x),\varphi(y)) + \epsilon/2 \\
&\leq& (\lambda\varphi(x) + \epsilon/4) + ((1-\lambda)\varphi(y) + \epsilon/4)
       + \epsilon/2 \\
&=&    \lambda\varphi(x) (1-\lambda)\varphi(y) + \epsilon.
\end{eqnarray*}

Therefore \(\varphi\) is convex.


%%%%%%%%%%%%%% 4
\begin{excopy}
Suppose $f$ is a complex measurable function on $X$, \(\mu\) is a
positive
measure on $X$, and
\begin{equation*}
 \varphi(p) = \int_X |f|^p\,d\mu = \|f\|_p^p \qquad (0<p<\infty).
\end{equation*}
Let \(E = \{p: \varphi(p)<\infty\}\). Assume \(\|f\|_\infty > 0\).
\begin{itemize}
 \itemch{a}
   If \(r<p<s\), \(r\in E\), and \(s\in E\), prove that \(p\in E\).
 \itemch{b}
   Prove that  \(\log\varphi\) is convex in the interior of $E$ and
   that  \(\varphi\) is continuous on $E$.
 \itemch{c}
   By \ich{a}, $E$ is connected. Is $E$ necessarily open? Closed? Can
   $E$ consist of a single point?
   Can $E$ be any connected subset of \((a,\infty)\)
 \itemch{d}
   If \(r<p<s\), prove that \(\|f\|_p \leq \max( \|f\|_r, \|f\|_s)\).
   Show that this implies the inclusion
   \(L^r(\mu) \cap  L^s(\mu) \subset L^p(\mu)\).
 \itemch{e}
   Assume that \(\|f\|_r < \infty\) for some \(r<\infty\) and prove
   that
   \begin{equation*}
     \|f\|_p \to \|f\|_\infty \qquad \textrm{as}\; p\to\infty.
   \end{equation*}
\end{itemize}
\end{excopy}

 Let \(X_0 = \{x\in X: |f(x)|<1\}\)
 and \(X_1 = \{x\in X: |f(x)|\geq 1\}\).

\begin{itemize}
\itemch{a}
 Compute
 \begin{eqnarray*}
 \varphi(p)
 &=& \int_X |f|^p\,d\mu
 = \int_{X_0} |f|^p\,d\mu + \int_{X_1} |f|^p\,d\mu
 \leq \int_{X_0} |f|^r\,d\mu + \int_{X_1} |f|^s\,d\mu \\
 &\leq& \int_{X} |f|^r\,d\mu + \int_{X} |f|^s\,d\mu
 \leq \varphi(r) + \varphi(s) < \infty.
 \end{eqnarray*}


\itemch{b}
  Let \(r < s < t\) be in the interior of $E$.
  Put \(p_1 = (t-s)/(t-r)\)
  and \(p_2 = (s-r)/(t-r)\)
  to form  the convex combination \(s = p_1 r + p_2 t\).

  Now from Lemma~\ref{llem:hlp:188} --- with \(k=2\),
  \(f_1 = |f|^r\) and
  \(f_2 = |f|^t\) we have
  \begin{eqnarray*}
    \varphi(s)
    &=&    \int_X |f|^s\,d\mu \\
    &=&    \int_X f_1^{p_1} f_2^{p_2}\,d\mu \\
    &\leq& \left( \int_X f_1\,d\mu \right)^{p_1}
           \left( \int_X f_2\,d\mu \right)^{p_2} \\
    &=&    \varphi(r)^{p_1} \varphi(t)^{p_2}.
  % \int_X |f|^s\,d\mu
  \end{eqnarray*}
  Now since \(\log\) is monotonic increasing, we get
  \begin{equation*}
  \log\varphi(s) \leq  \log(\varphi(r)^{p_1} \varphi(t)^{p_2} =
           p_1\log\varphi(r) + p_2\log\varphi(t).
  \end{equation*}
  Now that we have shown convexity in $E$,
  by Theorem~3.2 (\cite{RudinRCA80}), \(log(\varphi(x)\) is continuous in
  the interior of $E$. To show continuity on all of $E$,
  let \([a,c] = \overline{E}\) and put \(b=(a+b)/2\).

  If \(a\in E\), then \(b\in E\) and put
  \begin{equation*}
  g_a(x) = \max(|f(x)|^a, |f(x)|^b)
  \end{equation*}
  and clearly \(\int_X g_a(x)\,d\mu < \infty\).
  For \(s\in[a,b]\),  \(g_a\) is a dominated function for \(\varphi(s)\)
  and by Lebesgue's dominated convergence theorem, \(\varphi(s)\)
  is continuous at \(s=a\).

  If \(c\in E\), then \(b\in E\) and put
  \begin{equation*}
  g_c(x) = \max\left(|f(x)|^b, |f(x)|^c\right)
  \end{equation*}
  and clearly \(\int_X g_c(x)\,d\mu < \infty\).
  For \(s\in[b,c]\),  \(g_c\) is a dominated function for \(\varphi(s)\)
  and by Lebesgue's dominated convergence theorem, \(\varphi(s)\)
  is continuous at \(s=c\).

  Thus \(\varphi\) is continuous on $E$.

\itemch{c}
  The answer is ``Yes'' to the last question. That is, $E$
  can be any connected subset of \((0,\infty)\).

  First, the trivial cases:
  \begin{itemize}
  \item \(E=\emptyset\) with \(f:\R\to\R\) defined as \(f(x)=1\).
  \item \(E=\R^+\) with \(f:X\to\R\) defined as \(f(x)=0\)
        for any measurable space $X$.
  \end{itemize}

  Let's generalize the definition of \(\phi\) and $E$.
  For any function \(g:X\to\C\), let
  \begin{eqnarray*}
   \varphi_g(p) &=& \int_X |g|^p\,d\mu = \|g\|_p^p \qquad (0<p<\infty) \\
   E(g) &=& \{p: \varphi_g(p)<\infty\}.
  \end{eqnarray*}

  Say we have functions
 \(\{f_i\}_{i\in\N}\)
  \begin{equation*}
   f_i: X_i \to \R^+ \qquad(i\in\N)
  \end{equation*}
  where \(\R^+ = \{x\in\R: x\geq0\}\).
  % We'll use the abbreviations: \(\varphi_i(p) = \varphi_{f_i}(p)\).

  It is easy to see that \(E(f_1 + f_2) = E(f_1) \cap E(f_2)\).
  Similarly
  \begin{equation*}
  E\left(\sum_{i\in\N} f_i\right) = \bigcap_{i\in\N} E(f_i)\,.
  \end{equation*}
  We can simplify the fucntion summation by assuming that
  their domains are mutually disjoint. Without loss of generality,
  we can thing of disjoint union of domains.

  We will show that for any \(a\in\R^+\),
  we can build functions $g$, such that \(E(g)\)
  is \emph{half open line}, that is
  \(E(g) = \{x\in\R^+: 0<x<a\}\) or
  \(E(g) = \{x\in\R^+: a<x<a\}\).

  It is easy to see that any connected subset of \(\R^+\) can
  be represented as an intersection of countably many half open lines.


  Let \(X=\{x\in\R: x\geq 1\}\) with the normal Lebesgue's measure and let
  \(g_a(x) = x^a\) for some \(a\in\R\). Let's see how \(E(g_a)\) looks.
  If \(a=0\) then \(E(g_0) = \emptyset\).
  Otherwise \(a\neq 0\) and then for
  \begin{equation*}
  E(g_a) = \{p\in \R^+: pa < 1\}
  \end{equation*}
  we have two cases:
  \begin{itemize}
   \item[($-$)] If \(a<0\) then \(E(g_a) = \{p\in\R^+: p > 1/a\}\).
   \item[($+$)] If \(a>0\) then \(E(g_a) = \{p\in\R^+: p < 1/a\}\).
  \end{itemize}
  Thus we can find $a$ for any desired half open line we need.

\itemch{d}
 If either \(\|f\|_r = \infty\) or \(\|f\|_s = \infty\)
 then the inequality trivially follows.  Thus we may assume
 that both \(r,s\in E\).
 By negation, let's assume
 \begin{equation} \label{eq:3:fpfrfs}
 \|f\|_p > \max( \|f\|_r, \|f\|_s).
 \end{equation}
 We want to avoid dealing with an exceptional case, where
 $r$ or $s$ are on the boundary of $E$.
 From \ich{b} since \(\varphi\) is continuous on $E$,
 we can find \(r'\), \(s'\) such that \(r<r'<p<s'<s\)
 and still
 \(\|f\|_p > \max( \|f\|_{r'}, \|f\|_{s'})\).
 Since (By \ich{b} again) \(\log\varphi\) is convex in the interior of $E$,
 and so
 \begin{equation*}
 \log\varphi(p) \leq \bigl(\log\varphi(r') + \log\varphi(s')\bigr)/2
                \leq \max(\log\varphi(r'), \log\varphi(s')\bigr)
 \end{equation*}
 But \(\log\) is monotonically increasing, and so
 \(\varphi(p) \leq \max(\varphi(r'), \varphi(s')\bigr)\)
 contradiction to (\ref{eq:3:fpfrfs}).

 Now \(L^r(\mu) \cap  L^s(\mu) \subset L^p(\mu)\) follows by definitions.

\itemch{e}
 Assume $E$ is bounded by \(M<\infty\). Then
 \begin{equation*}
 \lim_{p\to\infty}\varphi(p) =
 \lim_{M<p\to\infty}\varphi(p) = \infty.
 \end{equation*}

 Otherwise, \(E=\R^+\) (we follow \cite{Hardy:1952:I} \textbf{192}).
 Two cases:
 \begin{itemize}
  \item[\(\bullet\)]
   Suppose \(M = \|f\|_\infty < \infty\).
   Then \(\|f\|_r \leq M\) and for any \(\epsilon>0\)
   we put \(H=\{ x\in X: f(x)> M-\epsilon\}\) with \(\xi=m(H)>0\).
   Then for any \(r\in\R^+\setminus\{0\}\) we have
   \begin{equation*}
   \xi(M-\epsilon)^r \leq \int_H |f|^rd\mu
                     \leq \int_X |f|^rd\mu \leq \|f\|_r^r\,.
   \end{equation*}
   Thus % \(M - \epsilon) \leq \|f\|_r
   \begin{equation*}
   \underline{\lim}_{r\to\infty} \|f\|_r \xi^{-r} \geq M - \epsilon
   \end{equation*}
   and consequently  \(\underline{\lim}_{r\to\infty} \|f\|_r  \geq M\).

  \item[\(\bullet\)]
   Suppose \(\|f\|_\infty = \infty\). Then for any \(M<0\)
   we put \(H=\{ x\in X: f(x)> M\}\) and \(\xi = m(H) > 0\).
   Similarly as the bounded case, we see that
   \(\underline{\lim}_{r\to\infty}  \|f\|_r \geq M\).
   Since $M$ can be arbitrary large,
   \(\underline{\lim}_{r\to\infty}  \|f\|_r = \infty\).
 \end{itemize}
 Thus, in both cases \(\lim_{r\to\infty} \|f\|_r = \|f\|_\infty\).


\end{itemize}

%%%%%%%%%%%%%% 5
\begin{excopy}
Assume, in addition to the hypotheses of Exercise~4, that
\begin{equation*}
 \mu(X) = 1.
\end{equation*}
\begin{itemize}
\itemch{a}
 Prove that \(\|f\|_r \leq \|f\|_s\) if \(0<r<s\leq \infty\).
\itemch{b}
 Under what conditions does it happen that \(0<r<s\leq \infty\)
 and \(\|f\|_r = \|f\|_s < \infty\) ?
\itemch{c}
 Prove that
  \(L^r(\mu) \supset L^s(\mu)\) if \(0<r<s\).
 Under what conditions do these two spaces contain the same functions.
\itemch{d}
 Assume that \(\|f\|_r < \infty\) for some \(r>0\), and prove that
 \begin{equation*}
   \lim_{p\to 0} \|f\|_p = \exp\left\{\int_X \log|f|\,d\mu\right\}
 \end{equation*}
 if \(\exp\{-\infty\}\) is defined to be $0$.
\end{itemize}
\end{excopy}

\begin{itemize}
%
\itemch{a}
If \(\|f\|_s=\infty\) the inequality trivially holds. Otherwise
We use lemma~\ref{lem:holder:eq} to compute, for simplification
assume \(f\geq 0\).
\begin{equation*}
\|f\|_r^r
 = \int_X f^r\,d\mu
 = \int_X (f^s)^{r/s}\cdot \mathbf{1}^{1-r/s}\,d\mu
 \leq \left(\int_X f^s\right)^{r/s} \cdot \left(\int_X \mathbf{1}\right)^{1-r/s}
 = \left(\int_X f^s\right)^{r/s}
\end{equation*}
Taking power of \(1/r\) gives the desired inequality \(\|f\|_r \leq \|f\|_s\).

\itemch{b}
 Following lemma~\ref{lem:holder:eq}, strict inequality happens
 when $f$ is not effectively constant.

\itemch{c}
The inclusion is immediate from \ich{a}. The spaces contain the same
functions (Spaces are ``equal'' except for their norm) iff $X$ has only finite
number of subsets with non zero measure.

If this condition is met. Let \(X=\disjunion_{j=1}^n X_j\) be the partition.
Meaning: \(\mu(X_j)>0\) and if \(A\subset X_j\) is measurable, then
\(\mu(A)=0\) or \(\mu(X\setminus A)=0\).
Then a measurable function \(f_{|X_j}=c_j\)
is constant \aded\ on each \(X_j\).
% assumes finite number of values except for set of measure zero.
It is easy to see now that for such function \(\|f\|_p < \infty\)
if \(c_j<\infty\) for \(1\leq i \leq n\).

Conversely, we can find inifinite enumerable partition
\(X=\disjunion_{j=1}^\infty X_j\), with \(\mu(X_j) = m_j > 0\).
We will construct a function \(f:X\to [0,\infty)\) such that
\(f\in L^r(X)\setminus L^s(X)\).
Since \(\mu(X)=1\), there are infinite subsets \(X_j\) with small measure
as desired.
Before picking values for $f$,
let's assume (or by picking a sub-family) that \(X_j\) are such that
\begin{equation*} \label{eq:LrLs:mj}
m_j \eqdef \mu(X_j) < (2^{-js} j)^{1/(s-r)}.
\end{equation*}
Hence,
\begin{equation*}
\left(\frac{1}{jm_j}\right)^r < \left(\frac{2^{-j}}{m_i}\right)^s.
\end{equation*}
Equivelantly,
\begin{equation}
l_j \eqdef (j m_j)^{-1/s} < \left(2^{-j}/m_i\right)^{1/r} \eqdef u_j.
\end{equation}
Setting ``mid-constants'' \(c_j = (l_j + u_j)/2\) we cann now define
\begin{equation*}
f(x) = \left\{ \begin{array}{ll}
               c_j \qquad & x\in X_j \\
               0          & \textrm{Otherwise}
               \end{array}\right..
\end{equation*}
To show that $f$ is satisfies our goal:
\begin{itemize}

\item[(i)] \(f\in L^r(X)\) \quad:
\begin{equation*}
\|f\|_r^r
 = \sum_{j=1}^\infty m_j c_j^r
 <  \sum_{j=1}^\infty m_j u_j^r
 \leq \sum_{j=1}^\infty m_j \left(\left(2^{-j}/m_i\right)^{1/r}\right)^r
 = \sum_{j=1}^\infty 2^{-j} = 1 < \infty
\end{equation*}

\item[(ii)] \(f\notin L^s(X)\) \quad:
\begin{equation*}
\|f\|_s^s
 = \sum_{j=1}^\infty m_j c_j^s
 >  \sum_{j=1}^\infty m_j l_j^s
 = \sum_{j=1}^\infty m_j \left((j m_j)^{-1/s} \right)^s
 = \sum_{j=1}^\infty 1/j = \infty
\end{equation*}

\end{itemize}

\itemch{d}
 Using \(\exp(x)=(\exp(x^p)^{1/p}\), we have:
 \begin{equation*}
  \exp\left(\int\log(|f|)\,d\mu\right)
   = \left(\exp\bigl(\int\log(|f|^p)\,d\mu\bigr)\right)^{1/p}
   \leq \left(\int|f|^p\,d\mu\right)^{1/p} = \|f\|_p.
 \end{equation*}
 The inequality is derived from comparing geometrical and arithmetical means,
 see (7) in the proof of \index{Jensen} Jensen's theorem
 (\cite{RudinRCA80}  page~64).

 Note that for \(x>0\)
 \begin{equation*}
  \varphi(x) = (t^x-1)/x
  = \log t \frac{e^{x\log t} - e^0}{x\log t - 0}
 \end{equation*}
 is an increasing function by convexity of \(\exp\).
 and so \(\varphi(x)\to \log t\) (decreasing)  as \(x\to 0\)
 by l'Hospital's rule. We also note the simple inequality
 \begin{equation} \label{eq:logt:leq:tm1}
 \log t \leq t-1.
 \end{equation}
 We now compute
 \begin{eqnarray}
  \exp\left(\int \log|f|\,d\mu\right)
  &\leq& \lim_{p\to 0} \|f\|_p          \notag \\
  &=& \overline{\lim_{p\to 0}} \|f\|_p         \notag \\
  &=& \exp\left( \overline{\lim_{p\to 0}}
                 \frac{1}{p}\log\bigl( \int |f|^p\,d\mu\bigr) \right) \notag\\
  &\leq& \exp \left(\overline{\lim_{p\to 0}}
                        \int(|f|^p-1)/p\,d\mu\right) \label{eq:5d:whyleq} \\
  &=&    \exp\left( \int \log|f|\,d\mu\right)        \label{eq:5d:whyeq}
 \end{eqnarray}
 Where (\ref{eq:5d:whyleq}) is by (\ref{eq:logt:leq:tm1})
 and   (\ref{eq:5d:whyeq})  is by Lebesgue's dominated convergence theorem.
 Hence
 \begin{equation*}
  \exp\left\{\int \log|f|\,d\mu\right\} =  \lim_{p\to 0} \|f\|_p
 \end{equation*}
\end{itemize}


%%%%%%%%%%%%%% 6
\begin{excopy}
Let $m$ be Lebesgue measure on \([0,1]\), and define \(\|f\|_p\)
with respect to $m$.
Find all functions \(\Phi\) on \([0,\infty)\) such that the relation
\begin{equation*}
 \Phi(\lim_{p\to 0} \|f\|_p) = \int_0^1(\Phi\circ f)\,dm
\end{equation*}
holds for every bounded, measurable, positive $f$. Show first that
\begin{equation*}
c\Phi(x)+(1-c)\Phi(1) = \Phi(x^c) \qquad (x>0, 0\leq c\leq 1).
\end{equation*}
Compare with Exercise~5\ich{d}.
\end{excopy}

Looking at Exercise~5\ich{d}, we clearly see that \(\log\) can be
one of such \(\Phi\) functions.
Define
\begin{equation*} % \label{eq:fc:chi}
 f(x) = f_c(x) x\cdot\chhi_{(0,c)} + 1\cdot \chhi_{(c,1)}
\end{equation*}
We compute the left side
\begin{equation*}
\Phi\left( \lim_{p\to 0} \|f\|_p\right)
= \Phi\left( \exp \bigl(\int_0^1 \log f\,dm\bigr)\right)
= \Phi\bigl( \exp (c\cdot\log x + (1-c)\log 1)\bigr)
= \Phi(x^c)
\end{equation*}
and the right side
\begin{equation*}
 \int_0^1 \Phi\circ f \,dm
 = \int_0^c \Phi(x)\,dm + \int_c^1 \Phi(x)\,dm
 = c\Phi(x) + (1-c)\Phi(1)
\end{equation*}
So the required equality is shown.

In the given equality
\begin{equation} \label{eq:Phi:lim}
 \Phi(\lim_{p\to 0} \|f\|_p) = \int_0^1(\Phi\circ f)\,dm
\end{equation}
We put
\begin{equation} \label{eq:ex:3.6:psi}
\psi(x) = \Phi(x) - \Phi(1),
\end{equation}
 getting
\begin{equation*}
 \psi(x^c)+\Phi(1)
  = c\bigl(\psi(x)+\Phi(1)\bigr) + (1-c)\bigl(\psi(x)+\Phi(1)\bigr)
\end{equation*}
Hence, when \(0\leq c\leq 1\).
\begin{equation} \label{eq:psi:xc}
\psi(x^c) = c\psi(x).
\end{equation}
If \(c>1\) then \(0\leq 1/c\leq 1\) and similarly
\begin{equation*}
\psi\bigl((x^c)^{1/c}\bigr) = (1/c)\psi(x^c)
\end{equation*}
and (\ref{eq:psi:xc}) holds for all \(c\geq 0\).

Hence for all \(x>0\),
\begin{equation*}
 \psi(x) = \psi\left(e^{\log x}\right) = \psi(e)\log x
\end{equation*}
and using (\ref{eq:ex:3.6:psi}) we see that any such \(\Phi\) must satisfy
\begin{equation*}
\Phi(x) = \psi(e)\log x + \Phi(1).
\end{equation*}
Clearly the converse, is true. That is for any \(a,b\in\R\)
the function \(\Phi(x) = a\log x + b\) satisfies the required condition.


%%%%%%%%%%%%%% 7
\begin{excopy}
For some measures, the relation \(r<s\) implies
\(L^r(\mu) \subset L^s(\mu)\);
for others, the inclusion is reversed;
and there are some for which \(L^r(\mu)\) does not contain \(L^s(\mu)\)
if \(r\neq s\). Give examples of these situations, and find conditions
on \(\mu\) under which these situations will occur.
\end{excopy}

For trivial atomic measure space \(\mu\) with only \(\{\emptyset,X\}\)
in its \salgebra, \(L^p(X,\mu)\) consists of measurable functions $f$
such that \(|f|<\infty \;\aded\).
In particular
\begin{equation*}
L^r(\mu)=L^s(\mu)\supset L^r(\mu)\subset L^s(\mu).
\end{equation*}
In Exercise~5(c) we saw a case (\(\mu(X)<\infty\) where for \(r<s\) we showed
\(L^s(\mu) \subsetneq L^r(\mu)\).
% Put \(t=(r+s)/2\).

Put \(t = (1/r+1/s)/2 > 0\).
Note that \(st>1\) and \(rt<1\) or equivalently \(-st < -1 < -rt\).

Now in \(\R^+\) with Lebesgue measure, the function \(f(x)=x^{-t}\) satisfies
\begin{equation*}
\|f\|_s^s = \int_0^\infty x^{-st},dm < \infty
\end{equation*}
while
\begin{equation*}
\|f\|_r^r = \int_0^\infty x^{-rt},dm > \infty.
\end{equation*}
Thus \(L^r(\mu) \nsupseteq L^s(\mu)\).


%%%%%%%%%%%%%% 8
\begin{excopy}
 If $g$ is a positive function on \((0,1)\) such that
\(g(x)\to\infty\) as \(x\to 0\),
then there is a convex function $h$ on \((0,1)\) such that \(h\leq g\)
and
\(h(x)\to\infty\) as \(x\to 0\).
True or false?
Is the problem changed of \((0,1)\) is replaced by \((0,\infty)\)
and \(x\to 0\) is replaced by \(x\to\infty\).
\end{excopy}

First part is true.
Let \((g(x)\) be as requried. We will now define a decreasing sequence
\seqan\ converging to $0$ by induction.
Let \(a_0=1\) and
let \(a_1>0\) be such that \(g(x)>1\) for all \(x<a_1\).
Now assume \(a_i\) are defined for \(i<k\).
Let \(a_k > 0\) be such that
\begin{itemize}
 \item  \(a_k\leq a_{k-1}/2\)
 \item  \(g(x) > k\) for all \(x < a_k\).
\end{itemize}
We now define $h$ on the segments \(I_i = [a_{i+1},a_{i}]\) by induction.
For \(x\in I_0\), we set \(h(x)=0\).
Assume $h$ was defined for the segments \(I_i\) for \(i<k\).
In particular, \(h(a_k)\) is defined.
For \(x\in I_k\) let \(h(x)\) get the values of the lines passing
through \(L=(0,k)\) and \(R=(a_k,h(a_k))\).
It is easy to see that $h$
\begin{itemize}
 \item is less than $g$.
 \item is continuous.
 \item converges to \(\infty\) as \(x\to 0\)
 \item is decreasing and its absolute slopes values increase,
       and thus it is convex.
\end{itemize}

Second part is false. Simply look at \(g(x)=\log(x)\). Any convex $h$
satisfying the requirements will have some \(0<a<b<\infty\)
with \(f(a)<f(b)\). But then the line passing through
\(A=(a,h(a))\) and
\(B=(b,h(b))\)
would intersect \(\log(x)\) at some point $M$, but then
\(h(M)<\log(x) = h(a) + (h(b)-h(a))(x-a)/(b-a)\)
contradicting $h$'s convexity.

%%%%%%%%%%%%%% 9
\begin{excopy}
Suppose $f$ is Lebesgue measurable on \((0,1)\), and not essentially bounded.
By Exercise~4\ich{e}, \(\|f\|_p\to\infty\) as \(p\to\infty\).
Can \(\|f\|_p\) tend to \(\infty\) arbitrarily slowly?
More precisely, is it true that for any function \(\Phi\) on \((0,\infty)\)
such that \(\Phi(p)\to\infty\) as \(p\to\infty\) one can find an $f$ such that
\(\|f\|_p\to\infty\) as \(p\to\infty\), but
\(\|f\|_p \leq \Phi(p)\) for all sufficiently large $p$?
\end{excopy}

The answer is \emph{yes}. Assume \(\Phi\) is as described above.
We will construct a ``step'' function $f$ as desired.

We can assume that \(\Phi\) is monotonically increasing.
Otherwise, we can simply look at
\begin{equation*}
 \Psi(x) = \inf_{w\geq x}\Phi(w)
\end{equation*}
instead.

For each \(k\in\Z\),
let \(p_k>0\) be such that \(\Phi(p)\geq k\) for all \(p>p_k\),
and define
\begin{equation*}
 \alpha_k
 \eqdef \inf_{p_1\leq p } \bigl(\Phi(p)/k\bigr)^p
 = \inf_{p_1\leq p \leq p_k } \bigl(\Phi(p)/k\bigr)^p.
\end{equation*}
Clearly,
\begin{equation*}
0 < 1/k^{p_k} \leq \alpha_k \leq 1.
\end{equation*}
Now, define ``interval lengths'' \(m_k = 2^{-k}\alpha_k\).
We can easily form disjoint open sub-intervals \(I_k\) of \([0,1]\)
such that \(m(I_k) = m_k\). Finally, we define
\begin{equation*}
f(x) = \sum_{k=1}^\infty \chhi_{I_k}(x) \cdot k =
    \left\{\begin{array}{ll}
             k & \qquad x\in I_k\\
             0 & \qquad \textrm{otherwise}
           \end{array}\right.
\end{equation*}
To show the required inequality, for \(p\geq p_1\)
\begin{equation*}
\|f\|_p^p
  = \sum_{k=1}^\infty m_k k^p
  \leq \sum_{k=1}^\infty 2^{-k} \bigl(\Phi(p)/k\bigr)^p k^p
  = \sum_{k=1}^\infty 2^{-k} \bigl(\Phi(p)\bigr)^p
  = \bigl(\Phi(p)\bigr)^p
\end{equation*}
Hence \(\|f\|_p \leq \Phi(p)\) for \(p\geq p_1\).


%%%%%%%%%%%%%% 10
\begin{excopy}
Suppose \(f_n\in L^p(\mu)\), for \(n=1,2,3,\ldots\), and
\(\|f_n-f\|_p\to 0\) and
\(f_n\to g\) \aded, as \(n\to \infty\).
What relation exists between $f$ and $g$?
\end{excopy}

By Theorem~3.12 \cite{RudinRCA80}, there is a subsequence of \(\{f_n\}\)
that converges pointwise to $f$ \aded. Obviousy this subsequence
converges to $g$ \aded. These two convergences may be missed
in two sets of measure $0$, and so is their union.
Thus \(f=g \aded\).

%%%%%%%%%%%%%% 11
\begin{excopy}
Suppose \(\mu(\Omega)=1\), and suppose $f$ and $g$ are positive measurable
functions on \(\Omega\) such that \(fg\geq 1\). Prove that
\begin{equation*}
 \int_\Omega f\,d\mu \cdot \int_\Omega g\,d\mu \geq 1.
\end{equation*}
\end{excopy}

Clearly \((fg)^{1/2} \geq 1\), hence by H\"older's inequality,
\begin{equation*}
1 \leq \int_\Omega f^{1/2} g^{1/2}\,d\mu
  \leq \left(\int_\Omega f\,d\mu\right)^{1/2}
       \left(\int_\Omega g\,d\mu\right)^{1/2}
\end{equation*}
Taking squares gives the desired inequality.


%%%%%%%%%%%%%% 12
\begin{excopy}
Suppose \(\mu(\Omega)=1\) and \(h:\Omega\to[0,\infty]\) is measurable. If
\begin{equation*}
 A = \int_\Omega h\,d\mu,
\end{equation*}
prove that
\begin{equation*}
 \sqrt{1+A^2} \leq \int_\Omega \sqrt{1+h^2}\,d\mu \leq 1 + A.
\end{equation*}
If \(\mu\) is Lebesgue measure on \([0,1]\) and if $h$ is continuous, \(h=f'\),
the above inequalities have simple geometric interpretation.
From this, conjecture (for general \(\Omega\)) under what conditions on $h$
equality can hold in either of the above inequalities, and prove your
conjecture.
\end{excopy}

We define a function \(\varphi\) and compute:
\begin{eqnarray}
 \varphi(x)  &=& \sqrt{x^2+1}         \label{eq:phi:sqrt:x2} \\
 \varphi'(x) &=& x/\sqrt{x^2+1}       \notag \\
 \varphi''(x) &=& 1/(x^2+1)^{3/2} > 0 \notag
\end{eqnarray}
Thus \(\varphi(x)\) is convex. Applying Jensen Theorem~3.3 to $h$ gives
\begin{equation*}
\sqrt{1+A^2} = \varphi\left(\int_\Omega h\,d\mu\right)
 \leq \int_\Omega (\varphi \circ h)\,d\mu
 = \int_\Omega \sqrt{1+h^2}\,d\mu.
\end{equation*}
the first desired inequality. The second inequality is immediate
by noting that \(\sqrt{1+x^2}\leq 1+x\)  for \(x\geq 0\), then integrating
over \(\Omega\).

For the \(h'=f\) case, here is the geometric interpretation.
The function $f$ is increasing.
The length of the graph curve of \((x,f(x))\) is
\(\int_0^1 \sqrt{1+h^2}\,d\mu\) which is greater or equal
the distance \(\sqrt{1+A^2}=d((0,f(0)),(1,f(1)))\) of its endpoints,
but is less or equal the Manhattan distnace \(1+A\).

The conjecture
\begin{itemize}
 \item[(i)]  \(\sqrt{1+A^2} = \int_\Omega \sqrt{1+h^2}\,d\mu\)
             \,iff\, $h$ is constant \aded.
 \item[(ii)]  \(\int_\Omega \sqrt{1+h^2}\,d\mu = 1 + A\)
             \,iff\, \(h=0\;\aded\).
\end{itemize}

\textbf{Proof.}
\newline
\textbf{(i).} If \(h(x)=c \;\textrm{a.e.}\)
then clearly \(A=c\) and the equality follows. Conversely,
if $h$ is not constant \aded, then
since \(varphi\) (defined in (\ref{eq:phi:sqrt:x2})) is strictly convex,
we can apply local lemma~\ref{lem:jensen:strict}
to get strict inequality
% \begin{equation*}
\(\sqrt{1+A^2} < \int_\Omega \sqrt{1+h^2}\,d\mu\).
% \end{equation*}
\newline
\textbf{(ii).} If \(h(x)=0 \;\textrm{a.e.}\) then clearly \(A=0\)
and the equality follows. Conversely, assume
\(\int_\Omega \sqrt{1+h^2}\,d\mu = 1 + A\).
Hence
\(\int_\Omega 1 + h - \sqrt{1+h^2}\,d\mu = 0\) and since the integrand
is non negative, \(1 + h - \sqrt{1+h^2} = 0\;\aded\), thus \(h=0\;\aded\).

%%%%%%%%%%%%%% 13
\begin{excopy}
Under what conditions on $f$ and $g$ does equality hold in the inclusions
of Theorem~3.8 and~3.9? You may have to treat the cases \(p=1\) and \(p=\infty\)
separately.
\end{excopy}

Call the domain of the functions $X$.
\begin{itemize}
 \item[\textbf{[3.8]}]
  Given conjugate \(1\leq \,p\,,q\,\leq \infty\),
  if \(f\in L^p(\mu)\)
  and \(g\in L^q(\mu)\), then
  \begin{equation} \label{eq:fg1:eq:fpgq}
      \|fg\|_1 = \|f\|_p\cdot\|g\|_q
  \end{equation}
  iff one of the following occurs:
\begin{itemize}
 \itemdim \emph{Normal}: \(1<p,q<\infty\) and
        the functions \(f^p\) and \(g^q\) are effectively proportional
        (see \cite{Hardy:1952:I} \textbf{189.}).
 \itemdim \emph{Supremum}:
            \(p=1\) and \(q=\infty\) and
            \(|g(x)| = \|g\|_\infty\) constant \aded\ on \(\supp f\)
 \itemdim \emph{Supremum}:
            \(q=1\) and \(p=\infty\) and
            \(|f(x)| = \|f\|_\infty\) constant \aded\ on \(\supp g\)
\end{itemize}

  \emph{Normal}: Assume \(1<p,q<\infty\),
  since \(1/p+1/q=1\) we have
  \(q/p = q - 1\).
  If \(f^p\) and \(g^q\) are effectively proportional, by symmetry, we can assume
  there is some scalar $a$ such that \(f^p = a g^q \aded\). Then
  \(|g|^{q/p+1} = |g|^q\). Now compute:
 \begin{eqnarray*}
  \|fg\|_1
    &=& \int |fg|\,d\mu
    = \int (a|g|^q)^{1/p}|g|\,d\mu
    = a^{1/p} \int |g|^{q/p+1}\,d\mu
    = a^{1/p} \int |g|^q\,d\mu \\
    &=& a^{1/p} \left(\int |g|^q\,d\mu\right)^{1/p + 1/q}
    = \left(\int a|g|^q\,d\mu\right)^{1/p}
      \left(\int  |g|^q\,d\mu\right)^{1/q} \\
    &=& \|f\|_p \|g\|_q
 \end{eqnarray*}

 Conversely, assume (\ref{eq:fg1:eq:fpgq}) holds. Then
 \begin{equation*}
 \int \left(|f|^p\right)^{1/p}g|
      \left(|g|^q\right)^{1/q}g|
        \,d\mu
 = \int |fg|\,d\mu
 = \|f\|_p \|g\|_q
 = \int \left(|f|^p\,d\mu\right)^{1/p}
   \int \left(|g|^q\,d\mu\right)^{1/q}.
 \end{equation*}
 By local lemma~\ref{lem:holder}
 with \(k=2\), we conclude that the functions \(|f|^p\) and \(|g|^q\)
 are effectively proportional.

 % \smallskip
 \medskip
 \emph{Supremum}: The two cases are symmetrical.
 Assume \(p=1\) and \(q=\infty\).
 If $g$ is constant \aded\ on \(\supp f\) then
 \begin{equation*}
 \int |fg|\,d\mu = \int_{\esssup f} |fg|\,d\mu
 = \left(\int_{\esssup f} |f|\,d\mu\right)\|g\|_\infty
 = \|f\|_1 \|g\|_\infty
 \end{equation*}
 Conversely, assume (\ref{eq:fg1:eq:fpgq}) holds. By negation
 assume that \(|g(x)|\) is not a constant \(\|g\|_\infty\) on \(E=\esssup f\).
 Then $g$ ``misses'' the supremum on $f$'s support. More precisely,
 there exists $a$ such that \(0\leq a < \|g\|_\infty\)
 such that the set
 \begin{equation*}
  B \eqdef \{x\in \esssup f: |g(x)| < a\} \subset \esssup \subset X
 \end{equation*}
 has \(\mu(B)> 0\). Now
 \begin{eqnarray}
 \int_X |fg|\,d\mu
 &=&    \notag
         \int_{X\setminus B} |fg|\,d\mu + \int_B |fg|\,d\mu \\
 &\leq& \notag
          \|g\|_\infty \int_{X\setminus B} |f|\,d\mu
        + a \int_B |f|\,d\mu \\
 &<&    \label{eq:gB:a}
          \|g\|_\infty \int_{X\setminus B} |f|\,d\mu
        + \|g\|_\infty \int_B |f|\,d\mu \\
 &=&    \notag
        \|f\|_1 \cdot \|g\|_\infty
 \end{eqnarray}
 Where the strict inequality in (\ref{eq:gB:a})
 by local lemma~\ref{lem:fgz:igz} gives a contradiction.

 \item[\textbf{[3.9]}]
  Given \(1\leq p \leq \infty\),
  if \(f,g\in L^p(\mu)\)
  then
  \begin{equation} \label{eq:fg:p:fpgp}
      \|f + g\|_p = \|f\|_p + \|g\|_p
  \end{equation}
  iff one of the following occurs:
   \begin{itemize}
    \itemdim \emph{Trivial}:  Both expressions are zero.
    \itemdim \emph{Manhattan}: \(p=1\) and
             \begin{equation} \label{eq:absfg:absfabsg}
              |f(x)+g(x)| = |f(x)| + g(x)\;\aded
             \end{equation}
    \itemdim \emph{Normal}: \(1< p <\infty\) and the functions
          $f$ and $g$ are effectively proportional
          (see \cite{Hardy:1952:I} \textbf{198.}).
    \itemdim \emph{Supremum}: \(p=\infty\) and for each \(\epsilon>0\)
          the set
          \begin{equation*}
           S_\epsilon =
               \bigl\{x\in X:
              \bigl|\left(\|f\|_\infty + \|g\|_\infty\right) - (f(x)+g(x))\bigr|
               < \epsilon\bigr\}
          \end{equation*}
          has positive measure.
   \end{itemize}
    \emph{Trivial}: Both expression are zero. For the rest of the cases
       (especially \emph{Normal}) we may assume \(\|f+g\|_p > 0\).

    \medskip
    \emph{Manhattan}:
    Assume \(p=1\).
    Clearly  \(|f(x)+g(x)| \leq |f(x)| + g(x)\;\aded\).
    Now \(\ref{eq:fg:p:fpgp}\) is equivalent to
    \begin{equation*}
    \int  |f| + |g| - |f+g|\,d\mu = 0.
    \end{equation*}
    this is equivalent by local lemma~\ref{lem:fgz:igz}
    to~(\ref{eq:absfg:absfabsg}).

    \medskip
    \emph{Normal}: Assume  \(1\leq p<\infty\).
    If $f$ and $g$ are effectively proportional, say \(f=ag\)
    for some scalar \(a>0\), then
    \begin{eqnarray*}
    \|f + g\|_p
    &=& \left(\int (a+1)^p|g|^p\,d\mu\right)^{1/p}
              = (a+1)\left(\int |g|^p\,d\mu\right)^{1/p} \\
       &=&   \left(\int |ag|^p\,d\mu\right)^{1/p}
            + \left(\int |g|^p\,d\mu\right)^{1/p} \\
    &=& \|f\|_p + \|g\|_p
    \end{eqnarray*}
    Conversely, assume (\ref{eq:fg:p:fpgp}) holds.
    Let \(S \eqdef \|f + g\|_p\),
    and $q$ be the conjugate exponent of $p$, that is \(1/p+1/q = 1\)
    or \(p=(p-1)q\).
    We compute
    \begin{eqnarray}
    S^p
    &=& \notag
     \|f + g\|_p^p  \\
    &=& \notag
     \int |f+g|^p\,d\mu \\
    &\leq& \label{eq:ex13:f+g}
           \int |f|\cdot |f+g|^{p-1}\,d\mu
         + \int |g|\cdot |f+g|^{p-1}\,d\mu \\
    &\leq& \label{eq:fg:p-1}
         \left(\int |f|^p\,d\mu\right)^{1/p}
            \left(\int |f+g|^{(p-1)q}\,d\mu\right)^{1/q}
         + \\
    &&   \notag
          \left(\int |g|^p\,d\mu\right)^{1/p}
            \left(\int |f+g|^{(p-1)q}\,d\mu\right)^{1/q} \\
    &=&  \notag
         \left(\|f\|_p + \|g\|_p\right) \|f+g\|_p^{p/q} \\
    &=&  \notag
         \left(\|f\|_p + \|g\|_p\right) \|f+g\|_p^{p-1}
    \end{eqnarray}
    The inequality (\ref{eq:fg:p-1}) is by applying
    Theorem~3.8 (\cite{RudinRCA80}) twice, and it is an equality
    iff both \(f^p\) and \(g^p\) are effectively proportional
    to \((f+g)^{p-1}\). Since \(\|f+g\|_p > 0\), by assumption,
    the above inequalities (divided by \(\|f+g\|_p^{p-1}\))
    are indeed equalities.
    Hence \(|f|^p\) and \(|g|^p\) are effectively proportional
    and so are \(|f|\) and \(|g|\).
    But since we equality also in (\ref{eq:ex13:f+g})
    we have \(|f+g|=|f|+|g| \;\aded\),
    and so $f$ and $g$ are effectively proportional.

    \medskip
    \emph{Supremum}: Assume \(p=\infty\).
    For any \(\mu\)-measurable function \(\varphi\)
    we have \(\|\varphi\|_\infty = \esssup \varphi\).
    Let \(M=\|f + g\|_\infty\) and
        \(N = \|f\|_\infty + \|g\|_\infty\).
    Clearly \(|f(x)+g(x)|\leq M \leq N\) for all \(x\in X\).
    If (\ref{eq:fg:p:fpgp}) holds, then for any \(\epsilon>0\)
    the set \(\{x\in X: |f(x)+g(x)| > M-\epsilon\}\) has non zero measure.
    But in this case, this set is exactly \(S_\epsilon\).
    Conversely, if \(\mu(S_\epsilon)>0\) for and \(\epsilon > 0\)
    then \(\|f+g|_\infty \geq M-\epsilon\) and so we get
    (\ref{eq:fg:p:fpgp}).


\end{itemize}

%%%%%%%%%%%%%% 14
\begin{excopy}
Suppose \(1<p<\infty\), \(f\in L^p=L^p((0,\infty))\),
relative to Lebesgue measure, and
\begin{equation*}
 F(x) = \frac{1}{x} \int_0^x f(t)\,dt \qquad (0<x<\infty)
\end{equation*}
\begin{itemize}
\itemch{a}
 Prove Hardy's inequality
 \begin{equation*}
  \|F\|_p \leq \frac{p}{p-1} \|f\|_p
 \end{equation*}
 which shows that the mapping \(f\to F\) carries \(L^p\) into \(L^p\).
\itemch{b}
 Prove that equality holds only if \(f=0\) a.e.
\itemch{c}
 Prove that the constant \(p/(p-1)\) cannot be replaced by a smaller one.
\itemch{d}
 If \(f>0\) and \(f\in L^1\), prove that \(F\notin L^1\).\newline
\end{itemize}
 \qquad \emph{Suggestions}: \ich{a} Assume first that \(f\geq 0\) and
 \(f\in C_c((0,\infty))\). Integration by parts gives
 \begin{equation*}
   \int_0^\infty F^p(x)\,dx = -p \int_0^\infty F^{p-1}(x)xF'(x)\,dx.
 \end{equation*}
 Note that \(xF' = f - F\), and apply
 \index{Holder@H\"older}
 H\"older's inequality to \(\int F^{p-1}f\).
 Then derive the general case.
 \ich{c}~Take \(f(x)=x^{-1/p}\) on \([1,A]\), \(f(x)=0\) elsewhere,
 for large $A$.
\end{excopy}


% \begin{itemize}

%%%%%%%%%%%%%%%%%%%%%%%%%%%%%%%%
\textbf{\ich{a}.}
Following the suggestion, we first assume \(0\leq f\in C_c((0,\infty))\).
Then
\begin{equation*}
F(x) \eqdef \int_0^x f(t)\,dt
\end{equation*}
and \(F(x)\) are differentiable.
Integration by parts (see theorem~6.22 \cite{RudinPMA85})\\

\begin{quote}
\footnotesize
 Reminder: Assume \(\Phi'=\phi\) and \(\Psi'=\psi\)
 \begin{equation*}
  \int_a^b \Phi(x)\psi(x)\,dx
  = \Phi(b)\Psi(b) - \Phi(a)\Psi(a) - \int_a^b \phi(x)\Psi(x)\,dx.
 \end{equation*}
\end{quote}

gives:
\begin{equation} \label{eq:int0x:Fpx}
 \int_0^x F^p(x)\cdot 1\,dx
 = \bigl(\lim_{x\to\infty} F^p(x)\cdot x\bigr) - F^p(0)\cdot 0
   - \int_0^x pF^{p-1}(x)F'(x)x\,dx .
\end{equation}
Since \(f(x)=0\) when \(x>b\) for some sufficiently large \(b<\infty\),
we can use the boundary  \(M\int_0^\infty |f(x)|\,dx\) to estimate
\begin{equation*}
 F^p(x)\cdot x
 \leq (M/x)^p x = Mx^{-(p-1)}.
\end{equation*}
Thus \(\lim_{x\to\infty} F^p(x)x = 0\) and also \(F(0) = 0\).

By differentiation (see theorem~6.20 \cite{RudinPMA85})
\begin{equation*}
F'(x)
 = \frac{d}{dx}\left(x^{-1}\int_0^x f(t)\,dt\right)
 = -x^{-2}\int_0^x f(t)\,dt + x^{-1}f(x) = (1/x)\bigl(F(x) + f(x)\bigr).
\end{equation*}
Thus
\begin{equation}  \label{eq:ex14:xF}
xF'(x) = f(x) - F(x).
\end{equation}
Collecting these resulted equalities,
we can substitute in~(\ref{eq:int0x:Fpx})
\begin{equation*}
 \int_0^\infty F^p(x)\,dx
  =  - p \int_0^\infty F^{p-1}(x)\bigl(f(x) - F(x)\bigr)\,dx
\end{equation*}
and simplify to
\begin{equation*}
(p-1) \int_0^\infty F^p(x)\,dx =  p \int_0^\infty F^{p-1}(x)f(x)\,dx
\end{equation*}
By H\"older inequality with $p$ and its conjugate exponent of \(q=p/(p-1)\).
\begin{eqnarray}
 \left(\|F\|_p\right)^p
    = \left|\int F^{p-1}f\right| \notag
 &\leq& \|F^{p-1}\|_q \|f\|_p    \label{eq:ex14:a} \\
 &=& \left(\int_0^\infty F^{q(p-1)}(x)\,dx\right)^{(p-1)/p} \|f\|_p \notag \\
 &=& \left(\|F\|_p\right)^{p-1} \|f\|_p \notag
\end{eqnarray}
Combining with the recent simplified equality
\begin{equation} \label{eq:ex:3.14a}
 \|F\|_p \leq (p/p-1) \|f\|_p.
\end{equation}

Theorem~3.14 (\cite{RudinRCA80}) shows that the functions with compact support
are dense in \(L^p((0,\infty\), so there (\ref{eq:ex:3.14a})
holds as well.

%%%%%%%%%%%%%%%%%%%%%%%%%%%%%%%%
\textbf{\ich{b}.}
If \(f=0\) equality clearly holds. Conversely, assume \(f=0\; \aded\)
does not hold, and by negation assume equality holds.

If \(f\geq0\;\aded\) does not hold, take
\(g(x)=|f(x)|\) and \(G(x)=(1/x)\int_0^x g(t)\,dt\).
Then
\begin{equation*}
\|F\|_p\leq \|G\|_p \leq \frac{p}{p-1}\|g\|_p = \frac{p}{p-1}\|f\|_p
\end{equation*}
and so the above inequalities reduce to equalities.

Thus we may assume \(f\geq0\). The inequality (\ref{eq:ex14:a})
must also be an equality, and by previous exercise, extending theorem~3.8
\cite{RudinRCA80} (see \ref{eq:fg1:eq:fpgq}) above),
the functions \((F^{p-1}(x))^q = F^p(x)\) and \(f^p(x)\) must be
effectively proportional and so are \(F(x)\) and \(f(x)\).
Hence there is a scalar \(a\neq 0\) such that
\(aF(x) = f(x)\;\aded\) for \(x\in(0,\infty)\).
Since \(F(x)\) is continuous, we can consider \(f(x)\) to be continuous,
otherwise we can look at \(aF(x)\). Using (\ref{eq:ex14:xF}),
we will solve the following first order ordinary partial equation
\begin{equation*}
\frac{d}{dx} \log(F(x)) = \frac{F'(x)}{F(x)} = \frac{a-1}{x}
\end{equation*}
Integrating gives
\begin{equation*}
 \log(F(x)) = \int_1^x F'(t)/F(t)\;dt + c_0 = (a-1)\int_1^x 1/t\;dt + c_0
 = b\log(x) + c
\end{equation*}
for some constants \(c_0\), \(c_1\).
Hence
\begin{equation*}
F(x) = e^{b\log(x)+c} = e^c x^b
\end{equation*}
But then \(f(x) = ae^c x^b\) is a contradiction to \(f \in L^p((0,\infty))\).

%%%%%%%%%%%%%%%%%%%%%%%%%%%%%%%%
\textbf{\ich{c}.}
Given \(A>1\), let
\begin{equation*}
f(x) = \left\{ \begin{array}{ll}
                 x^{-1/p}   & \quad  x\in[1,A] \\
                 0         & \quad  \textrm{otherwise}.
                \end{array}\right.
\end{equation*}

Now compute
\begin{equation*}
\|f\|_p = \left(\int_1^A (x^{-1/p})^p\,dx\right){1/p}
        = \left(\log(A)\right)^{1/p}
\end{equation*}

For \(1\leq x \leq A\)
\begin{equation*}
F(x)
 = \left(\int_1^x t^{-1/p}\;dt\right)/x
 = \frac{p}{(p-1)x}\bigl(x^{(p-1)/p} - 1\bigr)
\end{equation*}
\begin{equation*}
F^p(x)
 = \left(\frac{p}{p-1}\right)^p \left(\frac{x^{(p-1)/p} - 1}{x}\right)^p
\end{equation*}
We need to estimate the last factor.
For each (large) $x$ We look for \(\epsilon_x\) such that
\begin{equation} \label{eq:ex14:eps}
 \left(\frac{x^{(p-1)/p} - 1}{x}\right)^p > \frac{1-\epsilon_x}{x}
\end{equation}
Isolating
\begin{equation*}
\epsilon_x > 1 - x\left(\frac{x^{(p-1)/p} - 1}{x}\right)^p
\end{equation*}
Using l'Hospital rule;
\begin{equation*}
\lim_{x\to\infty} x\left(\frac{x^{(p-1)/p} - 1}{x}\right)^p
% = \lim_{x\to\infty} \left(\frac{x^{1/p}x^{(p-1)/p} - x^{1/p}}{x}\right)^p
= \lim_{x\to\infty} \left(\frac{x - x^{1/p}}{x}\right)^p
= \lim_{x\to\infty} 1 - x^{1/p-1}/p = 0.
\end{equation*}
So for each $x$ % \(1\leq x \leq A\)
we can find \(\epsilon_x>0\) such that (\ref{eq:ex14:eps}) holds
and \(\lim_{x\to\infty}\epsilon_x = 0\) (with \(A\to\infty\) as well).

Next, we need estimate for integration.
\begin{equation*}
\|F\|_p^p
= \int_0^\infty F^p(x)\,dx
\geq \int_1^A F^p(x)\,dx
\geq \left(\frac{p}{p-1}\right)^p \int_1^A \frac{1-\epsilon_x}{x}\,dx.
\end{equation*}
We will improve the last estimate.

Given \(\eta > 0\)
there is an $h$ such
that \(\epsilon_x < \eta\) for any \(x \geq h\).
By local lemma~\ref{lem:fg:bnless}
(replacing integral domain minimum), there exists \(A<\infty\) such that
\begin{equation*}
   \int_1^A \frac{1-\epsilon_x}{x}\,dx \geq + (1 - \eta) \int_1^A 1/x\,dx
\end{equation*}
Thus, by choosing sufficiently large $A$, we get
\begin{equation*}
\|F\|_p^p
\geq \left(\frac{p}{p-1}\right)^p (1 - \eta) \int_h^A f^p(x)\,dx
\end{equation*}
which shows that \(p/(p-1)\) is the minimal constant,
satisfying (\ref{eq:ex:3.14a}).

%%%%%%%%%%%%%%%%%%%%%%%%%%%%%%%%
\textbf{\ich{d}.}
By local lemma~\ref{lem:fgz:igz}, there exist some \(a<\infty\)
such that
\begin{equation*}
0 < A = \int_0^a f(x)\,dx < \infty
\end{equation*}
But now
\begin{equation*}
\|F\|_1
= \int_0^\infty F(x)\,dx
= \int_0^\infty \left((1/x)\int_0^x f(t)\,dt\right)\,dx
\geq \int_a^\infty A/x\,dx = \infty
\end{equation*}


% \end{itemize}


%%%%%%%%%%%%%% 15
\begin{excopy}
Suppose \(\{a_n\}\) is a sequence of positive numbers. Prove that
\begin{equation*}
 \sum_{N=1}^\infty \left(\frac{1}{N} \sum_{n=1}^N a_n\right)^p
 \leq
 \left(\frac{p}{p-1}\right)^p \sum_{n=1}^\infty a_n^p
\end{equation*}
if \(1<p<\infty\). \emph{Hint}: If \(a_n\geq a_{n+1}\), the result can be made
 to follow from Exercise~14.
 This special case implies the general case.
\end{excopy}

We will first show that the supremum of
\begin{equation*}
 \sup_\pi \sum_{N=1}^\infty \left(\frac{1}{N} \sum_{n=1}^N \pi(a_n)\right)^p
\end{equation*}
where \(\pi\) runs over all permutations of \N, is achieved with
\(\bigl(\pi(a_n)\bigr)_{n\in\N}\) monotonically decreasing.
Assume \((a_n)_{n\in\N}\) is monotonically decreasing, and
\((b_n)_{n\in\N}\) any of its permutations.
Then clearly, for any $N$
\begin{equation*}
 \sum_{n=1}^N a_n > \sum_{n=1}^N b_n
\end{equation*}
and so
\begin{equation*}
 \sum_{N=1}^\infty \left(\frac{1}{N} \sum_{n=1}^N a_n\right)^p
 \geq
 \sum_{N=1}^\infty \left(\frac{1}{N} \sum_{n=1}^N b_n\right)^p.
\end{equation*}

Thus, it is sufficient to prove the desired inequality for
decreasing sequence \((a_n)_{n\in\N}\).
Let's define \(f:(0,\infty)\to (0,\infty)\), by
\(f(x) = a_{\lceil x \rceil}\).
From the trivial equality
\begin{equation*}
 \sum_{n=1}^N a_n = \int_0^N f(x)
\end{equation*}
we can use the previous exercise~14, with the definition of \(F(x)\)
\begin{eqnarray*}
 \sum_{N=1}^\infty \left(\frac{1}{N} \sum_{n=1}^N a_n\right)^p
 &=& \sum_{N=1}^\infty \left(\frac{1}{N} \int_0^N f(x) \right)^p \\
 &=& \sum_{N=1}^\infty F^p(N) \\
 &=& \int_0^\infty F^p(x)\,dx = \|F\|_p^p \\
 &\leq& \left(\frac{p}{p-1} \|f\|_p\right)^p
        = \left(\frac{p}{p-1}\right)^p \int_0^\infty f^p(x)\,dx  \\
 &=&  \left(\frac{p}{p-1}\right)^p \sum_{N=1}^\infty a_n^p\;.
\end{eqnarray*}



%%%%%%%%%%%%%% 16
\begin{excopy}
Prove
\index{Egoroff}
Egoroff's theorem: If \(\mu(X)<\infty\), if \(\{f_n\}\) is a sequence of complex
measurable functions which converges pointwise at every point of $X$,
and if \(\epsilon > 0\), there is a measurable set \(E\subset X\), with
\(\mu(X\setminus E)<\epsilon\), such that \(\{f_n\}\)
converges uniformly on $E$.

(The conclusion is that by redefining the \(f_n\) on a set of arbitrarily small
measure we can convert a pointwise convergent sequence to a uniformly
convergent one: note the similarity with
\index{Lusin}
Lusin theorem.)

\qquad\emph{Hint}: Put
\begin{equation*}
 S(n,k) = \cap_{i,j>n} \left\{x: |f_i(x) - f_j(x)| < \frac{1}{k}\right\},
\end{equation*}
show that \(\mu(S(n,k))\to \mu(X)\) as \(n\to\infty\), for each $k$,
and hence that there is a suitable increasing sequence \(\{n_k\}\) such that
\(E = \cap S(n_k,k)\) has the desired property.

Show that the theorem does not extend to \(\sigma\)-finite spaces.

Show that the theorem does extend (with the same proof)
to the situation in which the sequences \(\{f_n\}\) are replaced by families
\(\{f_t\}\), where $t$ ranges over the positive reals, and the assumption
is that \(f_t(x) \to f(x)\), as \(t\to\infty\), for every \(x\in X\).
\end{excopy}

Let \(f(x) \eqdef \lim_{n\to\infty} f_n(x)\) for every $x$.
Following the hint. Clearly \(S(n,k)\subset S(n+1,k)\) for all $n$, $k$.
Now \((f_n(x))_{n\in\N}\) is a Cauchy sequence for every $x$.
Equivelantly,
for every $x$ and every $k$, we have \(x\in\cap_n S(n,k)\) for some $n$.
Thus
\begin{equation*}
\lim_{n\to\infty}\mu(S(n,k)) = \mu(X) < \infty.
\end{equation*}
Given \(\epsilon > 0\), for every \(k>0\), pick \(n_k\) such that
\begin{equation*}
\mu(S(n,k)) > \mu(X) 2^{-k}\epsilon
\end{equation*}
Now
\begin{equation*}
\mu(E)
 =  \mu\bigl(\cap S(n_k,k)\bigr)
 \geq \mu(X) - \sum_{k=1}^\infty 2^{-k}\epsilon
 = \mu(X) - \epsilon.
\end{equation*}

Here is \(\sigma\)-finite space example,
where the above does not hold.
Let \(f_n:\R\to\R\) defined by
\begin{equation*}
 f_n(x) = \chhi_{[-n,+n]}(x).
\end{equation*}
Clearly, \(f_n(x) \to 1 \) for every \(x\in \R\) but uniformly
only on bounded subsets.

Assume we have a family \((f_t)_{t\in\R^+}\) of functions
such that a limit \(f(x) = \lim_{t\to\infty} f_t(x)\) exists
for every \(x\in X\).
The result does  extend with similar proof.
We can define a sequence \((g_n)_{n\in\N}\)
\begin{equation} \label{eq:ex16:gnf}
 g_n(x) = f(x) + \sup_{t\geq n} |f_t(x) - f(x)|.
\end{equation}
Clearly \(\lim_{n\to\infty} g(x) = f(x)\) pointwise.
Fron what we have shown, for every \(\epsilon > 0\)
we can find a subset \(E\subset X\) such that \(\mu(X\setminus E) < \epsilon\)
and \((g_n)_{n\in\N}\) converges uniformly on $E$.
By the definition (\ref{eq:ex16:gnf})
\((f_n)_{n\in\N}\) converges uniformly on $E$ as well.


%%%%%%%%%%%%%% 17
\begin{excopy}
\begin{itemize}
\itemch{a}
 If \(0<p<\infty\), put \(\gamma_p = \max(1,2^{p-1})\), and show that
 \begin{equation*}
 |\alpha - \beta|^p \leq \gamma_p(|\alpha|^p + |\beta|^p)
 \end{equation*}
 for arbitrary numbers \(\alpha\) and \(\beta\).
\itemch{b}
   Suppose \(\mu\) is a positive measure on $X$, \(0<p<\infty\),
   \(f\in L^p(\mu)\), \(f_n\in L^p(\mu)\), \(f_n(x)\to f(x)\;\textrm{a.e.}\),
   and \(\|f_n\|_p \to \|f\|_p\) as \(n\to\infty\).
   Show that then \(\lim\|f-f_n\|_p = 0\),
   by completing the two proofs that are sketched below.
  \begin{itemize}
   \item[(i)]
    By Egoroff's theorem, \(X=A\cup B\) in such a way that
    \(\int_A|f|^p<\epsilon\), \(\mu(B)<\infty\), and \(f_n\to f\) uniformly
    on $B$, Fatou's lemma applied to \(\int_B|f_n|^p\), leads to
    \begin{equation*}
      \lim\sup \int_A |f_n|^p\,d\mu \leq \epsilon
    \end{equation*}
   \item[(ii)]
    Put \(h_n = \gamma_p(|f|^p + |f_n|^p) - |f-f_n|^p\),
    and use Fatou's lemma as in the proof of Theorem~1.34.
  \end{itemize}
\itemch{c}
 Show that the conclusion of \ich{b} is false if the hypothesis
 \(\|f_n\|_p \to \|f\|_p\) is omitted, even if \(\mu(X)<\infty\).
\end{itemize}
\end{excopy}


\begin{itemize}
\itemch{a}
 Let \(a=|\alpha|\) and \(b=|\beta|\).
 Since \(|\alpha - \beta| \leq a + b\), it is sufficient to show
 \begin{equation} \label{eq:3:17a}
   (a+b)^p \leq \gamma_p(a^p + b^p).
 \end{equation}
 If \(a=0\) (or \(b=0\), by symmetry)
 then (\ref{eq:3:17a}) is trivial. So we can assume \(a\neq 0 \neq b\).
 Put \(x = b/a > 0\).
 We have two cases:

 \paragraph{Case 1.} Assume \(0<p<1\) and \(\gamma_p = 1\).

 We first show that for \(x\geq 0\)
 \begin{equation*}
 (x + 1)^p \leq (x^p + 1)
 \end{equation*}
 or equivalently
 \begin{equation*}
 f(x) = (x^p + 1) - (x + 1)^p \geq 0
 \end{equation*}
 Clearly \(f(0) = 0\).
 Since \(p-1 < 0\) the function \(x\to x^{p-1}\) is decreasing (for \(x\neq 0\)).
 Hence $f$ is increasing for \(x>0\) since
 \begin{equation*}
 f'(x) = p\bigl(x^{p-1} - (x+1)^{p-1}\bigr) > 0
 \end{equation*}

 Now the inequality is derived:
 \begin{equation*}
 (a+b)^p =  (a+xa)^p = (x+1)^p a^p
 \leq a^p(x^p + 1)
 =    a^p + (xa)^p
 =    \gamma_p(a^p + b^p)
 \end{equation*}

 \paragraph{Case 2.} Assume \(p\geq 1\) and \(\gamma=2^{p-1}\).
 We first we want to show (\ref{eq:ex3.17:a}).
 Define a 2-points measurable space: \(X=\{0,1\}\)
 with \(\mu(\{0\}) = \mu(\{1\}) = 1\). Let \(f,g:X\to[0,\infty]\)
 defined by
 \begin{equation*}
  f(0) = x\,, \quad f(1) = 1\,, \qquad g = 1\,.
 \end{equation*}
 Let $q$ be the conjugate exponent of $p$.
 By Minkowski inequality (Theorem~3.5(2) \cite{RudinRCA80}), we have
 \begin{eqnarray*}
   x + 1 = \int_X fg\,d\mu
   &\leq&
        \left(\int_X f^p\,d\mu\right)^{1/p}
        \left(\int_X f^q\,d\mu\right)^{1/q} \\
   &=& \left(x^p+1\right)^{1/p} \left(1^q+q^q\right)^{1/q}
   = 2^{1/q} \left(x^p+1\right)^{1/p}.
 \end{eqnarray*}
 Raising to $p$ power, noting that \(p/q = p-1\) we now have
 \begin{equation} \label{eq:ex3.17:a}
  (x+1)^p \leq 2^{p-1} (x^p+1)
 \end{equation}

 Now we can derive
 \begin{equation*}
 (a+b)^p = (a+xa)^p =  a^p (x+1)^p
  \leq 2^{p-1} a^p(x^p + 1) = \gamma_p(a^p + b^p).
 \end{equation*}


\itemch{b}
 (Directly, \emph{not} following the hint). We use our local variant of Lebesgue's
 dominated theorem~\ref{lem:Lebesgue:domvar} (page \pageref{lem:Lebesgue:domvar}).
 Looking at \(f_n^p\in L^1(\mu)\) playing both roles of
 \(\{f_n\}\) and \(\{g_n\}\) there. Now the final result
 \begin{equation*}
  \lim_{n\to\infty} \|f_n^p - f^p \| = 0.
 \end{equation*}
 follows.


 Now going through the suggested hints
 (though unecessary after result established).
 Take arbitrary \(\epsilon > 0\).
 Let \(K= \{x\in X: f(x)=0\}\) and
 \begin{equation*}
   X_n = \{x\in X: 2^{n-1} \leq |f(x)|^p < 2^n\}
 \end{equation*}
 Clearly \(X = K \disjunion (\Disjunion_{n\in\Z} X_n)\) and
 \begin{equation*}
  \|f\|_p^p = \int_X |f|^p\,d\mu = \sum_{n\in\Z} \int_{X_n} |f|^p\,d\mu < \infty.
 \end{equation*}
 Hence \(\mu(X_n) < \infty\) for all \(n\in\Z\),
 and we can find some \(N<\infty\) such that
 \begin{equation*}
  T = \sum_{|n|>N} \int_{X_n} |f|^p\,d\mu < \epsilon/2.
 \end{equation*}
 We put
 \begin{equation*}
   A' = \Disjunion_{|n|>N} X_n
 \end{equation*}
 and we have \(\int_{A'} |f|^p\,d\mu < \epsilon/2\).
 We will now deal with
 \begin{equation*}
   B' \eqdef X \setminus A' = \Disjunion_{n = -N}^N X_n
 \end{equation*}
 \begin{equation*}
 \|f\|_p^p - T
  = \sum_{|n|\leq N} \int_{X_n} |f|^p\,d\mu
  \geq \sum_{|n|\leq N} 2^{n-1}\mu(X_n)
  \geq 2^{N-1} \sum_{|n|\leq N} \mu(X_n)
  = 2^{N-1} \mu(B').
 \end{equation*}
 We will concentrate now on the \(2N+1\) sets \(\{X_n\}\) for \(|n|<N\).
 Each has a finite \(\mu\) measure.
 Restricting \(\{f_n\}\) and $f$ to these sets and applying Egoroff's theorem
 on each \(X_n\)
 gives a partition \(X_n = A_n \disjunion B_n\)
 such that
 \begin{equation*}
    \int_{A_n} |f|^p\,d\mu < 2^{-2(|n| + 1)}\epsilon
 \end{equation*}
 and \(\{f_n\}\to f\) uniformly on \(B_n\).
 Since \(f_n\) is bounded on \(X_n\),
 \(\{|f_n|^p\}\to |f|^p\) uniformly on \(X_n\) as well,
 and also on any finite sub-union of them
 and in particular on
 \begin{equation*}
  B \eqdef \Disjunion_{n= -N}^N B_n \subset B'.
 \end{equation*}
 We put
 \begin{equation*}
  A = X \setminus B = A' \disjunion \Disjunion_{n= -N}^N A_n\,.
 \end{equation*}

 Combining results on $A$ and $B$ gives
 \begin{eqnarray}
      \int_X |f|^p\,d\mu
  &=& \lim_{n\to\infty} \int_X |f_n|^p\,d\mu \notag \\
  &=& \lim_{n\to\infty} \int_A |f_n|^p\,d\mu + \int_B |f_n|^p\,d\mu \notag \\
  &=& \limsup_{n\to\infty} \int_A |f_n|^p\,d\mu +
      \liminf_{n\to\infty} \int_B |f_n|^p\,d\mu  \label{eq:3.17b:1} \\
  &=& \int_B |f|^p\,d\mu + \limsup_{n\to\infty} \int_A |f_n|^p\,d\mu \notag
 \end{eqnarray}
 where (\ref{eq:3.17b:1}) is by local lemma~\ref{lem:limsup:liminf}.
 Now we get hint-(\emph{i}) estimate:
 \begin{equation*}
  \limsup_{n\to\infty} \int_A |f_n|^p\,d\mu
  = \int_X |f|^p\,d\mu - \int_B |f|^p\,d\mu
  = \int_A |f|^p\,d\mu < \epsilon\,.
 \end{equation*}

 Since \(\mu(B) < \infty\) (finite sum of positive numbers),
 and \(f_n^p\to f^p\) uniformly on $B$, clearly
 \begin{equation} \label{eq:3.17b:3}
  \lim_{n\to\infty} \int_B |f_n^p - f|\,d\mu = 0.
 \end{equation}
 Hence it is sufficient to show that (\ref{eq:3.17b:3}) holds
 also for $A$ instead of $B$, this will make it valid for $X$ as well,
 thus establishing the desired conclusion.

 \begin{eqnarray}
  2\gamma_p \int_A |f|^p\,d\mu
  &=& \notag
   \int_A \gamma_p |f|^p\,d\mu
   +
   \int_A \lim_{n\to\infty} \gamma_p |f_n|^p\,d\mu
   +
   \int_A \lim_{n\to\infty} |f - f_n|^p\,d\mu \\
  &=& \notag
   \int_A \lim_{n\to\infty} \gamma_p(|f|^p + |f_n|^p) - |f - f_n|^p\,d\mu \\
  &=& \notag
   \int_A \liminf_{n\to\infty} \gamma_p(|f|^p + |f_n|^p) - |f - f_n|^p\,d\mu \\
  &\leq& \label{eq:3.17b:fatou}
   \liminf_{n\to\infty} \int_A \gamma_p(|f|^p + |f_n|^p) - |f - f_n|^p\,d\mu \\
  &\leq& \notag
   \limsup_{n\to\infty} \int_A \gamma_p(|f|^p + |f_n|^p)\,d\mu
   +\liminf_{n\to\infty} \left(-\int_A |f - f_n|^p\,d\mu\right) \\
  &=& \notag
   \gamma_p \int_A |f|^p\,d\mu
   + \gamma_p \limsup_{n\to\infty} \int_A |f_n|^p\,d\mu
   - \limsup_{n\to\infty} \int_A |f - f_n|^p\,d\mu \\
 \end{eqnarray}

 Where (\ref{eq:3.17b:fatou}) is by Fatou lemma and \ich{a}.
 Thus
 \begin{equation*}
  \limsup_{n\to\infty} \int_A |f - f_n|^p\,d\mu  + \gamma_p \int_A |f|^p\,d\mu
  \leq \gamma_p \limsup_{n\to\infty} \int_A |f_n|^p\,d\mu.
 \end{equation*}
 Therefore
 \begin{equation*}
  \limsup_{n\to\infty} \int_A |f - f_n|^p\,d\mu  \leq \gamma_p \epsilon.
 \end{equation*}
 This shows (\ref{eq:3.17b:3}) for $A$ replaceing $B$ as needed.


\itemch{c}
 We will construct a counterexample.
 Let \(X=[0,1]\) with Lebesgue measure.
 Define a sequence of functions in \(L^1\)
 \begin{equation*}
   f_n(x) = n(n+1)\chhi_{[1/(n+1),1/n]}(x) \qquad \textrm{for}\; 1\leq n\in\N.
 \end{equation*}
Clearly \(\lim_{n\to\infty} f_n(x) = 0\) for every \(x\in[0,1]\), but
\(\|f_n\|_1 = 1\)

\end{itemize}


%%%%%%%%%%%%%% 18
\begin{excopy}
Let \(\mu\) be a positive measure on $X$. A sequence \(\{f_n\}\) of complex
measurable functions on $X$ is said to \emph{converge in measure}
to the measurable function $f$ if to every \(\epsilon>0\) there corresponds
an $N$ such that
\begin{equation*}
 \mu(\{x: |f_n(x) - f(x)| > \epsilon\}) < \epsilon
\end{equation*}
for all \(n>N\).
(This notion is of importance in probability theory.)
Assume \(\mu(X)<\infty\) and prove the following statements:
\begin{itemize}
 \itemch{a} If \(f_n(x)\to f(x)\;\aded\), then \(f_n\to f\) in measure.
 \itemch{b} If \(f_n\in L^p(\mu)\) and \(\|f_n-f\|_p \to 0\),
            then \(f_n\to f\) in measure; here \(1\leq p \leq \infty\).
 \itemch{c} If  \(f_n\to f\) in measure, then \(\{f_n\}\) has a subsequence
            which converges to $f$ \aded.
\end{itemize}
Investigate the convergences of \ich{a} and \ich{b}.
What happens to \ich{a}, \ich{b}, and \ich{c} if \(\mu(X)=\infty\),
for instance, if \(\mu\) is Lebesgue measure on \(\R^1\)?
\end{excopy}


\begin{itemize}
\itemch{a}
 Given \(\epsilon>0\), for each \(x\in X\) there is a least $m_x$
 such that \(|f_n(x) - f(x)| < \epsilon\) for all \(n \geq m_x\).
 For each \(m\in\N\) define
 \begin{eqnarray*}
   X_m
    &\eqdef& \{x\in X: m = m_x\} \\
    &=& \{x\in X: \forall n\geq m,\, |f_n(x) - f(x)| < \epsilon
        \; \wedge\; (m=1 \,\vee\, \exists n<m,\,  |f_n(x) - f(x)| \geq \epsilon
       \}.
 \end{eqnarray*}
 Clearly \(X=\disjunion_{n\in\N}X_i\), and \(X_i\) are measurable,
 so there is \(N_\epsilon\)  such that
 \begin{equation*}
 \mu\left(X\setminus \Disjunion_{i=1}^{N_\epsilon} X_i\right) < \epsilon.
 \end{equation*}
 This shows that \(f_n\to f\) in measure.
\itemch{b}
 By negation, assume there is \(\epsilon>0\)
 such that for each $N$, there exists \(n>N\)
 such that the subset
 \begin{equation*}
    E \eqdef \{x\in X: |f_n(x) - f(x)| \geq \epsilon\}
 \end{equation*}
 satisfies \(\mu(E) \geq \epsilon\).
 But then
 \begin{equation*}
   \int_X |f_n - f|^p\,d\mu
   \geq \int_E |f_n - f|^p\,d\mu
   \geq \epsilon^p\mu(E) = \epsilon^{p+1}
 \end{equation*}
 Thus
 \begin{equation*}
 \limsup_{n\to\infty} \|f_n-f\|_p =
 \limsup_{n\to\infty} \left(\int_X |f_n - f|^p\,d\mu\right)^{1/p}
 \geq \epsilon^{(p+1)/p} > 0
 \end{equation*}
 contradiction the assumption that \(\|f_n - f\|_p \to 0\).

\itemch{c}
 First let's illustrate the difficulty by constructing a case
 of sequence of functions converging in measure but converging \emph{nowhere}.
 Define for \(n\geq 1\)
 \begin{eqnarray*}
   s_n &=& \sum_{k=1}^n 1/k \\
   a_n &=& s_n - \lfloor s_n \rfloor \\
   b_n &=& \min(1, a_n + 1/(n+1)
 \end{eqnarray*}
 Note that \(a_n \leq b_n = a_{n+1} \leq a_n + 1/n\)
 and \(b_n - a_n \leq 1/(n+1)\).
 The sequence of characteristic functions  \(\chhi_{[a_n,b_n]}\) converges
 to $0$, but converges nowehere.

 Now back to the exercise task. For each \(\epsilon_k = 2^{-k}\)
 pick \(N_k\) such that for any \(n>N_k\)
  \begin{equation*}
   B_k \eqdef \mu(\{x: |f_n(x) - f(x)| > \epsilon_k\}) < \epsilon_k = s^{-k}.
  \end{equation*}
 Now take the subsequence \(\calF = (f_{N_k + 1})_{k\in\N}\).
 We now show that this sequence converges almost everywhere.
 The sequence \calF\ does not converge for \(x\in X\)
 iff $x$ belong to infinitely many subsets \(B_k\).
 That is
 \begin{equation*}
  x\in  = B = \bigcap_{j=1}^\infty \left( \bigcup_{k=j}^\infty B_k \right)\,.
 \end{equation*}
 But clearly \(\mu(B) = 0\).
\end{itemize}

Now assume \(X=\R^1\) with Lebesgue measure.
With \(f_n(x) = \chhi_{[n,n+1]}\), clearly \(f_n(x)\to 0\) for all \(x\in\R\)
but not in measure, thus \ich{a} does not hold.
But both \ich{b} and \ich{c} hold, since in their proofs above
we did not make use of the fact that \(\mu(X)<0\).

%%%%%%%%%%%%%% 19
\begin{excopy}
Define the \emph{essential range} of a function \(f\in L^\infty(\mu)\)
to be the set \(R_f\) consisting of all complex numbers $w$ such that
\begin{equation*}
 \mu(\{x: |f(x) - w | < \epsilon\}) > 0
\end{equation*}
for every \(\epsilon > 0\). Prove that \(R_f\) is compact.
What relation exists between the set \(R_f\) and the number \(\|f\|_\infty\)?

Let \(A_f\) be the set of all averages
\begin{equation*}
 \frac{1}{\mu(E)}\int_E f\,d\mu
\end{equation*}
where \(E\in\frakM\) and \(\mu(E)>0\).
What relations exist between \(A_f\) and \(R_f\)?
Is \(A_f\) always closed?
Are there measures \(\mu\) such that \(A_f\) is convex
for every \(f\in L^\infty(\mu)\)?
Are there measures \(\mu\) such that \(A_f\) fails to be convex for
some \(f\in L^\infty(\mu)\)?

How are these results affected if \(L^\infty(\mu)\) is replaced
by \(L^1(\mu)\), for instance?
\end{excopy}

Since \(f\in L^\infty(\mu)\), the essential range \(R_f\) is bounded in \C.
To show compactness, it is sufficent to show that \(R_f\) is closed.
By negation, let \(z\in \overline{R_f} \setminus R_f\).
For every open ball \(B = B(z,\epsilon)\)
with center $z$ and radii \(\epsilon > 0\)
there is \(w\in B\cap R_f\).
Now we take \(\eta = (\epsilon - |w-z|)/2\) and
\begin{equation*}
 G \eqdef \{x\in X: |f(x) - w| < \eta\} \subset
 H_\epsilon \eqdef \{x\in X: |f(x) - z| < \epsilon\}
\end{equation*}
since for \(|u-w|<\eta\;\Rightarrow\; |u - z|<\epsilon\) for any \(u\in\C\).
Since \(w\in R_f\), we have \(\mu(G) > 0\) but then \(\mu(H_\epsilon) > 0\).
This is contradiction to \(z\notin R_f\) since \(\epsilon\) was arbitrary.

It is easy to see that  \(\|f\|_\infty = \sup\{|z|: z\in R_f\}\).
Each number of \(A_f\) is a convex combination of numbers in \(R_f\).
Formally, \(A_f \subset \conv(R_f)\).

The set \(A_f\) need \emph{not} be closed. For example, let \(X=\N\)
with atomic measure \(\mu(\{n\}) = 1\) for all \(n\in X\),
and let \(f(n) = 1/n\). Clearly \(0\in \overline{A_f}\setminus A_f\).

There \emph{are} measures \(\mu\) for with \(A_f\) is convex
for every \(f\in L^\infty(\mu)\).
For example Lebesgue's measure on subsets of \(\R^n\) is shown
in \loclemma~\ref{llem:averages:convex}.

There \emph{are} measures for which \(A_f\) is not convex.
For example, let \(X=\{0,1\}\) with atomic counting measure \(\mu(A)=|A|\)
for all \(A\subset X\). For the function \(f(x) = x\), clearly
\(R_f=\{0,1\}\) and \(A_f=\{0,1/2,1\}\) which is not convex.

When replacing the space with \(L^1(\mu)\) we may get
a \emph{non} compact \(R_f\).
% \mldots


%%%%%%%%%%%%%%  20
\begin{excopy}
Suppose \(\varphi\) is a real function on \(\R^1\) such that
\begin{equation*}
 \varphi\bigl(\int_0^1 f(x)dx\bigr) \leq \int_0^1  \varphi(f)dx
\end{equation*}
for every real bounded measurable $f$. Prove that \(\varphi\) is then convex.
\end{excopy}

For any \(a,b\in\R\) and \(t\in [0,1]\)
define \(f:[0,1] \to \{a,b\}\) by
\begin{equation*}
 f(x) = \left\{\begin{array}{ll}
               a & \qquad x \in [0,t] \\
               b & \qquad x \in (t,1]
               \end{array}\right.
\end{equation*}
Now the assumed inequality gives
\begin{equation*}
  \varphi\bigl( ta + (1-t)b\bigr)
 =
  \varphi\left(\int_0^1 f(x)dx\right)
 \leq
  \int_0^1  \varphi(f)dx
 =
  t \varphi(a) + (1-t)\varphi(b)
\end{equation*}
which is the definition of convex function.



%%%%%%%%%%%%%% 21
\begin{excopy}
Call a metric space $Y$ a \emph{completion} of a metric space $X$ if $X$ is
dense in $Y$ and $Y$ is complete.
In Sec.~3.15 reference was made to ``the'' completion of a metric space.
State and prove a uniqueness theorem which justifies this terminology.
\end{excopy}

The uniqueness can be expressed by the following
\begin{llem}
Let \(Y_1\) and \(Y_2\) two completions of a metric space $X$.
Then there is a unique isometry (implying one-to-one onto continuous mapping)
\(T:Y_1\to Y_2\) inducing the identity on $X$.
\end{llem}
\begin{thmproof}
Given the \(Y_1\) and \(Y_2\) completions of $X$
define $T$ as the identity on $X$ and extend to \(Y_1\) as follows.
For each \(y_1\in Y_1\setminus X\)
there is Cauchy sequence \(\mathbf{x} = (x_n)_{n\in\N}\) in $X$
that converges to \(y_1\). Now \(\mathbf{x}\) is a Cauchy sequence
in \(Y_2\) and converges to a unique \(y_2\in Y_2\).
Define \(T(y_1) = y_2\). We need to show that the mapping is well defined.
Say \(\mathbf{w} = (w_n)_{n\in\N}\) in $X$ converges to \(y_1\) as well.
Then
\begin{equation*}
 x_1, w_1, x_2, w_2, \ldots x_n, w_n, \ldots
\end{equation*}
converges to \(y_1\) and this is a Cauchy sequence in $X$.
It must converge to a limit in \(Y_2\) which must be \(y_2\),
and so does \(\mathbf{w}\). Thus \(T(y_1)\) is independent on the choice
of converging sequence to \(y_1\).
\end{thmproof}

%%%%%%%%%%%%%% 22
\begin{excopy}
Suppose $X$ is a metric space in which every Cauchy sequence has a convergent
subsequence. Does it follow that $X$ is complete?
(See the proof of Theorem~3.11.)
\end{excopy}

The answer is ``Yes''.

If every Cauchy sequence \(\mathbf{x}\)
has a subsequence that converges to a limit $L$, then it can be easily
shown that  \(\mathbf{x}\) itself converges to $L$.

%%%%%%%%%%%%%% 23
\begin{excopy}
Suppose \(\mu\) us a positive measure on $X$, \(\mu(X)<\infty\),
\(f\in L^\infty(\mu)\), \(\|f\|_\infty > 0\), and
\begin{equation*}
\alpha_n = \int_X |f|^n\,d\mu \qquad (n=1,2,3,\ldots).
\end{equation*}
Prove that
\begin{equation*}
 \lim_{n\to \infty} \frac{\alpha_{n+1}}{\alpha_n} = \|f\|_\infty.
\end{equation*}
\end{excopy}

Put \(M = \|f\|_\infty\) and Define \(g = f/M\).
Now \(\|g\|_\infty = 1\) and
\begin{equation*}
 \frac{\alpha_{n+1}}{\alpha_{n}}
 =
 \frac{\int_X (Mg)^{n+1}\,d\mu}{\int_X (Mg)^{n}\,d\mu}
 =
 M \frac{\int_X g^{n+1}\,d\mu}{\int_X g^{n}\,d\mu}.
\end{equation*}
Hence it is sufficent to show that
\begin{equation} \label{3.23:gn}
 \lim_{n\to\infty} \frac{\int_X g^{n+1}\,d\mu}{\int_X g^{n}\,d\mu} = 1.
\end{equation}

By exercise~4\ich{e},
actually \(\lim_{n\to\infty} \int_X g^n\,d\mu = 1\)
and thus (\ref{3.23:gn}) follows.

%%%%%%%%%%%%%% 24
\begin{excopy}
Suppose \(\mu\) is a positive measure.
\(f\in L^p(\mu)\), \(g\in L^p(\mu)\).
\begin{itemize}
 \itemch{a}
  If \(0<p<1\), prove that
  \begin{equation*} % typo in book (extra '|')
   \int \bigl| |f|^p - |g|^p \bigr|\,d\mu \leq \int |f - g|^p \,d\mu
  \end{equation*}
  and that \(\Delta(f,g) = \int|f-g|^p\,d\mu\)
  defines a metric on \(L^p(\mu)\).
 \itemch{b}
  If \(1 \leq p < \infty\) and \(\|f\|_p\leq R\), \(\|g\|_p\leq R\),
  prove that
  \begin{equation*}
   \int \bigl| |f|^p - |g|^p \bigr|\,d\mu \leq 2pR^{p-1}\|f-g\|_p.
  \end{equation*}
  \emph{Hint}: Prove first, for \(x\geq 0\), \(y\geq 0\), that
  \begin{equation*}
   |x^p - y^p| \leq
   \left\{\begin{array}{ll}
          |x-y|^p                   & \quad \textrm{if}\; 0<p<1, \\
          p|x-y|(x^{p-1} + y^{p-1}) & \quad \textrm{if}\; 1\leq p < \infty.
          \end{array}\right.
  \end{equation*}
\end{itemize}
Note that \ich{a} and \ich{b} establish the continuity of the mapping
\(f\to |f|^p\) that carries \(L^p(\mu)\) into \(L^1(\mu)\).
\end{excopy}

Following the hints.
But symmetry of the expressions with regards to $x$ and $y$,
we can assume \wlogy\ that \(x>y\).

Assume \(0 < p < 1\). If \(y=0\) then
\begin{equation*}
 |x^p - 0^p| = x^p = |x-0|^p.
\end{equation*}
Otherwise, by dividing by \(y^p\) we need to show
\begin{equation} \label{eq:3.24:xp1}
 x^p - 1 \leq (x-1)^p
\end{equation}
or equivalently that
\begin{equation*}
 \varphi(x) = x^p - (x-1)^p - 1 \leq 0
\end{equation*}
for \(x\geq 1\).
Indeed \(\varphi(1) = 0\) and since \(x\to x^{p-1}\) is decreasing
\begin{equation*}
\varphi'(x) = p\bigl(x^{p-1} - (x-1)^{p-1}\bigr) < 0.
\end{equation*}
Hence \(\varphi\) is decreasing for \(x\geq 0\) and (\ref{eq:3.24:xp1}) holds.

Assume \(1\leq p < \infty\). \Wlogy, we can assume \(x\geq y\).
If \(y=0\)
\begin{equation*}
 x^p - 0^p \leq px^p = p(x-0)(x^{p-1} + 0^{p-1})
\end{equation*}
Otherwise, \(y>0\) and
\begin{equation*}
 x^p - y^p \leq p(x-y)(x^{p-1} + y^{p-1})
\end{equation*}
is equivalent --- dividing by \(y^p=yy^{p-1}\) --- to
\begin{equation*}
 (x/y)^p - 1 \leq p\bigl((x/y)-1\bigr)\bigl((x/y)^{p-1} + 1).
\end{equation*}
Hence it is sufficient to show
\begin{equation} \label{eq:3.24:xp2}
 x^p - 1 \leq p(x-1)(x^{p-1} + 1)
\end{equation}
or equivalently,
\begin{equation*}
 \varphi(x) = p(x-1)(x^{p-1} + 1) - (x^p - 1) \geq 0
\end{equation*}
for \(x\geq 1\). Clearly, \(\varphi(1) = 0\).
\begin{eqnarray*}
 \varphi'(x)
 &=& p\bigl( (x^{p-1} + 1) + (x-1)(p-1)x^{p-1} \bigr) - px^{p-1} \\
 &=& p\bigl( (x-1)(p-1)x^{p-1} + 1 \bigr) \\
 &\geq& 0.
\end{eqnarray*}
Thus \(\varphi\) is increasing for \(x\geq 1\) and  (\ref{eq:3.24:xp2}) holds.



\begin{itemize}
\itemch{a}
 The inequality is immediate from the integrands satisfy
 \(\bigl| |f|^p - |g|^p \bigr| \leq |f - g|^p\)
 by the hint's first (\(0<p<1\)) part.

 The function \(\Delta\) indeed is a metric since
 clearly \(\Delta(f,g) = 0\) iff \(f = g\,\aded\).
 and by the inequality we just justified,
 for every \(f,g,h \in L^p(\mu)\)
 \begin{equation*}
   \int \bigl| |f-g|^p - |g-h|^p \bigr|\,d\mu
    \leq \int |(f-g) - (g-h)|^p \,d\mu
    = \int |f - h|^p \,d\mu
 \end{equation*}
 But this shows
 \begin{equation*}
   \Delta(f,g) = \int |f-g|^p\,d\mu
    \leq \int |f - h|^p \,d\mu + \int |g - h|^p \,d\mu =
   \Delta(f,h) +  \Delta(h,g).
 \end{equation*}
 With the triangle shown, \(\Delta\) is a metric.

\itemch{b}
 Using the hint's inequality. Let $q$ be the conjugate exponent of $p$.
 Compute
 \begin{eqnarray}
   \int \bigl| |f|^p - |g|^p \bigr|\,d\mu
  &\leq& \label{eq:3.24b:hint}
   p \int |f-g|\cdot ( |f|^{p-1} + |g|^{p-1})\,d\mu \\
  &\leq& \label{eq:3.24b:mink}
   p \|f-g\|_p \left(\int (|f|^{p-1} + |g|^{p-1})^q\,d\mu\right)^{1/q} \\
  &\leq& \label{eq:3.24b:holder}
   p \|f-g\|_p \left(\int (|f|^{p-1})^q\,d\mu +
                     \int (|g|^{p-1})^q\,d\mu \right)^{1/q} \\
  &=& \label{eq:3.24b:expconj}
   p \|f-g\|_p \left(\int (|f|^p\,d\mu + \int (|g|^p\,d\mu \right)^{1/q} \\
  &=& \notag
   p \|f-g\|_p (\|f\|_p^p + \|g\|_p^p)^{1/q} \\
  &\leq& \notag
   p \|f-g\|_p (2R^p)^{1/q} =  p \|f-g\|_p 2^{1/q}R^{p-1} \\
  &\leq& \label{eq:3.24b:expconjgt1}
   2pR^{p-1}\|f-g\|_p. \notag
 \end{eqnarray}

Where
  (\ref{eq:3.24b:hint}) follows by the hint,
  (\ref{eq:3.24b:mink}) by Minkowski's inequality,
  (\ref{eq:3.24b:holder}) by H\"older's inequality,
  (\ref{eq:3.24b:expconj}) by $q$ being conjugate exponent (\((p-1)q = p\)),
  (\ref{eq:3.24b:expconjgt1}) by \(q\geq 1\).
\end{itemize}

% Added in 3rd edition

\begin{excopy}

Suppose \(\mu\) is a positive measure on $X$ and \(f: X\to(0,\infty)\) satisfies
\(\int_X f\,d\mu=1\).
Prove, for every \(E\subset X\) with \(0 < \mu(E) < \infty\), that
\begin{equation*}
\int_E (\log f)\,d\mu \leq \mu(E) \log \frac{1}{\mu(E)}
\end{equation*}
and, when \(0<p<1\).
\begin{equation*}
\int_E f^p\,d\mu \leq \mu(E)^{1-p}\,.
\end{equation*}
\end{excopy}

\begin{excopy}
If $f$ is a positive measure function on \([0,1]\), which is larger
\begin{equation*}
\int_0^1 f(x)\log f(x)\,dx 
\qquad \textnormal{or} \qquad
\int_0^1 f(s)\,ds \int_0^1 \log f(t)\,dt\; \textnormal{?} 
\end{equation*}
\end{excopy}


%%%%%%%%%%%%%%%
\end{enumerate}
%%%%%%%%%%%%%%%

 % -*- latex -*-
% $Id: rudinrca4.tex,v 1.2 2008/07/19 08:56:55 yotam Exp $

%%%%%%%%%%%%%%%%%%%%%%%%%%%%%%%%%%%%%%%%%%%%%%%%%%%%%%%%%%%%%%%%%%%%%%%%
%%%%%%%%%%%%%%%%%%%%%%%%%%%%%%%%%%%%%%%%%%%%%%%%%%%%%%%%%%%%%%%%%%%%%%%%
%%%%%%%%%%%%%%%%%%%%%%%%%%%%%%%%%%%%%%%%%%%%%%%%%%%%%%%%%%%%%%%%%%%%%%%%
\chapterTypeout{Elementary Hilbert Space Theory}

%%%%%%%%%%%%%%%%%%%%%%%%%%%%%%%%%%%%%%%%%%%%%%%%%%%%%%%%%%%%%%%%%%%%%%%%
%%%%%%%%%%%%%%%%%%%%%%%%%%%%%%%%%%%%%%%%%%%%%%%%%%%%%%%%%%%%%%%%%%%%%%%%
\section{Notes}

%%%%%%%%%%%%%%%%%%%%%%%%%%%%%%%%%%%%%%%%%%%%%%%%%%%%%%%%%%%%%%%%%%%%%%%%
\subsection{Completeness of the Trigonometric System}
\label{sec:comp:trig}

In showing the completeness of the trigonometric system,
on page~95, the text uses the following equality
\begin{equation*}
\frac{c_k}{\pi}\int_0^\pi \left(\frac{1 + \cos t}{2}\right)^k \sin t\,dt =
  \frac{2c_k}{\pi(k+1)}.
\end{equation*}
Let us show it in details. Define
\begin{equation*}
f(t) \eqdef \bigl((1+\cos(t))/2\bigr)^{k+1}.
\end{equation*}
Now
\begin{equation*}
f'(t)
= -\bigl(\sin(t)/2\bigr)\cdot(k+1)\bigl((1+\cos(t))/2\bigr)^k
% = -\bigl((k+1)\sin(t)/2\bigr)\cdot\bigl((1+\cos(t))/2\bigr)^k.
= \bigl(-(k+1)/2\bigr)\cdot\bigl((1+\cos(t))/2\bigr)^k\sin(t).
\end{equation*}
Hence
\begin{equation*}
\int_0^\pi \left(\frac{1 + \cos t}{2}\right)^k \sin t\,dt
 = \bigl(-2/(k+1)\bigr) \bigl(f(\pi) - f(0)\bigr)
 = \bigl(-2/(k+1)\bigr) (0 - 1)
 = 2/(k+1).
\end{equation*}
Which justifies the above quoted equality.


%%%%%%%%%%%%%%%%%%%%%%%%%%%%%%%%%%%%%%%%%%%%%%%%%%%%%%%%%%%%%%%%%%%%%%%%
%%%%%%%%%%%%%%%%%%%%%%%%%%%%%%%%%%%%%%%%%%%%%%%%%%%%%%%%%%%%%%%%%%%%%%%%
\section{Exercises} % pages 97-99

In this set of exercises, $H$ always denotes a Hilbert space.

%%%%%%%%%%%%%%%%%
\begin{enumerate}
%%%%%%%%%%%%%%%%%

%%%%%%%%%%%%%% 1
\begin{excopy}
If $M$ is a closed subspace of $H$, prove that
\(M = (M^\perp)^\perp\).
Is there a similar true statement for subspaces
$M$ which are not necessarily closed?
\end{excopy}

As discussed in section~4.9 \cite{RudinRCA80}, \(A^\perp\) is a closed
subspace for any \emph{subset} \(A\subset H\). Directly by definitions
we also have \(A \subset (A^\perp)^\perp\).
If we re-apply theorem~4.11(2) \cite{RudinRCA80}, with \(M^\perp\)
instead of $M$, we get again unique projections
$P$, $Q$ and \(P^\perp\)  of $H$ onto
$M$, \(M^\perp\) and \((M^\perp)^\perp\) respectably.
That is for any \(x\in H\) we have
\begin{equation*}
 x = Px + Qx = Qx + P^\perp x.
\end{equation*}
Hence \(P=P^\perp\) and so \(M = (M^\perp)^\perp\).
where $P$, $Q$ and \(P^\perp\) are projections of $H$ onto
$M$, \(M^\perp\) and \((M^\perp)^\perp\).

If $M$ is not closed, we cannot ensure the equality, as the following
example show. Let
\begin{equation*}
M \eqdef \{x\in\ell^2: \exists m < 0\,\forall n\geq m\; x(n) = 0\}.
\end{equation*}
Clearly $M$ is a non closed subspace of \(\ell^2\) and we have
\(M^\perp = \{0\}\) and
\(M \subsetneq \ell^2 = (M^\perp)^\perp\).


%%%%%%%%%%%%%% 2
\begin{excopy}
For \(n=1,2,3,\ldots\), let \(\{v_n\}\) be an independent set of vectors in $H$.
Develop a \emph{constructive} process which generates an orthonormal set
\(\{u_n\}\), such that \(u_n\) is a linear combination of \seqn{v}.
Note that this leads to a proof of the existence of a maximal orthonormal
set in a separable Hilbert spaces which makes no apeal to the Hausdorff
maximality principle.
(A space is \emph{separable}
\index{separable}
if it contains a countable dense subset.)
\end{excopy}

This is the
\index{Graham-Schmidt!Orthonormalization}
\index{Orthonormalization!Graham-Schmidt}
\emph{Graham-Schmidt Orthonormalization}.

By induction, Let \(u_1 \eqdef v_1/\|v_1\|\).
Let \(k\geq 1\).
Assume \(\{u_j\}\) defined for all \(1\leq j < k\).
Define
\begin{eqnarray*}
 {u'}_k &\eqdef& v_k - \sum_{j=1}^{k-1} \langle u_j,v_k \rangle \cdot u_j \\
 u_k    &\eqdef& {u'}_k \,/\, \|{u'}_k\|
\end{eqnarray*}


%%%%%%%%%%%%%% 3
\begin{excopy}
Show that \(L^p(T)\) is separable if \(1\leq p < \infty\),
but that \(L^\infty(T)\) is not separable.
\end{excopy}

Assume \(1\leq p < \infty\).
We will show that the set $Q$ of trigonometric polynomial
with (complex) rational coefficients is dense in \(L^p(T)\).
Clearly \(|Q| = \aleph_0\).

Assume \(\epsilon > 0\) and \(f\in L^p(T)\).
By Theorem~3.14 \cite{RudinRCA80} \(C_c(T)\) is dense in \(L^p(T)\).
Since $T$ is compact \(C_c(T)=C(T)\).
Take \(g\in C(T)\) such that \(\|g - f\|_p < \epsilon/3\).

By Theorem~4.25 \cite{RudinRCA80}, the trigonometric polynomial
are dense in C(T) in the \(\|\cdot\|_\infty\) norm.
Take a trigonometric polynomial $p$ such that
\(\|p - g\|_\infty < \epsilon/(6\pi)\), hence
\(\|p - g\|_p < \epsilon/3\).

For each coefficient \(c_j\) of $p$, we can find a sequence of
rationals \(\{q_{jk}\}_{k=1}^\infty\)
such that \(\lim_{k\to\infty} q_{jk} = c_j\).
Note that  the degree $N$ of $p$ is finite (\(c_j = 0\) if \(|j| > N\)).
Put
\begin{equation*}
h_k(t) \eqdef \sum_{j=-N}^N q_{jk} e^{ijt}
\end{equation*}
For each $k$,
we have \(\lim_{k\to\infty} q_{jk} e^{ijt} = c_k e^{ijt}\) uniformly.
Hence  \(\lim_{k\to\infty} h_k(t) = p(t)\) uniformly as well.
Therefore, we can pick some \(q_k \in Q\) such that
\(\|q - p\|_\infty < \epsilon/(6\pi)\), hence.
\(\|q - p\|_p < \epsilon/3\).
Combining the results, \(\|q - f\|_p < \infty\)
and so $Q$ is dense in \(L^P(T)\) which is thus separable.

We will now show that  \(L^\infty(T)\) is not separable.
For each \(r\in[-\pi,\pi)\) put
\begin{equation*}
 u_r \eqdef\chhi_{[-\pi,r]}.
\end{equation*}
Look at the set of \(R \eqdef \{u_r:  r\in[-\pi,\pi)\} \subset L^\infty(T)\).
Clearly \(|R| > \aleph_0\), but
for any two \(-\pi \leq r < s < \pi\) we have \(\|u_r - u_s\| = 1\).
Assume by negation there exists a countable dense set $D$ in \(L^\infty(T)\).
Put \(\epsilon = 1/3\). For each \(u_r\in R\) there exists some \(f_r\in D\)
such that \(\|u_r - f_r\|_\infty < 1/3\). By simple cardinality argument,
there must exist some pair \(r<s\) such that \(f_r = f_s\).
But then
\begin{equation*}
 \|u_r - u_s\|_\infty
 \leq \|u_r - f_r\|_\infty + \|f_r - f_s\|_\infty + \|f_s - u_s\|_\infty
 \leq 1/3 + 0 + 1/3 = 2/3 < 1
\end{equation*}
which us a contradiction.

%%%%%%%%%%%%%% 4
\begin{excopy}
Show that $H$ is separable if and only if $H$ contains a maximal orthonormal
system which is at most countable.
\end{excopy}

Assume $H$ is separable.
Let \(\{v_j\}_{j\in\N}\) a countable set which is dense in $H$.
By induction, we can through out all vectors \(v_k\)
which are depenent ob \(\{v_j\}_{1\leq j<k}\).
We can now use the
Graham-Schmidt!Orthonormalization (Exercise~2 above) to get orthonormal
system whose cardinality is at most \(|\N| = \aleph_0\).

Conversely, assume \(U=\{u_j\}_{j\in J}\) is a maximal orthonormal system
where \(|J| \leq \aleph_0\).
Clearly the set
of all finite linear combinations of $U$ with (complex) rational coefficients
\begin{equation*}
D \eqdef \left\{\sum_{j\in F} (q_j+ir_j)v_j :
    F\subset J \;\wedge\;
   |F|<\infty \;\wedge\;
   q_j, r_j \in \Q\right\}
\end{equation*}
is dense in $H$ and \(|D|=\aleph_0\).


%%%%%%%%%%%%%% 5
\begin{excopy}
If \(M = \{x: Lx = 0\}\), where $L$ is a continuous linear functional on $H$,
prove that \(M^\perp\) is a vector space of dimension $1$ (unless \(M=H\)).
\end{excopy}

Assume \(M\neq H\) and \(v\in H\setminus M\).
If by negation \(\dim(M^\perp)> 1 \) then there are
linearly independent \(v_1,v_2\in \M^\perp\).
Clearly \(v_1\neq 0 \neq v_2\)
and \(L(v_1)\neq 0 \neq L(v_2)\).
Put \(v \eqdef L(v_2)v_1 - L(v_1)v_2 \in M^\perp\).
By the independence of \(v_1,v_2\) we have \(v\neq 0\).
But
\begin{equation*}
L(v) = L\bigl( L(v_2)v_1 - L(v_1)v_2 \bigr) = L(v_2)L(v_1) - L(v_1)L(v_2) = 0.
\end{equation*}
which is a contradiction.


%%%%%%%%%%%%%% 6
\begin{excopy}
Let \(\{u_n\}\) (\(n=1,2,3,\ldots\)) be an orthonormal set in $H$.
Show that this gives an example of a closed and bounded set which is
not compact.
Let $Q$ be the set of all \(x\in H\) of the form
\begin{equation*}
x = \sum_1^\infty c_n u_n
   \qquad \left(\textrm{where} |c_n| \leq \frac{1}{n}\right).
\end{equation*}
Prove that $Q$ is compact. ($Q$ is called the Hilbert cube.)

More generally, let \(\{\delta_n\}\) be a sequence of positive numbers,
and let $S$ be the set of all \(x\in H\) od the form
\begin{equation*}
x = \sum_1^\infty c_n u_n
  \qquad \left(\textrm{where} |c_n| \leq \delta_n\right).
\end{equation*}
Prove that $S$ is compact if and only if \(\sum_1^\infty \delta_n^2 < \infty\).
Prove that $H$ is not locally compact.
\end{excopy}

The set \(U = \{u_n\}\) is clearly bounded by $1$.
It it was not closed and \(x\in\overline{U}\setminus U\) then
there would be a sequence $s$ in $U$ that converges to $x$.
\Wlogy, we can remove repetitions from the $s$.
But then $s$ cannot be a Cauchy sequence since \(\|u_j - u_k\|=2\)
for any \(j<k\).
Now let \(V = \{x\in H: \|x\| < 1/3\}\) be a zero neighborhood.
The family
\begin{equation*}
\{u_n + V\}_{n=1}^\infty
\end{equation*}
is an open covering of $U$ and clearly has no finite sub-covering.
Therefore, $U$ is not compact.

% Assume $G$ is a covering of the Hilbert cube $Q$.
Assume \(\sum_1^\infty \delta_n^2 < \infty\).
For each $n$, let
\begin{eqnarray*}
H_n &\eqdef& \{x\in H: x = \sum_{j=1}^n c_j u_j: |c_j|\}
             = \{x\in H: \forall m > n,\; \langle x,u_j\rangle = 0\}
             \subset S. \\
K_n &\eqdef& S \cap H_n
% K_n &\eqdef& \{\sum_1^n c_j u_j: |c_j| \leq \delta_j\} \subset S.
\end{eqnarray*}
Clearly \(K_n\) is homeomorphic to $n$-dimensional closed box in \(\C^n\)
and thus is compact (See Hiene Borel Theorem~2.41 \cite{RudinPMA85}).
\begin{quotation}
\small
Clearly \(\cup_n K_n \subset S\).
Equality \(\cup_n K_n = S\) holds iff there exist \(m<\infty\)
such that \(\delta_j = 0\) for all \(j>m\).
\end{quotation}

\iffalse
Let $G$ be an open covering of $S$.
For each $n$, the $G$ covering of $S$ is also covering of \(K_n\).
Let \(G_n \subset G\) a finite subcovering of \(K_n\).
We may also assume \(G_n \subset G_{n+1}\)
(otherwise, we simply annex \(G_n\) to \(G_{n+1}\)).
Suppose by negation that $S$ is \emph{not} compact.
Then \(\cup G_n \subsetneq S\) for each $n$.
Pick \(x_n \in S \setminus \cup G_n\).
We will now find a subsequence of \((x_n)_{n\in\N}\) that converges in $S$,
by getting subsequences that converge on projections on \(K_n\)'s.
Moving to double indexing, we put \(x_{0,n} = x_n\).
By induction assume that for $k$ the sequence \((x_{k,n})_{n\in\N}\)
is defined.
Pick a subsequences \((x_{k+1,n})_{n\in\N}\) whose projection on \(K_{k+1}\)
converges. Taking the diagonal subsequences, let \(y_n = x_{n,n}\).
Its projection (\(\langle\cdot,u_k\rangle\))
on \(K_k\) converges for all $k$.
Moreover its limit \(t = \sum_{k=1}^\infty \langle x,u_k \rangle u_k\)
converges in $S$ since \(\|t\| \leq \sum_{k=1}^\infty \delta_k^2\).
Thus \(t\in S\). Pick \(V\in G\) such that \(t\in V\).
There exist \(0<r\in\R\) such that \(B(t,r) \subset V\).
Let $m$ be such that
\begin{equation*}
\sum_{j=m+1}^\infty \delta_j^2 < r
\end{equation*}
Let \(t'\) be the projection of $t$ on \(H_m\).
So now \(t' \in K_n\).
\mldots
\fi

We will show that $S$ is homeomorphic to
the product space \(T = \prod_{k=1}^\infty \{z\in \C: |z|<\delta_k\)
with the weak topology.
By Tychonoff theorem (Appendix A~3 \cite{RudinFA79}) $T$ is compact.
The homeomorphism is given by the identity mapping. We need to show that
it is continuous in both directions.
\begin{itemize}
 \item
 Look at \(\Id: S \to T\). Let \(x\in S\) and note that \(\Id(x) = x\in T\).
 Let $V$ be a base neighborhood of $x$ in $T$. That is for some finite
 subset \(I \subset \N\) we have
 \begin{equation*}
 V = \{v\in T: \forall j\in I,\, |v_j - x_j| < \epsilon_j\}.
 \end{equation*}
 Now take
 \begin{equation*}
 U = \{s\in S: \|s-x\| < \min_{j\in J} \epsilon_j\}
 \end{equation*}
 and clearly \(\Id(U) = U \subset V\).

 \item
 Look at \(\Id: T \to S\).
 Let \(x\in T\) and note that \(\Id(x) = x\in S\).
 Pick a neighborhood
 \begin{equation*}
 V = \{s\in S: \|s-x\| = \epsilon
 \end{equation*}
 of \(x\in S\).
 There exist some \(M<\infty\) such that
\(\sum_{j=m+1}^\infty \delta_j^2 < \epsilon/2\)
 Now pick a base neighborhood
 \begin{equation*}
 U = \left\{v\in T:
     \forall 1\leq j\leq m,\, |v_j - x_j| < 2^{-(j+1)}\epsilon_j\right\}.
 \end{equation*}
 and clearly \(\Id(U) = U \subset V\).
\end{itemize}
Thus the identity is a homeomorphism. Since $T$ is compact so is $S$.


In particular, the Hilbert cube $Q$ is compact.

Conversely, assume that $S$ is compact.
If by negation \(\sum_1^\infty \delta_n^2 = \infty\).
then the family \(V_n \eqdef = \{x\in H: \|x\| < n\}\) of open sets
is a covering pf $S$ but has no finite sub-covering
since for each $n$ we can find some $m$ such that
\(\sum_{j=1}^m \delta_j^2 > n\) and then
\begin{equation*}
w \eqdef \sum_{j=1}^m \delta_j u_j \notin \bigcup_{j=1}^k V_n.
\end{equation*}
Therefore $G$ has no subcovering which is a contradiction.


%%%%%%%%%%%%%% 7
\begin{excopy}
Suppose \(\{a_n\}\) is a sequence of positive numbers
such that \(\sum a_n b_n < \infty \)
whenever \(b_n \geq 0\)
and
  \(\sum b_n^2 < \infty\).
Prove that
  \(\sum a_n^2 < \infty\).

\emph{Suggestion:} If \(sum a_n^2 = \infty\) then there are disjoint
sets \(E_k\) (\(k=1,2,3,\ldots\)) so that
\begin{equation*}
 \sum_{n\in E_k} a_n^2 > 1.
\end{equation*}
Define \(b_n\) so that \(b_n = c_k a_n\) for \(n\in E_k\). For suitably chosen
\(c_k\), \(\sum a_n b_n = \infty\) although \(\sum b_n^2 < \infty\).
\end{excopy}

Following the suggestion. Assume  \(\sum a_n^2 = \infty\)
and sets \(\{E_k\}_{k\in\N}\) such that \(\N = \disjunion E_k\) and
\(\sum_{n\in E_k} a_n^2 > 1\).
Define \(c_k\) so that
\begin{equation*}
\sum_{n\in E_k} (c_k a_n)^2  = c_k^2 \sum_{n\in E_k} a_n^2  = 1/k^2
\end{equation*}
Hence
\begin{equation*}
c_k \eqdef 1 \left/ \left( k \sqrt{\sum_{n\in E_k} a_n^2} \right)\right.
\end{equation*}
Now
\begin{equation*}
\sum_{n\in E_k} a_n b_n = c_k \sum_{n\in E_k} a_n^2
=  c_k \left. \sqrt{\sum_{n\in E_k} a_n^2} \right/ k
= 1/k
\end{equation*}
Hence
\(\sum_{n=1}^\infty b_n^2 < \infty\)
and
\(\sum_{n=1}^\infty a_n b_n = \infty\)
which contradicts the assumption on \(\{a_n\}\)
and so \(\sum a_n^2 < \infty\).


%%%%%%%%%%%%%%  8
\begin{excopy}
If \(H_1\) and \(H_2\) are two Hilbert spaces, prove thatone of them
is isomorphic to a subspace of the other. (Note that every closed subspace
of a Hilbert space is a Hilbert space.)
\end{excopy}

By the statemnt showm in section~4.19 of~\cite{RudinRCA87}, a Hilbert space
is determined by the cardinality of its orthonormal base.
By simply mapping the smaller orthonormal base of the two spaces
to the other we get an isomorphic embedding.

%%%%%%%%%%%%%% 9
\begin{excopy}
If \(A\subset[0,2\pi]\) and $A$ is measurable, prove that
\begin{equation*}
 \lim_{n\to\infty} \int_A \cos nx\,dx
 =
 \lim_{n\to\infty} \int_A \sin nx\,dx
 = 0.
\end{equation*}
\end{excopy}

By regularity of the Lebesgue measure, given \(\epsilon > 0\),
we can find a finite number of intervals \(\{I_k\}_{k=}^n\)
such that
\begin{eqnarray*}
 A &\subset& \bigcup_{k=1}^N I_k \\
 B &\eqdef& A\setminus \bigcup_{k=1}^N I_k \\
 m(B) &<& \epsilon/2.
\end{eqnarray*}
So it is sufficient to show that the above zero limits hold
for $A$ where $A$ is an interval.
The periods of \(\cos nx\) and \(\sin nx\) converge to zero
as \(n\to\infty\). Since  the intergal value
of a periodic function is determined by the ``residue'' interval
whose length is
\begin{equation*}
m(I) - \left\lfloor \frac{n m(I)}{2\pi} \right\rfloor \frac{2\pi}{n}
\leq 2\pi/n
\end{equation*}
and since the functions here are bounded, the limits are zero.

%%%%%%%%%%%%%%
\begin{excopy}
Let \(n_1 < n_2 < n_3 < \cdots\) be positive integers,
and let $E$ be the set of all \(x\in[0,2\pi]\) at which
\(\{\sin n_k x\}\) converges. Prove that \(m(E) = 0\).
\emph{Hint}: \(2\sin^2 \alpha = 1 - \cos 2\alpha\),
so \(\sin n_k x\to \pm 1/\sqrt{2}\;\aded\) on $E$,
by Exercise~9.
\end{excopy}

Assume by negation \(m(E) > 0\).
By previous exercise
\begin{equation*}
   \lim_{k\to\infty} \int_E \cos n_k x\,dm(x)
 = \lim_{k\to\infty} \int_E \sin n_k x\,dm(x) = 0
\end{equation*}
But Lebesgue's dominated convergence theorem
gives
\begin{eqnarray}
 \int_E \lim_{k\to\infty} \cos n_k x\,dm(x)  \label{eq:ex4:10}
      &=& \lim_{k\to\infty} \int_E \cos n_k x\,dm(x) = 0 \\
 \int_E \lim_{k\to\infty} \sin n_k x\,dm(x)  \notag
      &=& \lim_{k\to\infty} \int_E \sin n_k x\,dm(x) = 0
\end{eqnarray}

Let
\begin{eqnarray*}
E_+ &\eqdef& \{x\in E: \lim{k\to\infty} \cos n_k x > 0\} \\
E_0 &\eqdef& \{x\in E: \lim{k\to\infty} \cos n_k x = 0\} \\
E_- &\eqdef& \{x\in E: \lim{k\to\infty} \cos n_k x < 0\}
\end{eqnarray*}
Clearly \(m(E_+) > 0\) iff \(m(E_-) > 0\)
since otherwise \eqref{eq:ex4:10} would not hold.
But if \(m(E_+) > 0\) then we can apply again
the previous exercise, now to \(E_+\) which gives a contradicton
of
\begin{equation*}
\lim_{k\to\infty} \int_{E_+} \cos n_k x\,dm(x) > 0.
\end{equation*}
Thus \(m(E_0) = m(E) > 0\) and  \(\lim_{k\to\infty} \cos n_k x = 0\),
\aded\ on $E$.

The identity \(2\sin^2 \alpha = 1 - \cos 2\alpha\), now implies
\begin{equation*}
\lim_{k\to\infty} \sin n_k x = \pm\sqrt{2}/2
\end{equation*}
Appling previous exercise (again) on the two sets
\begin{equation*}
\{x\in E: \lim_{k\to\infty} \sin n_k x = -\sqrt{2}/2\}
\qquad
\{x\in E: \lim_{k\to\infty} \sin n_k x = \sqrt{2}/2\}
\end{equation*}
gives a contradiction with at least one of them.


%%%%%%%%%%%%%% 11
\begin{excopy}
Find a nonempty closed subset $E$ in \(L^2(T)\) that contains no element
of smallest norm.
\end{excopy}

We will construct \(\{f_n\}\) in \(L^2(T)\)  such that \(\|f_n\| = 1 + 1/n\).
We identify $T$ with \([0,2\pi)\).
We define (almost disjoint) sub-segments \(I_n = [a_{n-1},a_n]\)
of measure \(d_n = 2^{-n}\)
for \(n\geq 1\), by letting \(a_0 = 0\), and \(a_n = a_{n-1} + d_n\).
We define
\begin{equation*}
 f_n(t) = c_n\chhi_{I_n}(t).
\end{equation*}
We set
\begin{equation*}
c_n = (n+1) \bigm/ \left(n \|\chhi_{I_n}\|\right)
\end{equation*}
which ensures that \(\|f_n\| = 1 + 1/n\).
Clearly \(\|f_m - f_n\| > 1\) whenever \(m\neq n\).
The set \(F = \{f_n\}\) has no accumulation points and so $F$ is closed.


%%%%%%%%%%%%%% 11 2nd edition
\item[11b]
\begin{minipage}[t]{.8\textwidth}\footnotesize
[\textbf{Note:} This exercise was removed in the 3rd edition]

Prove that the identify
\begin{equation*}
4\langle x,y \rangle
=
    \|x+y\|^2
 -  \|x-y\|^2
 +  i\|x+iy\|^2
 -  i\|x-iy\|^2
\end{equation*}
is valid every inner product space, and show that it proves the
implication \ich{c}~\(\to\)~\ich{d} of Theorem~4.18.
\smallskip\hrule
\end{minipage}

Compute
\begin{eqnarray*}
 4\langle x,y \rangle
&=&
      2 \langle x,y \rangle
    + 2 \overline{\langle x,y \rangle}
    + 2 \langle x,y \rangle
    - 2 \overline{\langle x,y \rangle}  \\
&=&
      2 \langle x,y \rangle
    + 2 \overline{\langle x,y \rangle}
    + 2i\overline{i} \langle x,y \rangle
    + 2i^2\overline{\langle x,y \rangle}  \\
&=&
    \bigl(2 \langle x,y \rangle + 2 \langle y,x \rangle \bigr)
    + i \bigl(2 \langle x,iy \rangle + 2 \langle iy,x \rangle \bigr) \\
&=&
   \bigl(
      \langle x,x \rangle
    + \langle x,y \rangle
    + \langle y,x \rangle
    + \langle y,y \rangle \bigr)
 \\ &&
  -
   \bigl(
      \langle x,x \rangle
    - \langle x,y \rangle
    - \langle y,x \rangle
    + \langle y,y \rangle \bigr)
 \\
&&
  +
   i\bigl(
      \langle x,x \rangle
    + \langle x,iy \rangle
    + \langle iy,x \rangle
    + \langle iy,iy \rangle \bigr)
 \\ &&
  -
   i\bigl(
      \langle x,x \rangle
    - \langle x,iy \rangle
    - \langle iy,x \rangle
    + \langle iy,iy \rangle \bigr)
 \\
&=&
    \|x+y\|^2
 -  \|x-y\|^2
 +  i\|x+iy\|^2
 -  i\|x-iy\|^2
\end{eqnarray*}



%%%%%%%%%%%%%% 12
\begin{excopy}
The constants \(c_k\) in section~4.24 were shown to be such that
\(k^{-1}c_k\) is bounded.
Estimate the relevant integral more precisely and show that
\begin{equation*}
 0 < \lim_{k\to\infty} k^{-1/2} c_k < \infty.
\end{equation*}
\end{excopy}

Section~4.24 defines (see also \ref{sec:comp:trig}) \(c_k\) such that
\begin{eqnarray*}
Q_k(t) &\eqdef& c_k \left(\frac{1+\cos t}{2}\right)^k \\
\frac{1}{2\pi} \int_{-\pi}^\pi Q_k(t)\,dt &=& 1
\end{eqnarray*}

Thus
\begin{equation*}
c_k
= 2\pi \left/ \int_{-\pi}^\pi \left(\frac{1+\cos t}{2}\right)^k \,dt \right.
= 2^{k+1}\pi \left/ \int_{-\pi}^\pi (1+\cos t)^k \,dt \right..
\end{equation*}
Note that \(((1+\cos t)/2)^k\) is decreasing with $k$ and so a limit exists.

%%%%%%%%%%%%%% 13
\begin{excopy}
Suppose $f$ is a continuous function on \(\R^1\), with period $1$. Prove that
\begin{equation*}
 \lim_{N\to\infty} \frac{1}{N} \sum_{n=1}^N f(n\alpha) = \int_0^1 f(t)\,dt
\end{equation*}
for every irrational number \(\alpha\). \emph{Hint}: Do it first for
\begin{equation*}
 f(t) = \exp(2\pi ikt), \qquad k=0,\pm 1,\pm 2,\ldots
\end{equation*}
\end{excopy}

We use the \emph{binary modulus} notation for real numbers:
\begin{equation} \label{eq:4.13.bmod}
x \bmod 1 \eqdef x - \lfloor x\rfloor.
\end{equation}

Because of being periodic,
\(f(x) = f(x \bmod 1)\) for all \(x\in\R\).
Thus looking at $f$-evaluations of
\(\{n\alpha\}_{n=1}^N\) is equivalent
to looking at $f$-evaluations of
\begin{equation*}
A_N = \{n \alpha \bmod 1\}_{n=1}^N \subset (0,1)
\end{equation*}
Let \(X_N = \{x_n\}_{n=1}^N\) be an increasing re-ordering of \(A_N\).
Since \(\alpha\) is irrational, the sequence \(X_N\)
is strictly increasing. By defining
\begin{eqnarray*}
a_0 &=& 0 \\
a_k &=& (x_{k} + x_{k+1}) / 2 \qquad \textrm{for}\, 1\leq k < N \\
a_N &=& 1 \\
\end{eqnarray*}
We denote
\begin{eqnarray*}
\delta_N &\eqdef& \min_{1\leq n < N} a_{n+1} - a_{n} \\
\Delta_N &\eqdef& \max_{1\leq n < N} a_{n+1} - a_{n}.
\end{eqnarray*}

We have a partition of \([0,1]\).
Since $f$ is continuous, it is uniformly continuous on \([0,1]\).
By Theorem~6.7 of \cite{RudinPMA85}
the expression
\begin{equation*}
\lim_{N\to\infty} \frac{1}{N} \sum_{n=1}^N f(n\alpha)
= \lim_{N\to\infty} \frac{1}{N} \sum_{n=1}^N f(x_n)
\end{equation*}
\index{Riemann-Stieltjes}
becomes the Riemann-Stieltjes integral of $f$ over \([0,1]\)
provided we show that
\begin{equation} \label{eq:ex4.13:Delto:to0}
\lim_{N\to\infty} \Delta_N = 0.
\end{equation}


Clearly
\(\delta_N \leq 1/N\).
Thus there exists some integers \(j,k\in\N\) such that
\begin{equation*}
0 < d \eqdef (j\alpha \bmod 1) - (k\alpha \bmod 1) < 1/N.
\end{equation*}
Let \(M = \lceil 1/d \rceil \cdot \max(j,k\).
Looking at \(X_M\), we can see that it contains
all numbers of the form
\begin{equation*}
\{ md: 1\leq m \leq M\}
=
\{ (mj\alpha \bmod 1) - (mk\alpha \bmod 1): 1\leq m \leq M\}.
\end{equation*}
Hence \(\Delta_m \leq 1/N\) for all \(m\geq M\)
and so \eqref{eq:ex4.13:Delto:to0} is proved.


%%%%%%%%%%%%%% 14
\begin{excopy}
Compute
\begin{equation*}
 \min_{a,b,c} \int_{-1}^1 |x^3 - a - bx -cx^2|^2 \,dx
\end{equation*}
and find
\begin{equation*}
 \max \int_{-1}^1 x^3 g(x)\,dx,
\end{equation*}
where $g$ is subject to the restrictions
\begin{equation*}
  \int_{-1}^1 g(x)\,dx
  = \int_{-1}^1 x g(x)\,dx
  = \int_{-1}^1 x^2 g(x)\,dx
  = 0 ;
  \qquad
 \int_{-1}^1 |g(x)|^2\,dx=1.
\end{equation*}
\end{excopy}

Let's ortho-normalize the base of the $3$-dimensional subspace $S$
spanned by \(\{x^0, x^1, x^2\}\) in \(C[-1,1]\).
\begin{eqnarray*}
f_0(x) &=& \sqrt{2}/2 \\
f_1(x) &=& \sqrt{3/2}\cdot x \\
f_2(x) &=& x^2 - 1/3
\end{eqnarray*}
Note that
\begin{eqnarray*}
\langle f_2, f_0 \rangle
&=& \int_{-1}^1 (t^2 - 1/3)\cdot \sqrt{2}/2\,dt
   = (\sqrt{2}/2)\cdot \left(\left(\int_{-1}^1 t^2\,dt\right) - 2/3\right) \\
&=& (\sqrt{2}/2)\cdot \left(\left( (t^3/3)\bigm|_{-1}^1 \right) - 2/3\right)
= (\sqrt{2}/2)\cdot ( 2/3 - 2/3)
= 0
\end{eqnarray*}

The minimization we are looking for, is actually the \(L^2\) distance
between \(\tau(x) = x^3\) and~$S$.
We will project \(\tau\) on \(\{f_0,f_1,f_2\}\)
\begin{eqnarray*}
\langle \tau, f_0\rangle
  &=& \int_{-1}^1 t^3\cdot \sqrt{2}/2\,dt
      = (\sqrt{2}/2)\left( (1/4)t^4 \bigm|_{-1}^1 \right)
      = (\sqrt{2}/2)\cdot(1/4)\cdot 2
      = \sqrt{2}/4 \\
\langle \tau, f_1\rangle
 &=& \int_{-1}^1 t^3\cdot \sqrt{3/2}t,dt
      = \sqrt{3/2} \left( (1/5)t^5 \bigm|_{-1}^1 \right)
      =  \sqrt{2\cdot3}/5 \\
\langle \tau, f_2\rangle
 &=&  \int_{-1}^1 t^3\cdot (t^2 - 1/3)\,dt
       =  (t^6/6 - t^4/12)\bigm|_{-1}^1
       =  0
\end{eqnarray*}

\iffalse
\begin{eqnarray*}
\langle \tau, f_0\rangle &=&
  \int_{-1}^1 t^3\cdot \sqrt{2}/2\,dt \\
  &=& (\sqrt{2}/2)\left( (1/4)t^4 \bigm|_{-1}^1 \right) \\
  &=& (\sqrt{2}/2)\cdot(1/4)\cdot 2  \\
  &=& \sqrt{2}/4 \\
\langle \tau, f_1\rangle &=&
  \int_{-1}^1 t^3\cdot \sqrt{3/2}t,dt \\
 &=&  \sqrt{3/2} \left( (1/5)t^5 \bigm|_{-1}^1 \right) \\
 &=&  \sqrt{2\cdot3}/5 \\
\langle \tau, f_2\rangle
 &=&  \int_{-1}^1 t^3\cdot (t^2 - 1/3)\,dt \\
 &=&  (t^6/6 - t^4/12)\bigm|_{-1}^1 \\
 &=&  0
\end{eqnarray*}
\fi

Thus the projection is
\begin{equation*}
\tau'(x) \eqdef \sqrt{2}/4 f_0 + \sqrt{2\cdot3}/5 f_1
 = 3 x + 1/4.
\end{equation*}
The square distance is
\begin{equation*}
d^2(\tau,\tau')
 = \int_{-1}^1 t^3 - 3t - 1/4\,dt
 = (t^4/4 - 3t^2/2 - t) \bigm|_{-1}^1
 = 2
\end{equation*}
and so \(d(\tau,\tau') = \sqrt{2}\).

In order to find the maximal distance with $g$
we can simply apply exercise~\ref{ex:4.16}.
The condition on $g$
are exactly as required
there, namely \(g\in S^\perp\) and \(\|g\|=1\).
Thus the maximal distance is \(\sqrt{2}\).

\iffalse
We put
\begin{equation*}
\tilde{g}(x) \eqdef \tau(x) - \tau'(x) = x^3 - 3 x - 1/4
\end{equation*}
then \(\tilde{g} \in S^\perp\).
To normalize, we compute the norm
\begin{eqnarray*}
\|\tilde{g}\|^2
 &=& \int_{-1}^1 (t^3 - 3 t - 1/4)^2\,dt \\
 &=& \int_{-1}^1
      t^6   - 6t^4   - 2t^3  + 9 t^2 + 3t/2   + 1/16 \,dt \\
 &=& (t^7/7 - 6t^5/5 - t^4/2 + 3 t^3 + 3t^2/4 + t/16) \bigm|{-1}^1 \\
 &=& (t^4/2 + 3t^2/4) \bigm|{-1}^1 \\
 &=& 2(1 + 3/4) \\
 &=& 7/2 \\
\end{eqnarray*}
Finally we define $g$
\begin{eqnarray*}
g(x)
  \eqdef \tilde{g}(x)/\|\tilde{g}\|
  = \sqrt{7/2}t^3 - 3\sqrt{7/2}t - \sqrt{2\cdot 7}/8
\end{eqnarray*}
\fi


%%%%%%%%%%%%%% 15
\begin{excopy}
Compute
\begin{equation*}
 \min_{a,b,c} \int_0^\infty |x^3 - a - bx -cx^2|^2e^{-x} \,dx
\end{equation*}
State and solve the corresponding maximum problem,
as in Exercise~14.
\end{excopy}

With a little help from \texttt{http://integrals.wolfram.com}.

Let's ortho-normalize the base of the $3$-dimensional subspace $S$
spanned by \(\{x^0, x^1, x^2\}\) in \(C[0,\infty)\)
where the inner product is defined as
\begin{equation*}
\langle f,g \rangle \eqdef \int_0^\infty f(t)\overline{g(t)}e^{-t}\,dt
\end{equation*}
\index{Graham-Schmidt}
We use the Graham-Schmidt procedure.

\paragraph{Ortho-Normalize \(\phi_0(x)=x^0\)}.
\begin{equation*}
f_0(x) = 1
\end{equation*}

\paragraph{Ortho-Normalize \(\phi_1(x)=x^1\)}.
\begin{equation*}
\langle \phi_1, f_0 \rangle
 = \int_0^\infty t e^{-t}\,dt
 = \left(e{-t}(-1-t)\right)\bigm|_0^\infty
 = 0 - (-1) = 1
\end{equation*}
For
\begin{equation} \label{eq:ex4.15:psi1}
\psi_1 = \phi_1 - 1\cdot f_0
\end{equation}
we compute the norm
\begin{equation*}
\|\psi_1\|_2^2
 = \int_0^\infty \bigl(\psi_1(t)\bigr)^2 e^{-t}\,dt
 = \int_0^\infty (t - 1)^2 e^{-t}\,dt
 =  \left(e{-t}(-1-t^2)\right)\bigm|_0^\infty
 = 0 - 1\cdot (-1) = 1.
\end{equation*}
Hence we use \(\psi_1\) \eqref{eq:ex4.15:psi1} for
\begin{equation*}
f_1(x) = x - 1
\end{equation*}

\paragraph{Ortho-Normalize \(\phi_2(x)=x^2\)}.
\begin{equation*}
\langle \phi_2, f_0 \rangle
 = \int_0^\infty t^2 e^{-t}\,dt
 = \left((-t^2-2t-2)e{-t}\right)\bigm|_0^\infty
 = 2
\end{equation*}

\begin{equation*}
\langle \phi_2, f_1 \rangle
 = \int_0^\infty t^2(t-1) e^{-t}\,dt
 = \left((x^3-4x^2+8x-8) e^{-t}\right)\bigm|_0^\infty
 = 8
\end{equation*}

For
\begin{equation} \label{eq:ex4.15:psi2}
\psi_2 = \phi_2 - 2\cdot f_0 -8f_1.
\end{equation}
we compute the norm
\begin{eqnarray*}
\|\psi_2\|_2^2
 &=& \int_0^\infty \bigl(\psi_2(t)\bigr)^2 e^{-t}\,dt  \\
 &=& \int_0^\infty (t^2 - 2 - 8(t-1))^2 e^{-t}\,dt
      = \left((-t^4+12t^3-40t^2+16t-20)e{-t}\right)\bigm|_0^\infty \\
 &=& 20
\end{eqnarray*}

Hence we use \(\psi_2\) \eqref{eq:ex4.15:psi2} to get
\begin{eqnarray*}
f_2(x)
 &=& \psi_2(x)/\|\psi_2\| \\
 &=& (\sqrt{5}/10)(x^2 -8(x-1) -2) \\
 &=& (\sqrt{5}/10)(x^2 -8x + 6) \\
 &=& (\sqrt{5}/10)\cdot x^2 (-4\sqrt{5}/5)\cdot x^2 + 3\sqrt{5}/5
\end{eqnarray*}


The minimization we are looking for, is actually the \(L^2\) distance
between \(\tau(x) = x^3\) and~$S$.
We will project \(\tau\) on \(\{f_0,f_1,f_2\}\)
\begin{eqnarray*}
\langle \tau, f_0\rangle &=&
 \int_0^\infty t^3 e^{-t} \,dt
    = ((-t^3 -3t^2 -6t -6)e^{-t})\bigm|_0^\infty  \\
 &=& 6 \\
\langle \tau, f_1\rangle
&=& \int_0^\infty t^3(t-1)e^{-t} \,dt
    = ((-t^4 -3t^3 -9t^2 -18t -18)e^{-t})\bigm|_0^\infty  \\
 &=& 18 \\
\langle \tau, f_2\rangle
 &=&  \int_0^\infty t^3\,(\sqrt{5}/10)(t^2 - 8t + 6)\, e^{-t} \,dt \\
 &=&  (\sqrt{5}/10)
      \bigl((-t^5 + 3t^4 + 6t^3 +18t^2 +36t +36)e^{-t}\bigr)\biggm|_0^\infty \\
 &=&  \sqrt{5}\cdot 18/5
\end{eqnarray*}

Thus the projection is
\begin{equation*}
\tau'(x) \eqdef 6 f_0 + 18 f_1 + (\sqrt{5}\cdot 18/5) f_2
\end{equation*}
Let \(\tau = \tau' + \tau''\) directo decomposition.
The square distance is
\begin{eqnarray*}
d^2(\tau,\tau')
&=& \|\tau''\|^2 \\
&=& \|\tau\|^2 - \|tau'\|^2 \\
&=& \|\tau\|^2 - \sum_{j=0}^2 \langle \tau, f_j \rangle^2 \\
&=& \int_0^\infty t^6 e^{-t} \,dt
    - \bigl(6^2 + 18^2 +  (\sqrt{5}\cdot 18/5)^2\bigr) \\
&=& \left((-t^6 -6t^5 - 30t^4 - 120 t^3
          - 360t^2 -720t - 720)e^{-t}\right) \bigm|_0^\infty  \\
& & - (36 + 324 + 324/5) \\
&=& 720 - 360 - 324/5 \\
&=&  1476/5
\end{eqnarray*}



Again we apply exercise~\ref{ex:4.16}, to get
\begin{equation*}
\max \{\langle g, \tau\rangle : g\in S^\perp\;\wedge\; \|g\|=1\}
= d(\tau,\tau')
\end{equation*}



%%%%%%%%%%%%%% 16
\begin{excopy}
If \label{ex:4.16}
\(x_0\in H\) and $M$ is a closed linear subspace  of $H$, prove that
\begin{equation*}
 \min \bigl\{\|x - x_0\|: x\in M\bigr\}
 = \max \left\{|\langle x_0,y\rangle|: y\in M^\perp,\, \|y\|=1\right\}.
\end{equation*}
\end{excopy}

Using projections,
let \(x_0 = \mu + \nu\) be the decomposition,
such that \(\mu\in M\) and \(\nu \in M^\perp\).
By Theorem~4.11\ich{a} (\cite{RudinRCA87}),
\begin{equation*}
d = d(x_0,M) = \min \bigl\{\|x - x_0\|: x\in M\bigr\} = d(x_0,\mu) = \|\nu\|.
\end{equation*}
By Theorem~4.11\ich{d} (\cite{RudinRCA87}),
\(\|x_0\|^2 = \mu^2 + \nu^2\).

Clearly \(\nu = 0\) iff \(x_0 \in M\). In this case \(d=0\)
and for every \(y\in M^\perp\) we have \(\langle x_0, y\rangle = 0\)
and the desired equality holds.

Otherwise, \(\nu \neq 0\).
We pick \(y_1 = \nu/\|\nu\|\) and so
\begin{equation*}
\langle x_0, y_1\rangle
= \langle \mu + \nu, \nu\rangle / \| \nu \|
= (\langle \mu , \nu\rangle +  \langle \nu, \nu\rangle) / \| \nu \|
= 0 + \|\nu\|^2 / \|\nu\|
= \|\nu\|.
\end{equation*}
Hence
\begin{equation} \label{eq:ex4.15:dleq}
d \leq  \max \left\{|\langle x_0,y\rangle|: y\in M^\perp,\, \|y\|=1\right\}.
\end{equation}
Assume \(y\in M^\perp\) such that \(\|y\|=1\).
% and \(\langle x_0,y\rangle > d\).
Let $N$ be the $1$-dimensional subspace spanned by \(\nu\),
and denote its orthogonal subspace in \(M^\perp\) by \(N^\perp\).
Let \(y = v + w\) be the direct decomposition,
such that \(v\in N\) and \(w\in N^\perp \subset M^\perp\).
Thus there exists a scalar \(a\in\C\) such that \(v = ay_1\)
and since
\begin{equation*}
\|v\| + \|w\| = \|y\| = \|y_1\| = 1
\end{equation*}
we have \(|a| \leq 1\).
Now
\begin{eqnarray*}
\langle y, x_0 \rangle
&=& \langle v+w, \mu + \nu \rangle \\
&=&   \langle v, \mu \rangle
    + \langle w, \mu \rangle
    + \langle v, \nu \rangle
    + \langle w, \nu \rangle \\
&=& 0 + 0 + \langle v, \nu \rangle + 0 \\
&=& \langle a y_1, \|\nu\| y_1 \rangle \\
&=& a\|\nu\|
\end{eqnarray*}
Hence \(|\langle y, x_0 \rangle| \leq \|\nu\|\), which shows
the desired reversed inequality of \eqref{eq:ex4.15:dleq}.

\iffalse
Since
\begin{equation*}
  \langle x_0, \nu \rangle
= \langle x_0, \nu \rangle
= \langle x_0, x_0 - \mu \rangle
= \|x_0\|^2 - \langle x_0, \mu \rangle
\end{equation*}
\fi

%%%%%%%%%%%%%% 17
\begin{excopy}
Show that there is a continuous one-to-one mapping \(\gamma\) of \([0,1]\)
into $H$ such that
\(\gamma(b) - \gamma(a)\) is orthogonal to
\(\gamma(d) - \gamma(c)\) whenever
\(0\leq a \leq b \leq c \leq d \leq 1\).
(\(\gamma\) may be called a ``curve with orthogonal increments.'')
\emph{Hint}: Take \(H=L^2\), and constants characteristic functions
 of certain subsets of \([0,1]\).
\end{excopy}

Simply define \(\gamma(t) = {\chhi}_{[0,t]}\).


%%%%%%%%%%%%%% 18 2nd edition
\item[18b]
\begin{minipage}[t]{.8\textwidth}\footnotesize
[\textbf{Note:} This exercise was removed in the 3rd edition]

Give a direct proof of Theorem~4.16, i.e., one which does not
depend on the more general consideration of Sec.~4.15.
\smallskip\hrule
\end{minipage}


%%%%%%%%%%%%%% 18
\begin{excopy}
Define \(u_s(t) = e^{ist}\) for all \(s\in \R^1\).
Let $X$ be the complex vector space of all finite linear combinations
of these functions \(u_s\).
If \(f\in X\) and \(g\in X\), show that
\begin{equation*}
\langle f,g\rangle
= \lim_{A\to\infty} \frac{1}{2A} \int_{-A}^A f(t)\overline{g(t)}\,dt
\end{equation*}
exists. Show that this inner product makes $X$ into a unitary spce
whose completion is a non separable Hint space $H$.
Show that \(\{u_s: s\in\R^1\}\) is a maximal orthonormal set in $H$.
\end{excopy}

% Using the \eqref{eq:4.13.bmod} notation,
For \(b>0\) we use the real modulus notation:
\begin{equation*}
 a \bmod b \eqdef a - \lfloor \frac{a}{b} \rfloor b.
\end{equation*}
It is clear that
\begin{equation*}
\int_{-A}^A u_s(t)\,dt = \int_{-(A \bmod 2\pi/s) }^{A \bmod 2\pi/s} u_s(t)\,dt
\end{equation*}
If \(m, n \in \Z\) then
\begin{equation*}
\langle u_m,u_n\rangle
= \lim_{A\to\infty} \frac{1}{2A} \int_{-A}^A u_m(t)\overline{u_n(t)}\,dt
= \lim_{A\to\infty} \frac{1}{2A} \int_{-A}^A e^{i(m-n)t}\,dt.
\end{equation*}
Hence, if \(m=n\) then
\begin{equation*}
\langle u_m,u_n\rangle
= \lim_{A\to\infty} \frac{1}{2A} \int_{-A}^A e^{i\cdot 0\cdot t}\,dt.
= \lim_{A\to\infty} \frac{1}{2A} \int_{-A}^A 1\,dt
= 1.
\end{equation*}
Otherwise, \(m\neq n\). Put \(d = m - n \neq 0\) and so
\begin{eqnarray*}
|\langle u_m,u_n\rangle|
&=& \left|\lim_{A\to\infty} \frac{1}{2A} \int_{-A}^A u_d(t)\,dt\right| \\
&=&\left|\lim_{A\to\infty}
   \frac{1}{2A} \int_{-(A \bmod 2\pi/d)}^{A \bmod 2\pi/d} u_d(t)\,dt\right| \\
&=& \lim_{A\to\infty} \frac{1}{2A} %
   \left|\int_{-(A \bmod 2\pi/d)}^{A \bmod 2\pi/d} u_d(t)\,dt\right| \\
&\leq&  \lim_{A\to\infty} \frac{1}{2A} \cdot 4\pi/d \\
&=& 0
\end{eqnarray*}
Hence \(\{u_s\}_{s\in\Z}\) is an orthonormal set.




%%%%%%%%%%%%%% 19
\begin{excopy}
Fix a positive integer $N$,
put \(\omega = e^{2\pi i/N}\), prove the orthogonality relations
\begin{equation*}
 \frac{1}{N} \sum_{n=1}^N \omega^{nk} =
 \left\{\begin{array}{ll}
         1 \qquad \textrm{if}\quad k=0\\
         0 \qquad \textrm{if}\quad 1\leq k \leq N - 1
        \end{array}\right.
\end{equation*}
and use them to derive the identities
\begin{equation*}
\langle x, y \rangle = \frac{1}{N} \sum_{n=1}^N \|x + \omega^n y\|^2 \omega^n
\end{equation*}
that hold in every inner product space if \(N\geq 3\). Show also that
\begin{equation*}
\langle x, y \rangle
 = \frac{1}{2\pi} \int_{-\pi}^\pi \|x + e^{i\theta} y\|^2 e^{i\theta}\,d\theta.
\end{equation*}
\end{excopy}

If \(k=0\) then
\begin{equation*}
  \frac{1}{N} \sum_{n=1}^N \omega^{nk}
= \frac{1}{N} \sum_{n=1}^N \omega^{n0}
= \frac{1}{N} N\cdot 1
= 1.
\end{equation*}
Otherwise \( 1\leq k < N\). Put
\begin{equation*}
S_k = \sum_{n=1}^N \omega^{nk}
\end{equation*}
Since \(\Omega^{Nk} = (\Omega^N)^k = 1^k = 1\) we have
\begin{equation*}
\omega^k S_k
= \sum_{n=1}^N \omega^{nk + k}
= \left(\sum_{n=2}^N \omega^{nk}\right) +  \omega^{Nk + k}
= \left(\sum_{n=1}^N \omega^{nk}\right)
= S_k.
\end{equation*}
If by negation, \(S_k\neq 0\) then
 \(\omega^k= 1\) which is a contradiction. Hence
\begin{equation*}
\frac{1}{N}S_k = \frac{1}{N}0 = 0.
\end{equation*}

Assume now that \(N\geq 3\). Compute:
\begin{eqnarray*}
S
&\eqdef&
 \sum_{n=1}^N \|x + \omega^n y\|^2 \omega^n \\
&=&
 \sum_{n=1}^N \langle x + \omega^n y, x + \omega^n y\rangle \omega^n  \\
&=&
    \|x\|^2 \sum_{n=1}^N \omega^n
 +  \sum_{n=1}^N \left\langle \omega^n x,  \omega^{n}y \right\rangle
 +  \sum_{n=1}^N \left\langle \omega^{2n} y, x \right\rangle
 + \|y\|^2 \sum_{n=1}^N  \omega^n \overline{\omega^n} \omega^n \\
&=&
    \|x\|^2 \cdot 0
    + \sum_{n=1}^N \left\langle x, \overline{\omega^n} \omega^{n}y \right\rangle
    + \langle y, x \rangle \sum_{n=1}^N \omega^{2n}
    + \|y\|^2 \sum_{n=1}^N \omega^n \\
&=&
    0 +
    \sum_{n=1}^N \left\langle x, \overline{\omega^n} \omega^{n}y \right\rangle
    + 0 + 0 \\
&=& N \langle x, 1\cdot y \rangle.
\end{eqnarray*}
Hence
\begin{equation*}
\langle x, y \rangle = S/N.
\end{equation*}

Now for the analog integral equality.
\begin{eqnarray*}
I
&\eqdef&
 \int_{-\pi}^\pi \|x + e^{i\theta} y\|^2 e^{i\theta}\,d\theta \\
&=&
 \int_{-\pi}^\pi
    \langle x + e^{i\theta} y, x + e^{i\theta} y \rangle e^{i\theta}\,d\theta \\
&=&
    \|x\|^2 \int_{-\pi}^\pi e^{i\theta}\,d\theta
 +  \int_{-\pi}^\pi
         \left\langle e^{i\theta} x,  \omega^{n} y \right\rangle\,d\theta
 +  \int_{-\pi}^\pi
         \left\langle e^{2i\theta} y, x \right\rangle\,d\theta
 + \|y\|^2 \int_{-\pi}^\pi
         e^{i\theta} \overline{e^{i\theta}} e^{i\theta}\,d\theta \\
&=&
     \|x\|^2 \cdot 0
   + \int_{-\pi}^\pi
      \left\langle x, \overline{e^{i\theta}} e^{i\theta}y
     \right\rangle \,d\theta
    + \langle y, x \rangle \int_{-\pi}^\pi e^{2i\theta}\,d\theta
    + \|y\|^2 \int_{-\pi}^\pi e^{i\theta}\,d\theta \\
&=&
    0 +
    \int_{-\pi}^\pi
      \left\langle x, \overline{e^{i\theta}} e^{i\theta}y \right\rangle
    + \langle x, 0\cdot y\rangle  + 0 \\
&=& 2\pi \langle x, 1\cdot y \rangle.
\end{eqnarray*}
Hence
\begin{equation*}
\langle x, y \rangle = \frac{1}{2\pi}I.
\end{equation*}

\end{enumerate}

 % -*- latex -*-
% $Id: rudinrca5.tex,v 1.8 2006/04/30 19:06:58 yotam Exp $


%%%%%%%%%%%%%%%%%%%%%%%%%%%%%%%%%%%%%%%%%%%%%%%%%%%%%%%%%%%%%%%%%%%%%%%%
%%%%%%%%%%%%%%%%%%%%%%%%%%%%%%%%%%%%%%%%%%%%%%%%%%%%%%%%%%%%%%%%%%%%%%%%
%%%%%%%%%%%%%%%%%%%%%%%%%%%%%%%%%%%%%%%%%%%%%%%%%%%%%%%%%%%%%%%%%%%%%%%%
\chapterTypeout{Example of Banach Space Techniques} % 5

%%%%%%%%%%%%%%%%%%%%%%%%%%%%%%%%%%%%%%%%%%%%%%%%%%%%%%%%%%%%%%%%%%%%%%%%
%%%%%%%%%%%%%%%%%%%%%%%%%%%%%%%%%%%%%%%%%%%%%%%%%%%%%%%%%%%%%%%%%%%%%%%%
\section{Notes}

%%%%%%%%%%%%%%%%%%%%%%%%%%%%%%%%%%%%%%%%%%%%%%%%%%%%%%%%%%%%%%%%%%%%%%%%
\subsection{Baire's Category Theorem}

Other applications of Baire's theorem are the following

\begin{llem} \label{lem:count:1cat}
If $A$ is a countable subset of a complete metric space $X$,
then $A$ is of first category.
\end{llem}

\begin{thmproof}
As a set --- every point is a nowhere dense. 
Since  $A$ is a countable union of its elements, it is of first category.
\end{thmproof}

\begin{llem} \label{lem:gdel:2cat}
If $D$ is a dense \(G_\delta\) subset of a complete metric space $X$,
then $D$ is of second category.
\end{llem}

\begin{thmproof}
Let \(D=\cup_{n\in\N}G_n\) where \(G_n\) are open. By assumption
all \(G_n\) are dense. Put \(F_n = X \setminus G_n\), now each \(F_n\) is
closed and nowehere dense, since otherwise \(G_n\) would not be dense.
Thus \(F = \cup_{n\in\N}F_n\) is of first category, 
therefore, since $X$ is of second category, 
so is \(D = X \setminus F\).
\end{thmproof}



%%%%%%%%%%%%%%%%%%%%%%%%%%%%%%%%%%%%%%%%%%%%%%%%%%%%%%%%%%%%%%%%%%%%%%%%
\subsection{Working out Dirichlet's kernel}

\index{Dirichlet's kernel}

Let's verify the equality of \textbf{5.11}(9) of \cite{RudinRCA80}.
First note that
% \begin{equation*}
\(e^{i\theta} = \cos\theta + i\sin\theta\)
% \end{equation*}
and so
\begin{equation*}
e^{i\theta} - e^{-i\theta}
=     \bigl(\cos(\theta) - \cos(-\theta)\bigr)
  +  i\bigl(\sin(\theta) - \sin(-\theta)\bigr)
= 2i\sin\theta.
\end{equation*}
We use the above equality twice, in the following computation
\begin{eqnarray*}
D_n(t)
 &\eqdef& \sum_{k=-n}^n e^{ikt} \\
 &=&      \frac{e^{it/2} - e^{-it/2}}{e^{it/2} - e^{-it/2}}
           \sum_{k=-n}^n e^{ikt} \\
 &=& \left(\sum_{k=-n}^n e^{ikt}(e^{it/2} - e^{-it/2})\right)
     \biggm/
     \bigl(2i\sin(t/2)\bigr)\\
 &=& \bigl(e^{i(n+1/2)t} - e^{-i(n+1/2)t}\bigr) \bigm/
     \bigl(2i\sin(t/2)\bigr) \\
 &=& 2i\sin\bigl((n+1/2)t\bigr) \bigm/
     \bigl(2i\sin(t/2)\bigr) \\
 &=& \sin\bigl((n+1/2)t\bigr) / \sin(t/2). \\
\end{eqnarray*}

%%%%%%%%%%%%%%%%%%%%%%%%%%%%%%%%%%%%%%%%%%%%%%%%%%%%%%%%%%%%%%%%%%%%%%%%
\subsection{Parseval's Identity}
\index{Parseval}

Let's derive an equality in section~5.24. Given 
\begin{equation*}
f(z) = \sum_{n=0}^N b_n(z-z_0)^n
\end{equation*}
we have
\begin{align*}
 \frac{1}{2\pi}
  \int_\pi^\pi \left|f(z_0 + re^{i\theta}\right|^2\,d\theta
&= \frac{1}{2\pi} \int_\pi^\pi 
   \left(\sum_{n=0}^N b_n r^n e^{ni\theta}\right) 
   \cdot
   \overline{\left(\sum_{n=0}^N b_n r^n e^{ni\theta}\right)}
   \,d\theta \\
&= \frac{1}{2\pi} \int_\pi^\pi 
   \left(\sum_{n=0}^N b_n r^n e^{ni\theta}\right) 
   \cdot
   \left(\sum_{n=0}^N \overline{b_n} r^n e^{-ni\theta}\right)
   \,d\theta \\
&=  \frac{1}{2\pi}  \sum_{m,n=0}^N  \int_\pi^\pi 
   b_m \overline{b_n} r^{m+n} e^{(m-n)i\theta}
   \,d\theta \\
&=  \frac{1}{2\pi}  \sum_{m,n=0}^N  
    b_m \overline{b_n} r^{m+n} 
   \int_\pi^\pi e^{(m-n)i\theta}
   \,d\theta \\
&=  \sum_{n=0}^N |b_n|^2 r^{2n} 
\end{align*}


%%%%%%%%%%%%%%%%%%%%%%%%%%%%%%%%%%%%%%%%%%%%%%%%%%%%%%%%%%%%%%%%%%%%%%%%
\subsection{Working out Poisson's kernel}

\index{Poisson's kernel}

Let's verify the equality of \textbf{5.24}(8) of \cite{RudinRCA80}.
\begin{eqnarray*}
1+2\sum_1^\infty \left(ze^{-it}\right)^n
&=& 1 + 2\left(-1 + \sum_0^\infty \left(ze^{-it}\right)^n\right) \\
&=& -1 + 2/\left(1 - ze^{it}\right) \\
&=& \frac{ze^{-it} - 1 + 2}{1 - ze^{-it}} \\
&=& \frac{e^{it} + z}{e^{it} - z} \\
&=& \frac{1 + ze^{-it}}{1 - ze^{-it}} \\
&=& \frac{ (1 + ze^{-it})\overline{(1 - ze^{-it}) } }{
           (1 - ze^{-it})\overline{(1 - ze^{-it}) } } \\
&=& \frac{ 1 + ze^{-it} - \overline{ze^{-it}} - ze^{-it}\overline{ze^{-it}} }{
           |1 - ze^{-it}|^2 } \\
&=& \frac{ 1 + re^{i\theta}e^{-it} - re^{-i\theta}e^{it}
             - re^{i\theta}e^{-it}re^{-i\theta}e^{it} }{
           |1 - ze^{-it}|^2 } \\
&=& \frac{ 1 - r^2 + r\left(e^{i(\theta-t)} - e^{i(t-\theta)}\right)
         }{ |1 - ze^{-it}|^2 } \\
&=& \frac{ 1 - r^2 + 2ir\sin(\theta-t) }{ |1 - ze^{-it}|^2 } \\
\end{eqnarray*}


%%%%%%%%%%%%%%%%%%%%%%%%%%%%%%%%%%%%%%%%%%%%%%%%%%%%%%%%%%%%%%%%%%%%%%%%
%%%%%%%%%%%%%%%%%%%%%%%%%%%%%%%%%%%%%%%%%%%%%%%%%%%%%%%%%%%%%%%%%%%%%%%%
\section{Exercises} % pages 112-115

%%%%%%%%%%%%%%%%%
\begin{enumerate}
%%%%%%%%%%%%%%%%%

%%%%%%%%%%%%%% 1
\begin{excopy}
Let $X$ consist of two points $a$ and $b$,
put \(\mu(\{a\}) = \mu(\{b\}) = \half\),
and let \(L^p(\mu)\) be the resulting \emph{real} \(L^p\)-space.
Identify each real function $f$ on $X$ with the point \((f(a),f(b))\)
in the plane, and skecth the unit balls of \(L^p(\mu)\),
for\(0<p\leq \infty\).
Note that they are convex if and only if \(1\leq p \leq \infty\).
For which $p$ is the unit ball a square? A circle?
If \(\mu(\{a\}) \neq \mu(\{b\})\), how does the situation differ from the
preceding one?
\end{excopy}

The unit balls are \(B_p=\{(x,y)\in\R^2: x^p + y^p \leq 1\}\).
Assume \(1\leq p \leq \infty\) and let \((x_1,y_1),(x_2,y_2)\in B_p\).
By Minkowski's inequality (Theorem~3.5 \cite{RudinRCA87})
\[
\bigl((x_1+x_2)^p/2 + (y_1+y_2)^p/2\bigr)^{1/p}
\leq
   (x_1^p/2 + y_1^p/2)^{1/p}
 + (x_2^p/2 + y_2^p/2)^{1/p}
.\]
This is equivalent to
\(\|(x_1+x_2,y_1+y_2)\|_p \leq \|(x_1,y_1)\|_p + \|(y_2,y_2)\|_p\)
or
\[\left\|\bigl((x_1+x_2)/2,(y_1+y_2)/2\bigr)\right\|_p
 \leq
 \bigl(\|(x_1,y_1)\|_p + \|(y_2,y_2)\|_p\bigr)\,/\,2.
\]
which show convextiy.

If \(0<p<1\), take real \(\alpha\) such that \(\alpha^p = 2\).
Let
\begin{eqnarray*}
(x_1,y_1) &\eqdef& (\alpha, 0) \\
(x_2,y_2) &\eqdef& (0, \alpha)
\end{eqnarray*}
Now clearly
\(\|(x_1,y_1)\|_p =  \|(x_2,y_2)\|_p = 1\)
but
\[
   \|\bigl((x_1+x_2)/2,(y_1+y_2)/2\bigr)\|_p^p
=  (\alpha/2)^p)/2 + (\alpha/2)^p)/2
=  (\alpha/2)^p
=  2^{1-p} > 1.
\]
Hence \(B_p\) is not convex.

When \(p=1\) the unit ball is a square inscribed in the unit circle.
When \(p=2\) the unit ball is the unit circle.
When \(p=\infty\) the unit ball is a square circumscribed around the unit circle.

If \(\mu(\{a\}) \neq \mu(\{b\})\) the squares change to rectangles
and the circle eo ellipse.
% 1-800 340340

%%%%%%%%%%%%%% 2
\begin{excopy}
Prove that the unit ball (open or closed) is convex in every normed linear space.
\end{excopy}

Let $U$ be an open unit ball of a normed linear space
and let \(v_0,v_1\in U\).
For any \(t\in[0,1]\) let \(v_t \eqdef v_0 + t(v_1 - v_0)\).
We need to show that \(v_t\in U\).
We may assume that \(v_0\neq v_1\) and \(0<t<1\), since otherwise
we trivially have \(v_t=v_0\) or \(v_t=v_1\).
By definition of a norm
\[
\|v_t\| = \|(1-t)v_0 + tv_1\|
\leq \|(1-t)v_0\| + \|tv_1\|
=     (1-t)\|v_0\| + t\|v_1\| < (1-t)+t = 1.
\]

If $U$ is a closed unit ball,
we simply change the last strict inequality ($<$) into \(\leq\).


%%%%%%%%%%%%%% 3
\begin{excopy}
If \(1<p<\infty\), prove that the unit ball
of \(L^p(\mu)\) is
\index{strictly convex}
\emph{strictly convex};
this means if
\[ \|f\|_p = \|g\|_p = 1, \qquad f\neq g, \qquad h = \half(f+g), \]
then \(\|h\|_p < 1\).
(Geometrically, the surface of the ball contains no straight lines.)
Show that this fails in every \(L^1(\mu)\), and in every \(C(X)\).
(Ignore trivialities, such as spaces consisting of only one point.)
\end{excopy}

Let $X$ be the functions' domain.
For all \(x\in X\) we have
\[|h(x)| = \half|f(x)+g(x)| \leq \half(|f(x)|+g|(x)|).\]
If there is in equality, then by local lemma~\ref{llem:minkowski:eq},
there are non negative real constants $a$ and $b$ such that
\(af=bg\;\aded\), but since \(\|f\|_p = \|g\|_p\)
we have \(a=b\) and the case is trivial \(f=g\;\aded\).


%%%%%%%%%%%%%%
\begin{excopy}
Let $C$ be the space of all continuous functions on \([0,1]\),
with the supremum norm.
Let $M$ consist of all \(f\in C\) for which
\[ \int_0^\half f(t)\,dt - \int_\half^1 f(t)\,dt = 1. \]
Prove that $M$ is a closed subset of $C$ which contains no element
of minimal norm.
\end{excopy}

If \(f_n\to f\) in the supremum norm, clearly
\[ \lim_{n\to \infty} \int_0^\half f_n(t)\,dt - \int_\half^1 f_n(t)\,dt =
                      \int_0^\half f(t)\,dt - \int_\half^1 f(t)\,dt = 1. \]
Thus $M$ is closed.

We will now show that \(\|f\|_\infty > 1\) for every \(f\in M\).
Let \(f\in C([0,1])\) be such that \(\|f\|_\infty \leq 1\).
and let \(h=f(1/2)\).
Two cases:
\begin{itemize}
\item
If \(|h|<1\) then let \(\delta>0\) be such that \(|f(x)|<(1+|h|)/2\)
for all \(x\in (1/2-\delta,1/2+\delta)\). Now
\begin{eqnarray*}
\left|\int_0^\half f(t)\,dt - \int_\half^1 f(t)\,dt\right|
&=&
\left|\int_0^{1/2-\delta} f(t)\,dt
+ \int_{1/2-\delta}^\half f(t)\,dt
- \int_\half^{\half+\delta} f(t)\,dt
- \int_{\half+\delta}^1 f(t)\,dt\right| \\
&\leq& (1/2-\delta) + (\delta/2 + \delta/2)(1+|h|)/2 + (1/2 - \delta) \\
&=& 1 + 2\bigl((1+|h|)/2 - 1\bigr)\delta \\
&<& 1.
\end{eqnarray*}

\item
If \(|h|=1\), then let \(\delta>0\) be such that \(|f(x)-h| < 1/3\)
when \(x\in (\half-\delta,\half+\delta)\).
Note that if \(|y_i-h|<1/3\) for \(i=1,2\) then \(|y_1-y_2| < 2/3\).
Now
\begin{eqnarray*}
\left|\int_0^\half f(t)\,dt - \int_\half^1 f(t)\,dt\right|
&=&
\left|\int_0^{1/2-\delta} f(t)\,dt
+ \int_{1/2-\delta}^\half f(t)\,dt
- \int_\half^{\half+\delta} f(t)\,dt
- \int_{\half+\delta}^1 f(t)\,dt\right| \\
&\leq& (1/2 - \delta)
       + \int_0^\delta \left(f(t+h-\delta) - f(t+h)\right)dt
       + (1/2 - \delta) \\
&\leq& 1 - 2\delta + (2/3)\delta \\
&<& 1.
\end{eqnarray*}
\end{itemize}
In both case we see that \(f\notin M\).
Next we will show that for any \(\epsilon>0\) there exists \(f\in M\)
such that \(\|f\|_\infty > 1 + \epsilon\).
These results combined, show that $M$ has no minimal normed element.


\[
\left|\int_0^\half f(t)\,dt - \int_\half^1 f(t)\,dt\right|
\leq \int_0^\half |f(t)|\,dt + \int_\half^1 |f(t)|\,dt
= \int_0^1 |f(t)|\,dt \leq \|f\|_\infty
\]
Let \(\epsilon>0\). Define
\[
f(x) = \left\{
       \begin{array}{ll}
       -1 - \epsilon& \qquad 0\leq x \leq 1/2 - \delta \\
        (x-1/2) (1+\epsilon)/\delta
                        & \qquad 1/2 - \delta \leq x \leq 1/2 + \delta \\
       1 + \epsilon  & \qquad 1/2 + \delta \leq x \leq 1
       \end{array}\right.
\]
Clearly \(\|f\|_\infty = 1/2+\epsilon\) and if we pick
\(\delta\) such that
\[2(1+\epsilon)(1/2 - \delta) + 2\delta(1+\epsilon)/2 = 1\]
we can have \(f\in M\).
Simplifying:
\[2(1+\epsilon)(1/2 - \delta) + 2\delta(1+\epsilon)/2
  = (1+\epsilon)(2(1/2 - \delta) + \delta)
  = (1+\epsilon)(1 - \delta)\]
and solving the above,
gives \(\delta = 1 - 1/(1+\epsilon) = \epsilon/(1+\epsilon)\).



%%%%%%%%%%%%%% 5
\begin{excopy}
Let $M$ be the set of all \(f\in L^1([0,1])\), relative to Lebesgue measure,
such that
\[\int_0^1 f(t)\,dt = 1.\]
Show that $M$ is a closed subset of  \(L^1([0,1])\) which contains
infinitely many element of minimal norm.
(Compare this and Exercise~4 with Then~4.10.)
\end{excopy}

Let \(\{f_n\}\) be a sequence in $M$ such that
\(\lim_{n\to\infty}\|f_n - f\|_1 = 0\)
where \(f\in L^1([0,1])\). Now
\begin{eqnarray*}
\|f\|_1 &\leq& \|f_n - f\|_1 + \|f_n\|_1 \\
 \|f_n\|_1 &\leq& \|f_n - f\|_1 + \|f\|_1
\end{eqnarray*}
and so given \(\epsilon > 0\) for sufficiently large $n$
\[ 1 - \epsilon \leq \|f\|_1 \leq 1 + \epsilon\]
thus \(\|f\|_1=1\) and $M$ is closed.



%%%%%%%%%%%%%% 6
\begin{excopy}
Let $f$ be a bounded linear functional on a subspace $M$ of a Hilbert space $H$.
Prove that $f$ has a \emph{unique} norm-preserving extension to a bounded
linear functional on $H$, and that this extension vanishes on \(M^\perp\)
\end{excopy}

The case \(f=0\) is trivial, thus \wlogy by dividing by \(\|f\|\)
we may assume that \(\|f\|=1\).
Clearly for every
\(x=v+w\in H\) where \(v\in M\) and \(w\in M^\perp\)
we can define \(F(x) = f(v)\) which is a well defined extension
and clearly \(\|F\|=1\).

Now we will show that this extension is unique.
Let $F$ be an norm-preserving extension of $f$ on $H$.
By negation, assume that $F$ does not vanish on \(M^\perp\)
we can find \(w\in M^\perp\) such that \(\|w\|=1\)
and \(F(w)=a>0\). Pick and arbitrary \(\epsilon > 0\),
for which pick \(u\in M\) such that \(\|u\|=1\)
and \(h \eqdef f(u) > 1-\epsilon\).
We will contradict the assumtion if we show
that there exist a real $t$ such that
\[\frac{F(u+tw)}{\|u+tw\|} > 1.\]
Equivalently,
\[(h+ta)^2 = (F(u+tw))^2 > \|u+tw\|^2 = t^2 + 1.\]
that is
\[(1-a^2)t^2 - 2hat + (1-h^2) < 0.\]
Since the leaading coefficient \(1-a^2>0\),
by looking at the discriminant, there exist some $T$
satisfying the above inequality(ies), iff
\[4h^2a^2 - 4(1-a^2)(1-h^2) > 0\]
that is if
\begin{equation} \label{eq:ex:5.6:disc}
(h^2a^2)>(1-a^2)(1-h^2).
\end{equation}
Since \(\lim_{\epsilon\to 0} h = 1\),
When \(\epsilon\to 0\), the left side of \eqref{eq:ex:5.6:disc}
converges to \(a^2 > 0\) while zero is the limit of the right side.
Hence such $t$ exists, \(\|F\|>1\) contradicting the assumtion.


%%%%%%%%%%%%%% 7
\begin{excopy}
Construct a bounded linear functional on some subspace of some \(L^1(\mu)\)
which has two (hence inifinitely many) distinct norm-preserving
to \(L^1(\mu)\).
\end{excopy}

Define \(X=\{0,1\}\) and \(\mu(\{0\}) = \mu(\{1\}) = 1/2\)
Let
\[ M \eqdef \{f\in L^1(X,\mu): f(1) = 0\}\]
and the functional on $M$  \(\Lambda f = f(0)/2\).
If \(f\in M\) then \(\|f\|_1 = |f(0)|/2\), and so
\[ \|\Lambda\|
= \sup \{|\Lambda f|: f \in M\;\wedge\;\|f\|=1\}
= \sup \{|f(0)/2|: f \in M\;\wedge\;|f(0)|/2=1\}
= 1. \]

Now we can easily extend \(\Lambda\) to \(L^1(X,\mu)\)
by the following two extensions
\begin{eqnarray*}
\Lambda_1(f) &=& f(0)/2 \\
\Lambda_2(f) &=& f(0)/2 + f(1)/2
\end{eqnarray*}


%%%%%%%%%%%%%% 8
\begin{excopy}
Let $X$ be a normed linear space, and let \(X^*\) be its dual space, as defined
in Sec.~5.21, with the norm
\[\|f\| = \sup \{|f(x)|: \|x\|\leq 1\}.\]
\begin{itemize}
 \itemch{a} Prove that \(X^*\) is a Banach space.
 \itemch{b}
 Prove that the mapping \(f\to f(x)\) is, for each \(x\in X\),
 a bounded linear functional on \(X^*\), of norm \(\|x\|\).
 (This gives a natural imbedding of $X$ on its ``second dual'' \(X^{**}\),
 the dual space of \(X^*\).)
 \itemch{c} Prove that \(\{\|x_n\|\}\) is bounded if \mset{x_n} is a sequence
 in $X$ such that \(\{f(x_n)\}\) is bounded for every \(f\in X^*\).
\end{itemize}
\end{excopy}

\begin{itemize}
 \itemch{a}
 Let \(\Lambda_1,\Lambda_2 \in X^*\) and \(z_1,z_2 \in \C\)
 for any \(x\in X\), by definition, we have
 \[ (z_1\Lambda_1 + z_2\Lambda_2)(x) = z_1\Lambda_1(x) + z_2\Lambda_2)(x)  \]
 and so \(X^*\) is linear vector space over \(\C\).
 Also \(\|z\Lambda(x)\| = |z||\Lambda(x)|\) so to prove \(\|\cdot\|_{X^*}\)
 is a norm, we show the triangle inequality:
 \begin{eqnarray*}
 \|\Lambda_1 + \Lambda_2\| 
 &=& \sup\{|(\Lambda_1 + \Lambda_2)(x)|: x\in X,\;\|x|=1\} \\
 &=& \sup\{|\Lambda_1(x) + \Lambda_2(x)|: x\in X,\;\|x|=1\} \\
 &\leq& \sup\{|\Lambda_1(x)| + |\Lambda_2(x)|: x\in X,\;\|x|=1\} \\
 &\leq&    \sup\{|\Lambda_1(x)|: x\in X,\;\|x|=1\}
         + \sup\{|\Lambda_2(x)|: x\in X,\;\|x|=1\} \\
 &=& \|\Lambda_1\| + \|\Lambda_2\|.
 \end{eqnarray*}
 It is now left to show completeness. 
 Let \mset{\Lambda_n} be a Cauchy sequence in \(X^*\).
 Pick arbitrary \(x\in X\setminus\{0\}\). Now since
 \begin{eqnarray*}
 |\Lambda_m(x) - \Lambda_n(x)|
 = |(\Lambda_m - \Lambda_n)(x)|
 = \|x\|\cdot |(\Lambda_m - \Lambda_n)(x/\|x\|)|
 \leq \|x\|\cdot\|\Lambda_m - \Lambda_n\|
 \end{eqnarray*}
 \mset{\Lambda_n x} is a Cauchy sequence in \(\C\) and converges there,
 and for \(x=0\) as well.
 Thus we can define the limit 
 \[ \Lambda_x \eqdef \lim_{n\to\infty} = \lim_{n\to\infty} \Lambda_n x.\]
 Now for \(x_1,x_2\in X\) and \(z_1,z_2\in \C\)
 \begin{eqnarray*}
 \Lambda (z_1 x_1 + z_2 x_2)
 &=& \lim_{n\to\infty} \Lambda_n (z_1 x_1 + z_2 x_2) \\
 &=& \lim_{n\to\infty} \bigl(z_1\Lambda_n(x_1) + z_2 \Lambda(x_2)\bigr) \\
 &=& \bigl(
             \lim_{n\to\infty} z_1\Lambda_n(x_1)
           + \lim_{n\to\infty} z_2\Lambda_n(x_2) \bigr) \\
 &=& \bigl(  z_1\lim_{n\to\infty} \Lambda_n(x_1)
           + z_2\lim_{n\to\infty} \Lambda_n(x_2) \bigr) \\
 &=& z_1\Lambda(x_1) + z_2\Lambda(x_2).
 \end{eqnarray*}
 To show \(\Lambda\) is bounded, let \(\|x\| \leq 1\) in $X$, now
 \[ |\Lambda(x)| = |\lim_{n\to\infty} \Lambda_n(x)| 
    \leq = \lim_{n\to\infty} \|\Lambda_n\| < \infty\]
 and the limit is independent of $x$.

 Therefore \(X^*\) \emph{is} a banach space.
 
 \itemch{b} Let \(x\in X\). Since
 \[(z_1f_1+z_2f_2)(x) = z_1f_1(x) + z_2f_2(x)
   \qquad f_1,f_2\in X^*\;z_1,z_2\in \C\]
 clearly \(f\to f(x)\) is a linear functional on \(X^*\).
 Define \(x^{**}\in X^{**}\) by \(x^{**}(f) = f(x)\) for all \(f\in X^*\).
 If \(x=0\) then \(\|x^**\|_{X^{**}} = 0\) trivially, for \(x\neq 0\)
 \begin{eqnarray}
 \|x^{**}\|_{x^{**}} 
 &=& \sup\{|f(x)|: \|f\|_{X^*} = 1,\;f\in X^*\} \notag \\
 &=& \sup\bigl\{|f(x)|: \sup\{ |f(w)|: \|w\|=1,\; f\in X^*,\;w\in X\}=1\bigr\} 
     \notag \\
 &\leq& \sup\bigl\{|f(x)|: \sup\{ |f(x/\|x\|)|: f\in X^*\}=1\bigr\} 
        \label{eq:5.8:sup1} \\
 &\leq& \|x\| \label{eq:5.8:sup2}.
 \end{eqnarray}
 We now show the reverese inequality. Define a functional \(f_x\)
 on a 1-dimensional subspace of $X$ generated by $x$, as \(f_x(zx/\|x\|) = z\).
 Clearly \(\|f_x\| = f(x)/\|x\| = 1\)
 and applying Hahn-Banach theorem, we can extend
 \(f_x\) to a functional $f$ on whole $X$, such that 
 \(\|f\|_{X^*} = \|f_x\| = 1\).
 Using it in the supremum of \eqref{eq:5.8:sup1} and \eqref{eq:5.8:sup2} 
 we get an equality and so \(\|x^{**}\|_{X^{**}} = \|x\|\).
 
 \itemch{c}
 By negation, assume \mset{x_n} is unbounded.
 Let $W$ be the subspace of $X$ generated by \mset{x_n}.
 Two cases:

 \paragraph{Case 1.} Assume \(\dim(W) < \infty\) and \seqn{v} its base.
 Thus we have a unique representation 
 \(x = \sum_{i=j}^n a_j(x) v_j\), where \(a_j\in X^*\).
 Since \mset{x_n} is unbounded, there exist some $j$ such that
 \(\{a_j(x_i)\}_{i=1}^\infty\) is not bounded.

 \paragraph{Case 2.} Assume \(\dim(W)\infty\). 
 Let \(W_n\) be the subspace of $W$ generated by \seqn{x}.
 Clearly \(W = \cup_{n=1}^\infty W_n\).
 For convenience, we drop from \seqn{x} any \(x_j\) such that 
 \(x_j \in X_{j-1}\); note that zeros are dropped. 
 This case assumtion, guarantees that the sequence is still infinite.
 The functional $f$ to be constructed, 
 will furnish a contradiction, namely \mset{f(x_n)} unbounded for 
 the original, trivially, as well.

 For each \(w\in W\) there is a unique (finite) representation
 \[ w = \sum_{j=1}^{M(w)} a_j(w) (x_j/\|x_j\|).  \]
 Note that \(a_n(x_n) = \|x_n\|\).
 We will define a sequence of functionals \(f_n: W_n \to \C\).
 \begin{eqnarray*}
 % u_n &=& \overline{a_n(x_n)}/|x_n| \\
 f_n(w) &=& f\left(\sum_{j=1}^{M_w} a_j(w) x_j\right) 
       \eqdef \sum_{j=1}^{M_w} u_j a_j(w) \qquad w \in W_{M_w}
 \end{eqnarray*}
 By construction, 
 \begin{eqnarray*}
  % |u_n| &\in& \{0,1\} \\
  f_n(x_n) &=& \sum_{j=1}^n a_j(x_j) = \|x_n\| \\
  f_n &=& {f_m}_{|W_n}  \qquad \textrm{if}\, n\leq m
 \end{eqnarray*}
 Thus we can define the functional \(f = \cup f_n\), 
 for which \(f(x_n) = \|x_n\|\) for all $n$.
\end{itemize}


%%%%%%%%%%%%%% 9
\begin{excopy}
Let \(c_0\), \(\ell^1\), and \(\ell^\infty\) be Banach spaces consisting
of all complex sequences \(x=\{\xi_i\}\), \(\mbox{i=1,2,3,\ldots,}\) 
defined as follows:
\begin{alignat*}{3}
x&\in\ell^1 & &\quad\textrm{if and only if} 
   \quad && \|x\|_1 = \sum|\xi_i|<\infty.\\
x&\in\ell^\infty && \quad\textrm{if and only if} 
   \quad && \|x\|_\infty = \sup |\xi_i|<\infty.
\end{alignat*}
\(c_0\) is the subspace of \(\ell^\infty\) consisting of all \(x\in\ell^\infty\)
for which \(\xi_i \to 0\) as \(i\to\infty\).

Prove the following statements.
\begin{itemize}
 \itemch{a} If \(y=\{\eta_i\}\in \ell^1\) and \(\Lambda x = \sum \xi_i\eta_i\)
            for every \(x\in c_0\), then \(\Lambda\)
            is a~bounded linear functional on \(c_0\),
            and \(\|\Lambda\| = \|y\|_1\).
            Moreover, every \(\Lambda \in (c_0)^*\) is obtain in this way.
            In brief, \((c_0)^* = \ell^1\).

            (More precisely, these two space are not equal, the preceding
            statement, exhibits as isometric vector space isomorphism
            between them.
 \itemch{b} In the same sense, \((\ell^1)^* = \ell^\infty\).
 \itemch{c} Every \(y\in \ell^1\), induces a bounded linear functional on
            \(\ell^\infty\), as in \ich{a}.
            However, this does \emph{not} give all of \((\ell^\infty)^*\),
            since \((\ell^\infty)^*\) contains nontrivial functionals that
            vanish on all of \(c_0\).
 \itemch{d} \(c_0\) and \(\ell^1\) are separable but \(\ell^\infty\) is not.
\end{itemize}
\end{excopy}

\begin{itemize}
\itemch{a}
Clearly, \(\Lambda\) is linear.
Estimate:
\[ |\Lambda x| 
   = \left|\sum \eta_i\xi_i\right|
   \leq \sum |\eta_i \xi_i|
   \leq \left(\sum |\eta_i|\right) \left(\sup |x_i|\right) 
   = \|y\|_1 \cdot \|x\|\_infty.\]
Thus \(\|\Lambda\| \leq \|y\|_1\). 

For the reverse inequality, we may assume \(y\neq 0\).
let \(\|y\|_1 > \epsilon>0\),
let \(N>0\) be such that \(\sum_{i>N} |\eta_i| < \epsilon\).
% and let $M$ be the number of nonzero components of $y$ upto $N$;
% formally: \(M = |\{i\in\N: 1\leq i \leq N\,\wedge\, \eta_i=0\}|\).
In the next definitions and derivations we freely use \(0/0 = 0\).
Put \(u_i = \overline{\eta_i}/|\eta_i|\).
and define \(x=(\xi_i)_i\) by 
\[
   \xi_i = \left\{\begin{array}{ll}
                   u_i   & \qquad \textrm{if}\;  1\leq i\leq N \\
                   0     & \qquad \textrm{if}\;  i> N
                  \end{array}\right.\]

Now 
\[\Lambda x 
 = \sum \xi_i\eta_i
 = \sum_{i=1}^N \overline{\eta_i}\eta_i \,/\, |\eta_i|
 = \sum_{i=1}^N |\eta_i|
 > \|y\|_1 - \epsilon.\]
Since \(\epsilon\) can be arbitrarily small, \(\|\Lambda\|\geq \|y\|_1\).

Now Pick arbitrary \(f\in c_0^*\). 
Define \mset{e_i} in \(c_0\) as a function \(\N\to \C\)
by \(e_i(n) = delta_{in}\) that is \(e_i(i)=1\) and zero otherwise.
Define \(x=(\xi_i)_{i=1}^\infty\) by \(\xi_i = f(e_i)\)
and \(\Lambda\) as above. 
For each \(j\in \N\), we have
\[\Lambda e_j = \sum_{i=1}^\infty \xi_i e_j(i) = \xi_j.\]
Thus \(\Lambda\) and $f$ agree on the subspace $S$ generated by \mset{e_i}.
It is easy to see that \(c_0 = \overline{S}\) and thus \(\Lambda = f\)
that is every functional on \(c_0\) is given by the above ``producs-sum'' form.

\itemch{b}
 Let \(x=\{\xi_i\}\in \ell^1\) 
 \(y=\{\eta_i\}\in \ell^\infty\) and \(\Lambda x = \sum \eta_i\xi_i\).
 for 
 \[ |\Lambda x| \leq \sum |\eta_i\xi_i| 
    \leq \left(\sup \eta_i\right)\left( \sum |\eta_i\xi_i| \right)
    = \|y\|_\infty \cdot \|x\|_1.\]
 Hence \(\|\Lambda\| \leq \|y\|_\infty\).
 Given \(\epsilon>0\) pick some $J$ such that \(|\eta_j| \geq \|y\|-\epsilon\),
 and let \(e_j\in l_1\) 
 such that \(e_j(n) = 1\) if \(n\neq j\) and \(e_j(j)=1\).
 Now \(\Lambda e_j = \eta_j\) and so \(\|\Lambda\| \geq \|y\|-\epsilon\).
 Since \(\epsilon\) may be arbitrarily small, \(\|\Lambda\| \geq \|y\|_\infty\),
 hence \(\|\Lambda\| = \|y\|_\infty\).

 Now let \(f\in (\ell^1)^*\). Define \(e_j\) as before for all $j$, 
 let \(\eta_j= f(e_j)\) and let \(y=\{\eta_i\}\).
 Since \(\|e_j\|=1\) for all $J$, we have \(|\eta_j| \leq \|f\|\)
 and so \(y\in \ell^\infty\).

\itemch{c}
 Construct a functional \(f\in (\ell^\infty)^*\) as follows.
 It vanishes \(f(x)=0\) for all \(x\in c_0\). 
 Pick \(\mathbf{u}=(u_i)_{i\in\N}\)
 such that \(u_i=1\) for all \(i\in\N\). 
 Clearly \(u\in \ell^\infty\setminus c_0\).
 Now define \(f(u)=1\) and extend $f$ to \(\ell^\infty\) by Hahn-Banach theorem.
 Clearly $f$ is not in the image of \(\ell^1\) embedding in 
 \((\ell^\infty)^{**}\) 
 
\itemch{d}
 The extension field \(\Q(i)\in \C\) of the rational is actually
 \[\Q(i) = \{a+bi:\;a,b\in\Q\}.\]
 Clearly \(|\Q(i)|=\aleph_0\).
 Define the set $E$ of inifinite sequences of \(\Q(i)\) which 
 have only finite nonzero components. Formally:
 \[ E \eqdef \left\{x\in \bigl(Q(i)\bigr)^\N:\; 
                     \exists M,\; \forall n>m\,\Rightarrow x(n)=0\right\}.\]
 Now \(|E|=\aleph_0\) and $E$ is dense in \(c_0\) and \(\ell^1\).
 
 {\small
 \emph{Sketch:} Pick \(v\in c_0\) or \(v \in \ell^1\) and \(\epsilon>0\).
 \(c_0\)-case: \(\exists M, \forall n>m,\, |v_n|<\epsilon/2\). 
 \(\ell^1\)-case: \(\exists M, \sum{j=1}^M |v_j| <\epsilon/2\). 
 Then \(\epsilon/2\)-approximate $v$ trimmed above $M$, by some \(x\in E\).
 Now \(\|x-v\|<\epsilon/2+\epsilon/2=\epsilon\).
 }

 Define a subset
 \[V \eqdef  \{0,1\}^\N \subset \ell^\infty \subset \C^\N.\]
 Note that \(\|x-y\|\in\{0,1\}\) for all \(x,y\in V\).
 Clearly \(|V| = 2^{\aleph_0} > \aleph_0\).
 If by negation \(\ell^\infty\) is separable, let $D$ be a countable dense
 set in  \(\ell^\infty\). 
 Define a mapping \(\nu:V\to D\) by picking 
 for each \(v\in V\) an \(x\in D\) such that \(\|x-v\|<1/2\).
 Existence of \(\nu(v)\) is guaranteed by density of $D$.
 This map must be one-to-one, since if for some \(x\in D\)
 there exist \(v_1,v_2\in V\) such that \(\|x-v_i\|<1/2\) for \(i=1,2\)
 then \(\|v_1-v_2\|<1\) but then \(v_1=v_2\).
 But now a cardinality \(|V|<|D|\) contradiction.
\end{itemize}

%%%%%%%%%%%%%% 10
\begin{excopy}
If \(\sum \alpha_i\xi_i\) converges for every sequence \((\xi_i)\) such that
\(\xi_i \to 0\) as \(i\to\infty\), prove that \(\sum|\alpha_i| < \infty\).
\end{excopy}

This is a consequence of Exercise~9\ich{a} above.

%%%%%%%%%%%%%% 11
\begin{excopy}
For \(0<\alpha\leq 1\),
\index{Lip@\(\Lip\)}
let \(\Lip\alpha\) denote the space of all complex
functions $f$ on \([a,b]\) for which
\[M_f = \sup_{s\neq t} \frac{|f(s)-f(t)|}{|s-t|^\alpha} < \infty.\]
Prove  that \(\Lip\alpha\) is a Banach space, if \(\|f\| = |f(a)| + M_f\);
also if
\[\|f\| = M_f + \sup_x |f(x)|.\]
(The members of \(\Lip\alpha\) are said to satisfy
\index{Lipschitz condition}
a~\emph{Lipschitz condition} of order \(\alpha\).)
\end{excopy}

We first show that \(\|\cdot\|\) is indeed a norm.
\paragraph{Scalar Multiplication}.
Let \(z\in\C\). For all \(f\in \Lip\alpha\)
\[
M_{zf} 
= \sup_{s\neq t} \frac{|(zf)(s)-(zf)(t)|}{|s-t|^\alpha} 
= |z|\sup_{s\neq t} \frac{|f(s)-f(t)|}{|s-t|^\alpha} 
|z| M_f.\]
Hence
\[
\|zf\| = |(zf)(a)| + M_{zf} = |z|\cdot|f(a)| + |z|M_f = |z|\cdot\|f\|.
\]

\paragraph{Sub additivity}. For all \(f,g\in \Lip\alpha\)
\begin{eqnarray*}
\|f+g\| 
&=& |(f+g)(a)| + \sup_{s\neq t} \frac{|(f+g)(s)-(f+g)(t)|}{|s-t|^\alpha} \\
&=& |f(a) + g(a)| + 
  \sup_{s\neq t} \frac{\left|\bigl(f(s)-f(t)\bigr) + 
                             \bigl(g(s)-g(t)\bigr)\right|)}{|s-t|^\alpha} \\
&\leq& |f(a)| + |g(a)| + 
     + \sup_{s\neq t} \frac{|(f(s)-f(t)|}{|s-t|^\alpha} 
     + \sup_{s\neq t} \frac{|(g(s)-g(t)|}{|s-t|^\alpha} \\
&=& \|f\|+\|g\|.
\end{eqnarray*}

Clearly if \(\|f\|=0\) then \(f(a)=0\) and \(f'(t)=0\) for all \(t\in [a,b]\)
and so \(f=0\).

To show that \(\Lip\alpha\) is a Banach space it is left to show completeness.
Let \mset{f_n} be a Cauchy sequence in \(\Lip\alpha\).
Since \(|f_m(a)-f_n(a)| \leq \|f_m - f_n\|\), 
the evaluations \(\{f_n(a)\}\) is a Cauchy sequence in \(\C\).
Similarly for \(t\in(a,b]\)
\[
\frac{|(f_m - f_n)(t) - (f_m - f_n)(a)|}{|s-t|^\alpha} \leq \|f_m - f_n\|,\]
equivalently
\(|(f_m - f_n)(t) - (f_m - f_n)(a)| \leq |s-t|^\alpha \|f_m - f_n\|\)
hence \mset{f_n(t)} is a Cauchy sequence in \(\C\) as well.
Therefore we can define \(f(t) = \lim_{n\to\infty} f_n(t)\).
If by negation \(M_f = \infty\) then for any \(0<M<\infty\) we can find
\(s,t\in[a,b]\) such that \(s\neq t\) and \(|f(s)-f(t)|\geq |s-t|^\alpha M\).
But then we can find some $n$ such that \(f_n\) is sufficiently close 
to $f$ in \(\{s,t\}\) so \(|f_n(s)-f_n(t)|\geq |s-t|^\alpha (M-\epsilon)\).
But then \mset{M_{f_n}} is not bounded, which is a contradiction.


%%%%%%%%%%%%%% 12
\begin{excopy}
Let $K$ be triangle (two-dimensional figure) in the plane,
let $H$ be the set consisting of the vertices of $K$, of the form
\[ f(x,y) = \alpha x + \beta y + \gamma
   \qquad (\alpha, \beta,\, \textrm{and}\, \gamma\; \textrm{real}). \]
Show that to each \((x_0,y_0)\in K\) there corresponds
a unique measure \(\mu\) on $H$ such that
\[ f(x_0,y_0) = \int_H f\,d\mu. \]
(Compare Sec.~5.22.)

Replace $K$ by a square, let $H$ again be the set of its vertices, and let $A$
be as above.
Show that to each point of $K$ there still corresponds a measure on $H$,
with the above property, but that uniqueness is now lost.

Can you extrapolate to a more general theorem?
(Think of other figures, higher dimensional spaces.)
\end{excopy}

Denote the vertices $H$ of the triangle $K$ 
by \(v_i = (x_i,y_i\) for \(i=1,2,3\).
Since any triangle is convex, 
for any \((x_0,y_0)\in K\) there is a convex combination
\[(x_0,y_0) = \sum_{j=1}^3 w_j(x_j,y_j)\]
where the weights \(w_j\in[0,1]\) and \(\sum_{j=1}^3 w_j = 1\).
Put \(\mu(\{v_j\}) = w_j\). 
Now for any \(f(x,y) = \alpha x + \beta y + \gamma\) 
we compute:
\[
f(x_0,y_0)
 = \alpha x_0 + \beta y_0 + \gamma 
 = \sum_{j=1}^3  w_j (\alpha x_j + \beta y_j + \gamma) 
 = \sum_{j=1}^3 f(x_j,y_j) w_j 
 = \int_H^f\,d\mu.
\]

\paragraph{Uniqueness.} Let \(\mu_1\) and \(\mu_2\)
satsisfying the above condition for any $f$ of the above (affine) form.
\Wlogy, we may assume \(\mu_1(\{v_1\}) \neq \mu_2(\{v_1\})\).
We solve the following system of linear equations:
\begin{eqnarray*}
\alpha x_1 + \beta y_1 + \gamma &=& 1 \\
\alpha x_2 + \beta y_2 + \gamma &=& 0 \\
\alpha x_3 + \beta y_3 + \gamma &=& 0
\end{eqnarray*}
Where \(\alpha,\beta,\gamma\in\R\) are the unknowns.
The vertices are not co-linear, and there must exist a solution.
But now,
\[\int_H f\,d\mu_1 = \mu_1(\{v_1\} \neq \mu_2(\{v_1\}) = \int_H f\,d\mu_2\]
is a contradiction.

When $K$ is a squere we do the same, but now there are 
inifnite number of convex combinations of the verices.
Thus the representing measure is not unique in the interior of $K$.

As for higher dimension. In \(\R^n\) for \(n\geq 2\),
any  set $H$ of \(n-1\) points which do not lie in the same hyperplane
generate a compact convex hull $K$.
The condition on $H$ as equivalent of saying that they do not 
simultanously satisfy a single non trivial linear equation in \(\R^n\).
The ``extreme'' vertices of of $K$ are $H$.
Now, for any \(\mathbf{x}\in K\) there exist a unique measure 
\(\mu_{\mathbf{x}}\) on $H$ such that
\[f(\mathbf{x}) 
  = \int_H f\,d\mu_{\mathbf{x}} 
  = \sum_{v\in H} \mu_{\mathbf{x}}(v) \cdot f(v)\,.\]

%%%%%%%%%%%%%% 13
\begin{excopy}
Let \(\{f_n\}\) be a sequence of continuous complex functions on a (nonempty)
complete metric space $X$, such that \(f(x) = \lim f_n(x)\) exists
(as a complex number) for every \(x\in X\).
\begin{itemize}
 \itemch{a} Prove that there is an open set \(V\neq \emptyset\) and a number
            \(M<\infty\) such that \(|f_n(x)| < M\) for all \(x\in V\)
            and \(n=1,2,3,\ldots\).
 \itemch{b} If \(\epsilon>0\), prove that there is an open set \(V\neq\emptyset\)
            and an integer $N$ such that \(|f(x) - f_n(x)| \leq \epsilon\)
            if \(x\in V\) and \(n\geq N\).

\end{itemize}
\emph{Hint for \ich{b}}: For \(N=1,2,3,\ldots\), put
\[ A_N = \{x: |f_m(x)-f_n(x)|\leq \epsilon
\;\textrm{if}\; m\geq N\;\textrm{and}\; n\geq N\}.\]
Since \(X=\cup A_N\), some \(A_N\) has a nonempty interior.
\end{excopy}


\begin{itemize}
\itemch{a}
Let 
\[ B_M \eqdef  \{x\in X: \forall n\in\N,\, |f_n(x)|\leq M\} 
  = \bigcup_{n\in\N} \{x\in X: |f_n(x)|\leq M\}.\]
Clearly \(B_M\) are closed. For any \(x\in X\), 
the sequence \mset{f_n(x)} hence \mset{|f_n(x)|} is bounded.
Therefore, \(X=\cup_{M\in\N} B_M\) and
\index{Baire!category theorem}
by Baire's category theorem~5.6 since $X$ is complete, some \(B_M\) 
must have non empty interior $V$.

\itemch{b}
The sets 
\[ A_N = \{x: |f_m(x)-f_n(x)|\leq \epsilon\;\textrm{if}\; m,n\geq N\}\]
are closed. Similarly, \(X=\cup_{N\in\N} A_N\)
and so again by Baire's theorem, some \(A_N\) has non empty interior $V$.
\end{itemize}

%%%%%%%%%%%%%% 14
\begin{excopy}
Let $C$ be the space of all real continuous functions on \(I=[0,1]\)
with the supremum norm. Let $X$, be the subset of $C$ consisting of those
$f$ for which there exists a \(t\in I\) such that \(|f(s)-f(t)| \leq n|s-t|\)
for all \(s\in I\). Fix $n$ and prove that each open set in $C$ contains
an open set which does not intersect \(X_n\). (Each \(f\in C\) can be uniformly
approximated by a zigzag function $g$ with very large slopes and if
\(\|g-h\|\) is small, \(h\notin X_n\).)
Show that this implies the existence of a dense \(G_\delta\) in $C$ which
consists entirely of nowhere differentiable functions.
\end{excopy}


Fix $n$ and take a base open set in $C$ by picking \(f\in C\) and 
\(\epsilon>0\), and setting
\[V \eqdef \{\phi\in C: \|f-\phi\|_\infty < \epsilon/3\}.\]
In $I$ every continuous functions is uniformly continuous.
Hence there exists \(\delta>0\) such that \(|f(t)-f(s)|<\epsilon\)
whenever \(|s-t|<\delta\). 
Let 
\[s = \lceil 1\bigm/\,\min\bigl(\delta, \epsilon/(3(n+1))\bigr)\rceil\]
and split $I$ by \(x_i = i/(s+1)\) for \(0\leq i \leq s+1\).
Note that \(x_i - x_{i-1} < \delta/(n+1)\).
We will now construct \(g\in V\).
Intuitively, it will have on each subsegment \([x_{i-1},x_i]\) 
a slope of \(\pm(n+1)\), and it will increase or decrease so it follows $f$.
Formally, we define $g$ by induction on \([x_{i-1},x_i]\) for \(0< i \leq s+1\).
On the first subsegment \([x_0,x_1]\) let
\(g(x) = f(0) + (n+1)x\). Assume that $g$ is defined
on \([x_0,x_{i-1}]\), and by induction on \([x_{i-1}, x_i]\) by
\[g(x) = g(x_{i-1}) + \sigma (n+1)(x - x_{i-1})\]
where \(\sigma=1\) if \(g(x_{i-1}) < f(x{i-1})\) and 
\(\sigma= -1\) otherwise. It is easy to see that \(\|g-f\|_\infty < \epsilon/3\)
and that \(g\notin X_n\).
We now need a trivial lemma
\begin{llem}
Let \(f:[x_0,x_1]\to \R\) be defined by \(f(x) = ax+b\). 
For any \(\epsilon>0\)
such that \(\epsilon<|a|\) there exists \(\delta>0\) such that 
if \(h:[x_0,x_1]\to \R\) and \(\|f-h\|_\infty < \delta\)
then 
\begin{equation} \label{eq:ex5.14:lem}
 \max\left( \frac{h(t) - h(x_0)}{t-x_0}, 
              \frac{h(x_1) - h(t)}{x_1 - t} \right) < |a|-\epsilon. 
\end{equation}
for all \(t\in(x_0,x_1)\) we 
\end{llem}
\begin{thmproof}
\iffalse
By defining  \(\tilde{f}:[0,x_1-x_0]\to \R\)
as \(\tilde{f}(x) = |a|x\)
we see that we can assume that \(f(0)=0\) and \(a>0\).
Since we can similarly convert a $h$ 
and get the same ratios of \eqref{eq:ex5.14:lem}.
\fi
\Wlogy, we may assume \(a>0\), the negative case is analogous.
Let \(\delta = \epsilon(x_1-x_0)/4\), and assume \(\|f-h\|_\infty < \delta\).
For any \(t\in (x_0,x_1)\), there are two cases

\paragraph{High Case}: \((x_0+x_1)/2 \leq t < x_1\).
Let us estimate,
\begin{eqnarray*}
\frac{h(t) - h(x_0)}{t-x_0}
&=& \biggl( f(t) + (h(t)-f(t)) - \bigl(f(x_0) + (h(x_0)-f(x_0))\bigr)\biggr)
    \bigm /\,(t - x_0) \\
&\geq& \bigl(f(t) - f(x_0)\bigr) / (t - x_0) 
        - 2\delta / \bigl((x_1 - x_0)/2\bigr) \\
&=& a - 4\delta/(x_1 - x_0) \\
&=& a - \epsilon.
\end{eqnarray*}

\paragraph{Low Case}: \(x_0 < t \leq (x_0+x_1)/2\), similar derivation
as the high-case, but estimating the slope against \((x_0,h(x_0))\).
\end{thmproof}

From this lemma, back to this exercise, we see that we can pick 
some \(\delta>0\) such that for any \(h:I\to\R\) such that
\(\|g-h\|_\infty\), that is a neighborhood of $g$ the zigzag function, 
\(h\notin X_n\). 
Therefore, \[V_n \eqdef \inter{\left(C(I)\setminus X_n\right)}\] is dense.
By the corollary of theorem~5.6 in \cite{RudinRCA87}, 
\(\cap_{n\in\N} V_n\) is a dense \(G_\delta\) set, 
(and in particular non empty) that consists of nowhere differentiable functions.


%%%%%%%%%%%%%% 15
\begin{excopy}Let \(A=(a_{ij})\) be an infinite matrix with complex entries,
wher \(i,j=0,1,2,\ldots\).
$A$ associateswith each
sequence \(\{s_i\}\)
a sequence \(\{\sigma_i\}\), defined by
\[ \sigma_i = \sum_{j=0}^\infty a_{ij} s_j \qquad (i=1,2,3,\ldots), \]
provided that these series converge.

Prove that $A$ transforms every convergent sequence
\(\{s_i\}\)
to a sequence \(\{\sigma_i\}\) which converges to the same limit
if and only if the following conditions are satisfied:
\begin{alignat*}{2}
 \ich{a} & \qquad &\lim_{i\to\infty} a_{ij} &= 0\qquad \textrm{for each}\; j. \\
 \ich{b} & \qquad & \sup_i \sum_{j=0}^\infty |a_{ij}| &< \infty \\
 \ich{c} & \qquad & \lim_{i\to\infty} \sum_{j=0}^\infty a_{ij} &= 1.
\end{alignat*}
The process of passing from
\(\{s_i\}\) to \(\{\sigma_i\}\) is called
\index{summability method}
a~\emph{summability method}. Two examples are
\begin{eqnarray*}
a_{ij} &=&
   \left\{\begin{array}{ll}
          \frac{1}{i+1} & \quad \textrm{if}\; 0\leq j \leq i. \\
          0             & \quad \textrm{if}\; i < j,
          \end{array}\right. \\
\textrm{and}\qquad a_{ij} &=& (1-r_i)r_i^j, \qquad  0<r_i<1,\quad r_i\to 1.
\end{eqnarray*}
Prove that each of these also transforms some divergent sequence \(\{s_i\}\)
(even some unbounded ones) to a convergent sequences \(\{\sigma_i\}\).
\end{excopy}

Following the above notations, we denote the transform as \(\sigma = A(s)\).

\paragraph{Summability implies limits.}
Assume $A$ transforms convergent sequences to convergent sequences.

Assume by negation \ich{a} does not hold for some $j$.
Pick the sequences \mset{s_j}, such that \(s_{i} = \delta_{ij}\).
Clearly \(\lim_{i\to\infty}s_i = 0\), 
but \(\sigma_i = a_{ij}\) which does not converge to $0$, 
and so by contradiction, \ich{a} holds.

We will now show \ich{b}. We first show that 
\begin{equation} \label{eq:ex5.15:finsum}
\sum_{j=0}^\infty |a_{rj}| < \infty.
\end{equation}
for each $r$. By negation, assume \eqref{eq:ex5.15:finsum} does not hold 
for some fixed $r$.
We will construct, in appending steps, 
a~sequence \mset{s_j} such that converges to zero but
\((A(s))_r = \infty\).
We define a sequence of blocks of indices such that 
each block if $A$'s $r$th row is ``not too small''.
Formally, let \(M_0 = 0\) and let \(M_k<\infty\) be the minimal integer
such that \[\sum_{j=M_{k-1}}^{M_k} |a_{rj}| > 1.\]
Clearly \mset{M_k} is an infinitely increasing sequnce.
For \(k\in\N\) let 
\[s_j = e^{i\theta_j}/k \qquad 
 \textrm{where}\;
 M_{k-1} \leq j < M_k
 \;\textrm{and}\;
 \theta_j = -\Arg(a_{rj}).\]
Note that \(\lim_{j\to\infty}s_j = 0\) and \(a_{rj}s_j \geq 0\).
Compute the $r$ component of \(A(s)\):
\begin{eqnarray*}
\sigma_r 
&=& \sum_{j=0}^\infty a_{rj}s_j 
 = \sum_{k=1}^\infty \sum_{j=M_{k-1}}^{M_k} a_{rj}s_j \\
&=& \sum_{k=1}^\infty 
      \left(\sum_{j=M_{k-1}}^{M_k} a_{rj}e^{i\theta_j}\right) \bigm/\,k 
 = \sum_{k=1}^\infty \left(\sum_{j=M_{k-1}}^{M_k} |a_{rj}|\right) \bigm/\,k \\
&\geq& \sum_{k=1}^\infty 1/k \\
&=& \infty.
\end{eqnarray*}
Thus \(\sigma=A(s)\) is not a valid sequence, (limit cannot be defined at all
and thus \eqref{eq:ex5.15:finsum} is true, and we can denote
\[S_r \eqdef \sum{j=0}^\infty |a_{rj}.\]


Now assume by negation \ich{b} does not hold.
We will again construct, in appending steps, 
a~sequence \mset{s_j} that will provide a contradiction.
We will also build increasing sequences, 
\mset{b_k} of column blocks, and \mset{r_k} of rows, such that
\begin{eqnarray}
\sum_{j=0}^{b_k-1} |a_{r_k j}| &\leq& 1 \label{eq:ex5.15:head} \\
\sum_{j=b_k}^{b_{k+1}-1} |a_{r_k j}| &\geq& 2^k \label{eq:ex5.15:mid} \\
\sum_{j=b_{k+1}}^\infty |a_{r_k j}| &\leq& 1 \label{eq:ex5.15:tail}
\end{eqnarray}
for each \(k\in\N\).
Let \(b_0=0\). 
Assume \mset{s_j} is defined for all \(j<b_{k'}\)
By \ich{a}, there exists some \(\rho\) such that 
\(|a_{rj}| < 1/(b_{k'}+1)\)
for all \(r\geq \rho\) and all \(j<b_{k'}\).
To ensure our next row pick, let \(U_k = \max_{r\leq\rho}S_r\).
By our negation hypothesis, there exists \(r_k\) such that 
\[S_{r_k} \geq \max(k^2+2,U_k).\] 
Clearly \(r_k>\rho\), and we can find \(b_{k+1}\) such that 
\[\sum_{j=b_{k+1}}^\infty |a_{rj}| < 1.\]
Now define \(s_j\) for \(b_k \leq j < b_{k+1}\) by
\[s_j = e^{i\theta_j}/k \qquad  \textrm{where}\; \theta_j = -\Arg(a_{r_k j}).\]
By induction we complete the definitions of \mset{s_j} and 
the supporting sequences \mset{b_k} and \mset{r_k}.
Clearly \(\lim_{j\to\infty}s_j = 0\), but
\begin{eqnarray*}
|\lim_{r\to\infty}\sigma_r|
&=& \lim_{r\to\infty} |\sigma_r| 
= \lim_{r\to\infty} \left| \sum_{j=0}^\infty a_{rj}s_j \right| 
= \lim_{k\to\infty} \left| \sum_{j=0}^\infty a_{r_k j}s_j \right| \\
&\geq& \lim_{k\to\infty} 
       \left(
          \left| \sum_{j=b_k}^{b_{k+1}-1} a_{r_k j}s_j \right| 
         - \left| \sum_{j=0}^{b_k-1} a_{r_k j}s_j \right| 
         - \left| \sum_{j=b_{k+1}}^\infty a_{r_k j}s_j \right|
       \right) \\
&=& \lim_{k\to\infty} 
       \left(
          \left| \sum_{j=b_k}^{b_{k+1}-1} |a_{r_k j}| \right| / k
         - \left| \sum_{j=0}^{b_k-1} a_{r_k j}s_j \right| 
         - \left| \sum_{j=b_{k+1}}^\infty a_{r_k j}s_j \right|
       \right) \\
&\geq& \lim_{k\to\infty} 
       \left(
          (k^2+2k) / k
         - \sum_{j=0}^{b_k-1} |a_{r_k j}|
         - \sum_{j=b_{k+1}}^\infty |a_{r_k j}|
       \right) \\
&\geq& \lim_{k\to\infty} k+2 - 1 - 1 \\
&=& \infty.
\end{eqnarray*}
This contradicts the summability, hence \ich{b} is true.

Consider the constant sequences \(s_j=1\) for all \(j\in\N\).
It has \(\lim_{j\to\infty} s_j = 1\) which implies
 \(\lim_{j\to\infty} \sigma_j = 1\) which is actually equivalent
to \ich{c}.

\paragraph{Limits implies summability.}
Conversely, assume conditions \ich{a}, \ich{b}, \ich{c} hold.

\emph{Constant:} If \(s_j=c\) for all \(j\in\N\), 
then by \ich{c} we get \(\lim_{j\to\infty}\sigma_j = c\).
\emph{Vanishing:} Assume \(\lim_{j\to\infty s_j} s_j = 0\).
Pick arbitrary \(\epsilon>0\). 
Let $J$ be such that \(|s_j| < \epsilon/3\) whenever \(j\geq J\).
Put \(h = \max_{1\leq j \leq J} |s_j| + 1\).
By \ich{a} there exists \(\rho_1\) such that \(|a_rj| < \epsilon/(Jh+1)\)
for all \(r\geq\rho_1\) and \(0\leq j \leq J\).
Denote \(T_r = \sum{j=0}^\infty a_{rj}\).
By \ich{c}, pick \(\rho_2\) such that \(|T_r - 1| < \epsilon\)
whenever \(r\geq \rho_2\). 
Note that in this case, 
\[
\left|\left(T_r - \sum_{j=0}^J a_{rj}\right) - 1\right| 
\leq |T_r - 1| + J\epsilon/(Jh+1)
< 2\epsilon.\]
Let \(\rho = \max(\rho_1,\rho_2)\), now for \(r\geq \rho\)
\begin{eqnarray*}
|\sigma_r|
&=& \left|\sum_{j=0}^\infty a_{rj}s_j\right| 
 = \left|\sum_{j=0}^J a_{rj}s_j
        + \sum_{j=J+1}^\infty a_{rj}s_j\right| \\
&\leq&  \left|\sum_{j=0}^J a_{rj}s_j\right| 
      + \left|\sum_{j=J+1}^\infty a_{rj}s_j\right| \\
&\leq&  h\left|\sum_{j=0}^J a_{rj}\right| 
      + \epsilon\left|\sum_{j=J+1}^\infty a_{rj}\right| \\
&\leq&  hJ\epsilon/(Jh+1) + \epsilon(1+2\epsilon) \\
&<& 2\epsilon(\epsilon+1).
\end{eqnarray*}
Hence \(\lim_{r\to\infty} \sigma_r = 0\).

Now for arbitrary converging sequence, let \(\lambda = \lim_{j\to\infty} s_j\).
By the above two case, we have
\begin{eqnarray*}
\lim_{r\to\infty} \sigma_r 
&=& \lim_{r\to\infty} \sum_{j=0^\infty} a_{rj}s_j \\
&=& \lim_{r\to\infty} \sum_{j=0^\infty} a_{rj}(s_j - \lambda + \lambda) \\
&=& \left(\lim_{r\to\infty} \sum_{j=0^\infty} a_{rj}(s_j - \lambda)\right) + 
    \left(\lim_{r\to\infty} \sum_{j=0^\infty} a_{rj} \lambda\right) \\
&=& 0 + \lambda.
\end{eqnarray*}
Thus $A$ satisfies the summability condition.

\paragraph{Specific transformations.}
When 
\begin{eqnarray*}
a_{rj} &=&
   \left\{\begin{array}{ll}
          \frac{1}{r+1} & \quad \textrm{if}\; 0\leq j \leq r. \\
          0             & \quad \textrm{if}\; r < j,
          \end{array}\right.
\end{eqnarray*}
we observe the sequence \mset{s_j} defined by 
% \(s_j = (-1)^j\), that is \((+1,-1,+1,-1,\ldots)\).
% that is \((+1,-1,+1,-1,\ldots)\).
\(s_j = (-1)^j\sqrt{\lfloor j/2\rfloor}\), 
that is \((0,0,+1,-1,+\sqrt{2},-\sqrt{2},\ldots)\).
Clearly does not converge and is unbounded, but since
we observe that 
\[\sum_{j=0}^r s_j 
  = \left\{\begin{array}{ll}
          0 & \qquad r = 1 \bmod 2 \\
          \sqrt{r/2} & \qquad r = 0 \bmod 2 
          \end{array}\right.\]
we see that 
\begin{eqnarray*}
\sigma_r 
&=& \sum_{j=0}^r a_{rj}s_j
 = \left(\sum_{j=0}^r s_j\right)/(r+1) \\
&=& \left\{\begin{array}{ll}
          0 & \qquad r = 1 \bmod 2 \\
          \sqrt{r}/(r+1) & \qquad r = 0 \bmod 2 
          \end{array}\right.
\end{eqnarray*}
Hence \(\lim_{r\to\infty}\sigma_r = 0\).

When \(a_{kj} = (1-r_k)r_k^j\) 
where \(0<r_k<1\) and \(\lim_{k\to\infty} r_k\to 1\),
we pick
\(s_j = (-1)^j\), that is \((+1,-1,+1,-1,\ldots)\).
that is \((+1,-1,+1,-1,\ldots)\).
Clearly does not converge. Compute
\[
\sigma_k
= \sum_{j=0}^r a_{kj}s_j
 =  \sum_{j=0}^r (1-r_k)r_k^j \cdot (-1)^j
 = (1-r_k)\sum_{j=0}^r (-r_k)^j 
 = (1-r_k) \cdot \bigl(1/(1-r_k)\bigr) = 1.\]
In particular,  \(\lim_{k\to\infty}\sigma_k=1\).



%%%%%%%%%%%%%% 16
\begin{excopy}
Suppose $X$ and $Y$ are Banach spaces, and suppose \(\Lambda\)
is a linear mapping of $X$ into $Y$, with the following property:
For every sequence \mset{x_n} in in $X$ for which \(x = \lim x_n\),
and \(y = \lim \Lambda x_n\) exist,
it is true that \(y=\Lambda x\). Prove that \(\Lambda\) is continuous.

This is so called ``closed graph theorem''
\emph{Hint}: Let \(X \oplus Y\) be the set of all ordered pairs
\((x,y)\), \(x\in X\) and \(y\in Y\), with addition and scalar multiplication
defined componentwise.
Prove that \(X\oplus Y\) is a Banach space,
if \(\|x,y)\| = \|x\| + \|y\|\).
The graph $G$ of \(\Lambda\) is a subset of \(X\oplus Y\)
formed by the pairs \((x,\Lambda x)\), \(x\in X\). Note that our hypothesis
says that $G$ is closed; hence $G$ is a banach space.
Note that \((x,\Lambda x) \to x\) is continuous, one-to-one,
and linear and maps $G$ onto $X$.

Observe that there exist \emph{nonlinear} mappings
(of \(\R^1\) onto \(\R^1\), for instance)
whose graph is closed although they are
not continuous: \(f(x)=1/x\) if \(x=0\), \(f(0)=0\).
\end{excopy}

Define projections: 
\begin{alignat*}{2}
p_1 & : G \to X & \qquad p_1(x,\Lambda x) &= x \\
p_2 & : (X,Y) \to X & \qquad p_2(x,y) &= y
\end{alignat*}
These projections are linear and continuous, and \(p_1\) is also one-to-one.
By the open mapping theorem~5.9 (\cite{RudinRCA87}), \(p_1^{-1}\) 
is continuous. But \(\Lambda = p_2 \circ p_1^{-1}\) and thus is continuous.

%%%%%%%%%%%%%% 17
\begin{excopy}
If \(\mu\) is a positive measure, each \(f\in L^\infty(\mu)\) defines
a  multiplication operator \(M_f\)
on \(L^2(\mu)\) into \(L^2(\mu)\) such that \(M_f(g) = fg\).
Prove that \(\M_f\|\leq \|f\|_\infty\).
For which measures \(\mu\) is it true that \(\|M_f\| = \|f\|_\infty\)
for all \(f\in L^\infty(\mu)\)?
For which measures \(f\in L^\infty(\mu)\) does \(M_f\) map
\(L^2(\mu)\) onto \(L^2(\mu)\)?
\end{excopy}

Let \(f\in L^\infty(\mu)\) and \(g \in L^2(\mu)\).
Now
\[ \|M_f(g)\|_2 
   = \left(\int |fg|^2\,d\mu\right)^{1/2}
   \leq \left(\int (\|f\|_\infty |g|)^2\,d\mu\right)^{1/2}
   \leq \|f\|_\infty \left(\int |g|^2\,d\mu\right)^{1/2}.
\]
Hence \(\|M_f\| \leq \|f\|_\infty\).

Now pick some \(f\in L^\infty(X,\mu)\) and an arbitrary \(\epsilon>0\).
By definition, the set
\[ U \eqdef \{x\in X: |f(x)|>\|f\|_\infty - \epsilon\}\]
satisfies \(\mu(U)>0\). Assume that for any such set, 
there exists \(W\subset U\) such that \(\mu(W)<\infty\).
Now we consider \(g=\chi_W\).
Clearly \(\|g\|_2^2 = \mu(W)\). Also
 we see that 
\[
\|M_f(g)\|_2^2
= \int |fg|^2\,d\mu
= \int_W f^2\,d\mu
\geq (\|f\|_\infty - \epsilon)^2 \mu(W).\]
Hence \(\|M_f\| \geq \|f\|_\infty - \epsilon\)
and the equality \(\|M_f\| = \|f\|_\infty\) is established.

Finally, if \(1/f \in L^\infty(\mu)\), then 
\(M_f\) is invertible and \((M_f)^{-1} = M_{1/f}\).
In particular, in this case, \(M_f\) is onto.

%%%%%%%%%%%%%% 18
\begin{excopy}
Suppose \mset{\Lambda_n} is a sequence of bounded linear transformations
from a normed linear space $X$ to a Banach space $Y$,
suppose \(\|\Lambda_n\| \leq M <\infty\) for all $n$, and suppose
there is a dense set \(E\subset X\) such that
\mset{\Lambda_n x} converges for each \(x\in E\).
Prove that \mset{\Lambda_n x} converges for each \(x\in X\).
\end{excopy}

We will show that \mset{\Lambda_n x} is a Cauchy sequence.
Pick some arbitrary \(\epsilon>0\). Pick some \(x'\in E\) such that
\(\|x-x'\|<\epsilon/(3M)\).
Since \mset{\Lambda_n x'} is a Cauchy sequence, there exists
some $N$ such that \(\|\Lambda_m x' - \Lambda_n x'\| < \epsilon/3\)
whenever \(m,n>N\). But also
\begin{eqnarray*}
\|\Lambda_m x - \Lambda_n x\|
&\leq&
 \|\Lambda_m x - \Lambda_m x'\|
 + \|\Lambda_m x' - \Lambda_n x'\|
 + \|\Lambda_n x' - \Lambda_n x\| \\
&\leq& (\|\Lambda_m\| + \|\Lambda_n\|)\cdot\|x-x'\| 
      + \|\Lambda_m x' - \Lambda_n x'\| \\
&\leq& 2M\cdot\epsilon/(3M) + \epsilon/3 = \epsilon.
\end{eqnarray*}
Thus, \mset{\Lambda_n x} is a Cauchy sequence, 
since $Y$ is a Banach space this sequence converges.


%%%%%%%%%%%%%% 19
\begin{excopy}
If \(s_n\) is the $n$th partial sum of the Fourier series of a function
\(f\in C(T)\), prove that \(s_n/\log n \to 0\)
uniformly, as \(n\to \infty\), for each \(f\in C(T)\). That is prove that
\[ \lim_{x\to\infty} \frac{\|s\|_\infty}{\log n} = 0. \]

On the other hand, if \(\lambda_n/\log n \to 0\) prove that there exists an
\(f\in C(T)\) such that the sequence \(\{s_n(f;0)/\lambda_n\}\) is unbounded.
\emph{Hint}: Apply the reasoning of Exercise~18 and that of Sec.~5.11,
with a better estimate of \(\|D_n\|_1\), than used there.
\end{excopy}

We first prove the following Lemmas
(See also Theorems~I-8-1 and II-8-13 in \cite{Zyg:2002}).

\begin{llem} \label{llem:fog:ifoig}
Let $f$,\ and $g$ be integrable functions on each 
subinterval \([a,b']\) such that \(a\leq b' < b\).
Let
\begin{equation*}
F(x) = \int_a^x f(t)\,dt
\qquad
G(x) = \int_a^x g(t)\,dt.
\end{equation*}
Assume that \(g(x)\geq 0\) and that \(\lim_{x\to b} G(x) = \infty\).
If 
\begin{equation*}
\lim_{x\to b} f(x)/g(x) = 0
\end{equation*}
then
\begin{equation*}
\lim_{x\to b} F(x)/G(x) = 0
\end{equation*}
\end{llem}

In $o$-notation: the theorem says:
If \(f(x) = o(g(x))\)
then \(F(x) = o(G(x))\).

\begin{thmproof}
Pick arbitrary \(\epsilon>0\).
Let \(x_0\in[a,b)\) such that 
\(|f(x)/g(x)| < \epsilon/2\) for \(x_0 <x < b\).
Pick \(x_1 \in (x_0,b)\) such that for all \(x\geq x_1\) we have
\begin{equation*}
G(x) > 2\int_a^{X_0} |f(t)|\,dt \bigm/ \epsilon.
\end{equation*}

Thus for \(x\geq x_1\)
\begin{equation*}
|F(x)|
\leq 
   \int_a^{x_0} f(t)\,dt 
 + \int_{x_0}^x f(t)\,dt 
\leq 
   \int_a^{x_0} f(t)\,dt 
 + \epsilon G(x)/2
\leq \epsilon G(x).
\end{equation*}
\end{thmproof}

Clearly similar result and proof holds with 
reversed directions of $a$ and $b$.

The following lemma shows that for continuous functions 
\(\|s_n(f)\|_\infty = o(\log n)\).
\begin{llem} \label{llem:ex:5.19a}
Let \(g\in C(T)\) and \(s_n\) be the $n$-Fourier sum of $g$.
Then 
\begin{equation} \label{eq:ex:5.19}
\lim_{n\to\infty} s_n(f,x) / \log n = 0.
\end{equation}
\end{llem}
\begin{thmproof}
We freely identify $T$ with \(\{x: -\pi \leq x < \pi\}\).
Let \(D_n\) be the Dirichlet's kernel, 
that is \[s_n(f,t) = (D_n * f)(t).\]

Pick arbitrary \(\epsilon>0\).
Since $f$ is uniformly continuous (on $T$), 
there exists \(\delta > 0\)
such that 
\begin{equation} \label{eq:ex5:19:fcont}
|f(x-t) - f(x+t)| < \epsilon
\end{equation}
for all \(t < \delta\) and \(x\in T\).

If \(|t|\leq \pi/2\) then \(2t/\pi \leq \sin t\)
and also
\begin{equation} \label{eq:ex5:19:Dnltt}
|D_n(t)| 
= \left|\frac{\sin\bigl((n+1/2)t\bigr)}{\sin(t/2)}\right|
\leq 1 \bigm/ (\pi t).
\end{equation}

Combining \eqref{eq:ex5:19:fcont} and \eqref{eq:ex5:19:Dnltt}
we get that
\begin{equation*}
\lim_{t\to 0} \bigl(f(x-t) - f(x+t) \bigr) D_n(t)  \bigm/\, (1/t) = 0
\end{equation*}
uniformly for all \(x\in T\).
Using 
\[
\int_{\pi/n}^{\pi} \frac{1}{t}\,dt = \log(\pi) - \log(\pi/n) = \log n.
\]
with local lemma~\ref{llem:fog:ifoig}, we have:
\begin{equation} \label{eq:ex5:19:fDn:olog}
\lim_{n\to \infty} 
 \int_{\pi/n}^{\pi} \bigl(f(x-t) - f(x+t) \bigr) D_n(t)\,dt
   \bigm/\, \log n = 0.
\end{equation}
again, uniformly for all \(x\in T\).

We now estimate \(|s_n(f,x)|\) by integrating two domains.
\begin{eqnarray*}
2\pi |s_n(f,x)|
&=& \left| \int_{-\pi}^{\pi} f(x-t)D_n(t)\,dt \right| \\
&=& 
     \left|\int_0^{\pi/n} \bigl(f(x-t) - f(x+t)\bigr)D_n(t)\,dt\right| 
   + 
     \left|\int_{\pi/n}^{\pi} \bigl(f(x-t) - f(x+t)\bigr)D_n(t)\,dt\right| \\
&\leq&
     (\pi/n) \|f\|_\infty (2n+1)
   + \left|\int_{\pi/n}^{\pi} \bigl(f(x-t) - f(x+t)\bigr)D_n(t)\,dt\right| 
\end{eqnarray*}

Using the fact that \(\lim_{n\to\infty} (2n+1) / (n\log n) = 0\)
and \eqref{eq:ex5:19:fDn:olog} we get the desired \eqref{eq:ex:5.19}.
\end{thmproof}

The following may be viewed as a converse to local lemma~\ref{llem:ex:5.19a}.
It shows that \(\log n\) is the best estimate order for \(\|s_n\|\).
\begin{llem} 
Let \(g\in C(T)\) and \(s_n\) be the $n$-Fourier sum of $g$.
Then 
If \((\lambda_n)_{n\in\N}\) is a sequence of positive numbers
such that \(\lim_{n\to\infty}\lambda_n/\log n = 0\),
then there exists \(f\in C(T)\) such that
\(\{s_n(f;0)/\lambda_n\}_{n\in\N}\) is unbounded.
\end{llem}

\begin{thmproof}
% we may trivially ignore occurrences of \(\lambda_n = 0\).
By Secation 5.11 (\cite{RudinRCA87}) if \(\Lambda_n f = s_n(f;0)\)
then
\begin{equation*}
\| \Lambda_n\| = \|D_n\|_1 
 > \frac{4}{\pi}\sum_{k=1}^\infty \frac{1}{k} = c \log n
\end{equation*}
for some constant \(c>0\).
Hence
\begin{equation*}
\|\Lambda_n\|_\infty /\lambda_n > c \log n / \lambda_n
\end{equation*}
and so 
\begin{equation*}
\lim_{n\to\infty} \|\Lambda_n\|_\infty /\lambda_n 
\geq \lim_{n\to\infty} c \log n / \lambda_n = \infty.
\end{equation*}
and by the uniform bounded principle (theorem~5.8 \cite{RudinRCA87})
there must exists \(f\in C(T)\) such that the sequence
\(\{s_n(f;0)/\lambda_n\}_{n\in\N}\) is unbounded.
\end{thmproof}


\begin{excopy}
{\small [Appears in and refers to second edition].}\newline
Is the lemma of Sec.~4.15 valid on every Banach space?
In every normed linear space?
\end{excopy}

The lemma says:
\begin{quote}
\textsl{
If $V$ is a closed subspace of a Hilbert space $H$,
\(y\in H\),
\(y\notin V\),
and \(V^{*}\) is the space spanned by $V$ and $y$, then 
\(V^{*}\) is closed.
}
\end{quote}
% \newline

The proof is it is in the (old edition) text applies for Banach spaces
as it is. Now we will prove thje following lemma using 
Hahn Banach
\index{Hahn Banach}
theorem~5. \cite{}.

\begin{llem}
If $V$ is a closed subspace of a normed linear space $L$,
\(y\in L\)
and \(V^{*}\) is the space spanned by $V$ and $y$, then 
\(V^{*}\) is closed.
\end{llem}

\begin{thmproof}
We may assume \(y\notin L\) since otherwise the result is trivial.
Now let's define a functional on \(V^{*}\) by
\[ f(v + \lambda y) = \lambda \qquad v\in V,\; \lambda \in\C.\]
By Hahn-Banach theorem we can extend $f$ to all $N$.
Assume \(w\in \overline{V^*}\).
By having a norm, there exists a sequence \((v_n + \lambda_n y)_{n\in\N}\)
such that 
\[\lim_{n\to\infty} v_n + \lambda_n y = w.\]
But then 
\[f(w) = \lim_{n\to\infty} f(v_n + \lambda_n y) 
       = \lim_{n\to\infty} \lambda_n.\] 
Thus \(w = f(y)y + \lim_{n\to\infty} v_n\) and 
% so \(w - f(y)y = \lim_{n\to\infty} v_n\).
since $V$ is closed,
\(w - f(y)y\in V\) and so \(w\in V^*\) and we have shown that \(V^*\) is 
closed.
\end{thmproof}




\end{enumerate}

\nobreak
\begin{enumerate}

\setcounter{enumi}{19}

%%%%%%%%%%%%%% 20
\begin{excopy}
\begin{itemize}

\itemch{a}
Does there exist a sequence of continuous positive functions \(f_n\)
on \(\R^1\) such that \mset{f_n(x)} is unbounded if and only if $x$ is
rational?

\itemch{b}
Replace ``rational'' by irrational in \ich{a} and answer the resulting
question.

\itemch{c}
Replace ``\mset{f_n(x)} is unbounded''
by ``\(f_n(x)\to \infty\) as \(n\to\infty\)''
and answer the resulting analogues of \ich{a} and \ich{b}.
\end{itemize}
\end{excopy}

It is easy to see that for all cases, the existence 
is the same if positive functions are defined on \([0,1]\).
Simply by restricting functions, or uniting them with 
appropriate shifting to maintain continuity.

\begin{itemize}

%%%%%%%%%%
\itemch{a}

Define 
\[G_{m,n} \eqdef \{x\in\R: f_n(x)>m\}.\]

By definition \(\lim_{n\to\infty} f_n(x) = \infty\) iff
\[\forall M\exists N \forall n\geq N\;f_n(x)>M.\] 
Hence the set consists exactly of such points $x$ is
\[L \eqdef \bigcap_M \bigcup_N \bigcap_{n>N} G_{M,n}.\]

By definition \(\limsup_{n\to\infty} f_n(x) = \infty\) iff
\[\forall M\forall N \exists n\geq N\;f_n(x)>M.\]
Hence the set consists exactly of such points $x$ is
\[U \eqdef \bigcap_M \bigcap_N \bigcup_{n>N} G_{M,n}.\]

Thus the subset of \(\R^1\) for which  \mset{f_n(x)} is unbounded
is a \(G_\delta\) set. It cannot be \Q\ since otherwise,
by local lemmas~\ref{lem:count:1cat} and~\ref{lem:gdel:2cat},
\Q\ would be of second category.

%%%%%%%%%%
\itemch{b}

Following \cite{Myerson:1991:FCF}, we will show such sequence.
Interestingly, there it cites \cite{Gelb1996}, Chapter~7 Example~4.

Let \(\{q_j\}_{j\in\N}\) be an enumeration
of all the rationals in \((0,1)\).
Define \(f_n\) as a periodic function of period $1$,
thus we need to define it on \([0,1]\).
Let \(P_n = \{0,1\} \cup \{q_j: j\leq n\}\). 
We firrst define \(f_n\) on \(P_n\).
as follows in \([0,1]\) as follows.
\begin{align*}
f_n(0) = f_n(1) &= 0 \\
\forall j\leq n\quad f_n(q_j) &= j.
\end{align*}

The set \(P_n\) partitions \([0,1]\)
into \(n+1\) sub-segmnets, where we define \(f_n\) to be linear.
It is easy to see that \(f_n\) are continuous, monotonically increasing
and that for all \(x\in\R\)
\begin{equation*}
\lim_{n\to\infty} f_n(x) = 
\left\{\begin{array}{ll}
       j       & x - \lfloor x \rfloor = q_j \\
       \infty  &   x \in \R\setminus \Q
       \end{array}\right.
\end{equation*}



%%%%%%%%%%
\itemch{c}

The analogue for \ich{a} is \emph{false}.
Assume by negation there exists a sequence \(\{f_n\}_{n\in\N}\)
of continuous functions such that 
\begin{equation*}
L \eqdef \{x\in\R: \lim_{n\to\infty} f_n(x) = \infty\} = \Q.
\end{equation*}
Let \(\{q_n\}_{n\in\N}\) be an enumeration of \Q.
We will define a decreasing sequence of closed intervals
\(\{I_n\}_{n\in\N}\) such that 
for all \(n\in\N\)
the following hold:
\begin{align}
m(I_n) &> 0  \label{eq:ex:5.20:c0} \\
I_n &\supset I_{n+1} \label{eq:ex:5.20:c1} \\
q_n &\notin I_n \label{eq:ex:5.20:c2} \\
\forall x \in I_n,\; f_n(x) &\geq n \label{eq:ex:5.20:c3}
\end{align}
Since \(\lim_{n\to\infty} f_n(1/2)=\infty\), 
there exists some \(N_1\) such that 
\(f_n(1/2) > 2\) for all \(n\geq N_1\).
By taking a sufficiently small neighborhood \(V_1\) of \(1/2\)
and picking a closed sub-interval \(I_1 \subset V_1 \setminus \{q_1\}\).
We can ensure that 
\eqref{eq:ex:5.20:c0},
\eqref{eq:ex:5.20:c2} and
\eqref{eq:ex:5.20:c3} hold.

By induction assume that 
\(\{I_j\}_{j=1}^k\) were picked and that
the above 
\eqref{eq:ex:5.20:c0},
\eqref{eq:ex:5.20:c1},
\eqref{eq:ex:5.20:c2} and
\eqref{eq:ex:5.20:c3} hold for \(n< k\).
Since \(m(I_{k-1})>0\) we can
pick some rational \(\alpha \in I_{k-1} \cap \Q\).
Since 
Since \(\lim_{n\to\infty} f_n(\alpha)=\infty\), 
we can find some \(N_k\) such that 
\(f_n(\alpha) > k+1\) for all \(n\geq N_k\).
By taking a sufficiently small neighborhood \(V_k\subset I_{k-1}\) 
of \(\alpha\)
and picking a closed sub-interval \(I_k\subset V_k \setminus \{q_k\}\)
We can ensure that the above 
\eqref{eq:ex:5.20:c0},
\eqref{eq:ex:5.20:c1},
\eqref{eq:ex:5.20:c2} and
\eqref{eq:ex:5.20:c3} hold for \(n=k\) as well.

Since \(\cup_{n\in\N} I_n \neq \emptyset\) 
There exists \(c \in \cup_{n\in\N}\)
and \(c\notin \Q\) by \eqref{eq:ex:5.20:c2}.
Clearly \(\lim_{n\to\infty} f_n(c) = \infty\)
Which gives the contradiction \(c\in L=\Q\).


The analogue for \ich{b} is \emph{true}, since the example shown
for \ich{b} holds for here as well, since
that sequence converges everywhere, either for a real number
or infinity.

\end{itemize}

%%%%%%%%%%%%%% 21
\begin{excopy}
Suppose \(E\subset \R^1\) is measurable, and \(m(E)=0\).
Must there be a translate \(E+x\) of $E$ that does not intersect $E$?
Must there be a homeomorphism $H$ of \(\R^1\) onto \(\R^1\) so that
 \(h(E)\) does not intersect $E$?
\end{excopy}

The answer is no. It is sufficiently to defy the second conjecture.

We use exercise~2 in chapter~2 of \cite{RudinFA79} 
(althogh later in the order of Rudin's text books, but the exercise
does not require further knowledge than we already have).
Let \(I \eqdef [0,1] = E_0\disjunion F_0\) 
be a disjoint union of the unit segment
such that \(m(E_0)=0\) and \(F_0\) is of first category.

Using the translation notation \(E+a = \{x+a: x\in E\}\), define
\[
E = \bigcup_{n\in\Z} E_0 + n \qquad
F = \bigcup_{n\in\Z} F_0 + n \qquad.
\]
Clearly \(\R = E \disjunion F\) where 
such that \(m(E)=0\) and $F$ is of first category.

Assume by negation there is homeomorphism \(T:\R\to\R\) 
such that \(E\cap T(E) = \emptyset\).
But then \(I = F \cup T(F)\) contradiction to the fact that
the unit segment is of second category.


%%%%%%%%%%%%%% 22
\begin{excopy}
Suppose \(f\in C(T)\) and
\index{Lip@\(\Lip\)}
\(f\in \Lip\alpha\) for some \(\alpha > 0\). (See  Exercise~11.)
Prove that the Fourier series of $f$ converges to \(f(x)\),
by completing the following outline:
It is enough to consider the case \(x=0\),
\(f(0)=0\). The difference between the partial sums \(s_n(f;0)\)
and the integrals
\[ \frac{1}{\pi} \int_{-\pi}^\pi f(t)\frac{\sin nt}{t}\,dt \]
tends to $0$ as \(n\to \infty\).
The functions \(f(t)/t\) is in \(L^1(T))\).
\index{Riemann-Lebesgue lemma}
Apply the Riemann-Lebesgue lemma. More careful reasoning shows that the
convergence is actually uniform on $T$.
\end{excopy}

Define \(f_\tau(t) = f(t + \tau)\).
Now 
\begin{align*}
2\pi s_n(f;x)
&= \int_{\pi}^\pi f(t)D_n(x - t)\,dt
 = \int_{\pi}^\pi f(t+x)D_n(x - t + x)\,dt
 = \int_{\pi}^\pi f_x(t)D_n(-t)\,dt \\
& = 2\pi s_n(f_x;0).
\end{align*}
Clearly \(f\in\Lip_\alpha\) iff  \(f_x\in\Lip_\alpha\).
Thus it is sufficient to to show that \(\lim_{n\to\infty} s_n(f;0) = f(0)\).
By looking at \(g(t) = f(t) - f(0)\) we see that 
\(f\in\Lip_\alpha\) iff  \(g\in\Lip_\alpha\)
and \(s_n(f)\) differs from \(s_n(g)\) by the constant \(\hat{f}(0)\).
Thus we may also assume \(f(0)=0\).

Let \(E_n(t) = 2\sin(t)/t\)
and \(H_n(t) = D_n(t) - E_n(t)\).
Applying L'Hospital rules we have
\begin{align*}
\lim_{t\to 0} D_n(t) &= \lim_{t\to 0} \sin((n+1/2)t)/\sin(t/2) = 2n+1\\
\lim_{t\to 0} E_n(t) &= \lim_{t\to 0} 2\sin(nt)/t = 2n\\
\lim_{t\to 0} H_n(t) &= 1
\end{align*}
and also for \(G(t) = 1/\sin t - 1/t\)
\begin{equation*}
\lim_{t\to 0} G(t) % \frac{1}{\sin t} - \frac{1}{t}
= \lim_{t\to 0} \frac{t - \sin t}{t\sin t}
= \lim_{t\to 0} \frac{1 - \cos t}{\sin t + t\cos t} 
= \lim_{t\to 0} \frac{\sin t}{\cos t + \cos t - t\sin t} 
= \frac{0}{1+1+0} = 0.
\end{equation*}
Thus if we define \(G(0)=0\) then $G$ is continuous,
and we have \(U = \sup_{t\in[-\pi,\pi]} |G(t)|+1 < \infty\).
Now  for \(t\in [-\pi/2,\pi/2]\) we can estimate 
\begin{align}
|H_n(2t)|
&\leq \left|\frac{\sin\bigl((2n+1)t\bigr)}{\sin t} 
      - \frac{\sin(2nt)}{t}\right| \notag \\
&\leq \left|\frac{\sin\bigl((2n+1)t\bigr) - \sin(2nt)}{\sin t}\right|
      + \left| \sin(2nt) \left( \frac{1}{\sin t} - \frac{1}{t}\right) \right|
     \notag \\
&\leq \left| \frac{\sin t}{\sin t}\right| 
     + 1\cdot U = U + 2. \label{eq:5.22:sinn1t}
\end{align}
In \eqref{eq:5.22:sinn1t} we used the inequality
\begin{equation*}
\sin\bigl((2n+1)t\bigr) - \sin(2nt) \leq |\sin t|
\end{equation*}
for \(-\pi/2\leq t \leq \pi/2\) derived from
\begin{equation*}
\sin\bigl((2n+1)t\bigr) = \sin(2nt)\cos t + \cos(2nt)\sin t.
\end{equation*}
Note that the bound of \(H_n\) is independent of $n$.

Put \(M=\|f\|_\infty+1\).
Pick arbitrary \(\epsilon>0\), we may assume \(\epsilon < 1\).
Find \(\eta>0\)
such that 
\begin{itemize}
\item \(\eta < \epsilon\).
\item \(|f(t)| < \epsilon/MU\) for all \(t\in[-\eta,+\eta]\).
\end{itemize}

Put 
\(f_d(t) = f(t)/\sin(t/2)\)
and
\(f_e(t) = f(t)/(t/2)\).
Since the three functions in
\(\calF = \{f, f_d, f_e\}\)
are uniformly continuous in \(\T\setminus(-\eta,\eta)\), 
let \(\delta > 0\) be such that \(|g(t_1) - g(t_0)| < \eta\epsilon < \epsilon\)
for \(g\in \calF\)
whenever \(|t_1 - t_0| < \delta\). we may assume \(\delta<\eta\).

Pick \(n_0\in\N\) such that \(2\pi/n_0 < \delta)/MU\) and
take arbitrary \(n\geq n_0\).

Now
\begin{align}
\Delta_n 
&= \left|s_n(f;0) - \frac{1}{\pi}\int_{-\pi}^\pi f(t)\sin nt/t\,dt \right| 
   \notag \\
&= \left|\frac{1}{2\pi} \int_{-\pi}^\pi f(t)H_n(t)\,dt \right| 
 = 
   \frac{1}{2\pi}
   \left|
     \int_{-\eta}^\eta \cdots + \int_{\T\setminus[-\eta,+\eta]} \cdots
   \right| \notag \\
&\leq 
     \frac{1}{2\pi}
     \left(
         2MU\epsilon
       + \left|\int_{-\pi}^{-\eta} f(t)H_n(t),dt\right|
       + \left|\int_\eta^\pi f(t)H_n(t),dt\right|
     \right)  \label{eq:5.22:2int}
\end{align}

We will estimate the last two terms can also be as small as desired.
We will workout the last ($t$-positive) one. % , the other is similar.
For abbreviation, put \(\nu = (n+1/2)\).
The periods of 
\(\sin(\nu t)\) and \(\sin(nt)\) 
are \(\gamma_d=2\pi/\nu = 4\pi/(2n+1)\)
and \(\gamma_e=2\pi/n)\) respectably.
We will separate the integration segments to whole periods as available.
For \(\iota=d,e\) (as symbols), 
find the minimal \(l_\iota\) and maximal \(h_\iota\) such that 
\(\eta\leq l_\iota \gamma_\iota\) and \(h_\iota\gamma_\iota \leq \pi\).
The ``gaps'' size are
\begin{align*}
(l_d\gamma_d - \eta ) + (\pi - h_d\gamma_d) < 2\gamma_d &= 2\pi/(2n+1)\\
(l_e\gamma_e - \eta ) + (\pi - h_e\gamma_e) < 2\gamma_e &= 2\pi/n.
\end{align*}
Now
\begin{eqnarray*}
\left|\int_\eta^\pi f(t)H_n(t),dt\right|
&\leq&
    \left|\int_\eta^{\max(l_d\gamma_d,l_e\gamma_e)} f(t)H_n(t)\,dt\right| \\
& &  + \left|\sum_{k=l_d}^{h_d-1} \int_{k\gamma_d}^{(k+1)\gamma_d} 
           f_d(t)\sin(\nu t)\,dt\right| 
     + \left|\sum_{k=l_e}^{h_e-1} \int_{k\gamma_e}^{(k+1)\gamma_e}
           f_e(t)\sin(nt)\,dt\right| \\
& &  + \left|\int_{\min(l_d\gamma_d,l_e\gamma_e)}^\pi f(t)H_n(t)\,dt\right| \\
&\leq& 2\pi MU/n \\
& &  + \left|\sum_{k=l_d}^{h_d-1} \int_0^{\gamma_d}
           f_d(t+k\gamma_d)\sin\bigl(\nu (t+k\gamma_d)\bigr)\,dt\right| 
\\ & &
     + \left|\sum_{k=l_e}^{h_e-1} \int_0^{\gamma_e}
           f_e(t+k\gamma_e)\sin\bigl(n(t+k\gamma_e)\bigr)\,dt\right| \\
& &  + 2\pi MU/n \\
&\leq& 4\pi MU/n \\
& &  + \left|\sum_{k=l_d}^{h_d-1} \int_0^{\gamma_d/2}
           \bigl(f_d(t+k\gamma_d) - f_d(t+k\gamma_d + \gamma_d/2)\bigr)
           \sin\bigl(\nu (t+k\gamma_d)\bigr)\,dt\right| \\
& &  + \left|\sum_{k=l_e}^{h_e-1} \int_0^{\gamma_e/2}
           \bigl(f_e(t+k\gamma_e) - f_e(t+k\gamma_e + \gamma_e/2) \bigr)
           \sin\bigl(n(t+k\gamma_e)\bigr)\,dt\right| \\
&\leq& \epsilon
       + (h_d - l_d)\gamma_d\epsilon/2
       + (h_e - l_e)\gamma_e\epsilon/2. \\
&\leq& (1+\pi/2+\pi/2)\epsilon.
\end{eqnarray*}

Similar estimation can be done to 
\begin{equation*}
\left|\int_{-\pi}^{-\eta} f(t)H_n(t),dt\right|.
\end{equation*}
Thus the last two terms in \eqref{eq:5.22:2int} can be arbitrarily small,
and so 
\begin{equation*}
\lim_{n\to\infty}
  \left|s_n(f;0) - \frac{1}{\pi}\int_{-\pi}^\pi f(t)\sin nt/t\,dt \right| = 0.
\end{equation*}

Therefore, in order to see that \(\lim_{n\to\infty} s_n(f;0)=0\), 
it is sufficient to show that 
\begin{equation} \label{eq:5.22.suff}
\lim_{t\to 0} \int_{-\pi}^\pi f(t)\sin nt/t\,dt = 0.
\end{equation}

{\small
Note that it is \emph{not} necessarily that \(\lim_{t\to 0}f(t)/t = 0\).
}

Let $K$ be such that \(|f(x)-f(y)|/|x-y|^\alpha\) for all \(x\neq y\).
Then 
\begin{equation*}
|f(t)/t| 
= |f(t)| |t|^{\alpha-1}/|t|^\alpha
\leq K |t|^{\alpha-1}
\end{equation*}
for \(t\neq 0\). Thus
\begin{equation*}
\|f/t\|_1
= \frac{1}{2\pi} \int_{-\pi}^\pi |f(t)/t|\,dt 
\leq \frac{1}{2\pi} \int_{-\pi}^\pi K|t|^{\alpha-1}\,dt 
% = \frac{K}{\alpha\pi}\pi^\alpha.
= K\pi^{\alpha-1}/\alpha < \infty
\end{equation*}
which shows that \(f_e = f(t)/t \in L^1(\T)\).
Now
\begin{equation*}
\int_{-\pi}^\pi f(t)\sin nt/t\,dt 
= (2\pi/2i)\bigl(\hat{f_e}(n) - \hat{f_e}(n)\bigr).
\end{equation*}
By the Riemann-Lebesgue lemma \(\lim_{n\to\infty} \hat{f_e}(n) = 0\).
Hence \eqref{eq:5.22.suff} holds.

All the above bounds and minimal (\(\delta\), \(\eta\)) values
could have been taken for any value other than \(t=0\).
So we could have the limit converge uniformly
also for \(f_{\tau}(t) = f(t+\tau) - f(\tau)\).


\iffalse
Define \(f_\tau(t) = f(t + \tau)\).
Now 
\begin{align*}
2\pi s_n(f;x)
&= \int_{\pi}^\pi f(t)D_n(x - t)\,dt
 = \int_{\pi}^\pi f(t+x)D_n(x - t + x)\,dt
 = \int_{\pi}^\pi f_x(t)D_n(-t)\,dt \\
& = 2\pi s_n(f_x;0).
\end{align*}
Clearly \(f\in\Lip_\alpha\) iff  \(f_x\in\Lip_\alpha\).
Thus it is sufficient to to show that \(\lim_{n\to\infty} s_n(f;0) = f(0)\).
By looking at \(g(t) = f(t) - f(0)\) we see that 
similarly 
Clearly \(f\in\Lip_\alpha\) iff  \(g\in\Lip_\alpha\)
and \(s_n(f)\) differs from \(s_n(g)\) by the constant \(\hat{f}(0)\).
Thus we may also assume \(f(0)=0\).

{\small
Note that it is \emph{not} necessarily that \(\lim_{t\to 0}f(t)/t = 0\).
}

Let $K$ be such that \(|f(x)-f(y)|/|x-y|^\alpha\) for all \(x\neq y\).
Then 
\begin{equation*}
|f(t)/t| 
= |f(t)| |t|^{\alpha-1}/|t|^\alpha
\leq K |t|^{\alpha-1}
\end{equation*}
fot \(t\neq 0\). Thus
\begin{equation*}
\|f/t\|_1
= \frac{1}{2\pi} \int_{-\pi}^\pi |f(t)/t|\,dt 
\leq \frac{1}{2\pi} \int_{-\pi}^\pi K|t|^{\alpha-1}\,dt 
% = \frac{K}{\alpha\pi}\pi^\alpha.
= K\pi^{\alpha-1}/\alpha < \infty
\end{equation*}
which shows that \(f(t)/t \in L^1(\T)\).
\fi


%%%%%%%%%%%%%%%%%
\end{enumerate}

 % -*- latex -*-
% $Id: rudinrca6.tex,v 1.5 2008/07/19 08:56:55 yotam Exp $


%%%%%%%%%%%%%%%%%%%%%%%%%%%%%%%%%%%%%%%%%%%%%%%%%%%%%%%%%%%%%%%%%%%%%%%%
%%%%%%%%%%%%%%%%%%%%%%%%%%%%%%%%%%%%%%%%%%%%%%%%%%%%%%%%%%%%%%%%%%%%%%%%
%%%%%%%%%%%%%%%%%%%%%%%%%%%%%%%%%%%%%%%%%%%%%%%%%%%%%%%%%%%%%%%%%%%%%%%%
\chapterTypeout{Complex Measures} % 6

%%%%%%%%%%%%%%%%%%%%%%%%%%%%%%%%%%%%%%%%%%%%%%%%%%%%%%%%%%%%%%%%%%%%%%%%
%%%%%%%%%%%%%%%%%%%%%%%%%%%%%%%%%%%%%%%%%%%%%%%%%%%%%%%%%%%%%%%%%%%%%%%%
\section{Notes}

Intuitively the following lemma shows that \(f\,d\mu\) is regular.
\begin{llem} \label{llem:fdmu:abscont}
Let $f$ be a \(\mu\)-measurable function on $X$
such that \(\int|f|\,d\mu<\infty\).
For any \(\epsilon>0\) there exists \(\delta>0\) such that
\(\int_E |f|\,d\mu < \epsilon\) whenever \(\mu(E)<\delta\).
\end{llem}
\begin{thmproof}
Define the complex measure \(\lambda(E) = \int_E f\,d\mu\).
Clearly \(\lambda \ll \mu\). Theorem~6.11 gives the desired conclusion.
\end{thmproof}

%%%%%%%%%%%%%%%%%%%%%%%%%%%%%%%%%%%%%%%%%%%%%%%%%%%%%%%%%%%%%%%%%%%%%%%%
%%%%%%%%%%%%%%%%%%%%%%%%%%%%%%%%%%%%%%%%%%%%%%%%%%%%%%%%%%%%%%%%%%%%%%%%
\section{Exercises} % pages 132-134

%%%%%%%%%%%%%%%%%
\begin{enumerate}
%%%%%%%%%%%%%%%%%

%%%%%%%%%%%%%% 01
\begin{excopy}
If \(\mu\) is a complex measure on a \(\sigma\)-algebra \frakM,
and if \(E \in \frakM\), define
\begin{equation*}
\lambda(E) = \sup \sum |\mu(E_i)|,
\end{equation*}
the supremum being taken over all \emph{finite} partitions \(\{E_i\}\) of $E$.
Does it follow that \(\lambda = |\mu|\)?
\end{excopy}

Yes.

Clearly \(\lambda(E) \leq |\mu|(E)\). To show the opposite inequality,
let \(\epsilon > 0\). We can find some countable
partition \(\{E_i\}\) of $E$
such that
\begin{equation*}
 \sum_{i\in\N} |\mu(E_i)| > |\mu|(E) - \epsilon/2.
\end{equation*}
So  we can find some integer \(m<\infty\) such that
\begin{equation*}
 \sum_{i=1}^m |\mu(E_i)| > |\mu|(E) - \epsilon.
\end{equation*}
Hence \(\lambda(E) \geq |\mu|(E)\) and thus
\(\lambda(E) = |\mu|(E)\).


%%%%%%%%%%%%%%
\begin{excopy}
Prove that the example given at the end of Sec.~6.10 has the stated properties.
\end{excopy}

\paragraph{Property 1.}
On \(I=(0,1)\), let \(\mu\) be the Lebesgue measure
and \(\lambda\) the counting measure.
By negation, let \(\lambda = \lambda_s + \lambda_a\) be the
Lebesgue decomposition.
For any \(x\in I\), we have
 \(\lambda(\{x\}) = 1\) and  \(\mu(\{x\}) = 0\)
Hence
\(\lambda_s(\{x\}) = 0\) and  \(\lambda_a(\{x\}) = 1\).
Since \(\mu\perp \lambda_a\) we have \(I = A \disjunion B\)
such that for any Lebesgue measurable $E$, we have
\begin{equation*}
\mu(E) = \mu(E\cap A) \qquad \textrm{and} \qquad
\lambda_a(E) = \lambda_a(E\cap B).
\end{equation*}
Since
\(\lambda_a(\{x\}) = 1\)
for any \(x\in I\), we must have \(B=I\), but then \(\mu=0\)
which is a contradiction.

\paragraph{Property 2.}
Assume by negation \(h\in L^1(\lambda)\) such that \(d\mu = h\,d\lambda\).
If \(h=0\) then \(\mu=0\) so we may assume that \(h(x)\neq 0\)
for some \(x\in I\). But then
\begin{equation*}
0 = \mu(\{x\}) = h(x)\lambda(\{x\}) = h(x) \neq 0.
\end{equation*}
which is again a contradiction.


%%%%%%%%%%%%%%
\begin{excopy}
Prove that the vector space \(M(X)\) of  all complex regular Borel measures
on a locally compact Hausdorff space $X$ is a Banach space if
\(\|\mu\| = |\mu|(X)\).
\emph{Hint}: Compare Exercise~8, Chap.~5.
[That the difference of any two members of \(M(X)\) is in \(M(X)\)
was used in the first
paragraph of the proof of Theorem~6.19; supply a proof of this fact.]
\end{excopy}

Let \(\lambda,\mu \in M(X)\).
For \(E = \disjunion E_j\) use definition to compute
\begin{align}
\lambda(E) - \mu(E) \notag \\
&= \left(\sum_{j\in\N} \lambda(E_j)\right) -
   \left(\sum_{j\in\N} \mu(E_j)\right) \notag \\
&= \sum_{j\in\N} \lambda(E_j) - \mu(E_j) \label{eq:ex6.3} \\
&= \sum_{j\in\N} (\lambda - \mu)(E_j) \notag \\
&= (\lambda - \mu)(E).
\end{align}
The equality in \eqref{eq:ex6.3} holds since the series converge absolutely.

If we define \((a\mu)(E) = a(\mu(E)\) for \(a\in\C\), all vector space
equalities trivally hold. Thus we need only to show completeness.
Let \(\{\mu_j\}_{j\in\N}\) be a Cauchy sequence of measures.
That is for each \(\epsilon>0\) there exists some $N$ such that
\(|\mu_j - \mu_k|(X) < \epsilon\) if \(j,k>N\).
Since
\(|\mu_j - \mu_k|(E) \leq |\mu_j - \mu_k|(X)|\)
for any \(E\in\frakM\), the sequence
\(\{\mu_j(E)\}\) is a Cauchy sequence, and we can define
its limit \(\mu(E) = \lim_{j\to\infty} \mu_j(E)\).
It is easy to see that with \(\mu\), the vector space equalities still hold.
We need to show that \(\mu\in M(X)\).

We have the unique decompositions
\begin{align*}
\mu_j &= \lambda_j^+ - \lambda_j^- + i(\nu_j^+ - \nu_j^-) \\
\mu   &= \lambda^+ - \lambda^- + i(\nu^+ - \nu^-)
\end{align*}
where
\(\lambda_j^+\), \(\lambda_j^-\), \(\nu_j^+\) and \(\nu_j^-\)
are positive measures since
the mappings\(x\mapsto|x|\), \(\Re\) \(\Im\) are continuous,
and
\(\lambda^+\), \(\lambda^-\), \(\nu^+\) and \(\nu^-\)
are non-negative functions on \frakM.

For temporary abberviation, we use
\(\lambda=\lambda^+\) and \(\lambda_j = \lambda_j^+\).

Pick arbitrary \(K<\infty\) so we can estimate
\begin{align*}
\sum_{k\leq K} \lambda(E_k)
&= \sum_{k\leq K} \lim_{j\to\infty} \lambda_j(E_k)
 = \lim_{j\to\infty} \sum_{k\leq K} \lambda_j(E_k)
 = \lim_{j\to\infty} \lambda_j\left(\Disjunion_{k\leq K} E_k\right) \\
&\leq \lim_{j\to\infty} \lambda_j(E) \\
&= \lambda(E).
\end{align*}
Hence
\begin{equation*}
\sum_{k\in\N} \lambda(E_k) \leq \lambda(E).
\end{equation*}
Looking at \(f_j = \lambda_j(E_k)\) as function on \(k\in\N\)
with at the coundting measure on \N, we utilize
Lebesgue's dominated convergence theorem
in the following \eqref{eq:ex:6.3}.

Now
If \(E = \disjunion_{j\in\N} E_i\) all in \frakM, then
\begin{align}
\lambda(E)
&= \lim_{j\to\infty} \lambda_j(E)
 = \lim_{j\to\infty} \lambda_j\left(\Disjunion_{K\in\N} E_k\right) \notag \\
&= \lim_{j\to\infty} \sum_{k\in\N} \lambda_j(E_k) \notag \\
&= \sum_{k\in\N} \lim_{j\to\infty} \lambda_j(E_k)  \label{eq:ex:6.3} \\
&= \sum_{k\in\N} \lambda(E_k) \notag
\end{align}

Similarly, we can derive the equlaities for
\(\lambda^-\), \(\nu^+\) and \(\nu^-\). Then we sum them to get
\begin{equation*}
\mu(E) = \sum_{k\in\N}  \mu(E_k).
\end{equation*}


%%%%%%%%%%%%%%
\begin{excopy}
Suppose \(1 \leq p \leq \infty\), and $q$  is the exponent conjugate to $p$.
Suppose \(\mu\) is a positive \(\sigma\)-finite measure and $g$ is a measurable
function such that \(fg\in L^1(\mu)\) for every
\(f\in L^p(\mu)\). Prove that \(g\in L^q(\mu)\).
\end{excopy}

\iffalse
% By H\"older's inequality \(\|fg\|_1 \leq \|f\|_p \|g\|_q\).
Clearly \(f \mapsto \int fg\;d\mu\) is a linear functional on \(L^p(\mu)\).
Let \(X=\disjunion_{j\in\N} X_j\) a decomposition of the space
such that \(\mu(X_j)<\infty\). We set \(S_n = \disjunion_{j\leq n} X_j\)
and let \(g_n = g_{|S_n}\).
Using Exercise~2.5\ich{a} we have
\begin{equation*}
\|g_n\|_q = \|g_n\|_\infty \leq \|g\|_\infty.
\end{equation*}

Clearly \(g\in L^q(\mu)\) iff \(|g| L^q(\mu)\).
Since \(|\mu|(X)<\infty\), we have the constant \(1\in L^q\)
for all \(q\in[1,\infty]\). Hence also
\(g\in L^q(\mu)\) iff \(g\pm 1\in L^q(\mu)\).
Let \(X_n=\{x\in X: n-1 \leq |g(x)| < n\}\).

Assume first \(1<p,q<\infty\), and by negation, assume
\begin{equation} \label{eq:ex6.4:neg}
g\notin L^q(\mu).
\end{equation}
Hence we can have the estimatation.
\begin{equation*}
\sum_{n\in \N} n^q \mu(X_n)
 \geq \sum_{n\in \N} \int_{X_n} |g|^q\,d\mu \\
 = \int_X |g|^q\,d\mu
 = \|g\|q^q = \infty.
\infty
\end{equation*}

Define the function
\begin{equation*}
h(x) =
\left\{
 \begin{array}{ll}
 0 & \quad x\in X_0\\
 n^{q/p} & \quad x\in X_n \;\textrm{and}\; n>0
 \end{array}
\right.
\end{equation*}

Now
\begin{equation*}
\|h\|_p^p
= \int_X h^p\,d\mu
= \sum_{n=0}^\infty \int_{X_n} h^p\,d\mu
= \sum_{n\in\N}  \left(n^(q/p)\right)^p\,d\mu
= \sum_{n\in\N}  n^q\,d\mu
\end{equation*}
Hence \(\||g|-1\|_q = \infty\),

\fi

For each $n$ define
\begin{equation*}
g_n(x) =
\left\{
 \begin{array}{ll}
 g(x) & \quad |g(x)| \leq n\\
 ng(x)/|g(x)| & \quad |g(x)| > n
 \end{array}
\right.
\end{equation*}
Clearly \(\lim_{n\to\infty}g_n(x) = g(x)\;\aded\)\,.
Now we can define functionals \(\Lambda_n\in (L^p(\mu))^*\) by
\begin{equation*}
\Lambda_n(f) = \int_X fg_n\,d\mu.
\end{equation*}
With the estimatation
\begin{equation*}
|\lambda_n{f}|
\leq \int_X |fg_n|\,d\mu
\leq \int_X |fg|\,d\mu
< \infty
\end{equation*}
we use the
\index{Banach-Steinhaus}
\index{Steinhaus}
Banach-Steinhaus Theorem~5.8 (\cite{RudinRCA87})
to get a unifrom bound \(\|\Lambda_n\|\leq M\).
Now for each \(f\in L^p(\mu)\) we have
\begin{equation*}
\lim_{n\to\infty} \int_X fg_n\,d\mu = \int_X fg\,d\mu
\end{equation*}
and if we define \(\Lambda(f) = \int_X fg\,d\mu\) we
have \(\|\Lambda\| \leq M\).


Applying the uniquness part of Theorem~6.16 \cite{RudinRCA87}
we get the desired result for \(p<\infty\).

Now assume \(p=\infty\). Obviously the constant function
\(1\in L^\infty(\mu)\) and by assumption
\(|\int_X 1\cdot g\,d\mu|<\infty\). Hence \(f\in L^1{\mu}\).


%%%%%%%%%%%%%% 05
\begin{excopy}
Suppose $X$ consists of two points $a$ and $b$; define
\(\mu(\{a\}) = 1\),
\(\mu(\{b\}) = \mu(X) = \infty\), and
\(\mu(\emptyset) = 0\).
Is it true for this \(\mu\) ,
that \(L^\infty(\mu)\) is the dual space of \(L^1(\mu)\)?
\end{excopy}

No. Clearly
\begin{equation*}
L^1(\mu) = \{f\in \C^X: f(b)=0\}.
\end{equation*}
Now consider \(g_1,g_2\in L^\infty(\mu)\) defined as
\(g_1=1\), \(g_2(a)=1\) and \(g_2(b)=0\). Both induces the same
functional in \(\left(L^1(\mu)\right)^*\).


%%%%%%%%%%%%%%
\begin{excopy}
Suppose \(1 < p < \infty\), and prove that
\(L^q(\mu)\) is the dual space of \(L^p(\mu)\)
even if \(\mu\) is not \(\sigma\)-finite.
(as usual \(1/p+1/q=1\).)
\end{excopy}

For positive  \(\sigma\)-finite measure the claim has been proved.
Assume now that \((X,\frakM,\mu)\) is a complex measure space.
By Theorem~6.12 (\cite{RudinRCA87}) there exists a measurable
function $h$ such that \(\forall x\in X, |h(x)|=1\) and
\(d\mu = h\,d|\mu|\).
We note, that as \emph{sets of functions} we trivially have
\(L^p(\mu)=L^p(|\mu|)\) and \(L^q(\mu)=L^q(|\mu|)\).
The vector space operations are identical.
Also the topolgies are the same, since the norms are invariant,
that is
\begin{equation*}
\|f\|_{p,\mu}
= \left(\int |f|^p\,d\mu\right)^{1/p}
= \left(\int |f|^p\,d|\mu|\right)^{1/p}
\|f\|_{p,|\mu|}.
\end{equation*}

Hence, given a \(\Lambda\in (L^p(\mu))^*\)
it may be viewed as a functional on \(L^p(|\mu|)\).
From results on positive measures we have \(g\in L^q(|\mu|)\)
such that
\begin{equation*}
\Lambda(f) = \int_X fg\,d|\mu|
\end{equation*}
for all \(f\in L^p(\mu)\).
But then
\begin{equation*}
\Lambda(f) = \int_X f(g/h)\,d\mu
\end{equation*}
and clearly \(g/h\in L^q(\mu)\).
The uniqness of \(g/h)\) is proved in same way
as in the proof of Theorem~6.16 (\cite{RudinRCA87}).


%%%%%%%%%%%%%%
\begin{excopy}
Suppose \(\mu\) is a complex Borel measure on \([0,2\pi]\)
(or on the unit circle \T),
and define the Fourier coefficients of \(\mu\) by
\begin{equation*}
\widehat{\mu}(n) = \int e^{-int}\,d\mu(t) \qquad (n=0,\pm 1,\pm 2, \ldots).
\end{equation*}
Assume that \(\widehat{\mu}(n) \to 0 \) as \(n\to +\infty\)
and prove that then
\(\widehat{\mu}(n) \to 0 \) as \(n\to -\infty\).

\emph{Hint}: The assumption also holds with \(f\,d\mu\) in  place of \(d\mu\)
if $f$ is any trigonometric polynomial, hence if $f$ is continuous,
hence if $f$ is any bounded Borel function,
hence if \(d\mu\) is replace by \(d|\mu|\).
\end{excopy}

Following the hint. For each \(m\in\Z\) we have
\begin{equation*}
\lim_{n\to+\infty} \int e^{-int}e^{imt}\,d\mu(t)
= \lim_{n\to+\infty} \int e^{-i(n-m)t}\,d\mu(t)
= \lim_{n\to+\infty} \int e^{-int}\,d\mu(t)
= 0.
\end{equation*}
If for \(j=1,2\) we have
\begin{equation*}
\lim_{n\to+\infty} \int e^{-int}f_j(t)\,d\mu(t) = 0
\end{equation*}
Then
\begin{equation*}
\lim_{n\to+\infty} \int e^{-int}(f_1+f_2)(t)\,d\mu(t) = 0.
\end{equation*}
Thus we have similar convergence \(\lim_{n\to+\infty}\widehat{(f\mu)}(n)\)
for trigonometric polynomials $f$.
Similarly this holds for all continuous $f$, in particular for
the unique $h$ such that \(|\mu| = h\mu\) and \(|h(t)|=1\)
Now
\begin{align}
\widehat{\mu}(-n)
&= \int e^{int}\,d\mu(t)
 = \int e^{int}/h(t)\,d|\mu(t)|
 = \overline{\int \overline{e^{int}}/\overline{h(t)}\,d|\mu(t)|}
 = \overline{\int e^{-int}/\overline{h(t)}\,d|\mu(t)|} \notag \\
&= \overline{\int e^{-int}\left(h(t)/\overline{h(t)}\right)\,d\mu(t)}
   \label{eq:ex6.7}.
\end{align}
As before, the convergence to zero also holds for
\(h(t)/\overline{h(t)}\) and thus the values \eqref{eq:ex6.7}
converge to zero.


%%%%%%%%%%%%%%
\begin{excopy}
In the terminology of Exercise~7, find all \(\mu\) such that
\(\hat{\mu}\) is periodic, with respect to $k$
[This means that \(\hat{\mu}(n+k) = \hat{\mu}(n)\) for all integers $n$;
of course $k$ is also assumed to be an integer.]
\end{excopy}

Equivalently, we can look for all measurable functions $f$, such that
\begin{equation*}
\int_\T e^{-int}f(t)\,dm(t)
\end{equation*}
is periodic in $n$.


\paragraph{Example:} Given $k$, for each \(j\in\N_k\) let
\(K = \{e^{2j\pi i/k}: 0\leq j < k\}\) and set
\(\mu(\{P\}) = 1/k\) for each \(P\in K\)
and \(\mu(\T\setminus K) = 0\). Complete \(\mu\) to a measure,
then
\begin{equation*}
\widehat{\mu}(n) = \left\{
 \begin{array}{ll}
 1 & \quad \textrm{if}\; n = 0 \bmod k \\
 0 & \quad \textrm{if}\; n \neq 0 \bmod k
 \end{array}
 \right.
\end{equation*}
If \(\mu\ll m\) then \(d\mu = h\,dm\) for some \(h\in L^1(\T)\)
but \(\widehat{h}(n)\) cannot be periodic (unless \(h=0\)),
since \(\lim_{n\to\infty}\widehat{h}(n) = 0\).


%%%%%%%%%%%%%%
\begin{excopy}
Suppose that \(\{g_n\}\) is a sequence of positive continuous functions on
\(I=[0,1]\), that \(\mu\) is a positive Borel measure  on $I$, and that
\begin{itemize}
\itemch{i} \(\lim_{n\to\infty} g_n(x) = 0\quad \aded [m]\).
\itemch{ii} \(\int_I g_n\,dm = 1\) for all $n$,
\itemch{iii} \(\lim_{n\to\infty} \int_I fg_n\,dm = \int_I f\,d\mu\)
              for every \(f\in C(I)\).
\end{itemize}
Does it follow that \(\mu\perp m\)?
\end{excopy}

Yes. We will show that there exists a Borel subset \(D\subset I\) such that
\(m(D)=0\) and \(\mu(E) = \mu(E\cap D)\) for each Borel set $E$.

Consider the ``bad set''
\begin{equation*}
 B = \{x\in I: \limsup g_n(x) > 0\}.
\end{equation*}
Since \(m(B) = 0\) by \ich{i} we can redefine \(g_n(x)=0\)
for each \(x\in B\).
This redefinition removes the \aded\ restriction from \ich{i}, but
does not effect
the other assumptions \ich{ii}, \ich{iii} and not the desired conclusion.

Pick some \(k\in\N\).
\index{Egoroff}
By Egoroff's Theorem (Exercise~3.16 \cite{RudinRCA87})
there exists a measurable \(D_n \subset I\) such that
\(m(D_k) < 1/k\) and such that
\(\{g_n\}\) converges \emph{uniformly} to $0$ on \(C_k\)
where \(E_k = I \setminus D_k\).
We will show that \(\mu(E_k) = 0\).
\iffalse
Hence
\begin{equation*}
\lim_{n\to\infty} \int_{E_k} g_n\,dm
\leq \bigl(1 - m(D_k)\bigr) \lim_{n\to\infty} \sup_{x\in E_k} g_n(x)
= 0.
\end{equation*}
\fi
Given \(\epsilon>0\), we
apply Lusin Theorem~2.24 (\cite{RudinRCA87}) to the function
\(\chhi_{E_k}\) to get a function \(f_k\in C(I)\)
such that \(\|f_k\|_\infty = 1\) and
\begin{equation*}
m\bigl(\{x\in I: f(x) \neq \chhi_{E_k}(x)\}\bigr) < \epsilon.
\end{equation*}
\begin{align*}
\mu(E_k)
&= \int_I \chhi{E_k}\,d\mu \\
&\leq \int_I f_k\,d\mu + \epsilon
\end{align*}




%%%%%%%%%%%%%% 10
\begin{excopy}
Let \((X,\frakM,\mu)\) be a positive measure space. Call a set
\(\Phi \subset L^1(\mu)\)
\index{uniformly  integrable}
\emph{uniformly  integrable}
if to each \(\epsilon < 0\) corresponds \(\delta>0\) such that
\begin{equation*}
\left|\int_E f\,f\mu\right| < \epsilon
\end{equation*}
whenever \(f\in\Phi\) and \(\mu(E) < \delta\).
\begin{itemize}

\itemch{a}
Prove that every finite subset of \(L^1(\mu)\) is uniformly integrable.

\itemch{b}
Prove the following convergence theorem of \index{Vitali} Vitali:

\textsl{
If (i) \(\mu(X)<\infty\), (ii) \(\{f_n\}\) is uniformly integrable,
(iii) \(f_n(x)\to f(x)\;\aded\) as \(n\to\infty\),
and (iv) \(|f(x)|<\infty\;\aded\), then \(f\in L^1(\mu)\) and
\begin{equation*}
\lim_{n\to\infty} \int_X|f_n - f|\,d\mu = 0.
\end{equation*}
}
\emph{Suggestion}: Use \index{Egoroff} Egoroff's theorem

\itemch{c}
Show that \ich{b} fails if \(\mu\) is a Lebesgue measure on
\((-\infty,\infty)\), even if \(\{\|f_n\|_1\}\) is assumed to be bounded.
Hypothesis (i) can therefore  not be omitted in \ich{b}.

\itemch{d}
Show that the hypothesis (iv)  is redundant in \ich{b} for some \(\mu\)
(for instance, for Lebesgue measure on a bounded interval), but that there are
finite measures for which the omission of~(iv)
would make \ich{b} false.

\itemch{e}
Show that Vitali's theorem implies Lebesgue's dominated convergence theorem,
for finite measure space. Construct an example in which Vitali's theorem
applies although the hypothesis of
Lebesgue's theorem does not hold.

\itemch{f}
Construct a sequence \(\{f_n\}\), say on \([0,1]\), so that
\(f_n(x)\to 0\) for every $x$,
\(\int f_n\to 0\) but \(\{f_n\}\) 
is not uniformly integrable (with respect to Lebesgue's measure).

\itemch{g}
However, the following converse of Vitali's theorem is true:

\textsl{
If \(\mu(X)<\infty\), \(f_n\in L^1(\mu)\), and
\begin{equation*}
\lim_{n\to\infty} \int_E f_n\,d\mu
\end{equation*}
exists for every \(E\in \frakM\), then \(\{f_n\}\) is uniformly integrable.
}

Prove this by completing the following outline.

Define \(\rho(A,B) = \int |\chhi_A - \chhi_B|\,d\mu\).
Then \((\frakM,\rho)\) is a complete metric space
(modulo sets of measure $0$), and \(E\mapsto \int_E f_n\,d\mu\) is
a continuous for each $n$.
If \(\epsilon > 0\), there exist \(E_0\), \(\delta\), $N$
(Exercise~13, Chap~5) so that
\begin{equation} \label{eq:ex:6.10}
\left| \int_E (f_n - f_N)\,d\mu\right| < \epsilon
\qquad \textrm{if}\qquad
\rho(E, E_0) < \delta, \qquad n > N.
\end{equation}
If \(\mu(A)<\delta\), \eqref{eq:ex:6.10} holds with \(B = E_0 - A\)
and \(C = E_0 \cup A\)
in place of $E$.
Thus \eqref{eq:ex:6.10} holds with $A$ in place of $E$ and \(2\epsilon\)
in place of \(\epsilon\).
Now apply \ich{a} to \(\{\seq{f}{N}\}\): There exists \(\delta'>0\)
such that
\begin{equation*}
\left| \int_A f_n\,d\mu \right| < 3\epsilon
\qquad \textrm{if}\qquad
\mu(A) < \delta', \qquad n=1,2,3,\ldots.
\end{equation*}
\end{itemize}
\end{excopy}

We will need the following result.
\begin{llem}
If
\(\Phi=\{f_j: j\in J\}\) is uniformly  integrable,
then
\(\{|f_j|: j\in J\}\) is uniformly  integrable.
\end{llem} \label{llem:abs:unifinteg}
\begin{thmproof}
Define the complex plane quartans
\begin{align*}
\C_0 &= \{z\in\C: \Re(z)>0 \;\wedge\; \Im(z)\geq 0\} \\
\C_j &= \{e^{j\pi i/2}z: z\in \C_0 \} \quad j=1,2,3.
\end{align*}
Now
\begin{equation*}
\C = \{0\} \disjunion \Disjunion_{j=0}^3 \C_j.
\end{equation*}
Pick arbitrary \(\epsilon> 0\). Let \(\delta>0\) such that 
\(|\int_E f\,d\mu|<\epsilon\) for all \(f\in\Phi\) 
whenever \(\mu(E)<\delta\).
Take such $E$, and define \(E_j = E\cap \C_j\) for \(j=0,1,2,3\).
For each such $j$
\begin{align*}
\int_{E_j} |f|\,d\mu
&= \int_{E_j} |\Re(f) + \Im(f)|\,d\mu \\
&\leq  \int_{E_j} |\Re(f)|\,d\mu + \int_{E_j} |\Im(f)|\,d\mu \\
&=  \left|\int_{E_j} \Re(f)\,d\mu\right| 
  + \left|\int_{E_j} \Im(f)\,d\mu\right| 
 =  \left|\Re\left(\int_{E_j} f\,d\mu\right)\right| 
  + \left|\Im\left(\int_{E_j} f\,d\mu\right)\right| \\
&\leq  2\max\left(\left|\Re\left(\int_{E_j} f\,d\mu\right)\right|,
                  \left|\Im\left(\int_{E_j} f\,d\mu\right)\right|\right) \\
&\leq  \sqrt{2}\left|\int_{E_j} f\,d\mu\right|.
\end{align*}

Hence
\begin{equation*}
\int_E |f|\,d\mu
=    \sum_{j=0}^3 \int_{E_j} |f|\,d\mu
\leq \sqrt{2}\left|\sum_{j=0}^3 \int_{E_j} f\,d\mu\right| 
<    \sqrt{2}\epsilon
\end{equation*}
and thus \(\Phi\) is uniformly integrable.
\end{thmproof}



\begin{itemize}

\itemch{a}
By Theorems~1.29 and~6.11 \cite{RudinRCA87} $f$ is uniformly integrable
(by itself) if \(f\in L^1(\mu)\).
For finite set \(\{f_j\}_{j=1}^n\), for each \(\epsilon>0\)
we pick \(\delta = \min\{\delta_j: 1\leq j \leq n\}\)
where \(\delta_j\) corresponds to \(f_j\)


\itemch{b}
By removing a subset of measure zero, we may assume
\(f_n(x) \to f(x)\) for all \(x\in X\).

Pick arbitrary \(\epsilon>0\).
By Egoroff's Theorem (Exercise~3.16)  
and by being uniform integrable, 
we can find some some \(\delta>0\) and a set \(E\subset X\) such that 
\begin{enumerate}
\itemch{i} \(\mu(X\setminus E)<\delta\) .
\itemch{ii} On \(X\setminus E\) the convergence \(f_n \to f\) is uniform.
\itemch{iii} \(\int_E |f_n|\,d\mu < \epsilon\) for all $n$, 
      see local lemma~\ref{llem:abs:unifinteg}.
\end{enumerate}
By Fatou's Lemma (Theorem~1.28 \cite{RudinRCA87}) we have
\begin{equation*}
\int_E |f|\,d\mu 
= \int_E |\lim_{n\to\infty}f_n|\,d\mu 
= \int_E |\liminf_{n\to\infty}f_n|\,d\mu 
\leq \liminf_{n\to\infty} \int_E |f_n|\,d\mu < \sqrt{2}\epsilon
\end{equation*}
Pick $m$ such that \(|f_n(x)-f(x)| < \epsilon\)
for all \(n\geq m\) and all \(x\in X\setminus E\).
Thus
\begin{equation*}
\left|\int_X f\,d\mu\right|
\leq
  \left|\int_{X\setminus E} f\,d\mu\right|
+ \left|\int_E f\,d\mu\right|
\leq
  \int_{X\setminus E} (|f_n| + \epsilon)\,d\mu + \sqrt{2}\epsilon < \infty.
\end{equation*}
Thus \(f\in L^1(\mu)\).

Now
\begin{equation*}
\int_X |f-f_n|\,d\mu
= \int_{X\setminus E} |f-f_n|\,d\mu + \int_E |f-f_n|\,d\mu 
% \leq \mu(X)\epsilon + 2\sqrt{2}\epsilon
\leq (\mu(X) + 2\sqrt{2})\epsilon
\end{equation*}
and the desired convergence holds.

\itemch{c}
For a counterexample, simple take \(f_n(x) = \chhi_{[0,n]}\)
that are uniformly integrable.
Clearly \(\lim_{n\to\infty} f_n = 1 \notin L^1(\R,m)\).

\itemch{d}
For Lebesgue measure $m$ on \([0,1]\), say we dropy the 
\ich{iv} requirement.

If we take a singleton space \(X=\{x\}\) with \(\mu(\{x\}=1\)
and \(f_n(x)=n\), then the set is trivially uniformly integrable,
by ``suggesting'' \(\delta=1/2<1\).
Clearly \(f(x)=\infty\) and \(\int_X|f_n-f|\,d\mu=\infty\).

\itemch{e}
Given the assumptions of Lebesgue's dominated convergence theorem ---
\(f_n\to f\) and \(|f_n|\leq g\in L^1(\mu)\).
we Actually need to show that \(\Phi=\{f_n\}_{n\in\N}\) 
is uniformly integrable.
By \ich{a}, the function \(\{g\}\) is uniformly integrable.
Given \(\epsilon>0\) there exists some \(\delta>0\) 
such that \(|\int_E g\,d\mu|<\epsilon\) whenever \(\mu(E)<\delta\).
For such $E$ and for any $n$, by the dominating condition, 
\(|\int_E |f_n|\,d\mu|<\epsilon\) as well, thus
\(\Phi\) is uniformly integrable.

We now show an
example that satisfies Vitali's lemma assumptions but not
Lebesgue's theorem. For all \(n\in\N\)
let \(f_n: \R\to\R\) defined by 
\begin{align*}
a_0 &= 0 \\
a_n &= \sum_{j=1}^n 1/n \\
f_n &= \chhi_{[a_n,a_{n+1}]}.
\end{align*}
Clearly \(\lim_{n\to\infty}=0\) 
and also \(\lim_{n\to\infty} \|f_n\|_1 = 0\)
but there is no dominating function for \(\{f_n\}_{n\in\N}\)
in \(L^1(\R,m)\).


\itemch{f}
% Define \(f_n = n\chhi_{(0,1/n^2]}\).
Define
\begin{equation*}
f_n(x) = \left\{
\begin{array}{ll}
n^2        & \qquad 0 < x \leq 1/2n \\
-n^2       & \qquad 1/2n < x < 1/n \\
0          & \qquad x=0\;\vee\; x \geq 1/n
\end{array}\right.
\end{equation*}
Clearly \(\lim_{n\to\infty} f_n(x) = 0\) for all \(x\in[0,1]\)
and \(\int_{[0,1]} f_n\,dm = 0\) but
\begin{equation*}
\int_{[0,1/2n]} f_n\,dm = n/2.
\end{equation*}




\itemch{g}
The triangle inequality \(\rho(A,B)\leq \rho(A,C) + \rho(B,C)\)
is trivial. To show completeness we follow \cite{Oxtoby1980} chapter~10.
Say \(\{E_j\}_{j\in\N}\) a Cauchy sequence.
For each $j$ there is \(n_j > n_{j-1}\) such that \(\rho(E_m,E_n)<2^{-j}\)
whenever \(m,n\geq j\). Let \(F_j=E_{n_j}\), 
we have \(\rho(F_j,F_k)<2^{-j}\) for all \(k>j\). Define
\begin{equation*}
H_j = \bigcap_{k=j}^\infty F_k \qquad 
E = \bigcup_{j=1}^\infty H_j.
\end{equation*}
The set $E$ consists of points that belong to all 
but a finite number of sets \(\{f_j\}_{j\in\N}\).
If \(x\in E\vartriangle  F_j\) then there are two cases
\begin{itemize}
\itemch{i} 
\(x\in  E\setminus F_j\) then we look for the first $k$
such that \(x\in F_{j+k}\).
\itemch{ii} 
\(x\in  F_j\setminus E\) then we look for the first $k$
such that \(x\notin F_{j+k}\).
\end{itemize}
In both cases \(x\in F_{j+k-1} \vartriangle F_{j+k}\).

For each  \(x\in H_j\vartriangle F_j = F_j \setminus H_j\)
by looking at the first $k$ such that \(x\notin F_{j+k}\)
we see that
\begin{equation*}
x \in F_{j+k-1} \vartriangle F_{j+k}.
\end{equation*}

Using the simple rule
\((A\vartriangle B) \cup (A\vartriangle B) \subset B\vartriangle C \),
now clearly
\begin{equation*}
E\vartriangle  F_j
\subset (E\vartriangle H_j)\cup (H_j\vartriangle F_j)
\subset \bigcup_{k=1}^\infty F_{j+k-1} \vartriangle F_{j+k}.
\end{equation*}
Consequently
\begin{equation*}
m(E\vartriangle  F_j)
\leq \sum_{k=1}^\infty m(F_{j+k-1} \vartriangle F_{j+k})
\leq \sum_{k=1}^\infty 2^{-(j+k)} = 2^{1-j}.
\end{equation*}
For any \(n\geq n_j\) we have
\begin{align*}
\rho(E,E_n) 
&= m\bigl((E\vartriangle F_j) \vartriangle
          (E_{n_j}\vartriangle E_n)\bigr) \\
&\leq m(E\vartriangle F_j) + m(E_{n_j}\vartriangle E_n) 
< 2^{1-j}+2^{-j}.
\end{align*}

Thus completeness of the metric space $S$ of measurable sets 
modulu sets of zero-measure with the metric \(\rho\) was shown.

The continuity of \(E\mapsto \int_E f_n\,d\mu\) is immediate
by \(f_n\in L^1(\mu)\).
By Exercise~5.13(b) there exists $N$ and  an open set 
\(V = \{E\in S: \rho(E_0,E) < \delta)\}\)
of \(E_0\) such that \eqref{eq:ex:6.10} holds for any \(E\in V\) and \(n>N\).
If \(\mu(A)<\delta\) then \(B = E_0 \setminus A\in V\)
and \(C = E_0 \cup A\in V\).
Thus
\begin{equation*}
\left|\int_A f_n\right| 
= \left|\int_{E_0\setminus A} f_n + \int_{E_0\cup A} f_n\right| 
\leq \left|\int_{E_0\setminus A} f_n\right| + \left|\int_{E_0\cup A} f_n\right| 
< 2\epsilon.
\end{equation*}

We pick some \(0<\delta'<\delta\) such that
\(|\int_E f_j\,d\mu| < \epsilon\) whenever \(1\leq j \leq N\).
Now if \(\mu(A)<\delta'\) then
for any \(n>N\) we have
\begin{equation*}
\left|\int_A f_n\,d\mu\right|
\leq \left|\int_A f_N\,d\mu\right| + \left|\int_A (f_n - f_N)\,d\mu\right|
\leq \epsilon + 2\epsilon < 3\epsilon
\end{equation*}
If \(n\leq N\) then \(|\int_A f_n\,d\mu| < \epsilon < 3\epsilon\) trivially.
Thus
\begin{equation*}
\left|\int_A f_n\,d\mu\right| < 3\epsilon
\end{equation*}
for any \(n\in\N\) and $A$ such that \(\mu(A) < \delta'\).
Thus \(\{f_n\}_{n\in\N}\) are uniformly integrable, and by Vitali's lemma
the desired result follows.
\end{itemize}


%%%%%%%%%%%%%%
\begin{excopy}
Suppose \(\mu\) is a positive measure on $X$, \(\mu(X)<\infty\),
\(f_n \in L^1(\mu)\) for \(n=1,2,3,\ldots\),
\(f_n(x)\to f(x)\;\aded\),
and there exists \(p>1\) and \(C<\infty\) such that
\(\int_X |f_n|^p\,d\mu<C\) for all $n$. Prove that
\begin{equation*}
\lim_{n\to\infty} \int_X |f - f_n|\,d\mu = 0.
\end{equation*}
\emph{Hint}: \(\{f_n\}\) is uniformly integrable.
\end{excopy}

By negation there exists \(\epsilon>0\) and 
a sequences \(\{\delta_n\}_{n\in\N}\) and sets \(\{A_n\}_{n\in\N}\)
such that 
\begin{align*}
\mu(A_n) &< \delta_n = \epsilon/n \\
\left|\int_{A_n} f_n\,d\mu \right| &\geq \epsilon.
\end{align*}

Let $q$ be the exponent conjugate of $q$.
By H\"older inequality
\begin{equation*}
\left|\int_{A_n} f_n\,d\mu \right|
\leq \int_{A_n} |f_n|\,d\mu 
\leq 
\iffalse
     \left(\int_{A_n} |f_n|^p\,d\mu\right)^{1/p}
     \left(\int_{A_n} 1^q\,d\mu\right)^{1/q}
= 
\fi
\left(\int_{A_n} |f_n|^p\,d\mu\right)^{1/p} \bigl(\mu(A_n)\bigr)^{1/q}
\end{equation*}
Hence
\begin{equation*}
\int_{A_n} |f_n|^p\,d\mu
\geq \left( 
      \left|\int_{A_n} f_n\,d\mu \right| \bigm/ \bigl(\mu(A_n)\bigr)^{1/q}
     \right)^p
\geq \bigl( \epsilon / (\epsilon/n)^{1/q} \bigr)^p 
= (n\epsilon^{1-1/q})^p
= n^p\epsilon.
\end{equation*}
Which gives the contradiction
\begin{equation*}
\int_X |f_n|^p\,d\mu \geq \int_{A_n} |f_n|^p\,d\mu \geq n^p\epsilon > C
\end{equation*}
For sufficiently large $n$.


%%%%%%%%%%%%%%
\begin{excopy}
Let \frakM\ be the collection of all sets $E$ in the unit interval \([0,1]\) such that either $E$ or its complement is at most countable.
Let \(\mu\) be the counting measure on this \(\sigma\)-algebra \frakM.
If \(g(x) = x\) for \(0\leq x \leq 1\),
show that $g$ is not \frakM-measurable, although the mapping
\begin{equation*}
f \to \sum xf(x) = \int fg\,d\mu
\end{equation*}
makes sense for every \(f\in L^1(\mu)\) and defines a bounded linear functional
on \(L^1(\mu)\). Thus \((L^1)^* \neq L^\infty\) in this situation.
\end{excopy}

The set \(g^{-1}([0,1/2]) = [0,1/2]\) is not measurable, hence $g$ is not.
Indeed this measurable space is not \(\sigma\)-finite.

%%%%%%%%%%%%%%
\begin{excopy}
Let \(L^\infty = L^\infty(m) \), where $m$ is a Lebesgue measure on
\(I=[0,1]\). Show that there is a bounded linear functional \(\Lambda \neq 0\)
on \(L^\infty\) that is $0$ on \(C(I)\), and therefore there is no
\(g\in L^1(m)\) that satisfies
\(\Lambda f = \int_I fg\,dm\) for every \(f\in L^\infty\).
Thus \((L^\infty)^* \neq L^1\).
\end{excopy}

\(C(I)\) is a closed subspace of \((L^\infty)\).
By Hahn Banach Theorem
\index{Hahn Banach}
Theorem~5.16 \cite{RudinRCA87} 
and its consequence Theorem~5.19 there is a functional 
\(\Lambda\in (L^\infty)^*\) such that 
\(\Lambda f = 0\) for all \(f\in C(I)\) but 
\(\Lambda(\chhi_{[0,1/2]}) \neq 0\). Viewing \(C(I)\subset L^2\)
as a subspace of Hilbert space, if \(\Lambda\) may be represented
by inner multiplication by the zero function, but it surely cannot
represent \(\Lambda\) for the \(\chhi{[0,1/2]}\) case.


%%%%%%%%%%%%%%%%%
\end{enumerate}

 % -*- latex -*-
% $Id: rudinrca7.tex,v 1.15 2006/09/05 20:39:02 yotam Exp $


%%%%%%%%%%%%%%%%%%%%%%%%%%%%%%%%%%%%%%%%%%%%%%%%%%%%%%%%%%%%%%%%%%%%%%%%
%%%%%%%%%%%%%%%%%%%%%%%%%%%%%%%%%%%%%%%%%%%%%%%%%%%%%%%%%%%%%%%%%%%%%%%%
%%%%%%%%%%%%%%%%%%%%%%%%%%%%%%%%%%%%%%%%%%%%%%%%%%%%%%%%%%%%%%%%%%%%%%%%
\chapterTypeout{Differentiation} % 7

%%%%%%%%%%%%%%%%%%%%%%%%%%%%%%%%%%%%%%%%%%%%%%%%%%%%%%%%%%%%%%%%%%%%%%%%
%%%%%%%%%%%%%%%%%%%%%%%%%%%%%%%%%%%%%%%%%%%%%%%%%%%%%%%%%%%%%%%%%%%%%%%%
\section{Notes}

%%%%%%%%%%%%%%%%%%%%%%%%%%%%%%%%%%%%%%%%%%%%%%%%%%%%%%%%%%%%%%%%%%%%%%%%
\subsection{Notations}

\paragraph{Null Set.} We use the term
\emph{nullset}
\index{nullset}
of \cite{Oxtoby1980}
for a set of measure zero.

We denote the \emph{length} of an interval $I$ by \(\ell(I)\).

%%%%%%%%%%%%%%%%%%%%%%%%%%%%%%%%%%%%%%%%%%%%%%%%%%%%%%%%%%%%%%%%%%%%%%%%
\subsection{Outer Measure}

In Rudin's treatment \cite{RudinRCA87}, measure theory is developed
without relying in the notion of outer measure.
In contrast, Royden gives a the following definition
(\cite{Royden} Chapter~3, Section~2 page~56).

\paragraph{Definition.}
\index{outer measure}
\index{measure!outer}
The \index{outer measure} of a set \(A\subset \R\) is
\begin{equation*}
m^*(A) = \inf_{A\subset \cup I_n} \ell(I_n).
\end{equation*}
When can generalize this to sets in \(\R^n\) and also show that
\begin{equation*}
m^*(A) = \inf \{m(G): A\subset G \wedge G \;\mathrm{is\ open}\}.
\end{equation*}

%%%%%%%%%%%%%%%%%%%%%%%%%%%%%%%%%%%%%%%%%%%%%%%%%%%%%%%%%%%%%%%%%%%%%%%%
\subsection{Lemma of Vitali}

We bring the 
Lemma of Vitali
\index{Vitali!of Vitali}
from \cite{Royden}.

\begin{llem} \label{lem:vitali}
Let $E$ be a set of finite measure and \frakI\ a set of intervals
that covers $E$ in the sense of Vitali, that is
\begin{equation*}
\forall x\in E\, \forall \epsilon>0\;
 \exists I\in\frakI\, x\in I \wedge \ell(I) < \epsilon.
\end{equation*}
Then given \(\epsilon>0\) there exists a finite subset of intervals 
\(\{I_n\}_{n=1}^N \subset \frakI\) such that 
\begin{equation}
m\left(E \setminus \cup_{n=1}^N I_n\right) < \epsilon>0.
\end{equation}
\end{llem}

\textbf{Note} Originally, the lemma deals with \emph{outer} measure,
but here we use less generalized version.

\begin{thmproof}
\Wlogy\ we may assume that \(\{\ell(I): I\in\frakI\}\) is bounded,
otherwise we pick some finite measure open set \(O\supset E\) 
and intersect each interval with this opens set, producing new interval(s)
that still satisfy the requirements.
Form a sequence \(\{I_j\}_{j\in\N}\) by induction. Say 
\(\{I_j\}_{j=1}^n\) were picked. Define
\begin{align*}
U_n &= \cup_{j=1}^n \overline{I_j} \\
k_n &= \sup \{\ell(I): I\in\frakI \wedge I \cap U_n = \emptyset\}.
\end{align*}
If \(E\subset U_n\) we are done, otherwise \(0< k_n < \infty\).
Pick \(I_{n+1}\) such that \(\ell(I_{n+1}) > k_n/2\).
Clearly \(\lim_{n\to\infty} \ell(I_n) = 0\), hence there exists $N$
such that
\begin{equation*}
\sum_{n=N+1}^\infty \ell(I_n) < \epsilon/5.
\end{equation*}
Let \(x\in R\) and pick \(I\in\frakI\) such that \(I\cap U_N=\emptyset\).
Let $n$ be the smallest integer such that \(I\cap I_n \neq \emptyset\).
Clearly \(n>N\) 
and by our method of forming the sequences 
\(\ell(I) \leq k_{n-1} \leq 2(I_n)\). If \(c_n\) is the center of \(I_n\) then
\begin{equation*}
|x-c_n| \leq \ell(I) + \ell(I_n)/2 \leq \frac{5}{2} \ell(I_n).
\end{equation*}
Define  $5$-time expansions of \(I_n\)
\begin{equation*}
J_n = \left[c_n - \frac{5}{2} \ell(I_n),\, c_n + \frac{5}{2} \ell(I_n)\right]
\end{equation*}
and now  \(x\in J_n\) and so
\begin{equation*}
R \subset \cup_{n=N+1}^\infty J_n
\end{equation*}
Hence
\begin{equation*}
m(R) 
\leq \sum_{n=N+1}^\infty \ell(J_n)
= 5\sum_{n=N+1}^\infty \ell(I_n)
< \epsilon.
\end{equation*}
\end{thmproof}


%%%%%%%%%%%%%%%%%%%%%%%%%%%%%%%%%%%%%%%%%%%%%%%%%%%%%%%%%%%%%%%%%%%%%%%%
\subsection{Lebesgue Differentiation Theorem}

Here is a theorem given in \cite{Royden} Chapter~5, Section~1, page~100,
Theorem~3.

\begin{llem} \label{lem:leb:diff}
If \(f:[a,b]\to\R\) be a non decreasing function
then $f$ is differentiable almost everywhere.
The derivative \(f'\) is measurable and 
\begin{equation*}
\int_a^b f'(x)\,dx \leq f(b) - f(a). \label{eq:lem:leb:diff}
\end{equation*}
In particular, \(f'\in L^1([a,b],m)\).
\end{llem}
\begin{thmproof}
Let \(D\subset [a,b]\) be the set of points where $f$ is discontinuous.
Since $f$ is non decreasing we now that $D$ is countable.

Define
\begin{equation} \label{eq:7:DfMDfm}
D^+f(x) = \varlimsup_{h\to 0} \frac{f(x+h)-f(x)}{h}
\qquad
D^-f(x) = \varliminf_{h\to 0} \frac{f(x+h)-f(x)}{h}
\end{equation}
where \(x,x+h \in [a,b]\) of course.

We would like to be able to use limits such that that $h$ run over
some countable set for all \(x\in[a,b]\).
\begin{align} 
\Delta^+f(x) &\eqdef \label{eq:7:DfMQ} 
   \varlimsup_{\stackrel{h\to 0}{h\in\Q}} \frac{f(x+h)-f(x)}{h} \\
\Delta^-f(x) &\eqdef \label{eq:7:DfmQ}
   \varliminf_{\stackrel{h\to 0}{h\in\Q}} \frac{f(x+h)-f(x)}{h}
\end{align}

\paragraph{Claim:} 
\begin{align}
D^+f(x) &= \Delta^+f(x)  \label{eq:DfM:eqQ} \\
D^-f(x) &= \Delta^-f(x)  \label{eq:Dfm:eqQ}
\end{align}
We will show \eqref{eq:DfM:eqQ} and \eqref{eq:Dfm:eqQ} 
will follow from \(\varliminf f = -\varlimsup -f\).
Clearly \(\Delta^+f(x) \leq D^+f(x) \) since \(\Q\subset\R\).
For this claim, fix \(x\in[a,b]\) and let \(u \eqdef D^+f(x)\).
Pick an arbitrarily small \(\delta>0\).
There is \(h\neq 0\) such that \(|h|<\delta\)
\begin{equation*}
\bigl(f(x+h) - f(x)\bigr)/h > u - \epsilon \qquad x+h\in(a,b).
\end{equation*}
Let us assume that \(h>0\). The negative case can be similarly treated.
Put
\begin{equation*}
Q(x,h) = \{q\in\Q: q\geq h \wedge x+q\in[a,b]\}
\end{equation*}
Clearly \(Q(x,h)\neq \emptyset\) and 
 \(f(x+h)\leq f(x+q)\) for each \(q\in Q(x,h)\).
Thus
\begin{equation*}
\varlimsup_{\stackrel{q\to h}{q\in\Q(x,h)}}  \frac{f(x+q) - f(x)}{q}
\geq \varlimsup_{\stackrel{q\to h}{q\in\Q(x,h)}}  \frac{f(x+h) - f(x)}{q}
= \varlimsup_{\stackrel{q\to h}{q\in\Q(x,h)}}  \frac{f(x+h) - f(x)}{h}
= u.
\end{equation*}
Hence \(\Delta^+f(x) \geq D^+f(x)\) for any \(x\in[a,b]\)
and the claim \eqref{eq:DfM:eqQ} is shown.

Define the ``bad'' sets
\begin{align}
E       &\eqdef \left\{x\in[a,b]: D^-f(x) < D^+f(x)\right\} \notag \\
E_{u,v} &\eqdef \left\{x\in[a,b]: D^-f(x) < v < u < D^+f(x)\right\} 
        \label{eq:7:Euv}
\end{align}
and clearly
\begin{equation} \label{eq:ex7.14:Eu}
E = \bigcup_{u,v\in\Q} E_{u,v}
\end{equation}
Since 
\begin{align*}
\Delta^+f(x) &= \inf_{r\in\Q^+} 
    \sup\left\{ \frac{f(x+q)-f(x)}{q} :
                q\in\Q \swedge 0<|q|<r \swedge x+q\in[a,b] \right\}
\\
\Delta^-f(x) &= \sup_{r\in\Q^+} 
    \inf\left\{ \frac{f(x+q)-f(x)}{q} :
                q\in\Q \swedge 0<|q|<r \swedge x+q\in[a,b] \right\}
\end{align*}
the sets $E$ and \(E_{u,v}\) are measurable! They could be defined
by countable intersection and unions of open sets.
Noting that inverse image of intervals via $f$ are intervals since
$f$ is non decreasing.
This was 
the whole reason for the claim, thus avoiding the usage of outer measure
and using a simplified lemma (\ref{lem:vitali}) of Vitali.

We note that 
$f$ is differentiable on \([a,b]\setminus E\) 
and obviously \(E_{u,v} = \emptyset\) if \(u\geq v\).
Since the above \eqref{eq:ex7.14:Eu} is a countable union 
it will suffice to show that \(m(E_{u,v}) = 0\) whenever \(u < v\).
Fix some \(u,v\in\R\) such that \(v<u\).
Assume by negation 
\begin{equation} \label{eq:7:vit:sgt0}
s \eqdef m(E_{u,v}) > 0.
\end{equation}
Pick arbitrary \(\epsilon>0\) and take an open \(G \supset E_{u,v}\)
such that \(m(G) < s + \epsilon\)
For each \(x\in E_{u,v}\) we can find some $h$ such that
the interval \([x,x+h]\) or  \([x+h,x]\) is contained in $G$ and 
\begin{equation} \label{eq:7:vit:ltv}
\bigl(f(x+h) - f(x)\bigr)/h < v.
\end{equation}
By local lemma (Vitali)~\ref{lem:vitali} 
there is a finite set of disjoint open intervals 
\(\{I_j\}_{j=1}^N = \{(a_j,b_j)\}_{j=1}^N\) (where \(a_j<b_j\))
that 
cover a subset \(A = \cap_{j=1}^N I_j \cap  E_{u,v}\) 
and \(m(A) > s - \epsilon\). Summing \eqref{eq:7:vit:ltv} we get
\begin{equation} \label{eq:7:vit:ltvs}
\sum_{j=1}^N f(b_j) - f(a_j) 
< v\sum_{j=1}^N b_j - a_j < v\,m(G)  <  v(s+\epsilon).
\end{equation}
Now for each \(y\in A\) there is an interval
 \([y,y+k]\) or  \([y+k,y]\) is contained in some \(I_j\) such that 
\begin{equation} \label{eq:7:vit:gtu}
\bigl(f(y+k) - f(y)\bigr)/k > u.
\end{equation}
Applying the same lemma again, there are disjoint intervals
\(\{J_j\}_{j=1}^M = \{(c_j,d_j)\}_{j=1}^M\) (where \(c_j<d_j\))
that 
cover a subset \(B = \cap_{j=1}^M J_j \cap  A\) 
such that \(m(B) > s-2\epsilon\). 
Similar summation of~\eqref{eq:7:vit:gtu} gives
\begin{equation} \label{eq:7:vit:gtus}
\sum_{j=1}^M f(d_j) - f(c_j) > u \sum_{j=1}^M d_j - c_j >  u(s - 2\epsilon).
\end{equation}
For the segment \(I_n\), since $f$ is nondecreasing we have
\begin{equation*}
\sum_{\stackrel{1\leq j \leq M}{J_j \subset I_n}} f(d_j) - f(c_j) 
\leq f(b_n) - f(a_n).
\end{equation*}
Summing over all \(\{I_j\}_{j=1}^N\) gives
\begin{equation} \label{eq:vit:dcltba}
\sum_{j=1}^M f(d_j) - f(c_j) \leq \sum_{j=1}^N f(b_j) - f(a_j) 
\end{equation}
Combining 
\eqref{eq:7:vit:ltvs},
\eqref{eq:7:vit:gtus} and
\eqref{eq:vit:dcltba}
gives
\begin{equation*}
u(s - 2\epsilon) \leq v(s+\epsilon).
\end{equation*}
Since \(\epsilon\) is arbitrarily small, we have
\(us \leq vs\) and by \eqref{eq:7:Euv} \(s=0\) contradiction 
to~\eqref{eq:7:vit:sgt0}. Thus \(m(E)=0\) and $f$ is differentiable
almost everywhere.

To show the final part of this differentiation theorem, let 
\begin{equation*}
g_n(x) \eqdef n\bigl(f(x+1/n) - f(x)\bigr) 
                        \qquad \forall x>b,\; f(x)\eqdef f(b).
\end{equation*}
Clearly \(\lim_{n\to\infty} g_n(x) = f'(x)\) hence it is measurable.
Now by Fatou's lemma
\index{Fatou}
\begin{align*}
\int_a^b f'(x)\,dx
&\leq \varliminf_{n\to\infty} \int_a^b g_n(x)\,dx  
= \varliminf_{n\to\infty} n \int_a^b \bigl(f(x+1/n) - f(x)\bigr) \,dx  \\
&= \varliminf_{n\to\infty} 
   \left(n \int_b^{b+1/n} f(x)\,dx  - \int_a^{a+1/n} f(x)\,dx  \right)
 = \varliminf_{n\to\infty} 
   \left(f(b)  - n\int_a^{a+1/n} f(x)\,dx  \right) \\
& \leq f(b) - f(a).
\end{align*}
Hence \eqref{eq:lem:leb:diff} holds.
\end{thmproof}



%%%%%%%%%%%%%%%%%%%%%%%%%%%%%%%%%%%%%%%%%%%%%%%%%%%%%%%%%%%%%%%%%%%%%%%%
\subsection{Absolute Continuity and Bounded Variation.}

Trvial lemma showing that absolute continuity is
stronger than having bounded variation.

\begin{llem}
If \(f:[a,b]\to\C\) is an absolute continuous function,
then it has bounded variation.
\end{llem}
\begin{thmproof}
Pick \(\epsilon=1\) then there exists \(\delta>0\)
such that 
\(\sum_{j=1}^n |f(b_k) - f(a_k)| < \epsilon=1\)
whenever
\(\sum_{j=1}^n b_k - a_k| < \delta\)
where \(a \leq a_k \leq b_k \leq b\).

Now given an arbitrary partition \(a = t_0 < t_1 < \cdots < t_m = b\)
we can extended it to 
Now given a partition \(a = u_0 < u_1 < \cdots < u_n = b\)
such that 
\(u_k - u_{k-1} < \delta/2\) for all \(k\in\N_n\) and 
for any \(j\in\N_m\) there exists \(k\in\N_n\) 
such that \(t_j = u_k\).
We split the partitions to chunks bounded by \(\delta/2\) by defining
\begin{equation*}
c(k) = \min \{k\in\Z^+: u_k - a \geq n\delta/2\}.
\qquad 0\leq n \leq \lceil 2(b-a)/\delta \rceil.
\end{equation*}
Clearly \(u_{c(n)} - u_{c(n-1)} < \delta/2\).
Now
\begin{equation*}
\sum_{j=1}^m |f(t_j) - f(t_{j-1})|
\leq \sum_{j=1}^n |f(u_j) - f(u_{j-1})|
= \sum_k \sum_{j=c(k) + 1}^{c(k+1)} |f(u_j) - f(u_{j-1})|
\leq \lceil 2(b-a)/\delta \rceil\cdot 1.
\end{equation*}
Thus $f$ has bounded variation.
\end{thmproof}



%%%%%%%%%%%%%%%%%%%%%%%%%%%%%%%%%%%%%%%%%%%%%%%%%%%%%%%%%%%%%%%%%%%%%%%%
\subsection{Detailed Reference.}

In the \textsc{Notes and Comments} appendix for this chapter,
there is a reference for elementary proof of almost everywhere
differentiability of monotone function.
The full reference is:\\
\emph{A Geometric Proof of the Lebesgue Differentiation Theorem},
\textbf{Donald Austin},
Proceedings of the American Mathematical Society, 
Vol.~16, No.~2 (Apr.,~1965), pp.~ 220--221.


%%%%%%%%%%%%%%%%%%%%%%%%%%%%%%%%%%%%%%%%%%%%%%%%%%%%%%%%%%%%%%%%%%%%%%%%
%%%%%%%%%%%%%%%%%%%%%%%%%%%%%%%%%%%%%%%%%%%%%%%%%%%%%%%%%%%%%%%%%%%%%%%%
\section{Exercises} % pages 156-159

%%%%%%%%%%%%%%%%%
\begin{enumerate}
%%%%%%%%%%%%%%%%%


%%%%%%%%%%%%%% 1
\begin{excopy}
Show that \(|f(x)| \leq (Mf)(x)\) at every Lebesgue point of $f$ if 
\(f\in L^1(\R^k)\).
\end{excopy}

Say $x$ is a Lebesgue point
\index{Lebesgue point}
then
\begin{equation*}
\lim_{r\to\infty} \frac{1}{m(B_r)} \int_{B(x,r)} |f(y)-f(x)|\,dm(y) = 0.
\end{equation*}
Pick \(\epsilon>0\) and \(r>0\) such that
\begin{equation*}
D = \frac{1}{m(B_r)} \int_{B(x,r)} |f(y)-f(x)|\,dm(y) < \epsilon.
\end{equation*}
Hence
\begin{align*}
D 
&\geq \frac{1}{m(B_r)} \left( \int_{B(x,r)} |f(x)|\,dm(y) - 
                             \int_{B(x,r)} |f(y)|\,dm(y) \right) \\
&=  |f(x)| - \frac{1}{m(B_r)} \int_{B(x,r)} |f(y)|\,dm(y) 
\end{align*}
Combing the above, we get
\begin{equation*}
|f(x)| \leq \frac{1}{m(B_r)} \int_{B(x,r)} |f(y)|\,dm(y) + \epsilon.
\end{equation*}
Since \(\epsilon>0\) was arbitrary, we actually have 
the desired
\begin{equation*}
|f(x)| \leq \sup_{r>0} \frac{1}{m(B_r)} \int_{B(x,r)} |f|\,dm
\end{equation*}
inequality.


%%%%%%%%%%%%%%
\begin{excopy}
For \(\delta>0\), let \(I(\delta)\) be the 
segment \((-\delta,\delta) \subset \R^1\). Given \(\alpha\) and \(\beta\),
\(0\leq\alpha\leq \beta\leq 1\),
construct a measurable set \(E\subset \R^1\) so that the upper and lower
limits of 
\begin{equation*}
\frac{m\bigl(E\cap I(\delta)\bigr)}{2\delta}
\end{equation*}
are \(\beta\) and \(\alpha\) respectively, as \(\delta\rightarrow 0\).

(Compare this with Section~7.12.)
\end{excopy}

Let \(0< s,u < 1\). We will soon determine their actual values.
Define % the decreasing sequence \(s_n = s^n\) for \(n\geq 0\) and
the sequence of clopen intervals 
\begin{equation*}
I_n = (s^n,s^{n-1}] \qquad n \geq 0.
\end{equation*}
Clearly \((0,1] = \disjunion_{n\geq 0} I_n\).
Define ``inter-points'' and open sub-intervals.
\begin{align*}
t_n &= s^n + u\cdot m(I_n) \\
U_n &= (s^n, t_n) \subsetneq I_n \qquad n \geq 0.
\end{align*}
and let \(U = \cup_n U_n\), put \(U^- = \{x\in\R: -x \in U\)
and finally \(E = U \disjunion U^-\).
Clearly 
\begin{equation*}
\frac{m\bigl(E\cap I(\delta)\bigr)}{2\delta} 
= m\bigl(U\cap I(\delta)\bigr) / \delta.
\end{equation*}
Thus it is sufficiently to look at the limits of 
\(m(U\cap [0,\delta]) / \delta\).
For any \(\delta\) looking at the 
the subsequence of \(\{U_n\}_{n\geq 0}\) below \(\delta\)
we can see that
\begin{equation*}
\liminf_{\delta\to 0} m(U\cap [0,\delta]) / \delta = u.
\end{equation*}
Similarly, if we look at the ``shifted segments'' \(S_n = (t_{n+1},t_n)\)
we can see that
\begin{equation*}
\limsup_{\delta\to 0} m(U\cap [0,\delta]) / \delta 
= u / (s(1-u) + u) = u / (s + (1-s)u) > u.
\end{equation*}
Trivial solving give the desired values \(u=\alpha\)
and for
\begin{equation*}
u / (s(1-u) + u) = \beta
\end{equation*}
we have
\begin{equation*}
s 
= (u / \beta - u) / (1-u)
= (\alpha / \beta - \alpha) / (1-\alpha).
\end{equation*}

In Section~7.12 we see that the limit exists and must be $0$ or $1$
almost everywhere. 
So a case like constructed above may happen only on a nullset (measure zero). 

%%%%%%%%%%%%%%
\begin{excopy}
Suppose that $E$ is a measurable set of real numbers with arbitrarily
small periods. Explicitly,
this means that there are positive numbers \(p_i\), converging to $0$
as \(i\to \infty\), so that 
\begin{equation*}
E + p_i = E \qquad (i=1,2,3,\ldots).
\end{equation*}

Prove that then either $E$ or its complement has measure $0$.

\emph{Hint:} Pick \(\alpha\in\R^1\), 
put \(F(x) = m(E\cap[\alpha,x])\) for \(x>\alpha\), show that
\begin{equation*}
F(x + p_i) - F(x - p_i) = F(y + p_i) - F(y - p_i)
\end{equation*}
if \(\alpha + p_i < x < y\). 
What does this imply about \(F(x)\) if \(m(E)>0\)?
\end{excopy}

Let \(n = \lfloor (y-x)/2p_i \rfloor\).
Then \(x - p_i < (y-2np_i) - p_i < (y-2np_i) + p_i\).
Now
\begin{eqnarray*}
F(x + p_i) - F(x - p_i) 
&=& m(E \cap [x-p_i, x+p_i]) \\
&=& m(E \cap ([x-p_i, (y-2np_i) - p_i] \cup [(y-2np_i) - p_i, x+p_i]) \\
&=& m(E \cap ([x-p_i, (y-2np_i) - p_i]) \\
& & + m(E \cap [(y-2np_i) - p_i, x+p_i]) \\
&=& m((E - 2p_i) \cap ([x-p_i, (y-2np_i) - p_i]) \\
& & + m(E \cup [(y-2np_i) - p_i, x+p_i]) \\
&=& m(E \cap [(y-2np_i) - p_i, x+p_i]) \\
& & + m(E \cap ([x+p_i, (y-2np_i) + p_i]) \\
&=& m(E \cap [(y-2np_i) - p_i, (y-2np_i) + p_i]) \\
&=& m(E \cap [y - p_i, y + p_i]) \\
&=& F(y + p_i) - F(y - p_i)
\end{eqnarray*}

By Theorem~7.18 \cite{RudinRCA87}, \(F(x)\) is differentiable
almost everywhere. By the above equality, its value is constant
simply by looking at 
\begin{equation*}
F'(x) = \lim_{j\to\infty} \frac{F(x+p_j) - F(x-p_j)}{2p_j}.
\end{equation*}
This value is actually the 
metric density
\index{metric density}
of $E$ and by the discussion in Section~7.12 \cite{RudinRCA87}
this constant must be $0$ or $1$. If \(m(E)>0\) it must be $1$
and thus \(F(x) = x-\alpha\) for \(x>\alpha\).
Since \(\alpha\) was arbitrary, for any interval $I$ we have
\(m(E\cap I) = m(I)\) and so \(m(\R\setminus E) = 0\).


%%%%%%%%%%%%%%
\begin{excopy}
Call $t$ a 
\emph{period}
\index{period}
of the function $f$ on \(\R^1\) if \(f(x+t)=f(x)\) for all \(x\in\R^1\).
Suppose $f$ is a real Lebesgue measurable function with periods $s$ and $t$
whose quotient is irrational.
Prove that there is a constant $c$ such that \(f(x)=c\;\aded\),
but that $f$ need not be constant.

\emph{Hint}: Apply Exercise~3 to the sets \(\{f>\lambda\}\).
\end{excopy}

We construct a sequence \(\{p_j\}_{j\in\N}\) such that 
\(\lim_{j\to\infty} p_j = 0\) and \(p_j > 0\) are periods of $f$.
Assume \(s>t\). Let (Similar to Euclid GCD algorithm)
\begin{align*}
p_1 &= s \\
p_2 &= t \\
p_n &= p_{n-2} - p_{n-1}\left\lfloor \frac{p_{n-2}}{p_{n-1}}\right\rfloor
       < p_{n-1}.
\end{align*}
Assume by negation that  $f$ is not constant almost everywhere.
We can find some \(\lambda\) such that 
\(\{f\leq\lambda\}\) and its complement \(\{f>\lambda\}\)
each is not a nullset.
The sequence \(\{p_j\}_{j\in\N}\) is also periods of these sets.
But by previous exercise, one of these two sets must be a nullset.


%%%%%%%%%%%%%% 5
\begin{excopy}
If \(A \subset \R^1\) and \(B \subset \R^1\), 
define \(A+B = \{a+b: a\in A, b\in B\}\).
Suppose \(m(A)>0\) and \(m(B)>0\).
Prove that \(A+B\) contains a segment, by completing the following outline:

There are points \(a_0\) and \(b_0\) where $A$ and $B$ have 
\index{metric density}
metric density~$1$.
Choose a small \(\delta>0\).
Put \(c_0 = a_0 + b_0\).
For each \(\epsilon\), positive or negative, define \(B_\epsilon\) 
to be the set of all \(c_0 + \epsilon - b\) for which 
\(b\in B\) and \(|b-b_0| < \delta\).
Then \(B_\epsilon \subset  (a_0 + \epsilon - \delta, a_0 + \epsilon + \delta)\).
If \(\delta\) was well chosen and \(|\epsilon|\) is sufficiently small,
it follows that $A$ intersects \(B_\epsilon\), so that 
\(A+B \supset (c_0 - \epsilon_0, c_0 + \epsilon_0)\) 
for some \(\epsilon_0 > 0\).

Let $C$ be 
\index{Cantor}
Cantor's ``middle thirds'' set and show that \(C+C\) is an interval,
although \(m(C)=0\).

(See also Exercise~19, Chap. 9.)
\end{excopy}

In this exercise solution, whenever we use $X$, we mean that
the expression or statement holds both for $A$ and for $B$.
\Wlogy, we may assume \(a_0=b_0=0\),
thus we have \(c_0=0\).
Take \(\delta>0\) such that 
\begin{align*}
m\bigl(X\cap (-h,h)\bigr)/2h &> 2/3 \\ 
\end{align*}
for any \(0<h<\delta\).
Pick \(\epsilon_0 = \delta/3\).
For \(\epsilon < \epsilon_0\), we have
\begin{equation*}
X_\epsilon = \epsilon - \bigl(X \cap (-\delta,+\delta)\bigr)
\subset (\epsilon - \delta, \epsilon + \delta).
\end{equation*}
Clearly \(m(X_\epsilon) = m(X\cap(-\delta,\delta))\).
Now
\begin{equation*}
\bigl(X\cap(\epsilon-\delta,\epsilon+\delta)\bigr)
\setminus
\bigl(X\cap(-\delta,\delta)\bigr)
\supset 
(\epsilon-\delta,\epsilon+\delta) \setminus  (-\delta,\delta)
\supset (\delta,\delta + \epsilon)
\end{equation*}
therefore,
\begin{align*}
   m\bigl(X \cap (\epsilon-\delta, \epsilon+\delta)\bigr)
&\geq  m\bigl(X \cap (-\delta, +\delta)\bigr) - \epsilon.
\end{align*}
Thus
\begin{align*}
m\bigl(A \cap (\epsilon-\delta, \epsilon+\delta)\bigr) + m(B_\epsilon)
&\geq m\bigl(A \cap (-\delta, +\delta)\bigr) - \epsilon 
      + m\bigl(B\cap(-\delta,\delta)\bigr) \\
&\geq \bigl((2/3)\cdot2\delta - \epsilon\bigr) + (2/3)\cdot2\delta \\
&> 2\bigl((2/3)\cdot2\delta - \epsilon\bigr)
 > 2(4\delta/3 - \epsilon_0) = 2\delta.
\end{align*}
and so \(A\cap B_\epsilon \neq \emptyset\).

Equivalently, for all \(\epsilon\in(-\epsilon_0,+\epsilon_0)\), 
there are \(a\in A\) and \(b\in B\) such that \(a=\epsilon -b\),
that shows that \(\epsilon\in A+B\) and actually
\((-\epsilon,\epsilon) \subset A+B\).
 
\paragraph{Cantor Set.} 
We will show that \(C+C = [0,2]\).
Any arbitrary \(x\in[0,2]\) can be represented as
\begin{equation*}
x = \sum_{n=0}^\infty t_n 3^{-n}
\end{equation*}
where \(t_0\in\{0,1\}\) and \(t_n\in\{0,1,2\}\) for \(n\geq 1\).
We will construct \(a,b\in C\) such that \(x=a+b\).
Put \(c_{-1} = 0\), using
\begin{equation*}
v_n = t_n + 3c_{n-1}
\end{equation*}
we define \(a_n\),\(b_n\) and \(c_n\) 
by induction for \(n\geq 0\) by the following 6 cases
\begin{equation*}
(a_n,b_n,c_n) = 
\left
\{\begin{array}{ll}
(0,0,0) \quad & v_n = 0 \\
(0,0,1) \quad & v_n = 1 \\
(2,0,0) \quad & v_n = 2 \\
(2,0,1) \quad & v_n = 3 \\
(2,2,0) \quad & v_n = 4 \\
(2,2,1) \quad & v_n = 5
\end{array}\right.
\end{equation*}
Note that \(a_0 = b_0 = 0\), so  if we put
\begin{equation*}
a = \sum_{n=0}^\infty a_n 3^{-n} \in C
\qquad
b = \sum_{n=0}^\infty b_n 3^{-n} \in C
\end{equation*}
we get the desired \(x = a + b \in C + C\) .
 

%%%%%%%%%%%%%%
\begin{excopy}
Suppose $G$ is a subgroup of \(\R^1\) (relative to addition),
\(G\neq \R^1\), and $G$ is Lebesgue measurable. Prove that then \(m(G)=0\).

\emph{Hint}: Use Exercise~5.
\end{excopy}

Assume by negation \(m(G)>0\). By previous exercise \(G+G\) contains a segment.
But \(G+G=G\) since it is a group. Thus $G$ contains an interval 
\([a,b]\in G\) such that \(a<b\).
Pick arbitrary \(x\in \R\). Define
\begin{align*}
d &= b - a > 0\\
n &= \lfloor x / d \rfloor \in \Z \\
r &= x - nd \in [a,b) \subset G
\end{align*}
Now 
\begin{align*}
x = n(b-a) + r \in G
\end{align*}
and so we get the \(\R \subset G\) contradiction.


%%%%%%%%%%%%%%
\begin{excopy}
Construct a continuous monotonic function $f$ on \(\R^1\) so that $f$ is not
constant on any segment although \(f'(x)=0\;\aded\)
\end{excopy}

The function constructed in Section~7.16 Example~\ich{b} satisfies
the requirements.

%%%%%%%%%%%%%% 8
\begin{excopy}
Let \(V=(a,b)\) be a bounded segment in \(\R^1\).
Choose segments \(W_n\subset V\) in  such a way that their union $W$ is dense 
in $V$ and the set \(K = V \setminus W\) has \(m(K)>0\).
Choose continuous function \(\varphi_n\) so that 
\(\varphi_n(x)=0\) outside \(W_n\), \(0<\varphi_n(x)<2^{-n}\) in \(W_n\).
Put \(\varphi = \sum \varphi_n\) and define
\begin{equation*}
T(x) = \int_a^x \varphi(t)\,dt \qquad (a<x<b).
\end{equation*}
Prove the following statements:
\begin{itemize}
\itemch{a} $T$ satisfies the hypothesis of Theorem~7.26, with \(X=V\).
\itemch{b} $T$ is continuous, \(T'(x)=0\) on $K$, \(m(T(K)) = 0\).
\itemch{c} If $E$ is a nonmeasurable subset of $K$ 
           (See Theorem~2.22) and \(A=T(E)\), then \(\chhi_A\) 
           is Lebesgue measurable but \(\chhi_A \circ T\) is not.
\itemch{d} The functions \(\varphi_n\) can be so chosen that the resulting $T$
           is an \emph{infinitely differentiable} homeomorphism of $V$ onto some
           segment in \(\R^1\) and \ich{c} still holds.

\end{itemize}
\end{excopy}

Let \(\{q_j\}_{j\in\N}\) be a sequence consisting of all of \(\Q\cap V\).
Put \(d=b-a\) and let \(W_1 = (a_1,b_1)\) be an open segment such that
\(q_1 \in W_1\subset V\) and \(d_1 = b_1 - a_1 < 2^{-1}d\).
By induction, assume \(W_j\) where chosen for \(j<n\).
Pick the first \(q_k\) in our sequence such that 
\(q_k \notin \cup_{j<n} W_j\) and choose \(W_n = (a_n,b_n)\) such that
\begin{align*}
q_n &\in W_n \subset V \setminus \left(\cup_{j<n} W_j\right) \\
d_n &= b_n - a_n < 2^{-n} d
\end{align*}
Clearly the chosen intervals \(\{W_j\}_{j\in\N}\) are disjoint
and satisfy the requirements.

We now use Exercise~7.1 in \cite{RudinPMA85}
and  Example~3.11 (page 40) in \cite{Gelb1996}.
For each interval \(W_n = (a_n,b_n)\) we define 
\begin{equation*}
g_n(x) = \left\{\begin{array}{ll}
c_n\exp\left((a-b) \,/\, (x-a_n)^2(b_n-x)^2\right) \qquad & a_n < x < b_n \\
0  & \mathrm{otherwise}.
\end{array}
\right.
\end{equation*}
and choose \(c_n\) such that \(\varphi_n(x) \leq  2^{-n+1 } < 2^{-n}\).
Actually 
\begin{equation*}
c_n =   2^{-n+1 } \exp\left((b-a) \,/\, (x-a_n)^2(b_n-x)^2\right).
\end{equation*}

Note that if
\begin{equation*}
g_a(x) = 
\left\{
\begin{array}{ll}
e^{-1/(x-a_n)^2} \quad & a_n < x  \\
0  & \mathrm{otherwise}.
\end{array}\right.
\qquad
g_b(x) = 
\left\{
\begin{array}{ll}
e^{-1/(b_n-x)^2} \quad &  x < b_n \\
0  & \mathrm{otherwise}.
\end{array}\right.
\end{equation*}
Then \(g_a^{(k)}(a_n) = g_b^{(k)}(b_n) = 0\) for all \(k\in\N\)
and for \(x\in (a_n,b_n)\) we have
\begin{align*}
g_a(x)g_b(x) 
&= e^{-1/(x-a)^2} e^{-1/(b-x)^2} 
 = \exp\left(- 1/(x-a)^2 - 1/(b-x)^2\right) \\
&= \exp\left(\frac{(x-b) - (x-a)}{(x-a)^2 (b-x)^2}\right)
 = \exp\left(\frac{a-b}{(x-a)^2 (b-x)^2}\right) \\
&= \varphi_n(x)
\end{align*}
Each \(\varphi_n\) is continuous, and since their support sets
are disjoint, \(\varphi\) is also continuous.
Now that $T$ is defined we prove the above.
\begin{itemize}

\itemch{a}

Now we show that $T$
satisfies Theorem~7.26's requirements.
\begin{itemize}

\itemch{i} 
\(X=V\) is a union of open intervals and so $V$ is open.
The continuity of $T$ follows from continuity of \(\varphi\).

\itemch{ii}
$X$ is open and so measurable.
Let \(x_1,x_2\in V=(a,b)\) and  \(x_1 < x_2\).
There must be some \(W_n = (a_n,b_n) \subsetneq (x_1,x_2)\).
Since 
\begin{equation*}
\int_{W_n} \varphi_n(t)\,dt > 0
\end{equation*}
We have \(T(x_1) < T(x_2)\) and so $T$ is injective.
The fact that \(\varphi\) is continuous
also implies that $T$ is differentiable on $X$.
and \(T'(x)=\varphi(x)\) for all \(x\in X\).

\itemch{iii}
Trivially, \(m(T(V-X))=0\) since \(V-X=\emptyset\).
\end{itemize}

\itemch{b}
As previously shown in \ich{a}-\ich{ii}, 
\(T'(x)=\varphi(x)\) for all \(x\in X\),
in particular \(T(x)=0\) for \(x\in K\).

\itemch{c}
Let \(E\subset K\) be a non measurable set.
then \(A=T(E)\subset T(K)\) and clearly Lebesgue measurable
since \(m(T(K)) = 0\).

\itemch{d}
We already took this infinitely differentiability into account
in our construction of \(\{\varphi_n\}_{n\in\N}\).
\end{itemize}

%%%%%%%%%%%%%%
\begin{excopy}
Suppose \(0<\alpha < 1\). Pick $t$ so that \(t^\alpha = 2\). Then
\(t>2\) and the construction of Example~\ich{b} in Sec.~7.16 can be
carried out with \(\delta_n = (2/t)^n\). Show that the resulting function $f$
\index{Lip@\(\Lip\)}
\index{Lipschitz condition}
belongs to \(\Lip \alpha\) on \([0,1]\).
\end{excopy}

For each \(n\in\N\) it is easy to see that
\begin{equation*}
\sup_{0\leq x <y\leq1} \frac{|f(y)-f(x)|}{y-x}
= \frac{f\bigl((2/t)^n/2^n\bigr) - f(0)}{
        \left((2/t)^n/2^n - 0\right)^\alpha}
= \frac{2^{-n}}{\left((2/t)^n/2^n\right)^\alpha}
= \frac{2^{-n}}{t^{-n\alpha}}
= (t^\alpha/2)^n.
\end{equation*}
Since \((t^\alpha/2)=1\) we have \(f_n\in\Lip\alpha\)
which is a Banach space and closed. Hence \(f\in\Lip\alpha\).


%%%%%%%%%%%%%%
\begin{excopy}
If 
\index{Lip@\(\Lip\)}
\index{Lipschitz condition}
\(f\in \Lip 1\) on \([a,b]\), prove that $f$ is absolutely continuous
and that \(f'\in L^\infty\).
\end{excopy}

Pick arbitrary \(\epsilon>0\), choose some \(\delta < \epsilon\).
For any set of disjoint intervals \(\{(\alpha_j,\beta_j)\}_{j\in\N}\)
such that \(\sum_j (\beta_j - \alpha_j) < \delta\) we have
\begin{equation*}
\sum_j |f(\beta_j) - f(\alpha_j)|
\leq \sum_j 1\cdot|\beta_j - \alpha_j|
= \sum_j \beta_j - \alpha_j < \delta < \epsilon
\end{equation*}
Thus $f$ is absolutely continuous.
By Theorem~7.18 $f$ is differentiable \aded\ and 
by \(f\in\Lip 1\) we have \(\|f'\|_{\infty} \leq 1\) and thus
\(f'\in L^\infty\).


%%%%%%%%%%%%%% 11
\begin{excopy}
Assume that \(1<p<\infty\), $f$ is absolutely continuous on \([a,b]\),
\(f'\in L^p\), and \(\alpha = 1/q\) where $q$ is the exponent conjugate to $p$.
Prove that \(f\in \Lip\alpha\).
\end{excopy}

{\small (From Curtis T.~McMullen's\\
\texttt{\scriptsize
www.math.harvard.edu/{\~{}}ctm/home/text/class/harvard/212a/03/html/home/course/course.pdf})}

If \(a\leq x<y\leq b\) then
\begin{equation*}
|f(y)-f(x)| 
\leq \int_a^b |f'(t)| \cdot \chhi_{[x,y]}\,dt
\leq \|f'\|_p \cdot \|\chhi_{[x,y]}\|_q 
=    \|f'\|_p \cdot(y-x)^{1/q}.
\end{equation*}
Hence 
\begin{equation*}
\frac{|f(y)-f(x)|}{(y-x)^\alpha} < \|f'\|_p < \infty.
\end{equation*}
and \(f\in \Lip_\alpha\).





%%%%%%%%%%%%%% 12
\begin{excopy}
Suppose 
\label{ex:7.12}
\(\varphi: [a,b]\to\R^1\) is nondecreasing.
\begin{itemize}

\itemch{a} 
Show that there is a left-continuous nondecreasing $f$ on \([a,b]\),
so that \(\{f\neq\varphi\}\) is at most countable. 
[Left-continuous means: if \(a<x\leq b\) and \(\epsilon > 0\), then there is 
a \(\delta>0\) so that \(|f(x) - f(x-t)|<\epsilon\) whenever \(0<t<\delta\).]

\itemch{b}
Imitate the proof of Theorem~7.18 to show that there is a positive Borel measure
\(\mu\) on \([a,b]\) for which
\begin{equation*}
f(x) - f(a) = \mu([a,x)) \qquad (a\leq x \leq b).
\end{equation*}

\itemch{c}
Deduce from \ich{b} that \(f'(x)\) exists \aded \([m]\), 
that \(f'\in L^1(m)\), and that
\begin{equation*}
f(x) - f(a) = \int_a^b f'(t)\,dt + s(x) \qquad (a\leq x \leq b)
\end{equation*}
where $s$ is a nondecreasing and \(s'(x)=0\;\aded[m]\).

\itemch{d}
Show that \(\mu\perp m\) if and only if \(f'(x)=0\;\aded[m]\),
and that \(\mu \ll m\) if and only if $f$ is AC on \([a,b]\).

\itemch{e}
Prove that \(\varphi'(x) = f'(x)\;\aded[m]\).
\end{itemize}
\end{excopy}

\begin{itemize}
\itemch{a}
If \(x\in(a,b)\) is a point of discontinuity than
\begin{equation*}
\sup_{t<x} \varphi(t) < \inf_{t>x} \varphi(t) 
\end{equation*}
There could be at most countable number of such discontinuity points.
Otherwise we have an uncountable sum of positive numbers
which must be less than \(b-a\).
Now we define \(f(x) = \varphi(a)\) and for \(x\in(a,b)\)
\begin{equation} \label{eq:ex7.12e:f}
f(x) = \sup_{a\leq t < x} \varphi(t).
\end{equation}
which is clearly left-continuous and differs from \(\varphi\)
only where the latter is discontinuous.

\itemch{b}
Define \((g(x) = f(x) + x\), clearly $g$ is monotonic, one to one and 
left-continuous. Let \(V\subset (a,b)\) some open set.
We will show that \(g(V)\) is a Borel set.
Recalling that a monotonic function can have at most countable
number of points of discontinuity.
We can represent $V$ is a countable disjoint union of intervals
\(V = \disjunion_{j\in\N} I_j\) such that each \(I_j\)
is open, closed or half open, and $g$ is continuous on it.
Clearly, \(g(I_j)\) is an interval --- open, closed or half open.
Thus \(g(V)\) is a Borel set. By being one-to-one, 
\begin{equation*}
g\bigl((a,b)\setminus V\bigr) = \bigl(g(a),g(b)\bigr) \setminus g(V) 
\end{equation*}
and so \(g(C)\) is a Borel set for any closed set $C$
and consequently 
\(g(E)\) is a Borel set for any Borel set $E$.

Now we can define for each Borel set $E$
\begin{align*}
\nu(E) &= m\bigl(g(E)\bigr) \\
\mu(E) &= \nu(E) - m(E).
\end{align*}
If follows that for each \(x\in(a,b)\)
\begin{align*}
g(x) - g(a) &= \nu([a,x]) \\
f(x) - f(a) &= \nu([a,x]) - (x-a) = \mu([a,x)).
\end{align*}

\itemch{c}
By Lebesgue differentiation theorem (local lemma~\ref{lem:leb:diff})
\(f'\) exists almost everywhere and \(f'\in L^1([a,b],m)\). Put 
\begin{equation} \label{eq:ex7.12:sx}
s(x) = f(x) - f(a) - \int_a^x f'(t)\,dm(t).
\end{equation}
By Theorem~7.11 \cite{RudinPMA85} 
\begin{equation*}
\frac{d}{dx} \int_a^x f'(t)\,dm(t) = f'(x) \quad\aded
\end{equation*}
Hence \(s'(x) = 0\; \aded\).


\itemch{d}
Let 
\begin{equation*}
\mu = \mu_a + mu_s \qquad  d\mu = h\,dm + d\mu_s
\end{equation*}
be the unique and positive Lebesgue decomposition 
of Radon-Nikodym Theorem~6.10 \cite{RudinRCA87}. 
By Theorem~7.14 \cite{RudinRCA87}.
\begin{equation*}
\lim_{t\to 0+} \frac{\mu\bigl([x+t)\bigr)}{m\bigl([x+t)\bigr)}
= \lim_{t\to 0+} \frac{\mu\bigl([x+t)\bigr)}{t}
= h(x) \quad \aded[m].
\end{equation*}
But
\begin{equation*}
\lim_{t\to 0+} \frac{\mu\bigl([x+t)\bigr)}{t} = f'(x) \quad \aded[m].
\end{equation*}
So by the the uniqueness of $h$, we have \(f' = h\;\aded[m]\).
Hence, by \eqref{eq:ex7.12:sx} we have \(\mu_s([a,x)] = s(x)\).

\paragraph{Equivalence 1.}
Assume \(\mu\perp m\), then \(\mu_a = h\,dm \perp m\).
Since $h$ is positive \(h = 0\;\aded[m]\) and so \(f'=0\;\aded[m]\).
Conversely, assume  \(f'=0\;\aded[m]\). Then \(\mu_a = h\,dm = 0\)
and thus \(\mu_a \perp m\) trivially.
Since \(\mu_s \perp m\) so is \(\mu\).

\paragraph{Equivalence 2.}
Assume  \(\mu \ll m\). Then $f$ maps nullsets to nullsets and 
by Theorem~7.18 \cite{RudinRCA87} $f$ is absolutely continuous.
Conversely, assume $f$ is absolutely continuous.
then by Theorem~7.18 \cite{RudinRCA87} 
\begin{equation*}
f(x) - f(a) = \int_a^x f'(t)\,dt  \qquad a\leq x \leq b
\end{equation*}
Thus \(\mu_s = 0\) and so \(\mu = \mu_a\) and \(\mu \ll m\).

\itemch{e}
We know that by construction \eqref{eq:ex7.12e:f}
 \(f \leq \varphi\) 
and \(f = \varphi\;\aded[m]\) and both are nondecreasing.
Pick \(x\in[a,b]\) such that \(f(x) = \varphi(x)\) and \(f'(x)\) exists.
Hence if \(h>0\) and \(x\pm h \in [a,b]\) then
\begin{equation*}
     \frac{\varphi(x-h) - \varphi(x)}{-h} 
\leq f'(x) 
\leq \frac{\varphi(x+h) - \varphi(x)}{h}.
\end{equation*}
Therefore
\begin{align*}
     \varliminf_{h\to x^-} \frac{\varphi(x+h) - \varphi(x)}{h}
&\leq \varlimsup_{h\to x^-} \frac{\varphi(x+h) - \varphi(x)}{h} \\
&\leq f'(x) \\
&\leq \varliminf_{h\to x^+} \frac{\varphi(x+h) - \varphi(x)}{h}
\leq \varlimsup_{h\to x^+} \frac{\varphi(x+h) - \varphi(x)}{h}.
\end{align*}

We will show 
\begin{equation} \label{eq:ex7.12e:desired}
\varlimsup_{h\to x^+} \frac{\varphi(x+h) - \varphi(x)}{h} = f'(x).
\end{equation}
The analog equality
\begin{equation*}
\varliminf_{h\to x^-} \frac{\varphi(x+h) - \varphi(x)}{h}  = f'(x)
\end{equation*}
can be shown with similar arguments. Thus the above inequalities
are actually equalities. Hence \(\varphi'(x) = f'(x)\).

By definition of \(\varliminf\), we have
a strictly decreasing sequence 
\(\{h_j\}_{j\in\N}\) 
of positive numbers
such that
\begin{align*}
\lim_{j\to\infty} h_j &= 0 \\
\lim_{j\to\infty} \frac{\varphi(x+h_j) - \varphi(x)}{h_j}
&=\varlimsup_{h\to x^+} \frac{\varphi(x+h) - \varphi(x)}{h}
\end{align*}
By the construction of $f$ in \eqref{eq:ex7.12e:f} 
the ``modified'' set \(S=\{x\in[a,b]: f(x)\neq \varphi(x)\}\)
is countable.
We also note that for \(\varphi(x_1) \leq f(x_2)\)
whenever \(a\leq x_1\leq x_2 \leq b\).
Thus in each segment \(H_n = [x + h_{n}, x + h_{n-1}]\)
we can find a sequence \(G_n = \{g_j\}_{j\in\N}\) such that
\begin{align*}
f(g_j) &= \varphi(g_j) \qquad (\mathrm{since}\;g_j\notin S) \\
\lim_{j\to\infty} g_j &= x + h_{n} 
\end{align*}
Hence
\begin{equation*}
\lim_{j\to\infty} f(g_j) \geq \varphi(x + h_{n})
\end{equation*}
and therefore
\begin{equation*}
\lim_{j\to\infty} \frac{\varphi(g_j) - \varphi(x)}{g_j - x}
 \geq \lim_{j\to\infty} \frac{f(g_j) - f(x)}{g_j - x}
 \geq \frac{\varphi(x+h_{n}) - \varphi(x)}{h_{n}}.
\end{equation*}
By a diagonal process we can pick 
a sequence \(\{x_j\}_{j\in\N}\) such that for all $n$
\begin{align*}
x_n &\in G_n \subset H_n \\
\lim_{n\to\infty} \frac{f(x_n) - f(x)}{x_n - x}
&\geq  \frac{\varphi(x+h_{n}) - \varphi(x)}{h_{n}} + \frac{1}{n}
\end{align*}
Clearly 
\begin{align*}
\lim_{n\to\infty} x_n &= x \\
f'(x) &= \lim_{n\to\infty} \frac{f(x_n) - f(x)}{x_n - x}
 \geq  \lim_{n\to\infty} \frac{\varphi(x+h_{n}) - \varphi(x)}{h_{n}}.
\end{align*}
Hence \eqref{eq:ex7.12e:desired} is true.

\end{itemize}


%%%%%%%%%%%%%% 13
\begin{excopy}
Let \(BV\) be the class of all $f$ on \([a,b]\) that have bounded variation
on \([a,b]\), as defined after Theorem~7.19. Prove the following statements.
\begin{itemize}
\itemch{a} 
Every monotonic bounded function on \([a,b]\) is in \(BV\).

\itemch{b}
If \(f \in BV\) is real, there exists bounded monotonic functions
\(f_1\) and \(f_2\) so that \(f=f_1-f_2\).\\
\emph{Hint}: Imitate the proof of Theorem~7.19.

\itemch{c}
If \(f\in BV\) is left-continuous then \(f_1\) and \(f_2\) 
can be chosen in \ich{b} so as to be also left-continuous.

\itemch{d}
If \(f\in BV\) is left-continuous then there is a Borel measure \(\mu\) 
on \([a,b]\) that satisfies
\begin{equation*}
f(x) - f(a) = \mu([a,x)) \qquad (a\leq x \leq b);
\end{equation*}
\(\mu\ll m\) if and only if $f$ is AC on \([a,b]\).

\itemch{e}
Every \(f\in BV\) is differentiable \(\aded[m]\), and \(f'\in L^1(m)\).
\end{itemize}
\end{excopy}

\begin{itemize}
\itemch{a}
Let $f$ be monotonic on \([a,b]\).
If 
\begin{equation*}
a\leq t_0 < t_1 < \cdots < t_N\leq b
\end{equation*}
then clearly 
\begin{equation*}
\sum_{j=1}^N |f(t_j) - f(t_{j-1})|
= \left| \sum_{j=1}^N \bigl(f(t_j) - f(t_{j-1})\bigr) \,\right|
= |f(t_N)-f(t_0)|.
\end{equation*}
Therefore,
\begin{equation*}
\sup_{a=t_0 < t_1 < \cdots < t_N=b} \sum_{j=1}^N |f(t_j) - f(t_{j-1})|
= |f(b)-f(a)|
\end{equation*}

\itemch{b}
By Theorem~7.19 \cite{RudinRCA87}
if
\begin{equation}  \label{eq:Fx:ex:7.13b}
F(x) = 
\sup_{a=t_0 < t_1 < \cdots < t_N=x} \sum_{j=1}^N |f(t_j) - f(t_{j-1})|
= |f(b)-f(a)|
\end{equation}
then \(F\pm f\) are nondecreasing and continuous.
Now
\begin{equation*}
f_1 = (F+f)/2 \qquad f_2 = -(F-f)/2
\end{equation*}
satisfy the requirements.

\itemch{c}
Assume $f$ is left-continuous, then $F$ defined in \eqref{eq:Fx:ex:7.13b}
is left-continuous as well. To show this pick \(x\in[a,b]\) and \(\epsilon>0\).
Take \(\delta>0\) such that \((a\leq x-\delta\) and 
\begin{equation*}
\sum_{j=1}^n |f(b_j) - f(a_j)| < \epsilon
\end{equation*}
whenever 
\begin{equation*}
\sum_{j=1}^n b_j - a_j < \delta \qquad  \forall j\, a\leq a_j\leq b_j\leq b
\end{equation*}
Now we can see that 
\begin{equation*}
F(x-\delta) + \epsilon \geq F(x).
\end{equation*}
Thus our construction satisfy the left-continuous requirement.

\itemch{d}
Assume such required Borel measure \(\mu\) exists.
Now $f$ maps sets of measure $0$ to sets of measure $0$ since 
\(\mu \ll m\). By Theorem~7.18 \cite{RudinRCA87} $f$ is absolutely continuous.

Conversely, assume $f$ is absolutely continuous.
By Theorem~7.20 \cite{RudinRCA87} \(f'\in L^1(m)\) and we can define
\begin{equation*}
\mu(E) = \int_E f'(x)\,dm(x)
\end{equation*}
for each Borel set $E$. Now \(\mu\) is a Borel measure and \(\mu \ll m\).

\itemch{e}
By \ich{b} we can represent \(f=f_1-f_2\) where \(f_1,f_2\) are monotonic.
As we saw in Exercise~12\ich{a} these functions of at most countable
number of points where they are not continuous.
By Exercise~\ref{ex:7.12}\ich{c}
\(f_j'\) exist \aded\ and \(f_j'\in L^1(m)\) for \(j=1,2\)
and so is \(f' = f_1' - f_2'\).

\paragraph{Note:} Here we \emph{cannot} show that 
\(f(x)-f(a) = \int_a^x f't)\,dt\). Absolute continuity is missing for this.
\end{itemize}

%%%%%%%%%%%%%% 
\begin{excopy}
Show that the product of two absolutely continuous functions on \([a,b]\)
is absolutely continuous. Use this to derive a theorem about integration
by parts.
\end{excopy}

Let \(f_1,f_2\) be two absolutely continuous functions and \(\epsilon>0\).
Thus there is a \(\delta>0\) such that 
\begin{equation*}
\sum_{j=1}^N |f(t_j) - f(t_{j-1})|<\epsilon
\end{equation*}
if 
\begin{equation} \label{eq:ex7.14}
\sum_{j=1}^N |t_j - t_{j-1}|<\delta.
\end{equation}
Put \(f = f_1\cdot f_2\) and 
for abbreviation put \(a_k = f_1(a_k)\) and \(b_k = f_2(a_k)\). 
Using:
\begin{align*}
|f(t_j) - f(t_{j-1})|
&= |(f_1\cdot f_2)(t_j) - (f_1\cdot f_2)(t_{j-1})|
 = |a_j b_j - a_{j-1}b_{j-1}| \\
&= \left|
    \bigl(a_{j-1} + (a_j - a_{j-1})\bigr)
    \bigl(b_{j-1} + (b_j - b_{j-1})\bigr) - a_{j-1}b_{j-1}
   \right| \\
&= \bigl|
      (a_j - a_{j-1})b_{j-1} 
    + (b_j - b_{j-1})a_{j-1} 
    + (a_j - a_{j-1}) (b_j - b_{j-1}) 
   \bigr| \\
&\leq 
    |a_j - a_{j-1}|\cdot \|f_2\|_\infty
    + |b_j - b_{j-1}|\cdot \|f_1\|_\infty
    + |a_j - a_{j-1}|\cdot |b_j - b_{j-1}| \\
&\leq 
    |a_j - a_{j-1}|\cdot 3\|f_2\|_\infty
    + |b_j - b_{j-1}|\cdot \|f_1\|_\infty
\end{align*}
Assuming \eqref{eq:ex7.14} we have:
\begin{align*}
\sum_{j=1}^N |f(t_j) - f(t_{j-1})|
&\leq
    3\|f_2\|_\infty \sum_{j=1}^N  |a_j - a_{j-1}|
  + \|f_1\|_\infty \sum_{j=1}^N  |b_j - b_{j-1}|  \\
&\leq \left(\|f_1\|_\infty + 3\|f_2\|_\infty \right) \epsilon
\end{align*}
Thus $f$ is absolutely continuous as well.

Now assume \(u,v: [a,b]\to\C\) are absolutely continuous functions.
By the above result, they are simultaneously differentiable almost everywhere.
Hence by Leibnitz's rule
\begin{equation*}
\frac{d\bigl(f(x)g(x)\bigr)}{dx} = f'(x)g(x) + f(x)g'(x) \; \aded
\end{equation*}
Consequently
\begin{equation*}
f(b) - f(a) =
\int_a^b \frac{d\bigl(f(x)g(x)\bigr)}{dx}\,dx 
= \int_a^b f'(x)g(x)\,dx + \int_a^b f(x)g'(x)\,dx.
\end{equation*}



%%%%%%%%%%%%%% 15
\begin{excopy}
Construct a monotonic functions $f$ on \(\R^1\) so that \(f'(x)\) exists
(finitely) for every \(x\in \R^1\), but \(f'\) is not a continuous function.
\end{excopy}

Based on \cite{Gelb1996} Chapter~3 counterexample~2 page~36. Let
\begin{equation*}
f(x) = \left\{
\begin{array}{ll}
x^2\sin(1/x) - 2x^2 + x \quad & \mathrm{if}\; x < 0 \\
0            \quad & \mathrm{if}\; x =  0 \\
x^2\sin(1/x) + 2x^2 + x \quad & \mathrm{if}\; x > 0
\end{array}
\right.
\end{equation*}
Now
\begin{equation*}
f'(x) = \left\{
\begin{array}{ll}
2\bigl(x\sin(1/x) + 2|x|\bigr) + \bigl(2-\cos(1/x)\bigr)
    \quad & \mathrm{if}\; x\neq 0 \\
1   \quad & \mathrm{if}\; x=0 \\
\end{array}
\right.
\end{equation*}
Clearly \(f'>0\) but is not continuous in $0$ since \(\cos(1/x)\) is not.


%%%%%%%%%%%%%%
\begin{excopy}
Suppose \(E\subset [a,b]\), \(m(E)=0\).
Construct an absolutely continuous monotonic function $f$ on \([a,b]\)
so that \(f'(x)=\infty\) at every \(x\in E\).

\emph{Hint}: \(E \subset \cap V_n\), \(V_n\) open, \(m(V_n)<2^{-n}\).
Consider the sum of characteristic functions of these sets.
\end{excopy}

Following the hint.
Let \(V_n\) as suggested, but also require \(V_{n+1}\subset V_n\).
This could be easily done, by possibly replacing
\begin{equation*}
V_n \leftarrow \cap_{k\leq n} V_n
\end{equation*}
Define
\begin{equation*}
g(x) = \sum_{n\in\N} \chhi_{V_n}
\end{equation*}
Thus \(g(x)=0\) iff \(x\in E\) and \(g\in L^1([a,b])\).
Now define
\begin{equation*}
f(x) = \int_a^x g(t)\,dt.
\end{equation*}
By Exercise~10\ich{a} of Chapter~6 \cite{RudinRCA87}
$g$ is uniformly integrable (by itself).
Thus $f$ is absolutely continuous.
Now let \(x\in E\) and \(M\in\N\).
There is some \(\delta>0\) such that
\((x-\delta,x+\delta) \subset V_M\) and 
\begin{equation*}
\frac{f(x+h) - f(x)}{h} 
= \frac{1}{h}\int_x^{x+h} g(t)\,dt
\geq \frac{1}{h}\int_x^{x+h} \sum_{n=1}^M \chhi_{V_n}\,dt
= Mh/h = M.
\end{equation*}
Hence \(f'(x)=\infty\).



 By Theorem~7.19
$f$ is differentiable \aded\ and
\begin{equation*}
f(x) = \int_a^x f'(t)\,dt \qquad \forall x\in[a,b]
\end{equation*}


%%%%%%%%%%%%%% 17
\begin{excopy}
Suppose \(\{\mu_n\}\) is a sequence of positive Borel measures on \(\R^k\) and
\begin{equation*}
\mu(E) = \sum_{n=1}^\infty \mu_n(E).
\end{equation*}
Assume \(\mu(\R^k)<\infty\). Show that \(\mu\) is a Borel measure.
What is the relation between Lebesgue decomposition of the \(\mu_n\) 
and that of \(\mu\)?

Prove that 
\begin{equation*}
(D\mu)(x) = \sum_{n=1}^\infty (D\mu_n)(x) \quad \aded[m].
\end{equation*}
Derive corresponding theorems for sequences \(\{f_n\}\) 
of positive nondecreasing functions on \(\R^1\) and their sums \(f=\sum f_n\).
\end{excopy}

Let \(\mu_n = \mu_{n,a} + \mu_{n,s}\) be the Lebesgue decomposition.
Clearly 
\(\sum_{n=1}^\infty \mu_{n,a}\)
and
\(\sum_{n=1}^\infty \mu_{n,s}\)
are positive Borel measures and
\begin{equation*}
\sum_{n=1}^\infty \mu_{n,a} \ll m 
\qquad
\sum_{n=1}^\infty \mu_{n,s} \perp m.
\end{equation*}
By uniqueness, these make the Lebesgue decomposition of \(\mu\). 

By Theorem~7.14 \cite{RudinRCA87} and the above observation, if 
\(d\mu = f\,dm + d\mu_s\) and
\(d\mu_n = f_n\,dm + d\mu_{n,s}\)
then 
\begin{equation*}
(D\mu)(x) = f(x) = \sum_{n=1}^\infty f_n(x) = \sum_{n=1}^\infty (D\mu_n)(x) 
\quad\aded[m].
\end{equation*}

Say \(\{f_n\}_n\in\N\) are positive nondecreasing functions on \(\R^1\)
such that 
\begin{equation*}
f = \sum_{n=1}^\infty f_n \in L^\infty(\R,m).
\end{equation*}
Then
\begin{equation*}
f'(x) = \sum_{n=1}^\infty f_n'(x) \qquad \aded[m].
\end{equation*}
This can be shown by using Exercise~12 and replacing $a$ and \(f(a)\)
by \(-\infty\) and 
\begin{equation*}
\lim_{t\to-\infty} f(t) = \inf_{t\in\R} f(t).
\end{equation*}


%%%%%%%%%%%%%% 18
\begin{excopy}
Let \(\varphi_0(t) = 1\) on \([0,1)\), \(\varphi_0(t) = -1\)
on \([1,2)\), extended \(\varphi_0\) to \(\R^1\) so as to have a period $2$,
and define \(\varphi_n(t) = \varphi_0(2^n t)\), \(n=1,2,3,\ldots\)

\textsl{
Assume that \(\sum |c_n|^2 < \infty\) and prove that the series
\begin{equation} \label{eq:ex7.18}
\sum_{n=1}^\infty c_n\varphi_n(t)
\end{equation}
converges then for almost every $t$.
}

Probabilistic Interpretation: The series \(\sum(\pm c_n)\) converges
with probability $1$.\\
\emph{Suggestion}: \(\{\varphi_n\}\) is orthonormal on \([0,1]\), 
hence \eqref{eq:ex7.18} is the Fourier series of some \(f\in L^2\).
If \(a = j\cdot 2^{-N}\), \(b = (j+1)\cdot 2^{-N}\), \(a<t<b\),
and \(s_N = c_1\varphi_1 + \cdots + c_N\varphi_N\), then, for \(n>N\),
\begin{equation*}
s_N(t) = \frac{1}{b-a} \int_a^b s_N\,dm = \frac{1}{b-a} \int_a^b s_n\,dm,
\end{equation*}
and the last integral converges to \(\int_a^b f\,dm\), as \(n\to\infty\).
Show that \eqref{eq:ex7.18} at almost every Lebesgue point of $f$.
\end{excopy}

If \(m<n\) then within any segment \([j 2^{-n},  (j+1) 2^{-n}]\) clearly
\(\varphi_m\) is constant, while \(\varphi_n\) equally alternates signs
in \(2^{n-m}\) subsegments. Thus \(\varphi_m \perp \varphi_n\).
This argument also shows that the above equality of the integrals holds.

The existence of $f$ is implied by the completeness of 
the Hilbert space \(L^2([0,1],m)\).

Let $x$ be a Lebesgue point
\index{Lebesgue point}
of $f$ such that \(x\in B\) --- the countable \(B = \{j2^{-n}: j,n\in\Z^+\}\). 
We will define
nicely shrinking sets
\index{nicely shrinking sets}
for $x$ with \(\alpha = 1/2\). 
Let \(r_n=2^{-n}\) 
and pick \(E_n = [jr_n/2, (j+1)r_n/2]\subset [0,1]\)
such that \(x\in E_n\). 
We have 
\(E_n \subset B(x,r_n)\) and
\begin{equation*}
m(E_n) = r_n/2 = \alpha r_n = m\bigl(B(x,r_n)\bigr).
\end{equation*}
Now by the above observation and Theorem~7.10 \cite{RudinRCA87}
\begin{equation*}
f(x) 
= \lim_{n\to\infty} \frac{1}{m(E_n)} \int_{E_n} f\,dm
= \lim_{n\to\infty} s_n(x).
\end{equation*}
Thus the series converges for all Lebesgue points except for at most 
a countable set.


%%%%%%%%%%%%%% 19
\begin{excopy}
Suppose $f$ is continuous on \(\R^1\), \(f(x)>0\) if \(0 < x < 1\), \(f(x)= 0\)
otherwise. Define
\begin{equation*}
  h_c(x) = \sup \{n^c f(nx): n = 1,2,3,\ldots\}.
\end{equation*}
Prove that
\begin{itemize}
\itemch{a} \(h_c\) is in \(L^1(\R^1)\) if \(0<c<1\),
\itemch{b} \(h_1\) is in weak \(L^1\) but not in \(L^1(\R^1)\),
\itemch{c}  \(h_c\) is not weak \(L^1\) if \(c>1\).
\end{itemize}
\end{excopy}

\textbf{Remimder:} A function $f$ is 
\emph{weak} \(L^1\)\
\index{weak L1@weak \(L^1\)}
if 
\begin{equation*}
\lambda\cdot m\left( \{x: |f(x)| > \lambda\}\right)
\end{equation*}
is a bounded function of \(\lambda\).

For abbreviation we put \(M = \|f\|_\infty\).
For any $c$ we have.
\begin{align*}
\int_{\R^1} |h_c(t)|\,dt
&=    \int_0^1 h_c(t)\,dt
 =    \sum_{m=1}^\infty \int_{1/(m+1)}^{1/m} h_c(t)\,dt \\
&=    \sum_{m=1}^\infty \int_{1/(m+1)}^{1/m} 
      \bigl(\sup_{n\in\N} n^c f(nt)\bigr)\,dt 
 =    \sum_{m=1}^\infty \int_{1/(m+1)}^{1/m} 
      \bigl(\sup_{1\leq n\leq m+1} n^c f(nt)\bigr)\,dt
\end{align*}
and also put \(I_m=[1/(m+1),1/m]\)
the mid-third \(K= [1/2,2/3]\) and using continuity, let 
\begin{equation*}
T = \inf\{f(x): x\in K\} = \min\{f(x): x\in K\} > 0.
\end{equation*}

\begin{itemize} 
\itemch{a}
If \(0<c<1\) then
\begin{align*}
\int_{\R^1} |h_c(t)|\,dt
&\leq \sum_{m=1}^\infty \int_{1/(m+1)}^{1/m} (m+1)^c M\,dt \\
&=    M \sum_{m=1}^\infty (m+1)^c \left(\frac{1}{m+1} - \frac{1}{m}\right) 
 =    M \sum_{m=1}^\infty (m+1)^c \frac{1}{m(m+1)} \\
&=    M \sum_{m=1}^\infty m^{-1}(m+1)^{c-1} \\
&\leq M \sum_{m=1}^\infty (m+1)^{c-2} \\
&\leq M \int_1^\infty x^{c-2}\,dx 
=     \frac{M}{c-1}x^{c-1}\bigm|_1^\infty = M(1-c) < \infty.
\end{align*}
Hence \(h_c\in L^1(\R)\).

\itemch{b}
clearly \(m\cdot I_{2m} \subset K\) for 
any \(m\geq 1\). Thus we have the following  ``generous'' estimation:
%  (m/2) [1/(m+1),1/m]  \subset [1/3,2/3]
\begin{align*}
\int_{\R^1} |h_1(t)|\,dt
&\geq \sum_{m=1}^\infty \int_{I_m} h_1(t)\,dt 
 \geq \sum_{m=1}^\infty \int_{I_{2m}} h_1(t)\,dt 
 \geq \sum_{m=1}^\infty \int_{I_{2m}} 
      \bigl(\sup_{1\leq n\leq 2m+1} n^1 f(nt)\bigr)\,dt \\
&\geq \sum_{m=1}^\infty \int_{I_{2m}} m T 
 =    T \sum_{m=1}^\infty m \ell(I_{2m})  
 =    T \sum_{m=1}^\infty m \frac{1}{2m(2m+1)} \\
&=    (T/2) \sum_{m=1}^\infty  \frac{1}{2m+1}
 \geq (T/2) \sum_{m=1}^\infty  \frac{1}{2m+2}
 =    (T/4) \sum_{m=2}^\infty  \frac{1}{m} = \infty.
\end{align*}
Hence \(h_1 \notin L^1(\R)\).

To show that \(h_1\) is weak \(L^1\) we first assume that \(\lambda < M\).
Then
\begin{equation*}
\lambda \cdot m\bigl( \{x\in\R: |h_1(x)| > \lambda\}\bigl)
\leq M \cdot \ell([0,1]) = M.
\end{equation*}
Now assume that \(\lambda \geq M\). If \(|h_1(x)|> \lambda\)
then \(n\cdot f(nx) > \lambda\) for some $n$. 
Hence \(nM > \lambda\) and  \(x < 1/n\).
Thus \(n \geq \lfloor \lambda/M \rfloor \geq 1\) and 
\begin{equation*}
0 < x < 1/\lfloor \lambda/M \rfloor.
\end{equation*}
Since
\begin{equation*}
\lambda < \left(\lfloor \lambda/M\rfloor + 1\right)M 
        \leq 2 \lfloor \lambda/M\rfloor M 
\end{equation*}
we can estimate
\begin{align*}
\lambda \cdot m\bigl( \{x\in\R: |h_1(x)| > \lambda\}\bigl) 
&<  \lambda /\lfloor \lambda/M \rfloor < 2M.
\end{align*}
In both cases the expression is bounded for any \(\lambda\) and 
\(h_1\) is weak \(L^1\).

\itemch{c}
By checking odd and even cases \(\lfloor n/2\rfloor I_n \subset K\)
for all \(n\geq 2\). 
It is sufficient to consider \(\lambda \geq T\).
If 
\begin{equation*}
n \geq 2\left\lceil \left(\frac{\lambda}{T}\right)^{1/c} \right\rceil
  \geq        \left(\frac{\lambda}{T}\right)^{1/c} 
\end{equation*}
then
\begin{equation*}
n^c T > \lambda.
\end{equation*}
Therefore if \(x \in I_{2n+1} \cup I_{2n}\) then \(nx \in K\) 
and \(h_c(x) \geq \lambda\), consequently
\begin{equation*}
(0,1/2n] \subset \{x\in\R: h_c(x)> \lambda\}.
\end{equation*}
Now since \(\lambda/T \geq 1\) we see that
\begin{align*}
\lambda \cdot m\bigl( \{x\in\R: |h_1(x)| > \lambda\}\bigl) 
&\geq \lambda \cdot \ell\bigl((0,1/n]\bigr) \\
&\geq \lambda \bigm/ 
      \left\lceil (\lambda/T)^{1/c} \right\rceil 
 \geq \lambda \bigm/ 2 (\lambda/T)^{1/c} \\
&= \left(T^{-c}/2\right) \lambda^{1-1/c}
\end{align*}
is clearly unbounded for \(\lambda\).
Hence \(h_c\) is not weak \(L^1(\R)\).
\end{itemize}


%%%%%%%%%%%%%% 20
\begin{excopy}
\begin{itemize}
\itemch{a}
For any set \(E\subset \R^2\), the boundary \(\partial E\) of $E$ is,
by definition the closure of $E$ minus the interior of $E$. Show that
$E$ is Lebesgue measurable whenever \(m(\partial E) = 0\).

\itemch{b}
Suppose that $E$ is the union of a (\emph{possibly uncountable}) collection
of \emph{closed} discs in \(\R^2\) whose radii are at least $1$.
Use \ich{a} to show that $E$ is Lebesgue measurable.

\itemch{c} 
Show that the conclusion of \ich{b} is true even when the
radii are unrestricted.

\itemch{d}
Show that some unions of closed discs of radius $1$
are not Borel sets (See Sec.~2.21.)

\itemch{e}
Can discs be replaced by triangles, rectangles, arbitrary polygons, etc.,
in all this?
What is the relevant geometric property?
\end{itemize}
\end{excopy}

\begin{itemize}

\itemch{a}
For every set $E$ we have
\begin{equation*}
E = \inter{A} \disjunion (E \cap \partial E).
\end{equation*}
Now if \(m(\partial E) = 0\)
then also \(m(E \cap \partial E) = 0\) and thus $E$ is measurable
as a union of two measurable sets.


\itemch{b}

{\small Using a hint of Dmitry Ryabogin
see 
\linebreak[1]
\texttt{http://www.math.ksu.edu/\~{}ryabs/tar13.pdf}
\linebreak[1]
or
\texttt{http://www.math.ksu.edu/\~{}ryabs/tar13.dvi}{}.}

Let \(\{D_j\}_{j\in J}\) be a family of closed discs
with radii \(\geq 1\). 
Let \(U = \cup_{j\in J} D_j\) and \(V = \cup_{j\in J} \inter{D_j}\).
Let $Y$ be the boundary of the $U$. Since $Y$ is closed, it is measurable.
Assume by negation that \(m(Y)>0\). 
By Lebesgue's density theorem~7.7 \cite{RudinRCA87}
there are points (actually almost all)
\(b\in Y\) such that their denisty is $1$.
Pick arbitrary \(b\in Y\) and let \(\iota = 1/4\)
(Actually and value \(0 < \iota < 1/2\) will do).
We will show that the density 
\begin{equation*}
\lim_{r\to 0^+} \frac{m\bigl(Y\cap B(b,r)\bigr)}{m(B_r)} \leq 1 - \iota < 1
\end{equation*}
which will provide the desired contradiction.
Let \(0<r<\iota\) and \(0<\epsilon<r\).
Since $b$ is a boundary point,
there exist some disc with radius \(\rho\geq 1\)
whose center $c$ such that \(\|b-c\| < \rho + \epsilon\).
\Wlogy, assume \(b=(0,0)\) and \(c=(\rho + \epsilon, 0)\).
Let $S$ be the intersection
of the discs \(B(b,r)\) and \(B(c,\rho + \epsilon)\).
These boundaries of these discs intersects in two points
\((r\cos\alpha, \pm r \sin\alpha)\). Denote the following points
\begin{alignat*}{2}
Q &= (r\cos\alpha, r \sin\alpha) & & \qquad\textrm{Positive intersection} \\
P &= (r\cos\alpha, 0)            & & \qquad\textrm{Projection of\ }\; Q \\
A &= (\epsilon, 0)               & & \qquad\textrm{Nearest point of}\; 
                                     B(c,\rho+\epsilon)\;\textrm{to}\; b \\
R &= (r, 0)
                                 & & \qquad\textrm{Nearest point of}\; 
                                     B(b,r)\;\textrm{to}\; c \\
\end{alignat*} 
Now we estimate the area which is the measure of the intersection
\begin{align*}
m\bigl(B(b,r) \cap B(c,\rho + \epsilon)\bigr)
&= 2m\bigl(B(b,r) \cap B(c,\rho + \epsilon) 
           \cap \{(x,y)\in\R^2: y\geq 0\} \bigr) \\
&\geq 2 m\bigl(\vartriangle(A,R,Q)\bigr) 
 \geq 2 m\bigl(\vartriangle(b,R,Q)\bigr) 
 \geq 2 m\bigl(\vartriangle(P,R,Q)\bigr) \\
&= 2 \frac{\|P-R\|\cdot \|Q-P\|}{2}
 = r(1-\cos\alpha)\cdot r\sin\alpha.
\end{align*}

To use this estimatation, we need to see how \(\alpha\)
that clearly satisfies \(0<\alpha < \pi/2\)
depends on $r$, \(\rho\) and \(\epsilon\).
Equating the square distance of \(\overline{Qc}\) by Pythagoras
\begin{equation*}
\rho^2 = \|P-c\|^2 + \|Q-P\|^2 
= \bigl((\rho+\epsilon)-r\cos\alpha\bigr)^2 +(r\sin\alpha)^2
= (\rho+\epsilon)^2-2r(\rho+\epsilon)\cos + r^2
\end{equation*}
hence
\begin{equation*}
\cos\alpha = \frac{r^2 + (\rho+\epsilon)^2 - \rho^2}{2r(\rho+\epsilon)} > 0
\end{equation*}
and so
\begin{equation*}
\lim_{\epsilon\to 0^+} \cos\alpha = r/2\rho \leq r/2.
\end{equation*}
For each (sufficiently small) \(r>0\), We can pick 
a disc \(D_{j(r)} = D_j\) of the family with distance of 
\(\epsilon\geq 0\) to $x$ 
(radius \(\rho\) and center $c$ such that \(\|c-x\| = \rho+\epsilon\))
such that \(\alpha_r = \alpha > \pi/2 - r\).

We start estimating the density of $Y$ in $x$. 
We note that \(U \subset \inter{V} \subset Y\),
hence \(U \cap Y = \emptyset\).
\begin{equation} \label{eq:ex7.20:DYx}
D_Y(x) = 
\lim_{r\to0+} \frac{m\bigl( B(x,r) \cap Y\bigr)}{m(B_r)}
\leq
1 - \lim_{r\to0+} \frac{m\bigl( B(x,r) V\bigr)}{m(B_r)}\;.
\end{equation}
Focusing on the last limit 
\begin{align*}
\lim_{r\to 0+} \frac{m\bigl( B(x,r) \cap V\bigr)}{m(B_r)}
&\geq \lim_{r\to 0+} \frac{m\bigl( B(x,r) \cap D_{j(r)}\bigr)}{m(B_r)} 
 \geq \lim_{r\to 0+} \frac{r^2(1-\cos\alpha_r)\sin\alpha_r}{\pi r^2} \\
&\geq \lim_{r\to 0+} (1-\cos\alpha_r)\sin\alpha_r / \pi = 1/\pi > 0.
\end{align*}
Returning to \eqref{eq:ex7.20:DYx} we see that 
\begin{equation*}
D_Y(x) \leq 1 - 1/\pi < 1.
\end{equation*}
which is the the desired contradiction.


\itemch{c}
The statement of \ich{b} can 
be trivially to any lower bound \(r\geq 0\) instead of $1$ for the radii.
So now if \calF\ is any collection if closed circles,
let \(\calF(r)\) be the sub-collection consisting of closed circles 
of \calF\ whose radii is at least $r$.
Now since
\begin{equation*}
\calF = \cup_{n=1}^\infty \calF(1 - 1/n)
\end{equation*}
we have
\begin{equation*}
\bigcup_{C\in\calF} C
= \cup_{n=1}^\infty \left(\bigcup_{C\in \calF(1 - 1/n)} C\right).
\end{equation*}
Thus any union of closed circles, is a countable union of 
Lebesgue measurable sets
and so the union is Lebesgue measurable as well.

\itemch{d}
Pick a non measurable set \(A\subset \R^1\). Now let
\begin{equation*}
U 
= \cup_{a\in A} B\bigl((a,1),1\bigr)
= \cup_{a\in A} \{(x,y)\in\R^2: (x-a)^2+(y-1)^2 \leq 1\}.
\end{equation*}
Assume by negation that $U$ is a Borel set. Then so would be
\begin{equation*}
A \times \{0\} = U \cap \left(\R\times \{0\}\right).
\end{equation*}
Now $A$ is generated by a countable sequence of countable unions
and countable intersections of the 
\(\sigma\)-algebra base of open sets. Restricting these open sets to \(\R^1\)
results with open sets in \(\R^1\). Producing the same countable processes
in \(\R^1\) now generates $A$ that must be Borel, which is a contradiction.


\itemch{e}
Instead of disc, the required property is that the geometric shape
would have a lower bound on the internal angles. Not ``too accute'' angle.
Thus perfect polygons would work.
\end{itemize}

%%%%%%%%%%%%%% 21
\begin{excopy}
  If $f$ is a real function on \([0,1]\) and
\begin{equation*}
  \gamma(t) = t + if(t)
\end{equation*}
the length of the graph of $f$ is, by definition, the total variation
of \(\gamma\) on \([0,1]\). Show that this length is finite if and
only if \(f\in BV\).  (See Exercise~13.)  Show that it is equal to
\begin{equation*}
  \int_0^1 \sqrt{1 + [f'(t)]^2}\,dt
\end{equation*}
if $f$ is absolutely continuous.
\end{excopy}


We will deal with more general case of a curve, 
following Theorem~(8.4)(iii) of \cite{Saks37}.

\begin{llem}
Let \(\phi:[0,1]\to \R^2\) be a continuous curve given by 
\(\phi(t) = \bigl(x(t),y(t)\bigr)\).
Let
\begin{equation*}
d(t_0,t_1) = \|\phi(t_1) - \phi(t_0)\| = 
\sqrt{\bigl(x(t_1) - x(t_0)\bigr)^2 + \bigl(y(t_1) - y(t_0)\bigr)^2}
\qquad (0\leq t_0 \leq t_1 \leq 1)
\end{equation*}
the distance of the curve \(\phi\) on the parameter range \([t_0,t_1]\).
Let 
\begin{equation*}
S(\phi,\alpha;t) = \sup_{\alpha = a_0 < a_1 < \cdots < a_n = t}
                   \sum_{j=1}^n d(a_j,a_{j-1})
\qquad (0\leq \alpha \leq t \leq 1)
\end{equation*}
be the length of the curve on the parameter range \([\alpha,t]\).
\begin{itemize}
\itemch{i} The length $S$ of the curve is finite iff \(x(t)\) and \(y(t)\)
           have bounded variation.
\itemch{ii} If  \(x(t)\) and \(y(t)\) are absolutely continuous, then
\begin{equation} \label{eq:7.20:Seq}
S(\phi,0;1) = \int_0^1 \sqrt{\bigl(x'(t)\bigr)^2 + \bigl(y'(t)\bigr)^2}.
\end{equation}
\end{itemize}
\end{llem}
\begin{thmproof}
Assume the length is finite.
Let \(0=a_0 < a_1 < \cdots a_n = 1\) be any partition, then
\begin{equation*}
\sum_{j=1}^n |x(a_j) - x(a_{j-1})|
\leq \sum_{j=1}^n d(a_j,a_{j-1}) \leq S(\phi,0;1) < \infty
\end{equation*}
and so \(x(t)\) has bounded variation.
Simiar estimation can be done with \(y(t)\) that  has bounded variation as well.

Conversely, assume  \(x(t)\) and \(y(t)\) have  bounded variation.
Again \(0=a_0 < a_1 < \cdots a_n = 1\) be any partition, then
\begin{equation*}
\sum_{j=1}^n d(a_j,a_{j-1}) 
\leq \sum_{j=1}^n |x(a_j) - x(a_{j-1})| + |y(a_j) - y(a_{j-1})| < \infty
\end{equation*}
and the curve has finite length and \ich{i} is proved.

Now assume that  \(x(t)\) and \(y(t)\) are absolutely continuous.


We first show that so is \(S(\phi,0,t)\).
Let \(\epsilon>0\) and \(\delta\) be such that 
\(\sum_{k=1}^n |x(b_j) - x(a_j)| < \epsilon\) 
and
\(\sum_{k=1}^n |y(b_j) - y(a_j)| < \epsilon\) 
whenever \(\sum_{k=1}^n b_j - a_j < \delta\) where \(0 \leq a_j < b_j \leq 1\).
Now \(S(t) = S(\phi,0;t)\) is clearly monotonically increasing, 
and we have
\begin{eqnarray*}
\sum_{j=1}^n S(b_j) - S(a_j) 
&=&
\sum_{j=1}^n \sup_{a_j = t_0 < t_1 < \cdots < t_n = b_j} 
             \sum_{k=1}^n  d(t_{k-1},t_k) \\
&\leq&
\sum_{j=1}^n \sup_{a_j = t_0 < t_1 < \cdots < t_n = b_j} 
             \sum_{k=1}^n  |x(k) - x(t_{k-1})| + |y(k) - y(t_{k-1})|   \\
&\leq&
\sum_{j=1}^n 
\left(\sup_{a_j = t_0 < t_1 < \cdots < t_n = b_j} 
             \sum_{k=1}^n  |x(k) - x(t_{k-1})| \right) \\
&& +\;
\left(\sup_{a_j = t_0 < t_1 < \cdots < t_n = b_j} 
             \sum_{k=1}^n  |y(k) - y(t_{k-1})| \right)
\\
&<& 2\epsilon
\end{eqnarray*}

Thus \(S(t)\) is absolutely continuous as well.
By Theorem~7.18 \cite{RudinRCA87} \(x(t)\), \(y(t)\) and \(S(\phi,0;t)\)
are differentiable almost everywhere and consequently
are differentiable \emph{simultaneously} almost everywhere.
Let \(J\subset[0,1]\)
the set of points all three functions are differentiable
and \(t\in J\).
We have
\begin{equation*}
S(t+h) -S(t) 
\geq d(t+h,t) 
= \sqrt{\bigl(x(t+h) - x(t)\bigr)^2 + \bigl(y(t+h) - y(t)\bigr)^2}
\end{equation*}
for any \(h>0\). Hence
\begin{align*}
S'(t)
&\geq \lim_{h\to 0^+}  
   \sqrt{\bigl(x(t+h) - x(t)\bigr)^2 + \bigl(y(t+h) - y(t)\bigr)^2} \bigm/ h \\
&= \lim_{h\to 0^+}  
   \sqrt{\left(\bigl(x(t+h) - x(t)\bigr)/h\right)^2 + 
         \left(\bigl(y(t+h) - y(t)\bigr)/h\right)^2}  \\
&= \sqrt{\bigl(x'(t)\bigr)^2 + \bigl(y'(t)\bigr)^2}.
\end{align*}
We will show the reversed inequality holds almost everywhere.
For any interval \(I=(\alpha,\beta)\subset[0,1]\) we use the notations
\begin{align*}
x(I) &= |x(\beta) - x(\alpha)| \\
y(I) &= |y(\beta) - y(\alpha)| \\
S(I) &= S(\phi,\alpha;\beta). 
\end{align*}
Define the ``bad set''
\begin{equation*}
A = \left\{t\in J: 
        S'(t) > \left(\bigl(x'(t)\bigr)^2 + \bigl(y'(t)\bigr)^2\right)^{1/2}
        \right\}. 
\end{equation*}
and its subsets
\begin{align}
A_n = \bigl\{ & t\in J:   \notag \\
      & \forall I=(\alpha,\beta)\ni t,\; 0<m(I) < 1/n  \notag \\
      & \Rightarrow  \label{eq:7.21:An}
        S(I)/m(I)
        \geq \left(
                \bigl(x(I)/m(I)\bigr)^2 + \bigl(y(I)/m(I)\bigr)^2
             \right)^{1/2} + 1/n
        \bigr\}
\qquad (n\in\N).
\end{align}
(Notice that \(g'(t) = \lim_{m(I)\to 0+} g(I)/m(I)\)
if $g$ is differentiable at $t$.)
Clearly \(A = \cup_{n\in\N} A_n\).

Fix $n$ and pick arbitrary \(\epsilon>0\).
There exist a ``sufficiently rich'' partition 
\begin{equation*}
0 = t_0 < t_1 < \cdots t_p = 1
\end{equation*}
putting \(J_k = [t_{k-1},t_k]\),
such that 
\(m(J_k) = t_k - t_{k-1} < 1/n\) for \(k\in\N_p\) and 
\begin{equation} \label{eq:7.21:Sleq}
S([0,1]) = \sum_{k=1}^p S(J_k) \leq \sum_{k=1}^p d(t_{k-1},t_k) + \epsilon
\end{equation}
On the other hand, by \eqref{eq:7.21:An} we have
\begin{equation} \label{eq:7.21:Sgeq}
S(J_k) \geq d(t_{k-1},t_k) + m(J_k)/n
\end{equation}
whenever \(j_k \cap A_n \neq \emptyset\). 
By \eqref{eq:7.21:Sleq} \eqref{eq:7.21:Sgeq} we have
\newcommand{\sumJkAn}{
            \sum_{\stackrel{1\leq k \leq p}{J_k\cap A_n \neq \emptyset}}}
\begin{equation*}
m(A_n) 
\leq \sumJkAn m(J_k) 
\leq n \sumJkAn S(J_k) - d(t_{k-1},t_k) 
\leq n\epsilon.
\end{equation*}
Since \(\epsilon\) was arbitrarily picked, \(m(A_n) = 0\) 
hence \(m(A) = 0\) and the desired reversed inequality holds.
Therefore
\begin{equation} \label{eq:ex7.21:Sdif}
S'(t) = \sqrt{\bigl(x'(t)\bigr)^2 + \bigl(y'(t)\bigr)^2}
\end{equation}
which completes the proof.
\end{thmproof}

For this specifc exercise, we note that 
in a curve of a function graph
\(x(t)=t\) and \(x'(t)=1\).
Now the lemma provides the solution.


%%%%%%%%%%%%%% 22
\begin{excopy}
\begin{itemize}
\itemch{a}
Assume that both $f$ and  its maximal function \(Mf\) are in \(L^1(\R^k)\).
Prove that then \(f(x)=0\;\aded[m]\).
\emph{Hint}: To every other \(f\in L^1(\R^k)\) corresponds a constant
\(c=c(f)>0\) such that
\begin{equation*}
  (Mf)(x) \geq |x|^{-k}
\end{equation*}
whenever \(|x|\) is sufficiently large.

\itemch{b}
Define \(f(x) = x^{-1}(\log x)^{-2}\) if \(0<x<\frac{1}{2}\),
\(f(x)=0\) on the rest of \(\R^1\).
Then \(f\in L^1(\R^1)\).
Show that
\begin{equation*}
  (Mf)(x) \geq |2x \log(2x)|^{-1} \qquad (0<x<1/4)
\end{equation*}
so that \(\int_0^1 (Mf)(x)\,dx = \infty\).
\end{itemize}
\end{excopy}

\begin{itemize}

\itemch{a}
Assume by negation \(\|f\|_1\neq 0\).
Then there exist and integer \(a\in\N\) such that 
\[\alpha = \int_{B(0,a)} |f(x)|\,dm(x) > 0.\]
Assume \(x\in\R^k\) and \(\|x\|\geq a\). 
We note that \(m(B_r) = c_k \pi r^k\) 
(where \(c_k = \pi^{k/2}/\Gamma(n/2+1)\)).
Now
\begin{align*}
(Mf)(x)
&= \sup_{r>0} \frac{1}{m(B_r)} \int_{B(x,r)} |f(t)|\,dt \\
&\geq \frac{1}{m(B_{|x|+a})} \int_{B(x,|x|+a)} |f(t)|\,dt 
 \geq \frac{1}{m(B_{|x|+a})} \int_{B(0,a)} |f(t)|\,dt \\
&= \alpha / m(B_{|x|+a}).
\end{align*}
Define and estimate for each integer \(n>a\)
\begin{align*}
S_n
&= \int_{B(0,n)} (Mf)(x)\,dm(x) \\
&\geq \sum_{r=a+1}^n \int_{B(0,r)\setminus B(0,r-1) } (Mf)(x)\,dm(x) 
 \geq \sum_{r=a+1}^n 
     \int_{B(0,r)\setminus B(0,r-1) } \alpha / m(B_{|x|+a}) \,dm(x) \\
&\geq \sum_{r=a+1}^n 
     m\bigl(B(0,r)\setminus B(0,r-1)\bigr) \alpha / m(B_{2r}) \\
&= \sum_{r=a+1}^n \alpha c_k\bigl(r^k - (r-1)^k\bigr) 
                  \bigm/ \bigl( c_k (r+a)^k \bigr) 
 = \frac{\alpha}{2^k} 
   \sum_{r=a+1}^n \bigl(r^k - (r-1)^k\bigr) \bigm/ r^k \\
&\geq \frac{\alpha}{2^k} 
      \sum_{r=a+1}^n \frac{1}{r}.
\end{align*}
Since \(\lim_{n\to\infty} S_n = \infty\) we get the
contradiction \(Mf \notin L^1(\R^k)\).

\itemch{b}
Put \(F(x) = -1/\log(x)\). Now
\begin{equation*}
F'(x) = (-1/x) \cdot \left(-\bigl(\log(x)\bigr)^{-2}\right) = f(x)
\end{equation*}
when \(0<x<1/2\).
Hence
\begin{equation*}
\|f\|_1 
= \lim_{h\to 0^+} \int_h^{1/2} f(x)\,dx
= -1/\log(1/2) - \lim_{h\to 0^+}  1/\log(h)
= -1/\log(1/2) 
< \infty.
\end{equation*}
Thus \(f\in L^1(\R^1)\).
But if  \(0<x<1/4\) then
\begin{equation*}
(Mf)(x) 
= \sup_{r} \frac{1}{2r} \int_{x-r}^{x+r} f(t)\,dt
\geq \frac{1}{2x} \int_0^{2x} f(t)\,dt
= \bigl(F(2x) - F(0)\bigr)/2x 
= |2x\log(2x)|^{-1} 
\end{equation*}
Hence
\begin{align*}
\int_0^1 (Mf)(x)\,dx
&\geq \int_0^{1/4} (Mf)(x)\,dx \\
&\geq \int_0^{1/4} \frac{1}{2x} \cdot \frac{-1}{\log(2x)} \,dx \\
&= \frac{-1}{2}\left.\left(\log(|\log(2x)|)\right)\right|_0^{1/4} 
 = \frac{-1}{2} \bigl(\log(1/2) - \lim_{x\to +\infty}\log(x)\bigr) \\
&= \infty.
\end{align*}
Indeed \(Mf \notin L^1([0,1])\).
\end{itemize}


%%%%%%%%%%%%%% 23
\begin{excopy}
The definition of Lebesgue points, as made in Sec.~7.6, applies to individual
integrable functions, not to the equivalence classes discussed in Sec.~3.10.
However, if \(F\in L^1(\R^k)\) is one of these equivalence classes, one may call
a point \(x\in \R^k\)
a \emph{Lebesgue point of}
\index{Lebesgue point}
$F$ if there is a complex number, let us call it \((SF)(x)\) to be $0$
at those points \(x\in \R^k\) that are not Lebesgue points of $F$.

Prove the following statement: If \(f\in F\), and $x$ is a Lebesgue
points of $f$, then $x$ is also a Lebesgue point of $F$, 
and \(f(x)(SF)(x)\).  Hence \(SF\in F\).

Thus $S$ ``selects'' a member of $F$ that has a \emph{maximal} set
of Lebesgue points.
\end{excopy}

Assume $x$ is
\index{Lebesgue point}
a~Lebesgue point of $f$.
The set $E$ of non Lebesgue points of $f$ has measure zero by 
Theorem~7.7 \cite{RudinRCA87}. Thus
changing $f$ on $E$ does not effect the values of 
\begin{equation*}
A(x) \frac{1}{m(B_r)} \int_{B(x,r} |f(y) - f(x)|\,dm(y)
\end{equation*}
except possibly for \(x\in E\).
Hence Lebesgue points of $f$ are Lebesgue points of $F$ as well.

%%%%%%%%%%%%%%%%%
\end{enumerate}

 % -*- latex -*-

%%%%%%%%%%%%%%%%%%%%%%%%%%%%%%%%%%%%%%%%%%%%%%%%%%%%%%%%%%%%%%%%%%%%%%%%
%%%%%%%%%%%%%%%%%%%%%%%%%%%%%%%%%%%%%%%%%%%%%%%%%%%%%%%%%%%%%%%%%%%%%%%%
%%%%%%%%%%%%%%%%%%%%%%%%%%%%%%%%%%%%%%%%%%%%%%%%%%%%%%%%%%%%%%%%%%%%%%%%
\chapterTypeout{Integration on Product Spaces} % 8


%%%%%%%%%%%%%%%%%%%%%%%%%%%%%%%%%%%%%%%%%%%%%%%%%%%%%%%%%%%%%%%%%%%%%%%%
%%%%%%%%%%%%%%%%%%%%%%%%%%%%%%%%%%%%%%%%%%%%%%%%%%%%%%%%%%%%%%%%%%%%%%%%
\section{Notes}

%%%%%%%%%%%%%%%%%%%%%%%%%%%%%%%%%%%%%%%%%%%%%%%%%%%%%%%%%%%%%%%%%%%%%%%%
\subsection{Product of Complex Measures} \label{subsec:prod:complex:measures}

Following Theorem~8.6 that assumes the measures are \(\sigma\)-finite
which also implies that thy are positive, there is 
a definition of products \(\mu\times\lambda\) with the similar assumptions.
We need to generalize the notion for complex measures as well.
Given a measure \(\mu\) we can define the following decomposition
based on 
\index{Jordan Decomposition}
Jordan Decomposition.
\begin{alignat*}{2}
\mu_{r} &= \Re(\mu) \qquad& \mu_{i} &= \Im(\mu) \\
\mu_{r+} &= (|\mu_r|+\mu_r)/2  \qquad& \mu_{i+} &= (|\mu_i|+\mu_i)/2 \\
\mu_{r-} &= (|\mu_r|-\mu_r)/2  \qquad& \mu_{i-} &= (|\mu_i|-\mu_i)/2
\end{alignat*}
Now by distributive law for measures we get the desired expected generalization
of the definition of product space to complex (finite!) measures.


%%%%%%%%%%%%%%%%%%%%%%%%%%%%%%%%%%%%%%%%%%%%%%%%%%%%%%%%%%%%%%%%%%%%%%%%
\subsection{Young's Inequality} \label{subsec:young:ineq}

We bring here a proof of 
\index{Young!inequality}
Young's inequality based on lecture notes from \cite{Viaclovsky:18125:lec20}.

\begin{lthm}
Let \(p, q, r \in[1,\infty]\) such that
\begin{equation*}
\frac{1}{r} = \frac{1}{p} + \frac{1}{q} - 1
\end{equation*}
if \(f \in L^p(\R)\) and \(g\in L^q(\R)\), 
then \(f \ast g\) exists \aded\ and \(f\ast g\in L^r(\R)\). 
Moreover,
\begin{equation*}
\|f\ast g\|_r \leq \|f\|_p \cdot \|g\|_q\,. \label{eq:young:ineq}
\end{equation*}
\end{lthm}


\begin{thmproof}
If \(f=0\,\aded\) or \(g=0\,\aded\) then the inequality is trivial.
Without loss of generality, let \(\|f\|_p = \|g\|_q = 1\)
Since otherwise we can look at 
\(f/\|f\|_p\) and \(g/\|g\|_q\) instead. The general
case follows from the nonnegative
case, so assume \(f, g \geq 0\). 
Let \(p'\) and \(q'\) be the exponential conjugate of $p$ and $q$ respectably.
Using \(1/r+1/q'+1/p'=1\) and 
applying H\"older’s inequality Theorem~3.5,
\begin{align*}
(f\ast g)(x)
&= \int_{\R}\left(f(y)^{p/r}g(x-y)^{q/r}\right)f(y)^{1-p/r}g(x-y)^{1-q/r}\,dy \\
&\leq 
  \left(\int_{\R} f(y)^pg(x-y)^q\,dy\right)^{1/r}
     \left(\int_{\R} f(y)^{(1-p/r)q'}\,dy\right)^{1/q'}
        \left(\int_{\R} f(x-y)^{(1-q/r)p'}\,dy\right)^{1/p'}\,.
\end{align*}
Since 
\begin{equation*}
(1-p/r)q' = p \qquad\textnormal{and}\qquad (1-q/r)p' =q
\end{equation*}
we have
\begin{equation*}
(f\ast g)(x) 
\leq \left(\int_{\R} f(y)^pg(x-y)^q\,dy\right)^{1/r}\cdot1\cdot1.
\end{equation*}
Hence
\begin{equation*}
(f\ast g)^r(x) \leq \int_{\R} f(y)^pg(x-y)^q\,dy
\end{equation*}
That is \((f\ast g)^r \leq f^p \ast g^q\).
Now
\begin{align}
\|f\ast g\|_r^r
&= \int_{\R} (f\ast g)^r(x)\,dx \notag \\
&\leq  \int_{\R} (f^p \ast g^q)(x)\,dx = \|(^p \ast g^q\|_1 \notag \\
&\leq \|f^p\|_1 \|g^q\|_1 \label{eq:young:conv1} \\
&=  \|f\|_p^p \cdot \|g\|_q^q = 1. \notag
\end{align}
The inequality~\eqref{eq:young:conv1} is given by Theorem~8.14.
Thus desired \eqref{eq:young:ineq} was shown.
\end{thmproof}

%%%%%%%%%%%%%%%%%%%%%%%%%%%%%%%%%%%%%%%%%%%%%%%%%%%%%%%%%%%%%%%%%%%%%%%%
%%%%%%%%%%%%%%%%%%%%%%%%%%%%%%%%%%%%%%%%%%%%%%%%%%%%%%%%%%%%%%%%%%%%%%%%
\section{Exercises} % pages 174-177

%%%%%%%%%%%%%%%%%
\begin{enumerate}
%%%%%%%%%%%%%%%%%


%%%%%%%%%%%%%% 1
\begin{excopy}
Find a monotone class \frakM\ in \(\R^1\) which is not a \salgebra,
even though \(\R^1 \in \frakM\) and \(\R^1\setminus A \in \frakM\)
for every \(A\in \frakM\).
\end{excopy}

Take
\begin{equation*}
\frakM = \bigl\{\emptyset, \R^1, \; 
                \{0\}, \R^1\setminus\{0\}, \;
                \{1\}, \R^1\setminus\{1\}  \bigr\}.
\end{equation*}
Note that it is a monotone class but \(\{0,1\}\notin \frakM\).


%%%%%%%%%%%%%% 2
\begin{excopy}
Suppose $f$ is a Lebesgue measurable nonnegative real function on \(\R^1\)
and \(A(f)\) is the 
\emph{ordinate set}
\index{ordinate set}
of $f$. This is the set of all points \((x,y)\in\R^2\) for which \(0<y<f(x)\).
\begin{itemize}
\itemch{a} Is it true that \(A(f)\) is Lebesgue measurable, 
           in the two dimensional sense?
\itemch{b} If the answer to \ich{a} is affirmative, 
           is the integration of $f$ over \(\R^1\) 
           equal to the measure of \(A(f)\)?
\itemch{c} Is the graph of $f$ a measurable subsection of \(\R^2\)?
\itemch{d} If the answer to \ich{c} is affirmative, 
           is the measurable of the graph equal to zero?
\end{itemize}
\end{excopy}

\begin{itemize}
\itemch{a}
Yes.
Define simple functions
\begin{equation*}
f_n(x) = \lfloor nf(x)\rfloor / n \qquad (n\geq 1).
\end{equation*}
Clearly \(\lim_{n\to\infty} f_n = f\).
Since \(A(f_n)\) is a countable union of rectangles, it is measurable.
So is \(A(f) = \cup_n A(f_n)\).

\itemch{b}
Yes. Using the notations if \ich{a}, we have 
\begin{equation*}
m_2\bigl(A(f_n)\bigr) 
= \sum_{k=0}^\infty (k/n)\cdot m\bigl(\{x\in\R^1: f_n(x) = k/n\}\bigr)
= \int_{\R^1} f_n(t)\,dm(t).
\end{equation*}
The desired equality is derived 
by Lebesgue monotone convergence theorem.

\itemch{c}
Yes, see \ich{d}.

\itemch{d}
Yes.
Pick \(\epsilon>0\). Define
\begin{equation*}
s(x) = \epsilon \cdot 2 ^{-\lfloor x \rfloor} > 0
\end{equation*}
Clearly 
\begin{equation*}
\int_{\R} s(x)\,dx = 2 \epsilon \sum_{n=0}^\infty 2^{-n} = 4\epsilon.
\end{equation*}
Now 
\begin{equation*}
G 
= \{(x,f(x)): x\in\R\} 
\subset \{(x,y)\in\R^2: f(x)-s(x) < y < f(x)+s(x)\}.
\end{equation*}
and 
\begin{align*}
m(G) 
&\leq m\bigl(\{(x,y)\in\R^2: f(x)-s(x) < y < f(x)+s(x)\}\bigr) \\
&=  m\bigl(\{(x,y)\in\R^2: 0 < y < 2s(x)\}\bigr)
= 2\int_R s(x)\,dx = 8\epsilon.
\end{align*}
Since \(\epsilon\) is arbitrary, we have \(m(G)=0\).
\end{itemize}


%%%%%%%%%%%%%% 3
\begin{excopy}
Find an example of a positive continuous function $f$
in the open unit square in \(\R^2\),
whose integral (relative to Lebesgue measure)
is finite but such that \(\varphi(x)\) 
(in the notation of Theorem~8.8)
is infinite for some \(x \in (0,1)\).
\end{excopy}

Let \(U = (0,1)^2 \subset \R^2\) be the open unit square
and \(C = (1/2,0)\) a point on its boundary.
For \(n\geq 1\) define the open triangle
\begin{equation*}
T_n = \triangle\bigl(C-(1/n, 0),\; C+(1/n, 0),\; C+(0, 1/n)\bigr).
\end{equation*}
Clearly \(T_n \supset T_{n+1}\) is a decreasing sequence, 
\(m(T_n) = 1/n^2\) and \(\cap_n T_n = \emptyset\).
By  Urysohn's Lemma~2.12 \cite{RudinRCA87}, 
\index{Urysohn's lemma}
there exists
\(f_n:U\to[0,1]\) such that 
\begin{equation*}
U \setminus T_n \prec f_n \prec T_{n+1}.
\end{equation*}
Let \(f = \sum_{n=1}^\infty f_n\).
Now
\begin{equation*}
\int_U f(x,y)\,dm(x,y)
= \sum_{n=1}^\infty \int_U f_n(x,y)\,dm(x,y)
\leq \sum_{n=1}^\infty \int_U \chhi_{T_n}(x,y)\,dm(x,y)
= \sum_{n=1}^\infty 1/n^2
< \infty.
\end{equation*}
But for \(x=1/2\)
\begin{align*}
\int_0^1 f_x(y)\,dy 
&= \int_0^1 f(1/2,y)\,dy
= \sum_{n=1}^\infty \int_0^1 f(1/2,y)\,dy \\
&\geq \sum_{n=1}^\infty \int_0^1 \chhi_{T_{n+1}}(1/2,y)\,dy
= \sum_{n=1}^\infty 1/(n+1)
= \infty.
\end{align*}
 

%%%%%%%%%%%%%% 4
\begin{excopy}
Suppose \(1\leq p \leq \infty\), \(f\in L^1(\R)\) and \(g\in L^p(\R)\).
\begin{itemize}
\itemch{a}
Imitate the proof of Theorem~8.14 to show that the integral defining
\((f\ast g)(x)\) exists for almost all $x$, that \(f\ast g \in L^p(\R)\),
and that 
\begin{equation*}
\| f \ast g \|_p \leq \|f\|_1 \|g\|_p.
\end{equation*}

\itemch{b}
Show that equality can hold in \ich{a} if \(p=1\) and if \(p=\infty\),
and find the conditions under which this happens.

\itemch{c}
Assume \(1 < p < \infty\) and equality holds in \ich{a}.
Show that then either 
\(f=0\;\aded\) or 
\(g=0\;\aded\)

\itemch{d}
Assume \(1\leq p \leq \infty\), \(\epsilon>0\), and show that there exists
\(f\in L^1(\R^1)\) and \(g\in L^p(\R^1)\) such that 
\begin{equation*}
\| f \ast g \|_p > (1-\epsilon) \|f\|_1 \|g\|_p.
\end{equation*}
\end{itemize}
\end{excopy}

For a generalization with 
\index{Young}
Young inequalities, see \cite{EdwFA} Theorem~9.5.1 page~655.

\begin{itemize}
\itemch{a}
Let's first assume \(p=\infty\), then
\begin{align*}
\|f \ast g\|_\infty
&= \esssup_{x\in\R} \left| \int_{\R} f(y)g(x-y)\,dy\right|
\leq \esssup_{x\in\R}  \int_{\R} |f(y)g(x-y)|\,dy \\
&\leq \|g\|_\infty \esssup_{x\in\R} \int_{\R} |f(y)|\,dy
 = \|f\|_1 \|g\|_\infty
\end{align*}

So now we may assume \(p<\infty\). We use 
% generalization of
 H\"older's inequality 
\index{Holder@H\"older}
Theorem~3.5 \cite{RudinRCA87}
% (see local lemma~\ref{llem:hlp:188}).
Let \(q = p / (p - 1)\) be the exponent conjugate.
\begin{align}
|(f \ast g)(t)|
&\leq \int_{\R} |f(t-s)g(s)|\,ds \notag \\
&= \int_{\R} 
    \left(|f(t-s)|^{1/p} |g(s)|\right) 
    \left(|f(t-s)|^{1/q}\right) 
    \,ds \notag \\
&\leq \label{eq:ex8.4a:holder}
      \left( \int_{\R} \left(|f(t-s)|^{1/p} |g(s)|\right)^p\,ds \right)^{1/p}
      \left( \int_{\R} \left(|f(t-s)|^{1/q}\right)^q \,ds \right)^{1/q} \\
&=   \left( \int_{\R} |f(t-s)| \cdot |g(s)|^p\,ds \right)^{1/p}
      \left( \int_{\R} |f(t-s)| \,ds \right)^{1/q} \notag \\
&=   \left( \int_{\R} |f(t-s)| \cdot |g(s)|^p\,ds \right)^{1/p} \|f\|_1^{1/q}
     \notag
\end{align}
The above inequality holds for almost all $t$. Hence
\begin{align}
\|f \ast g\|_p^p
&\leq \|f\|_1^{p/q} \notag
      \int_\R 
        \left(\int_{\R} |f(t-s)| \cdot |g(s)|^p\,ds \right)^{(1/p)p}\,dt \\
&= \|f\|_1^{p/q} \notag
   \int_\R \int_{\R} |f(t-s)| \cdot |g(s)|^p\,ds\,dt \\
&= \|f\|_1^{p/q} \label{eq:8.4:fub}
   \int_\R \int_{\R} |f(t-s)| \cdot |g(s)|^p\,dt\,ds \\
&= \|f\|_1^{p/q} \notag
   \int_\R |g(s)|^p \int_{\R} |f(t-s)| \,dt\,ds \\
&= \|f\|_1^{p/q} \cdot \|g\|_p^p \cdot \|f\|_1  \notag
 = \|f\|_1^{p/(p/(p-1)) + 1}  \|g\|_p^p \\
&=  \|f\|_1^p \cdot \|g\|_p^p  \notag
\end{align}
The \eqref{eq:8.4:fub} equality is by Fubini Theorem~8.8 \cite{RudinRCA87}.
Thus \(\|f \ast g\|_p =  \|f\|_1 \|g\|_p\) as desired.

\itemch{b}
\emph{Note:} This is a result of 
\index{Riesz-Thorin}
\index{Thorin}
Riesz-Thorin theorem.

% Clearly if \(f=0\,\aded\) or \(g=0\,\aded\) then equality holds.
Let \(h = f\ast g\).
By the proof of Theorem~8.14, when \(p=1\), there is an equality iff
\begin{equation*}
|h(x)| 
= \left|\int_{-\infty}^\infty f(x-y)g(y)\,dy\right|
= \int_{-\infty}^\infty |f(x-y)g(y)|\,dy
\end{equation*}
This happens iff \(f(x-y)g(y)\) 
has the same argument almost everywhere on~$x$ and~$y$.
Equivalently, since \(x-y\) covers all of \(\R\), iff
$f$ and $g$ wach has constant argument \aded.

When \(p=\infty\) then equality holds 
if for any \(\epsilon>0\) there exists some \(x\in\R\) 
and \(\theta\in[0,2\pi]\)
such that 
\begin{equation*}
\int_{\R}\left| f(y)g(x-y)- e^{i\theta}|f(y)|\cdot\|g\|_\infty\right|\,dx 
< \epsilon.
\end{equation*}


\itemch{c}
Assme \(1<p<\infty\).
If equality holds, then equality must hold for almost all \(t\in\R\) in 
\eqref{eq:ex8.4a:holder}.
By Local~Lemma~\ref{llem:hlp:188} (H\"older), both expressions:
\begin{equation*}
|f(t-s)|^{1/p} |g(s)|  \qquad  |f(t-s)|^{1/q}
\end{equation*}
must be effectively proportional for all \(t\in\R\).
Equivalently, 
\begin{equation*}
|g(s)|  \qquad  |f(t-s)|^{1/q-1/p}
\end{equation*}
must be effectively proportional. 
Hence \(f=0\,\aded\) or \(g=0\,\aded\).

\itemch{d}
Let \(f_n(x) = n\chhi_{[-1/n,+1/n]}/2\) and \(g = \chhi_{[0,1]}\).
Clearly \(\|f_n\|_1=1\) and
\(\|g\|_p = 1\). Now 
\begin{align*}
(f_n\ast g)(x) 
&= \int_{-\infty}^\infty f_n(x-t)g(t)\,dt
 = \int_0^1 f_n(x-t)\,dt
 = (n/2) \int_0^1 \chhi_{[-1/n,+1/n]}(x-t)\,dt \\
&= \left\{\begin{array}{ll}
    n(x+1/n)/2 \qquad &  -1/n \leq x \leq 1/n \\
    1    \qquad    1/n \leq x \leq 1-1/n \\
    n(x-(1-1/n))/2 \qquad &  -1/n \leq x \leq 1/n \\
    0 \qquad              & \textnormal{otherwise}
   \end{array}\right.
\end{align*}
By Lebesgue convergence theorems
\(\lim_{n\to\infty} \|f_n\ast g\|_p = 1\).
Hence for any \(\epsilon>0\) we can find some $n$ such that 
\begin{equation*}
\| f_n \ast g \|_p > 
(1-\epsilon) =
(1-\epsilon) \|f\|_1 \|g\|_p.
\end{equation*}
\end{itemize}


%%%%%%%%%%%%%% 5
\begin{excopy}
Let $M$ be the Banach space of all complex Borel measures on \(\R^1\).
The norm in $M$ is \(\|\mu\| = |\mu|(\R^1)\).
Associate to each Borel set \(E \subset \R^1\) the set
\begin{equation*}
E_2 = \{ (x,y): x+y \in E\} \subset \R^2.
\end{equation*}
If \(\mu\) and \(\lambda \in M\) define their convolution \(\mu \ast \lambda\)
to be the set function given by
\begin{equation*}
(\mu \ast \lambda)(E) = (\mu \times \lambda)(E_2)
\end{equation*}
for every Borel set \(E\subset \R^1\);
\(\mu\times\lambda\) is as in Definition~8.7.
\begin{itemize}
\itemch{a}
Prove that \(\mu \ast \lambda \in M\) and that 
\(\|\mu \ast \lambda\| \leq \|\mu\| \, \|\lambda \|\).

\itemch{b}
Prove that \(\mu \ast \lambda \) is the unique \(\nu\in M\) such that 
\begin{equation*}
\int f\,d\nu = \int\int f(x+y)\,d\mu(x)\,d\nu(y)
\end{equation*}
for every \(f\in C_0(\R^1)\). (all integrals extend over \(\R^1\).)

\itemch{c}
Prove that convolution in $M$ is commutative, associative, and distributive
with respect to addition.

\itemch{d}
Prove the formula
\begin{equation*}
(\mu \ast \lambda)(E) = \int \mu(E-t)\,d\lambda(t)
\end{equation*}
for every \(\mu\) and \(\lambda \in M\) and every Borel set $E$. Here
\begin{equation*}
E - t = \{x - t: x\in E\}.
\end{equation*}

\itemch{e}
Define \(\mu\) to be 
\emph{discrete} 
\index{discrete!measure} 
if \(\mu\) is concentrated on a countable set;
define \(\mu\) to be 
\emph{continuous}
\index{continuous!measure}
if \(\mu(\{x\}) = 0\) for every \(x\in\R^1\);
let $m$ be Lebesgue measure on \(\R^1\)
(note that \(m\notin M\)).
Prove that  \(\mu \ast \lambda\) is discrete if both \(\mu\) and \(\lambda\)
are discrete, that \(\mu \ast \lambda\) is continuous if \(\mu\) is continuous
and \(\lambda\in M\), and that \(\mu \ast \lambda \ll m \) if \(\mu \ll m\).

\itemch{f}
Assume
\(d\mu = f\,dm\), 
\(d\lambda = g\,dm\),
\(f\in L^1(\R^1)\), 
\(g\in L^1(\R^1)\), 
and prove that \\
\(d(\mu\ast\lambda) = (f\ast g)\,dm\).

\itemch{g}
Properties \ich{a} and \ich{c} show that the Banach space $M$ is what one calls
\index{commutative!Banach algebra}
\emph{commutative Banach algebra}.
Show that \ich{e} and \ich{f} imply that the set of all discrete measures in $M$
is a subalgebra of $M$,
that the continuous measures form an ideal in $M$, and that the absolutely
continuous measures (relative to $M$) form an ideal in $M$ which is isomorphic 
(as an algebra) to \(L^1(\R^1)\).

\itemch{h}
Show that $M$ has a unit, i.e., show that there exists a \(\delta\in M\)
such that \(\delta \ast \mu = \mu\) for all \(\mu \in M\).

\itemch{i}
Only two properties of \(\R^1\) have been used in this discussion: \(\R^1\)
is a commutative group (under addition), and there exists a translation
invariant Borel measure $M$ on \(\R^1\) which is not identically $0$ and 
which is finite on all compact subsets of \(\R^1\).
Show that the same results hold if \(\R^1\)
is replaced by \(\R^k\) or by $T$ (the unit circle) or by \(T^k\)
(the $K$-dimensional torus, the cartesian product of $k$
copies of $T$),
as soon as the definitions are properly formulated.
\end{itemize}
\end{excopy}

\emph{Note:} The definition of \(\mu\times\lambda\) where 
\(\mu\) and \(\lambda\) are complex measures requires a generalization
of the definition done in section~8.7 of \cite{RudinRCA87},
see \ref{subsec:prod:complex:measures} above.

Utilizing the notations of Theorem~8.6 gives:
\begin{align*}
(E_2)_t &= \{u\in\R: (t,u)\in E_2\} = \{u\in\R: t+u\in E\} 
    = \{x-t: x\in E\} = E-t \\
(E_2)^u &= \{t\in\R: (t,u)\in E_2\} = \{t\in\R: t+u\in E\} 
    = \{y-u: y\in E\} = E-u.
\end{align*}

\begin{itemize}
\itemch{a}
Let \(\frakM\) be the set of all Borel sets \(E\subset \R\)
such that \(E_2\) is a Borel set in \(\R^2\).

% Assume \(E\subset\R\) is a Borel set. We need to whow that 
% \(E_2\) is a Borel set in \(\R^2\).x
% We will do this in steps.

% First we assume that 
If $E$ is open, then it is trivial to see that~\(E_2\) is open. 
If \(E\in\frakM\) then 
\begin{equation*}
C_2 
= \{(x,y)\in\R^2: x+y \in \R\setminus E\}
= \R^2 \setminus \{(x,y)\in\R^2: x+y \in E\}
= \R^2 \setminus E_2\,,
\end{equation*}
hence \((\R\setminus E)\in\frakM\).

Let \(E = \cup_{n\in\N}B_n\) a Partition of Borel sets
and assume \(B_n\in\frakM\) for all \(n\in\N\), then
\begin{equation*}
E_2 
= \{(x,y)\in\R^2: x+y \in \cup_{n\in\N}B_n\}
= \bigcup_{n\in\N}B_n \{(x,y)\in\R^2: x+y \in B_n\}
\end{equation*}
hence \(E\in\frakM\).
Therefore \frakM\ is a \salgebra\ that contains the open sets,
and so it contains all the Borel sets.


Now we have to show that \(\mu\ast\lambda\) is a measure.
Let \(E=\disjunion_{n\in\N} B_n\) a partition of Borel sets.
\begin{align*}
(\mu\ast\lambda)(E)
&= (\mu\times\lambda)\left(\{(x,y)\in\R^2: x+y\in\disjunion_{n\in\N} B_n\}\right)
  \\
&= \sum_{n\in\N}(\mu\times\lambda)\left(\{(x,y)\in\R^2: x+y \in B_n\}\right) \\
&= \sum_{n\in\N}(\mu\ast\lambda)(B_n)
\end{align*}

We note that if \(A,B\subset\R\) are disjoint, then
\(A_2,B_2\subset\R^2\) are disjoint as well. Hence
\begin{align*}
\left|\mu\ast\lambda\right|
&= |\mu\ast\lambda|(\R)
 = \sup_{\R=\disjunion_{n\in\N} F_n} 
   \left(\sum_{n\in\N} \left| (\mu\ast\lambda) (F_n)\right|\right)
 = \sup_{\R=\disjunion_{n\in\N} F_n} 
   \left(\sum_{n\in\N} \left| (\mu\times\lambda) (F_n)_2\right|\right)
   \\
&\leq (|\mu|\times|\lambda|)(\R^2) 
 = \|\mu\|\cdot\|\lambda\|.
\end{align*}
The last equality is a trivial applying of 
\index{Fubini}
Fubini's Theorem~8.8.

\itemch{b}
The Borel measure is determined by its value on open sets.
In the case of \(\R\) therefore, it is determined by the value on intervals.
But for any interval $I$ we can aproximate \(\chhi_I\) by functions
from \(C_0(\R)\). Thus the uniqueness follows.

\itemch{d}
\emph{Note:} before \ich{c} since we will need this one there.
First we have the set manipulation
\begin{equation*}
(E_2)^y 
= \{x\in\R: (x,y)\in E_2\}
= \{x\in\R: x+y\in E\}
= \{t-y\in\R: t\in E\}
= E - y.
\end{equation*}
Now
\begin{equation*}
(\mu \ast \lambda)(E) 
= (\mu\times\lambda)(E_2) 
= \int_{\R} \mu\left(E_2)^t\right)\,d\lambda(t)
= \int_{\R} \mu(E-t)\,d\lambda(t)
\end{equation*}

\itemch{c}
Let \(\lambda,\mu,\nu\in M\) and a Borel set \(E\subset\R\),
and \(E_2\) defined as above.

\paragraph{Convolution is commutative.}
\begin{equation*}
(\lambda\ast \mu)(E)
= (\lambda\times\mu)(E)
= (\mu\times\lambda)(E)
= (\mu\ast\lambda)(E)
\end{equation*}

We note the set equality:
\begin{gather*}
(E^y)_2 = (E-y)_2 
 = \{(t,u)\in\R^2: t+u\in E-y\}
 = \{(t,u)\in\R^2: t+u+y\in E\} \\
%
\begin{align*}
\bigl((E-y)_2\bigr)^s 
&= \{r\in\R: (r,s)\in (E-y)_2\}
 = \{r\in\R: r+s\in E-y\} \\
&= \{r\in\R: r+s+y\in E\}
 = E-(y+s)
\end{align*}
\end{gather*}

\paragraph{Convolution is associative.}
Using Theorem~8.6 and its notations
\begin{align*}
\bigl((\lambda\ast \mu)\ast \nu\bigr)(E)
&=\bigl((\lambda\ast \mu)\times \nu\bigr)(E_2)
 % = \int_{\R} \nu\bigl((E_2)_x\bigr)\,d(\lambda\ast \mu)(x) \\
 = \int_{\R} (\lambda\ast \mu)\bigl((E_2)^y\bigr)\,d\nu(y) \\
&= \int_{\R} (\lambda\ast \mu)(E-y)\,d\nu(y) 
 = \int_{\R} (\lambda\times \mu)\bigl((E-y)_2\bigr)\,d\nu(y) \\
&= \int_{\R} 
   \left(
      \int_{\R}\lambda\left(\bigl((E-y)_2\bigr)^s\right)\,d\mu(s)
   \right)\,d\nu(y) \\
&= \int_{\R} \left(\int_{\R}\lambda(E-y-s)\,d\mu(s)\right)\,d\nu(y)
\end{align*}

Summerizing the above
\begin{equation} \label{eq:measure:conv:assoc}
\bigl((\lambda\ast \mu)\ast \nu\bigr)(E)
= \int_{\R} \left(\int_{\R}\lambda(E-y-s)\,d\mu(s)\right)\,d\nu(y)
\end{equation}

We now use commutativity and \eqref{eq:measure:conv:assoc} twice:
\begin{equation*}
\bigl((\lambda\ast \mu)\ast \nu\bigr)(E)
\bigl((\mu\ast \lambda)\ast \nu\bigr)(E)
= \int_{\R} \left(\int_{\R}\mu(E-y-s)\,d\lambda(s)\right)\,d\nu(y)
\end{equation*}
and
\begin{equation*}
\bigl(\lambda\ast (\mu\ast \nu)\bigr)(E)
\bigl((\mu\ast \nu)\ast \lambda\bigr)(E)
= \int_{\R} \left(\int_{\R}\mu(E-y-s)\,d\nu(s)\right)\,d\lambda(y)
\end{equation*}
By Fubini's Theorem~8.8 the last integrals of the last two equalities
are equal
\begin{equation*}
  \int_{\R} \left(\int_{\R}\mu(E-y-s)\,d\lambda(s)\right)\,d\nu(y)
= \int_{\R} \left(\int_{\R}\mu(E-y-s)\,d\nu(s)\right)\,d\lambda(y)
\end{equation*}
hence,
\begin{equation*}
  \bigl((\lambda\ast \mu)\ast \nu\bigr)(E) 
= \bigl(\lambda\ast (\mu\ast \nu)\bigr)(E).
\end{equation*}

\paragraph{Convolution is distributive.}
\begin{align*}
\bigl((\lambda+\mu)\ast \nu)(E)
&= \int_{\R} (\lambda+\mu)(E-y)\,d\nu(y)
     \int_{\R} \lambda(E-y)\,d\nu(y)
   + \int_{\R} \mu(E-y)\,d\nu(y) \\
&=  (\lambda+\mu)\ast \nu)(E)
  + (\mu+\mu)\ast \nu)(E)
\end{align*}
The distribution of other side follows by commutativity.

\itemch{e}
\textbf{Discreteness.}
Assume both \(\mu\) and \(\lambda\) are discrete.
Let \(A,B\subset\R\) the countable sets on which the measures are concentrated
respectably. We freely use the notations like \(\mu(\{x\})=\mu(x)\).
Now
\begin{align*}
(\lambda\ast\mu)(E)
&= \int_{\R} \lambda(E-y)\,d\mu(y)
 = \sum_{y\in B} \lambda(E-y)\mu(y)
 = \sum_{y\in B} \left(\sum_{x\in A\cap (E-y)}\lambda(x)\right)\mu(y) \\
&= \sum{x\in A}\sum_{y\in B} \chhi_E(x+y)\lambda(x)\cdot\mu(y)
\end{align*}
Therefore \(\lambda\ast\mu\) is concentrated in \(\{x+y: (x,y)\in A\times B\}\).
and 
\begin{equation*}
(\lambda\ast\mu)(w) = \sum_{\stackrel{(x,y)\in  A\times B}{x+y=w}} \lambda(x)\cdot\mu(y).
\end{equation*}


\textbf{Continuity.}
Assume \(\mu\) is continuous. Pick arbitrary \(x\in\R\).
\begin{equation*}
(\mu\ast\lambda)(x) 
= \int_{\R} \mu(x-y)\,d\lambda(y) = 
= \int_{\R} 0\,d\lambda(y) = 0.
\end{equation*}
Hence \(\mu\ast\lambda\) is continuous.

\textbf{Absolute Continuity.}
Assume \(\mu \ll m\) and \(m(E)=0\). 
Clearly \(\mu(E-y)=0\) for each \(y\in\R\) since \(m(E-y)=0\)
by construction of the Lebesgue measure. Now
\begin{equation*}
(\mu\ast \lambda)(E) 
= \int_{\R} \lambda(E-y)\,d\mu(y)
= \int_{\R} 0\,d\mu(y) = 0
\end{equation*}

\itemch{f}
Take a Borel set $E$. Using the fact that $m$ is a measure that 
is invariant under translation and 
\index{Fubini}
Fubini's Theorem~8.8
\begin{align*}
(\mu\ast\lambda)(E)
&= \int_{\R} \chhi_E\,d(\mu\ast\lambda)
 = \int_{\R} \mu(E-y)\,d\lambda(y)
 = \int_{\R} \left(\int_{E-y}\!f(x)\,dm(x)\right)g(y)\,dm(y) \\
&= \int_{\R} \left(\int_E f(x-y)\,dm(x)\right)g(y)\,dm(y) \\
&= \int_{\R} \left(\int_{\R} \chhi_E(x) f(x-y)g(y)\,dm(x)\right)\,dm(y) \\
&= \int_{\R} \left(\int_{\R} \chhi_E(x) f(x-y)g(y)\,dm(y)\right)\,dm(x) \\
&= \int_{\R} \chhi_E(x) \left(\int_{\R} f(x-y)g(y)\,dm(y)\right)\,dm(x) 
 = \int_{\R} \chhi_E(x) (f\ast g)(x)\,dm(x) \\
&= \int_{\R} \chhi_E(x)\,d(f\ast g)(x)m(x)
\end{align*}


\itemch{g}
Discrete measure are subalgebra by \ich{e} and \ich{f}.
Closure under addition and scalar multiplication is trivial.

Let $C$ be the family of continuous measures.
Again, closure under addition and scalar multiplication is trivial.
By \ich{e} and \ich{f} we have shown that $C$ is an ideal.

Let $A$ be the family of absolutely continuous measures with respect to $m$.
Again, closure under addition and scalar multiplication is trivial.
If \(\mu\ll m\) then by 
\index{Radon-Nikodym}
Radon-Nikodym Theorem~6.10 trivially extended to complex measures,
there exists $m$-measurable $f$
such that \(f\,dm = d\mu\). With this,\ich{e} and \ich{f} $A$ is an ideal
and the mapping \(\mu \to f\) is an isomorphism.


\itemch{h}
Put \(\delta(E) = \chhi_E(0)\). Verification is trivial.

\itemch{i}
For \(E\subset \T^k\) we need to modify the definition 
\begin{equation*}
E_2 = \left\{(u_1,u_2)\in \left(\T^k\right)^2: \exists u\in E,\;
         \forall j\in\N_k,\; (z_1)j\cdot(z_2)_j = u_j\right\}
\end{equation*}
Now all arguments can be easily applied to \(\R^k\) and \(\T^k\).
\end{itemize}


%%%%%%%%%%%%%% 6
\begin{excopy}
(Polar coordinates in \(\R^k\).)
Let \(S_{k-1}\) be the unit sphere in \(\R^k\), 
i.e., the set of all \(u \in \R^k\) whose distance from the origin $0$ is $1$.
Show that every \(x\in \R^k\), except for \(x=0\), has a unique representation
of the form \(x=ru\), where $r$ is a positive real number and \(u\in S_{k-1}\).
Thus \(\R^k\setminus\{0\}\) may be regarded as the cartesian product
\((0,\infty)\times S_{k-1}\).

Let \(m_k\) be the Lebesgue measure on \(\R^k\), and define a measure
\(\sigma_{k-1}\) on \(S_{k-1}\) as follows:
If \(A \subset S_{k-1}\) and $A$ is a Borel set, let \(\overline{A}\)
be the set of all points \(ru\), where \(0 < r < 1\)
and \(u\in A\), and define
\begin{equation*}
\sigma_{k-1}(A) = k \cdot m_k(\overline{A}).
\end{equation*}
Prove that the formula
\begin{equation*}
\int_{R^k} f\,dm_k 
= \int_0^\infty r^{k-1}\,dr \int_{S_{k-1}} f(ru)\,d\sigma_{k-1}(u)
\end{equation*}
is valid for every nonnegative Borel function $f$ on \(\R^{k}\).
Check that this coincides with familiar results 
when \(k=2\)
and when \(k=3\).

\emph{Suggestion}: If \(0 < r_1 < r_2\) and if $A$ 
is an open subset of \(S_{k-1}\), let $E$ be the set of all \(ru\) with
\(r_1 < r < r_2\), \(u\in A\),
and verify that the formula holds for the characteristic function of $E$.
Pass from these to characteristic functions of Borel sets in \(\R^k\).
\end{excopy}

Pick $E$ as suggested. Now
\begin{align*}
\int_{R^k} \chhi_E\,dm_k 
&= m_k(E) 
 = m_k\left(\{ru\in \R^k: u\in A\;\wedge\; r_1<r<r_2\}\right) \\
&=   m_k\left(\{ru\in \R^k: u\in A\;\wedge\; r<r_2\}\right)
   - m_k\left(\{ru\in \R^k: u\in A\;\wedge\; r\leq r_1\}\right) \\
&= m_k(\overline{A})\left(r_2^k - r_1^k\right)
 = k\int_{r_1}^{r_2} r^{k-1}m(\overline{A})\,dr \\
&= \int_{r_1}^{r_2} r^{k-1}\sigma_{k-1}(A)\,dr
 = \int_{r_1}^{r_2} r^{k-1}\left(\int_A 1\,d\sigma_{k-1}(u)\right)\,dr \\
&= \int_0^\infty r^{k-1}\,dr \int_{S_{k-1}} \chhi_E(ru)\,d\sigma_{k-1}(u)
\end{align*}
Thus the desired equality holds for \(f=\chhi_E\).

Now every Borel function \(f\geq 0\) can be monotonically approximated
by simple functions of the forum
\begin{equation*}
s(x) = \sum_{j\in\N} a_j\chhi_{E_j}
\end{equation*}
where \(a_j\geq 0\) and \(E_j\) are sets of the suggested type.
Note: For such sets, \(A_j\) sets can be picked so they are intersection
of \(B_k(u;\epsilon)\) with \(u\in S_{k-1}\) and \(\epsilon>0\).
By Lebesgue's monotone~ convergence Theorem~1.34
the equality holds for non-negative Borel functions.


%%%%%%%%%%%%%% 7
\begin{excopy}
Suppose \((X,\calG, \mu)\)
and \((Y,\calF, \lambda)\)
are finite measure spaces, and suppose \(\psi\) is a measure on
\(\calG\times\calF\) such that 
\begin{equation*}
\psi(A\times B) = \mu(A) \mu(B)
\end{equation*}
whenever \(A\in\calG\) and \(B\in\calF\).
Prove that then \(\psi(E) = (\mu\times\lambda)(E)\) 
for every \(E\in \calG\times\calF\).
\end{excopy}

Let \calE\ be the family if sets $E$ for which the desired equality holds.
Clearly all the elementary sets are in \calE\ and it is monotonic.
By Theorem~8.3  \(\calG\times\calF \subset \calE\).


%%%%%%%%%%%%%% 8
\begin{excopy}
\begin{itemize}
\itemch{a}
Suppose $g$ is a real function on \(\R^k\) such that each section \(f_x\)
is a Borel measurable and each section \(f^y\) is continuous.
Prove that $f$ is Borel measurable on \(\R^2\).
Note the contrast between this and Example~8.9(c).
\itemch{b}
Suppose $g$ is a real function on \(\R^k\) which is continuous in each 
of the $k$ variables separately.
More explicitly, for every choice of \(x_2,\ldots\,x_n\), the mapping
\(x_1 \to g(\seqn{x})\) is continuous, etc.
Prove that $g$ is a Borel function.
\end{itemize}
\emph{Hint}: If \((i-1)/n = a_{i-1} \leq x \leq a_i = i/n\), put
\begin{equation*}
f_n(x,y) = 
\frac{a_i - x}{a_i - a_{i-1}} f(a_{i-1}, y) 
+
\frac{x - a_{i-1}}{a_i - a_{i-1}} f(a_i, y).
\end{equation*}
\end{excopy}

\begin{itemize}
\itemch{a}
We first look at the function \(g_2:\R^2\to\R^2\) defined by
\(g_2(x,y) = (x,f_a(y))\).
Pick a base open set 
\begin{equation*}
G = (u-\delta_x,u+\delta_x)\times(v-\delta_y,u+\delta_y) \subset \R^2.
\end{equation*}
Now
\begin{equation*}
g_2^{-1}(G) = (u-\delta_x,u+\delta_x)\times f_a^{-1}(v-\delta_y,u+\delta_y).
\end{equation*}
which is a cartesian product of two Borel sets, hence
\(g_2^{-1}(G)\) is a Borel set and \(g_2\) is a Borel function.
By Theorem~1.12\ich{d} \(xg_2(x,y) = x\cdot f_a(x,y)\) are Borel
functions for all \(a\in\R\).
Hence the functions \(f_n\) (defined in the hint) are Borel functions.

Clearly for \(\lim_n f_n(x,y) = f(x,y)\) for all \((x,y)\in\R^2\).
If \(G\subset \R^2\) is any open set then
\begin{equation*}
f^{-1}(G) = \bigcup_{m\in\N} \left(\bigcap_{n\geq m} f_n^{-1}(G)\right)
\end{equation*}
Since each \(f_n^{-1}(G)\) is a Borel set, 
by being a \salgebra\ \(f^{-1}(G)\) is a Borel set. 
Therefore $f$ is Borel measurable.

\itemch{b}

Just for comparison with continuity, recall the following example:
\begin{equation*}
f(x,y) = \left\{
\begin{array}{ll}
0 & \textnormal{if}\; x= y = 0 \\
\frac{xy}{x^2+y^2}  \quad & \textnormal{otherwise}
\end{array}\right.
\end{equation*}
Now 
\begin{align*}
\lim_{x\to 0} f(x,0) &= 0 \\
\lim_{y\to 0} f(0,y) &= 0 \\
\lim_{x\to 0} f(x,ax) &= \frac{a}{1+a^2} 
   = \left(\frac{1}{a}\right)\,
     \bigm/\,
     \left(1+\left(\frac{1}{a}\right)^2\right)
     \qquad (a\neq 0)
\end{align*}

By induction. If \(k=1\) then trivially $g$ is continuous,
and every continuous function is also a Borel function.
Now assume that \(k=n\) and that the claim holds for all \(k<n\).
For each fixed \(x_n\) the function \(\tilde{g}:\R^{n-1}\to\R\) defined by 
\begin{equation*}
\tilde{g}(x_1,x_2,\ldots,x_{n-1}) = g(x_1,x_2,\ldots,x_{n-1},x_n)
\end{equation*}
is Borel.
Similar to previous item, 
for each \((x_1,x_2,\ldots,x_{n-1})\in\R^{n-1}\) and 
\begin{equation*}
x_n \in [a_{i-1},a_i] = [(i-1)/n, i/n]
\end{equation*}
we define
\begin{equation*}
f_n(x_1,\ldots,x_n) = 
\frac{a_i - x_n}{a_i - a_{i-1}} g(x_1,\ldots,x_{n-1},a_{i-1}) 
+
\frac{x_n - a_{i-1}}{a_i - a_{i-1}} f(x_1,\ldots,x_{n-1},a_i).
\end{equation*}
Clearly \(\lim_{n\to\infty} f_n = f\) pointwise, and by similar argumnets 
the limit of Borel function is Borel as well. Thus $f$ is a Borel function.
\end{itemize}


%%%%%%%%%%%%%% 9
\begin{excopy}
Suppose $E$ is a dense set in \(\R^1\) and $f$ is a real function on \(\R^2\)
such that 
\ich{a} \(f_x\) is Lebesgue measurable for each \(x\in E\) and 
\ich{b} \(f^y\) is continuous for almost all  \(y\in \R^1\).
Prove that $f$ is a Lebesgue measurable on \(\R^2\).
\end{excopy}

Enumerate $E$ as \(\{a_j\}_{j\in\N}\) such that for each $n$ 
there exists some $n$ such that 
for each \(x\in[-n,n]\) we can there exists \(j,k\in\N_m\)
such that \(a_j\leq x \leq a_k\) and \(a_k-a_j < 1/n\).
(Note that \(2n^2 \leq m < \infty)\).) Let
\(K_n = \left[\min(\{a_j: j\leq n\}), \max(\{a_j: j\leq n\})\right]\)

Let $D$ be the set of all \(y\in\R\) such that \(f^y\) is not continuous.
Clearly \(m(D)=0\) and also 
\(m_2\left(\{(x,y)\in\R^2: y\in D\}\right) = m_2(\R\times D) = 0\).

Now define 
\begin{equation*}
f_n(x,y) = \left\{
\begin{array}{ll}
0 & y \in D \;\vee\; x \notin K_n \\
\frac{a_i - x}{a_i - a_{i-1}} f(a_{i-1}, y) 
+
\frac{x - a_{i-1}}{a_i - a_{i-1}} f(a_i, y) & \textnormal{otherwise}.
\end{array}
\right.
\end{equation*}
Clearly \(f_n\) are Lebesgue measurable, and
\(\lim_{n\to\infty} f_n = f \aded\) Thus $f$ is Lebesgue measurable.


%%%%%%%%%%%%%% 10
\begin{excopy}
Suppose $f$ is a real function on \(\R^2\),
\(f_x\) is a Lebesgue measurable for each $x$,
and \(f^y\) is continuous for each $y$.
Suppose \(g: \R^1\to \R^1\) is Lebesgue measurable,
and put \mbox{\(h(y) = f(g(y),y)\)},
Prove that $h$ is Lebesgue measurable on \(\R^1\).

\emph{Hint}: Define \(f_n\) as in Exercise~8, put \(h_n(y) = f_n(g(y),y)\).
show that each \(h_n\) is measurable and that \(h_n(y) \to h(y)\).
\end{excopy}

Following the hint. For each \(y\in\R\) and \(n\in\N\)
find 
\begin{equation*}
a_{i-1} = (i-1)/n \leq g(y) \leq i/n = a_i
\end{equation*}
and define
\begin{equation*}
h_n(y) 
= f_n(g(y),y) 
= n\left(a_i-g(y)\right)\cdot f(a_{i-1},y) + n(g(y)-a_{i-1})\cdot f(a_i,y).
\end{equation*}
By Theorem~1.9\ich{c} \(h_n\) is Lebesgue measurable.
Fix \(y\in\R\) since \(a_{i}-a_{i-1} = 1/n\) and \(f^y\) is a continuous section
we have \(\lim_{n\to\infty}h_n(y) = h(y)\).
By Corollary~\ich{a} of Theorem~1.14 $h$ is Lebesgue measurable.


%%%%%%%%%%%%%% 11
\begin{excopy}
Let \(\calB_k\) be the \salgebra\ of all Borel sets in \(\R^k\).
Prove that \(\calB_{m+n} = \calB_m \times \calB_n\).
This is relevant in Theorem~8.14.
\end{excopy}

% It is trivial to see that \(\calB_{m+n} \supseteq \calB_m \times \calB_n\).
We start with a topological lemma.
\begin{llem} \label{lem:open-countable}
If \(G\in\R^n\) is open, then it is a union of a countable collection
of $n$-cubes.
\end{llem}
\begin{thmproof}
Let \(D=\Q^n\cap G\) a countable dense set in $G$.
For each \(x\in D\) let \(C_x\) be the maximal open cube whose center is $x$
and \(C_w\subset G\). Clearly \(\cup_{w\in D} C_w \subset G\).
Assume by negation \(y\in G \setminus \cup_{w\in D} C_w\).
Let \(\delta>0\) be such that 
\begin{equation*}
C_y(\delta) := \{x\in \R^n: \|x-y\|_1<\delta\} \subset G.
\end{equation*}
Pick some \(x\in C_y(\delta/2)\cap G\), then 
\(y\in C_x\), with the contradiction \(y\in \cup_{w\in D} C_w\).
\end{thmproof}

By the Lemma every open set \(G\in \R^{m+n}\) is a countable union
of open ``rectangles'' \(X\times Y\subset \R^m\times\R^n\) 
where \(X\in\R^m\) and \(Y\in\R^n\) are open sets.
Hence \(G\in \calB_m \times \calB_n\).
But \(\calB_{m+n}\) is a minimal \salgebra\ that contains all open sets
in \(\R^{m+n}\) and so \(\calB_{m+n} \subset \calB_m \times \calB_n\).

To show the opposite inclusion it suffices to show that if
an arbitrary elementary set
\(E = E_m\times E_n \in \calB_m \times \calB_n\)
then \(E\in \calB_{m+n}\).
But since \(E_m\in \calB_m\) then by minimality of the 
Borel \salgebra\ \(\calB_m\) we also have (its inverse projection)
\(E_m\times\R^n\in\calB_{m+n}\).
Similarly \(R^m\times E_n\in\calB_{m+n}\). Now
\begin{equation*}
E = E_m\times\R^n \;\cap\;R^m\times E_n \in \calB_{m+n}.
\end{equation*}


%%%%%%%%%%%%%% 12
\begin{excopy}
Use Fubini's
\index{Fubini}
Theorem~8.8 and the relation
\begin{equation*}
\frac{1}{x} = \int_0^\infty e^{-xt}\,dt \qquad (x>0)
\end{equation*}
to prove that 
\begin{equation*}
\lim_{A\to\infty} \int_0^A \frac{\sin x}{x}\,dx = \frac{\pi}{2}.
\end{equation*}
\end{excopy}

\begin{align*}
\lim_{A\to\infty}\int_0^A \frac{\sin x}{x}\,dx 
&= \int_0^\infty \sin x\left( \int_0^\infty e^{-xt}\,dt\right)\,dx 
 = \int_0^\infty \left(\int_0^\infty e^{-xt}\frac{e^{ix}-e^{-ix}}{2i}\,dt\right)dx \\
&= \frac{1}{2i}\int_0^\infty \left(\int_0^\infty e^{-xt}(e^{ix}-e^{-ix})\,dt\right)dx
   \\
&= \frac{1}{2i}\int_0^\infty 
        \left(\int_0^\infty e^{(-t+i)x}-e^{-(t+i)x}\,dx\right)dt 
   \qquad \textnormal{(Fubini)}
   \\
&= \frac{1}{2i}\int_0^\infty \left(
       \frac{e^{(-t+i)x}}{-t+i} - \frac{e^{-(t+i)x}}{-t-i}
    \right)\biggm|_{x=0}^\infty \,dt \\
&= \frac{1}{2i}\int_0^\infty 
        \left(0 - \frac{1}{-t + i}\right)
        - \left(0 - \frac{1}{-t - i}\right)\,dt \\
&= \frac{1}{2i}\int_0^\infty 
        \frac{1}{-t - i} - \frac{1}{-t + i}\,dt
 = \frac{1}{2i}\int_0^\infty \frac{2i}{t^2 + 1}\,dt
 = \int_0^\infty \frac{1}{t^2 + 1}\,dt \\
&= \left(\tan^{-1}(t)\right)\bigm|_{t=0}^\infty
 = \tan^{-1}(\infty) - \tan^{-1}(0) = \pi/2
\end{align*}


%%%%%%%%%%%%%% 13
\begin{excopy}
If \(\mu\) is a complex measure on a \salgebra\ \frakM, show that every 
set \(E\in\frakM\) has a subset $A$ for which 
\begin{equation*}
|\mu(A)| \geq \frac{1}{\pi} |\mu|(E).
\end{equation*}
\emph{Suggestion}: There is a measurable real function \(\theta\) so that
\(d\mu = e^{i\theta}\,d|\mu|\).
Let \(A_\alpha\) be the subset of $E$ where \(\cos(\theta - \alpha)>0\), 
show that
\begin{equation*}
\Re[e^{-i\alpha}\mu(A_\alpha)] = \int_E \cos^+(\theta - \alpha)\,d|\mu|.
\end{equation*}
and integrate with respect to \(\alpha\) (as in Lemma~6.3).

Show by an example, that \(1/\pi\) is the best constant in this inequality.
\end{excopy}

Compute
\begin{align*}
e^{-i\alpha} \mu(A_\alpha)
&= e^{-i\alpha}\int_{A_\alpha} 1\,d\mu
 = \int_{A_\alpha} e^{i(\theta - \alpha)}\,d|\mu|
 = \int_{A_\alpha} \cos(\theta-\alpha)\,d|\mu| 
   + i\int_{A_\alpha} \sin(\theta - \alpha)\,d|\mu|.
\end{align*}
Hence
\begin{equation*}
\Re[e^{-i\alpha}\mu(A_\alpha)]
= \int_{A_\alpha} \cos(\theta-\alpha)\,d|\mu| 
= \int_E \cos^+(\theta-\alpha)\,d|\mu|.
\end{equation*}

Integrate by \(\alpha\)
\begin{align*}
\int_0^{2\pi} \Re[e^{-i\alpha}\mu(A_\alpha)]\,d\alpha
&= \int_0^{2\pi} 
   \left(\int_E \cos^+(\theta(x)-\alpha)\,d|\mu|(x)\right)\,d\alpha  \\
&= \int_E \left(
      \int_0^{2\pi} \cos^+(\theta(x)-\alpha)\,d\alpha
          \right)\,d|\mu|(x)  \qquad\textnormal{(Fubini)} \\
&= \int_E \left(
      \int_0^{2\pi} \cos^+(-\alpha)\,d\alpha \right)\,d|\mu|(x) 
 = \int_E 2\,d|\mu|(x) 
 = 2|\mu|(E).
\end{align*}

Assume by negation
\(|\mu(A_\alpha)| <  |\mu|(E)/\pi\) 
for all \(\alpha\in[0,2\pi]\). Then
\begin{align*}
\left|\int_0^{2\pi} \Re[e^{-i\alpha}\mu(A_\alpha)]\,d\alpha\right|
&\leq \int_0^{2\pi} \left|\Re[e^{-i\alpha}\mu(A_\alpha)]\right|\,d\alpha
 \leq \int_0^{2\pi} |\mu(A_\alpha)|\,d\alpha \\
&< \int_0^{2\pi} |\mu|(E)/\pi\,d\alpha
  = 2|\mu|(E)
\end{align*}
that contradicts the previous established equality.
Hence there exists some \(\alpha\)
such that \(|\mu(A_\alpha)| \geq |\mu|(E)/\pi\).

\paragraph{Best Constant.}
Consider the unit circle \(\gamma:[0,2\pi]\to\C\) 
defined by \(\gamma(t) = e^{it}\) with the complex measure
\(d\mu = \lambda'(t)\,dm(t)\).
Clearly \(|\mu|(\gamma^*) = 2\pi\).

For any \(E\subset \gamma^*\) and any \(u\in\C\) such that \(|u|=1\)
we put \(E_u = \{uz: z\in E\}\).
It is easy to see that \(E_u\subset \gamma^*\) and that 
\(|\mu(E)| = |\mu(E_u)|\).

Take 
\begin{equation*}
A = \{\gamma(t): t\in[0,2\pi] \wedge\;\Re(\gamma'(t)) \geq 0\}
  = \{z\in\gamma^{*}: \Im(z) \leq 0\}
\end{equation*}
Now
\begin{equation*}
\mu(A) = \int_{\pi}^{2\pi} = 1\cdot(\gamma^{it})'\,dt
 = \gamma(2\pi) - \gamma(\pi) = 2 = |\mu|(\gamma^*)/\pi.
\end{equation*}

We will show that $A$ consist of the positive part of \(\mu\)
and \(\gamma^*\setminus A\) of the negative.
Let \(P \subset A\) and \(N \subset \gamma^*\setminus A\), then
using the definition of $A$ we have
\begin{equation*}
\Re(\mu(P)) 
= \int_P \Re\left((\gamma^{it})'\right)\,dt
= \int_P \left|\Re\left((\gamma^{it})'\right)\right|\,dt
\geq 0.
\end{equation*}
and
\begin{equation*}
\Re(\mu(N)) 
= \int_N \Re\left((\gamma^{it})'\right)\,dt
= \int_N -\left|\Re\left((\gamma^{it})'\right)\right|\,dt
\leq 0.
\end{equation*}

Assume by negation there exists some \(E\subset\gamma^*\) such that 
\(|\mu(E)| > 2\). \Wlogy\ (or by replacing $E$ by some \(E_u\)
we may assume that \(\mu(E) > 0\) is real.
Now we get the following contradiction
\begin{align*}
\mu(E) 
&= \Re(\mu(E\cap A)) + \Re(\mu(E\setminus A))
 \leq \Re(\mu(E\cap A))
 = \Re(\mu(E\cap A)) \\
&\leq \Re(\mu(E\cap A)) + \Re(\mu(A\cap E))
 = \Re(\mu(A)) 
 = \mu(A)
 = 2.
\end{align*}

%%%%%%%%%%%%%% 14
\begin{excopy}
Complete the following proof of Hardy's
\index{Hardy} 
inequality
(Chap.~3 Exercise~14).
Suppose \(f\geq 0\) on \((0,\infty)\), \(f\in L^p\), \(1<p<\infty\), and
\begin{equation*}
F(x) = \frac{1}{x}\int_0^x f(t)\,dt.
\end{equation*}
Write \(xF(x) = \int_0^x f(t)t^\alpha t^{-\alpha}\,dt\),
where \(0<\alpha<1/q\), use H\"older's inequality 
to get an upper bound for
\(F(x)^p\), and integrate to obtain
\begin{equation*}
\int_0^\infty F^p(x)\,dx 
\leq 
(1-\alpha q)^{1-p} (\alpha p)^{-1} \int_0^\infty f^p(t)\,dt.
\end{equation*}
Show that the best choice of \(\alpha\) yields
\begin{equation*}
\int_0^\infty F^p(x)\,dx 
\leq 
\left(\frac{p}{p-1}\right)^{p} \int_0^\infty f^p(t)\,dt.
\end{equation*}
\end{excopy}

In \cite{Garling2007} Section~7.3,
given \(\mu\)-measurable \(f:\Omega\to\C\)
\index{maximal function!Muirhead}
\index{Muirhead!maximal function}
Muirhead's maximal function is introduced, 
\begin{equation*}
f^\dagger(t) = \sup\left\{\frac{1}{t}\int_E|f|\,d\mu: \mu(E)\leq t\right\}
\qquad \textnormal{where}\; 0 < t < \mu(\Omega).
\end{equation*}

The treatment there (\cite{Garling2007}) uses
the following notation
\begin{equation*}
\lambda_f(t) = \mu(f > t) \qquad (t\geq 0)
\end{equation*}
for the distribution function.


See \cite{Garling2007} Theorem~8.1.1 and its Corollary.

Using H\"older's inequality:
\begin{equation*}
\int_0^x f(t)\,dt
= \int_0^x f(t)t^\alpha t^{-\alpha}\,dt
\leq \left(\int_0^x \left(f(t)t^\alpha\right)^p\,dt\right)^{1/p}
      \cdot \left(\int_0^x t^{-\alpha q}\,dt\right)^{1/q}
\end{equation*}
Simplifying the last term
\begin{equation*}
\left(\int_0^x t^{-\alpha q}\,dt\right)^{1/q}
= \left(\frac{x^{1-\alpha q}}{1-\alpha q}\right)^{1/q}.
\end{equation*}
Combining with the inequality gives
\begin{align*}
F(x)^p
&\leq \left(x^{-1} \left(\frac{x^{1-\alpha q}}{1-\alpha q}\right)^{1/q}\right)^p
      \left(\int_0^x \left(f(t)t^\alpha\right)^p\,dt\right)
 = (1-\alpha q)^{-p/q} x^{-\alpha p - 1}
   \left(\int_0^x \left(f(t)t^\alpha\right)^p\,dt\right) \\
&= (1-\alpha q)^{1-p} x^{-\alpha p - 1}
   \left(\int_0^x \left(f(t)t^\alpha\right)^p\,dt\right).
\end{align*}
The power of $x$ was simplified
by \((-1 + (1-\alpha q)\frac{1}{q})p = -p + p/q - \alpha p = -\alpha p - 1\).

We now integrate using 
\index{Fubini}
Fubini's Theorem~8.8 on \(\{(x,t)\in\R^2: 0\leq t \leq x\}\)
\begin{align*}
\int_0^\infty F^p(x)\,dx 
&\leq \int_0^\infty
  \left((1-\alpha q)^{1-p} x^{-\alpha p - 1}
   \left(\int_0^x \left(f(t)t^\alpha\right)^p\,dt\right)\right)\,dx
   \\
&= (1-\alpha q)^{1-p}
   \int_0^\infty
   \left( x^{-\alpha p - 1}
   \left(\int_0^x \left(f(t)t^\alpha\right)^p\,dt\right)\right)\,dx
   \\
&= (1-\alpha q)^{1-p}
   \int_0^\infty 
     \left(\int_t^\infty x^{-\alpha p - 1}
                 \left(f(t)t^\alpha\right)^p\,dx\right)\,dt
   \\
&= (1-\alpha q)^{1-p}
   \int_0^\infty 
     \left(\left(f(t)t^\alpha\right)^p
       \int_t^\infty x^{-\alpha p - 1}
                 \,dx\right)\,dt.
\end{align*}
The inner integral is simplified as follows:
\begin{equation*}
\int_t^\infty x^{-\alpha p - 1}
= -\frac{1}{\alpha p}\left(x^{-\alpha p}\right)\biggm|_{x=t}^\infty
= -(\alpha p)^{-1}\left(0 - t^{-\alpha p}\right)/p = (\alpha p)^{-1} t^{-\alpha p}.
\end{equation*}
Back to previous integration
\begin{align*}
\int_0^\infty F^p(x)\,dx 
 &\leq (1-\alpha q)^{1-p}
   \int_0^\infty \left(f(t)t^\alpha\right)^p (\alpha p)^{-1} t^{-\alpha p}/p\,dt
 \\
 &= (1-\alpha q)^{1-p} (\alpha p)^{-1} \int_0^\infty \bigl(f(t)\bigr)^p\,dt
\end{align*}

\paragraph{Best Constant.}
We want to minimize
\begin{equation*}
b(\alpha) = (1-\alpha q)^{1-p} (\alpha p)^{-1}
\end{equation*}
where \(0 < \alpha < 1/q\)
This is equivalent to maximize  
\begin{equation*}
c(\alpha) = (1-\alpha q)^{p-1} \alpha p.
\end{equation*}
where \(0 < \alpha q < 1\).
Clearly in this domain \(c(\alpha) > 0\). We differentiate and equate to zero
\begin{gather*}
c'(\alpha) =  (1 - \alpha q)^{p-1} - \alpha q (p-1)(1 - \alpha q)^{p-2} = 0 \\
1 - \alpha q = \alpha q(p-1) \\
\alpha = \frac{1}{pq}\,.
\end{gather*}
Hence the best constant is
\begin{align*}
b(\alpha) 
&= (1-\alpha q)^{1-p} (\alpha p)^{-1}
= \left(1-\frac{1}{p}\right)^{1-p} \left(\frac{1}{q}\right)^{-1}
= \left(\frac{1}{q}\right)^{1-p} \left(\frac{1}{q}\right)^{-1}
= \left(\frac{1}{q}\right)^{-p}
\\
&= \left(1-\frac{1}{p}\right)^{-p}
= \left(\frac{p-1}{p}\right)^{-p}
= \left(\frac{p}{p-1}\right)^{p}\,.
\end{align*}

%%%%%%%%%%%%%% 15
\begin{excopy}
Put \(\varphi(t) = 1 - \cos t\) if \(0\leq t \leq 2\pi\), 
\(\varphi(t) = 0\) for all other real $t$.
For \(-\infty < x < \infty\), define
\begin{equation*}
f(x) = 1, 
\qquad g(x) = \varphi'(x),
\qquad h(x) = \int_{-\infty}^x \varphi(t)\,dt.
\end{equation*}
Verify the following statements about convolutions of these functions:
\begin{itemize}
\item[(i)] \((f\ast g)(x) = 0\) for all $x$.
\item[(ii)] \((g\ast h)(x) = (\varphi \ast \varphi)(x) > 0\) on \((0,4\pi)\).
\item[(iii)] Therefore \((f\ast g)\ast h = 0\), 
             whereas \(f\ast (g\ast h)\) is a positive constant.
\end{itemize}
But convolution is supposedly associative, by Fubini's Theorem~8.8
\index{Fubini}
(Exercise~5(c)). What went wrong?
\end{excopy}

First compute the functions
\begin{equation*}
g(x) = \left\{\begin{array}{ll}
0            & x < 0 \\
\sin x \quad & 0 < x < 2\pi \\
0            & x > 2\pi
\end{array}
\right.
\qquad
h(x) = \left\{\begin{array}{ll}
0                & x \leq 0 \\
x - \sin x \quad & 0 \leq x \leq 2\pi \\
2\pi             & x \geq 2\pi
\end{array}
\right.
\end{equation*}

\begin{itemize}
\item[(i)]
Compute:
\begin{equation*}
(f\ast g)(x) 
= \int_\R f(x-t)g(t)\,dt
= \int_\R g(t)\,dt
= \int_0^{2\pi} \sin(t)\,dt 
= 0
\end{equation*}

\item[(ii)]
We note that \(0 < x-t < 2\pi\) iff  \(x-2\pi < t < x\).
Compute:
\begin{align*}
(g\ast h)(x)
&= \int_\R g(x-t)h(t)\,dt
 = \int_0^{2\pi} g(x-t)\cdot (t - \sin t)\,dt
% = \int_\R g(t)h(x-t)\,dt
% = \int_0^{2\pi} \sin(t)\cdot h(x-t)\,dt
 \\
&= \left\{\begin{array}{ll}
   \int_{\max(x-2\pi,0)}^{\min(x,2\pi)} \sin (x-t)\cdot (t - \sin t)\,dt 
        \quad & x \in [0,4\pi] \\
   0  & x \notin [0,4\pi]
   \end{array}\right.
\end{align*}
We have 
\begin{itemize}
\item[\(\circ\)] \(\max(x-2\pi,0) < \min(x,2\pi)\) when \(0<x<4\pi\),
\item[\(\circ\)] \(t-\sin t > 0\) for all \(t>0\),
\item[\(\circ\)] \(\sin(x-t)>0\) when \(0 < x-t < 2\pi\).
\end{itemize}
Hence \((g\ast h)(x)>0\) when \(0<x<4\pi\).

\item[(iii)]
Clearly by linearity of convolution.

\end{itemize}
The associativity holds in Exercise~5\ich{c} for complex measures, 
meaning finite measures on~\R. Viewing these as functions, 
the associativity holds for functions in \(L^1(\R)\) but clearly
here \(f\notin L^1(\R)\).

%%%%%%%%%%%%%% 16
\begin{excopy}
Prove the following analogue of Minkowski's inequality
\index{Minkowski}
, for \(f\geq 0\):
\begin{equation*}
\left\{\int \left[\int f(x,y)\,d\lambda(y)\right]^p\,d\mu(x)\right\}^{1/p}
\leq
\int \left[\int f^p(x,y)\,d\mu(x) \right]^{1/p}\,d\lambda(y).
\end{equation*}
Supply the required hypothesis.
(Many further developments of this theme may be found in [9].)
\end{excopy}

See also Theorem~2.4 in \cite{LiebLoss200104}.

Let \((X,\mu)\) and \(Y,\lambda\) be \(\sigma\)-finite measurable spaces
and \(f:X\times Y\to  \R^\oplus\) measurable function.

\emph{Note:} If \(Y = \{0,1\}\) and \(\lambda\) is the counting measure,
then we get the known Minkowski's inequality
of Theorem~3.5(2). We follow similar idea as in the proof there.

We may assume that $f$ and \(\supp f\) are bounded
and generalize the result by applying
Lebesgue's monotone convergence Theorem~1.34.

Denote the two sides
and the inner integral of the left side of the desired inequality as
\begin{align*}
L &= \left(\int_X 
        \left(\int_Y f(x,y)\,d\lambda(y)\right)^p d\mu(x)\right)^{1/p} \\
R &= \int_Y\left(\int_X \left(f(x,y)\right)^pd\mu(x) \right)^{1/p} d\lambda(y) \\
\psi(x) &= \int_Y f(x,y)\,d\lambda(y).
\end{align*}
Fix \(v\in Y\), the 
\index{H\"older}
H\"older inequality of of Theorem~3.5(1) gives
\begin{equation*}
\int_X f(x,v)\left(\psi(x)\right)^{p-1} d\mu(x)
\leq \left(\int_X \left(f(x,v)\right)^p\,d\mu(x)\right)^{1/p} 
     \left(\int_X \left(\psi(x)\right)^{(p-1)q} d\mu(x)\right)^{1/q}
\end{equation*}
Using Fubini's Theorem~8.8 and \(p = (p-1)q\), we integrate both sides
\begin{align}
A
&= \int_Y 
    \left(\int_X f(x,v)\left(\psi(x)\right)^{p-1} d\mu(x)\right)\,d\lambda(v)
 = \int_X 
    \left(\int_Y f(x,v)\left(\psi(x)\right)^{p-1} d\lambda(v)\right)\,d\mu(x)
   \notag \\
&= \int_X 
    \left(\psi(x)\right)^{p-1} \left(\int_Y f(x,v)\,d\lambda(v)\right)\,d\mu(x)
 = \int_X \left(\psi(x)\right)^p d\mu(x) = L^p
   \notag \\
&\leq  \label{eq:minkowski:gen1}
    \left(
      \int_Y
        \left(\int_X \left(f(x,v)\right)^p\,d\mu(x)\right)^{1/p} 
      d\lambda(v)
    \right)
    \left(\int_X \left(\psi(x)\right)^p d\mu(x)\right)^{1/q} 
  \\
&= \left(\int_X \left(\psi(x)\right)^p d\mu(x)\right)^{1/q} \cdot R.
  \notag
\end{align}

We now have \(L = A^{1/p}\) and establish an inequality
\(A \leq A^{1/q} R\).
Three cases regarding $A$ are to be considered.

\paragraph{Case 1 (Zero).} 
If \(A=0\) then \(L = A^{1/p}=0\) and the desired inequality is trivial.  

\paragraph{Case 2 (Infinity).}
The case of \(A=\infty\) is impossible because of
our assumption that \(\|f\|_\infty<\infty\) 
and \((\mu\times\lambda)(\supp f) < \infty\).

\paragraph{Case 3 (Normal).}
If \(0 < A < \infty\) then we can safely divide the inequality
we got by \(A^{1/q}\) and since \(1-1/q=1/p\) we get the desired inequality,
\(A^{1/p} \leq R\) or explicitly
\begin{equation*}
 \left(\int_X 
    \left(\int_Y f(x,y)\,d\lambda(y)\right)^p d\mu(x)\right)^{1/p} 
\leq
 \int_Y\left(\int_X \left(f(x,y)\right)^pd\mu(x) \right)^{1/p} d\lambda(y).
\end{equation*}


%%%%%%%%%%%%%%%%%
\end{enumerate}

 %%%%%%%%%%%%%%%%%%%%%%%%%%%%%%%%%%%%%%%%%%%%%%%%%%%%%%%%%%%%%%%%%%%%%%%%
%%%%%%%%%%%%%%%%%%%%%%%%%%%%%%%%%%%%%%%%%%%%%%%%%%%%%%%%%%%%%%%%%%%%%%%%
%%%%%%%%%%%%%%%%%%%%%%%%%%%%%%%%%%%%%%%%%%%%%%%%%%%%%%%%%%%%%%%%%%%%%%%%
\chapterTypeout{Fourier Transforms}

%%%%%%%%%%%%%%%%%%%%%%%%%%%%%%%%%%%%%%%%%%%%%%%%%%%%%%%%%%%%%%%%%%%%%%%%
%%%%%%%%%%%%%%%%%%%%%%%%%%%%%%%%%%%%%%%%%%%%%%%%%%%%%%%%%%%%%%%%%%%%%%%%
\section{Comments and Clarifications}

%%%%%%%%%%%%%%%%%%%%%%%%%%%%%%%%%%%%%%%%%%%%%%%%%%%%%%%%%%%%%%%%%%%%%%%%
\subsection{Convolution} 

\begin{llem} \label{lem:conv:commut}
Convolution is commutative. If \(f,g\in L^1\) then
\(f\ast g = g\ast f\).
\end{llem}
\begin{thmproof}
\begin{align*}
(f\ast g)(x)
&= \int_\R f(x-y)g(y)\,dm(y)
 = \int_\R f(t)g(x-t)(dx/dt)\,dm(t)
 = \int_\R g(x-t)f(t)\,dm(t) \\
&= (g \ast f)(x)
\end{align*}
\end{thmproof}


%%%%%%%%%%%%%%%%%%%%%%%%%%%%%%%%%%%%%%%%%%%%%%%%%%%%%%%%%%%%%%%%%%%%%%%%
\subsection{Transform Formulas} \label{subsec:xform:formulas}


\newcommand{\sqdivtpi}{\ensuremath{\frac{1}{\sqrt{2\pi}}}}
In Theorem~9.2.
Suppose \(f\in \Lp1\), and \(\alpha,\lambda \in\R\).
\begin{itemize}

 \itemch{a} If \(g(x)=f(x)e^{i\alpha x}\) then
    \begin{eqnarray*}
      \Hat{g}(t)
            & = & \intR g(x)e^{-ixt}\,dm(x) \\
            & = & \intR f(x)e^{i\alpha x}e^{-ixt}\,dm(x) \\
            & = & \intR f(x)e^{-i(t-\alpha)x}\,dm(x) \\
            & = & \Hat{f}(t-\alpha).
    \end{eqnarray*}

  \itemch{b} If \(g(x)=f(x-\alpha)\) then
    \begin{eqnarray*}
      \Hat{g}(t)
            & = & \intR g(x)e^{-ixt}\,dm(x) \\
            & = & \intR f(x-\alpha)e^{-i((x-\alpha)+\alpha)t}\,dm(x) \\
            & = & e^{-i\alpha t}\intR f(x-\alpha)e^{-i(x-\alpha)t}\,dm(x) \\
            & = & \Hat{f}(t-\alpha)e^{-i\alpha t}.
    \end{eqnarray*}

  \itemch{d} If \(g(x)=\overline{f(-x)}\) then
    \begin{eqnarray*}
      \Hat{g}(t)
            & = & \intR g(x)e^{-ixt}\,dm(x) \\
            & = & \intR \overline{f(-x)}e^{-ixt}\,dm(x) \\
            & = & \intR \overline{f(-x)e^{-i(-x)t}}\,dm(x) \\
            & = & \overline{\intR f(-x)e^{-i(-x)t}\,dm(x)} \\
            & = & \overline{\Hat{f}(t)}.
    \end{eqnarray*}

  \itemch{e}
    Given \(g(x) = f(x/\lambda)\) and \(\lambda>0\)
    we use the substitution \(y=x/\lambda\) and \(dx/dy=\lambda\)
    \begin{align*}
    \widehat{g}(x)
     &= \int_{-\infty}^\infty g(x)e^{-itx}\,dm(x)
     = \int_{-\infty}^\infty f(x/\lambda)e^{-itx}\,dm(x) 
     = \int_{-\infty}^\infty f(y)e^{- it \lambda y}\lambda\,dm(yx) \\
     &= \lambda \widehat{f}(\lambda t)
    \end{align*}


\end{itemize}


%%%%%%%%%%%%%%%%%%%%%%%%%%%%%%%%%%%%%%%%%%%%%%%%%%%%%%%%%%%%%%%%%%%%%%%%
\subsection{Auxiliary Functions}

In Section~9.7, the following functions are defined:
\begin{gather}
H(t) = e^{-|t|} \\
h_\lambda(x) = \int_{-\infty}^\infty H(\lambda t)e^{itx}\,dm(t) 
 \qquad (\lambda > 0)
\end{gather}
Then text says:
\begin{quote}
A simple computation gives
\begin{equation}
 \tag{Rudin(3)}
h_\lambda(x) = \sqrt{\frac{2}{\pi}} \frac{\lambda}{\lambda^2 + x^2}
\end{equation}
\end{quote}
Let's compute it in details, in two ways.
\begin{itemize}

\item

Assuming \(\lambda>0\):
\begin{eqnarray*}
h_\lambda(x)
  & = & \intR H(\lambda t)e^{itx}dm(t) =
  % & = &
        \intR e^{-|\lambda t|}e^{itx}dm(t) \\
  & = & \int_{-\infty}^0 e^{(ix+\lambda)t}dm(t) +
        \int_0^\infty e^{(ix-\lambda)t}dm(t) \\
  & = & \left.\left(\sqdivtpi\frac{1}{ix+\lambda}e^{(ix+\lambda)t}
        \right)\right|_{-\infty}^0 +
        \left.\left(\sqdivtpi\frac{1}{ix-\lambda}e^{(ix-\lambda)t}
        \right)\right|_0^\infty \\
  & = & \sqdivtpi\frac{1}{ix+\lambda} - 0 +
        0 - \sqdivtpi\frac{1}{ix-\lambda} \\
  & = & \sqdivtpi\,\frac{-2\lambda}{(ix+\lambda)(ix-\lambda)} \\
  & = & \sqrt{\frac{2}{\pi}}\frac{\lambda}{\lambda^2+x^2}\>.
\end{eqnarray*}

\item
We use the indefinite integral formula 
(See \cite{Apostol1961}~6.17, Exercise~20)
\begin{equation*}
\int e^{ax}\cos(bx)\,dx = \frac{e^{ax}(a\cos(bx) + b\sin(bx))}{a^2+b^2} + C.
\end{equation*}
\iffalse
We use the indefinite integral formula
\begin{equation*}
\int 1/(a+x^2)\,dx = \arctan\left(x/\sqrt{a}\,\right)/\sqrt{a}
\end{equation*}
\fi
\iffalse
which can be verified by
\begin{equation*}
1=\frac{d}{dx}\tan\bigl(\arctan(\x)\bigr)
= \frac{d}{dx}\arctan(x) \frac{d}{dx}\tan\bigl(\arctan(\x)\bigr)
\end{equation*}
\fi

Now
\begin{align*}
h_\lambda(x) 
&= \int_{-\infty}^\infty H(\lambda t)e^{itx}\,dm(t)
  =  \int_{-\infty}^0 e^{-\lambda t} e^{itx}\,dm(t) 
   + \int_0^\infty e^{-\lambda t} e^{itx}\,dm(t)  \\
&= \int_0^\infty e^{-\lambda t} (e^{itx} + e^{-itx})\,dm(t)
 = \int_0^\infty e^{-\lambda t} 2\cos(tx)\,dm(t)  \\
&= \frac{2}{\sqrt{2\pi}}
   \left.\frac{e^{-\lambda t}(-\lambda \cos(tx) + x\sin(tx))}{\lambda^2+x^2} 
   \,\right|_{t=0}^{t=\infty} 
 = \sqrt{\frac{2}{\pi}}\cdot\frac{\lambda}{\lambda^2+x^2}
\end{align*}
\end{itemize}

Following in the text is the equality
\begin{equation} \tag{Rudin(4)}
\intR h_\lambda(x)dm(x) = 1
\end{equation}
that we will now verify.

From the derivative equation \(\arctan'(x) = 1/(1+x^2)\) we have
the indefinite integral:
\[\int \frac{1}{\lambda^2+x^2}dx = \arctan'(x/\lambda)/\lambda.\]
To get Equation~(3) there:
\begin{eqnarray*}
\intR h_\lambda(x)dm(x)
 & = & \sqrt{\frac{2}{\pi}}\sqdivtpi \intR \frac{\lambda}{\lambda^2+x^2}dx \\
 & = & (1/\pi)
       \left.\bigl(\arctan(x/\lambda)\bigr)\right|_{-\infty}^{\infty} \\
 & = & \frac{1}{\pi}\left(\frac{\pi}{2} - \frac{-\pi}{2}\right) = 1.
\end{eqnarray*}

%%%%%%%%%%%%%%%%%%%%%%%%%%%%%%%%%%%%%%%%%%%%%%%%%%%%%%%%%%%%%%%%%%%%%%%%
\subsection{Equality in Theorem 9.9}

The proof of Theorem~9.9 uses the equality
\begin{equation*}
h_\lambda(y) = \lambda^{-1}h_1\left(\frac{y}{\lambda}\right)\,.
\end{equation*}
Working it out:
\begin{equation*}
h_1(y/\lambda) 
= \sqrt{\frac{2}{\pi}}\frac{1}{1^2+(y/\lambda)^2}
= \sqrt{\frac{2}{\pi}}\frac{\lambda^2}{\lambda^2+y^2}
= \lambda h_\lambda(y).
\end{equation*}


%%%%%%%%%%%%%%%%%%%%%%%%%%%%%%%%%%%%%%%%%%%%%%%%%%%%%%%%%%%%%%%%%%%%%%%%
%%%%%%%%%%%%%%%%%%%%%%%%%%%%%%%%%%%%%%%%%%%%%%%%%%%%%%%%%%%%%%%%%%%%%%%%
\section{Additional Results}

%%%%%%%%%%%%%%%%%%%%%%%%%%%%%%%%%%%%%%%%%%%%%%%%%%%%%%%%%%%%%%%%%%%%%%%%
\subsection{Generalization of Theorem~9.2} \label{sec:gen:thm9.2}

Theorem~9.2 has several items, each of which gives some equality
under some conditions. Here we generalize item~\ich{f}.

\begin{llem} \label{lem:g-eq-xnf}
Assume \(f\in L^1\).
If \(g(x) = x^nf(x)\) and \(g\in L^1\) then
\(\hat{f}\) is $n$-times differentiable and
\begin{equation}
\hat{g}(t) = i^nD^n\hat{f}(t). \label{eq:gtDn}
\end{equation}
\end{llem}
\begin{thmproof}
The case \(n=1\) was proved in Theorem~9.2\ich{f}.
Assume that \eqref{eq:gtDn} holds for \(1\leq n < k\).
Put \(h(x) = x^kf(x) = xg(x)\). By induction
\begin{equation*}
\hat{h}(t) 
= i\hat{g}'(t) 
= iD^1\left(i^{k-1}D^{k-1}\hat{f}\right)(t)
= i^kD^k\hat{f}(t).
\end{equation*}
\end{thmproof}


%%%%%%%%%%%%%%%%%%%%%%%%%%%%%%%%%%%%%%%%%%%%%%%%%%%%%%%%%%%%%%%%%%%%%%%%
\subsection{Continuity with Domain Transformations}

Theorem~9.5 deals with continuity of changing a function
by shifting its \(\R^1\) domain.
Let's generalize it to other kind of transformations.

\begin{llem} \label{lem:9.5:gen}
Let \((X,d,\mu)\) be a metric and positive measurable space such
that \(\mu(A)<\infty\) whenever \(A\subset X\) is compact
and let \(f\in L^p(X)\).
For each \(\epsilon>0\) there exists \(\delta > 0\) such that if 
\(T:\,X\to X\) is continuous and \(d(T(x),x)<\delta\) for all \(x\in X\)
then \(\|f - f\circ T\|_p < \epsilon\).
\end{llem}
\begin{thmproof}
By theorem~3.14 there exists \(g\in C_c(X)\) such that 
\(\|f-g\|_p < \epsilon/3\). Put \(K=\supp g\).
Let \(\delta>0\)
be such that \(|g(x_1)-g(x_2)|<\epsilon/(3\mu(K))\) 
whenever \(|x_1-x_2|<\delta\).
\end{thmproof}


%%%%%%%%%%%%%%%%%%%%%%%%%%%%%%%%%%%%%%%%%%%%%%%%%%%%%%%%%%%%%%%%%%%%%%%%
\subsection{Inversion --- 4 Steps to Identity}

The Inversion~Theorem~9.11 actually goes only half way.
The following lemma completes it.

\begin{llem}
Denote \(\calF(f) = \hat{f}\).
If \(f\in L^2\) then
\begin{equation} \label{eq:F4}
\calF^4(f) = \calF(\calF(\calF(\calF(f)))) = f \;\aded
\end{equation}
\end{llem}
Notes
\begin{itemize}
\item This can be summarized by \(\calF^4 = \Id\).
\item By Theorem~9.14, if \(\hat{f}\in L^1\) then
\begin{equation*}
f(x) = \int_{\infty}^\infty \hat{f}(t) e^{ixt}\,dm(t)\;\aded
\end{equation*}
\end{itemize}
\begin{thmproof}
By Plancherel Theorem~9.13\ich{d}
\begin{equation*}
f(x) = \lim_{A\to\infty}\int_{-A}^A \hat{f}(t) e^{ixt}\,dm(t) \;\aded
\end{equation*}
Therefore, with
accepting the limit as the defined of the Fourier transform in \(L^2\) 
\begin{equation*}
f(-x) = \lim_{A\to\infty}\int_{-A}^A \hat{f}(t) e^{-ixt}\,dm(t) = \calF^2(f)(x)
\;\aded
\end{equation*}
Applying this twice gives the desired \eqref{eq:F4}.
\end{thmproof}

Intuitively $f$ vanishes at infinitely if \(f\in L^1\),
but it is not difficult to construct a counterexample.
The intuition is true with additional requirement.

\begin{llem} \label{lem:fdf-L1-then-f-inC0}
If \(f\in L^1(\R)\)  and differentable and \(f'\in L^1\) then
\(f\in C_0\).
\end{llem}
\begin{thmproof}
Let 
\(l = \varliminf_{x\to\infty}|f(x)|\)
and
\(h = \varlimsup_{x\to\infty}|f(x)|\).
We need to show that \(l=h=0\).
Similarly we can do the same with 
opposite direction \(\lim_{x\to-\infty}\) limits 
and consequently \(f\in C_0\).

First we will show that \(l=0\). If by negation \(l>0\), 
then \(|f(x)|\geq l\) for all \(x\geq M\) for some \(M<\infty\)
and thus 
\begin{equation*}
\int_{-\infty}^\infty |f(x)|\,dm(x)
\geq \int_M^\infty|f(x)|\,dm(x) 
\geq \int_M^\infty l\,dm(x) = \infty.
\end{equation*}
A contradiction to \(f\in L^1\). 
Hence \(\varliminf_{x\to\pm\infty}|f(x)| = 0\).

Now assume by negation \(h>0\).
Since \(f'\in L^1\) there exist some \(M<\infty\) such that
\begin{equation} \label{eq:intfx-lt4}
\int_M^\infty |f(x)|\,dm(x) < h/4.
\end{equation}
But we can find some \(x_l,x_h>M\) such that
\begin{align*}
|f'(x_l)| &< h/3 \\
|f'(x_h)| &> 2h/3
\end{align*}
and \wlogy\ assume that \(x_l < x_h\). Now
\begin{equation*}
\int_M^\infty |f(x)|\,dm(x) 
\geq \int_{x_l}^{x_h} |f'(x)|\,dm(x) 
\geq \left|\int_{x_l}^{x_h} f'(x)\,dm(x)\right|
= f(x_h) - f(x_l)
> h/3
\end{equation*}
which contradicts \eqref{eq:intfx-lt4}.
\end{thmproof}

%%%%%%%%%%%%%%%%%%%%%%%%%%%%%%%%%%%%%%%%%%%%%%%%%%%%%%%%%%%%%%%%%%%%%%%%
\subsection{Back from Derivative to Multiplication}

The following lemma is an inverse variant of Theorem~9.2\ich{f},
that we have generalized before (\ref{sec:gen:thm9.2}).
\begin{llem} \label{lem:fourierdif}
Let $f$ be a differentable function on \(\R\).
Put \(g(x)=f'(x)\).
If \(f,g\in L^1\) then
\begin{equation} \label{eq:fourierdif}
\hat{g}(t) = it\hat{f}(t) \qquad (t\in\R).
\end{equation}
\end{llem}
\begin{thmproof}
By local lemma~\ref{lem:fdf-L1-then-f-inC0} \(f\in C_0\).
Integrating by parts
\begin{equation*}
\hat{g}(t) 
= \int_{-\infty}^\infty f'(x)e^{-ixt}\,dm(x)
= \frac{1}{\sqrt{2\pi}}f(x)e^{-ixt}\bigm|_{-\infty}^\infty 
  - (-it)\int_{-\infty}^\infty f(x)e^{-ixt}\,dm(x)
= 0 - 0 + it\hat{f}(t).
\end{equation*}
\end{thmproof}

Let's extend this formula
\begin{llem} \label{lem:fourierdifn}
Let $f$ be an $n$-times differentable function on \(\R\).
Put \(g(x)=D^n(f)(x)\).
If \(D^k(f)\in L^1\) for \(0\leq k \leq n\) then
\begin{equation} \label{eq:Fourierdifn}
\hat{g}(t) = \widehat{D^n(f)}(t) = (it)^n\hat{f}(t) \qquad (t\in\R).
\end{equation}
\end{llem}
\begin{thmproof}
The previous locall lemma \ref{lem:fourierdif}
establishes \eqref{eq:Fourierdifn} for \(n=1\).
Assume by induction that \eqref{eq:Fourierdifn} holds for \(1\leq n < k\).
Now
\begin{equation*}
\widehat{D^k(f)}(t) 
= \widehat{D(D^{k-1}(f))}(t) 
= it \widehat{D^{k-1}(f)}(t) 
= it \cdot (it)^{k-1}\hat{f}(t)
= (it)^k\hat{f}(t).
\end{equation*}
\end{thmproof}

%%%%%%%%%%%%%%%%%%%%%%%%%%%%%%%%%%%%%%%%%%%%%%%%%%%%%%%%%%%%%%%%%%%%%%%%
%%%%%%%%%%%%%%%%%%%%%%%%%%%%%%%%%%%%%%%%%%%%%%%%%%%%%%%%%%%%%%%%%%%%%%%%
\section{Exercises} % pages 193-195

%%%%%%%%%%%%%%%%%%%%%%%%%%%%%%%%%%%%%%%%%%%%%%%%%%%%%%%%%%%%%%%%%%%%%%%%
\section{Local Lemmas} 

\begin{llem}
If \(m\in\Z\) and
\begin{equation*}
  f(x) = x^m e^{-x^2}
\end{equation*}
then
\begin{equation}
  \lim_{x\to\infty} f(x) = 0. \label{eq:lim:xm:ex2:eq0}
\end{equation}
\end{llem}
\begin{thmproof}
Since \(f(-x) = \pm f(x)\) we can restrict our attention to \(0<x\to+\infty\).
By induction on~$m$. Clearly \eqref{eq:lim:xm:ex2:eq0} holds for \(m \leq 0\).
Assume it holds for all \(m < k\).
Now by L'Hospital rules and induction assumption
\begin{equation*}
\lim_{x\to\infty} f(x) 
= \lim_{x\to\infty} \frac{mx^{m-1}}{2x e^{x^2}}
= (m/2) \lim_{x\to\infty} x^{m-2} e^{-x^2} = 0.
\end{equation*}
\end{thmproof}


\begin{llem} \label{llem:9.calF}
Let \calF\ be a family of functions of the following form
\begin{equation*}
f(x) = \sum_{j=1}^n a_jx^{m_j}e^{-x^2}
\end{equation*}
where \(a_j\in\C\) and \(m\in\Z^+\).
If \(f \in \calF\) then \(f'\in\calF\).
\end{llem}
\begin{thmproof}
Differentiate \(f(x) = x^me^{-x^2}\) gives
\begin{equation*}
f'(x) = mx^{m-1} e^{-x^2} -2x^{m+1}e^{-x^2} \in \calF.
\end{equation*}
Since differentiation is linear and \calF\ is a linear space,
and since \(f'\in \calF\) for a $f$ in a base, \(f'\in \calF\) also for
all \(f\in \calF\).
\end{thmproof}


\begin{llem} \label{lem:fxfL1}
Assume $f$ is measurable and \(x^nf(x)\) are bounded for \(n=1,2\).
Then \(f\in L^1\).
\end{llem}
\begin{thmproof}
For \(n=0,1,2\) let \(A_{1n}\) be such that 
\(|x^nf(x)| \leq A_{1n}\) (See Exercise~9.\ref{ex:Amn})
\begin{equation*}
\|f\|_1 
= \int |f|
= \int_{-1}^1 |f(t)|\,dt + \int_{|t|>1} |f(t)|\,dt
\leq 2A_{10} + \int_{|t|>1} A_{12}t^{-2}\,dt
= 2(A_{10} + A_{12}).
\end{equation*}
Hence \(f\in L^1\).


\begin{llem} \label{lem:fxfL2}
Assume $f$ is measurable and \(x^nf(x)\) are bounded for \(n=0,1\).
Then \(f\in L^1\).
\end{llem}
Let
\begin{align*}
L &= \{x\in\R\setminus[-1,1]: |f(x)| < 1\} \\
H &= \{x\in\R\setminus[-1,1]: |f(x)| \geq 1\}
\end{align*}
\begin{align*}
\|f\|_2^2 
&= \int_{\R} |f|^2
= \int_{-1}^1 |f|^2 + \int_L |f|^2 + \int_H |f|^2 \\
&\leq 2A_{10}^2 + \int_L |f| + \int_H (A_{11}/t)^2\,dt \\
&\leq 2A_{10}^2 + \|f\|_1 + 2A_{11}^2.
\end{align*}
Hence \(f\in L^2\).
\end{thmproof}


\begin{llem} \label{lem:fxfL12}
Assume $f$ is measurable and \(x^nf(x)\) are bounded for \(0 \leq n \leq 2\).
Then
 \(f\in L^1\cap L^2\).
and
 \(\hat{f}\in L^2\cap C_0\).
\end{llem}
\begin{thmproof}
Combining the results of previous local lemmas 
\ref{lem:fxfL1} and \ref{lem:fxfL2}
we have \(f\in L^1\cap L^2\).
By 
Theorem~9.6 and
\index{Plancherel}
Plancherel Theorem~9.13 \(\hat{f}\in L^2\cap C_0\).
\end{thmproof}

\paragraph{Derivative of rational polynomial.}

\begin{llem} \label{lem:ratpoly:vansihderivf}
Let \(f(x) = p(x)/q(x)\) where \(p(x)\) and \(q(x)\) are 
real polynomials defined on~\(\R\) and \(q(x)\neq 0\) for all \(x\in\R\)
and \(\deg(p) < \deg(q)\).
Then 
\begin{itemize}
\itemch{a} \(f\in C^\infty(\R)\).

\itemch{b} The derivatives are rational polynomials of the form
\begin{equation} \label{eq:ratpoly:vansihderivf:denom}
f^{(n)}(x) = \frac{s_n(x)}{\bigl(q(x)\bigr)^{2^n}}
\end{equation}
for some polynomial \(s_n(x)\) such that 
\(\deg(s_n(x)) < \deg(q(x))^{2^n} = 2^n\deg(q(x))\).


\itemch{c} All derivatives vanish at infinity
\begin{equation} \label{eq:ratpoly:vansihderivf}
\lim_{x\to\pm\infty} f^{(n)}(x) = 0 \qquad (n \in\Z^+).
\end{equation}

\end{itemize}
\end{llem}
\begin{thmproof}
Clearly \(\deg(q) \geq 1\) and \(\lim_{x\to\pm\infty}q(x) = \pm\infty\).
Assume first \(n=0\) 
then \eqref{eq:ratpoly:vansihderivf:denom} is trivial 
and \eqref{eq:ratpoly:vansihderivf}
follows by applying applying L'Hospital rule \(\deg(p)\)-times.

By induction, 
assume \eqref{eq:ratpoly:vansihderivf:denom} 
and \eqref{eq:ratpoly:vansihderivf} holds for \(n=k\).
Now
\begin{equation*}
f^{(k+1)}(x) 
= \left(\frac{s_k(x)}{\bigl(q(x)\bigr)^{2^k}}\right)'
=   \frac{{s_k}'(x)}{\bigl(q(x)\bigr)^{2^k}} 
  - \frac{{s_k}(x) \bigl(q(x)\bigr)^{2^k}}{%
          \left(\bigl(q(x)\bigr)^{2^k}\right)^2} 
= \frac{s_{k+1}(x)}{\bigl(q(x)\bigr)^{2^{k+1}}}
\end{equation*}
where \(s_{k+1}(x) = (q(x))^{2^k}({s_k}'(x) - s_k(x))\).
Clearly
\begin{equation*}
\deg(s_{k+1}(x)) < \deg\left(\bigl(q(x)\bigr)^{2^{k+1}}\right)
\end{equation*}
hence \eqref{eq:ratpoly:vansihderivf:denom} 
and \eqref{eq:ratpoly:vansihderivf} holds for \(n=k+1\) as well.
\end{thmproof}

%%%%%%%%%%%%%%%%%%%%%%%%%%%%%%%%%%%%%%%%%%%%%%%%%%%%%%%%%%%%%%%%%%%%%%%%
\section{The Exercises} % pages 193-195


%%%%%%%%%%%%%%%%%
\begin{enumerate}
%%%%%%%%%%%%%%%%%


%%%%%%%%%%%%%% 1
\begin{excopy}
Suppose \(f \in L^1\), \(f>0\).
Prove that \(|\hat{f}(y)| < \hat{f}(0)\) for every \(y\neq 0\).
\end{excopy}

We have
\begin{equation*}
\hat{f}(0)
= \int_{-\infty}^\infty f(t)e^{-i0t}\,dm(t)
= \int_{-\infty}^\infty f(t)\,dm(t)
= \|f\|_1 \geq 0.
\end{equation*}
Looking at the mutually disjoint \(G_n = f^{-1}\bigl([n-1,n)\bigr)\)
it is easy to see that \(\|f\|_1 > 0\). Actually the same reasoning
shows that
\begin{equation*}
\int_a^b f(t)\,dm(t) > 0
\end{equation*}
Whenever \(a<b\), simply by looking at \(G_n \cap [a,b]\).


For any \(y\in\R\)
\begin{equation*}
|\hat{f}(y)|
= \left| \int_{-\infty}^\infty f(t)e^{-iyt}\,dm(t) \right|
\leq  \int_{-\infty}^\infty |f(t)e^{-iyt}|\,dm(t)
= \int_{-\infty}^\infty f(t)\,dm(t)
= \hat{f}(0).
\end{equation*}
Assume \(y\neq 0\),
we still need to show strict inequality.
Let us have the following simple lemma
\begin{llem} \label{lem:aubuLYab}
If \(a,b>0\) and \(|u_1|=|u_2|=1\) and \(u_1 \neq u_2\),
then
\begin{equation} \label{eq:ex9.1:llem}
a + b > |u_1a + u_2 b|
\end{equation}
\end{llem}
\begin{thmproof}
Since
\begin{equation*}
|u_1a + u_2 b| = |u_1a/u_2 + b|
\end{equation*}
We may assume \(u_2=1\) and \(u = u_1 = e^{i\theta}\)
with \(\theta \neq 0 \mod 2\pi\) that is \(\cos\theta < 1\).
Thus

\begin{equation*}
|ua + b|^2
= (a\sin\theta)^2 + (a\cos\theta) + b)^2
= a^2 + 2ab\cos\theta + b^2
\end{equation*}
\begin{equation*}
(a+b)^2 - |ua + b|^2 = 2ab(1-\cos\theta) > 0.
\end{equation*}
Hence \((a+b)^2 > |ua + b|^2\) which gives \eqref{eq:ex9.1:llem}.
\end{thmproof}

Back to the exercise.
\begin{align*}
\hat{f}(0) - |\hat{f}(y)|
&=
   \int_{-\infty}^\infty f(t)\,dm(t)
   - \left| \int_{-\infty}^\infty f(t)e^{-iyt}\,dm(t) \right| \\
&\geq \int_0^\infty
           \bigl(f(t) + f(-t)\bigr) -
           \left| f(t)e^{-iyt} + f(-t)e^{iyt}\right| \,dm(t)
\end{align*}
By the lemma~\ref{lem:aubuLYab}, the last integrand is positive \aded,
thus \(\hat{f}(0) - |\hat{f}(y)| > 0\)
which gives the desired result.



\iffalse
Pick arbitrary \(\epsilon\in(0,\|f\|_1/2)\).
By Theorem~3.14 \cite{RudinRCA87} there exists
a~continuous function $g$ such that \(\|g-f\|_1 < \epsilon\).
We may assume that \(g\geq 0\), otherwise we pick \(|g|\).
By continuity and choice of \(\epsilon\) there exists \(\xi\)
such that \(g(\xi)>0\) and \(a,\delta>0\) such that
\(g(x)\geq a\) whenever \(x\in[\xi-\delta,\xi+\delta]\).
\fi

%%%%%%%%%%%%%% 2
\begin{excopy}
Compute the Fourier transform of the characteristic function
of an interval. For \(n=1,2,3,\ldots\), let
\(g_n\) be the characteristic if \([-n,n]\),
let $h$ be the characteristic
function of \([-1,1]\) and compute \(g_n \ast h\) explicitly.
(The grpah is piecewise linear).
Show that \(g_n \ast h\) is the Fourier transform
of a~function \(f_n\in L^1\); except for a multiplicative constant,
\begin{equation*}
f_n(x) = \frac{\sin x\,\sin nx}{x^2}
\end{equation*}
Show that \(\|f_n\|_1 \to \infty\) and conclude that the mapping
\(f \to \hat{f}\) maps \(L^1\) into a \emph{proper} subset of \(C_0\).
Show however, that the range of this mapping is dense in \(C_0\).
\end{excopy}

Let \(f=\chhi_{[a,b]}\) then
\begin{align*}
\hat{f}(x)
 &= \int_{-\infty}^\infty f(x)\cdot e^{-ixt}\,dm(t)
  = \int_a^b e^{-ixt}\,dm(t)
  = (e^{-ixb} - e^{-ixa})/\bigl(-ix\sqrt{2\pi}\,\bigr) \\
 &= i(e^{-ixb} - e^{-ixa})/\bigl(\sqrt{2\pi}x\bigr).
\end{align*}
Hence 
\begin{equation*}
\hat{g_n}(x) 
 = i(e^{-inx} - e^{inx})/\bigl(\sqrt{2\pi}x\bigr)
 = -i^2((e^{inx} - e^{-inx})/i)\bigl(\sqrt{2\pi}x\bigr)
 = \frac{1}{\sqrt{2\pi}}\sin(nx)/x.
\end{equation*}
Similarly,
\begin{equation} \label{eq:fourier:chi}
\widehat{\chhi_{[-\lambda,\lambda]}}(x) = \frac{1}{\sqrt{2\pi}}\sin(\lambda x)/x
\end{equation}
for any \(\lambda\in\R^{+}\).


Now we compute the convolution.
\begin{align*}
(g_n \ast h)(x)
 &= \int_{\infty}^\infty g_n(x-t)h(t)\,dm(t) \\
 &= \int_{-1}^1 g_n(x-t)\,dm(t)
  = \left\{
    \begin{array}{ll}
    2              & |x| \leq n - 1 \\
    n+1-|x| \qquad & n-1 \leq |x| \leq n+1 \\
    0 \qquad       & |x| \geq n + 1
    \end{array}\right.
\end{align*}

Put \(\varphi_n = g_n \ast h\) and compute:
\begin{align*}
\hat{\varphi_n}(t) 
 &= \int_{-\infty}^\infty \varphi_n(x)e^{-itx}\,dm(x)
 =  \int_{-\infty}^\infty 
         \left(\int_{-\infty}^\infty g_n(x-y)h(y)\,dm(y)\right)
         e^{-itx}\,dm(x) \\
 &= \int_{-\infty}^\infty 
          \left(\int_{-\infty}^\infty g_n(x-y)e^{-it(x-y)}\,dm(x)\right)
         h(y)e^{-ity}\,dm(y) \\
 &= \int_{-\infty}^\infty 
          \left(\int_{-\infty}^\infty g_n(x)e^{-itx}\,dm(x)\right)
         h(y)e^{-ity}\,dm(y) \\
 &= \int_{-\infty}^\infty \hat{g_n}(x) h(y)e^{-ity}\,dm(y) 
  = \hat{g_n}(x)  \int_{-\infty}^\infty h(y)e^{-ity}\,dm(y) \\
 &= \hat{g_n}(x) \hat{h}(x) = \beta_n \frac{\sin(x)\sin(nx)}{x^2}
\end{align*}
for some \(\beta_n\).

By the inversion formula (Theorem~9.11 \cite{RudinRCA87})
and the fact that \(f_n\) is an even function
\begin{align*}
\varphi_n(t) 
&= \int_{\infty}^\infty \hat{\varphi_n}(x) e^{itx}\,dm(x)
 = \int_{\infty}^\infty \hat{g_n}(x)\hat{h}(x) e^{itx}\,dm(x) \\
&= \beta_n \int_{\infty}^\infty f_n(x) e^{itx}\,dm(x) 
 = \beta_n \int_{\infty}^\infty f_n(x) e^{-itx}\,dm(x) \\
&= \beta_n \hat{f_n}(t).
\end{align*}


We will now show
\begin{equation} \label{eq:ex9.1:fn1:inf}
\lim_{n\to\infty}\|f_n\|_1 = \infty.
\end{equation}
If \(0<x\leq\pi/2\) then \(\sin x/x \geq 2/\pi > 1/2\).
\begin{align*}
\|f_n\|_1
&= \int_{-\infty}^\infty \left|\frac{\sin x\,\sin nx}{x^2}\right|\,dm(x)
 \geq \int_0^1 (\sin x)\cdot|\sin nx|/x^2\,dm(x) \\
&\geq \int_0^1 (|\sin nx|/x)(\sin x/x)\,dm(x)
 \geq (1/2)\int_0^1 (|\sin nx|/x)\,dm(x)
\end{align*}
We will soon estimate the last integral.
The function \(\sin nx\) has \(\lfloor n/2\pi \rfloor\)
complete periods within \([0,1]\).
Since
\begin{equation*}
\sin \pi/3 = \sin 2\pi/3 = -\sin 4\pi/3 = -\sin 5\pi/3 = 1/2,
\end{equation*}
in each period $w$,
the lengths total of the two sub-intervals where \(|\sin nx| \geq 1/2\)
is \(w/3\). The total lengths of these sub-intervals in \([0,1]\)
is at least
\begin{equation*}
(2\pi/n) \lfloor n/2\pi \rfloor / 3 \geq 1/4
\end{equation*}
for \(n>2\pi\).
Hence
\begin{align*}
\int_0^1 (|\sin nx|/x)\,dm(x)
&\geq \sum_{k=0}^{\lfloor n/2\pi \rfloor}
  \int_{2\pi k/n}^{2\pi(k+1)/n} |\sin nx|/x\,dm(x) \\
&\geq (1/4)\sum_{k=1}^{\lfloor n/2\pi \rfloor}
         (1/2)\cdot (1/k)
 \geq (1/8)\sum_{k=1}^{\lfloor n/7\rfloor} 1/k
\end{align*}
which clearly converges to \(+\infty\) as \(n\to\infty\)
and \eqref{eq:ex9.1:fn1:inf} is shown.


By Theorem~9.6 (\cite{RudinRCA87}) the Fourier transform 
is a continuous mapping of \(L^1\) to \(C_0\).
If by negation (similar to Theorem~5.15 \cite{RudinRCA87})
the mapping were \emph{onto} then by Theorem~5.10 (\cite{RudinRCA87})
there would exist \(\delta>0\) such that 
\begin{equation*}
 \delta \|f_n\|_1 \leq \|\hat{f_n}\|_\infty = 1
\end{equation*}
for all $n$, which is a contradiction to the shown \(\lim_{n\to\infty}\|f_n\|_1 = \infty\).

%%%%%%%%%%%%%% 3
\begin{excopy}
Find
\begin{equation*}
\lim_{A\to\infty} \int_{-A}^A \frac{\sin \lambda t}{t}e^{itx}\,dt
  \qquad (\infty < x < \infty)
\end{equation*}
where \(\lambda\) is a positive constant.
\end{excopy}

Using \eqref{eq:fourier:chi} of previous exercise, 
and Theorem~9.13\ich{d} (\cite{RudinRCA87})
\begin{equation*}
\lim_{A\to\infty} \int_{-A}^A \frac{\sin \lambda t}{t}e^{itx}\,dt
= \sqrt{2\pi} \chhi_{[-\lambda,\lambda]}.
\end{equation*}

%%%%%%%%%%%%%% 4
\begin{excopy}
Give examples of \(f\in L^2\) such that \(f\notin L^1\)
but \(\hat{f}\in L^1\). Under what circumstances can this happen?
\end{excopy}

The function in previous exercise, namely \(f(x) = \sin(\lambda t)/t\) 
is an example. 
The question is --- for which \(f\in L^2\setminus L^1\)
we have \(\hat{f}\in L^1\)\,?

%%%%%%%%%%%%%% 5
\begin{excopy}
If \(f\in L^1\) and \(\int|t\hat{f}(t)|\,dm(t) < \infty\),
prove that $f$ coincides \aded\ with a differentiable function
whose derivative is
\begin{equation*}
i \int_{-\infty}^\infty t\hat{f}(t)e^{ixt}\,dm(t).
\end{equation*}
\end{excopy}

By Theorem~9.6 \(\hat{f}\in L^1\), hence
\begin{equation*}
\int_{-\infty}^\infty |\hat{f}(t)|\,dm(t)
\leq \int_{-1}^1 |\hat{f}(t)|\,dm(t) 
     + \int_{\R\setminus[-1,1]} |t\hat{f}(t)|\,dm(t)
< \infty.
\end{equation*}
and \(\hat{f}\in L^1\).

By the inversion Theorem~9.11, we can define
\begin{equation*}
g(x) = \int_{-\infty}^\infty \hat{f}(t)e^{ixt}\,dm(t)
\end{equation*}
and \(f(x)=g(x)\,\aded\)

Differentiate
\begin{align}
g'(x) 
&= \lim_{h\to 0} (g(x+h)-g(x))/h \notag \\
&= \lim_{h\to 0} 
    \left(
     \int_{-\infty}^\infty \hat{f}(t)(e^{i(x+h)t} - e^{ixt})\,dm(t)
    \right) \,\bigm/\, h \notag \\
&= \lim_{h\to 0} 
    \left(
     \int_{-\infty}^\infty \hat{f}(t)(e^{i(x+h)t} - e^{ixt})/h\,dm(t)
    \right) \notag \\
&= 
     \int_{-\infty}^\infty \hat{f}(t)
    \left(
          \lim_{h\to 0} (e^{i(x+h)t} - e^{ixt})/h
    \right) 
    \,dm(t)    \label{eq:ex9.5:lbgdom} \\
&= i\,\int_{-\infty}^\infty t\hat{f}(t)e^{ixt}\,dm(t) \notag
\end{align}

The \eqref{eq:ex9.5:lbgdom} equality is justified by the same argument
of section~9.3(\emph{a}) and the fact that for sufficiently small $h$
we have \(|(e^{i(x+h)t} - e^{ixt})/h|<2\) together with \(\hat{f}\in L^1\).

%%%%%%%%%%%%%% 6
\begin{excopy}
Suppose  \(f\in L^1\), $f$ is differentiable almost everywhere,
and \(f'\in L^1\). Does it follow that the Fourier transform
of \(f'\) is \(ti\hat{f}(t)\)?
\end{excopy}

Consider
\begin{equation*}
f(x) = \left\{
  \begin{array}{ll}
  e^{-x} & x \geq 0\\
  0     & x < 0
  \end{array}
  \right.
\end{equation*}
and \(f'(x) = -f(x)\,\aded\) Now
\begin{equation*}
\hat{f}(t) 
 = \int_0^\infty e^{-x}e^{-ixt}\,dm(x)
 = \int_0^\infty e^{-x(1+it)}\,dm(x)
 = \frac{-1}{1+it}\,e^{-x(1+it)}\bigm|_0^\infty
 = 1/(1+it)
\end{equation*}
Clearly the conjecture here fails.


%%%%%%%%%%%%%% 7
\begin{excopy} 
Let 
\label{ex:Amn}
$S$ be the class of all functions $f$ on \(\R^1\) which have
the following property:
$f$ is infinitely differentiable,
and there are numbers \(A_{m n}(f)<\infty\),
for $m$ and \(n=0,1,2,\ldots\), such that
\begin{equation*}
\left| x^nD^m f(x)\right| \leq A_{m n}(f) \qquad (x\in\R^1).
\end{equation*}
Here $D$ is the ordinary differentiable operator.

Prove that the Fourier transform maps $S$ onto $S$.
Find examples of members of $S$.
\end{excopy}

\textbf{Note:} This $S$ space is called
\index{Schwartz space}
Schwartz space in other texts.

Assume \(f\in S\).
In order to show that \(\hat{f}\in S\) we need to show
\(\hat{f}\) is infinitely differentiable and 
that \(A_{mn}(\hat{f})<0\) exist for all \(m,n\in\Z^+\).

\begin{equation*}
\hat{f}(t) = \int_{\R} f(x)e^{-ixt}\,dm(x)
\end{equation*}
Now
\begin{align*}
|\hat{f}(t)| 
 &\leq \int_{\R} |f(x)|\,dm(x)
  = \int_{-1}^1 |f(x)|\,dm(x) + \int_{|x|>1} |f(x)|\,dm(x) \\
 & \leq \int_{-1}^1 A_{00}(f)\,dm(x) + 2\int_1^\infty A_{02}(f)x^{-2}\,dm(x)
  = 2(A_{00}(f) + A_{02}(f)) /\sqrt{2\pi}.
\end{align*}
Hence 
\begin{equation} \label{eq:A00}
A_{00}(\hat{f}) \leq 2(A_{00}(f) + A_{02}(f)) /\sqrt{2\pi}\,.
\end{equation}

If \(g(x) = x^nf(x)\) then clearly \(g\in S\).
By local lemma~\ref{lem:g-eq-xnf}
\(\hat{g}(t) = i^nD^n\hat{f}(t)\) and so by applying the preceding result
\begin{equation*}
|D^n\hat{f}(x)| = |\hat{g}(t)| \leq 2(A_{00}(g) + A_{02}(g)) /\sqrt{2\pi}.
\end{equation*}


If \(g(x) = (D^nf)(x)\) then clearly \(g\in S\).
By local lemma~\ref{lem:fourierdifn}
\(\hat{g}(t) = (it)^n\hat{f}(t)\).

Let \(f\in S\).
By local lemma~\ref{lem:g-eq-xnf}, \(\hat{f}\) is infinitely differentable.
Let \(g(t) = t^m(D^n(\hat{f}))(t)\).
Put \(g_d(t) = (D^n(\hat{f}))(t)\)
and \(h(x) = x^nf(x) \in S\)
By the same local lemma~\ref{lem:g-eq-xnf}
\begin{equation*}
  D^n\hat{f}(t) = (-i)^n\hat{h}(t)
\end{equation*}
Applying \eqref{eq:A00} we have 
\begin{equation*}
A_{n0}(\hat{g}) \leq A_{00}(\hat{h}) < \infty
\end{equation*}

By local lemma~\ref{lem:fourierdifn} applied to \(t^mg_d(t)\) we get
\begin{equation*}
t^m(D^n(\hat{f}))(t) = (-i)^m\widehat{D^m(h)}(t)
\end{equation*}
It is easy to see that \(D^m(h)\in S\) since $S$ is closed under addition.
Thus 
\begin{equation*}
A_{nm}(\hat{f}) = A_{n0}(\widehat{D^mh} < 0
\end{equation*}
and therefore \(\hat{f}\in S\).

\iffalse
If \(f\in C^1(\R)\) and its derivative is bounded then 
for each \(\epsilon>0\) 
we have \(|f(s)-f(t)|<\epsilon\) whenever 
\(|s-t|<\delta=\epsilon / (\|f'\|_\infty+1)\), 
that is $f$ is uniformly continuous.
Hence any member of $S$ is uniformly continuous.

Also if \(f\in S\) and \(m\in\{0,1\}\) then
\begin{equation*}
\int |f| = \int x^2|f|/x^2 \leq A_{02}x^{-2} < \infty
\end{equation*}
and so \(S\subset L^1(\R)\).
Almost directly by definition, if \(f\in S\) and \(g(x) = x^kf(x)\) 
for some \(k\geq 0\)
then \(g\in S\) as well.

Let \(f\in S\), and let \(g(x) = ixf(x)\).
With the above and theorem~9.2\ich{f} \({\hat{f}}' = \hat{g}\).
\fi % false

As an example,
members of \calF\ defined in local lemma~\ref{llem:9.calF} are in $S$.


%%%%%%%%%%%%%% 8
\begin{excopy}
If $p$ and $q$ are conjugate exponents, 
\(f\in L^p\), \(g\in L^q\) and \(h = f\ast g\),
prove that $h$ is uniformly continuous. If also \(1<p<\infty\), then
\(h\in C_0\): show that this fails for some \(f\in L^1\), \(g\in L^\infty\).
\end{excopy}

Along the lines of the proof of Theorem~9.5.\\
Fix \(\epsilon > 0\). There is a continuous \(\phi\) (Theorem~3.14) such that
\begin{align*}
\supp(\phi) &\subset [-A,A] \\
\|f-\phi\|_p &< \epsilon\,.
\end{align*}
\Wlogy, we can enlarge $A$ such that 
\begin{equation*}
\|f\chhi_{\R\setminus[-A,A]}\|_p
= \left(\int_{\R\setminus[-A,A]}\|_p |f(x)|^p\,dm(x)\right)^{1/p} < \epsilon.
\end{equation*}
By uniform continuity of \(\phi\) there exists \(\delta\in (0, A)\)
such that 
\begin{equation*}
 |\phi(s) - \phi(t)| < A^{-1/p}\epsilon\,.
\end{equation*}
We set 
\(f_a(x) = f(a-x)\) and
\(\phi_a(x) = \phi(a-x)\).
Now
\begin{align*}
|h(s)-h(t)|
&= |(f\ast g)(s) - (f\ast g)(t)| \\
&= \left|\int f(s-x)g(x)\,dm(x) - \int (f(t-x)g(x)\,dm(x) \right| \\
&= \left|\int (f(s-x) - f(t-x))g(x)\,dm(x) \right| \\
&\leq \int |(f(s-x) - f(t-x))g(x)|\,dm(x)  \\
&= \int \left|f(s-x) - f(t-x)\right|\cdot|g(x)|\,dm(x) \\
&\leq \|f_s - f_t\|_p \cdot \|g\|_q \\
&\leq \left(
           \|f_s - \phi_s\|_p + \|\phi_s - \phi_t\|_p +\|\phi_t - f_t\|_p 
     \right) \cdot \|g\|_q \\
&= \left(2\|f - \phi\|_p + \|\phi_s - \phi_t\|_p \right) \cdot \|g\|_q \\
&\leq (2\epsilon + (2A (A^{-1/p}\epsilon)^p)^{1/p}) \cdot \|g\|_q \\
&= 4\epsilon \|g\|_q\,.
\end{align*}
Therefore $h$ is uniformly continuous.

As a counterexample, let \(f = \chhi_{[0,1]} \in L^1\)
and \(g = 1 \in L^\infty\). Then also \(h = f \ast g = 1 \notin C_0\).


%%%%%%%%%%%%%% 9
\begin{excopy}
Suppose \(1\leq p < \infty\), \(f\in L^p\), and
\begin{equation*}
g(x) = \int_x^{x+1} f(t)\,dt.
\end{equation*}
Prove that \(g\in C_0\). What can you say about $g$ if \(f\in L^\infty\).
\end{excopy}

Let \(f\in L^p\). Pick some \(\epsilon>0\), then there exist some \(M<\infty\)
such that \(\int_{|t|>M} |f(t)|\,dt < \epsilon\).
Obviously 
\begin{equation*}
\left|\int_x^{x+1}f(t)\,dt\right|
\leq \int_x^{x+1}|f(t)|\,dt
\leq \int_{|t|>M+1}|f(t)|\,dt < \infty.
\end{equation*}
Hence \(g\in C_0\).

If \(f\in L^\infty\) then $g$ is not necessarily in \(C_0\).
For example, if \(f=1\) then \(g=f\in L^\infty \setminus C_0\).


%%%%%%%%%%%%%% 10
\begin{excopy}
Let \(C^\infty\) be the class of all infinitely differentiable complex functions
on \(\R^1\), and let \(C_c^\infty\) consist of all \(g\in C^\infty\) 
whose supports are compact.
Show that  \(C_c^\infty\) does not consist of $0$ alone.

Let \(L_{\textrm{loc}}^1\) be the class 
of all $f$ which belong to \(L^1\) locally;
that is, \(f\in L_{\textrm{loc}}^1\) provided that $f$ is measurable
and \(\int_I |f|<\infty\) for every bounded interval $I$.

If \(f\in L_{\textrm{loc}}^1\) and \(g\in C_c^\infty\), prove that 
\(f\ast g \in C^\infty\).

Prove that there are sequences \(\{g_n\}\) in \(C_c^\infty\), such that
\begin{equation*}
\|f\ast g_n - f\|_1 \to 0
\end{equation*}
as \(n\to \infty\), for every \(f\in L^1\).
(Compare Theorem~9.10.)
Prove that  \(\{g_n\}\) can also be so chosen that 
\((f\ast g_n)(x) \to f(x)\,\aded\), for every \(f\in L_{\textrm{loc}}^1\);
in fact for suitable \(\{g_n\}\) the convergence occurs at every point
$x$ at which $f$ is derivative of its indefinite integral.

Prove that \((f \ast h_\lambda)(x) \to f(x) \,\aded\) if \(f\in L^1\), 
as \(\lambda\to 0\), and that \(f\ast h_\lambda \in C^\infty\),
although \(h_\lambda\) does not have compact support.
(\(h_\lambda\) is defined in Sec~9.7.)
\end{excopy} 

\paragraph{Example of non trivial \(C_c^\infty\) function.}
Define
\begin{equation*}
v(x) = \left\{
\begin{array}{ll}
e^{-1/x^2}e^{-1/(x-1)^2} \qquad &x\in(0,1) \\
0 & x \notin (0,1)
\end{array}
\right.
\end{equation*}
It can be shown that \(D^n(v)(0) = D^n(v)(1) = 0\) and so 
\(0\neq v \in C_c^\infty\).

\paragraph{Convolution resulting in \(\C^\infty\).}
Assume \(f\in L_{\textrm{loc}}^1\) and \(g\in C_c^\infty\)
and put \(h = f\ast g\). We may assume that \(\supp g \subset [-A,A]\)
for some \(A < \infty\).
By definition, local lemma~\ref{lem:conv:commut}
Using the substitution \(t=x-y\) we have
\begin{equation*}
h(x) = \int_{-A}^A f(x-y)g(y)\,dm(y) = \int_{-A-x}^{A-x} g(x-y)f(y)\,dm(y)
\end{equation*}
Note that the fact that the support of $g$ is bounded
ensures that the integrals above are finite because 
$f$ is locally in \(L^1\).

The difference ratio of $g$ is bounded since 
\begin{equation} \label{eq:gCcdinf:difrat}
\left|\frac{g(x-y)-g(x+s-y)}{s}\right| \leq \|g'\|_\infty < \infty.
\end{equation}
The last inequality holds becuase \(g'\in C_c^\infty\).

The difference ratio, for \(s>0\)
\begin{align*}
d(x,s) 
&= (h(x+s)-h(x))/s \\
&= \frac{1}{s}\int_{-A-x-s}^{A-x+s} (g(x-y)-g(x+s-y))f(y)\,dm(y) \\
&= \int_{-A-x-s}^{A-x+s} \frac{g(x-y)-g(x+s-y)}{s}f(y)\,dm(y) 
\end{align*}
The inequality \eqref{eq:gCcdinf:difrat} 
together with the discussion in section~9.3\ich{a}
allow us to use Lebesgue's dominated convergence theorem~1.34.
\begin{align*}
h'(x)
&= \lim_{s\to 0}d(x,s) \\
&= \lim_{s\to 0} \frac{1}{s}\int_{-A-x-s}^{A-x+s} (g(x-y)-g(x+s-y))f(y)\,dm(y) \\
&= \int_{-A-x-s}^{A-x+s} \lim_{s\to 0} \frac{g(x-y)-g(x+s-y)}{s}f(y)\,dm(y) \\
&= \int_{-A-x-s}^{A-x+s} \lim_{s\to 0} \frac{g(x-y)-g(x+s-y)}{s}f(y)\,dm(y) \\
&= \int_{-\infty}^\infty g'(x-y)f(y)\,dm(y) \\
&= (f \ast g')(x).
\end{align*}
Therefore, \(f \ast g\) is differentiable, and 
since \(g'\in C_c^\infty\), by induction  
\(f \ast D^kg\) is differentiable
and \(D(f \ast D^kg) = D^{k+1}g\) for all \(k\in\Z^+\) and so 
\(f \ast g \in C^\infty\).


\paragraph{Convolution approximation to identity in \(L^1\).}
Define
\begin{equation}
g_n(x) = \left\{%
\begin{array}{ll}
a_n\exp(-1/(x+1/n)^2)\exp(-1/(x-1/n)^2) \qquad & x\in (-1/n,1/n) \\
0                                       \qquad & x\notin (-1/n,1/n)
\end{array}\right.
\end{equation}
where \(a_n\) is defined such that \(\|g_n\|_1 = 1\).
We will now show that 
\begin{equation} \label{eq:limCc-ast-gn}
\lim_{n\to\infty} \|\varphi\ast g_n - \varphi\|_1 = 0
\qquad \forall \varphi \in C_c(\R)
\end{equation}

% Let \(f\in L^1(\R)\) and 
Let \(\varphi\in C_c(\R)\) and pick some \(\epsilon > 0\).
Since its support is bounded, \(\varphi\) is \emph{uniformly} continuous.
Thus we can find some \(m<\infty\) such that
\(|\varphi(t) - \varphi(s)|<\epsilon/A\) whenever \(|t-s|<2/m\).
For any \(n>m\)
\begin{align*}
\|\varphi \ast g_n - \varphi\|_1
&= \left| \int_{-\infty}^\infty
        \left( \int_{-\infty}^\infty 
          \varphi(x-y) \cdot g_n(y)\,dm(y) - \varphi(x)
        \right)\,dm(x) \right| \\
&\leq \int_{-\infty}^\infty
        \left| \int_{-\infty}^\infty 
          \varphi(x-y) \cdot g_n(y)\,dm(y) - \varphi(x)
        \right|\,dm(x) \\
&=    \int_{-\infty}^\infty
        \left| \int_{-\infty}^\infty 
          \bigl(\varphi(x-y) - \varphi(x)\bigr)\cdot g_n(y)\,dm(y)
        \right|\,dm(x) \\
&=    \int_{-\infty}^\infty
        \left| \int_{-1/n}^{1/n}
          \bigl(\varphi(x-y) - \varphi(x)\bigr)\cdot g_n(y)\,dm(y)
        \right|\,dm(x) \\
&\leq  \int_{-\infty}^\infty
         \left( \int_{-1/n}^{1/n}
          \bigl|\varphi(x-y) - \varphi(x)\bigr|\cdot g_n(y)\,dm(y)
         \right)\,dm(x) \\
&=    \int_{-1/n}^{1/n}
         \left( \int_{-\infty}^\infty
          \bigl|\varphi(x-y) - \varphi(x)\bigr|\cdot g_n(y)\,dm(x)
         \right)\,dm(y) \\
&=    \int_{-1/n}^{1/n}
         \left( \int_{-A}^{A+y}
          \bigl|\varphi(x-y) - \varphi(x)\bigr|\cdot g_n(y)\,dm(x)
         \right)\,dm(y) \\
&\leq  \int_{-1/n}^{1/n}
         \left( \int_{-A}^{A+y}
          (\epsilon/A)\cdot g_n(y)\,dm(x)
         \right)\,dm(y) \\
&\leq    (\epsilon/A) \int_{-1/n}^{1/n}(A+y)g_n(y)\,dm(y) \\
&=    (\epsilon/A)\left((2A/n) + \int_{-1/n}^{1/n} y g_n(y)\,dm(y)\right) \\
&\leq \epsilon(2/n + 1).
\end{align*}
Thus the claim \eqref{eq:limCc-ast-gn} holds.

By theorem~3.14 we can find \(\varphi\in C_c(\R)\)
such that
\(\|f-\varphi\|_1 < \epsilon\).
\iffalse
Find \(0<A<\infty\) such that \(\int_{|x|>A}|f(x)|\,dx < \epsilon\)
and \(supp \varphi \subset [-A,A]\).
By Lusin's theorem~2.24 we can approximate \(f\cdot\chhi_{[-A,A]}\)
with a function \(\varphi\in C_c(\R)\) such that
\begin{equation*}
m\left(\{x\in\R: f(x) \neq \varphi(x)\}\right) < \epsilon\,/
\end{equation*}
\fi

Now
\begin{align*}
\|f\ast g_n - f\|_1
&\leq 
    \|f\ast g_n - \varphi \ast g_n\|_1
  + \|\varphi \ast g_n - \varphi\|_1
  + \|\varphi - f\|_1 \\
&\leq \|(f-\varphi)\ast g_n\|_1 + \|\varphi \ast g_n - \varphi\|_1 + \epsilon \\
&\leq \epsilon\cdot 1 +  \|\varphi \ast g_n - \varphi\|_1 + \epsilon
\end{align*}
With \eqref{eq:limCc-ast-gn}
\begin{equation*}
\lim_{n\to\infty} \|f\ast g_n - f\|_1 = 0
\end{equation*}

\paragraph{Convolution approximation to identity almost everywhere.}
Pick an arbitrary finite interval \(I\subset \R\).
We know that \(\|f\chhi_I\|_1 < \infty\).
Let \(X_n = \{X\in I: n - 1 \leq |f(x)| < n\}\), 
clearly \(I=\disjunion_{n\in\N} X_n\).
Now
\begin{equation*}
\sum_{n\in\N} (n-1)\cdot m(X_n) 
 \leq \|f\cdot\chhi_I\|_1
 < \infty.
\end{equation*}
and the last inequality is because \(f\in L_{\textrm{loc}}^1\).
 % \leq \sum_{n\in\N} n\cdot m(X_n)\,.
We can find some \(N>2\) such that  
\begin{equation*}
\sum_{n>N} m(X_n) < \sum_{n>N} (n-1)\cdot m(X_n) < \epsilon\,.
\end{equation*}
Pick \(\delta = \epsilon/(N \max(m(I),1))\) and now 
\begin{equation*}
\int_E |f(x)|\,dm(x) < 2\epsilon
\end{equation*}
for any \(E\subset I\) such that \(m(E) < \delta\).
\index{Lusin}
By Lusin theorem~2.24, we can find a \(\psi\in C_c(\R)\)
such that  \(|\psi(x)|\leq |f(x)|\) for all \(x\in\R\)
and if \(B = \{x\in\R: f(x)\neq \psi(x)\}\)
then \(m(B) < \epsilon / \|f\chhi_I\|_1\). Now for any \(x\in I \setminus B\)
\begin{align*}
|(f\ast g_n)(x) - f(x)|
&\leq  |(f \ast g_n)(x) - (\psi \ast g_n)(x)|
     + |(\psi \ast g_n)(x) - \psi(x)|
     + |\psi(x) - f(x)| \\
&\leq \left\|f-\psi\right\|_1 \cdot \left\|g_n\right\|_1 
      + |(\psi \ast g_n)(x) - \psi(x)|
      + 0 \\
&= \int_B|f(t) - \psi(t)|\,dm(t) + |(\psi \ast g_n)(x) - \psi(x)| \\
&\leq 2\epsilon + |(\psi \ast g_n)(x) - \psi(x)|.
\end{align*}
Using \eqref{eq:limCc-ast-gn} again, shows that 
\begin{equation*}
\lim_{n\to\infty}|(f\ast g_n)(x) - f(x)| = 0 \qquad \forall x\in I\setminus B.
\end{equation*}
Since \(\epsilon = m(B)\) was arbitrary the limit actually holds
for $x$ almost everywhere in $I$, and since~$I$ was arbitrarily
chosen, consequently the limit holds for almost everywhere in~\(\R\).

\paragraph{Convergence with auxilary function.}
In the proof of the inversion theorem~9.11 it was shown that 
\begin{equation*}
\lim_{n\to\infty}(f\ast h_{\lambda_n})(x) = f(x)\quad\aded
\end{equation*}
for any seqence \(\{\lambda_n\}\) such that 
\(\lim_{n\to\infty} \lambda_n = 0\)
\emph{without} the requirement of \(\hat{f}\in L^1(\R)\).
Again with the argument of section~9.3\ich{a}
\begin{equation*}
\lim_{\lambda\to 0}(f\ast h_\lambda)(x) = f(x)\quad\aded
\end{equation*}

\paragraph{Convolution result in \(C^\infty\)}
Let \(f\in L^1(\R)\) and let \(g(x) = (f\ast h_\lambda)(x)\). Now
\begin{align*}
\bigl(g(x+s)-g(x)\bigr) \bigm/ s
&= \frac{1}{s} 
   \int_{-\infty}^\infty \bigl(f(x-y) - f(x+s-y)\bigr)h_\lambda(y)\,dm(s) \\
&= \frac{1}{s}\left(
   \int_{-\infty}^\infty f(x-y) h_\lambda(y)\,dm(s) -
   \int_{-\infty}^\infty f(x-y)h_\lambda(y-s)\,dm(s)
   \right) \\
&= -\int_{-\infty}^\infty f(x-y)\frac{h_\lambda(s) - h_\lambda(y-s)}{s}\,dm(s)
\end{align*}
The limit exists
\begin{equation*}
g'(x) = -(f \ast {h_\lambda}')(x)\,.
\end{equation*}
Since \(h_\lambda\) satisfies the condition 
of local lemma~\ref{lem:ratpoly:vansihderivf} we have
the derivatives \({h_\lambda}^{(n)}\) bounded and vanish at infinity
for all $n$. Thus we can reapply the above manipulation, to show that 
\begin{equation*}
(f\ast h_\lambda)^{(n)} = (f\ast (h_\lambda)^{(n)})
\end{equation*}
in particular \(f\ast h_\lambda \in C^\infty(\R)\).

%%%%%%%%%%%%%% 11
\begin{excopy}
Find conditions of $f$ and/or \(\widehat{f}\) which ensure the correctness of 
the following formal argument: If
\begin{equation*}
 \varphi(t) = \frac{1}{2\pi} \int_{-\infty}^\infty f(x)e^{-itx}\,dx
\end{equation*}
and
\begin{equation*}
 F(x) = \sum_{k = -\infty}^\infty f(x + 2k\pi)
\end{equation*}
then $F$ is periodic with period \(2\pi\), the $n$th Fourier coefficient of $F$
is \(\varphi(n)\), hence
\(F(x) = \sum \varphi(n)e^{inx}\). In particular,
\begin{equation*}
\sum_{k = -\infty}^\infty f(2k\pi) = \sum_{n = -\infty}^\infty \varphi(n).
\end{equation*}

More generally
\begin{equation} \label{eq:ex9.11}
\sum_{k = -\infty}^\infty f(k\beta) 
= \alpha \sum_{n = -\infty}^\infty \varphi(n\alpha).
\qquad \textnormal{if}\; \alpha>0, \beta>0,\, \alpha\beta = 2\pi.
\end{equation}
What does \eqref{eq:ex9.11} say about the limit, as \(\alpha\to 0\),
of the right-hand side 
(for ``nice'' functions, of course)?
Is this in agreement with the inversion theorem?
\\
\index{Poisson summation formula}
[\eqref{eq:ex9.11} is known as the Poisson summation formula.]
\end{excopy}

(See also Watkins (quoting Bump) \cite{Watkins:fnleqn}.)

Note the use of different factors, and thus
\begin{equation*}
\varphi(t) = \frac{1}{\sqrt{2\pi}} \widehat{f}(t).
\end{equation*}
% http://www.secamlocal.ex.ac.uk/people/staff/mrwatkin/zeta/bump-fnleqn.ps


We assume that $f$ is 
\index{Schwartz}
a~Schwartz function (see exercise\ref{ex:Amn}). 
That is $f$ is infinitely differentiable
and \(x^nD^m f(x)\) are bounded for any \(m,n\in\Z^+\).
Let \(A_{mn}\) be such that \(|x^nD^m f(x)| < A_{mn}\).
We have 
\begin{equation*}
\left|(x+ka)^2f(x+ka)\right| \leq A_{12}
\end{equation*}
hence
\begin{equation*}
\sum_{k = -\infty}^\infty |f(x + ka)|
\leq A_{12} \sum_{k = \infty}^\infty|x+ka|^{-2} < \infty.
\end{equation*}

\iffalse
Pick arbitrary \(x\in\R\) and \(a>0\).
We will now show that \(\sum_{k = -\infty}^\infty f(x + ka)\) 
absolutely converges.
Suffices to show that 
\begin{equation}
\sum_{k \geq k_0} |f(x + ka)| < \infty
\end{equation}
for some \(k_0\). We pick \(k_0\) such that \(x+k_0a > 1\).

Clearly the set \(\{k\in\Z: |x+ka|\leq 1\}\) is finite, and so
\begin{equation*}
A := \left|\sum_{\overset{k\in\Z}{|x+ka|\leq 1}} f(x + ka)\right| < \infty.
\end{equation*}

\begin{align*}
\left|\sum_{k = -\infty}^\infty f(x + ka)\right|
&\leq A + \sum_{\overset{k\in\Z}{|x+ka|>1}} |f(x + ka)| \\
&\leq A + \sum_{\overset{k\in\Z}{|x+ka|>1}} |(x+ka)\cdot f(x + ka)| 
\end{align*}
\fi 

By exercise~\ref{ex:Amn}
\begin{itemize}

\item Clearly \(f\in L^1\).

\item 
The Fourier transform \(\widehat{f}\) is a~Schwartz function as well.
Hence \(\widehat{f}\in L^1\)

\item The series
\begin{gather*}
\sum_{k = -\infty}^\infty f(x + 2k\pi) \\
\sum_{n = -\infty}^\infty \varphi(n) 
\end{gather*}
absolutely converge.
\end{itemize}

As defined above, \(F(x)\) converges absolutely and so its 
derivatives. Clearly it has a period of \(2\pi\) and
thus \(F\in C^\infty(\T)\). Also
\begin{align*}
\int_{-\pi}^\pi F(x)\,dx 
&= \int_{-\pi}^\pi \left(\sum_{k = -\infty}^\infty f(x + 2k\pi)\right)\,dx
   = \sum_{k = -\infty}^\infty \int_{-\pi}^\pi f(x + 2k\pi)\,dx \\
&= \int_{-\infty}^\infty f(x + 2k\pi)\,dx < \infty
\end{align*}
The fact that \(F\in L^1(\T)\) can also be derived 
by the fact that it is continuous on the compcat \T\ and thus bounded.
This argument also shows \(F\in L^p(\T)\) for all \(1\leq p \leq \infty\).

According to the analysis of section~4.26, we define the Fourier coefficient
\begin{equation*}
\widehat{F}(n) = \frac{1}{2\pi} \int_{-\pi}^\pi F(t)e^{-int}\,dt\,.
\end{equation*}
Hence
\begin{align*}
\widehat{F}(n) 
&= \frac{1}{2\pi} \int_{-\pi}^\pi 
     \left(\sum_{k = -\infty}^\infty f(t + 2k\pi)\right)e^{-int}\,dt
 = \frac{1}{2\pi} \sum_{k = -\infty}^\infty
     \int_{-\pi}^\pi f(t + 2k\pi)e^{-int}\,dt \\
&= \frac{1}{2\pi} \int_{-\infty}^\infty f(t)e^{-int}\,dt 
 = \varphi(n)\,.
\end{align*}

Since \(F\in L^2(\T) \subset L^1(\T)\) we have
\begin{equation*}
F(x) = \sum_{n\in\Z} \widehat{F}(n)e^{inx} = \sum_{n\in\Z} \varphi(n)e^{inx}
\end{equation*}
The desired equality above is given by evaluating \(F(0)\).

Now for \(\beta > 0\) define
\begin{align*}
f_\beta(x) &= f(\beta x/(2\pi))\\
\varphi_\beta(t) 
  &= \frac{1}{2\pi} \int_{-\infty}^\infty f_\beta(x)e^{-itx}\,dx
   = \frac{1}{\sqrt{2\pi}} \widehat{f_\beta}(t)
\end{align*}
By Theorem~9.2\ich{d} 
\begin{equation*}
\widehat{f_\beta}(x) = (2\pi/\beta)\widehat{f}(2\pi x/\beta)\,.
\end{equation*}
Clearly \(f_\beta\) is as Schwartz function. By what was shown
and putting \(\alpha = 2\pi/\beta\) we have
\begin{align*}
\sum_{k = -\infty}^\infty f(k\beta) 
&= \sum_{k = -\infty}^\infty f_\beta(2k\pi) 
 = \sum_{n = -\infty}^\infty \varphi_\beta(n)
 = \sum_{n = -\infty}^\infty \frac{1}{\sqrt{2\pi}} \widehat{f_\beta}(n) \\
&= \sum_{n = -\infty}^\infty 
   \frac{1}{\sqrt{2\pi}} (2\pi/\beta)\widehat{f}\bigl((2\pi/\beta)n\bigr)
 = (2\pi/\beta) \sum_{n = -\infty}^\infty 
   \varphi\bigl((2\pi/\beta)t\bigr) \\
&= \alpha \sum_{n = -\infty}^\infty  \varphi(\alpha n)
\end{align*}

Looking at the limit
\begin{equation*}
\lim_{\alpha\to 0} \alpha \sum_{n = -\infty}^\infty \varphi(n\alpha)
= \lim_{\beta\to \infty} \sum_{k = -\infty}^\infty f(k\beta) 
= f(0)
\end{equation*}

%%%%%%%%%%%%%% 12
\begin{excopy}
Take \(f(x) = e^{-|x|}\) in exercise~11 and derive the identity
\begin{equation*}
\frac{e^{2\pi\alpha} + 1}{e^{2\pi\alpha} - 1}
= \frac{1}{\pi} \sum_{n=-\infty}^\infty \frac{\alpha}{\alpha^2 + n^2}.
\end{equation*}
\end{excopy}

Using this $f$ (and \(1/\alpha\) instead of \(\alpha\))
\begin{align}
\sum_{k = -\infty}^\infty f(2k\pi\alpha) 
&= \sum_{k = -\infty}^\infty e^{-|2k\pi\alpha|}
= 1 + 2\sum_{k = 1}^\infty \left(e^{-2\pi\alpha}\right)^k
= 1 + 2\frac{e^{-2\pi\alpha}}{1 - e^{-2\pi\alpha}}
= \frac{e^{-2\pi\alpha} + 1}{1 - e^{-2\pi\alpha}} \notag \\
&= \frac{e^{2\pi\alpha} + 1}{e^{-2\pi\alpha}-1}  \label{eq:sum:fkpia}
\end{align}

As for \(\varphi\)
\begin{align}
2\pi\cdot\varphi(n/\alpha)
&= \int_{-\infty}^\infty e^{-|x|}e^{-i(n/\alpha)x}\,dx
 =   \int_{-\infty}^0 e^{x-i(n/\alpha)x}\,dx
   + \int_0^\infty e^{-x-i(n/\alpha)x}\,dx \notag \\
&=  \frac{1}{1-in\alpha}\left.\left(e^{x-i(n/\alpha)x}\right)\right|_{-\infty}^0
  + \frac{1}{-1-in\alpha}\left.\left(e^{-x-i(n/\alpha)x}\right)\right|_0^\infty 
     \notag \\
&=  \frac{1}{1-in/\alpha} - \frac{1}{-1-in/\alpha}
 =  \frac{(1 + in/\alpha) + (1 - in/\alpha)}{1^2 - (in/\alpha)^2} \notag \\
&= \frac{2\alpha^2}{n^2+\alpha^2}  \label{eq:2pi:varphi:na}
\end{align}

Using the \eqref{eq:ex9.11} of previous exercise 
(with \(1/\alpha\) replacing \(\alpha\)) 
and the above
\eqref{eq:sum:fkpia},
\eqref{eq:2pi:varphi:na} 
we get
\begin{align*} 
\frac{e^{2\pi\alpha} + 1}{e^{2\pi\alpha} - 1}
&= \sum_{k = -\infty}^\infty f(2k\pi\alpha) 
= \frac{1}{\alpha} \sum_{n = -\infty}^\infty \varphi(n/\alpha)
= \frac{1}{\alpha} 
     \sum_{n = -\infty}^\infty 
        \frac{1}{2\pi}\cdot \frac{2\alpha^2}{n^2+\alpha^2} \\
&= \frac{1}{\pi} \sum_{n = -\infty}^\infty  \frac{\alpha}{n^2+\alpha^2}\,.
\end{align*}


%%%%%%%%%%%%%% 13
\begin{excopy}
If \(0 < c < \infty\), define \(f_c(x) = \exp(-cx^2)\).
\begin{itemize}
\itemch{a} 
  Compute \(\widehat{f_c}\). \emph{Hint}: If \(\varphi = \widehat{f_c}\),
    an integration by parts gives \(2c\varphi'(t) + t\varphi(t) = 0\).
\itemch{b}
  Show that there is one (and only one) $c$ for which \(\widehat{f_c} = f_c\).
\itemch{c}
  Show that \(f_a \ast f_b = \gamma f_c\); find \(\gamma\) and $c$ 
  explicitly in terms of $a$ and $b$.
\itemch{d}
  Take \(f = f_c\) in Exercise~11. What is the resulting identity?
\end{itemize} 
\end{excopy}

\begin{itemize}
\itemch{a} 
Put \(f_c(x) = \exp(-cx^2)\) and \(\varphi = \widehat{f_c}\).
We will first compute \(\varphi(0)\).

The following equality is shown 
in \cite{RudinPMA85}~Section~8.31.
\begin{equation*}
\int_{-\infty}^\infty e^{-x^2}\,dx = \sqrt{\pi}.
\end{equation*}
By substituting \(y=\sqrt{c}x\) using \(dx/dy=1/\sqrt{a}\) we get
\begin{equation*}
\int_{-\infty}^\infty e^{-cx^2}\,dx 
= \int_{-\infty}^\infty e^{-y^2}/sqrt{c}\,dy = \sqrt{\pi/c}.
\end{equation*}
Hence
\begin{equation*}
\varphi(0)
= \frac{1}{\sqrt{2\pi}} \int_{-\infty}^\infty e^{-cx^2}e^{-i0x}\,dx \\
= \frac{1}{\sqrt{2\pi}} \cdot \sqrt{\pi/c} 
= 1 / \sqrt{2c}.
\end{equation*}

For any \(t\in\R\)
\begin{align*}
\varphi(t) 
&= \frac{1}{\sqrt{2\pi}} \int_{-\infty}^\infty e^{-cx^2}e^{-itx}\,dx \\
&= \frac{1}{\sqrt{2\pi}} 
  \left.\left(e^{-cx^2}\cdot
        \left(\frac{-1}{it}\right)e^{-itx}
  \right)\right|_{-\infty}^\infty
  - \frac{1}{\sqrt{2\pi}} \int_{-\infty}^\infty 
        \frac{2cx}{it}\cdot  e^{-cx^2}e^{-itx}\,dx \\
&= \frac{2ci}{\sqrt{2\pi}t} \int_{-\infty}^\infty x e^{-cx^2-itx}\,dx
\end{align*}

Its derivative
\begin{equation*}
\varphi'(t) 
= \frac{1}{\sqrt{2\pi}} \int_{-\infty}^\infty (-ix)e^{-cx^2}e^{-itx}\,dx
= \frac{-i}{\sqrt{2\pi}} \int_{-\infty}^\infty xe^{-cx^2}e^{-itx}\,dx
\end{equation*}
Combining the above equalities gives
\begin{equation*}
\frac{t}{2c}\varphi(t)
= \frac{i}{\sqrt{2\pi}} \int_{-\infty}^\infty xe^{-cx^2}e^{-itx}\,dx
= -\varphi'(t) 
\end{equation*}
and so we get the hint which is a homogeneous differential equation
\begin{equation*}
2c\varphi'(t) + t\varphi(t) = 0
% \varphi'(t) = \frac{-t}{2c}\varphi(t).
\end{equation*}
whose solution is
\begin{equation*}
\varphi(t) 
= \varphi(0)\exp\left(-\int_0^t\frac{t}{2c}\,dt\right)
= \frac{1}{\sqrt{2c}}\exp\left(\frac{-t^2}{4c}\right).
\end{equation*}

\itemch{b}
By previous item,
\(c=\half\) is the unique $c$ such that \(f_c = \widehat{f_c}\).

\itemch{c}
Let \(g = f_a \ast f_b\). By Theorem~9.2\ich{c} and previous item
\begin{align*}
\widehat{g}(t) 
= \widehat{f_a}(t)\widehat{f_b}(t)
&= \frac{1}{\sqrt{2a}}\exp\left(\frac{-t^2}{4a}\right) \cdot
  \frac{1}{\sqrt{2b}}\exp\left(\frac{-t^2}{4b}\right)
= \frac{1}{2\sqrt{ab}} \exp\left(\frac{-t^2}{4a}+\frac{-t^2}{4b}\right) \\
&= \frac{1}{2\sqrt{ab}} \exp\left(\frac{-(a+b)t^2}{4ab}\right).
\end{align*}
Thus if \(g = \gamma f_c\) then by looking at \(\widehat{f_c}\) 
we must have
\(c = (a+b)/(4ab)\) and 
\begin{equation*}
\gamma / \sqrt{2c} = 1/(2\sqrt{ab}).
\end{equation*}
Hence
\begin{equation*}
\gamma 
= \frac{\sqrt{2c}}{2\sqrt{ab}}
= \frac{\sqrt{2(a+b)}}{2\sqrt{ab(4ab)}}
= \frac{\sqrt{2(a+b)}}{4ab}
\end{equation*}
 
\itemch{d}
By \cite{RudinPMA85}~Section~8.21
\begin{equation*}
\int_{-\infty}^\infty e^{-x^2}\,dx = \sqrt{\pi}.
\end{equation*}
Hence, using the substitution \(y^2=cx^2\) and \(dx/dy=\sqrt{1/c}\) 
we have
\begin{equation*}
  \int_{-\infty}^\infty e^{-cx^2}\,dx 
= \int_{-\infty}^\infty e^{-y^2}\sqrt{1/c}\;dy = \sqrt{\pi/c}
\end{equation*}

Now look at a Fourier transform (without a constant factor).
Given the substitution \(x = (1/\sqrt{c})y - it/(2c)\) we can 
have a ``sqaure -- linear-free'' expression
\begin{equation*}
-cx^2 -itx 
= -c\left(\frac{1}{\sqrt{c}}y - \frac{it}{2c}\right)^2
  -it\left(\frac{1}{\sqrt{c}}y - \frac{it}{2c}\right)
= -y^2 - \frac{t^2}{4c}
\end{equation*}
and \(dx/dy = 1/\sqrt{c}\). Now
\begin{equation*}
\int_{-\infty}^\infty e^{-cx^2}e^{-itx}\,dx
= \int_{-\infty}^\infty e^{-y^2 - t^2/(4c)}\frac{1}{\sqrt{c}}\,dy
= \frac{e^{-t^2/(4c)}}{\sqrt{c}}
   \int_{-\infty}^\infty e^{-y^2}\,dy
= \sqrt{\frac{\pi}{c}}e^{-t^2/(4c)}
\end{equation*}
Hence if \(f_c(x) = \exp(-cx^2)\) then
\begin{equation} \label{eq:gaussian:fourier}
\widehat{f_c}(t) 
= \frac{1}{\sqrt{2\pi}} \sqrt{\frac{\pi}{c}}e^{-t^2/(4c)}
= \frac{1}{\sqrt{2c}} e^{-t^2/(4c)}
\end{equation}

We note that 
in the exercise~11
\(\varphi\) 
is defined with another constant, namely
\(\varphi = (1/\sqrt{2\pi})\widehat{f}\).
Thus 
\begin{equation*}
\alpha \sum_{n = -\infty}^\infty \varphi(n\alpha)
= \frac{\alpha}{\sqrt{2\pi}} \sum_{n = -\infty}^\infty 
  \frac{1}{\sqrt{2c}} e^{-(n\alpha)^2/(4c)}
= \frac{\alpha}{2\sqrt{\pi c}} \sum_{n = -\infty}^\infty  e^{-(n\alpha)^2/(4c)}
\end{equation*}

Now the equality
\begin{equation} 
\sum_{k = -\infty}^\infty f(k\beta) 
= \alpha \sum_{n = -\infty}^\infty \varphi(n\alpha).
\qquad \textnormal{if}\; \alpha>0, \beta>0,\, \alpha\beta = 2\pi.
\end{equation}
of exercise~11 with \(f=f_c\) becomes
\begin{equation*}
\sum_{k = -\infty}^\infty e^{-c(k\beta)^2}
= \frac{\alpha}{2\sqrt{\pi c}} \sum_{n = -\infty}^\infty  e^{-(n\alpha)^2/(4c)}
\end{equation*}
With \(\alpha=1\) and \(\beta=2\pi\) it becomes
\begin{equation*}
\sum_{k = -\infty}^\infty e^{-c(2\pi k)^2}
= \frac{1}{2\sqrt{\pi c}} \sum_{n = -\infty}^\infty  e^{-n^2/(4c)}
\end{equation*}
\end{itemize}


%%%%%%%%%%%%%% 14
\begin{excopy}
The Fourier transform can be defined for \(f\in L^1(\R^k)\) by
\begin{equation*}
\Hat{f}(y) = \int_{\R^k} f(x)e^{-ix\cdot y}\,dm_k(x)\qquad (y\in\R^k),
\end{equation*}
where \((x\cdot y) = \sum \xi_j \eta_j\) if 
\(x= (\seq{\xi}{k})\),
\(y= (\seq{\eta}{k})\)
and \(m_k\) is Lebesgue measure on \(\R^k\),
divided by \((2\pi)^{k/2}\) for convenience. Prove the inversion theorem and
\index{Plancherel theorem}
the Plancherel theorem in this context, as well as the analogue of Theorem~9.23.
\end{excopy}

This is actually redoing most of the chapter~(9) generalizing 
the domain of functions from real line to \(\R^k\).
This approach is taken in \cite{LiebLoss200104} Chapter~5 from the start.

Let's proceed, we put in brackets the sections theorems numbers
being generalized.

We generalize the definition [9.1(4)] of 
\index{convolution}
convolution.
Let \(f,g\in L^1(\R^k)\), then
\begin{equation*}
(f \ast g)(x) = \int_{\R^k} f(x-y)g(y)\,dm_k(y).
\end{equation*}

The generalization of Theorem~9.2 items: 
\ich{a},
\ich{b}
\ich{c},
\ich{d}
follows easily.

The generalization of Theorem~9.5 is trivial.
We carry over the definition
\begin{equation*}
f_y(x) = f(x-y) \qquad (x,y\in \R^k).
\end{equation*}
% We can also consider multi-valued (in \(\C^n\)) functions.
The change needed in the proof worth noting is to take 
$k$-power of \(\epsilon\).

Generalization of Theorem~9.6.
\begin{llem} \label{lem:9.6:kdim}
If \(f\in L^1(\R^k)\), then \(\widehat{f}\in C_0(\R^k)\) and 
\begin{equation*}
\|\widehat{f}\|_\infty \leq \|f\|_1.
\end{equation*}
\end{llem}
\begin{thmproof}
The inequality is obvious from the definition of the transform
and noting that \(|e^{-i(t\cdot x)}|=1\).

If \(\lim_{n\to 0}t_n=t \in \R^k\), then
\begin{equation*}
\left|\widehat{f}(t_n) - \widehat{f}(t)\right|
\leq \int_{\R^k} 
     |f(x)|\cdot\left|e^{-i(t_n\cdot x)} - e^{-i(t\cdot x)}\right|\,dm(x).
\end{equation*}
The integrand is bounded by \(2|f(x)|\) and vanishes as \(n\to\infty\).
Hence \(\lim_{n\to\infty} \widehat{f}(t_n)=\widehat{f}(t)\), 
by the dominated convergence theorem~1.34. Thus \(\widehat{f}\) 
is continuous.

% By definitions and using \(e^{\pi i}=-1\)
% Let \(u=(1,1,\ldots,1)\in\R^k\), hence 
For each \(t\in\R^k\setminus\{0\}\) we can pick ``an inverse''
\(\tau(t)\in\R^k\) such that \(t\cdot\tau_t = 1\) in the following way.
Let \(m\in \N_k\) be the minimal such that
\(|t_m|=\max\{|t_j|: 1\leq j \leq k\}\)
and define
\begin{equation*}
\tau(t)_j = \left\{%
\begin{array}{ll}
1/t_m \quad& \textrm{if}\; j=m \\
0          & \textrm{if}\; j\neq m \\
\end{array}\right.
\end{equation*}

Note that if \(\lim_n \|t_n\|_p=\infty\) then 
\begin{equation} \label{eq:taut:to0}
\lim_n \|\tau(t_n)\|_p = 0  \qquad (1\leq p \leq \infty)
\end{equation}

Now
\begin{align*}
\widehat{f}(t) 
&= \int_{\R^k} f(x)e^{-i(t\cdot x)}\,dm_k(x)
 = -\int_{\R^k} f(x)e^{-i(t\cdot (x + \pi\tau(t))}\,dm_k(x) \\
&= -\int_{\R^k} f(x - \pi\tau(t))e^{-i(t\cdot x)}\,dm_k(x).
\end{align*}
Hence 
\begin{equation*}
2\widehat{f}(t) 
= \int_{\R^k} \bigl(f(x) - f(x - \pi\tau)\bigr) e^{-i(t\cdot x)}\,dm_k(x)
\end{equation*}
so that
\begin{equation*}
2\|\widehat{f}(t)\| \leq \|f - f_{\pi\tau(t)}\|_1.
\end{equation*}
By previous local lemma~\ref{lem:9.6:kdim} and \eqref{eq:taut:to0}
\begin{equation*}
\lim_{\|t\|\to\infty} \|f - f_{\pi\tau(t)}\|_1 
= \lim_{\|\tau(t)\|\to 0} \|f - f_{\pi\tau(t)}\|_1 = 0.
\end{equation*}
Hence \(\widehat{f}\in C_0(\R^k)\).
\end{thmproof}


Define [9.7]
\begin{align*}
H(t) &= e^{-\sum_{1\leq j\leq k}|t_j|} \qquad t \in \R^k \\
h_\lambda(x) &= \int_{\R^k} H(\lambda t) e^{i(t\cdot x)}\,dm(t) 
 \qquad \lambda > 0, \; x\in\R^k
\end{align*}

Compute \(h_\lambda(x)\) using the identity established in the 
$1$-dimensioal case in the text.
\begin{align*}
h_\lambda(x) 
&= \int_{\R^k} e^{-\lambda\sum_{1\leq j\leq k}|t_j|} 
               e^{i\sum_{1\leq j\leq k} t_jx_j}\,dm(t) \\
&= 
    \int_{-\infty}^\infty
    \int_{-\infty}^\infty
    \cdots
    \int_{-\infty}^\infty
      e^{\sum_{1\leq j\leq k}-\lambda|t_j| + i t_jx_j}
    \,dm(t_k) 
     \cdots
    \,dm(t_2) 
    \,dm(t_1) 
    \\
&= 
    \int_{-\infty}^\infty
    e^{-\lambda|t_1|+it_1x_1}
    \int_{-\infty}^\infty
    e^{-\lambda|t_2|+it_2x_2}
    \cdots
    \int_{-\infty}^\infty
      e^{-\lambda|t_k| + t_kx_k}
    \,dm(t_k) 
     \cdots
    \,dm(t_2) 
    \,dm(t_1) 
    \\
&= \prod_{j=1}^k \sqrt{\frac{2}{\pi}}\frac{\lambda}{\lambda^2+x_j^2} \\
&= \left(\frac{2}{\pi}\right)^{k/2} \lambda^k 
   \prod_{j=1}^k \frac{1}{\lambda^2+x_j^2}
\end{align*}

The intergal is similarly (using Fubini's theorem~8.8) computed
\begin{align}
\int_{\R^k}h_\lambda(x)\,dm(x)
&=  \int_{-\infty}^\infty
    \int_{-\infty}^\infty
    \cdots
    \int_{-\infty}^\infty
     \left(\frac{2}{\pi}\right)^{k/2} \lambda^k 
       \prod_{j=1}^k \frac{1}{\lambda^2+x_j^2}
    \,dm(t_k) 
     \cdots
    \,dm(t_2) 
    \,dm(t_1) 
    \notag \\
&=  \int_{-\infty}^\infty
      \sqrt{\frac{2}{\pi}}\frac{\lambda}{\lambda^2+x_1^2}
    \cdots
    \int_{-\infty}^\infty
      \sqrt{\frac{2}{\pi}}\frac{\lambda}{\lambda^2+x_k^2}
    \,dm(t_k) 
     \cdots
    \,dm(t_1) 
    \notag \\
&= 1^k = 1. \label{eq:kdim:9.7(4)}
\end{align}
Note that \(0<H(t)\leq 1\) and \(\lim_{\lambda\to 0} H(\lambda t) = 1\).

The next lemma will give us a convolution [(9.8)] equality.
\begin{llem} \label{lem:9.8:kdim}
If \(f\in L^1(\R^k)\), then
\begin{equation} \label{eq:9.8:kdim}
(f\ast h_\lambda)(x) 
= \int_{\R^k} H(\lambda t) \widehat{f}(t)e^{i(x\cdot t)}\,dm(t).
\end{equation}
\end{llem}
\begin{thmproof}
Using Fubini's theorem~8.8
\begin{align}
(f\ast h_\lambda)(x) 
&= \int_{\R^k} f(x-y)
     \left(\int_{\R^k} H(\lambda t)e^{i(t\cdot y)}\,dm(t)\right)\,dm(y) 
     \notag \\
&= \int_{\R^k} H(\lambda t)
     \left(\int_{\R^k} f(x-y)e^{i(t\cdot y)}\,dm(y)\right)\,dm(t)
     \notag \\
&= \int_{\R^k} H(\lambda t)
     \left(\int_{\R^k} f(y)e^{i(t\cdot (x-y))}\,dm(y)\right)\,dm(t)
     \label{eq:kdim:9.8}
     \notag \\
&= \int_{\R^k} H(\lambda t)e^{i(t\cdot x)}
     \left(\int_{\R^k} f(y)e^{-i(t\cdot y)}\,dm(y)\right)\,dm(t)
     \notag \\
&= \int_{\R^k} H(\lambda t)e^{i(t\cdot x)}\widehat{f}(t)\,dm(t).
\end{align}
In \eqref{eq:kdim:9.8} we actually changed \((x-y)\)
by a ``new'' variable $y$.
\end{thmproof}

Generalizing [9.9]
\begin{llem} \label{lem:9.9:kdim}
If \(g\in L^\infty(\R^k)\) and $g$ is continuous at a point $x$, then
\begin{equation}
\lim_{\lambda\to 0} (g \ast h_\lambda)(x) = g(x). \label{eq:9.9:kdim}
\end{equation}
\end{llem}
\begin{thmproof}
We will integrate using the following
\begin{align*}
h_\lambda(y)
&= \left(\frac{2}{\pi}\right)^{k/2} \lambda^k
   \prod_{j=1}^k \frac{1}{\lambda^2+y_j^2}
  \\
&= \left(\frac{2}{\pi}\right)^{k/2} \frac{\lambda^k}{\lambda^{2k}}
   \prod_{j=1}^k \frac{1}{1/\lambda^2}\frac{1}{\lambda^2+y_j^2}
=  \left(\frac{2}{\pi}\right)^{k/2} 
   \prod_{j=1}^k \frac{1}{1+(y_j/\lambda)^2}
 \\
&= h_1(y/\lambda) / \lambda^k
\end{align*}
On account of \eqref{eq:kdim:9.7(4)}, we have
\begin{align*}
\lim_{\lambda\to 0} (g \ast h_\lambda)(x) 
&= \int_{\R^k} \bigl(g(x-y) - g(x)\bigr)h_\lambda(y)\,dm(y) \\
&= \int_{\R^k} \bigl(g(x-y) - g(x)\bigr)\lambda^{-k}h_1(y/\lambda)\,dm(y) \\
&= \int_{\R^k} \bigl(g(x-\lambda s) - g(x)\bigr)
               \lambda^{-k}h_1(s) \prod_{j=1}^k \frac{dy_j}{ds_j}\,dm(s) \\
&= \int_{\R^k} \bigl(g(x-\lambda s) - g(x)\bigr)h_1(s) \,dm(s) 
\end{align*}
Looking at the last integrand
\begin{align*}
\left|\bigl(g(x-\lambda s) - g(x)\bigr)h_1(s)\right| 
     &\leq 2\|g\|_\infty h_1(s) \\
\forall s\in\R^k,\quad 
  \lim_{\lambda\to 0} \bigl(g(x-\lambda s) - g(x)\bigr)h_1(s) &= 0.
\end{align*}
Hence \eqref{eq:9.9:kdim} follows from
the dominated convergence theorem~1.34.
\end{thmproof}

Generalizing the approximation via convolution [9.10]
\begin{llem} \label{lem:9.10:kdim}
If \(1\leq p < \infty\) and \(f\in L^P(\R^k)\), then
\begin{equation}
 \lim_{\lambda\to 0} \|f\ast h_\lambda - f\|_p = 0. \label{eq:9.10:kdim} 
\end{equation}
\end{llem}
\begin{thmproof}
Since \(h_\lambda L^(\R^k)\), where \(1/p+1/q=1\), 
\(f\ast h_\lambda)(x)\) is defined for every \(x\in\R^k\).
By~\eqref{eq:kdim:9.7(4)} we have
\begin{equation*}
(f\ast h_\lambda)(x) - f(x) 
= \int_{\R^k} \bigl(f(x-y)-f(x)\bigr)h_\lambda(y)\,dm(y)
\end{equation*}
and 
\index{Holder@H\"older}
H\"older inequality (theorem~3.3) gives
\begin{align*}
|(f\ast h_\lambda)(x) - f(x)|
&\leq \int_{\R^k} \bigl|f(x-y)-f(x)\bigr|h_\lambda(y)\,dm(y) \\
&=    \int_{\R^k} \left(\bigl|f(x-y)-f(x)\bigr|^p 
           h_\lambda^{1/p}(y)\right) h_\lambda^{1/q}(y)
          \,dm(y) \\
&\leq 
  \left( \int_{\R^k} \bigl|f(x-y)-f(x)\bigr|^p
        h_\lambda(y) \,dm(y) \right)^{1/p} 
  \left(  \int_{\R^k} \left(h_{1/q}\lambda(y)\right)^q\,dm(y)\right)^{1/q} \\
&= \left( \int_{\R^k} \bigl|f(x-y)-f(x)\bigr|^p h_\lambda(y)
                     \,dm(y) \right)^{1/p}
\end{align*}
The last equality holds becuse of \eqref{eq:kdim:9.7(4)}.
Taking power over the above, gives
\begin{equation*}
|(f\ast h_\lambda)(x) - f(x)|^p
\leq \int_{\R^k} \bigl|f(x-y)-f(x)\bigr|^p h_\lambda(y) \,dm(y) 
\end{equation*}
Integrating with respect to $x$ and apply using Fubini's theorem~8.8:
\begin{align}
\| f\ast h_\lambda - f\|_p^p 
&\leq \int_{\R^k} \left(\bigl|f(x-y)-f(x)\bigr|^p h_\lambda(y) \,dm(y)
                 \right)\,dm(x) \notag \\
&= \int_{\R^k} \left(\bigl|f(x-y)-f(x)\bigr|^p \,dm(x)
                 \right) h_\lambda(y) \,dm(y) \notag \\
&= \int_{\R^k} \|f_y - f\|_p^p h_\lambda(y) \,dm(y). \label{eq:fyfpp:hl}
\end{align}
If \(g(y) = \|f_y - f\|_p^p\), then $g$ is bounded and continuous,
by a trivial $k$-dimensioal generalization of theorem~9.5, 
and \(g(0)=0\). Hence the expression in \eqref{eq:fyfpp:hl}
tends to $0$ as \(\lambda\to 0\), by local lemma~\ref{lem:9.9:kdim}
\end{thmproof}

Now we arrive to the generalization of the inversion theorem [9.11].
\begin{llem} \label{lem:9.11:kdim}
If \(f,\widehat{f}\in L^1(\R^k)\) and if 
\begin{equation*}
g(x) = \int_{\R^k} \widehat{f}(t)e^{ixt}\,dm(t)
\end{equation*}
then \(g\in C_0\) and \(f(x)=g(x)\;\aded\)
\end{llem}
\begin{thmproof}
We use the result of local lemma~\ref{lem:9.8:kdim} 
The integrands on the right side of \eqref{eq:9.8:kdim} are bounded
by \(|\widehat{f}(t)\), and since \(\lim_{\lambda\to 0} H(\lambda t) = 0\),
the right side of \eqref{eq:9.8:kdim} converges to \(g(x)\)
for every \(x\in\R^k\), by the dominated convergence theorem~1.34.

By combining local lemma~\ref{lem:9.10:kdim} and theorem~3.12
there is a sequence \(\{\lambda_n\}\) 
such that \(\lim_{n\to\infty}\lambda_n=0\) and 
\begin{equation*}
\lim_{n\to\infty} (f\ast h_{\lambda_n})(x) = f(x)\quad\aded
\end{equation*}
Hence \(f(x)=g(x)\;\aded\)
and \(g\in C_0(\R^k)\) by local lemma~\ref{lem:9.6:kdim}.
\end{thmproof}

\index{Plancherel}
\paragraph{Generalization of Plancherel Theorem [9.13].}
\begin{llem}
Once can associate to each \(f\in L^2(\R^k)\) a function
\(\widehat{f}\in L^2(\R^k)\) so that the following properties hold:
\begin{itemize}

\itemch{a}
If \(f \in L^1(\R^k)\cap L^2(\R^k)\), 
then \(\widehat{f}\) is the previously defined Fourier transform of $f$.

\itemch{b}
For every \(f \in L^2(\R^k)\), \(\|\widehat{f}\|_2 = \|f\|_2\).

\itemch{c}
The mapping \(f \to \widehat{f}\) is a 
\index{Hilbert}
Hilbert space isomorphism of \(L^2(\R^k)\) onto \(L^2(\R^k)\)

\itemch{d}
The following symmetric relation exists between $f$ and \(\widehat{f}\):
If
\begin{equation*}
\varphi_A(t) \int_{[-A,A]^k} f(x)e^{-i(x\cdot t)}\,dm(x)
\qquad \textrm{and} \qquad
\psi_A(t) \int_{[-A,A]^k} \widehat{f}(x)e^{-i(x\cdot t)}\,dm(x),
\end{equation*}
then
\begin{equation*}
\lim_{A\to+\infty} \|\varphi_A - \widehat{f}\|_2 = 0 \\
\qquad \textrm{and} \qquad
\lim_{A\to+\infty} \|\psi_A - f\|_2 = 0
\end{equation*}
\end{itemize}
\end{llem}
\begin{thmproof}
Our objective is the relation
\begin{equation} \label{eq:plancherel:iso:kdim}
\|\widehat{f}\|_2 = \|f\|_2 \qquad (f\in L^1(\R^k)\cap L^2(\R^k)).
\end{equation}
We fix \(f\in L^1(\R^k)\cap L^2(\R^k)\), 
put \(\widetilde{f}(x) = \overline{f(-x)}\),
(clearly \(\widetilde{f} \in L^1(\R^k)\cap L^2(\R^k)\))
and define \(g = f \ast \widetilde{f}\). Then
\begin{equation*}
g(x) 
= \int_{\R^k} f(x-y)\overline{f(-y)}\,dm(y)
= \int_{\R^k} f(x+y)\overline{f(y)}\,dm(y)
= \langle f_x, f \rangle,
\end{equation*}
where the inner product is taken in \(L^2(\R^k)\).

\begin{quotation}
Note that \(\widetilde{f}\) ``negates the direction'' of the variable,
while it gets ``negated back'' in the convolution $g$.
Thus, in the integration is with respect to variables going
in the ``same direction''
\end{quotation}

By generalization (mentioned above) of theorem~9.5 
\(x\to f_x\) is a continuous mapping, and the continuity of 
the inner product, $g$ is a continuous function.
\index{Schwartz}
Schwartz inequality shows that
\begin{equation*}
|g(x)| \leq \|f_x\|_2\cdot \|f_x\|_2 =  \|f\|_2^2
\end{equation*}
so that $g$ is bounded. Also \(g\in L^1(\R^k)\) since
\(f,\widetilde{f}\in L^1(\R^k)\).
Hence we map apply local lemma~\ref{lem:9.8:kdim} 
\begin{equation} \label{eq:gconvh0}
(g\ast h_\lambda)(0) 
= \int_{\R^k} H(\lambda t) \widehat{g}(t)\,dm(t).
\end{equation}
Since $g$ is continuous and bounded, local lemma~\ref{lem:9.9:kdim}
show sthat
\begin{equation} \label{eq:lim-gconvh0}
\lim_{\lambda\to 0} (g \ast h_\lambda)(0) = g(0) = \|f\|_2^2.
\end{equation} 


Theorem~9.2\ich{c}+\ich{d} 
(trivially generalized to \(\R^k\)) shows that 
\begin{equation*}
\widehat{g} 
= \widehat{f} \cdot\,\widehat{\tilde{f}}
= \widehat{f} \cdot\,\overline{\widehat{f}}
= |\widehat{f}|^2
\end{equation*}
and since \(H(\lambda t)\)
increases to $1$ as \(\lambda\to 0\), 
the monotone convergence theorem~1.26 gives
\begin{equation} \label{eq:limint:Hg}
\lim_{\lambda\to 0} \int_{\R^k} H(\lambda t)\widehat{g}\,dm(t)
= \int_{\R^k} \left| \widehat{f}(t)\right|\,dm(t).
\end{equation}

Now
\eqref{eq:gconvh0},
\eqref{eq:lim-gconvh0} and
\eqref{eq:limint:Hg}
shows that \(\widehat{f}\in L^2(\R^k)\) and that
the \eqref{eq:plancherel:iso:kdim} holds.
The part of the work for \(L^1(\R^k)\cap L^2(\R^k)\)) is complete.


Let $Y$ be the space of all Fourier transforms \(\widehat{f}\)
of functions \(f\in L^1(\R^k)\cap L^2(\R^k)\).
By \eqref{eq:plancherel:iso:kdim}, \(Y\subset L^2(\R^k)\).
We claim that $Y$ is dense in \(L^2(\R^k)\),
that is \(Y^\perp = \{0\}\).

The functions 
\begin{equation*}
f_{\alpha,\lambda}(x) = e^{i(\alpha\cdot x)}H(\lambda x)
\qquad \forall \alpha\in\R^k,\; \forall\lambda>0
\end{equation*}
are in \(L^1(\R^k)\cap L^2(\R^k)\). Their Fourier transforms
\begin{equation*}
\widehat{f_{\alpha,\lambda}}(t)
= \int_{\R^k} = e^{i(\alpha\cdot (t-y))}H(\lambda (t-y))\,dm(y)
= \int_{\R^k} = e^{i(\alpha\cdot x)}H(\lambda x)\,dm(x)
= h_\lambda(\alpha-t)
\end{equation*}
are therefore in $Y$. If \(w\in L^2(\R^k)\cap Y^\perp\), it follows that
\begin{equation*}
(h_\lambda \ast \overline{w})(\alpha)
= \int_{\R^k} h_{\lambda}(\alpha-t)\overline{w(t)}\,dm(t) = 0
\qquad \forall \alpha\in\R^k.
\end{equation*}
hence \(w=0\), by local lemma~\ref{lem:9.10:kdim} and therefore
$Y$ is dense in \(L^2(\R^k)\).

We temporary notate \(\widehat{f}\) by \(\Phi f\).
Collecting our result so far, shows that 
\begin{equation*}
\Phi: L^1(\R^k)\cap L^2(\R^k) \longrightarrow Y
\end{equation*}
is an \(L^2(\R^k)\)-isometry whose domain domain and range
are both dense subspaces of \(L^2(\R^k)\).
There is a unique continuous extension of \(\Phi\) from 
the whole \(L^2(\R^k)\) as a domain, and thus \ich{b} holds.

Since \(L^2(\R^k)\) is a complete metric space (as a Hilbert space),
by Lemma~4.16 this extsnion of \(\Phi\) is \emph{onto}
\(L^2(\R^k)\).
In order to show that this extended \(\Phi\) is a
\emph{Hilbert} space isomorphism, we need to show that 
the inner product is maintained. But the inner product is determined
by the norm:
\begin{equation*}
\langle v,w\rangle = 
\left(
\|v+w\|_2^2
- \|v-w\|_2^2
+ i\|v+iw\|_2^2
- i\|v-iw\|_2^2
\right) \bigm/ 4.
\end{equation*}
Hence by \ich{b} 
\begin{equation*}
\forall f,g\in L^2(\R^k),\quad 
\langle f,g\rangle = \langle \widehat{f},\widehat{g}\rangle.
\end{equation*}
and thus \ich{c} holds.

To prove \ich{d}, let 
\begin{equation*}
k_A = \chhi_{[-A,A]^k}.
\end{equation*}
Then \(k_Af \in L^1(\R^k)\cap L^2(\R^k)\) and by definitions
\begin{equation*}
\varphi_A = \widehat{k_A f}.
\end{equation*}
Since \(\lim_{A\to\infty} \|f - k_A f\|_2 = 0\), 
it follows from \ich{b} that 
\begin{equation*}
\lim_{A\to\infty} \|\widehat{f} - \varphi_A\|_2
= \lim_{A\to\infty} \|\widehat{f - k_A f}\|_2 = 0.
\end{equation*}
Similarly 
\begin{equation*}
\lim_{A\to\infty} \|f - \psi_A\|_2
= \lim_{A\to\infty} \|\Phi^{-1}(f - k_A f)\|_2 = 0.
= \lim_{A\to\infty} \|f - k_A f\|_2 = 0.
\end{equation*}
\end{thmproof}

\paragraph{Complex Homomorphisms of \(L^1(\R^k)\).}
Now we generalize theorem~9.23.
\begin{llem}
To every complex homomorphism \(\varphi\) on \(L^1(\R^k)\setminus\{0\}\)
such that
\begin{equation} \label{eq:9.23:kdim:lem}
\varphi(f\ast g) = \varphi(f)\varphi(g) \qquad (\forall f,g\in(L^1(\R^k))
\end{equation}
there corresponds a unique \(t\in\R^k\) such that
\begin{equation} \label{eq:9.23:kdim:beta}
\forall f\in L^1(\R^k),\quad  \varphi(f) = \widehat{f}(t).
\end{equation}
\end{llem}
\begin{thmproof}
By theorem~6.16, there exists a unique
\(\beta \in L^\infty(\R^k)\) such that 
\begin{equation} \label{eq:9.23:kdim:eu:beta}
\varphi(f) = \int_{\R^k} f(x)\beta(x)\,dm(x) \qquad (\forall f\in L^1(\R^k)).
\end{equation}
Looking at \eqref{eq:9.23:kdim:lem}'s left side
\begin{align}
\varphi(f \ast g)
&= \int_{\R^k} (f\ast g)(x)\cdot\beta(x)\,dm(x) \notag \\
&= \int_{\R^k} \beta(x)
     \left(\int_{\R^k} f(x-y)g(y)\,dm(y)\right)\,dm(x) \notag \\
&= \int_{\R^k} g(y)
     \left(\int_{\R^k} f_y(x)\beta(x)\,dm(x)\right)\,dm(y) \notag \\
&= \int_{\R^k} g(y)\varphi(f_y)\,dm(y). \label{eq:9.23:kdim:left}
\end{align}
Looking at the right side
\begin{equation}  \label{eq:9.23:kdim:right}
\varphi(f)\varphi(g) = \varphi(f)\int_{\R^k} g(y)\cdot\beta(y)\,dm(y).
\end{equation}
Combining
\eqref{eq:9.23:kdim:left} and
\eqref{eq:9.23:kdim:right} gives
\begin{equation} \label{eq:9.23:kdim:lr}
\int_{\R^k} g(y)\varphi(f_y)\,dm(y) 
= \varphi(f)\int_{\R^k} g(y)\cdot\beta(y)\,dm(y)
\end{equation}

% Fix \(f\in L^1(\R^k)\) such that \(\varphi(f)\neq 0\). 
Since  \eqref{eq:9.23:kdim:lr} holds for any \(g\in L^1(\R^k)\)
by the uniqueness of \(\beta\) it is wasy to show that
\begin{equation} \label{eq:vfby=vfy}
\varphi(f)\beta(y) = \varphi(f_y) \; \aded(y)
\end{equation}
\begin{quotation}
In the text \cite{RudinRCA87} the above \eqref{eq:vfby=vfy}
is established after fixing $f$ so \(\varphi(f)\neq 0\).
We do not need it so soon, thus it can be used for any \(f\in L^1(\R^k)\).
\end{quotation}
But \(y\to f_y\) is a continuous mapping
(generalizationof theorem~9.5) and \(\varphi\) 
is continuous on  \(L^1(\R^k)\).
Hence \eqref{eq:vfby=vfy}'s right side is continuous
function of \(y\in\R^k)\).
By
picking some \(f\in L^1(\R^k)\) such that \(\varphi(f)\neq 0\). 
and redefining 
\begin{equation*}
\beta(y) = \varphi(f_y) / \varphi(f)
\end{equation*}
then \(\beta\) may get changed on set of measure~0 at most, 
thus \eqref{eq:9.23:kdim:beta} still holds. Now \(\beta\) is continuous
and now \eqref{eq:vfby=vfy} holds for \emph{all} \(y\in\R^k\).
By replacing $y$ by \(x+y\) and $f$ by \(f_x\)
in \eqref{eq:vfby=vfy}, we obtain
\begin{equation*}
\varphi(f)\beta(x+y)
= \varphi(f_{x+y})
= \varphi((f_x)_y)
= \varphi(f_x)\beta(y) 
= \varphi(f)\beta(x)\beta(y) .
\end{equation*}
By picking (again) $f$ such that \(\varphi(f)\neq 0\)
we get
\begin{equation} \label{eq:9.23:kdim:beta:hom}
\forall x,y\in\R^k,\quad \beta(x+y) = \beta(x)\beta(y).
\end{equation}
Since \(\beta\) is not identically zero, 
\eqref{eq:9.23:kdim:beta:hom}implies \(\beta(0)=1\)
and the continuity of \(\beta\) shows hat the is a \(\delta>0\)
such that
\begin{equation} % \label{eq:9.23:kdim:beta:delta}
\int_{[0,\delta]^k}\beta(y)\,dy = c \neq 0.
\end{equation}
Then
\begin{equation} \label{eq:9.23:kdim:beta:delta}
c\beta(x)
= \int_{[0,\delta]^k}\beta(x)\beta(y)\,dy
= \int_{[0,\delta]^k}\beta(x + y)\,dy
= \int_{[x,x+\delta]^k}\beta(y)\,dy.
\end{equation}
Since \(\beta\) is continuous, the last integral is a differentiable
function of $x$ in each axis.
hence \eqref{eq:9.23:kdim:beta:delta} shows that \(\beta\)
is differentiable.
Differentiating \eqref{eq:9.23:kdim:beta:hom}
with respect to each axis of $y$, then put \(y=0\);
the result is
\begin{equation} \label{eq:9.23:kdim:diffeq}
\frac{\partial\beta(x)}{\partial x_j} = A_j\beta(x), 
\qquad A_j = \frac{d\beta(0)}{dx_j}
\qquad (1\leq j \leq k)
\end{equation}
Hence the partial derivatives of \(\beta(x)e^{-A_jx_j}\) are
\begin{align*}
\frac{\partial}{\partial x_j} \left(\beta(x)e^{-A_jx_j}\right)
&= \left(\frac{\partial}{\partial x_j} \beta(x)\right)e^{-A_jx_j}
   + -A_j\beta(x)e^{-A_jx_j} \\
&= \left(\left(\frac{\partial}{\partial x_j} \beta(x) 
              \right) - A_j\beta(x)\right)e^{-A_jx_j}
= 0
\end{align*}
for \(1\leq j \leq k\). Similarly the partial derivatives of
\begin{equation*}
\beta(x)\prod_{j=1}^ke^{-A_jx_j}
= \beta(x)e^{-\sum_{j=1}^k A_jx_j} = \beta(x)e^{-A\cdot x}
\end{equation*}
are
\begin{align*}
\frac{\partial}{\partial x_j} \left(\beta(x)e^{-A\cdot x}\right)
&= \left(\frac{\partial}{\partial x_j} \beta(x)\right)e^{-A\cdot x})
   -A_j \beta(x)e^{-A\cdot x} \\
&= \left(\left(\frac{\partial}{\partial x_j} \beta(x)\right)
        -A_j \beta(x)\right)e^{-A\cdot x}
= 0.
\end{align*}
Hence \(\beta(x)e^{-A\cdot x}\)
is constant in respect of each of \(\{x_j\}_{j=1}^k\).
Since \(\beta(0)=1\) we obtain
\begin{equation*}
\beta(x) = e^{A\cdot x} \qquad (A\in\C^k,\; x\in\R^k).
\end{equation*}
Since \(\beta\in L^\infty(\R^k)\) it is bounded (also continuous)
and by looking at each axis separately, we see that
\(\forall j\in\N_k,\;\Re(A_i)=0\). Thus $A$ consists
of pure imaginary numbers, and there exist \(t\in R^k\)
such that 
\begin{equation*}
\beta(x) = e^{-it\cdot x}.
\end{equation*}
By looking back  at \eqref{eq:9.23:kdim:eu:beta} we get the desired
\begin{equation*}
\varphi(f) = \int_{\R^k}f(x)e^{-it\cdot x}\,dm(x) = \widehat{f}(t).
\end{equation*}
evaluation of a Fourier transform.
\end{thmproof}


%%%%%%%%%%%%%% 15
\begin{excopy}
If \(f\in L^1(\R^k)\), $A$ is a linear operator on \(\R^k\), 
and \(g(x) = f(Ax)\),
how is \(\Hat{g}\) related to \(\Hat{f}\)?
If $f$ is invariant under rotations, i.e., if \(f(x)\) depends only 
on the euclidean distance of $x$ from the origin, prove that the same 
is true for \(\Hat{f}\).
\end{excopy}

\begin{quote}
This is a generalization of theorem~9.2\ich{e}.
But then we should  either
\textbf{(i)} think of $A$ as \(1/\lambda\),
or 
\textbf{(ii)} 
think of $A$ as \(\lambda\) \emph{and} swap between $f$ and $g$.
\end{quote}

If \(\rank{A} < \dim(A) = k\) then the mapping is not onto 
and its image \((A(\R^k)\) is a subspace of dimension \(<k\)
so \(m(A(\R^k)) = 0\) in \(\R^k\).
Thus $g$ depends on values of $f$ \emph{only} on a set of measure~$0$.
So we cannot get dependency of \(\widehat{g}\) on \(\widehat{f}\)

Now we may assume that $A$ is regular, that is, it is invertible.
We need to use some change of variable technique.
We can use either theorem~10.9 of \cite{RudinPMA85}
or theorem theorem~7.26 of our current~\cite{RudinRCA87}.
The former deals with \emph{continuous} functions
while the latter with \emph{measurable} functions but only 
of \emph{positive} values. We will use the first option.

To use theorem~10.9 of \cite{RudinPMA85}, 
let us first assume that \(f\in C_c(\R^k)\).
We have the inverse \(A^{-1}\).
Note that in our Euclidean real case, $A$ can be represented 
as an \(k\times k\) matrix $A$.
\index{Jacobian}
The Jacobian matrices  of the \(\R^k\)-mappings are
\(J_A=A\) and \(J_{A^{-1}}=A^{-1}\), the equalities hold
since $A$ is linear (so we often drop the '$J$' symbol), 
and their determinants 
which are constant since $A$ is linear and satisfy
\begin{equation*}
\left|J_{A^{-1}}\right|\cdot\left|J_A\right| = |A^{-1}|\cdot |A| = 1.
\end{equation*}
We note that there exists a linear mapping \(A^*\)
such that 
\(\langle Av,w\rangle =  \langle v, A^*w\rangle\)
for all \(v,w\in\R^k\).
See \cite{Lang94} chapter~\textsf{XIII} sections~\S5 and~\S7.

Since \(g(x) = f(Ax)\), equivalently we have \(g(A^{-1}x) = f(x)\).
\begin{align}
\widehat{f}(t) 
&= \int_{\R^k} g\left(A^{-1}y\right)e^{-i(t\cdot y)}\,dm(y) \notag \\
&= \int_{\R^k} g\bigl(A^{-1}A(x)\bigr)e^{-i(t\cdot Ax)} |J_A|\,dm(x) 
   \label{eq:fourier:jacobian:c} \\
&= |A|\int_{\R^k} g(x)e^{-i(A^*t\cdot x)} \,dm(x) 
   \label{eq:fourier:innerprod} \\
&= |A|\widehat{g}(A^*t).
\end{align}
In \eqref{eq:fourier:jacobian:c} the substitution \(y=Ax\) is used.
See also \cite{EdwFA}~5.15.4.

We also note that \(A^* = A^T\) the transpose. To see this let
\begin{equation*}
v = (\seq{v}{k}) \qquad w = (\seq{w}{k})
\end{equation*}
be arbitrary vectors in \(\R^k\). Now 
\begin{alignat*}{2}
(Av)_j &= \sum_{m=1}^k A_{j,m}v_m
\qquad
  & \langle Av,w \rangle 
     &= \sum_{j=1}^k (Av)_jw_j 
     =\sum_{j=1}^k  \left(\sum_{m=1}^k A_{j,m}v_m\right)w_j \\
(A^Tw)_j &= \sum_{m=1}^k A_{m,j}w_m
\qquad
  & \langle v,A^Tw \rangle 
     &= \sum_{j=1}^k v_jA^Tw_j 
     =\sum_{j=1}^k v_j \sum_{m=1}^k A_{m,j}w_m \\
\end{alignat*}
Looking at the double-sums , each with \(k^2\) terms, 
we see that they are just permutations of each other.
Hence 
\begin{equation*}
\langle Av,w\rangle = \langle v, A^*w\rangle
\end{equation*}
Hence \(A^* = A^T\) as matrices.
See also \cite{Herstein1975}~Theorem~6.10.2.

Now since \(C_c(\R^k)\) are dense in \(L^p(\R^k)\)
and we can converge to and \(f\in L^p(\R^k)\)
by functions \(g\in C_c(\R^k)\) 
such that \(|g(x)|\leq |f(x)|\,\aded(x)\), 
by Lebesgue dominated convergence theorem~1.34
\begin{equation}  \label{eq:fourier:jacobian}
\widehat{f}(t) = |A|\,\widehat{g}(A^*t) \qquad (\;g(x)=f(Ax)\;)
\end{equation}
or equivalently
\begin{equation}  \label{eq:fourier:jacobian:alt}
\widehat{g}(t) = |A|^{-1}\,\widehat{f}\left(\left(A^*\right)^{-1}t\right) 
\qquad (\;g(x)=f(Ax)\;)
\end{equation}
for every \(f\in L^1(\R^k)\).

\paragraph{Dependence on Euclidean distance.}
Now assume that \(f(x)\) depends on \(\|x\|_2\).
Then for any rotations mapping $A$, we have \(f(x) = f(Ax)\).
We also note that if $A$ is a rotation, then \(A^T = A^{-1}\) 
and \(|A|=1\) and  all the eigenvalues have absolute value~$1$.
See \cite{Herstein1975}~Lemma~6.10.5.
By what we have shown, 
\begin{equation*}
\widehat{f}(x) = |A|\,\widehat{f}(A^*x) = \widehat{f}(A^Tx)
\end{equation*}
for all \(A^T\) such that $A$ is a rotation.
The mapping \(A\to A^T=A^{-1}\) is (bijection) 
\emph{onto} automorphism of the group of rotations.
Hence \(\widehat{f}\) similarly depends only on the Euclidean distance.

% theorem 10.9 \cite{RudinPMA85}

%%%%%%%%%%%%%% 16
\begin{excopy}
The
\index{Laplacian}
\emph{Laplacian} 
of a function $f$ on \(\R^k\) is 
\begin{equation*}
\Delta f = \sum_{j=1}^k \frac{\partial^2f}{\partial x_j^2},
\end{equation*}
provided that the partial derivatives exist. What is the relation between 
\(\Hat{f}\) and \(\Hat{g}\) if \(g = \Delta f\)
and all necessary integrability conditions are satisfied?
It is clear that the Laplacian commutes with translations.
Prove that it also commutes with rotations, i.e. that 
\begin{equation*}
 \Delta(f \circ A) =  \Delta(f) \circ A
\end{equation*}
whenever $f$ has continuous second derivatives 
and $A$ is a rotation of \(\R^k\).
(Show that it is enough to do this under the additional assumption 
that $f$ has compact support.)
\end{excopy}

We first compute the 
\index{Laplacian}
Laplacian of the exponential 
function we convolute with in the Fourier transform.
\begin{align*}
\Delta e^{-i(x\cdot t)}
&= \sum_{j=1}^k \frac{\partial^2}{\partial x_j^2} e^{-i(x\cdot t)} 
 = \sum_{j=1}^k \frac{\partial}{\partial x_j} (-it_j) e^{-i(x\cdot t)} 
 = \sum_{j=1}^k (-it_j) \frac{\partial}{\partial x_j} e^{-i(x\cdot t)} \\
&= \sum_{j=1}^k (-it_j)^2 e^{-i(x\cdot t)} 
 = e^{-i(x\cdot t)} \sum_{j=1}^k -t_j^2  \\
&= -|t|^2e^{-i(x\cdot t)}
\end{align*}
Now assume \(f\in C_c^2(\R^k\)). 
That is, $f$ is sufficiently differentiable 
and it has (and so have its  derivatives) a compact support. Then
using the inversion theorem (local lemma~\ref{lem:9.11:kdim}) we get
\begin{equation*}  \label{eq:laplacian:fourier:inv}
\Delta f(x)
 = \Delta\int_{\R^k} \widehat{f}(t) e^{i(x\cdot t)}\,dm(t) 
 = \int_{\R^k} \widehat{f}(t) \Delta e^{i(x\cdot t)}\,dm(t) 
 = \int_{\R^k} \left(-|t|^2\right)\widehat{f}(t) e^{i(x\cdot t)}\,dm(t)
\end{equation*}
The inversion theorem also implies uniqueness of the Fourier transform.
That is if \(\widehat{g} = \widehat{h}\) then \(g=h\)
or similarly, more suitable to our case, if
\begin{equation}
\int_{\R^k}\widehat{g}(t)e^{i(x\cdot t)}\,dm(x) =
\int_{\R^k}\widehat{h}(t)e^{i(x\cdot t)}\,dm(x)
\end{equation}
then \(\widehat{g}(t)=\widehat{h}(t)\;\aded(t)\).
Hence by \eqref{eq:laplacian:fourier:inv} we have
\begin{equation} \label{eq:laplacian:fourier}
\widehat{\Delta f}(t) = -|t|^2\widehat{f}(t).
\end{equation}
By utilizing convergence theorems, \eqref{eq:laplacian:fourier}
can applied to every \(f\in L^1(\R^k)\cap C^2(\R^k)\),
that is to smooth functions not necessarily with compact support.

\paragraph{Commuting with rotation.}
Let us first explore the expression \(\Delta(f \circ A)\)
directly without referring to the Fourier transform.
We will do it merely to get some intuition for 
the complexity involved --- or later the saved complexity.
Let $A$ be a rotation operator. Since $A$ is linear
\begin{equation*}
 \frac{\partial^2}{\partial x_j^2}A(x) = 0 \qquad (\forall j \in N_k).
\end{equation*}
Temporarily fix $j$ to compute partial derivatives.
\begin{equation*}
\frac{\partial}{\partial x_j} (f \circ A)(x)
= \frac{\partial}{\partial x_j} (f \circ A)(x) 
= \left(\frac{\partial}{\partial x_j}(f \circ A)(x)\right)
   \cdot
   \frac{\partial A(x)}{\partial x_j}
\end{equation*}
We proceed to second derivative
\begin{align*}
\frac{\partial^2}{\partial x_j^2} (f \circ A)(x)
&=   \frac{\partial}{\partial x_j} 
     \left(
      \left(\frac{\partial}{\partial x_j}(f \circ A)(x)\right)
      \cdot
      \frac{\partial A(x)}{\partial x_j}
     \right) \\
&=    \frac{\partial^2 (f \circ A)(x)}{\partial x_j^2}
      \cdot
      \frac{\partial A(x)}{\partial x_j} 
      +
       \frac{\partial (f \circ A)(x)}{\partial x_j}
       \cdot
       \frac{\partial^2 A(x)}{\partial x_j^2}
       \\
&=    \frac{\partial^2 (f \circ A)(x)}{\partial x_j^2}
      \cdot
      \frac{\partial A(x)}{\partial x_j} 
\end{align*}

Back to the Laplacian
\begin{equation*}
\Delta(f \circ A)(x) 
= \sum_{j=1}^k 
     \frac{\partial^2}{\partial x_j^2} (f \circ A)(x) 
= \sum_{j=1}^k 
      \frac{\partial^2 (f \circ A)(x)}{\partial x_j^2}
      \cdot
      \frac{\partial A(x)}{\partial x_j} 
\end{equation*}

Now looking at Fourier transforms.
We will use the that \(|A|=1\) when $A$ is a rotation.
First transform the left side of the desired equality.
By \eqref{eq:laplacian:fourier} and \eqref{eq:fourier:jacobian:alt}
\begin{equation} \label{eq:fourier:laplacian:rot:left}
 \widehat{\Delta(f \circ A)}(t) 
 = -|t|^2 \widehat{(f \circ A)}(t)
 = -|t|^2 |A|^{-1} \widehat{f}\left(\left(A^*\right)^{-1}t\right)
 = -|t|^2 \widehat{f}\left(\left(A^*\right)^{-1}t\right)
\end{equation}
Transform the right side of the desired equality using
\eqref{eq:fourier:jacobian} 
and again  \eqref{eq:laplacian:fourier} 
 with \eqref{eq:fourier:jacobian:alt}
\begin{align}
 \widehat{\bigl((\Delta f) \circ A\bigr)}(t) 
&= |A|^{-1}\,\widehat{(\Delta f)}\left(\left(A^*\right)^{-1}t\right) 
 = |A|^{-1}
   \cdot 
   \left(-\left|\left(A^*\right)^{-1}t\right|^2\right)
   \cdot
   \widehat{f}\left(\left(A^*\right)^{-1}t\right) \notag 
   \\
   \label{eq:fourier:laplacian:rot:right}
&=  -|t|^2\cdot \widehat{f}\left(\left(A^*\right)^{-1}t\right) 
\end{align}
As we did in previous exercise, 
the implied uniqueness by the inversion theorems applied to 
\eqref{eq:fourier:laplacian:rot:left} and
\eqref{eq:fourier:laplacian:rot:right}
gives the desired equality
\begin{equation*}
 \Delta(f \circ A) = \Delta(f) \circ A\,.
\end{equation*}

\paragraph{Compact support.} Any function $f$ in \(L^1(\R^k)\)
and in particular any sufficiently smooth function in \(L^1(\R^k)\)
can be approximated in \(L^1(\R^k)\) by function \(g\in C_c^2(\R^k)\).
We can also ensure that \(|g(x)|\leq |f(x)|\;\aded\).
Hence 
the arguments above could be applied only to functions with compact support
and later by Lebesgue's dominated convergence theorem~1.34
be generalized to \(L^1(\R^k)\cap C^2(\R^k)\).


%%%%%%%%%%%%%% 17
\begin{excopy}
Show that every Lebesgue measurable character of \(\R^1\) is continuous.
Do the same for \(\R^k\).
(adapt part of the proof of Theorem~9.23.)
Compare with Exercise~18.
\end{excopy}

Let \(\varphi\) be a character of \(\R^k\).
Then \(\varphi\chhi_{\restriction[0,1]}\in L^1(\R)\)
and by theorem~7.11 there exists (almost everywhere) \(b\in(0,1)\) such that
\begin{equation*}
c := \int_0^b \varphi(y)\,dy \neq 0.
\end{equation*}
Now for any \(x\in\R\)
\begin{equation*}
c \varphi(x) 
= \int_0^b \varphi(y)\varphi(x)\,dx = 
= \int_0^b \varphi(y + x)\,dx = 
= \int_x^{b+x} \varphi(y + x)\,dx
\end{equation*}
Hence 
\begin{equation*}
\varphi(x) = \frac{1}{c} \int_x^{b+x} \varphi(y + x)\,dx
\end{equation*}
is continuous.

\begin{quote}
The generalization of continuity cannot be derived
by separate variable continuity, as the example
\(f(0,0)=0\) and otherwise \(f(x,y)=xy/(x^2+y^2)\) shows
(see \cite{Gelb1996}~Chapter~9).
\end{quote}

Now assume \(\varphi\) be a character of \(\R^k\).
Clearly \(\varphi(0)=1\).
For every \(a\in \R^k\) and \(j\in\N_k\) we can define,
by binding to ``\(a \setminus a_j\)'',
a character on \(\R^1\) by
\begin{equation*}
\varphi_{a,j}(t) 
= \varphi\bigl((a_1,\ldots,a_{j-1},t,a_{j+1},\ldots,a_k)\bigr).
\end{equation*}
Hence \(\varphi\) is continuous as a function of each axis.
Let \(\epsilon>0\) and pick \(\delta>0\) such that
for all \(j\in \N_k\), if \(|t|<\delta\) then
\begin{equation*}
\left|\varphi\left(
   \overbrace{0,\ldots,0}^{j-1\;\textrm{times}},
   t,
   \overbrace{0,\ldots,0}^{n-j\;\textrm{times}}\right) - 1
\right| < \epsilon.
\end{equation*}
Now if \(\|x-a\|_2 < \delta\) then
\begin{align*}
|\varphi(x) - \varphi(a)|
&\leq \sum_{j=1}^k 
  \left|
   \varphi(x_1,\ldots        ,x_j,a_{j+1},\ldots,a_k)
   -
   \varphi(x_1,\ldots,x_{j-1},a_j,\ldots,a_k) \right| \\
&= \sum_{j=1}^k 
  \left|\left((\varphi\left(
           \overbrace{0,\ldots,0}^{j-1\;\textrm{times}},
           \,x_j - a_j,\,
           \overbrace{0,\ldots,0}^{n-j\;\textrm{times}}\right)
           - 1
        \right)
   \varphi(x_1,\ldots,x_{j-1},a_j,\ldots,a_k)\right| \\
&\leq k\epsilon.
\end{align*}
Hence \(\varphi\) continuous at $a$ and thus continuous in \(\R^k\).

%%%%%%%%%%%%%% 18
\begin{excopy}
Show (with the aid of the Hausdorff maximality theorem) that there exist real
\emph{discontinuous} functions $f$ on \(\R^1\) such that 
\begin{equation} \label{eq:ex9.18}
f(x + y) = f(x) + f(y)
\end{equation}
for all $x$ and \(y\in \R^1\).

Show that if \eqref{eq:ex9.18} holds and $f$ is Lebesgue measurable then
$f$ is continuous.

Show that if \eqref{eq:ex9.18} holds and the graph of $f$ is
not dense in the plane, then $f$ is continuous.

Find all continuous functions which satisfy \eqref{eq:ex9.18}
\end{excopy}

From \eqref{eq:ex9.18} it is easy that
\begin{equation} \label{eq:ex9.18:Q}
\forall x\in\R,\;\forall q\in\Q:\quad f(qx) = q\cdot f(x).
\end{equation}

\paragraph{Discontinuous example.}
Take 
\index{Hamel}
Hamel base $H$ of \(\R\) over \(\Q\) starting with \(1,\sqrt{2}\in H\).
Define 
\(f(1)=1\) and \(f(h)=0\) for all \(h\in H \setminus\{1\}\).
Clearly $f$ can be extended to a span of any finite sub-base.
By Zorn Lemma $f$ can be extended to the whole \(\R\)
and obviously $f$ is discontinuous while its additive rule holds.

\paragraph{Measurable Character.}
Assume that $f$ is measurable.
Define
\begin{equation*}
\varphi(x) = \exp\left(i\Re\bigl(f(x)\bigr)\right).
\end{equation*}
It is easy to see that \(\varphi\) is a measurable character. 
By previous exercise
\(\varphi\) is continuous and there exist 
some (unique) \(t\in\R\) such that \(\varphi(x) = \exp(itx)\)
for all \(x\in \R\). Hence
\begin{equation*}
\Re\bigl(f(x)\bigr) = tx + 2\pi k_x \qquad (k_x \in \Z).
\end{equation*}
for all \(x\in\R\).
Assume by negation that \(k_w\neq 0\) for some \(w\in\R\).
Let 
\begin{equation} \label{eq:ex9.18:Z}
d = \frac{\Re\bigl(f(w)\bigr) - tw}{2\pi} \in \Z
\end{equation}
By \eqref{eq:ex9.18:Q} we have
\begin{equation*}
\Re\bigl(f(w/(2d))\bigr) = \Re\bigl(f(w)\bigr) \,\bigm/\, (2d)
\end{equation*}
and so
\begin{equation*}
\frac{\Re\bigl(f(w/(2d))\bigr) - tw/(2d)}{2\pi}
= \frac{\Re\bigl(f(w))\bigr) - tw}{4\pi d}
= \half \notin \Z
\end{equation*}
which is a contradiction to \eqref{eq:ex9.18:Z}.
Therefore \(\Re(f(x)) = tx\) and so \(\Re\circ f\) is continuous.

Similarly we can show that \(\Im\circ f\) is continuous, 
therefore $f$ is continuous.


\paragraph{Dense Graph}
If $f$ is \emph{not} continuous, then by what we just saw it is not linear
and we can find
\begin{equation*}
v_1 = \bigl(x_1,f(x_1)\bigr)
\qquad
v_2 = \bigl(x_2,f(x_2)\bigr)
\end{equation*}
such that \(\{v_1,v_2\}\) are linearly independent 
in the vector space \(\R^2\) over \R\ and so they span it.
Pick \((x,y)\in\R^2\), and let \(a_1,a_2\in\R\) be such that
\((x,y)=a_1v_1+a_2v_2\).
We can find two rational sequences \(\{q_{jk}\}_{k=1}^\infty\)
such that \(\lim_{k\to\infty} q_{jk} = a_j\) for \(j=1,2\).
Hence
\begin{equation*}
\lim_{k\to\infty} q_{1k}v_1 + q_{2k}v_2 = a_1v_1+a_2v_2 = (x,y).
\end{equation*}
Since \(q_{1k}v_1 + q_{2k}v_2\) are in the graph of $f$
it is dense in \(\R^2\).

\paragraph{Description of continuous functions.}
If $f$ is continuous, then since \(f(q) = q\cdot f(1)\) for all \(q\in\Q\)
by continuity we also have \(f(x) = x\cdot f(1)\) for all \(x\in\R\).
Hence the continuous functions that satisfy \eqref{eq:ex9.18}
are exactly the linear functions.


%%%%%%%%%%%%%% 19
\begin{excopy}
Suppose $A$ and $B$ are measurable subsets of \(\R^1\), 
having finite positive measure.
Show that the convolution \(\chhi_A \ast \chhi_B\) is continuous 
and not identically zero. Use this to prove that \(A+B\) contains a segment.

(A different proof was suggested in Exercise~5, Chap.~7.)
\end{excopy}

Put \(h = \chhi_A \ast \chhi_B\).
Fix \(\epsilon>0\) and by theorem~3.14, pick \(g\in C_c(\R)\)
such that \(\|\chhi_A - g\|_1 < \epsilon/m(B)\).
% Define \(g_a(x) = g(x+a)\) and cleary \(g_a\) are uniformly continuous.
Cleary $g$ is uniformly continuous.
Pick \(\delta>0\) such that
\(|g(s)-g(t)| < \epsilon/m(B)\) whenever \(|s-t|<\delta\).
If \(|s-t|<\epsilon/m(B)\) then
\begin{align*}
h(s) - h(t)
&= (\chhi_A \ast \chhi_B)(s) - (\chhi_A \ast \chhi_B)(t)  \\
&= \int \chhi_A(s-x)\chhi_B(x)\,dm(x) -
   \int \chhi_A(t-x)\chhi_B(x)\,dm(x) \\
&= \int \bigl(\chhi_A(s-x) - \chhi_A(t-x)\bigr)\chhi_B(x)\,dm(x) \\
&= \int_B \chhi_A(s-x) - \chhi_A(t-x)\,dm(x)
\end{align*}
Now
\begin{eqnarray*}
|h(s) - h(t)|
&=& \left| \int_B \chhi_A(s-x) - \chhi_A(t-x)\,dm(x) \right| \\
&\leq& \int_B |\chhi_A(s-x) - \chhi_A(t-x)|\,dm(x) \\
&\leq& 
   \int_B |\chhi_A(s-x) - g(s-x)|\,dm(x) +
   \int_B |g(s-x) - g(t-x)|\,dm(x) + \\
&& \int_B |g(t-x) - \chhi_A(t-x)|\,dm(x) \\
&\leq& 3\epsilon.
\end{eqnarray*}
Hence $h$ is (uniformly!) continuous.

%%%%%%%%%%%%%%%%%
\end{enumerate}

 %%%%%%%%%%%%%%%%%%%%%%%%%%%%%%%%%%%%%%%%%%%%%%%%%%%%%%%%%%%%%%%%%%%%%%%%
%%%%%%%%%%%%%%%%%%%%%%%%%%%%%%%%%%%%%%%%%%%%%%%%%%%%%%%%%%%%%%%%%%%%%%%%
%%%%%%%%%%%%%%%%%%%%%%%%%%%%%%%%%%%%%%%%%%%%%%%%%%%%%%%%%%%%%%%%%%%%%%%%
%chapter 10
\chapterTypeout{Elementary Properties of Holomorphic Functions}

\newcommand{\itwopi}{\frac{1}{2\pi}}
\newcommand{\itwopii}{\frac{1}{2\pi i}}

%%%%%%%%%%%%%%%%%%%%%%%%%%%%%%%%%%%%%%%%%%%%%%%%%%%%%%%%%%%%%%%%%%%%%%%%
%%%%%%%%%%%%%%%%%%%%%%%%%%%%%%%%%%%%%%%%%%%%%%%%%%%%%%%%%%%%%%%%%%%%%%%%
\section{Notes}

%%%%%%%%%%%%%%%%%%%%%%%%%%%%%%%%%%%%%%%%%%%%%%%%%%%%%%%%%%%%%%%%%%%%%%%%
\subsection{Theoerm 10.6 --- Power Series is Holomorphic}

In the proof of theorem~10.6 the following equality
\begin{equation} \label{eq:thm:10.6}
\left[ \frac{z^n - w^n}{z-w} - nw^{n-1} \right]
= (z-w)\sum_{k=1}^{n-1} kw^{k-1} z^{n-k-1}
\end{equation}
is used for \(n\geq 2\) and \(z\neq w\). Let's work it out.
In order to compute the fraction we start with
\begin{align*}
(z-w)\sum_{k=0}^{n-1} w^k z^{n-k-1}
&=  \left(\sum_{k=0}^{n-1} w^k z^{n-k}\right)
  - \left(\sum_{k=0}^{n-1} w^{k+1} z^{n-k-1}\right) \\
&=  \left(\sum_{k=0}^{n-1} w^k z^{n-k}\right)
  - \left(\sum_{k=1}^{n} w^{k} z^{n-k}\right) \\
&= z^n + \left(\sum_{k=0}^{n-1} \left(w^k z^{n-k} - w^k z^{n-k}\right)\right) - w^n\\
&= z^n - w^n
\end{align*}
Hence
\begin{equation} \label{eq:thm:10.6:frac}
\frac{z^n - w^n}{z-w} = \sum_{k=0}^{n-1} w^k z^{n-k-1}
\end{equation}

Finally,
\begin{align*}
(z-w)\sum_{k=1}^{n-1} kw^{k-1} z^{n-k-1}
&= \left(\sum_{k=1}^{n-1} kw^{k-1} z^{n-k}\right) -
   \left(\sum_{k=1}^{n-1} kw^{k} z^{n-k-1}\right) \\
&= \left(\sum_{k=0}^{n-2} (k+1)w^{k} z^{n-k-1}\right) -
   \left(\sum_{k=1}^{n-1} kw^{k} z^{n-k-1}\right) \\
&= (0-1)w^0z^{n-1} +
   \left(\sum_{k=1}^{n-2} (k_1-k)w^{k} z^{n-k-1}\right) -
   (n-1)w^{n-1}z^0 \\
&= z^{n-1} + \left(\sum_{k=1}^{n-2} w^kz^{n-k-1}\right) + w^{n-1} - nw^{n-1} \\
&= \left(\sum_{k=0}^{n-1} w^kz^{n-k-1}\right) - nw^{n-1} \\
&= \frac{z^n - w^n}{z-w} - nw^{n-1}
\end{align*}
We used \eqref{eq:thm:10.6:frac} in the last equality,
hence \eqref{eq:thm:10.6} holds.


%%%%%%%%%%%%%%%%%%%%%%%%%%%%%%%%%%%%%%%%%%%%%%%%%%%%%%%%%%%%%%%%%%%%%%%%
\subsection{Theoerm 10.7 --- Geometric Equality}

The proof of Theoerm 10.7 uses the following equality
\begin{equation*}
\sum_{n=0}^\infty \frac{(z - a)^n}{\left(\varphi(\zeta) - a\right)^{n+1}}
  = \frac{1}{\varphi(\zeta) - a}.
\end{equation*}
Let us derive it.
\begin{align*}
\sum_{n=0}^\infty \frac{(z - a)^n}{\left(\varphi(\zeta) - a\right)^{n+1}}
 &= \frac{1}{\varphi(\zeta) - a}
  \sum_{n=0}^\infty \left(\frac{z - a}{\varphi(\zeta) - a}\right)^n
  = \frac{1}{\varphi(\zeta) - a}
    \cdot
    \frac{1}{1 - \frac{z - a}{\varphi(\zeta) - a}} \\
 &= \frac{1}{\varphi(\zeta) - a) - (z - a)}
  = \frac{1}{\varphi(\zeta) - a}.
\end{align*}

%%%%%%%%%%%%%%%%%%%%%%%%%%%%%%%%%%%%%%%%%%%%%%%%%%%%%%%%%%%%%%%%%%%%%%%%
\subsection{Theoerm 10.10 --- Winding Number Equality}

The proof of Theoerm~10.0 defines
\begin{equation*}
 \varphi(t) = \exp\left\{\int_\alpha^t \frac{\gamma'(s)}{\gamma(s) - z}ds\right\}
\end{equation*}
and later claims that the derivative of \(\varphi/(\gamma - z)\)
is zero in \(V = [\alpha, \beta] \setminus S\) where $S$ is where \(\gamma\)
is not differentiable.
Let us show it in $V$.
Put
\begin{equation*}
h(t) = \int_\alpha^t \frac{\gamma'(s)}{\gamma(s) - z}ds
\end{equation*}
so \(\varphi(t) = \exp(h(t))\) and
\(h'(t) = \frac{\gamma'(t)}{\gamma(t) - z}\)
and \(\varphi'(t) = h'(t)\varphi(t)\).\\
Now \(\frac{d}{dt} \frac{\varphi(t)}{\gamma(t) - z} = N/D\)
where \(D=(\gamma(t) - z)^2\) and
\begin{equation*}
  N = \varphi'(t)(\gamma(t) - z) - \varphi(t)\gamma'(t) 
   = h'(t)e^{h(t)}(\gamma(t) - z) - e^{h(t)}\gamma'(t) 
    = e^{h(t)}\gamma'(t) - e^{h(t)}\gamma'(t) = 0.
\end{equation*}



%%%%%%%%%%%%%%%%%%%%%%%%%%%%%%%%%%%%%%%%%%%%%%%%%%%%%%%%%%%%%%%%%%%%%%%%
\subsection{Theoerm 10.15 --- Cauchy Fromula in Convex Region}

Let us work out the last derivation in the proof of theoerm~10.15.
\begin{equation*}
0 = \itwopii\int_\gamma g(\xi)\,d\xi
= \itwopii\int_\gamma \frac{f(\xi)-f(z)}{\xi - z}\,d\xi
\end{equation*}
Hence
\begin{equation*}
\itwopii\int_\gamma \frac{f(\xi)}{\xi - z}\,d\xi
= \frac{f(z)}{2\pi i}\int_\gamma \frac{d\xi}{\xi - z}\,d\xi
= f(z)\cdot\Ind_\gamma(z)
\end{equation*}


%%%%%%%%%%%%%%%%%%%%%%%%%%%%%%%%%%%%%%%%%%%%%%%%%%%%%%%%%%%%%%%%%%%%%%%%
\subsection{Theoerm 10.25 --- Inequality}

Let us show the inequality used in the proof of theoerm~10.25.

Given a polynomial \(P(x) = \sum_{j=0}^n a_nz^n\) where \(a_n=1\)
and
\begin{equation} \label{eq:thm10.25:r}
r > 1 + 2|a_0| + \sum_{j=1}^{n-1} |a_j|
\end{equation}
we need to show that
\begin{equation} \label{eq:thm10.25}
\left| P(re^{i\theta})\right| > |P(0)| = |a_0|.
\end{equation}
By \eqref{eq:thm10.25:r} we have
\begin{equation*}
r^n
> r^n + 2|a_0|r^n + \sum_{j=1}^{n-1} |a_j|r^n
> 2|a_0| + \sum_{j=1}^{n-1} |a_j|r^j
\end{equation*}
In the last inequality we used the fact that \(r>1\).
Hence
\begin{equation*}
r^n - \sum_{j=0}^{n-1} |a_j|r^n > |a_0|.
\end{equation*}
Now clearly
\begin{equation*}
\left| P(re^{i\theta})\right|
= \left| r^ne^{ni\theta} + \sum{j=1}^{n-1} a_j r^j e^{ji\theta}\right|
\geq r^n - \sum{j=1}^{n-1} |a_j| r^j > |a_0| = |P(0)|.
\end{equation*}
Hence \eqref{eq:thm10.25} is true.

%%%%%%%%%%%%%%%%%%%%%%%%%%%%%%%%%%%%%%%%%%%%%%%%%%%%%%%%%%%%%%%%%%%%%%%%
\subsection{Theoerm 10.26 --- Deriving the Estimation}

Let's finalize the derivation in the proof of theoerm~10.26.
By Theoerm~10.22 for
\begin{equation*}
f(z) = \sum_{n=0}^\infty c_n(z-a)^n
\end{equation*}
we have
\begin{equation*}
\sum_{n=0}^\infty |c_n|^2 r^{2n}
= \frac{1}{2\pi}\int_0^{2\pi} \left|f(a+re^{it})\right|^2\,d\theta \leq M^2
\end{equation*}
for any \(r<R\). Hence \(|c_n|r^n < M\) for all \(r<R\) and for all $n$.
Thus \(|c_n| < M/R^n\) for all $n$. Now clearly
\begin{equation*}
f^{(n)}(a) = n!c_n
\end{equation*}
and so
\begin{equation*}
\left|f^{(n)}(a)\right| = n!|c_n| \leq n!M/R^n.
\end{equation*}


%%%%%%%%%%%%%%%%%%%%%%%%%%%%%%%%%%%%%%%%%%%%%%%%%%%%%%%%%%%%%%%%%%%%%%%%
\subsection{Theoerm 10.28 --- Compact Argument}

In the proof of theoerm~10.28, we are given a compact \(K \subset \Omega\)
and the proof claims that there exists \(r>0\) such that
\begin{equation*}
E := \bigcup_{z\in K} \overline{D}(z;r)
\end{equation*}
is a compact subset of \(\Omega\).
Here is the justification.

Take the distance 
\begin{equation*}
d=d(K,\Omega^c)=\inf\{d(z,w):z\in K, w\in \C\setminus\Omega\}.
\end{equation*}
By compactness argument and the fat that \(\C\) is a Hausdorff space, \(d>0\).
Put \(r=d/2\) and
for each \(z\in K\) we pick \(r_z>0\) such that \(D(z,r) \subset \Omega\).
By compactness of $K$, there is a finite subcover
\begin{equation*}
E = \subset \bigcup_{z\in I} D(z,r) \subset \Omega.
\end{equation*}
where \(I\subset K\) is finite.


%%%%%%%%%%%%%%%%%%%%%%%%%%%%%%%%%%%%%%%%%%%%%%%%%%%%%%%%%%%%%%%%%%%%%%%%
\subsection{Lemma 10.29 --- Working out an Integral}

In the proof of Lemma~10.29.
we use the formula for integration oevr path devleopped
in section~10.8, namely for a smooth \(\gamma:[\alpha,\beta]\to\C\) path
\begin{equation*}
\int_\gamma f(x)\,dz = \int_\alpha^\beta f(\bigl(\gamma(t)\bigr)\gamma'(t)\,dt.
\end{equation*}

with \(\zeta(t) = (1-t)z+tw\) we can compute
\begin{align*}
f(w) - f(z)
&= f(\zeta(1)) - f(\zeta(0)) \\
&= \int_{\zeta([0,1])} f'(\xi)\,d\xi
 = \int_{\zeta([0,1])} f'(\zeta(t))\zeta'(t)\,dt
 = \int_{\zeta([0,1])} f'(\zeta(t))(w-z)\,dt \\
&= (w-z)\int_{\zeta([0,1])} f'(\zeta(t))\,dt
 = (w-z)\int_{\zeta([0,1])} f'(\zeta(t))\,dt
\end{align*}
Hence
\begin{equation*}
\int_0^1 \left[f'(\zeta(t)) - f'(a)\right]\,dt
= \int_0^1 f'(\zeta(t))\,dt - f'(a)
= \bigl((f(w) - f(z)\bigr)/(w-z) - f'(a).
\end{equation*}

It is tempting to try prroving this lemma directly without integration.
For this we need to have a local bound on \(f'\) which is given
by the fact that \(f^{(2)}\) exists and is continuous.
The idea is to show continuity at \((0,0)\) and then for sufficiently small
\(r>0\) and \(z,w\in D'(0,r)\) estimate
\begin{equation*}
\left|\frac{f(z)-f(w)}{z-w} - f'(0)\right|
\leq \left|\frac{f(z)-f(w)}{z-w} - f'(w)\right| + |f'(w) - f(0)|
\end{equation*}
and show that it can be small as desired.

%%%%%%%%%%%%%%%%%%%%%%%%%%%%%%%%%%%%%%%%%%%%%%%%%%%%%%%%%%%%%%%%%%%%%%%%
\subsection{Lemma 10.30 --- Adding Details}

The proof of theoerm~10.30 has several arguments that may use
clarifying details.

\paragraph{Setting $c$.} We can set
\begin{equation*}
c := \left|r\varphi'(z_0)\right|/5
\end{equation*}

\paragraph{Onto neighborhood.}
We pick \(\lambda\) such that
\begin{equation} \label{eq:thm10.30:lambda}
|\lambda - \varphi(a)| < c.
\end{equation}
Then after we show
\begin{equation*}
\min_\theta \left| \lambda - \varphi(a + re^{i\theta})\right|
\geq
  \left(\min_\theta \left| \varphi(a + re^{i\theta}) - \varphi(a)\right|\right)
  - |\varphi(a) - \lambda|
> 2c - c = c
\end{equation*}
the proof utilizes corollary of theoerm~10.24. Let's show how.
Set \(\psi(z) = \lambda - \varphi(z)\).
The corollary says that if
\(\psi(z)\) has no zeros in \(D(a;r)\) then
\begin{equation*}
|\psi(a)| = |\lambda - \varphi(a)|
\geq \min_\theta \left| \lambda - \varphi(a + re^{i\theta})\right|.
\end{equation*}
But the last inequality contradicts \eqref{eq:thm10.30:lambda}, hence
\(\psi\) has a zero in \(D(a;r)\).


%%%%%%%%%%%%%%%%%%%%%%%%%%%%%%%%%%%%%%%%%%%%%%%%%%%%%%%%%%%%%%%%%%%%%%%%
\subsection{Lemma 10.32 --- Differentiation}

Given \(g'/g = h'\) for some \(h\in H(\Omega)\).
Put \(\eta = g\cdot\exp(-h)\). Now
\begin{equation*}
\eta'
= g'\cdot\exp(-h) + g(-h)'\cdot\exp(-h)
= g'\cdot\exp(-h) + g(-g'/g)'\cdot\exp(-h)
= (g'-g')\cdot\exp(-h)
= 0.
\end{equation*}

%%%%%%%%%%%%%%%%%%%%%%%%%%%%%%%%%%%%%%%%%%%%%%%%%%%%%%%%%%%%%%%%%%%%%%%%
\subsection{Lemma 10.32 --- Details}

The term
\index{deleted neighborhood}
\index{deleted!neighborhood}
\emph{deleted neighborhood} of \(z_0\) means \(V\setminus\{z_0\}\)
where $V$ is a neighborhood of \(z_0\).

Tracing the proof. Assuming by negation \(f'(z_0) = 0\).
By theoerm~10.32 for a neighborhood $V$ of \(z_0\) we have
\begin{equation*}
f(z) = f(z_0) + \bigl(\varphi(z)\bigr)^m
\qquad \textnormal{where}\;\varphi\in H(V),\; m > 1.
\end{equation*}
and \(\varphi\) is \emph{onto} some \(D(0;r)\).
So if we look at inverse image
\begin{equation*}
A := \{\varphi^{-1}(\rho e^{2\pi k/m}\} \qquad \textnormal{for}\; \rho < r.
\end{equation*}
Now \(|A|=m\) and \(f(A)\) has one value, thus $f$ is $m$-to-$1$.
This contradiction shows that \(f'(z_0) \neq 0\).

%%%%%%%%%%%%%%%%%%%%%%%%%%%%%%%%%%%%%%%%%%%%%%%%%%%%%%%%%%%%%%%%%%%%%%%%
\subsection{Theoerm 10.37 --- Details}

There is a typo in the end of the stated Theorem.
Instead of
\begin{quote}
\textsl{\ldots\ if \(x\in D_-\) \ldots}
\end{quote}
It should be
\begin{quote}
\textsl{\ldots\ if \(z\in D_-\) \ldots}
\end{quote}

There are illustrations of the paths defined in the proof 
in \figurename{\ref{fig:10-37}}.
\begin{figure}[ht]
 % \centering
 % \captionsetup[subfloat]{nearskip=-3pt}
%
\subfloat[\(C(s)\) and \(\gamma(s)\)]{%
\begin{minipage}[b]{0.4\textwidth}
\centering
\input{10-37-a}
\end{minipage}
}
%
\hspace{0.1\textwidth}
%
\subfloat[\(f(s)\)]{%
\begin{minipage}[b]{0.4\textwidth}
\centering
\input{10-37-b}
\end{minipage}
}
%
\\[20pt]%
%
\subfloat[\(g(s)\)]{%
\begin{minipage}[b]{0.4\textwidth}
\centering
\input{10-37-c}
\end{minipage}
}
%
\hspace{0.1\textwidth}
%
\subfloat[\(h(s)\)]{%
\begin{minipage}[b]{0.4\textwidth}
\centering
\input{10-37-d}
\end{minipage}
}
%
\caption{Paths in the Proof of Theorem 10.37}
\label{fig:10-37}
\end{figure}


%%%%%%%%%%%%%%%%%%%%%%%%%%%%%%%%%%%%%%%%%%%%%%%%%%%%%%%%%%%%%%%%%%%%%%%%
\subsection{Lemma 10.39 --- Details}

Given \(\gamma = (\gamma_1 - \alpha) / (\gamma_0 - \alpha)\)
perfrom the computation:
\begin{equation*}
\frac{\gamma'}{\gamma}
= \frac{\gamma_1'(\gamma_0-\alpha) - (\gamma_1-\alpha)\gamma_0'}{
               (\gamma_0 - \alpha)^2}
   \cdot \frac{\gamma_0 - \alpha}{\gamma_1 - \alpha}
= \frac{\gamma_1'(\gamma_0-\alpha) - (\gamma_1-\alpha)\gamma_0'}{
               (\gamma_1 - \alpha)(\gamma_0 - \alpha)}
= \frac{\gamma_1'}{\gamma_1-\alpha} - \frac{\gamma_0'}{\gamma_0-\alpha}.
\end{equation*}

%%%%%%%%%%%%%%%%%%%%%%%%%%%%%%%%%%%%%%%%%%%%%%%%%%%%%%%%%%%%%%%%%%%%%%%%
%%%%%%%%%%%%%%%%%%%%%%%%%%%%%%%%%%%%%%%%%%%%%%%%%%%%%%%%%%%%%%%%%%%%%%%%
\section{Edition 2 (Old) Exercises} % pages 193-195

Some exercises in edition~2 do not appear or are different than in edition~3.
We bring some of them here

%%%%%%%%%%%%%%%%%
\begin{enumerate}
%%%%%%%%%%%%%%%%%

\iffalse
%%%%%%%%%%%%%% 1
\begin{excopy}
   If $A$ and $B$ are disjoint subsets of the plane, if $A$ is compact,
   and if $B$ is closed, then there exists
   a~\(\delta > 0\) such that \(|\alpha-\beta| \geq 0\) for all
   \(\alpha \in A\) and \(\beta \in B\). Prove this with an arbitrary
   metric space in place of the plane.
\end{excopy}
  Let $d$ be the metric. Define
  \[G_n = \{x \in A: d(x,B) > 1/n\}\]
  for all \(n>0\). It is clear that for all \(x\in A\) \(d(x,B)>0\)
  and so \(\cup G_n = A\). Since A is compact there exists $m$ such that
  \(\cup_{n=1}^m G_n = A\).
  Hence, for all \(x\in A\) \(d(x,B)>1/m\)
  Put \(\delta=1/m\) that satsify the requirement.
\fi

\setcounter{enumi}{1}
%%%%%%%%%%%%%% 2
\begin{excopy}
    At the end of section~10.8 occurs the definition of the length of a path
    \(\gamma\) as
    \[\int_\alpha^\beta|{\gamma^\prime}(t)|dt.\]
    Does this agree with the definitions given in Exercise~10, Chapter 8?
    Length of a graph of~$f$ is
       \[f_s(1) + \int_0^1 \sqrt{1 + |f^\prime(t)|^2}dt.\]
\end{excopy}

\text{Note:} This is of course referring to exercise in
edition~2 (\cite{RudinRCA80}).
In Edition~3, it appears in chapter~7, exercise~21.
Actually in this exercise we have shown in \eqref{eq:ex7.21:Sdif}
what we need here. There we used the result for a specific case
of \(x(t)=t\).

%%%%%%%%%%%%%%
\end{enumerate}


%%%%%%%%%%%%%%%%%%%%%%%%%%%%%%%%%%%%%%%%%%%%%%%%%%%%%%%%%%%%%%%%%%%%%%%%
%%%%%%%%%%%%%%%%%%%%%%%%%%%%%%%%%%%%%%%%%%%%%%%%%%%%%%%%%%%%%%%%%%%%%%%%
\section{The Exercises} % pages 227-230

Exercises of \emph{current} 3rd edition.

%%%%%%%%%%%%%%%%%
\begin{enumerate}
%%%%%%%%%%%%%%%%%

%%%%%%%%%%%%%% 01
\begin{excopy}
The following fact was tacitly used in this chapter:
If $A$ and $B$ are disjoint subsets of the plane,
if $A$ is compat and $B$ is closed, then there exists a \(\lambda > 0\)
such that \(|\alpha - \beta| \geq \lambda\) for all
\(\alpha \in A\) and \(\beta \in B\).
Prove this, with an arbitrary metric space in place of the plane.
\end{excopy}

Let \((X,d)\) be a matric space and $A$ and $B$ as described in $X$.
For each \(x\in A\) pick some \(\lambda_x>0\) such that
\(D(x;\lambda_x) \cap B = \emptyset\). Such \(\delta_x\) always exists
since otherwise \(x \in \overline{B}=B\).

Clearly \(\{D(x;\lambda_x)\}_{x\in A}\) is an open covering of the compact $A$,
hence there is a finite set \(F\subset X\) such that
\begin{equation*}
 X \subset \bigcup_{x\in F} D(x;\lambda_x).
\end{equation*}
Hence we can pick a \(\delta = \min(\{\delta_x: x\in F\}) > 0\)
that satisfies the requirement.


%%%%%%%%%%%%%% 02
\begin{excopy}
Suppose that $f$ is an entire function,
and that in every  power series
\begin{equation*}
f(z) = \sum_{n=0}^\infty c_n(z-a)^n
\end{equation*}
at least one coefficient is $0$. Prove that $f$ is a polynomial.
\\ \emph{Hint:} \(n!\,c_n = f^{(n)}(a)\).
\end{excopy}

For each \(a\in D(0,1)\) let \(n_a\) be the minimal \(n\in\Z^+\) such that
\(c_n=0\) for the power series representation around $a$.
By cardinality argument \(\|D(0,1)|=2^{\aleph_0} > \aleph_0 = \Z^+\)
there is some \(m\in\Z^+\) and infinite subset \(A\subset D(0,1)\)
such that \(m=n_a\) for each \(a\in A\).
But then \(f^{(m)}\) has infinite zeros in \(D(0,1)\)
and hence \(f^{(m)}(z) = 0\) for all \(z\in\C\),
hence $f$ is a polynomial of degree  \(<m\).

%%%%%%%%%%%%%% 03
\begin{excopy}
If $f$ and $g$ are entire function, and \(|f(z)| \leq |g(z)|\)
for every $z$. What conclusions can you draw?
\end{excopy}

\textbf{Claim:}
Under these conditions, \(f(z) = ag(z)\) for all \(z\in\C\) for some
\(a\in\C\) such that \(a\leq 1\).

If \(g = 0\) then so is \(f = 0\) and we are done.

We claim that \(q(z) = f(z)/g(z)\) for \(z\notin Z(g)\)
can be always be extended to an entire function.

So we may now assume that $g$ is not constantly zero, and by theoerm~10.18
\(Z(g)\) has no limit point.
Assume \(g(w) = 0\)
then bt theoerm~10.18 we have the following representations
\begin{equation*}
f(z) = (z-w)^m\tilde{f}(z)
\qquad
g(z) = (z-w)^n\tilde{g}(z)
\end{equation*}
and \(\tilde{f}(w) \neq 0 \neq \tilde{f}(w)\).
If by negation \(m < n\) then for suffuciently small \(\delta>0\)
\begin{equation*}
\delta^{n-m} < \bigl|\ \tilde{f}(z+\delta) / \tilde{g}(z+\delta)\bigr|
\end{equation*}
but then \(|f(z+\delta)| > |g(z+\delta)|\) which is a contradiction.
Hence by theoerm~10.20 \(q=f/g\) has a removable singularity at $z$
and
\begin{equation*}
q(z) = \left\{\begin{array}{ll}
\tilde{f}(z)/\tilde{g}(z) \qquad& m=n \\
0 & m > n
\end{array}\right.
\end{equation*}

Similarly $q$ can be defined for all \(Z(f)\),
hence we can view it as an entire function. By the given condition,
$q$ is bounded. By
\index{Lioville}
Lioville's theoerm~10.23 $q$ is constant. Hence \(f(z)= q(0) g(z)\).


%%%%%%%%%%%%%% 04
\begin{excopy}
Suppose that $f$ is an entire function, and
\begin{equation*}
|f(x)| \leq A + B|z|^k
\end{equation*}
for all $z$, where $A$, $B$, and $k$ are positive numbers.
Prove that $f$ must be a polynomial.
\end{excopy}

Let \(f(z) = \sum c_nz^n\). By
\index{Lioville}
theoerm~10.22 for any \(r\geq 0\) we have
\begin{equation*}
\sum_{n=0}^\infty |c_n|^2 r^{2n}
= \frac{1}{2\pi}\int_{-\pi}^\pi \left| f(re^{i\theta})\right|^2\,d\theta
\leq \frac{1}{2\pi} (2\pi)\cdot(A+Br^k)^2
= A^2+2ABr^k + B^2r^{2k}
\end{equation*}
Clearly \(c_n = 0\) whenever \(n > 2k\), otherwise the above inequality
would fail. Hence $f$ is a polynomial.

%%%%%%%%%%%%%% 05
\begin{excopy} 
Suppose 
\label{ex:fn:uniform}
\(\{f_n\}\) is a uniformly bounded sequence of holomorphic function
in \(\Omega\) such that  \(\{f_n(z)\}\) converges for every \(z\in \Omega\).
Prove that the convergence is uniform on every compact subset of \(\Omega\).
\\ \emph{Hint:} Apply the dominated convergence theorem to the Cauchy formula
for \(f_n - f_m\).
\end{excopy}

Define the pointwise limit \(f(z) = \lim_{n\to\infty} f_n(z)\).
Since \(\{f_n\}\) is a uniformly bounded, we have
\begin{equation*}
M := \sup_n \|f_n\|_\infty < \infty.
\end{equation*}

Let us temporarily restrict the function to some disc
\(\overline{D}(a;r/2)\subset D(a;r) \subset\Omega\).
Let \(\gamma(t) = a+re^{it}\), hence \(\gamma^* = \partial(D(a;r))\).
By the Cauchy formula
\begin{equation*}
\left| \frac{f_n(w)}{w - z}\right| \leq 2M/r
\qquad (z\in \overline{D}(a;r/2),\, w \in \gamma^*)
\end{equation*}
For abbreviation, let \(w_t = a+re^{it}\).
Now for \(z\in \overline{D}(a;r/2)\) we have
\begin{align*}
|f_m(z) - f_n(z) 
&\leq \frac{1}{2\pi} \int_{\gamma^*} \frac{|f_m(w)-f_n(w)}{|w-z|}\,d|w|
 \leq \frac{1}{\pi r}  \int_0^{2\pi} |f_m(w_t)-f_n(w_t)|\,dt \\
& \leq \frac{1}{\pi r}  
       \left(\int_0^{2\pi} |f_m(w_t)-f(w_t)|\,dt +
             \int_0^{2\pi} |f_n(w_t)-f(w_t)|\,dt\right)
\end{align*}
The last integrands are each dominated by \(2M\).
By Lebesgue's dominated convergence theorem~1.34,
for each \(\epsilon>0\) there exists $N$ such that 
the last 2-intergrals expression is \(<\epsilon\)
if \(m,n\geq N\), for \emph{any} \(z\in \overline{D}(a;r/2)\).
Hence \(\{f_n(z)\}\) converges uniformly on \(\overline{D}(a;r/2)\).

Any compact \(K\subset\Omega\) can be covered by finite set
of such \(\{\overline{D}(a_j;r/2): j\in J\}\) closed discs
with \(|J|<\infty\) and \(D(a_j;r)\subset \Omega\).

Returning to the original definitions of \(\{f_n\}\) on \(\Omega\),
the  \(\{f_n(z)\}\) converges uniformly on each closed disc, 
and therefore the sequence converges uniformly on a finite union of them.
Hence  \(\{f_n(z)\}\) converges uniformly on $K$.

%%%%%%%%%%%%%% 06
\begin{excopy}
There is a region \(\Omega\) that \(\exp(\Omega) = D(1;1)\).
Show that \(\exp\) is one-to-one in \(\Omega\),
but that there are many such \(\Omega\). Fix one, and define
\(\log z\), for \(|z-1|<1\), to be that \(w\in\Omega\) for which \(e^w = z\).
Prove that \(\log'(z) = 1/z\). Find the coefficients \(a_n\) in
\begin{equation*}
\frac{1}{z} = \sum_{n=0}^\infty a_n(z-1)^n
\end{equation*}
and hence the coefficients \(c_n\) in the expansion
\begin{equation*}
\log z = \sum_{n=0}^\infty c_n(z-1)^n.
\end{equation*}
In what other discs can this be done?
\end{excopy}

For each \(z\in D(1;1)\) there exists a unique \(\theta\in (-\pi,\pi)\) 
and \(r>0\) such that \(z=re^{i\theta}\).
Let 
\begin{equation*}
\Omega = \left\{\log(r) + i\theta:  
           re^{i\theta} \in D(1;1) \;\wedge\; \theta\in (-\pi,\pi)\right\}.
\end{equation*}
Clearly \(\Omega_n = \{z + 2\pi ni: z\in \Omega\}\)
can be a region as required for all \(n\in\Z\).

Now with the above definion of \(\log\), we have
\begin{equation*}
(\exp\circ\log)(z) = \Id_\Omega(z) \qquad (z\in\Omega).
\end{equation*}
Hence for \(z\in\Omega\) we have 
\begin{equation*}
1 
= \log'(z)\cdot (\exp'\circ \log)(z)
= \log'(z)\cdot (\exp\circ \log)(z) 
= \log'(z)\cdot z.
\end{equation*}
Thus \(\log'(z) = 1/z\) for \(z\in \Omega\).

\paragraph{Computing coefficients.}
Differentiation of \(1/z\) gives
\begin{equation*}
\frac{d^n(z^{-1})}{dz^n} = 
% z^{-1}, -z^{-2}, -2z^{-3}, 6z^{-4}, -24z^{-5}
 (-1)^n \cdot n!\cdot z^{-n-1}
\end{equation*}
Hence, the coefficients of the power series of \(1/z\) around $1$ are
\begin{equation*}
a_n = \frac{d^n(z^{-1})}{dz^n}(z=1)/n!
=  (-1)^n \cdot n!\cdot 1^{-n-1} / n! = (-1)^n
\end{equation*}

The coefficients of the ``primitive'' \(\log(z)\) are:

This can be done on any open disc that does contains the zero.
\begin{align*}
c_0 &= 0 \\
c_n &= -(-1)^n/n \qquad (n > 0)
\end{align*}
That is \((c_n)=0,1,-1/2,1/3,-1/4,\ldots\).

%%%%%%%%%%%%%% 07
\begin{excopy}
If \(f\in H(\Omega)\), the Cauchy formula for the derivatives of $f$,
\begin{equation*}
f^{(n)}(z)
= \frac{n!}{2\pi i} \int_\Gamma \frac{f(\xi)}{(\xi - z)^{n+1}}\,d\xi
\qquad (n=1,2,3,\ldots)
\end{equation*}
is valid under conditions on $z$ and \(\Gamma\).
State these and prove the formula.
\end{excopy}

Sufficent conditions are:
\begin{itemize}
\item \(\Gamma\) is a cycle in \(\Omega\).
\item \(\Ind_\Gamma(\alpha)=0\) for every \(\alpha\in \C\setminus\Omega\).
\item \(z\notin \Gamma^*\).
\item \(\Ind_\Gamma(z)=1\).
\end{itemize}
By Cauchy's formula theorem~10.15 we have
\begin{equation*}
f(z) = \itwopii\int_\Gamma \frac{f(w)}{w-z}\,dw.
\end{equation*}
This shows the desired formula holds for \(n=0\).
By induction assumes that it holds for \(n=k\), that is
\begin{equation*}
f^{(k)}(z)
= \frac{k!}{2\pi i} \int_\Gamma \frac{f(w)}{(w - z)^{k+1}}\,dw
\end{equation*}
Now
\begin{align}
f^{(k+1)}(z)
&= \lim_{h\to 0} \left(f^{(k)}(z+h)- f^{(k)}(z+h)\right)/h \notag \\
&= \lim_{h\to 0} \frac{k!}{2\pi i} 
   \left(
    \int_\Gamma 
    f(w)
    \left(
       \frac{1}{(w - (z+h))^{k+1}} - \frac{1}{(w - z)^{k+1}}
    \right)\,dw  
    \right) \bigm/ h
    \notag \\
&= \frac{k!}{2\pi i} \int_\Gamma \lim_{h\to 0} 
    f(w)
    \left(
       \frac{1}{(w - (z+h))^{k+1}} - \frac{1}{(w - z)^{k+1}}
    \right)\bigm/h\,dw  \label{eq:cauchy:diflim} \\
&= \frac{k!}{2\pi i} \int_\Gamma 
    f(w) \frac{d\,\left((w-z)^{-(k+1)}\right)}{dz}\,dw  \notag \\
&= \frac{k!}{2\pi i} \int_\Gamma f(w)(k+1)(w-z)^{-(k+2)}\,dw \notag \\
&= \frac{(k+1)!}{2\pi i} \int_\Gamma \frac{f(w)}{(w - z)^{k+2}}\,dw \notag
\end{align}
The justification of \eqref{eq:cauchy:diflim} is based on the fact
that \(\Gamma^*\) is compact it is it sufficient to use
sequences as limits and fintally utilize exercise~\ref{ex:fn:uniform} above.

%%%%%%%%%%%%%% 08
\begin{excopy}
Suppose $P$ and $Q$ are polynomials, the  degree of $Q$ exceeds that of $P$
by at least $2$,
and the rational function \(R = P/Q\) has no pole on the real axis.
Prove that the integral of $R$ over \((-\infty,\infty)\)
is \(2\pi i\) times the sum of the residues of $R$ in the upper half plane.
[Replace the integral over \((-A,A)\)  by one over a suitable semicircle,
and apply the residue theorem.] What is the analogous statement for the lower
half plane? Use this method to compute
\begin{equation*}
 \int_{-\infty}^\infty \frac{x^2}{1+x^4}\,dx.
\end{equation*}
\end{excopy}

Let \(a>0\) be such that all the roots of $Q$ are in \(D(0;a)\).
For each \(A>a\) consider the closed path 
\(\gamma_A\: [-A, A+\pi \to\C\) that is defined as follows
\begin{equation*}
\Gamma_A(t) = \left\{%
\begin{array}{ll}
t & t \leq A \\
Ae^{i(t-A)} \quad & t \geq A
\end{array}\right.
\end{equation*}
Now define the upper arc \(U_A=\{z\in\C: |z|=A \wedge \Im(z)>0\}\) and
\begin{equation*}
I(A) 
= \int_U P(z)/Q(z)\,dz 
= \int_0^\pi P(e^{iAt})/Q(e^{iAt})\frac{dz}{dt}\,dt
= iA\int_0^\pi P(e^{iAt})e^{iAt}/Q(e^{iAt})\frac{dz}{dt}\,dt
\end{equation*}
Hence \(|I(A)| \leq \pi A \sup_{z\in A} |P(e^{iAt})/Q(e^{iAt})|\).
Since \(\deg(P) \leq \deg(Q)+2\) we have
\begin{equation*}
\lim_{A\to\infty} I(A) = 0.
\end{equation*}
\index{residue theorem}
The set of poles of $R$ is exactly the set \(Z_Q\) of zeros of $Q$.
We define 
\begin{equation*}
Z_{Q^+} = \{z\in Z_Q: \Im(z)>0\}
\qquad
Z_{Q^-} = \{z\in Z_Q: \Im(z)<0\}
\end{equation*}
By the residue theorem~10.42.
\begin{align*}
\int_{-\infty}^\infty R(x)\,dx 
&= \lim_{A\to\infty} 
   \left(\int_{\Gamma_A}  R(z)\,dz - \int_{U_A} P(z)/Q(z)\,dz\right) \\
&= \lim_{A\to\infty} \int_{\Gamma_A}  R(z)\,dz 
   - \lim_{A\to\infty} \int_{U_A} P(z)/Q(z)\,dz 
 = \lim_{A\to\infty} \int_{\Gamma_A}  R(z)\,dz \\
&= 2\pi i \sum_{z\in Z_{Q^+}} \Res(R;z).
\end{align*}

The analogous statement for \(Z_{Q^-}\) is
\begin{equation*}
\int_{-\infty}^{\infty} R(x)\,dx 
= -\int_{\infty}^{-\infty} R(x)\,dx 
= 2\pi i \sum_{z\in Z_{Q^-}} \Res(R;z).
\end{equation*}
Hence if all the zeros if \(Q(z)\) are on one side, then the integral
over the real line is zero. 
Note that if all the coefficients of \(Q(z)\) are real,
then \(Q(z)=0\) iff \(Q(\overline{z})=0\).

The set of zeros of \(x^4+1\) is \(\{e^{(2k+1)\pi i/4}: k=0,1,2,3\}\).
Hence the zeros above the real line are \(\{e^{\pi i/4}, e^{3\pi i/4}\}\)
or \(\{q,qi\}\) where \(q=e^{\pi i/4}=(1+i)\sqrt{2}/2\).
Now \(Q(z)=\prod_{k=0^3} (z-qi^k)\) hence
\begin{equation*}
\Res(R;q_j) = q^2/ \prod_{0\leq k\leq3 \wedge k\neq j} (z-qi^k) \qquad (j=0,1,2,3)
\end{equation*}
Applying it gives
\begin{align*}
 \int_{-\infty}^\infty \frac{x^2}{1+x^4}\,dx
 &= 2\pi i \sum_{z\in Z_Q} \Res(R;z)
  = 2\pi i \left( \Res(R; e^{\pi i/4}) + \Res(R; e^{\pi i/4})\right) \\
 &= 2\pi i \left( q^2/\bigl((q-qi)(q+q)(q+qi)\bigr) 
                 -q^2/\bigl((qi-q)(qi+q)(qi+qi)\bigr)\right) \\
 &= (2\pi i/q) \cdot
    \left(\frac{1}{(1-i)2(1+i)} - \frac{1}{(-1+i)(1+i)2i}\right) \\
 &= (\pi i/q)  \left(\frac{1}{2}  - \frac{1}{-2i}\right) 
  = \pi q (1-i)/2 
  = \sqrt{2}\pi(1+i)(1-i)/4 \\
 &= \sqrt{2}\pi/2 \simeq 2.2214414690791831
\end{align*}


%%%%%%%%%%%%%% 09
\begin{excopy}
Compute \(\int_{-\infty}^\infty e^{-itx}/(1+x^2)\,dx\) for real $t$,
by methods described on Exercise~8.
Check your answer against the inversion theorem for Fourier transforms.
\end{excopy}

Put \(f_t(z) = e^{-itz}/(1+z^2)\). 
Assume first that \(t\leq 0\). 
If \(\Im(z)\geq 0\) then \(\Re(-itz) \leq 0\) and so \(|e^{-itz}|\geq 1\).
% Thus \(|e^{-itz}/(1+z^2)| \leq 1/z^2\).
Consider the 
 upper plane half circle with radius \(r>1\) defined by:
\begin{equation*}
D_r = \{z\in\C: |z|\geq r \wedge \Im z \geq 0\}.
\end{equation*}
Integration over its boundary gives:
\begin{equation*}
\int_{\partial D_r} e^{-itz}/(1+z^2)\,dz 
= (2\pi i) \Res(e^{-itz}/(1+z^2); i)
= (2\pi i) e^{-iti}/(i-(-i)) = \pi e^t
\end{equation*}
On the arc part of \(D_r\) 
we have \(|e^{-itz}/(1+z^2)| \leq |1/z^2| = 1/r^2\).
Hence
\begin{equation*}
\int_{-\infty}^\infty f_t(x)\,dx
= \lim_{r\to\infty} \int_{\partial D_r} f_t(z)\,dz 
= \Res(e^{-itz}/(1+z^2); i) 
= \pi e^t
\end{equation*}
Since \(f_t(x) = f_{-t}(-x)\) for any \(t\in\R\) we have
\(\int_{-\infty}^\infty f_t(x)\,dx = \int_{-\infty}^\infty f_{-t}(x)\,dx\).
Therefore
\begin{equation*}
\int_{-\infty}^\infty e^{-itx}/(1+x^2)\,dx = \pi e^{-|t|} \qquad(t\in\R).
\end{equation*}

\paragraph{Fourier Inversion.}
\begin{align*}
\int_{-\infty}^\infty e^{itx} e^{-|t|}\,dt
&= \int_{-\infty}^0 e^{itx} e^{t}\,dt  + \int_0^\infty e^{itx} e^{-t}\,dt
 = \int_{-\infty}^0 e^{(1+ix)t}\,dt  + \int_0^\infty e^{(-1+ix)t}\,dt \\
&=   \left(e^{(1+ix)t}/(1+ix)\right)\biggm|_{t=-\infty}^0
   + \left(e^{(-1+ix)t}/(-1+ix)\right)\biggm|_{t=0}^\infty \\
&= e^{(1+ix)0}/(1+ix) - e^{(-1+ix)0}/(-1+ix) 
 = 1/(1+ix) - 1/(-1+ix) \\
&= 2/(x^2+1)
\end{align*}
Using \(\pi e^{-|t|}\) instead of \(e^{-|t|}\) 
and factoring with \(1/\sqrt{2\pi}\) provides 
the expected equality with the original \(1/(1+x^2)\) function.



%%%%%%%%%%%%%% 10
\begin{excopy}
Let \(\gamma\) be a poitively oriented unit circle, and compute
\begin{equation*}
\itwopii \int_\gamma \frac{e^z - e^{-z}}{z^4}\,dz
\end{equation*}
\end{excopy}

Using the residue theorem~10.42 and Taylor expansion
\begin{align*}
\itwopii \int_\gamma \frac{e^z - e^{-z}}{z^4}\,dz
&= \itwopii \int_\gamma z^{-4}\left(\sum_{n=0}^\infty (z^n - (-z)^n)/n!\right)\,dz \\
&= \itwopii \int_\gamma 2z^{-4}\left(\sum_{k=0}^\infty z^{2k+1}/(2k+1)!\right)\,dz \\
&= 2\cdot\Res\left(\sum_{k=0}^\infty z^{2k-3}/(2k+1)!;\,0\right)
 = 2\cdot\Res\left(z^{2\cdot 1-3}/(2\cdot 1+1)!;\,0\right) = \\
&= 2/3! = 1/3
\end{align*}


%%%%%%%%%%%%%% 11
\begin{excopy}
Suppose \(\alpha\) is a complex number, \(|\alpha|\neq 1\), and compute
\begin{equation*}
\int_0^{2\pi} \frac{d\theta}{1 - 2\alpha \cos\theta + \alpha^2}
\end{equation*}
by integrating \((z-\alpha)^{-1}(z-1/\alpha)^{-1}\) over the unit circle.
\end{excopy}

Putting \(z=e^{i\theta}\) (or \(\theta = -i\log(z)\))
then \(dz/d\theta = iz\) and \(d\theta/dz = -i/z\)
and when \(|z|=1\)
then \(\bar{z}=1/z\) and \(\cos \theta = (z+1/z)/2\).
Hence
\begin{align*}
\int_0^{2\pi} \frac{d\theta}{1 - 2\alpha \cos\theta + \alpha^2}
&= -i \int_{|z|=1} \bigl(1 - \alpha(z+1/z) + \alpha^2\bigr)^{-1} \bigm/ z\,dz \\
&= -i \int_{|z|=1} \bigl(z - \alpha(z^2+1) + z\alpha^2\bigr)^{-1}\,dz \\
&= (i/\alpha)  \int_{|z|=1} \left(z^2 - (\alpha+1/\alpha)z+1\right)^{-1}\,dz \\
&= (i/\alpha)  \int_{|z|=1} (z-\alpha)^{-1}(z-1/\alpha)^{-1}\,dz \\
&= (2\pi i^2/\alpha) \cdot \left\{%
   \begin{array}{ll}
   (1/\alpha - \alpha) \quad & \alpha > 1 \\
   (\alpha - 1/\alpha) \quad & \alpha < 1 \\
   \end{array}\right. \\
&= 2\pi / |\alpha^2 - 1|
\end{align*}


%%%%%%%%%%%%%% 12
\begin{excopy}
Compute
\begin{equation*}
\int_{-\infty}^\infty \left(\frac{\sin x}{x}\right)^2 e^{itx}\,dx
\qquad (\textnormal{for real }\; t).
\end{equation*}
\end{excopy}

Using Python and MatPlotLib, I have a conjecture, that the integral equals:
\begin{equation*}
\left\{\begin{array}{ll} %
\pi(2-|t|)/4 \qquad & |t| \leq 2 \\
0 & |t| \geq 2
\end{array}\right.
\end{equation*}

We follow the ideas of section~10.44.

The function \((\sin z/z)^2 e^{itz}\) has a removable singularity at \(z=0\)
and can be defined there as $1$ and be considered an entire function.
Since \(2i \sin z = e^{iz} - e^{-iz}\) we have
\begin{equation*}
-4\sin^2 z = \left(e^{iz} - e^{-iz}\right)^2 = e^{2iz} + e^{-2iz} - 2\,.
\end{equation*}
Thus
\begin{equation*}
(\sin z/z)^2 e^{itz} = -e^{i(t+2)z}/4z^2 - e^{i(t-2)z}/4z^2 + e^{itz}/2z^2\,.
\end{equation*}

Consider the path \(\Gamma_A\) from \(-A\) to \(-1\)
then lower unit circle and finally from $1$ to $A$.
Put
\begin{align*}
% \frac{1}{\pi}\varphi_A(s) = \itwopii \int_{\Gamma_A} \frac{e^{isz}}{z}\,dz\,.
\psi_s(z) &= \frac{e^{isz}}{z^2} \\
\varphi_A(s) 
  &= \int_{\Gamma_A} \frac{e^{isz}}{z^2}\,dz
  = \int_{\Gamma_A} \psi_s(z)\,dz\,.
  % = \int_{-A}^A \psi_s(x)\,dx\,. Not equal 
\end{align*}
Now
\begin{equation} \label{eq:10.12:sin2phis}
\int_{-A}^A \left(\frac{\sin x}{x}\right)^2 e^{itx}\,dx
% = -\varphi_A(t+2)/4 - \varphi_A(t-2)/4 + (1/2)\int_{\Gamma_A} z^{-2}\,dz
= \varphi_A(t)/2 - \bigl(\varphi_A(t+2)/4 - \varphi_A(t-2)\bigr)/4 
\end{equation}

We complete \(\Gamma_A\) into a closed path in two ways.
\begin{itemize}
\item With the lower half circle of radius $A$. Inside this closed path, 
      \(\psi_s(z)\) is analytic.
\item With the upper half circle of radius $A$. Inside this closed path, 
      \(\psi_A(z)\) has one pole at \(z=0\)
      with residue \(is\) (consider \(((isz)^1/1!)/z^2 = is/z\)).
\end{itemize}

For both path closing options, we use the substitution \(z = Ae^{i\theta}\)
and \(dz/d\theta = iAe^{i\theta}\)

Using the first choice of path closing gives:
\begin{align}
\varphi_A(s) 
&= \int_{|z|=A \wedge \Im(z)<0} \psi_s(z)\,dz
 = \int_{-\pi}^0 \psi_s(Ae^{i\theta}) \cdot iAe^{i\theta} \,d\theta \notag \\
&= (i/A) \int_{-\pi}^0 \exp(isAe^{i\theta})/e^{i\theta}\,d\theta.
   \label{eq:ex10.12:lowcirc}
\end{align}
Using the second choice with the residue in the pole gives:
\begin{align}
\varphi_A(s) 
&= 2\pi i\cdot \Res(\psi_s;0) - \int_{|z|=A \wedge \Im(z)>0} \psi_s(z)\,dz
 = -2\pi s - \int_0^\pi iAe^{i\theta} \psi_s(Ae^{i\theta})\,d\theta \notag \\
&= -2\pi s - (i/A) \int_{-\pi}^0 \exp(isAe^{i\theta})/e^{i\theta}\,d\theta.
   \label{eq:ex10.12:upcirc}
\end{align}
Since
\begin{equation*}
\left|\exp(isAe^{i\theta})\right| = \exp\bigl(-As\sin(\theta)\bigr),
\end{equation*}
and that is \(<1\) if $s$ and \(\sin(\theta)\) of the same sign.
Hence as \(A\to\infty\)
the integral of \eqref{eq:ex10.12:lowcirc} when \(s<0\)
and the integral of \eqref{eq:ex10.12:upcirc} when \(s>0\)
converges to~$0$.
Therefore
\begin{equation}
\lim_{A\to\infty} \varphi_A(s) = 
 \left\{\begin{array}{ll}%
 -2\pi s \qquad & \textnormal{if\ }\; s > 0\\
 0       \qquad & \textnormal{if\ }\; s \leq 0
 \end{array}
 \right. \label{eq:10.12:varphiA}
\end{equation}

\iffalse
We compute the rational integral term of \eqref{eq:10.12:sin2phis}.
Within the real axis
\begin{equation*}
\int_{\R\setminus(-1,1)} \frac{dx}{x^2} 
 = 2\int_1^{\infty} x^{-2}\,dx 
 = 2\left(-x^{-1}\right)\bigm|_1^\infty = 2\bigl(0-(-1)\bigr)=2
 \end{equation*}
and within the arc part
\begin{align*}
\int_{-\pi}^0 e^{-2i\theta}\frac{dz}{d\theta}\,d\theta 
&= i \int_{-\pi}^0 e^{-2i\theta}\cdot e^{i\theta}\,d\theta 
 = i \int_{-\pi}^0 e^{-i\theta}\,d\theta
 = i\left.\left(\frac{1}{-i}e^{-i\theta}\right)\right|_{-\pi}^0 \\
&= -\left.e^{-i\theta}\right|_{-\pi}^0 = % BADBAD 1 - \bigl(-(-1)\bigr) = 0.
   -\bigl(1 - (-1)\bigr) = -2
\end{align*}
Adding to:
\begin{equation}
\int_{\Gamma_A} \frac{dz}{z^2}
 = \int_{\R\setminus(-1,1)} \frac{dx}{x^2} 
   + \int_{-\pi}^0 \exp(isAe^{i\theta})e^{i\theta}\,d\theta \\
 = 2 - 2 = 0 \label{eq:10.12:intrat}
\end{equation}
\fi % false

Applying \eqref{eq:10.12:varphiA} % and \eqref{eq:10.12:intrat} 
to \eqref{eq:10.12:sin2phis} gives
\begin{align*}
\int_{-\infty}^\infty (\sin x/x)^2 e^{itx}\,dx
&= \lim_{A\to\infty} 
   \left(\varphi_A(t) -\bigl(\varphi_A(t+2) + \varphi_A(t-2)\bigr)/4)
   \right)\\
&= \left\{\begin{array}{ll}%
   0 + 0 + 0  \qquad & t \leq -2 \\
   0 + 2\pi(t+2)/4 + 0 \qquad & -2 \leq t \leq 0 \\
   -2\pi t/2 + 2\pi(t+2)/4 + 0 \qquad & 0 \leq t \leq 2 \\
   -2\pi t/2 + 2\pi(t+2)/4 + 2\pi(t-2)/4 \qquad & t \geq 2
   \end{array}\right. \\
&= \left\{\begin{array}{ll}%
   0  \qquad & t \leq -2 \\
   \pi t/2+\pi \qquad & -2 \leq t \leq 0 \\
   -\pi t/2 + \pi \qquad & 0 \leq t \leq 2 \\
   0 \qquad & t \geq 2
   \end{array}\right. \\
&= \max\left(0, \pi-|\pi t/2|\right)
\end{align*}

%%%%%%%%%%%%%% 13
\begin{excopy}
Compute
\begin{equation*}
\int_0^\infty \frac{dx}{1 + x^n} \qquad (n=2,3,4,\ldots).
\end{equation*}
[For even $n$, the method of Exercise~8 can be used.
However, a different path can be chosen, which simplifies the computation
and which also works for odd $n$:
from $0$ to $R$ to \(R = \exp(2\pi i/n)\) to $0$.]
\\
\phantom{AAAA}\emph{Answer:} \((\pi/n)\sin(\pi/n)\).
\end{excopy}

Put 
\begin{equation*}
f(z) = \frac{1}{1 + z^n}.
\end{equation*}
Assume first that n is even (\(n/2\in\Z\)) and \(n\geq 2\).
The roots of \(z^n+1\) are \mbox{\(\{e^{\pi i(2k+1)/n}\!: k\in \Z_n\}\)}.
Put \(\alpha_k = e^{\pi i(2k+1)/n}\) for \(k\in\Z_n\).
The roots in the upper place are
 \(\{\alpha_k: k\in \Z_{n/2}\}\).
Using the result of exercise~8 above and the fact that the integrand is 
an even function, we have
\begin{align*}
\int_0^\infty \frac{dx}{1 + x^n}
&= \half\cdot \int_{-\infty}^\infty \left(1 + x^n\right)^{-1}\,dx
 = \pi i \cdot \sum_{k\in\Z_{n/2}} 
        \Res\left( (1 + z^n)^{-1}; \alpha_k \right) \\
&= \pi i \cdot \sum_{k\in\Z_{n/2}}\; 
      \prod_{j\in \Z_n\setminus\{k\}} \left(\alpha_k - \alpha_j)\right)^{-1}
\end{align*}

\iffalse
Let's compute the residue of the ``first'' pole.
\begin{align*}
\Res(f, \alpha_0) 
&= \prod_{j\in \Z_n\setminus\{0\}} (\alpha_0 - \alpha_j)^{-1}
 = \lim_{z\to\alpha_0} \frac{z-\alpha_0}{\prod_{j\in \Z_n} (z - \alpha_j)}
 = \lim_{z\to\alpha_0} \frac{z-\alpha_0}{z^n+1} 
 = \lim_{z\to\alpha_0}\frac{1}{nz^{n-1}} \\
&= e^{\pi i/n}/n
\end{align*}
\fi 

Let's compute the residue of a pole.
\begin{align*}
\Res(f, \alpha_k) 
&= \prod_{j\in \Z_n\setminus\{k\}} (\alpha_k - \alpha_j)^{-1}
 = \lim_{z\to\alpha_k} \frac{z-\alpha_k}{\prod_{j\in \Z_n} (z - \alpha_j)}
 = \lim_{z\to\alpha_k} \frac{z-\alpha_k}{z^n+1} 
 = \lim_{z\to\alpha_k}\frac{1}{nz^{n-1}} \\
&= \alpha_k^{-(n-1)}/n
 %= \left(\alpha_k/\alpha_k^n\right)\bigm/n
 = \alpha_k^{-n}\alpha_k/n = -e^{(2k+1)\pi i/n}/n
% = e^{((2k+1)\pi i/n)(1-n))}/n
% = e^{(2k+1)\pi i(1-n)/n)}/n
\end{align*}
In particular 
\begin{equation*}
\Res(f, \alpha_0) = -\alpha_0/n0) = -e^{\pi i /n}/n.
\end{equation*}

Using the hint, We consider the closed path \(\Gamma_R\)
which is the sum of
\begin{equation*}
\begin{array}{ll}
\gamma_1:[0,R]\to\C & \gamma_1(t) = t \\
\gamma_2:[0,2\pi/n]\to\C \qquad& \gamma_2(t) = e^{it} \\
\gamma_3:[0,R]\to\C & \gamma_3(t) = e^{2\pi i/n}(R-t) 
\end{array}
\end{equation*}

Note that if \(z\in \gamma_3^*\) then 
\(z^n = |z|^n\) and so \((1+z^n)^{-1} = (1+|z|^n)^{-1}\).
For any \(R>1\) we have
\begin{equation*} 
\int_{\Gamma_R} \frac{dz}{1+z^n} = 2\pi i \Res(f;\alpha_0) 
= - 2\pi i e^{-\pi i/n}/n.
\end{equation*}
Now
\begin{align*}
\lim_{R\to\infty} \int_{\Gamma_R} \frac{dz}{1+z^n}
&= \lim_{R\to\infty} 
   \left( \int_{\gamma_1}\cdots +\int_{\gamma_2}\cdots +\int_{\gamma_3}\cdots \right)
 = \lim_{R\to\infty} \left( \int_{\gamma_1}\cdots +\int_{\gamma_3}\cdots \right) \\
&= \lim_{R\to\infty} \left( 
        \int_0^R (x^n+1)^{-1}\,dx +
        \int_0^R \left(\bigl((R-x)e^{2\pi i /n}\bigr)^n+1\right)^{-1}
                 \frac{dz}{dx}
                 \,dx\right) \\
&= (1 - e^{2\pi i /n}) \lim_{R\to\infty}  \int_0^R (x^n+1)^{-1}\,dx
\end{align*}
Hence 
\begin{align*}
 \int_0^\infty (x^n+1)^{-1}\,dx 
 &= \frac{2\pi i \left(-e^{\pi i/n}/n\right)}{1 - e^{2\pi i /n}}
  = \frac{-2\pi i }{ne^{-\pi i/n}(1 - e^{2\pi i /n})}
  = \frac{2\pi i }{n(e^{\pi i/n} - e^{-\pi i /n})} \\
 &= \frac{\pi}{n(e^{\pi i/n} - e^{-\pi i /n})/(2i)}
  = \frac{\pi}{n\cdot\sin(\pi/n)} \\
 &= (\pi/n)\bigm/\sin(\pi/n).
\end{align*}


%%%%%%%%%%%%%% 14
\begin{excopy}
Suppose \(\Omega_1\) and \(\Omega_2\) are plane regions, $f$ and $g$ are
nonconstant functions defined on
\(\Omega_1\) and \(\Omega_2\), respectively,
and \(f(\Omega_1) \subset \Omega_2\).
Put \(h = g\circ f\).
If $f$ and $g$ are holomorphic, we know that $h$ is  holomorphic.
Suppose we know that $f$ and $h$ are holomorphic.
Can we conclude anything about $g$?
What if we know that $g$ and $h$ are holomorphic?
\end{excopy}

In the second case we \emph{cannot} know much about $g$.
Take \(f(z)=0\) and \(g(z)=|z|\).
Clearly \(h=0\) and $g$ is not holomorphic.
A More interesting question would be if we also assume that 
$F$ is \emph{onto} \(\Omega_2\).


In the second case we \emph{cannot} know much about $f$.
For example, take \(g=0\) and \(h=0\) constant functions, 
and the equality holds for any \(f \in \Omega_2^{\Omega_1}\).

%%%%%%%%%%%%%% 15
\begin{excopy}
Suppose \(\Omega\) is a region, \(\varphi \in H(\Omega)\),
\(\varphi'\) has no zero in \(\Omega\),
\(f\in H(\varphi(\Omega))\),
\(g = f\circ \varphi\), \(z_0\in\Omega\), and
\(w_0 = \varphi(z_0)\).
Prove that if $f$ has a zero of order $m$ at \(w_0\),
then $g$ also has a zero of order $m$ at \(z_0\).
How is this modified if \(\varphi'\) has a zero of order $k$ at \(z_0\)?
\end{excopy}

Clearly \(g'(z_0) = f'(w_0)\varphi'(z_0)\).
Then  both \(g'\) and \(f'\) 
has zero of order \(m-1\) in \(w_0\) and \(z_0\) respectively.
Hence  $f$ has a zero of order~$m$ at~\(w_0\).
It is now easy to see that the zero order of~\(\varphi\) at~\(z_0\)
adds to the order $g$ at~\(z_0\).

%%%%%%%%%%%%%% 16
\begin{excopy}
Suppose \(\mu\) is a complex measure on a measure space~$X$,
\(\Omega\)~is an open set in the plane,
\(\varphi\)~is a bounded function on \(\Omega\times X\)
such that \(\varphi(z,t)\) is a measurable function of $t$,
for each \(z \in \Omega\), and \(\varphi(z,t)\) is holomorphic in \(\Omega\),
for each \(t\in X\). Define
\begin{equation*}
f(x) = \int_X \varphi(z,t)\,d\mu(t)
\end{equation*}
for \(z\in\Omega\). Prove that \(f\in H(\Omega)\).
\emph{Hint:} Show that to every compact \(K \subset \Omega\) there corresponds
a constant \(M<\infty\) such that
\begin{equation*}
\left| \frac{\varphi(z,t) - \varphi(z_0,t)}{z - z_0}\right| < M
\qquad (z \;\textnormal{ and }\; z_0\in K,\; t\in X).
\end{equation*}
\end{excopy}

Obviously, in the above formula we assume \(z\neq z_0\).
Let $U$ be some upper bound of \(|\varphi|\).
For all~\(t\in X\),
denote the holomorphic functions \(\varphi_t(z) = \varphi(z,t)\)
and the ratios
\begin{equation*}
\Delta_t(z_1,z_0) = \frac{\varphi_t(z_1) - \varphi_t(z_0)}{z_1 - z_0}
\qquad (\textnormal{for all distinct}\; z_0,z_1\in\Omega).
\end{equation*}
Recall that \(D'(a;r) = \{z: 0 < |z-a|<r\}\).

Let \(K\subset \Omega\) be a compact set.
Pick some arbitrary \(w\in K\).
Then we can find some \(r>0\) such that \(\overline{D(w;r)}\subset \Omega\).
We pick \(v_0\in D(w,r/2)\) and
we will estimate \(|\Delta_t(z,v_0)|\) for all \(z\in\Omega\). 

\paragraph{Local case.}
Clearly \(D(v_0;r/2)\subset D(w;r)\subset\Omega\) and
\(|{\varphi_t}'(v_0)| \leq 2U/r\) for all \(t\in X\) by Theorem~10.26.
% This shows that \(|\Delta_t(w,v_0)| \leq 2U/r\) 
for all \(t\in X\).

Pick \(v_1\in D'(v_0;r/2)\). 
Use the path segment 
\(\gamma(\tau) = v_0 + \tau(v_1-v_0)\) with \(\tau\in[0,1]\).
Then
\begin{equation*}
\varphi_t(v_1) 
= \varphi_t(v_0) + \int_\gamma {\varphi_t}'(z)\,dz
= \varphi_t(v_0) + \int_0^1 {\varphi_t}'(\gamma(\tau))\gamma'(\tau)\,d\tau
\end{equation*}
Hence
\begin{equation*}
\left|\varphi_t(v_1) - \varphi_t(v_0)\right|
\leq \int_0^1 \left|{\varphi_t}'(\gamma(\tau))\gamma'(\tau)\right|\,d\tau
\leq 2|v_1-v_0|U/r.
\end{equation*}
Subsequently
\begin{equation} \label{eq:ex10.16:estimate:in}
|\Delta_t(v_1,v_0)| \leq 2U/r 
\qquad (\textnormal{for all} v_0\in D(w;r/2),\;v_1 \in D(v_0;r/2)\,).
\end{equation}

\paragraph{Outside case.}
When \(v_1\in\Omega \setminus D(v_0;r/2)\)
\begin{equation} \label{eq:ex10.16:estimate:out}
|\Delta_t(v_1,v_0)| 
\leq \frac{|\varphi_t(w) - \varphi_t(v)|}{|r/2|} \leq 4U/r.
\end{equation}

To combine the case, define
\begin{equation*}
\Omega_M := 
\left\{z\in\Omega: 
  \forall t\in X, \forall \zeta\in\Omega\setminus\{z\},\; |\Delta_t(z,\zeta)|<M
\right\}.
\end{equation*}

From \eqref{eq:ex10.16:estimate:in} and \eqref{eq:ex10.16:estimate:out}
we know that there exists \(M = M_w\) (for example \(M_w = 4U/r + 1\))
such that 
\begin{equation*}
D(w;r/2) \subset \Omega_M
\end{equation*}

Since $w$ was arbitrarily chosen, \(\Omega \subset \cup_{M\in\N}\Omega_M\)
and since $K$ is compact we can find some \(M<0\) such that 
\(|\Delta_t(z_1,z_0)|<M\) for all \(t\in X\) and all distinct \(z_0,z_1\in K\).
Thus establishing the hint.

In order for $f$ to be differentiable at $z$, it is sufficient that
for any sequence \(\{z_n\}_{n\in\N}\) there exists a limit 
\begin{equation*}
f'(z) = \lim_{n\to\infty} \frac{f(z_n) - f(z)}{z_n - z} 
 \qquad (\text{where}\; z_n \neq z)
\end{equation*}
Now
\begin{equation*}
\frac{f(z_n) - f(z)}{z_n - z} 
= \frac{\int_X \varphi(z_n,t)\,d\mu - \int_X \varphi(z,t)\,d\mu}{z_n - z} \\
= \int_X \frac{\varphi(z_n,t) -  \varphi(z,t)}{z_n - z}\,d\mu
\end{equation*}

There exists some \(\delta>0\) 
such that \({D(z,\delta)}\subset\Omega\).
For sufficient large $m$, for all \(n>m\) 
we have \(z_n\in \overline{D(z,\delta/2)}\) and by its compatcness
and the established hint, \(\Delta_t(z_n,z)\) are bounded
and we can apply 
\index{Lebesgue}
Lebesgue's Dominated Theorem~1.34 that here gives 
\begin{equation*}
f'(z) 
= \lim_{n\to\infty} \int_X \Delta_t(z_n,t)\,d\mu
= \int_X \lim_{n\to\infty} \Delta_t(z_n,t)\,d\mu
= \int_X \varphi_t'(z)\,d\mu.
\end{equation*}

%%%%%%%%%%%%%% 17
\begin{excopy}
Determine the regions in which the following functions are defined
and holomorphic:
\begin{equation*}
f(z) = \int_0^1 \frac{dt}{1+tz}, \qquad
g(z) = \int_0^\infty \frac{e^{tz}}{1+t^2}\,dt, \qquad
h(z) = \int_{-1}^1 \frac{e^{tz}}{1+t^2}\,dt\,.
\end{equation*}
\emph{Hint}: either use Exercise~16, or combine
\index{Morera}
Morera's theorem with
\index{Fubini}
Fubini's.
\end{excopy}

\begin{itemize}

\item The function \(f(z)\) is holomorphic on any region \(\Omega\) for which 
\begin{equation*}
\{1/(1+tz): t\in[0,1]\;\wedge\; z\in\Omega\}
\end{equation*}
is bounded. This happens when there exists \(r > 0\) such that 
\begin{equation*}
\Omega \subset \{w\in\C: \forall z\in (-\infty,-1],\, |w-z|>r\}.
\end{equation*}

\item
The function \(g(z)\) is holomorphic on any region \(\Omega\) 
such that \(\Re(z)\leq 0\) for all \(z\in\Omega\).
Assume \(\Omega\) satisfies this condition.
Define
\begin{equation*}
g_n(z) = \int_0^n \frac{e^{tz}}{1+t^2}\,dt, \qquad
\end{equation*}
By previous exercise, \(g_n\) are holomorphic on \(\Omega\).
\index{Lebesgue}
By Lebesgue's Dominated Theorem~1.34 
\(\lim_{n\to\infty} g_n(z) = g(z)\) and it is easy to see that 
the convergence is uniform on any bounded subset of \(\Omega\),
in particular on triangle boudnaries. Thus the condition
\index{Morera}
of Morera's Theorem~10.17 holds, hence \(g\in H(\Omega)\). 

\item The function \(h(z)\) is holomorphic on any region \(\Omega\) for which 
\begin{equation*}
\{\exp(tz): t\in[-1,1] \;\wedge\; z\in\Omega\}
\end{equation*}
is bounded. This happens when there exists some \(M<\infty\) such that 
\begin{equation*}
\forall w\in\Omega,\;\Re(w) \leq M.
\end{equation*}

\end{itemize}

%%%%%%%%%%%%%% 18
\begin{excopy}
Suppose \(f\in H(\Omega)\),
\(\overline{D}(a;r)\subset \Omega\),
\(\gamma\) is a positively oriented circle with center at $a$ and radius $r$,
and $f$ has no zero on \(\gamma^*\). For \(p=0\), the integral
\begin{equation*}
\itwopii \int_\gamma \frac{f'(z)}{f(z)} z^p\,dz
\end{equation*}
is equal to the number of zeros of $f$ in \(D(a;r)\).
What is the value of this integral (in terms of zeros of $f$)
for \(p=1,2,3,\ldots\)?
What is the answer if \(z^p\) is replaced by any \(\varphi\in H(\Omega)\)?
\end{excopy}

For \(p = 0\) the claim is shown in Theorem~10.43\ich{a}.
Otherwise the number it still gives the number of zeros,
except that if $f$ has zero of order $k$ in \(z=0\) then 
the number is reduced by \(\min(p, k)\).
Similarly of \(\varphi\) has zeros of order \(p_j\) on \(z_j\)
where $f$ has zeros of order \(k_j\)
then the number is reduced by \(\sum_j \min(p_j,k_j)\).


%%%%%%%%%%%%%% 19
\begin{excopy}
Suppose \(f\in H(U)\), \(g\in H(U)\),
and neither $f$ nor $g$ has a zero in $U$. If
\begin{equation*}
\frac{f'}{f}\left(\frac{1}{n}\right) =
\frac{g'}{g}\left(\frac{1}{n}\right)
\qquad (n=1,2,3,\ldots)
\end{equation*}
find another simple relation between $f$ and $g$.
\end{excopy}

Both \((f'/f), (g'/g)\in H(U)\).
By the Corollary to Theorem~10.18 we have \(f'/f = g'/g\) in \(H(U)\).
Using line segments from $0$ to \(z=te^{i\theta}\in U\) 
as the path: \(\gamma(\tau) = \tau\cdot e^{i\theta}\) with \(0\leq \tau\leq t\), 
Define
\begin{equation*}
F(z) 
= \int_0^t (f'/f)(\tau e^{i\theta})\cdot e^{i\theta}\,d\tau
= e^{i\theta} \int_0^t \frac{d(\log\circ f)}{d\tau}(\tau e^{i\theta})\,d\tau
\end{equation*}
Similarly, we define \(G(z)\) by integrating \(g/g'\).
Clearly $F$ and $G$ differ by a constant, say \(c=F(z)-G(z)\). Hence
\begin{equation*}
(\log\circ f)(z) = (\log\circ g)(z) + c
\end{equation*}
By taking exponents, we get the relation
\begin{equation*}
f(z) = e^c \cdot g(z).
\end{equation*}

%%%%%%%%%%%%%% 20
\begin{excopy}
Suppose \(\Omega\) is a region,
\(f_n\in H(\Omega)\) for \(n=1,2,3,\ldots\),
none of the functions \(f_n\) has a zero in \(\Omega\),
and \(f_n\) converges to $f$ uniformly on compact subsets of \(\Omega\).
Prove that either $f$ has no zero in \(\Omega\) or \(f(z)=0\)
for all \(z\in\Omega\).
\end{excopy}

Consider \(N = f^{-1}(0) \subset\Omega\).
If \(N=\emptyset\) or \(N = \Omega\) we are done.

By negation assume \(\emptyset \subsetneq N \subsetneq \Omega\).
By Theorem~10.18 there exists an isolated zero.
Hence we have 
\(B(z;r) \subset \Omega\setminus N\) for some \(z\in N\) and \(r>0\).
We look at the circle path \(\gamma\) whose image \(\gamma^*\) 
is the boundary of \(\overline{B(z;r/2)}\). 
By Theorem~10.28 the derivatives 
\({f_n}\) converge uniformly to \(f'\) on \(\gamma*\).
But now Theorem~10.43\ich{a} gives the following
\begin{equation*}
0 = \lim_{n\to\infty} N_{f_n} = N_f = 1
\end{equation*}
contradiction.


%%%%%%%%%%%%%% 21
\begin{excopy}
Suppose \(f\in H(\Omega)\), \(\Omega\) contains the closed unit disc, and
\(|f(z)| < 1\) if \(|z|=1\).
How many fixed points must $f$ have in the disc?
That is, how many solutions does the equation \(f(z)=z\) have there?
\end{excopy}

By looking at \(h(z) = z/2\) we can see that the minimal number of
solutions for such $f$ cannot exceed $1$.

Let \(U_0 = \{z\in\C: |z|\leq 1\}\) be the unit circle.
Define by induction \(U_k = f(U_{k-1})\) for \(k\geq 1\).
Also by induction we can see that \(U_k \subset U_{k-1}\).
These are compact sets and thus have non empty intersection
\(X = \cap_{k\in\Z^+} U_k\). Clearly \(f(X) = X\).

\index{maximum modulus}
By the maximum modulus Theorem~10.24
\index{Cauchy's estimates}
and Cauchy's estimates Theorem~10.26, we have \(|f'(z)|<1\) for all \(z\in U\).
Moreover, \(s = \max_{z\in U}|f'(z)|<1\).
Hence the diameter of the compact sets \(\{U_k\}_{k\in\Z^+}\) is decreasing,
such that \(\diam(U_k) \leq s^k\). Hence $X$ must be a singleton.

%%%%%%%%%%%%%% 22
\begin{excopy}
Suppose \(f\in H(\Omega)\), \(\Omega\) contains the closed unit disc,
\(|f(z)| > 2\) if \(|z|=1\), and \(f(0)=1\).
Must $f$ have a zero in the unit disc?
\end{excopy}

If by negation there was no zero, then \(g = 1/f \in H(\Omega)\)
and 
\begin{equation*}
|g(0)| = 1 > 1/2 > \max\{|f(z)|: |z|=1\}
\end{equation*}
that contradicts the
\index{Maximum Modulus}
Maximum Modulus Theorem~10.24.

%%%%%%%%%%%%%% 23
\begin{excopy}
Suppose \(P_n(z) = 1 + z/1! + \cdots + z^n/n!\),
\(Q_n(z) = P_n(z) - 1\), where \(n=1,2,3,\ldots\),
% none of the functions
What can you say about the location of the zeros
of \(P_n\) and \(Q_n\) for large $n$?
Be as specific as you can.
\end{excopy}

The polynomials \(P_n\) have zeros that converge to infinity.
More accurately, for each \(R<0\) there exists some \(m<\infty\)
such that \(P_n\) has \emph{no} zeros in \(B(0;R)\) for all \(n>m\).
Otherwise, there would be an increasing sub-sequence of indices 
\(\{s_n\}_{n\in\N}\) and zeros \(\{z_n\}_{n\in\N}\) in \(B(0;R)\) 
such that \(P_{s_n}(z_n) = 0\) and \(\lim_{n\to\infty} z_n = z\in B(0;R)\).
Now \(\lim_{n\to\infty} P_{s_n}(z) = \exp(z)\) uniformly on \(B(0;R)\)
which leads to the \(\exp(z) = 0\) contradiction.

In addition to \(Q_n(0)=0\) for all \(n\in\N\),
the polynomials \(Q_n\) have zeros that converge to \(\{2\pi i k: \;k\in\Z\}\).
But the other roots diverge. More accurately,
for all \(R<0\) and \(r>0\)  there exists some \(m<\infty\)
such that \(Q_n\) has \emph{no} zeros in
\begin{equation*}
\{z\in\C:\; |z|\leq R \;\wedge\; \Re(z)\geq r\}
\end{equation*}
for all \(n>m\).


%%%%%%%%%%%%%% 24
\begin{excopy}
Prove the following form of
\index{Rouche}
Rouche's theorem:
Let \(\Omega\) be the interior of a compact set $K$ in the plane.
Suppose $f$ and $g$ are continuous on $K$ and holomorphic in \(\Omega\),
and \(|f(z)-g(z)|<|f(z)|\) for all \(z\in K\setminus \Omega\).
Then $f$ and $g$ have the same number of zeros in \(\Omega\).
\end{excopy}

Let $D$ be a connected component of  \(\Omega\).
We will show that $f$ and $g$ have the same number of zeros in $D$.
If by negation $f$ has have infinite number of zeros in any such $D$,
then then \(f_{\restriction D}=0\) and by continuity
 \(f_{\restriction \partial D}\) but this contradicts the assumption 
that \(|f(z)-g(z)|<|f(z)|\) for all \(z\in K\setminus \Omega\).
Thus the number of zeros of $f$ in $D$ is finite.
Once we establish that this number of zeros of $g$ in $D$ is the same,
we can sum up the zeros in all such components and get the desired
result. Let \(N = \{z\in D:\; f(z) = 0\}\).

The boundary \(\partial D \subset K\). For any \(\epsilon>0\),
by introducing sufficiently small squares,
we can find a path \(\gamma_\epsilon\) that is ``\(\epsilon\)-near'' the boundary.
More accurately, \(d(z,K)<\epsilon\) for all \(z\in \gamma_\epsilon^*\).

Since the number of zeros of $f$ in $D$ is finite, we can pick 
\(\epsilon < d(N,\partial D)\).
Hence \(f(z)\neq 0\) for all \(z\in \gamma_\epsilon^*\).
Since $f$ and $g$ are continuous, we can further require \(\epsilon\)
to sufficiently small such that the inequality \(|f(z)-g(z)|<|f(z)|\)
also holds for all \(z\in\gamma_\epsilon^*\).
The needed step to complete is to apply Theorem~10.43\ich{b}
which shows that the numbers of zeros of $f$ and $g$ in $D$ are the same.


%%%%%%%%%%%%%% 25
\begin{excopy}
Let $A$ be the
\index{annulus}
annulus \(\{x: r_1 < |z| < r_2\}\), where \(r_1\) and \(r_2\) are given
positive numbers.
\begin{itemize}

\itemch{a} Show that the Cauchy formula
\begin{equation*}
f(z) = \itwopii \left(\int_{\gamma_1} + \int_{\gamma_2}\right)
       \frac{f(\zeta)}{\zeta - z}\,d\zeta
\end{equation*}
is valid under the following conditions: \(f\in H(A)\),
\begin{equation*}
r_1 + \epsilon < |z| < r_2 - \epsilon,
\end{equation*}
and
\begin{equation*}
\gamma_1(t) = (r_1+\epsilon)e^{-it},
\qquad
\gamma_2(t) = (r_2-\epsilon)e^{it},
\qquad
(0\leq t \leq 2\pi).
\end{equation*}

\itemch{b} Show by means of \ich{a} that every \(f\in H(A)\)
can be decmposed info a sum \(f=f_1+f_2\), when \(f_1\) is holomorphic
outside \(\overline{D}(0;r_1)\)
and  \(f_2 \in H(D(0;r_1))\).
The decomposition is unique if we require that 
\(f_1(z)\to 0\) as \(|z|\to\infty\).

\itemch{c} Use this decomposition to associate with each \(f\in H(A)\) its
so-called
\index{Laurent series}
``Laurent series''
\begin{equation*}
\sum_{-\infty}^\infty c_n z^n
\end{equation*}
which converges to $f$ in $A$. Show that there is only one such series for 
each $f$. Show that it converges to $f$ uniformly on compact subsets of $A$.

\itemch{d} If \(f\in H(A)\) and $f$ is bounded in $A$, show that the components
\(f_1\) and \(f_2\) are also bounded.

\itemch{e} How much of the foregoing can you extend to the case \(r_1=0\)
(or \(r_2=\infty\), or both)?

\itemch{f} How much of the foregoing can you extend  to region bounded 
by finitely many (more than two) cycles?
\end{itemize}
\end{excopy}

See \cite{Gamelin2003} Chapter~VI, Section~1.
\begin{itemize}

\itemch{a}
This result follows by applying \index{Cauchy} Cauchy's Theorem~10.35
for \(\Gamma = \gamma_1 + \gamma_2\).

\itemch{b}
We define
\begin{align*}
f_1(z) &= \itwopii \int_{\gamma_1} \frac{f(\zeta)}{\zeta - z}\,d\zeta \\
f_2(z) &= \itwopii \int_{\gamma_2} \frac{f(\zeta)}{\zeta - z}\,d\zeta\,.
\end{align*}
Clearly \(f = f_1 +f_2\) and \(\lim_{z\to\infty} f_1(z) = 0\).
Say  \(f = g_1 +g_2\) and \(\lim_{z\to\infty} g_1(z) = 0\).
Then \(f_1 - g_1 = g_2 - f_2\) in $A$.
The function \(f_1 - g_1\) is holomorphic outside of \(\gamma_1\)
and the function \(g_2 - f_2\) is holomorphic inside \(\gamma_2\).
Since they agree on the intersection, we have an entire function $h$
\begin{align*}
h(z) &= g_2(z) - f_2(z) \qquad (|z| < r_1) \\
h(z) &= f_1(z) - g_1(z) \qquad (|z| < r_2)
\end{align*}
But \(\lim_{z\to\infty} h(z) = 0\), hence \(h(z)=0\) for all \(z\in\C\)
and the uniqueness follows.

\itemch{c}
We have
\begin{equation*}
f_2(z) = \sum_{n=0}^\infty c_n z^n
\end{equation*}
We can compute the coefficients \(c_n\) using Theorem~10.7
\begin{equation*}
c_n = \itwopii \int_{\gamma_2} \frac{f(\zeta)}{\zeta^{n+1}} \qquad (n\geq 0).
\end{equation*}

Since \(\lim_{z\to\infty} f_1(z) = 0\), we can put \(z=1/w\) and define
\begin{equation*}
g_1(w) = f(1/w) = f(z) \qquad (|z|>r  \Leftrightarrow |w|<1/r)
\end{equation*}
and with \(g_1(0)=0\), we have \(g_1 \in H(\{w\in\C: |w|<1/r\})\).
Hence it as a power series 
\begin{align*}
g_1(w) &= \sum_{n=1}^\infty b_n w^n \\
f_1(z) &= g_1(1/z) = \sum_{n-1}^\infty b_n z^{-n}
\end{align*}
By putting \(c_{-n} = b_n\) we get the desired representation for $f$.

For each \(n\in\Z\), we consider \(f(z)/z^{n+1}\). Now for any sub-annulus of $A$
we can find sufficiently small \(\epsilon>0\) such that it is contained
between the circles of \(\gamma_1^*\) and  \(\gamma_2^*\) and we can compute
\begin{equation*}
\int_{\gamma_1^*} \frac{f(z)\,dz}{z^{n+1}}
 = \int_{\gamma_1^*} \frac{1}{z^{n+1}}\sum_{n\in\Z} c_k z^k\,dz
 = \sum_{n\in\Z} c_k \int_{\gamma_1^*} z^{k-n-1}\,dz
 = 2\pi i c_{n}
\end{equation*}
Since the last integral vanishes whenever \(k-n-1\neq -1\).
Hence
\begin{equation*}
c_n = \itwopii \int_{\gamma_1^*} \frac{f(z)\,dz}{z^{n+1}}.
\end{equation*}

By the arguments of section~10.5, 
the power series of
\begin{equation*}
f_2(z) = \sum_{n=0}^\infty c_n z^n
\end{equation*}
converges absolutely in \(\overline{D(0;\rho)}\) for every \(\rho<r_2\).
Similarly by looking on \(g_1\), the power series of 
\begin{equation*}
f_1(z) = \sum_{n=-1}^{-\infty} c_n z^n
\end{equation*}
converges absolutely in \(\{z\in\C: |z|\geq \rho\}\) for every \(\rho> r_1\).
Now for every compact \(K\subset A\) we can find 
\(\rho_1\) and \(\rho_2\) such that
\begin{equation*}
K \subset \{z\in\C: \rho_1 \leq |z| \leq \rho_2\}
\qquad \textnormal{and}\qquad 
r_1 < \rho_1 < \rho_2 < r_2\
\end{equation*}
and clearlt the whole power series converges absolutely in $K$.x

If by negation there was another power series, 
then by  positive and negative powers split, we would get 
another decomposition for $f$ that contradicts the uniqueness established 
in~\ich{b}.

\itemch{d}
Put \(r_3 = (r_1+r_2)/2\). 
Clearly \(f_2\) is bounded in~\(B_2 = \overline{D(0;r_3)}\)
and  \(f_1\) is bounded in~\(B_1 = \C\setminus D(0;r_3)\).
If \(f = f_1+f_f2\) is bounded in $A$, then 
\(f_1\) must be bounded in \(A\setminus B_2\) and
\(f_2\) must be bounded in \(A\setminus B_1\).
Consequently, \(f_1\) and \(f_2\) are bounded on~$A$.

\itemch{e}
The case \(r_1=0\) and \(r_2=1\) applies for the function 
\begin{equation*}
f(z) = \sum_{n\geq -1} z^n = 1/z + 1/(1-z)
\end{equation*}

The cases of \(r_2=\infty\) requires different interpration of ``\(\epsilon\)''
for selecting \(\gamma_2\). We can use the following condition;
for any \(M<\infty\), let \(\gamma_2(t) = Me^{it}\) for \(0\leq t\leq 2\pi\).
Function that can be applied for such case have essential singularity 
in infinity, for example \(f(z) = \sin(z)\).
This can be combined with \(r_1=0\) as well (add \(1/z\)).

\itemch{f}
With $n$ circles all centered at the origin, there are \(n-1\) annulus regions.
All the above applies to each such annulus.

\end{itemize}

%%%%%%%%%%%%%% 26
\begin{excopy}
It is required to extend the function 
\begin{equation*}
\frac{1}{1-z^2} + \frac{1}{3-z}
\end{equation*}
in the series of the form \(\sum_{-\infty}^\infty c_n z^n\).

How many such expansions are there?
In which region is each of them valid?
Find the coefficients \(c_n\) explicitly for each of these expansions.
\end{excopy}

We use the identity:
\begin{equation*}
f(z) = \frac{1}{1-z^2} + \frac{1}{3-z}
= \frac{1}{2}\left(\frac{1}{z-1} - \frac{1}{z+1}\right) + \frac{1}{3-z}
\end{equation*}

Consider the following case:
\begin{alignat*}{2}
\frac{1}{z-1} &= \sum_{n=0}^\infty z^n && \qquad (|z|<1) \\
\frac{1}{z+1} &= \sum_{n=0}^\infty (-1)^n z^n  && \qquad (|z|<1) \\
\frac{1}{3-z} &= \sum_{n=0}^\infty 3^{-(n+1)} z^n  && \qquad (|z|<3) \\
\frac{1}{z-1} 
 &= \frac{1}{z(1-(1/z))} = 
  \sum_{n<0} z^n && \qquad (|z|>1) \\
\frac{1}{z+1} 
 &= \frac{1}{z(1-(-1/z))} 
 = \sum_{n<0} (-1)^{n+1} z^n && \qquad (|z|>1) \\
\frac{1}{3-z} 
  &= \frac{-1}{z} \cdot \frac{1}{1-3/z} 
  = \sum_{n<0}^\infty (-1)^{n+1}\cdot 3^{n} z^n  && \qquad (|z|>3)
\end{alignat*}

Gathering the results:
\begin{alignat*}{2}
f(z) &= \sum_{n=-1}^{-\infty} 3^{-(n+1)} z^n + (2/3)z^0 + \sum_{n=1}^\infty z^n  
  & \qquad & (|z|<1) \\
f(z) &= \sum_{n=-1}^{-\infty} \left((-1)^{n+1} + 1\right)z^n + 
        \sum_{n=0}^\infty 3^{-(n+1)} z^n 
     & \qquad & (1<|z|<3) \\
f(z) &= \sum_{n=-1}^{-\infty} \left(1 - (-1)^{n}(1+3^n) \right)z^n
     & \qquad & (|z|>3)
\end{alignat*}


%%%%%%%%%%%%%% 27
\begin{excopy}
Suppose \(\Omega\) is a horizontal strip determined by the inequalities
\(a<y<b\), say.
Suppose \(f\in H(\Omega)\) and \(f(z) = f(z+1)\) for all \(z\in\Omega\).
Prove that $f$ has a Fourier expansion in \(\Omega\),
\begin{equation*}
f(z) = \sum_{-\infty}^\infty c_n e^{2\pi inz},
\end{equation*}
which converges uniformly in \(\{z: a+\epsilon \leq y \leq b - \epsilon\}\),
for every \(\epsilon > 0\).
\emph{Hint:} The map \(z\to e^{2\pi i z}\) converts $f$ to a function in an 
annulus.

Find the integral formula by means on which the coefficients \(c_n\) can be 
computed from $f$.
\end{excopy}

Denote \(z=x+iy\) with \(x,y\in\R\). The map
\begin{equation*}
\varphi(z) = e^{2\pi i z} = e^{2\pi i (x+iy)} = e^{-2\pi y}\cdot e^{2\pi i x}
\end{equation*}
maps the stripe \(\Omega = \{x+iy\in\C: x\in\R \wedge a<y<b\}\)
onto the annulus
\begin{equation*}
A = \{z\in\C: e^{-2\pi b} < |z| < e^{-2\pi a}\}.
\end{equation*}
The mapping is not injective, but if \(\varphi(x_1+iy_1) = \varphi(x_2+iy_2)\)
where \(x_i,y_i\in\R\) then \(y_1=y_2\) and \(x_2-x_1\in\Z\).
Hence we can define \(g:A\to\C\) by \(g(w) = f(z)\) where \(w=\varphi(z)\)
which is well defined since \(f(z)=f(z+1)\) for all \(z\in\Omega\).

But the mapping holomorphic and 
is locally injective, that is for each \(z\in\Omega\)
there is a neighborhood \(V_z\subset\Omega\) such that \(\varphi\)
is one-to-one in \(V_z\).
Hence \(g\in H(A)\) by Theorem~1.30.

By previous exercise there exists a sequence \((c_n)_{n\in\Z}\)
of complex numbers such that 
\begin{equation*}
g(w) = \sum_{n\in\Z} c_n w^n \qquad (w\in A).
\end{equation*}
But
\begin{equation*}
c_nw^n = c_n  \left(e^{2\pi i z}\right)^n = e^{2\pi i nz}
\end{equation*}
substituting the last expression in the previous power-series gives
the desired \index{Fourier} Fourier expansion.


%%%%%%%%%%%%%% 28
\begin{excopy}
Suppose \(\Gamma\) is a closed curve in the plane, with parameter interval
\([0,2\pi]\).
Take \mbox{\(\alpha \notin \Gamma^*\).}
Approximate \(\Gamma\) uniformly by trigonometric polynomials \(\Gamma_n\).
Show that \(\Ind_{\Gamma_m}(\alpha) = \Ind_{\Gamma_n}(\alpha)\) if $m$ and $n$
are sufficiently large. Define this common value to be \(\Ind_{\Gamma}(\alpha)\).
Prove that the result does not depend on the choice of \(\{\Gamma_n\}\);
prove that Lemma~10.39 is now true for closed curves, and use this to give 
a~different proof of Theorem~1.40.
\end{excopy}

By Theorem~4.25 we for each $n$, we can find a trigonometric polynomial
\(P_n:\T\to\C\) such that \(\max\{|\Gamma(t)-P_n(t)|: t\in\T\}< 1/n\).
The continuous image \(\Gamma^*\) is compact, hence 
if \(\alpha\notin \Gamma^*\) then \(d(\alpha,\Gamma^*)>0\).
Hence, we can find some \(k<\infty\) such that \(d(\alpha,\Gamma^*)>1/k\).
Now by Lemma~10.39
\begin{equation} \label{eq:ex10.29:Pmn}
\Ind_{P_m}(\alpha) = \Ind_{P_n}(\alpha) 
\end{equation}
for all \(m, n \geq 2k\).
Hence 
\begin{equation*}
\Ind_\Gamma(\alpha) := \lim_{n\to\infty}{P_m}(\alpha)
\end{equation*}
is well defined and is independent of the choice 
of the trigonometric polynomials, since with any replacment of them,
Lemma~10.39 could still be applied and \eqref{eq:ex10.29:Pmn} will hold.

Now we can simplify the proof of Theorem~10.40.
Instead of introducing the paths 
\begin{equation*}
\gamma_k(s) 
 = H\left(\frac{i}{n},\frac{k}{n}\right)(ns + 1 -i)
 + H\left(\frac{i-1}{n},\frac{k}{n}\right)(i - ns)
\qquad (i\in\N_n, \; k\in \Z_n, \; i-1\leq ns\leq i)
\end{equation*}
we can use the curves (now, not necessarily paths)
\begin{equation*}
\gamma_k(s) = H(s,k/n) \qquad (k\in \Z_n)
\end{equation*}
and proceed with the original proof.


%%%%%%%%%%%%%% 29
\begin{excopy}
Define 
\begin{equation*}
f(z) = \frac{1}{\pi} \int_0^1 r\,dr \int_{-\pi}^\pi \frac{d\theta}{re^{i\theta}+z}.
\end{equation*}
Show that \(f(z) = \overline{z}\) if \(|z|<1\) and that 
\(f(z) = 1/z\) if \(|z|\geq 1\).

Thus $f$ is not holomorphic in the unit disc, 
although the integrand is a holomorphic function of~$z$. 
Note the contrast between this, on the one hand, and Theorem~10.7 
and Exercise~16 on the other.

\emph{Suggestion:} Compute the inner integral separately 
for \(r<|z|\) and for \(r>|z|\).
\end{excopy}

We first compute the inner intergral. We define
\begin{equation*}
h(r,z) = \int_{-\pi}^\pi \frac{d\theta}{re^{i\theta}+z}.
\end{equation*}
\paragraph{Assuming \(|r|<z\).} In this case \(z\neq 0\) and we have
\begin{equation*}
\frac{1}{re^{i\theta}+z}
= \frac{1}{z}\cdot\frac{1}{1-((-r/z)e^{i\theta})}
= \frac{1}{z}\sum_{n=0}^\infty \left(-\frac{r}{z}\right)^n e^{in\theta}.
\end{equation*}
Hence
\begin{align*}
% \int_{-\pi}^\pi \frac{d\theta}{re^{i\theta}+z}
h(r,z)
&= \frac{1}{z} \int_{-\pi}^\pi
    \left( \sum_{n=0}^\infty \left(-\frac{r}{z}\right)^n e^{in\theta} 
    \right)\,d\theta
 = \frac{1}{z} \sum_{n=0}^\infty \left(-\frac{r}{z}\right)^n 
     \int_{-\pi}^\pi e^{in\theta} \,d\theta
 = \frac{1}{z} \left(-\frac{r}{z}\right)^0 \cdot 2\pi \\
&= 2\pi/z
\end{align*}


\paragraph{Assuming \(|r|>z\).} In this case we have
\begin{equation*}
\frac{1}{re^{i\theta}+z}
= \frac{1}{re^{i\theta}}\cdot\frac{1}{1-(-z/r)e^{-i\theta}}
= \frac{1}{re^{i\theta}}\sum_{n=0}^\infty (-z/r)^n e^{-in\theta}.
\end{equation*}
Hence
\begin{equation*}
% \int_{-\pi}^\pi \frac{d\theta}{re^{i\theta}+z}
h(r,z)
 = \int_{-\pi}^\pi \frac{1}{re^{i\theta}}
    \left(\sum_{n=0}^\infty \left(\frac{-z}{r}\right)^n e^{-in\theta}\right) d\theta 
 = \sum_{n=0}^\infty \frac{(-z)^n}{r^{n+1}}
    \int_{-\pi}^\pi e^{-i(n+1)\theta}\,d\theta 
 = 0.
\end{equation*}

Back to the outer integral, we again have two cases.
\paragraph{Assume \(|z|\leq 1\).}
\begin{align*}
f(z) 
&= \frac{1}{\pi} \left(
   \int_0^{|z|} r\cdot h(r,z)\,dr + \int_{|z|}^1 r\cdot h(r,z)\,dr \right) 
 = \frac{1}{\pi} \left(
      \int_0^{|z|} \frac{2\pi r}{z}\,dr +  0\right) \\
&= (2/z) \left.\left(r^2/2\right)\right|_0^{|z|}
 = |z|^2/z
 = \overline{z}\,.
\end{align*}

\paragraph{Assume \(|z|\geq 1\).}
\begin{align*}
f(z) 
&= \frac{1}{\pi} \int_0^1 r\cdot h(r,z)\,dr 
 = \frac{1}{\pi} \int_0^1 (2\pi r/z) \,dr 
 = (1/z)\left.\left(r^2\right)\right|_{0}^1
 = 1/z\,.
\end{align*}


%%%%%%%%%%%%%% 30
\begin{excopy}
Let \(\Omega\) be the plane minus two points, and show that some closed paths
\(\Gamma\) in \(\Omega\) satisfy assumtion~(1) of Theorem~10.35 without being
\index{null-homotopic}
\index{null!homotopic}
null-homotopic in~\(\Omega\).
\end{excopy}

Intuitively, make a double ``eight''-pattern, where each ``eight''
is in different direction.
Say the two deleted points are $1$ and $3$.
Let's for the path as the a polyline passing thru the following
points in the given order:
{\newcommand{\qtoq}{\;\to\;}
\begin{align*}
2 \qtoq 3-i \qtoq 4 \qtoq 3+i &\qtoq 2 \qtoq 1-i \qtoq 0 \qtoq 1+i \qtoq \\
2 \qtoq 3+i \qtoq 4 \qtoq 3-i &\qtoq 2 \qtoq 1+i \qtoq 0 \qtoq 1-i \qtoq 2.
\end{align*}
}
Of course, there are similar paths that are nicer
in the sense that they cross themselves only 
in a finite number of points~($5$?).

%%%%%%%%%%%%%%%%%
\end{enumerate}



 %%%%%%%%%%%%%%%%%%%%%%%%%%%%%%%%%%%%%%%%%%%%%%%%%%%%%%%%%%%%%%%%%%%%%%%%
%%%%%%%%%%%%%%%%%%%%%%%%%%%%%%%%%%%%%%%%%%%%%%%%%%%%%%%%%%%%%%%%%%%%%%%%
%%%%%%%%%%%%%%%%%%%%%%%%%%%%%%%%%%%%%%%%%%%%%%%%%%%%%%%%%%%%%%%%%%%%%%%%
%chapter 11
\chapterTypeout{Harmonic Functions}

%%%%%%%%%%%%%%%%%%%%%%%%%%%%%%%%%%%%%%%%%%%%%%%%%%%%%%%%%%%%%%%%%%%%%%%%
%%%%%%%%%%%%%%%%%%%%%%%%%%%%%%%%%%%%%%%%%%%%%%%%%%%%%%%%%%%%%%%%%%%%%%%%
\section{Notes}

%%%%%%%%%%%%%%%%%%%%%%%%%%%%%%%%%%%%%%%%%%%%%%%%%%%%%%%%%%%%%%%%%%%%%%%%
\index{Laplacian}
\subsection{The Laplacian}

\newcommand*{\partialby}[1]{\frac{\partial}{\partial #1}}
\newcommand*{\fracpart}[2]{\frac{\partial #1}{\partial #2}}
\newcommand*{\dpartial}[2]{\frac{\partial^2 #1}{\partial #2^2}}
\newcommand*{\px}{\partialby x}
\newcommand*{\py}{\partialby y}

Assuming \(f = u + iv\) and \(f_{xy} = f_{yx}\) we have
\begin{align*}
4\partial\tilde{\partial}f 
 &= 4\partial\left(\half\left(\px + i\py\right)(u + iv)\right)
  = 2\partial\bigl(u_x + iv_x + i(u_y + iv_y)\bigr) \\
 &= \left(\px - i\py\right)\bigl(u_x - v_y + i(v_x + u_y)\bigr) \\
 &= u_{xx} - v_{yx}  + i(v_{xx} + u_{yx}) 
    -i\bigl(u_{xy} - v_{yy} + i(v_{xy} + u_{yy})\bigr) \\
 &= (u + iv)_{xx} + (u + iv)_{yy} = \Delta f.
\end{align*}

\subsubsection{Polar Coordinates}

Using
\begin{equation*}
x = r\cos\theta \qquad
y = r\sin\theta \qquad
r = \sqrt{x^2+y^2} \qquad
\tan \theta = y/x
\end{equation*}
Asssume \(u(x,y)\) is sufficiently differentiable with continuous
partial derivatives.

Using the chain rule
\begin{equation*}
\fracpart{u}{x} 
  = \cos \theta \fracpart{u}{r} 
    - \frac{\sin \theta}{r}\fracpart{u}{\theta} 
  \qquad
\fracpart{u}{y} 
  = \sin \theta \fracpart{u}{r} 
    + \frac{\cos \theta}{r}\fracpart{u}{\theta}
\end{equation*}

Continuing
\begin{align*}
\dpartial{u}{x}
 &= \cos^2\theta \dpartial{u}{r}
    - \frac{2\sin\theta \cos\theta}{r} 
      \frac{\partial^2 u}{\partial r\,\partial \theta}
    + \frac{\sin^2 \theta}{r^2} \dpartial{u}{\theta}
    + \frac{\sin^2 \theta}{r} \fracpart{u}{r}
    + \frac{2\sin\theta \cos\theta}{r^2}\fracpart{u}{\theta} \\
\dpartial{u}{y}
 &= \sin^2\theta \dpartial{u}{r}
    + \frac{2\sin\theta \cos\theta}{r} 
      \frac{\partial^2 u}{\partial r\,\partial \theta}
    + \frac{\cos^2 \theta}{r^2} \dpartial{u}{\theta}
    + \frac{\cos^2 \theta}{r} \fracpart{u}{r}
    - \frac{2\sin\theta \cos\theta}{r^2}\fracpart{u}{\theta}
\end{align*}
Adding the above gives
\begin{equation*}
\Delta u 
= \dpartial{u}{x} + \dpartial{u}{y}
 = \dpartial{u}{r} + \frac{1}{r}\fracpart{u}{r} 
    + \frac{1}{r^2}\dpartial{u}{\theta}
 = \frac{1}{r}\partialby{r}\left(r\fracpart{u}{r}\right)
    + \frac{1}{r^2}\dpartial{u}{\theta}
\end{equation*}


%%%%%%%%%%%%%%%%%%%%%%%%%%%%%%%%%%%%%%%%%%%%%%%%%%%%%%%%%%%%%%%%%%%%%%%%
\subsection{Proof of Theorem~11.9}

The proof of Theorem~11.9 refers to Theorem~10.7
for showing that $f$ is holomorphic. But instead, it should 
refer to Exercise~16 of Chapter~10 (\ref{ex:10.16}).

%%%%%%%%%%%%%%%%%%%%%%%%%%%%%%%%%%%%%%%%%%%%%%%%%%%%%%%%%%%%%%%%%%%%%%%%
\subsection{Harnack's Theorem}

\index{Harnack}
Let's look a the proof of Harnack's Theorem~11.11.
The first double inequality has a minor error.
The middle expression is missing square sign in the denominator.
It should be
\begin{equation*}
\frac{R^2 - r^2}{R^2 - 2rR\cos(\theta - t) + r^{\mathbf{2}}}
\end{equation*}

% The proof of Harnack's Theorem~11.11, specifically 
The second
double-inequality makes use of \textbf{11.10}(1).

\iffalse
of the following observation.

Say $u$ is a real harmonic function, so \(u(z) = \Re(f(z)\) for some
\(f\in H(\Omega)\). 
Let \(\gamma(t) = a + Re^{it}\)
and \(\Gamma = \{\gamma(t): -\pi \leq t < \pi\}\). Now
\(\gamma`(t) = Rie^{it}\) and
\begin{align*}
u(a) 
 &= \Re(f(a))
 = 
   \Re\left(
     \dtwopii
     \int_{\Gamma} \frac{f(a+z)}{(a+z)-a}\,dz
   \right) 
 \\
 &= \frac{1}{2\pi} 
   \Re\left(
    \frac{1}{i}
     \int_{-\pi}^\pi \frac{f(a+\gamma(t))}{Re^{it}}\cdot Rie^{it}\,dt
   \right).
 \\
 &= \frac{1}{2\pi} 
   \Re\left(
     \int_{-\pi}^\pi f(a+\gamma(t))\,dt
   \right).
\end{align*}
\fi


%%%%%%%%%%%%%%%%%%%%%%%%%%%%%%%%%%%%%%%%%%%%%%%%%%%%%%%%%%%%%%%%%%%%%%%%
%%%%%%%%%%%%%%%%%%%%%%%%%%%%%%%%%%%%%%%%%%%%%%%%%%%%%%%%%%%%%%%%%%%%%%%%
\section{The Exercises} % pages 249-252

%%%%%%%%%%%%%%%%%
\begin{enumerate}
%%%%%%%%%%%%%%%%%

%%%%%%%%%%%%%% 
\begin{excopy}
Suppose $u$ and $v$ are real harmonic functions in a plane regular \(\Omega\).
Under what conditions is \(uv\) harmonic?
(Note that the answer depends strongly on the fact that the question
is one about \emph{real} functions.)
Show that \(u^2\) cannot be harmonic in \(\Omega\), unless $u$ is constant.
For which \(f \in H(\Omega)\) is \(|f|^2\) harmonic?
\end{excopy}

Compute
\begin{equation*}
(uv)_{xx} 
= \left((uv)_x\right)_x
= \left(u_xv + uv_x\right)_x
= u_{xx}v + 2u_xv_x + uv_{xx}
\end{equation*}

Hence
\begin{align*}
\Delta(uv) 
 &= u_{xx}v + 2u_xv_x + uv_{xx} + u_{yy}v + 2u_yv_y + uv_{yy} \\
 &= \Delta(u)v + u\Delta(v) + 2(u_xv_x + u_yv_y).
\end{align*}

So \(uv\) is harmonic when \(u_xv_x + u_yv_y = 0\) in \(\Omega\).
Clearly \(\Delta(u^2) = 2(u_x^2 + u_y^2) \geq 0\)
and equality holds iff \(u_x = u_y = 0\).

%%%%%%%%%%%%%% 
\begin{excopy}
Suppose $f$ a complex function in a region \(\Omega\) and both
$f$ and \(f^2\) are hannonic in \(\Omega\). Prove that
either $f$ or \(\overline{f}\) are holnmorphic in \(\Omega\).
\end{excopy}

By the previous exercise, \((f_x)^2 + (f_y)^2 = 0\).
Regardless whether \(f_x = f_y = 0\) holds or not, we have
\begin{equation*}
f_x = \pm i\,f_y.
\end{equation*}
By continuity, only one variant of \(\pm\) holds for all \(\Omega\).
Put \(f = u + iv\) with real valued functions $u$ and $v$.
Now
\begin{equation*}
u_x + iv_x = -v_y + iu_y
\qquad\textnormal{or}\qquad
u_x + iv_x = v_y - iu_y.
\end{equation*}
Hence the Cauchy-Riemann equation holds for \(\overline{f}\) or $f$.

%%%%%%%%%%%%%% 3
\begin{excopy}
If $u$ is a harmonic function in a region \(\Omega\),
what can you say about the set of points at which the
gradient of $u$ is $0$? (Thus is the set which \(u_x = u_y = 0\).)
\end{excopy}

% From Section~1.11, every harmonic function has continuous partial derivative
% of all orders.
We will show that
the vanishing set $K$ is either the whole region where $u$ is constant
in each connected component of \(\Omega\),
or it consists of just isolated points.

By Section~11.10, every real harmonic function is locally the real part
of holomorphic function.

Assume $u$ is harmnonic in a connected region
(\wlogy, it is \(\Omega\) and the set
\begin{equation*}
 K = \{z\in\Omega: u_x(z) = u_y(z) = 0\}
\end{equation*}
has accumulation point \(z_0 \in \Omega\).
Note that both \(\Re(u)\) and \(\Im(u)\) are harmnonic.
By the previous, remark there exist a neighborhood \(V\subset\Omega\)
and holomorphic functions $f$ and $G$ defined on $V\ni z_0$ such that
\begin{equation*}
 u(z) = \Re(f(z)) + i\Re(g(z)) \qquad (z\in V).
\end{equation*}
Now for \(z\in V\)
\begin{align*}
u_x(z) &= \Re(f_x(z)) + i\Re(g_x(z)) = (\Re(f))_x(z) + i(\Re(g))_x(z) \\
u_y(z) &= \Re(f_y(z)) + i\Re(g_y(z)) = (\Re(f))_y(z) + i(\Re(g))_y(z)
\end{align*}
and Cauchy-Riemann equations show that 
\begin{alignat*}{2}
\Re(u)_x &= \Re(f)_x = \Im(f)_y &\qquad \Im(u)_x &= \Re(g)_x = \Im(g)_y \\
\Re(u)_y &= \Re(f)_y = -\Im(f)_x &\qquad \Im(u)_y &= \Re(g)_y = -\Im(g)_x 
\end{alignat*}
For \(z\in K\), the above functions vanish, hence
\begin{align*}
f'(z) = g'(z) = 0 \qquad (z\in K).
\end{align*}
By Theorem~10.18 \(f'(z)=g'(z) = 0\) for all \(z\in V\)
and consequently for all \(z\in \Omega\).
Hence $f$ and $g$ are constant functions and so is $u$.

%%%%%%%%%%%%%% 4
\begin{excopy}
Prove that every partial derivative of every harmonic function is harmonic.

Verify, by direct computation, that \(P_r(\theta - t)\) is, for each fixed $t$,
a harmonic function of \(re^{i\theta}\).
Deduce (without referring to holomorphic functions) that the Poisson integral
\(P[d\mu]\) of every finite
Borel measure \(\mu\) on $T$ is harmonic in $U$, by showing that every partial
derivative of \(P[d\mu]\) is equal to
the integral of the corresponding partial derivative of the kernel.
\end{excopy}

See also \cite{Lang199304} Chapter~\textsf{VIII} Section~\S3 Example~3.

Let
\begin{equation*}
f(z) = \frac{e^{it} + z}{e^{it} - z} = \frac{u + x+iy}{u - x - iy}.
\end{equation*}

First order differentiation:
\begin{align*}
\partialby{x}f(z) 
&= \partialby{x} \frac{u + x+iy}{u - x - iy}
 = \frac{(u - x - iy) + (u + x+iy)}{(u-z)^2}
 = \frac{2u}{(u-z)^2}
 \\
\partialby{y}f(z) 
&= \partialby{y} \frac{u + x+iy}{u - x - iy}
 = \frac{i(u - x - iy) + i(u + x+iy)}{(u-z)^2}
 = \frac{2iu}{(u-z)^2}
\end{align*}

Second order differentiation:
\begin{align*}
\dpartial{}{x}f(z) 
 &= \partialby{x}\frac{2u}{(u-x-iy)^2}
  = \frac{-2u(2x+2iy-2u)}{(u-z)^2}
  = 4u\frac{(-x-iy+u)}{(u-z)^2}
 \\
\dpartial{}{y}f(z) 
 &= \partialby{y}\frac{2iu}{(u-x-iy)^2}
  = \frac{-2iu(2ix-2y-2iu)}{(u-z)^2}
  = 4u\frac{(x+iy-u)}{(u-z)^2}
\end{align*}

By summing the above, and noting that the partial differentiation
commutes with the \(\Re\) operator,
we get 
\begin{align*}
\Delta P_r(\theta - t)
= \left(\dpartial{}{x}+\dpartial{}{y}\right) \Re(f(z))
= \Re\left(\left(\dpartial{}{x}+\dpartial{}{y}\right) f(z)\right)
= \Re(0) = 0.
\end{align*}

Let us explicitly define 
\begin{equation*}
P[d\mu](re^{i\theta}) = \dtwopii \int_{-\pi}^\pi P_r(\theta - t)\,d\mu(t).
\end{equation*}

Put \(m(t) = \mu(\{x: -\pi < x < t\})\).
We can find 4 increasing functions \(a_j(t)\) such that
\begin{equation*}
m(t) = (a_1(t) - a_2(t)) + i\left(a_3(t) - a_4(t)\right).
\end{equation*}
Applying Theorem~9.42 of \cite{RudinPMA85} separately to each \(a_i\)
and summing we can generalize that theorem so we can use \(m(t)\)
in place of \(a(t)\) in that theorem.
Thus

\iffalse
Let $f$ be a harmonic function.
Since its real and imaginary parts are real parts of holomorphic functions
(by Section~11.10), $f$ itself has partial derivatives of all orders.
By Theorem~9.41 \cite{RudinPMA85} we can change the order
of partial derivative axes. Hence
\begin{align*}
\Delta(f_x)
&= (f_x)_{xx} + (f_x)_{yy}
 = f_{xxx} + f_{xyy}
 = f_{xxx} + f_{yyx}
 = \left(f_{xx} + f_{yy}\right)_x
 = \left(\Delta(f)\right)_x
 = 0.
\end{align*}
Similarly we agve \(\Delta(f_y) = 0\).

Putting \(z = re^{i\theta} = x + iy\), we have:
\begin{equation*}
\renewcommand{\currentprefix}{ex11.4}
P(\theta -t) = \Re\left((e^{it}+z)/(e^{it}-z)\right)
\end{equation*}
Thus \(P(z) \in H(U)\).

Let \(e^{it} = x_t + iy_t\), then
\begin{align*}
 P_r(\theta -x)
&= \Re\left((x_t + iy_t + x+iy)/\left(x_t + iy_t -(x+iy)\right)\right) \\
&= \Re\left(((x_t+x) + i(y_t + y))/\left((x_t-x) + i(y_t - y)\right)\right) \\
&= \Re\left(\left((x_t+x) + i(y_t + y)\right)
           \cdot\left((x_t-x) - i(y_t - y)\right)
         / \left((x_t-x)^2 + (y_t - y)^2\right)\right) \\
&= \Re\frac{(x_t^2 - x^2) + (y_t^2 - y^2)
           + i\left((x_t-x)(y_t + y) + (x_t+x)(y_t - y)\right)}{
             (x_t-x)^2 + (y_t - y)^2} \\
&= (x_t^2 - x^2 + y_t^2 - y^2) / \left((x_t-x)^2 + (y_t - y)^2\right)
\end{align*}

Thus
\begin{align*}
\px P_r(\theta -x)
=& \px \left((x_t^2 - x^2) - (y_t^2 - y^2)\right)
   \bigm/ \left((x_t-x)^2 + (y_t - y)^2\right) \\
=&  \left(-2x \left((x_t-x)^2 + (y_t - y)^2\right)
    - \left((x_t^2 - x^2) - (y_t^2 - y^2)\right)(2x -2)\right)
    \\
 & \bigm/
     \left((x_t-x)^2 + (y_t - y)^2\right)^2 \\
=& - (4x + 2)\left((x_t-x)^2 + (y_t - y)^2\right)
     \bigm/ \left((x_t-x)^2 + (y_t - y)^2\right)^2 \\
\end{align*}
\fi

\iffalse
Looking at a kernel $k$, we freely use \(k(z) = k(r,\theta)\).
Asuming the given conditions, we want to show
\begin{equation} \locallabel{eq:need:lim}
\frac{\partial}{\partial x}\left( \int_{-\pi}^\pi k(r,\theta -t)\,d\mu(t)\right)
= 
\int_{-\pi}^\pi \frac{\partial}{\partial x}\left( k(r,\theta -t)\right)\,d\mu(t)
\end{equation}

It is sufficient to show that for each nonzero real sequence \(\{h_n\}_{n\in\N}\)
such that \mbox{\(\lim_{n\to\infty} h_n = 0\)}.
Using 
\begin{equation*}
z + h_n = r_n e^{i\theta_n}
\end{equation*}
we need to show
\begin{multline} \locallabel{need:limn}
\lim_{n\to\infty}\frac{1}{h_n}
 \left( \int_{-\pi}^\pi 
  \bigl(k(r,\theta -t)-k(r_n, \theta_n-t)\bigr)\,d\mu(t)\right)
 \\
 =
\int_{-\pi}^\pi 
  \left( 
    \lim_{n\to\infty}\frac{1}{h_n}
      \bigl(k(r,\theta -t)-k(r_n, \theta_n-t)\bigr)
  \right)
  \,d\mu(t)
\end{multline}

% Consider Theorem 9.42 in Rudin's PMA

% Consider cases:  z=0,   z\neq 0
Two cases. \\
\textbf{Case~1.} \(z\in\R\). \\
Then \(\theta = \theta_n = 0\). Now \localeqref{need:limn} becomes
\begin{equation} \locallabel{eq:need:lim}
\frac{\partial}{\partial x}\left( \int_{-\pi}^\pi k(r, -t)\,d\mu(t)\right)
= 
\int_{-\pi}^\pi \frac{\partial}{\partial x}\left( k(r, -t)\right)\,d\mu(t)
\end{equation}
\fi

\iffalse
\textbf{Case~1.} \(z=0\). \\
Then \(r=0\), \(\theta = \theta_n = 0\), \(h_n = r_n\)
and \eqref{eq:11.4:need:limn} becomes
\begin{equation*}
\lim_{n\to\infty}\frac{1}{h_n}
  \left( \int_{-\pi}^\pi k(0)-k(h_n, -t)\bigr)\,d\mu(t)\right)
 =
\int_{-\pi}^\pi 
  \left( \lim_{n\to\infty}\frac{1}{h_n} \bigl(k(0)-k(h_n, -t)\bigr)  
  \right)  \,d\mu(t)
\end{equation*}

The ``directonal'' derivative
\begin{equation*}
 \lim_{n\to\infty}\frac{1}{h_n} \bigl(k(0)-k(h_n, -t)\bigr)
 = e^{i\Arg(t)}\cdot k'(0).
\end{equation*}

\textbf{Case~2.} \(z\neq 0\).
\fi

\unfinished

%%%%%%%%%%%%%% 6
\begin{excopy}
Suppose \(f \in H(\Omega)\) and $f$ has no zero in \(\Omega\).
Prove that \(\log|f|\) is harmonic in \(\Omega\), by computing its
Laplacian. Is there an easier way?
\end{excopy}

\begin{align*}
\log(|f|)
 &= \log\left( \left(\left((f+\overline{f})/2\right)^2 
    + \left((\overline{f} - f)/2\right)^2\right)^{\half}\right)
 = \log\left( \left(\sqrt{2}/2\right)\left(f^2 
    + {\overline{f}}^2\right)^\half\right) \\
 &= \half \log\left(f^2 + {\overline{f}}^2\right) + \log(\sqrt{2}/2)
\end{align*}

\unfinished

%%%%%%%%%%%%%% 
\begin{excopy}
Suppose \(f \in H(U)\), where $U$ is the open unit disc, $f$ is one-to-one in
$U$, \(\Omega = f(U)\), and \(f(z) = \sum c_n z^n\).
Prove that the area of \(\Omega\) is 
\begin{equation*}
\pi \sum_{n=1}^\infty n |c_n|^2.
\end{equation*}

Hint: The Jacobian of $f$ is \(|f'|^2\).
\end{excopy}


%%%%%%%%%%%%%% 
\begin{excopy}
\ich{a} If \(f \in H(\Omega)\), \(f(z) \neq 0\) for \(z \in \Omega\),
and \(=\infty \alpha < \infty\), prove that
\begin{equation*}
\Delta(|f|^\alpha) = \alpha^2 |f|^{\alpha-2} |f'|^2,
\end{equation*}
by proving the formula
\begin{equation*}
\partial\overline{\partial}(\psi \circ (f\overline{f})) 
 = (\varphi \circ |f|^2)\cdot|f'|^2,
\end{equation*}
in which \(\psi\) twice differentiable on \((0, \infty)\) and
\begin{equation*}
\varphi(t) = t \psi''(t) + \psi'(t).
\end{equation*}

\ich{b}
Assume \(f \in H(\Omega)m\) and \(\Phi\) is a complex function with domain
\(f(\Omega)\), which has continuous
second-order partial derivatives. Prove that
\begin{equation*}
 \Delta[\Phi \circ f] = [(\Delta \Phi) \circ f] \cdot|f'|^2.
\end{equation*}
Show that this specializes to the result of \ich{a} 
if \(\Phi(w) = \Phi(|w|)\).
\end{excopy}


%%%%%%%%%%%%%% 8 
\begin{excopy}
Suppose \(\Omega\) is a region, \(f_n\in H(\Omega)\) for \(n=1,2,3,\ldots\),
\(u_n\) is the real part off \(f_n\), \(\{u_n\}\) converges 
uniformly on compact subsets of \(\Omega\), and \(\{f_n(z)\}\) converges for
 at least one \(z \in \Omega\). Prove that then \(\{f_n\}\)
converges uniformly on compact subsets of \(\Omega\).
\end{excopy}


%%%%%%%%%%%%%% 
\begin{excopy}
Suppose $u$ is a Lebesgue measurable function in a region \(\Omega\), and $u$ 
is locally in \(L^1\). This means that
the integral of \(|u|\) over any compact subset of \(\Omega\) is finite.
 Prove that $u$ is harmonic if it satisfies the
following form of the mean value property:
\begin{equation*}
u(a) = \frac{1}{\pi r^2} \iint\limits_{D(a;r)}  u(x, y)\,dx\,dy
\end{equation*}
whenever \(\overline{D}(a;r) \subset \Omega\).
\end{excopy}


%%%%%%%%%%%%%% 10
\begin{excopy}
Suppose \(I=[a,b]\) is an interval on the real axis, 
\(\varphi\) is a continuous function on $I$, and
\begin{equation*}
f(z) = \dtwopii \int_a^b \frac{\varphi{t}}{t - z}\,dt \qquad (z \notin I).
\end{equation*}
Show that
\begin{equation*}
\lim_{\epsilon\to 0}[f(x+i\epsilon) - f(x-i\epsilon)] \qquad (\epsilon > 0)
\end{equation*}
exists for every real $x$, and find it in terms of \(\varphi\).

How is the result affected if we assume merely that \(\varphi \in L^1\)?
What happens then at points $x$ at
which \(\varphi\) has right- and left-hand limits?
\end{excopy}


%%%%%%%%%%%%%% 
\begin{excopy}
Suppose that \(I=[a,b]\), \(\Omega\) is a region, \(I \subset \Omega\),
$f$ is continuous in \(\Omega\), and \(f \in H(\Omega - I)\). Prove that
actually \(f \in H(\Omega)\).

Replace $I$ by some other sets for which the same conclusion can be drawn.
\end{excopy}


%%%%%%%%%%%%%% 
\begin{excopy}
\index{Harnack} (Harnack‘s Inequalities) 
Suppose \(\Omega\) is a region, $K$ is a compact subset of \(\Omega\),
\(z_0 \in \Omega\), Prove that
there exist positive numbers \(\alpha\) and \(\beta\) 
(depending on \(z_0\), $K$, and \(\Omega\)) such that
\begin{equation*}
\alpha u(z_0) \leq u(z) \leq \beta u(z_0)
\end{equation*}
for every positive harmonic function $u$ in \(\Omega\) and for all \(z \in K\).

If \(\{u_n\}\) is a sequence of positive harmonic functions in \(\Omega\)
 and if \(u_n(z_0)\to 0\), describe the behavior
of \(\{u_n\}\) in the rest of \(\Omega\). Do the same if \(u_n(z_0)\to \infty\).
 Show that the assumed positivity of \(\{u_n\}\) is
essential for these results.
\end{excopy}


%%%%%%%%%%%%%% 13
\begin{excopy}
Suppose $u$ is a positive harmonic function in $U$ and \(u(0) = 1\).
How large can \(u(\half)\) be? How small?
Get the best possible bounds.
\end{excopy}


%%%%%%%%%%%%%% 
\begin{excopy}
For which pairs of lines \(L_1\), \(L_2\) do there exist real functions,
hamonic in the whole plane, that are
$0$ at all points of \(L_1 \cup L_2\) without vanishing identically?
\end{excopy}


%%%%%%%%%%%%%% 
\begin{excopy}
suppose $u$ is a positive harmonic function in $U$, 
and \(u(re^{i\theta}) \to 0\) as \(r\to 1\), for every \(e^{i\theta} \neq 1\).
Prove
that there is a constant $c$ such that
\begin{equation*}
u(re^{i\theta}) = cP_r(\theta).
\end{equation*}
\end{excopy}


%%%%%%%%%%%%%% 16 
\begin{excopy}
Here is an example of a harmonic function in $U$ which is not identically $0$ but all of whose radial
limits are $0$:
\begin{equation*}
u(z) = \Im\left[\left(\frac{1+2}{1-z}\right)^2\right].
\end{equation*}/
Prove that this $u$ is not the Poisson integral of any measure on $T$ 
and that it is not the difference of
two positive harmonic functions in $U$.
\end{excopy}


%%%%%%%%%%%%%% 
\begin{excopy}
Let \(\Phi\) be the set of all positive harmonic functions $u$ in $U$
 such that \(u(0) = 1\). Show that \(\Phi\) is 
a~convex set and find the extreme points of \(\Phi\). (A point $x$ in a convex
 set \(\Phi\) is called an extreme point of
\(\Phi\) if $x$ lies on no segment both of whose end points lie in \(\Phi\)
 and are different from $x$.) \emph{Hint}: If $C$ is the
convex set whose members are the positive Borel measures on $T$,
 of total variation $1$, show that the
extreme points of $C$ are precisely those \(\mu \in C\)
 whose supports consist of only one point of $T$.
\end{excopy}


%%%%%%%%%%%%%% 
\begin{excopy} 18
Let \(X^*\) be the dual space of the Banach space $X$. 
A sequence \(\{\Lambda_n\}\) in \(X^*\) is said to converge weakly
to \(\Lambda \in X^*\) if \(\Lambda_n x \to \Lambda x\) as \(n \to \infty\),
 for every \(x \in X\). Note that \(\Lambda_n \to \Lambda\) weakly whenever
\(\Lambda_n \to \Lambda\) in the
norm of \(X^*\). (See Exercise~8, Chap.~5.) The converse need not be true.
 For example, the functionals
\(f\to \hat{f}(n)\) on \(L^2(T)\) tend to $0$ weakly (by the Bessel inequality),
 but each of these functionals has norm $1$.
Prove that \(\{\| \Lambda_n\|\}\) must be bounded if \(\{\Lambda_n\}\)
 converges weakly.
\end{excopy}


%%%%%%%%%%%%%% 19
\begin{excopy}
\ich{a} Show that \(\delta P_r(\delta) > 1\) if \(\delta = 1 - r\).\\
\ich{b} If \(\mu \geq 0\), \(u = P[d\mu]\), and \(I_\delta \subset T\)
is the are with center $1$ and length \(2\delta\), show that
\begin{equation*}
\mu(I_\delta) \leq \delta\mu(1 - \delta)
\end{equation*}
and that therefore
\begin{equation*}
(M\mu)(1) \leq \pi(M_{\textnormal{rad}}\,u)(1).
\end{equation*}
\ich{c} If, furthermore, \(\mu \perp m\), show that
\begin{equation*}
u(re^{i\theta}) \to \infty \qquad \aded\,[\mu].
\end{equation*}
\emph{Hint}: Use Theorem~7.15.
\end{excopy}


%%%%%%%%%%%%%% 20
\begin{excopy}
Suppose \(E \subset T\), \(m(E) = 0\).
 Prove that there is an \(f \in H^\infty\), with \(f(0) = 1\), that has
\begin{equation*}
\lim_{r\to 1} f(re^{i\theta}) = 0
\end{equation*}
at every \(e^{i\theta} \in E\).

\emph{Suggestion}: Find a lower semicontinuous 
 \(psi \in L^1(T)\), \(\psi > 0\), \(\psi = +\infty\) at every point of $E$.
 There
is a holomorphic $g$ whose real part is \(P[\psi]\). Let \(f= 1/g\).
\end{excopy}


%%%%%%%%%%%%%% 21
\begin{excopy}
Define \(f \in H(U)\) and \(g \in H(U)\) by 
\(f(z) = \exp \{(1 + z)/(1 - z)\}\), \(g(z) = (1 - z) \exp \{-f(z)\}\). Prove
that
\begin{equation*}
g^*(e^{i\theta}) = \lim_{r\to 1} g(re^{i\theta})
\end{equation*}
exists at every \(e^{i\theta} \in T\), that \(g^* \in C(T)\),
 but that $g$ is not in \(H^\infty\).

\emph{Suggestion}: Fix $s$, put
\begin{equation*}
 z_t = \frac{t + is - 1}{t + is + 1} \qquad (0 < t < \infty).
\end{equation*}
For certain values of $s$, \(|g(z_t)| \to \infty\) as \(t \to \infty\).

\end{excopy}


%%%%%%%%%%%%%% 22
\begin{excopy}
Suppose $u$ is harmonic in $U$, and 
\(\{u_r: 0 \leq r < 1\}\) is a uniformly integrable subset of 
\(L^1(T)\). (See
Exercise~10, Chap.~6.) Modify the proof of Theorem~11.30 to show that
\(u = P[f]\) for some \(f \in L^1(T)\).

\end{excopy}


%%%%%%%%%%%%%% 23
\begin{excopy}
Put \(\theta_n = 2^{-n}\) and define
\begin{equation*}
u(z) = \sum_{n=1}^\infty n^{-2}\{P(z,e^{i\theta_n}) - P(z,e^{-i\theta_n})\},
\end{equation*}
for \(z \in U\). Show that $u$ is the Poisson integral of a measure on $T$,
 that \(u(x) = 0\) if \(-1 < x < 1\), but
that
\begin{equation*}
u(1 — \epsilon + i\epsilon)
\end{equation*}
is unbounded, as \(\epsilon\) decreases to $0$. 
(Thus $u$ has a radial limit, but no nontangential limit, at $1$.)

\emph{Hint}: lf \(\epsilon = \sin \theta\) is small and 
\(z = 1 — \epsilon + i\epsilon\), then
\begin{equation*}
 P(z,e^{i\theta}) - P(z,e^{-i\theta}) > 1/\epsilon.
\end{equation*}
\end{excopy}


%%%%%%%%%%%%%% 24
\begin{excopy}
Let \(D_n(t)\) be the 
\index{Dirichlet} Dirichlet kernel, as in Sec.~5.11, define the 
\index{Fejer@Fej\'er} Fej\'er 
kernel by
\begin{equation*}
K_N = \frac{1}{N+1} (D_0 + D_1 + \cdots + D_n),
\end{equation*}
put \(L_N(t) = \min(N, \pi^2/Nt^2)\). Prove that
\begin{equation*}
K_{N_1}(t) = \frac{1}{N} \cdot \frac{1 - \cos Nt}{1 - \cos t} \leq L_N(t)
\end{equation*}
and that \(\int_R L_N\,d\sigma \leq 2\).

Use this to prove that the arithmetic means
\begin{equation*}
\sigma_N = \frac{S_0 + S_1 + \cdots + S_N}{N + 1}
\end{equation*}
of the partial sums \(s_n\) of the Fourier series of a functionf
\(f \in L^1(T)\) converge to \(f(e^{i\theta})\) at every Lebesgue
point of $f$ (Show that \(\sup |\sigma_N|\) is dominated by \(Mf\),
 then proceed as in the proof of Theorem~11.23.)
\end{excopy}

%%%%%%%%%%%%%% 25
\begin{excopy}
If \(1 \leq p \leq \infty\) and \(f \in L^1(\R^1)\), prove that
 \((f * h_\lambda)(x)\) is a harmonic function of \(x + i/\lambda\) in the upper
half plane. 
(\(h_\lambda\) is defined in Sec.~9.7; 
it is the Poisson kernel for the half plane.)
\end{excopy}

%%%%%%%%%%%%%%%%%
\end{enumerate}



 %%%%%%%%%%%%%%%%%%%%%%%%%%%%%%%%%%%%%%%%%%%%%%%%%%%%%%%%%%%%%%%%%%%%%%%%
%%%%%%%%%%%%%%%%%%%%%%%%%%%%%%%%%%%%%%%%%%%%%%%%%%%%%%%%%%%%%%%%%%%%%%%%
%%%%%%%%%%%%%%%%%%%%%%%%%%%%%%%%%%%%%%%%%%%%%%%%%%%%%%%%%%%%%%%%%%%%%%%%
\chapterTypeout{The Maximum Modulus Principle}


 %%%%%%%%%%%%%%%%%%%%%%%%%%%%%%%%%%%%%%%%%%%%%%%%%%%%%%%%%%%%%%%%%%%%%%%%
%%%%%%%%%%%%%%%%%%%%%%%%%%%%%%%%%%%%%%%%%%%%%%%%%%%%%%%%%%%%%%%%%%%%%%%%
%%%%%%%%%%%%%%%%%%%%%%%%%%%%%%%%%%%%%%%%%%%%%%%%%%%%%%%%%%%%%%%%%%%%%%%%
\chapterTypeout{Approximations by Rational Functions}

%%%%%%%%%%%%%%%%%%%%%%%%%%%%%%%%%%%%%%%%%%%%%%%%%%%%%%%%%%%%%%%%%%%%%%%%
%%%%%%%%%%%%%%%%%%%%%%%%%%%%%%%%%%%%%%%%%%%%%%%%%%%%%%%%%%%%%%%%%%%%%%%%
\section{Notes}

%%%%%%%%%%%%%%%%%%%%%%%%%%%%%%%%%%%%%%%%%%%%%%%%%%%%%%%%%%%%%%%%%%%%%%%%
\subsection{Note in Theorem 13.6}

Theorem~13.6
considers \(S^2\) to be the complex plane and $K$ a compact subset.
Before the proof the text notes
that \(S^2 \setminus K\) has at most countably many components.

This can be shown by noting that \(G = S^2 \setminus K\)
is open, and for any \(z\in G\) there is an open neighborhood
\(V = \{\zeta\in \C: |\zeta - z|<\epsilon\}\)
for some \(\epsilon > 0\) and 
such that \(z \in V \subset G\). Clearly $V$ is connected
so it is a subset of a connected component of $G$.

Also, $V$ must contain
some point of the form \(x + iy\) where \(x,y\in\Q\).
Now since \(\C_{\Q} =\{x + iy: x,y\in\Q\}\) is a countable set,
there could not be more components than \(|\C_{\Q}|\).

\else
 % \setcounter{chapter}{0}  % -*- latex -*-
% $Id: rudinrca1.tex,v 1.2 2006/09/08 07:29:12 yotam Exp $

%%%%%%%%%%%%%%%%%%%%%%%%%%%%%%%%%%%%%%%%%%%%%%%%%%%%%%%%%%%%%%%%%%%%%%%%
%%%%%%%%%%%%%%%%%%%%%%%%%%%%%%%%%%%%%%%%%%%%%%%%%%%%%%%%%%%%%%%%%%%%%%%%
%%%%%%%%%%%%%%%%%%%%%%%%%%%%%%%%%%%%%%%%%%%%%%%%%%%%%%%%%%%%%%%%%%%%%%%%
\chapterTypeout{Abstract Integration}

%%%%%%%%%%%%%%%%%%%%%%%%%%%%%%%%%%%%%%%%%%%%%%%%%%%%%%%%%%%%%%%%%%%%%%%%
%%%%%%%%%%%%%%%%%%%%%%%%%%%%%%%%%%%%%%%%%%%%%%%%%%%%%%%%%%%%%%%%%%%%%%%%
\section{Notes}

%%%%%%%%%%%%%%%%%%%%%%%%%%%%%%%%%%%%%%%%%%%%%%%%%%%%%%%%%%%%%%%%%%%%%%%%
\subsection{Lebesgue's Monotone Convergence --- Proof Fix}

\index{Lebesgue}
In Theorem~1.26 (page~22), it says:
\begin{quotation}
 \mldots, there exists \(\alpha \in [0,\infty)\) such that
\end{quotation}
It should be:
\begin{quotation}
 \mldots, there exists \(\alpha \in [0,\infty]\) such that
\end{quotation}

%%%%%%%%%%%%%%%%%%%%%%%%%%%%%%%%%%%%%%%%%%%%%%%%%%%%%%%%%%%%%%%%%%%%%%%%
\subsection{Lebesgue's Dominated Convergence --- Variant}

\index{Lebesgue}
Theorem~1.34 on page 27 requires
\begin{equation*}
|f_n(x)| \leq g(x) \qquad \textrm{(}n=1,2,3,\ldots; x\in X\textrm{),}
\end{equation*}
Instead, it could require:
\begin{eqnarray*}
g_n &\to& g \\
\int_X g_n d\mu  &\to& \int_X g d\mu  \\
|f_n(x)| &\leq& g(x).
\end{eqnarray*}
Let's have it explicitly.
%%%%%%%%%%%%%%%%
\begin{llem} \label{lem:Lebesgue:domvar}
Suppose \(\{f_n\}\) is a sequence of complex measurable functions on $X$ such that
\begin{equation*}
 f(x) = \lim_{n\to\infty} f_n(x)
\end{equation*}
exists for every \(x\in X\). If there is a a sequence of functions \(\{g_n\}\)
in \(L^1(\mu)\) and a function \(g\in L^1(\mu)\)
such that
\begin{eqnarray*}
   |f_n(x)| &\leq& g_n(x) \qquad (\forall n\in\N,\;\forall x\in X) \\
  \lim_{n\to\infty} g_n(x) &=& g(x) \qquad (\forall x\in X) \\
  \lim_{n\to\infty} \int_X g_n\,d\mu  &=& \int_X g\,d\mu
\end{eqnarray*}
then
\begin{eqnarray}
  f &\in& L^1(\mu) \notag \\
  \lim_{n\to\infty} \int_X |f_n - f|\,d\mu  & = & 0 \label{eq:leb:domv1} \\
  \lim_{n\to\infty} \int_X f_n\,d\mu  & = & \int_X f\,d\mu \label{eq:leb:domv2}
\end{eqnarray}
\end{llem}
Note: The orginal theorem~1.34 (\cite{RudinRCA80}) follows using \(g_n=g\).

\begin{thmproof}
Since
\begin{equation*}
 |f| = \lim_{n\to\infty} |f_n| \leq \lim_{n\to\infty} |g_n| = |g|
\end{equation*}
and $f$ is measurable, \(f\in L^1(\mu)\).
Since \(|f-f_n|\leq g_n + g\),
Fatou's lemma applies to \(g + g_n - |f_n - f|\) and yields
\begin{eqnarray*}
 2 \int_X g\,d\mu
 &=& \int_X g\,d\mu
     + \int_X \lim_{n\to\infty}g_n
     - \int_X \lim_{n\to\infty}|f_n - f|\,d\mu \\
 &=& \int_X \left(\lim_{n\to\infty}(g + g_n  - |f_n - f|\right)\,d\mu \\
 &=& \int_X \liminf_{n\to\infty} g + g_n  - |f_n - f|\,d\mu \\
 &\leq& \liminf_{n\to\infty}\int_X g + g_n  - |f_n - f|\,d\mu \\
 &=& \int_X g\,d\mu
     + \lim_{n\to\infty} \int_X g_n\,d\mu
     \liminf_{n\to\infty} \left(-\int_X |f_n - f|\,d\mu\right) \\
 &=& 2\int_X g\,d\mu - \limsup_{n\to\infty} \int_X |f_n - f|\,d\mu .
\end{eqnarray*}
Since \(2\int_X g\,d\mu\) is finite
\begin{equation*}
 \liminf_{n\to\infty} \int_X |f_n - f|\,d\mu \leq 0.
\end{equation*}
which shows (\ref{eq:leb:domv1}).
The last assertion (\ref{eq:leb:domv2}) follows immediately from
\begin{equation*}
 \left| \int_X f_n\,d\mu - \int_X f\,d\mu\right|  \leq \int_X |f_n - f|\,d\mu.
\end{equation*}
\end{thmproof}





%%%%%%%%%%%%%%%%%%%%%%%%%%%%%%%%%%%%%%%%%%%%%%%%%%%%%%%%%%%%%%%%%%%%%%%%
%%%%%%%%%%%%%%%%%%%%%%%%%%%%%%%%%%%%%%%%%%%%%%%%%%%%%%%%%%%%%%%%%%%%%%%%
\section{Exercises} % pages 32-33

%%%%%%%%%%%%%%%%%
\begin{enumerate}
%%%%%%%%%%%%%%%%%

%%%%%%%%%%%%%%
\begin{excopy}
Does there exist an infinite \salgebra\ %
which has only countably many members?
\end{excopy}

No.

Say be negation a \salgebra\ \M\ in $X$ %
has countably many members \(\{A_i\}_{i=1}^\infty\).
We will build an infinite countable family out of \M,
mutually disjoint, that will be a base for \M.

For each \(x\in X\), let
\begin{equation} \label{eq:cntcap}
G_x = \bigcap_{x\in A_i} A_i.
\end{equation}
Clearly, \(x\in G_x\in \M\). The latter membership relation
by the face that the intersection in (\ref{eq:cntcap}) is
of at most countable number of sets.
We observe that if \(G_x \cap G_y \neq \emptyset\)
then \(G_x = G_y\).

By negation, if \(G_x \neq G_y\), then
\(G_x\setminus G_y \neq \emptyset\)
or
\(G_y\setminus G_x \neq \emptyset\).
\(G_x\cap G_y \subsetneq G_y\)
Put \(G = G_x\cap G_y \in \M\)
and so
\(G\subsetneq G_x\)
or
\(G \subsetneq G_y\).
\Wlogy, \(G \subsetneq G_x\).
If \(x\in G\) then $G$ participates in the intersection of (\ref{eq:cntcap})
and thus \(G_x\subset G\) leading to the \(G_x \subsetneq G_x\) contradiction.
Otherwise, \(x\neq G\) and so \(x \in G_y^c \in \M\)
and so \(G_y^c\) participates in (\ref{eq:cntcap}) and so
\(x\in G_x \subset G_y^c = G_x^c\) which is also a contradiction.

Thus the family \(\calB = \{G_x\}_{x\in X}\) is a subset of \M\
of disjoint (ignoring repetitions) subsets of $X$.
For each \(A\in \M\) we define
\begin{equation*}
A' = \bigcup_{x\in A} G_x.
\end{equation*}
Clearly, \(A\subset A'\). To show the reverse inclusion,
let \(w\in A'\).
Then for some \(x\in A\) we have \(w \in G_X\).
But this means that for every set \(S\in M\),
if \(x\in S\) then \(w\in S\). In particular, \(x\in A\) and so
\(A'\subset A\), hence \(A=A'\).

We showed that every set in \M\ is a disjoint union
of some subset of \calB.
Thus, \calB\ cannot be finite and by being subset of \M\
is countable. Therefore, the union of every subset of
\calB\ is in \M, and by being disjoint
\(|\M| = 2^{|\calB|} > \aleph_0\).


%%%%%%%%%%%%%%
\begin{excopy}
Prove and analogue of Theorem~1.8 for $n$ functions.
\end{excopy}

\begin{lthm}
Let \(\{u_i\}_{i=1}^n\) be real measurable functions on a measurable
space $X$, let
\(\Phi\)\ be a continuous mapped of \(\R^n\) into a topological space $Y$,
and define
\begin{equation*}
h(x) = \Phi(u_1(x),u_2(x),\ldots,u_n(x))
\end{equation*}
for \(x\in X\). Then \(h: X\to Y\) is measurable.
\end{lthm}
\begin{thmproof}
Put \(f(x) = (u_1(x),u_2(x),\ldots,u_n(x))\).

Then $f$ maps $X$ into \(\R^n\). Since
\(h = \Phi\circ f\), Theorem~1.7 shows that it is enough to prove
the measurability of $f$.

If $B$ is any open box in \(\R^n\), with sides parallel to the axes,
then $B$ is a cartesian product of $n$ segments \(\seqn{I}\), and
\begin{equation*}
f^{-1}(B) = \bigcap_{j=1}^n u_j^{-1}(I_j),
\end{equation*}
which is measurable, by our assertions on \(u_j\).
Every open set $V$ in \(\R^n\) is a countable union of such boxes \(B_i\),
and since
\begin{equation*}
f^{-1}(V) = f^{-1}\left(\bigcup_{i=1}^\infty B_i\right)
          = \bigcup_{i=1}^\infty f^{-1}(B_i),
\end{equation*}
\(f^{-1}\) is measurable.
\end{thmproof}

%%%%%%%%%%%%%%
\begin{excopy}
Prove that if $f$ is a real function on a measurable space $X$
such that \(\{x:f(x)\geq r\}\) is measurable for every rational $r$,
then $f$ is measurable.
\end{excopy}

Let \(\alpha\in \R\). Choose some decreasing sequence \(\{q_i\}_{i\in\N}\)
such that \(q_i\in\Q\) and \(q_i\to\alpha\). Now
\begin{equation*}
f^{-1}((\alpha,\infty]) = \bigcup_{i} f^{-1}((q_i,\infty]).
\end{equation*}
By Theorem~1.12(c), $f$ is measurable.

%%%%%%%%%%%%%%
\begin{excopy}
Let \(\{a_n\}\) and \(\{b_n\}\) be sequences in \([-\infty,\infty]\),
and prove the following assertions:

\begin{itemize}

\item[(a)] \qquad
   \(\displaystyle
      \limsup_{n\to \infty} (-a_n) =
     -\liminf_{n\to \infty} a_n
   \).

\item[(b)] \qquad
   \(\displaystyle
      \limsup_{n\to \infty} (a_n + b_n) \leq
      \limsup_{n\to \infty} a_n +
      \limsup_{n\to \infty} b_n\)

 provided none of the sums is of the form \(\infty - \infty\).

\item[(c)] If \(a_n\leq b_n\) for all $n$, then
 \[\liminf_{n\to\infty} a_n \leq \liminf_{n\to\infty} b_n\,.\]
\end{itemize}
Show by an example that strict inequality can hold in (b).
\end{excopy}

Let us have formalized definitions:
\begin{eqnarray*}
 \limsup_{n\to \infty} a_n  &=& \inf_{n\in\N}\, \sup_{i\geq n} a_i \\
 \liminf_{n\to \infty} a_n  &=& \sup_{n\in\N}\, \inf_{i\geq n} a_i
\end{eqnarray*}

\begin{itemize}
 \item[(a)]
  \begin{eqnarray*}
    \limsup_{n\to \infty} (-a_n)
     &=& \inf_{n\in\N}\, \sup_{i\geq n} -a_i \\
     &=& \inf_{n\in\N}\, -\inf_{i\geq n} a_i \\
     &=& -\sup_{n\in\N}\, \inf_{i\geq n} a_i \\
     &=& -\liminf_{n\to \infty} a_n
  \end{eqnarray*}

 \item[(b)]

 Let
 \begin{eqnarray}
  A_k &=& \sup_{n\geq k} a_n       \label{eq:Ak:limsup} \\
  B_k &=& \sup_{n\geq k} b_n       \label{eq:Bk:limsup} \\
  S_k &=& \sup_{n\geq k} a_n+b_n   \notag \\
 \end{eqnarray}
 If by negation, \(S_k > A_k + b_k\) then put
 \(\epsilon = S_k - (A_k+b_k)\).
 By definition, there exists \(m\geq k\) such that
 \begin{equation*}
 a_m+b_m > S_k - \epsilon \geq A_k + b_k
 \end{equation*}
 and so \(a_m > A_k\) or \(b_m > B_k\), but each is a contradiction.
 Hence,
 \begin{equation*}
  \sup_{n\geq k} a_n+b_n \leq \sup_{n\geq k} a_n + \sup_{n\geq k} b_n
 \end{equation*}
 for any \(k\in\N\).
 Clearly  \(\{A_k\}_{k\in\N}\) and \(\{A_k\}_{k\in\N}\)
 are decreasing and their infimum is their existing limit. Thus

 \begin{eqnarray*}
   \limsup_{n\to \infty} (a_n + b_n)
   &=& \inf_{n\in\N}\, \sup_{i\geq n} a_i+b_i \\
   &\leq& \inf_{k\in\N}\,   \left(\sup_{n\geq k} a_i
                          +       \sup_{n\geq k} b_i\right) \\
   &=& \inf_{k\in\N} A_k + B_k \\
   &=& \lim_{k\in\N} A_k + B_k \\
   &=& \lim_{k\in\N} A_k + \lim_{k\in\N} B_k \\
   &=& \inf_{k\in\N} A_k + \inf_{k\in\N} B_k \\
   &=& \limsup_{n\in\N} a_n + \limsup_{n\in\N} a_n
 \end{eqnarray*}

  Let \(a_n = (-1)^n n\) and \(b_n = (-1)^{n+1} n\) and
  so
  \(a_n+b_n = ((-1)^n + (-1)^{n+1}) n = 0\)
  But clearly \(\limsup a_n = \limsup b_n = \infty\).

 \item[(c)]
  Define the monotonic increasing sequences:
  \begin{eqnarray*}
  A_k &=& \inf_{n\geq k} a_n  \\
  B_k &=& \inf_{n\geq k} b_n  \\
  \end{eqnarray*}
  and so
  \begin{equation*}
  \liminf_{n\to\infty} a_n = \sup A_k = \lim A_k
      \leq \lim B_k = \sup B_k = \liminf_{n\to\infty} b_n\,.
  \end{equation*}

\end{itemize}

%%%%%%%%%%%%%%
\begin{excopy}
Prove that the set of points at which a sequence of measurable
real functions converges is a measurable set.
\end{excopy}

Let \(\{f_n\}_{n\in\N}\) be a sequence of measurable real functions on $X$.
Let \(\overline{f} = \limsup f_n\)
and \(\underline{f} = \liminf f_n\).
For any \(r\in \R\),
\begin{equation*}
E_r = \overline{f}^{-1}([r,\infty])
 = \bigcap_{n\in\N} \bigcup_{k\geq n} f_k^{-1}([r,\infty]).
\end{equation*}
Thus \(\overline{f}\) is measurable and similarly, so is \(\underline{f}\).
Now \(d = \overline{f} - \underline{f}\) is also measurable.
Surely
\begin{equation*}
E_0 = X \setminus d^{-1}\left([-\infty,0)\cup(0,\infty]\right) = d^{-1}(0)
\end{equation*}
is measurable. But \(E_0\) is exactly the set where
\(\overline{f}\) and \(\underline{f}\) agree
which is the set of point where \(\{f_n\}_{n\in\N}\) converge.


%%%%%%%%%%%%%%
\begin{excopy}
Let $X$ be an uncountable set, let \M\ be the collection of
all sets \(E\subset X\) such that either $E$ or \(E^c\)
is at most countable,
and define \(\mu(E)=0\) in the first case,
\(\mu(E)=1\) in the second.
Prove that \(\mu\) is a \salgebra\ in $X$ and that \(\mu\) is a measure on \M.
\end{excopy}

Clearly \(\emptyset, X\in \M\) and
\(\mu(\emptyset) = 0\)
and \(\mu(X) = 1\).
By definition, \M\ is closed under complement operation.
Now let \(\{A_n\}_{n\in\N}\) be a countable family in \M.
If some \(A_i\) is uncountable, then so is \(U = \cup A_i\in \M\)
and the \(\mu(U) = 1\).
Otherwise, since \(\aleph_0 \times \aleph_0 = \aleph_0\),
then \(|U| = |\cup A_i| = \aleph_0\) and so \(U\in M\) with \(\mu(U) = 0\).
Thus \M\ is a \salgebra.


%%%%%%%%%%%%%%
\begin{excopy}
Suppose \(f_n:X\to[0,\infty]\) is measurable for \(n=1,2,3,\ldots\),
\(f_1 \geq f_2 \geq f_3 \geq \cdots \geq 0\),
\(f_n(x)\to f(x)\) as \(n\to\infty\), for every \(x\in X\),
and \(f_1\in L^1(\mu)\).
Prove that then
\begin{equation*}
 \lim_{n\to\infty}\int_X f_n d\mu = \int_X fd\mu
\end{equation*}
and show that this conclusion does \emph{not} follow if the condition
``\(f_1\in L^1(\mu)\)'' is omitted.
\end{excopy}

\index{Lebesgue}
This is an application of Lebesgue's Dominated Convergence Theorem,
with \(g(x) = f_1(x)\) serving as the dominating function.

Say the condition \(f_1\in L^1(\mu)\)'' is omitted.
Let, \(X = [0,1]\) with the natural measure $m$, and for \(n>=1\), let
\begin{equation*}
f_n(x) = \left\{\begin{array}{l@{\qquad}l}
                \infty &  0\leq x<1/n \\
                0      &  1/n \leq x \leq 1
                \end{array}\right.
\end{equation*}
Now we easily see that \(f_n\to 0\) and so
\begin{equation*}
\int_{[0,1]} \lim_{n\to\infty} f_n(x)dm
= 0 < \infty
= \lim_{n\to\infty} \int_{[0,1]} f_n(x)dm.
\end{equation*}

%%%%%%%%%%%%%%
\begin{excopy}
Put \(f_n = \chhi_E\) if $N$ is odd, \(f_n = 1 - \chhi_E\) if $n$ is even.
What is the relevance of this example
\index{Fatou's lemma}
to Fatou's lemma.
\end{excopy}

In this case, the inequality of Fatou's Lemma becomes strict.
Say the measure is $m$ on $X$.
\begin{equation*}
\int_X \liminf_{n\to\infty} f_n(x)dm
= 0 < \min(\mu(E),\mu(X\setminus E))
= \liminf_{n\to\infty} \int_X f_n(x)dm.
\end{equation*}

%%%%%%%%%%%%%%
\begin{excopy}
Suppose \(\mu\) is a positive measure on $X$, \(f:X\to[0,\infty]\)
is measurable, \(\int_X fd\mu = c\), where \(0<c<\infty\),
and \(\alpha\) is a constant. Prove that
\begin{equation*}
\lim_{n\to\infty} \int_X n\log \left[ 1 + (f/n)^\alpha\right]d\mu
 =\left\{\begin{array}{l@{\qquad}l}
         \infty & \textrm{if }\; 0<\alpha<1,\\
         c      & \textrm{if }\; \alpha = 1,\\
         0      & \textrm{if }\; 1 < \alpha < \infty.
         \end{array}
         \right.
\end{equation*}


\emph{Hint}: If \(\alpha\geq 1\), then the integrands are dominated by
\(\alpha f\).
If \(\alpha<1\),
\index{Fatou's lemma}
Fatou's lemma  can be applied.
\end{excopy}

First we need to show some inequalities that
will establish the hint.
\begin{llem} \label{llem:apbp:geq}
If \(a \geq b \geq 0\) and \(a,\alpha\geq 1\), then
\(a^\alpha - b^\alpha \geq a - b\).
\end{llem}

\begin{thmproof}
Since \(a\geq b\) and \(\alpha-1\geq 0\) we have
\(a^{\alpha-1} \geq b^{\alpha-1}\) and so
\begin{equation*}
a^{\alpha-1} - 1 \geq b^{\alpha-1} - 1
\end{equation*}
The left hand side must be non negative.
We look at the sign of the right side of the last inequality.
In either case, we can deduce:
\begin{equation*}
a (a^{\alpha-1} - 1) \geq b(b^{\alpha-1} - 1).
\end{equation*}
The above is equivalent to
\begin{equation*}
a^\alpha - b^\alpha \geq a - b.
\end{equation*}
\end{thmproof}


\begin{llem} \label{llem:nlogx:leq}
Let \(n\in\N\)
and let \(x,\alpha\in\R\) such that \(x\geq 0\) and \(\alpha\geq 1\)
then
\begin{equation}
n\log\left(1 + \left(\frac{x}{n}\right)^\alpha\right) \leq \alpha x.
\end{equation}
\end{llem}

\begin{thmproof}
We assume \(n\in\N\) and the requirements of the lemma hold for
\(x,\alpha\in\R\).
Let
\begin{displaymath}
g(t) = e^t - t.
\end{displaymath}
Clearly, \(g(0)=1\) and for \(t\geq 0\)
we have \(g'(t) = e^t - 1 \geq 0\) and thus
\begin{equation} \label{eq:etmtg1}
e^t - t \geq 1 \qquad \textrm{for}\, t\geq 0.
\end{equation}
\begin{displaymath}
f(x) = e^{\alpha x/n} - x/n.
\end{displaymath}
If \(\alpha=1\) then from (\ref{eq:etmtg1}, we have \(f(x)\geq 1\)
for \(x\geq 0\).

Using Lemma~\ref{llem:apbp:geq}
with \(a=e^{x/n}\) and \(b=x/n\),
we see that
\begin{equation*}
e^{\alpha x/n} - (x/n)^\alpha \geq e^{x/n} - x/n \geq 1.
\end{equation*}

Hence
\begin{equation*}
1 + (x/n)^\alpha \leq e^{\alpha x/n}.
\end{equation*}
Equivalently,
\begin{equation*}
\log\left(1 + (x/n)^\alpha\right) \leq \alpha x/n,
\end{equation*}
that is,
\begin{equation*}
n\log\left(1 + (x/n)^\alpha\right) \leq \alpha x.
\end{equation*}
\end{thmproof}

% Hey, I found a hand-written (Hebrew) workout I made about 20 years ago!
% Here is a sort of edited copy.

Put
\begin{equation*}
f_n = n\log\left[ 1 + (f/n)^\alpha\right]
    = n^{1-\alpha}
      \log\left[\left(1 + f^\alpha/n^\alpha\right)^{n^\alpha}\right]
\end{equation*}
Note that if \(f(x)=0\) then \(f_n(x)=0\) and thus
\begin{equation*}
X' = f^{-1}\{(0,\infty]\}\subset X
\end{equation*}
and for every $n$ we have \(\int_X f_n = \int_{X'} f_n\)
and we may assume that \(f,f_n > 0\).

Now
\begin{eqnarray*}
\lim_{n\to\infty} f_n
 &=& \left(\lim_{n\to\infty} n^{1-\alpha}\right) \cdot
     \log\left(\lim_{n\to\infty}
         \left(1 + f^\alpha/n^\alpha\right)^{n^\alpha}\right) \\
 &=& f^\alpha  \lim_{n\to\infty} n^{1-\alpha} \\
 &=&  \left\{\begin{array}{l@{\qquad}l}
         \infty & \textrm{if }\; 0<\alpha<1,\\
         f      & \textrm{if }\; \alpha = 1,\\
         0      & \textrm{if }\; 1 < \alpha < \infty.
         \end{array}
      \right.
\end{eqnarray*}

Assume \(\alpha<1\).
\index{Fatou's lemma}
From Fatou's lemma,
\begin{equation*}
\lim_n \int f_n \geq \int \lim_n f_n = \infty.
\end{equation*}

Assume \(\alpha\geq 1\).
From local lemma~\ref{llem:nlogx:leq} we have \(f_n\leq \alpha f\).
\index{Lebesgue's!Dominated Convergence Theorem}
\index{Dominated Convergence Theorem}
Using Lebesgue's Dominated Convergence Theorem,
\begin{equation*}
\lim_n\int_X f_n\, d\mu = \int \lim_n f_n\,d\mu
 = \left\{\begin{array}{l@{\qquad}l}
         \int_X f = c     & \textrm{if }\; \alpha = 1,\\
         \int_X 0 = 0     & \textrm{if }\; 1 < \alpha
         \end{array}
  \right.
\end{equation*}




%% This is a newer trial, not remembering  the good old
\iffalse
First we need to show some inequalities that
will establish the hint.
\begin{llem} \label{llem:apbp:geq}
If \(a \geq b \geq 0\) and \(a,\alpha\geq 1\), then
\(a^\alpha - b^\alpha \geq a - b\).
\end{llem}

\begin{thmproof}
Since \(a\geq b\) and \(\alpha-1\geq 0\) we have
\(a^{\alpha-1} \geq b^{\alpha-1}\) and so
\begin{equation*}
a^{\alpha-1} - 1 \geq b^{\alpha-1} - 1
\end{equation*}
The left hand side must be non negative.
We look at the sign of the right side of the last inequality.
In either case, we can deduce:
\begin{equation*}
a (a^{\alpha-1} - 1) \geq b(b^{\alpha-1} - 1).
\end{equation*}
The above is equivalent to
\begin{equation*}
a^\alpha - b^\alpha \geq a - b.
\end{equation*}
\end{thmproof}


\begin{llem} \label{llem:nlogx:leq}
Let \(n\in\N\)
and let \(x,\alpha\in\R\) such that \(x\geq 0\) and \(\alpha\geq 1\)
then
\begin{equation}
n\log\left(1 + \left(\frac{x}{n}\right)^\alpha\right) \leq \alpha x.
\end{equation}
\end{llem}

\begin{thmproof}
We assume \(n\in\N\) and the requirements of the lemma hold for
\(x,\alpha\in\R\).
Let
\begin{displaymath}
g(t) = e^t - t.
\end{displaymath}
Clearly, \(g(0)=1\) and for \(t\geq 0\)
we have \(g'(t) = e^t - 1 \geq 0\) and thus
\begin{equation} \label{eq:etmtg1}
e^t - t \geq 1 \qquad \textrm{for}\, t\geq 0.
\end{equation}
\begin{displaymath}
f(x) = e^{\alpha x/n} - x/n.
\end{displaymath}
If \(\alpha=1\) then from (\ref{eq:etmtg1}, we have \(f(x)\geq 1\)
for \(x\geq 0\).

Using Lemma~\ref{llem:apbp:geq}
with \(a=e^{x/n}\) and \(b=x/n\),
we see that
\begin{equation*}
e^{\alpha x/n} - (x/n)^\alpha \geq e^{x/n} - x/n \geq 1.
\end{equation*}

Hence
\begin{equation*}
1 + (x/n)^\alpha \leq e^{\alpha x/n}.
\end{equation*}
Equivalently,
\begin{equation*}
\log\left(1 + (x/n)^\alpha\right) \leq \alpha x/n,
\end{equation*}
that is,
\begin{equation*}
n\log\left(1 + (x/n)^\alpha\right) \leq \alpha x.
\end{equation*}
\end{thmproof}


Back to the exercise.
We have three cases:
\begin{itemize}

 \item \(\alpha=1\)

 \begin{eqnarray*}
  \lim_{n\to\infty} n\log (1 + f/n)
   &=& \lim_{n\to\infty} \log \left((1 + f/n)^n\right) \\
   &=& \log \left(\lim_{n\to\infty} (1 + f/n)^n\right) \\
   &=& \log e^f = f
 \end{eqnarray*}
 Thus, using Lebesgue's dominated convergence theorem
 we have
 \begin{equation*}
 \lim_{n\to\infty} \int_X n\log \left(1 + f/n\right)d\mu
  = \int_X f\,d\mu = c.
 \end{equation*}

 \item \(\alpha<1\)

 \begin{eqnarray*}
      \left(1 + (x/n)^\alpha\right)^n
 &=& (1 + n^{1-\alpha}x^\alpha/n)^n \\
 &=& \sum_{k=0}^n \binom{n}{k}\left(n^{1-\alpha}x^\alpha/n\right)^k \\
 &\geq& \sum_{k=1} \cdots \\
 &\geq& n(n^{1-\alpha}x^\alpha/n) \\
 &=& n^{1-\alpha}x^\alpha.
 \end{eqnarray*}

 Thus
 \begin{equation*}
 \lim_{n\to\infty} n\log(1 + (x/n)^\alpha)
 = \lim_{n\to\infty} \log\left((1 + (x/n)^\alpha)^n\right) = \infty.
 \end{equation*}

 Again, using Lebesgue's dominated convergence theorem
 we have:
 \begin{equation*}
 \lim_{n\to\infty} \int_X n\log \left(1 + (f/n)^\alpha\right)d\mu \\
 = \int_X
      \left(\lim_{n\to\infty} n\log \left(1+(f/n)^\alpha\right)\right)d\mu \\
 = \infty.
 \end{equation*}

 \item \(\alpha>1\)

\end{itemize}
\fi % and of clumsy new trial


%%%%%%%%%%%%%%
\begin{excopy} % 10
Suppose \(\mu(X)<\infty\), \(\{f_n\}\) is a sequence of bounded complex
measurable functions on $X$, and \(f_n\to f\) uniformly on $X$. Prove that
\begin{equation*}
\lim_{n\to\infty} \int_X f_n d\mu = \int_X fd\mu,
\end{equation*}
and show that the hypothesis ``\(\mu(X)<\infty\)'' cannot be omitted.
\end{excopy}

Taking \(\epsilon = 1\), there is \(M_1>0\) such that
for any \(n\geq M_1\), we have \(|f_n(x) - f(x)| < M_1\) for all \(x\in X\).
Thus \(|f| < |f_n| + M_1\). In particular, $f$ is bounded.
Also the constant function \(g(x) = \|f\| + M_1\) dominates \(\{f_n\}\).
Applying Lebesgue's dominated convergence theorem gives the desired result

%%%%%%%%%%%%%%
\begin{excopy} % 11
Show that
\begin{equation*}
 A = \bigcap_{n=1}^\infty \bigcup_{k=n}^\infty E_k
\end{equation*}
in Theorem~1.41, and hence provided th theorem without any reference
to integration.
\end{excopy}

The set $A$ is defined as the set of all \(x\in X\)
that belong to infinitely many \(E_k\).
If \(x\in X\) then \(x \in U_k = \cup_{k=n}^\infty E_k\) for any \(n>0\).
Conversely, if \(x\notin A\) then there must be some \(N>0\)
such that \(x\notin E_k\) for all \(k\geq N\).
Clearly, \(x\notin U_N\) and the set equality is shown.

Now, let's quote the Theorem:
\begin{quotation}
\setcounter{quotethm}{40} % to get 41
  \begin{quotethm}
   Let \(\{E_k\}\) be a sequence of measurable sets in $X$, such that
   \begin{equation} \label{eq:thm41}
        \sum_{k=1}^\infty \mu(E_k) < \infty.
   \end{equation}
   Then almost all \(x\in X\) lie in at most finitely many of the sets \(E_k\).
  \end{quotethm}
  In view of this exercise, we have to show that \(\mu(A) = 0\).
  But the fact that the series in (\ref{eq:thm41}) conversges
  means that for any \(\epsilon>0\) there is an $N$ such that
  \(\sum_{k\geq N} \mu(E_k) < \epsilon\) and so
  \begin{equation*}
    \mu(A)
    = \mu \left(\bigcap_{n=1}^\infty \bigcup_{k=n}^\infty E_k\right)
    \leq \mu \left(\bigcup_{k=N}^\infty E_k\right) < \epsilon.
  \end{equation*}
  Thus \(\mu(A) = 0\).
\end{quotation}



%%%%%%%%%%%%%%
\begin{excopy}
Suppose \(f\in L^1(\mu)\). Prove that to each \(\epsilon\)
there exists a \(\delta > 0\) such that
\(\int_E |f|d\mu < \epsilon\) whenever \(\mu(E) < \delta\).
\end{excopy}

Let $X$ be the space on which \(\mu\) is defined.
\begin{equation*}
E_n = \{x\in X: n - 1 \leq |f(x)| < n
 \qquad \textrm{for } n\geq 1.
\end{equation*}
Clearly
\begin{equation*}
X = \Disjunion_{n=1}^\infty E_n\,.
\end{equation*}
Now,
\begin{equation} \label{eq:intf:sum:Ek}
\int_X |f|d\mu = \sum_{n=1}^\infty \int_{E_n} |f|d\mu < \infty
\end{equation}

Let \(\epsilon > 0\). By (\ref{eq:intf:sum:Ek}), there is
\(N>0\) such that
\begin{equation}
\sum_{n=N}^\infty \int_{E_n} |f|d\mu < \epsilon/2.
\end{equation}
Put \(H = \cup_{i=1}^{N-1} E_i\)
and \(T = \cup_{i=N}^\infty E_i\).
Note that \(X = H \disjunion T\).

Take \(\delta = \epsilon/2N\).
Now if $E$ is \(\mu\) measurable, and \(\mu(E)<\delta\) then
\begin{eqnarray*}
\int_E |f|d\mu
&=& \int_{E\cap H} |f|d\mu + \int_{E\cap T} |f|d\mu \\
&\leq& N\mu(E\cap H) + \int_{T} |f|d\mu \\
&\leq& N\mu(E) + \int_T |f|d\mu \\
&=& \epsilon/2  + \epsilon/2 = \epsilon.
\end{eqnarray*}


%%%%%%%%%%%%%%%
\end{enumerate}
%%%%%%%%%%%%%%%

 % \setcounter{chapter}{1}  % -*- latex -*-

%%%%%%%%%%%%%%%%%%%%%%%%%%%%%%%%%%%%%%%%%%%%%%%%%%%%%%%%%%%%%%%%%%%%%%%%
%%%%%%%%%%%%%%%%%%%%%%%%%%%%%%%%%%%%%%%%%%%%%%%%%%%%%%%%%%%%%%%%%%%%%%%%
%%%%%%%%%%%%%%%%%%%%%%%%%%%%%%%%%%%%%%%%%%%%%%%%%%%%%%%%%%%%%%%%%%%%%%%%
\chapterTypeout{Positive Borel Measures}

%%%%%%%%%%%%%%%%%%%%%%%%%%%%%%%%%%%%%%%%%%%%%%%%%%%%%%%%%%%%%%%%%%%%%%%%
%%%%%%%%%%%%%%%%%%%%%%%%%%%%%%%%%%%%%%%%%%%%%%%%%%%%%%%%%%%%%%%%%%%%%%%%
\section{Notes}

While working on Exercise~\ref{ex:2:10}, I worked out the following
lemma, that eventually were not used. Here they are, so they
will not be lost.

\begin{llem} \label{llem:interval:subsum}
Assume the disjoint union
\begin{equation*}
 \Disjunion_{i\in\N} I_i \subset [0,1]
\end{equation*}
where \(I_i\) are intervals of the form
\((a_i,b_i)\),
\((a_i,b_i]\),
\([a_i,b_i)\) or
\([a_i,b_i]\). If
\begin{equation*}
 \sum_{i\in\N} m(I_i) =  \sum_{i\in\N} (b_i - a_i) < 1
\end{equation*}
then there exists \(u\in[0,1]\) such that for any open interval $J$,
where \(u\in J\subset [0,1]\), the following inequality
\begin{equation*}
 \sum_{i\in\N} m(I_i \cap J) < m(J)
\end{equation*}
holds.
\end{llem}

\begin{thmproof}
Let \(U = \disjunion_{i\in\N}\).
We will construct a decreasing sequence of clsoed intervals \(\{K_i\}_{i\in\N}\)
such that for all \(i\in\N\)
\begin{itemize}
 \item \(K_{i+i} \subset K_i\)
 \item \(m(K_i) = 2^{-i}\)
 \item \(\sum_{i\in\N} m(I_i\cap K_i) < m(K_i) = 2^{-i}\).
\end{itemize}
Put \(K_0 = [0,1]\). Now by induction, assume \(K_i = [\alpha,\beta]\)
is defined and for which satisfies the above requirements.
Put \(\gamma = (\alpha+\beta)/2\), consider the two subintervals
\(L=[\alpha,\gamma]\) and
\(R=[\gamma,\beta]\). The first two requirements hold for \(K_{i+1}\)
and at least one of them satisfies the third, and we pick it as \(K_{i+1}\).
Now let \(u = \cap_{i\in\N} K_i\) (identifying a singleton with the element),
such $u$ exists as an intersection of non empty compact sets (and also is
unique).

Now assume \(u\in J = (a,b)\) an open interval.
Then there is some (sufficiently small) \(K_j \subset J\).
Using \(m(J) = m(K_j) + m(J\setminus K_j)\) we have
\begin{eqnarray*}
 \sum_{i\in\N} m(I_i \cap J)
 &=& \sum_{i\in\N} m\left(I_i \cap (K_j \disjunion (J\setminus K_j)\right) \\
 &=& \sum_{i\in\N} {   m(I_i \cap K_j)
                     + m\left(I_i \cap (J\setminus K_j)\right)} \\
 &=& \sum_{i\in\N} m(I_i \cap K_j) +
     \sum_{i\in\N} m\left(I_i \cap (J\setminus K_j)\right) \\
 &<&     m(K_j) + \sum_{i\in\N} m\left(I_i \cap (J\setminus K_j)\right) \\
 &\leq&  m(K_j) + m(J\setminus K_j) \\
 &=& m(J).
\end{eqnarray*}
\end{thmproof}


\begin{llem} \label{llem:sumintervals:}
Assume the unit interval is a disjoint union
\begin{equation*}
 [0,1] = \Disjunion_{i\in\N} I_i
\end{equation*}
where \(I_i\) are intervals of the form
\((a_i,b_i)\),
\((a_i,b_i]\),
\([a_i,b_i)\) or
\([a_i,b_i]\). Then
\begin{equation*}
 \sum_{i\in\N} m(I_i) =  \sum_{i\in\N} (b_i - a_i)= 1.
\end{equation*}
\end{llem}

\begin{thmproof}
For any finite sub-sum \(\sum_{i=1}^N m(I_i) \leq 1\), hence
\(\sum_{i\in\N} m(I_i)\leq 1\).
By negation, we assume \(\sum_{i\in\N} m(I_i)< 1\).
Now the assumptions of the previous lemma~\ref{llem:interval:subsum} hold
giving \(u\in[0,1]\) such that for any open interval \(J\subset[0,1]\) we have
\(\sum_{i\in\N} m(I_i \cap J) < m(J)\).
By assumptions there must be (a unique) $j$ such that \(u\in I_j\).
Now there are two cases.
\begin{itemize}
 \item[(\emph{i})]
   (Internal) $u$ is inetrnal point of \(I_j\), then
   we can pick an open interval $J$ such that \(u\in J \subset I_j\).
   Note, that here we use the induced topology, so $u$ may
   also be $0$ or $1$.
   For the sake of unifying with the proof continuation, we denote
   a dummy \(j' = j + 1\).
 \item[(\emph{ii})]
   (Boundary) \(u\in\partial I_j\) that is (\(I_j\) is not open and)
   \(u=a_j\in I_j\) or \(u=b_j\in I_j\).  In that case there must be
   \(j'\), the index of the neighbor interval, such that
   \(u\in\partial I_{j'}\).  \(L=I_j\disjunion I_{j'}\) is an
   interval.  Now we can pick an open interval $J$ such that
   \(u\in J\subset L=I_j\disjunion I_{j'}\) (where $L$ is an interval).
\end{itemize}
In both cases \(J \subset I_j \disjunion I_{j'}\).
Now by lemma~\ref{llem:interval:subsum} \(\sum_{i\in\N} m(I_i\cap J) < m(J)\),
but
\begin{equation*}
 \sum_{i\in\N} m(I_i\cap J)
 \geq   m(I_j \cap J) +  m(I_{j'} \cap J) \\
 =     m\left( (I_j \disjunion I_{j'}) \cap J\right)  = m(J)
\end{equation*}
which is a contradiction.
\end{thmproof}


Let us generalize 
\index{Lusin}
Lusin theorem~2.24.
\begin{llem}
Let \(f:\R\to\C\) be a measurable function.
For each \(\epsilon>0\) there exists a continuous function \(g:\R\to\C\)
such that 
\begin{align*}
m\bigl(\{x\in\R: f(x)\neq g(x)\}\bigr) &< \epsilon\\
\forall x\in\R,\quad |g(x)| &\leq |f(x)|\,.
\end{align*}
\end{llem}
Note that the lemma does \emph{not} imply \emph{uniform} continuity of $g$.
\\
\begin{thmproof}
Pick an \(\epsilon>0\), \wlogy\ \(\epsilon<1/2\).
For each \(n\in\Z\) consider the restriction
\(f_n:[n,n+2]\to\C\) of $f$ to \([n,n+2]\) 
(actually, \(f_n = f_{\restriction[n,n+2]}\)).
By Lusin theorem~2.24 we can find \(g_n:[n,n+2]\to\C\)
such that 
\begin{align*}
m\bigl(\{x\in[n,n+2]: f(x)\neq g_n(x)\}\bigr) &< \epsilon_n = 2^{|n|+2}\epsilon\\
\forall x\in[n.n+2],\quad |g_n(x)| &\leq |f(x)|\,.
\end{align*}
We will connect \(g_n\) to define $g$.
For every $n$ pick \(t_n\in(n, n+1)\) such that 
\begin{equation*}
g_{n-1}(t_n) = g_n(t_n) = f_n(t_n) = f(t_n).
\end{equation*}
The existence of \(t_n\) is ensured by \(\epsilon_n < 1/2\)
and thus \(g_n\) and \(g_{n+1}\) could differ from $f$ in \([n,n+1]\)
ona set of measure  at most less than $1$.
For each \(x\in\R\) there is a unique \(n\in\Z\) such that 
\(t_n \leq t_{n+1}\),  and we define \(g(x) = g_n(x)\).
Clearly $g$ is continuous and differs from $f$ on a set whose
measure is at most
\begin{equation*}
\sum_{n\in\Z}\epsilon_n = \epsilon \sum_{n\in\Z}  2^{|n|+2} < \epsilon\,.
\end{equation*}
\end{thmproof}

%%%%%%%%%%%%%%%%%%%%%%%%%%%%%%%%%%%%%%%%%%%%%%%%%%%%%%%%%%%%%%%%%%%%%%%%
%%%%%%%%%%%%%%%%%%%%%%%%%%%%%%%%%%%%%%%%%%%%%%%%%%%%%%%%%%%%%%%%%%%%%%%%
\section{Exercises Support}

%%%%%%%%%%%%%%%%%%%%%%%%%%%%%%%%%%%%%%%%%%%%%%%%%%%%%%%%%%%%%%%%%%%%%%%%
\subsection{Topology}

We need to establish some more set-theoretic topological results.
Using the
\index{Stone-Cech compactification@Stone-\Cech\ compactification}
Stone-\Cech\ compactification (\cite{Dug1966}, \textsf{XI 8.3}).

\paragraph{Definition} (\cite{Dug1966}, \textsf{VII 7.1}).
A Hausdorff space $X$ is
\index{completely regular}
\emph{completely regular}
\index{Tychonoff}
(or Tychonoff)
if for each point \(p\in X\) and a closed \(A\subset X\) such that \(p\notin A\)
there is a continuous \(f:X\rightarrow I=[0,1]\) such that \(f(p) = 1\)
and \(\forall x\in A, f(x)=0\).

\paragraph{Definition} (\cite{Dug1966}, \textsf{VII~7.1}).
Let $X$ be a completely regular topological space.
Let \(P=C(X,I)\) be the set of continuous functions \(f:X\rightarrow I\)
where \(I=[0,1]\) the unit interval. The product space \(I^P\)
is compact by the Tychonoff theorem  (\cite{Dug1966}, \textsf{XI~1.4}).
We define the map
\begin{eqnarray} \label{eq:stonecech:rho}
 \rho: X & \rightarrow & I^P \\
 \rho(x) &=& (\lambda(x))_{\lambda \in C(X,I)} \notag
\end{eqnarray}
The Stone-\Cech\ compactification is
\begin{equation*}
 \beta(X) = \overline{\rho(X)},
\end{equation*}
where the closure is on \(I^P\).

\begin{llem} \label{llem:stonecech:omega1}
The Stone-\Cech\ compactification of \([0,\omega_1)\) is homeomorphic
to \([0,\omega_1]\).
\end{llem}
\begin{thmproof}
For each \(f\in C([0,\omega_1],I\), its restriction
\begin{equation*}
 f_{|[0,\omega_1)} \in C([0,\omega_1),I).
\end{equation*}
From lemma~\ref{llem:Vickery} every \(g \in C([0,\omega_1),I)\),
can be extended to \(\overline{f}\in C([0,\omega_1],I)\)
by defining \(\overline{f}(\omega_1)\) as the tail value.
Thus,
\begin{equation*}
 C([0,\omega_1),I) \cong C([0,\omega_1],I)
\end{equation*}
By using the notations of (\ref{eq:stonecech:rho}),
\begin{equation*}
 \rho\left(C([0,\omega_1),I)\right) \subset
 \rho\left(C([0,\omega_1],I)\right)
\end{equation*}
Since \([0,\omega_1]\) is compact, it image is compact
and so equals to its closure. We now have:
\begin{equation*}
 \rho\left(C([0,\omega_1),I)\right) \subset
 \overline{\rho\left(C([0,\omega_1),I)\right)} \subset
 \overline{\rho\left(C([0,\omega_1],I)\right)} =
 \rho\left(C([0,\omega_1],I)\right)
\end{equation*}
\end{thmproof}


\begin{llem}
Let \(K_1,K_2\subset [0,\omega_1]\) be uncountable compact.
Then \(K_1\cap K_2\) is uncountable compact.
\end{llem}
\begin{thmproof}
Clearly \(K = K_1\cap K_2\) is compact.
By negation assume $K$ it is countable.
For \(i=1,2\), let \(H_i = K_i\cap [0,\omega_1)\)
(note that \(\omega_1\in K_i\)),
% both \(H_i\) are uncountable and compact in the inherited topology of
and \(H=K\setminus \{\omega_1\}\).
For $H$, there exists by lemma~\ref{llem:countable:ub}
an upper bound \(b\in[0,\omega_1)\).
Since \(H_i\cap[0,b)\) is countable, we define
in the space \([0,\omega_1)\)
new compact (in the inherited topology) subsets
\begin{equation*}
 L+i = H_i \cap [b,\omega_1) = K_i \cap [b,\omega_1)
 \qquad \textrm{for}\; i=1,2.
\end{equation*}

\end{thmproof}


\iffalse
%%% NOT TRUE!!  only if Y closed.
\begin{llem}
Let $X$ be a topological space, \(Y\subset X\) a subspace
with the topology inherited from $X$.
If \(K\subset X\) compact then \(K\cap Y\) is compact in $Y$
\end{llem}
\begin{thmproof}
Let \(\{V_i\}_{i\in I}\) be an open cover in $Y$ of \(K\cap Y\).
By the definition of the inherited topology, there are open sets
 \(\{V_i\}_{i\in I}\) in $X$ such that \(U_i = V_i \cap Y\) (for \(i\in I\).
For each \(i\in I\) we define \(W_i = V_i
\end{thmproof}
\fi


%%%%%%%%%%%%%%%%%%%%%%%%%%%%%%%%%%%%%%%%%%%%%%%%%%%%%%%%%%%%%%%%%%%%%%%%
%%%%%%%%%%%%%%%%%%%%%%%%%%%%%%%%%%%%%%%%%%%%%%%%%%%%%%%%%%%%%%%%%%%%%%%%
\section{Exercises} % pages 58-61

%%%%%%%%%%%%%%%%%
\begin{enumerate}
%%%%%%%%%%%%%%%%%

%%%%%%%%%%%%%%
\begin{excopy}
Let \(\{f_n\}\) be a sequence of real nonnegative functions on \(\R^1\),
and consider the following four statements:
\begin{itemize}
 \itemch{a}
   If \(f_1\) and \(f_2\) are upper semicontinuous,
   then \(f_1 + f_2\) is upper semicontinuous.
 \itemch{b}
   If \(f_1\) and \(f_2\) are lower semicontinuous,
   then \(f_1 + f_2\) is lower semicontinuous.
 \itemch{c}
   If each \(f_n\) is upper semicontinuous, then \(\sum_1^\infty f_n\)
   is upper semicontinuous.
 \itemch{d}
   If each \(f_n\) is lower semicontinuous, then \(\sum_1^\infty f_n\)
   is lower semicontinuous.
\end{itemize}
Show that three of these atr true and one is false.
What happens if the word ``nonnegative'' is omitted?
Is the truth of the statements affected if \(\R^1\) is replaced
by a general topological space?
\end{excopy}

We will show that only \ich{c} is false.

\begin{itemize}
 \itemch{a}
  True, for
  \begin{equation*}
  \{x\in\R: f_1(x)+f_2(x) < a\}
  = \bigcup_{\alpha\in\R} \left(\{x\in\R: f_1(x)< \alpha\} \cap
                             \{x\in\R: f_2(x) < a-\alpha\} \right).
  \end{equation*}
  and since the intersection of two open sets is open, the above
  set is open as an infinite union of open sets.

 \itemch{b}
  True, for
  \begin{equation*}
  \{x\in\R: f_1(x)+f_2(x) > a\}
  = \bigcup_{\alpha\in\R} \left(\{x\in\R: f_1(x)> \alpha\} \cap
                             \{x\in\R: f_2(x) > a-\alpha\} \right).
  \end{equation*}
  and since the intersection of two open sets is open, the above
  set is open as an infinite union of open sets.
 \itemch{c}
 False. Let's construct the following sequence of upper semicontinuous
 functions. For \(n\geq 1\), let
 \begin{equation*}
  f_n(x) = \left\{\begin{array}{lc}
        1 & \qquad\textrm{if }\  1/(n+1) \leq x \leq 1/n \\
        0 & \textrm{Otherwise}
                  \end{array}\right..
 \end{equation*}
 Putting \(F = \sum f_n\).
 Now clearly \(\{x\in R: x < 1/2\} = \{0\}\) a singleton which is clearly
 \emph{not} open.

 \itemch{d}
 True.
 Using \ich{b} and induction, we see that  \(F_n = \sum_{k=1}^n f_k\)
 is lower semicontinuous. Now for any \(x\in\R\) such that
 \(\sum_{k=1}^\infty f_k(x) > a\) there exists $n$ such that \(F_n(x) > a\),
 and so
 \begin{equation*}
  \{x\in\R: \sum_{k=1}^\infty f_k(x)\ > a\} =
  \bigcup_n \{x\in\R: F_n(x)\ > a\}.
 \end{equation*}
 is open.
\end{itemize}


%%%%%%%%%%%%%%
\begin{excopy}
Let $f$ be an arbitrary complex function on \(\R^1\), end define
\begin{eqnarray*}
 \varphi(x,\delta) & = & \sup\{|f(s) - f(t)|: s,t\in (x-\delta, x+\delta)\}\\
 \varphi(x)        & = & \inf\{\varphi(x,\delta): \delta > 0\}.
\end{eqnarray*}
Prove that \(\varphi\) is upper semicontinuous, that $f$ is continuous
at a point $x$ if and only if \(\varphi(x)=0\), and hence that the set of
points of continuity of an arbitrary complex function is a \(G_\delta\).

Formulate and prove an analogous statement for general topological
space in place of \(\R^1\).
\end{excopy}

Assume \(\alpha \in \R\) and \(G=\varphi^{-1}(-\infty,\alpha)\).
Let \(b\in G\setminus\inter{G} = \) a boundary point.
Now for any \(\delta>0\), there exists
\(w\in (b-\delta/2,b+\delta/2)\) such that \(w\notin G\).
Hence there are \(x,y\in(w-\delta/2,w+\delta/2)\) such that
\(|f(x)-f(y)| \geq \alpha\). But \(x,y\in (b-\delta,b+\delta)\) as well,
and since \(\delta\) is arbitrary, \(\varphi(b)\geq \alpha\) and
so \(b\notin G\) hence $G$ is open showing that \(\varphi\) is
upper semicontinuous.

If $f$ is continuous at $w$, then for any
% (replacing classic roles of \(\epsilon\) and \(\delta\))
\(\epsilon>0\) there exists \(\delta>0\) such that
\(|f(w+h)-f(w)| < \epsilon/2\) whenever \(|h|<\delta\).
In such case, clearly for any \(x,y\in (w-\delta,w+\delta)\)
we have
\begin{equation*}
|f(x) - f(y)| \leq
|f(x) - f(w)| + |f(w) - f(y)| < \epsilon/2 + \epsilon/2 = \epsilon.
\end{equation*}
That \(\varphi(w) < \epsilon\) and since \(epsilon>0\) was arbitrary,
\(varphi(w)=0\).
Conversely, if \(varphi(w)=0\) then for any \(\epsilon>0\)
there exists \(\delta>0\) such that for any \(x,y\in (w-\delta,w+\delta)\)
we have \(|f(x)-f(y)| < \epsilon\). In particular,
 \(|f(x)-f(w)| < \epsilon\) for any \(x \in (w-\delta,w+\delta)\),
hence $f$ is continuous at $w$.

The set of points where an arbitrary function $f$ is continuous, is
 \(\cap_n \varphi^{-1}(-\infty, 1/n)\) and by what was just shown
this set is an intersection of countably many open sets, that is
a \(G_\delta\) type of set.

\paragraph{Generalization} Let \(X,T\) be a topological space
(points are in $X$ and $T$ is the family of opens sets in $X$).
Define
\begin{eqnarray*}
 \varphi(x,V) & = & \sup\{|f(s) - f(t)|: s,t\in V\}
                             \qquad \textrm{where}\; x\in V\in T\\
 \varphi(x)        & = & \inf\{\varphi(x,V): V\in T\}.
\end{eqnarray*}
The proof goes similarly, just replacing \(\delta\) and \(delta/2\)
with sub-neighborhoods.


%%%%%%%%%%%%%%
\begin{excopy}
Let $X$ be a metric space, with metric \(\rho\).
For any non empty \(E\subset X\), define
\begin{equation*}
 \rho_E(x) = \inf\{\rho(x,y): y\in E\}.
\end{equation*}
Show that \(\rho_E\) is a uniformly continuous function on $X$.
If $A$ and $B$ are disjoint nonempty closed subsets of $X$, examine
the relevance of the function
\begin{equation*}
  f(x) = \frac{\rho_A(x)}{\rho_A(x) +  \rho_B(x)}
\end{equation*}
\index{Urysohn's lemma}
to Urysohn's lemma.
\end{excopy}

Let \(x\in X\) and \(\emptyset \neq E \subset X\).
For any \(\epsilon>0\), there exists \(u\in E\) such that
\(\rho(x,u) < \rho_E(x) + \epsilon\).
Now
\begin{equation*}
\rho_E(y) \leq \rho(y,u)
          \leq \rho(x,y) + \rho(x,u)
             < \rho(x,y) + \rho_E(x) + \epsilon.
\end{equation*}
Hence \(\rho_E(y) - \rho_E(x) \leq \rho(x,y)\) and by symmetry,
\begin{equation*}
|\rho_E(y) - \rho_E(x)| \leq \rho(x,y).
\end{equation*}
From which continuity of \(\rho_E\) follows.

If a non empty \(A\subset X\) is closed and \(x\in X\setminus A\)
then \(\rho_A(x) > 0\), since otherwise \(x\in \overline{A}\).
Hence by \(A\cap B = \emptyset\), the denominator of $f$ satisfies
\(\rho_A(x)+\rho_B(x) > 0\) for any \(x\in X\) and so $f$ is
well defined and continuous.

Now this gives a simple algabraic construction of a function
as desired in Urysohn's Lemma. We can easily see
that \(f(x)=0\) for \(x\in A\) and
that \(f(x)=1\) for \(x\in B\).

%%%%%%%%%%%%%% 4
\begin{excopy}
Examine the proof of the
\index{Riesz theorem}
Riesz theorem and prove the following two statements:
\begin{itemize}
 \itemch{a}
   If \(E_1 \subset V_1\) and \(E_2 \subset V_2\), where \(V_1\) and \(V_2\)
   are disjoint open sets, then \(\mu(E_1\cup E_2) = \mu(E_1) + \mu(E_2)\),
   even if \(E_1\) and \(E_2\) are not in \frakM.
 \itemch{b}
   If \(E\in \frakM_F\) then
   \(E = N\cup K_1\cup K_2 \cup \cdots\), where \(\{K_i\}\)
   is a disjoint countable collection of compact sets and \(\mu(N) = 0\).
\end{itemize}
\end{excopy}


\begin{itemize}
 \itemch{a}
   By \textsc{step~i} of the proof of Riesz Theorem (\cite{RudinRCA80} page~44),
   we know that
   \begin{equation*}
   \mu(E_1\cup E_2) \leq \mu(E_1) + \mu(E_2).
   \end{equation*}
   For the opposite inequality, let \(\epsilon>0\) and pick
   some open set $V$ such that \(E_1\cup E_2\subset V\) and
   \(\mu(V) \leq \mu(E_1\cup E_2) + \epsilon\).
   We now have:
   \begin{eqnarray*}
   \mu(E_1) + \mu(E_2)
    & \leq & \mu(V\cap V_1) + \mu(V\cap V_2) \\
    & = & \mu\left(V\cap (V_1 \cup V_2)\right) \\
    & \leq & \mu(V) \leq \mu(E_1\cup E_2) + \epsilon.
   \end{eqnarray*}
   Since \(\epsilon\) was arbitrary, we have
   \begin{equation*}
   \mu(E_1) + \mu(E_2) \leq \mu(E_).
   \end{equation*}

 \itemch{b}
   By \textsc{step~viii} of the proof of Riesz Theorem
   (\cite{RudinRCA80} page~47), we know that \frakM\ contains all
   Borel sets with finite \(\mu\) measure.
   Now define by induction, \(K_1\) a compact set such that
   \(K_1\subset E\) and \(\mu(K_1) > \mu(E)/2\).

   If \(\{K_i\}_{1\leq i<n}\) are defined, we know that
   \(D_{n-1} = E\setminus \cup_{i<n} K_i \in \frakM_F\) as such
   we can define \(K_n\)
   as a compact set, such that \(K_n \subset D_{n-1}\)
   and \(\mu(K_n) > \mu(D_{n-1}E)/2\).
   We can easily see that
   \begin{equation*}
    \sum_{i=1}^n \mu(K_i) \geq \sum_{i=1}^n 1/2^n = 1 - 1/2^{n+1}.
   \end{equation*}
   Put \(N = E \setminus \cup_{i=1}^\infty K_i\)
   and since \(\mu(E) = \sum_{i=1}^\infty \mu(K_i)\),
   we conclude that \(\mu(N) = 0\).
\end{itemize}

\end{enumerate}

\index{Lebesgue}
In Exercises 5 to 8, $m$ stands for Lebesgue's measure on \(R^1\).
\nobreak
\begin{enumerate}

\setcounter{enumi}{4}

%%%%%%%%%%%%%% 5
\begin{excopy}
Let $E$ be
\index{Cantor}
Cantor's familiar ``middle thirds'' set.
Show that \(m(E) = 0\), even though $E$ and \(\R^1\) have the same cardinality.
\end{excopy}

After removing the thirds on the $n$ step, the set \(C_n\) ``remain'' with
\(m(C_N) = (2/3)^n\). By being in the \(\sigma\)-algebra,
the Cantor set $C$ has a measure
\begin{equation*}
m(C) =
\lim_{n\rightarrow\infty}m(C_n) = \lim_{n\rightarrow\infty}(2/3)^n = 0.
\end{equation*}

For compuing the cardinality, let's consider the ternary representation
of each \(\alpha\in [0,1]\),
\begin{equation*}
 \alpha = \sum_{i=1}^\infty t_i(\alpha) 3^{-i}
    \qquad \textrm{where}\; t_i(\alpha) \in \{0,1,2\}
\end{equation*}
Numbers, except for $0$,  with finite representation
(with \(t_i(\alpha)=0\) for all \(i > N\) for some $N$)
such as
\begin{equation*}
\alpha = \sum_{i=1}^N t_i(\alpha) 3^{-i}
\end{equation*}
where \(0\neq t_N(\alpha) \in \{1,2\}\) ---
also have an infinite representation as in:
\begin{equation*}
\sum_{i=1}^N t_i(\alpha) 3^{-i} =
\sum_{i=1}^{N-1} t_i(\alpha) 3^{-i}
+ (t_N(\alpha)-1) 3^{-N}
+ \sum_{i=N+1}^\infty 2\cdot 3^{-i}
\end{equation*}

In such cases,
we resolve the ambiguity, by choosing:
\begin{itemize}
 \item the infinite representation if \(t_N = 1\).
 \item the finite representation if \(t_N = 2\).
\end{itemize}

Now the Cantor set $C$, is exactly the numbers in \([0,1]\)
whose ternary representation (with the above choice made)
does \emph{not} contain any factor (digit) $1$, that is
\(1 \neq t_i\in \{0,2\}\) for all $i$.

Using \(b_i=t_i/2\in \{0,1\}\), we build
 the following map \(f:C\rightarrow [0,1]\)
\begin{equation*}
f(\alpha)
  = f\left(\sum_{i=1}^\infty t_i(\alpha) 3^{-i} \right)
  = \sum_{i=1}^\infty (t_i(\alpha)/2) 2^{-i}.
\end{equation*}
Maps the Cantor set \emph{onto} the unit interval, using \emph{binary}
representation. This shows directly that the cardinality of $C$ is
the same as that of \([0,1]\) which is known to be \(|\R^1|\).

%%%%%%%%%%%%%% 6
\begin{excopy}
Construct \label{ex:disc:K}
a totally disconnected compact set \(K\subset \R^1\) such that
\(m(K) > 0\).
($K$ is to have no connected subset consisting of more than one point.)

If $v$ is lower semicontinuous  and \(v\leq \chhi_K\), show that actually
\(v \leq 0\). Hence \(\chhi_K\) cannot be approximated by lower semicontinuous
function, in the sense of
\index{Vitaly}
\index{Carath\'eodory}
Vitaly-Carath\'eodory Theorem.
\end{excopy}

Construct the following similar to Cantor set.
Start from the unit interval,
in each step we ``break'' any segment of the previous step,
but (contrary to Cantor's) ensuring that the sum of open segment
taken out is ``much'' less than $1$.

More formally, we start (after step-0), with \(D_0 = [0,1]\).
After step $n$, \(D_n\) is a union of \(2^n\) closed intervals.
In step $n$, from each interval \([a,b]\) of \(D_{n-1}\)
we substract from its center an open sub-interval with length of \(2^{-(2n+1)}\)
Hence in this step we substract a total of
\begin{equation*}
 2^n \cdot 2^{-(2n+1)} = 2^{-n-1}.
\end{equation*}
The total length of open intervals removed by all steps upto step $n$ is
\begin{equation*}
 \sum_{k=1}^n 2^{-k-1} = (1/2) \sum_{k=1}^n 2^{-k} < 1/2
\end{equation*}
Thus \(K = \cap_n D_n\) is compact, \(m(D) \geq 1/2\)
and it is totally disconnected, since any interval eventually gets cut.

Now if $v$ satisfies the assumptions of the exercise, and by negation
\(v(w) > 0\) for some \(w\in \R\) then for \(a = v(w)/2\) then
\(V = v^{-1}(a,\infty)\) is open and \(w\in V\) thus there is an open
interval $I$ for which \(f(x)>a\) for all \(x\in I\),
but since $K$ is totally disconnected, there must be some \(r\in I\setminus K\)
giving a the contradiction \(\chhi(r) = 0\).

%%%%%%%%%%%%%% 7
\begin{excopy}
Given \(\epsilon > 0\), construct an open set \(E\subset [0,1]\) which is dense
in \([0,1]\), such that \(m(E)=\epsilon\).
(To say that $A$ is dense in $B$ means that the closure of $A$ contains $B$.)
\end{excopy}

Of course we should assume \(\epsilon \leq 1\).
Put \(\Q\cap[0,1]\) into q sequence \((q_i)_{i=1}^\infty\).
Now define the following sequence open sets \((G_k)_{k=1}^\infty\).
Given \(k\leq 1\), let \(H_k =  \cup_{j<k} \overline{G_j}\)
(closed, where \(H_1 = \emptyset\)),
let $i$ be the minimal such that \(q_i\notin H_k\).
Pick an open interval \(I_k\) of \(q_i\) with length \(l_k\)
such that \(l_k \leq 2^{-k}\epsilon\)
(Note: by ``open'' we mean here --- open \emph{in} \([0,1]\).
Thus \([0,a)\) and \((b,1]\) are considered open).
If \(l_k < 2^{-k} \epsilon\), also pick some finite number
of open intervals \(\{J_{k,m}\}\) such that
\begin{equation*}
G_k = I_k \cup\,\bigcup_{m=1}^{N_k} J_{k,m}
\end{equation*}
satisfies
\begin{equation*}
m(G_k) = 2^{-k} \qquad \textrm{and} \qquad
  G_k \cap \left(\cup_{j<k} G_j\right) = \emptyset.
\end{equation*}
Let \(G = \cup G_k\), surely \(m(G) = \epsilon\) and \(\Q\cap [0,1] \subset G\),
and so $G$ is dense in \([0,1]\).


%%%%%%%%%%%%%% 8
\begin{excopy}
Construct a Borel set \(E\subset \R^1\) such that
\begin{equation*}
 0 < m(E\cap I) < m(I)
\end{equation*}
for every nonempty segment $I$. Is it possible to have \(m(E) < \infty\)
for such a set?
\end{excopy}


Given \(0<a<1\) we will first build a set \(F\subset [0,1]\) such that
\(m(F) = a\) and
\begin{equation*}
0 < m(F\cap I) < m(I)
\end{equation*}
We will build $F$, and its complement \(G = [0,1] \setminus F\)
as an countable infinite union of Borel sets,
\(F = \cup_n F_n\)
and
\(G = \cup_n G_n\)
with measures \(m(F_i) = 2^{-i}a\) and \(m(G_i) = 2^{-i}(1-a)\).
At each stage,
\begin{equation*}
 R_n = [0,1] \setminus
       \left( \bigcup_{i=1}^n F_i \cup \bigcup_{i=1}^n G_i\right)
\end{equation*}
will be a countable family of intervals ---
close, open are half closed and open. That is, of
any of the forms:
\([a,b]\),
\((a,b)\),
\([a,b)\) or
\((a,b]\).

\paragraph{Building \(F_1\).}
For any interval \(I\subset[0,1]\).
Similar to Exercise~\ref{ex:disc:K} above, we construct a totally disconnected
subset \(F_1\) of [0,1]
whose complement is a countable family of open intervals.
We carefully choose the sizes of the open intervals, so
that \(m(F_1) = a/2\).

\paragraph{Building \(G_1\).}
The set \(F_1^c = [0,1]\setminus G_1\) consists of countably many \(\cal{N}\)
intervals (where \(\cal{N}<\infty\) or \({\cal{N}} = \aleph_0\)).
Within each interval \(I_i\) of \(F_1^c\) we build a totally disconnected
subset \(T_i\), such that the sum of measures of these sets equals \((1-a)/2\).
This is done either by choosing a measure of
\((1-a)/\cal{N}\) if  \(\cal{N}<\infty\),
or a measure of \(2^{i+1}(1-a)\) otherwise.
We let \(G_1= \cup_i T_i\).


\paragraph{Building \(F_n\).}
Assume \(\{F_i\}_{i=1}^{n-1}\)
and \(\{G_i\}_{i=1}^{n-1}\) are built,
The set
\begin{equation*}
 R_{n-1} = [0,1] \setminus
  \left( \bigcup_{i=1}^{n-1} F_i \cup \bigcup_{i=1}^{n-1} G_i\right)
\end{equation*}
consists of countably many intervals.
Within each of these intervals we build a totally disconnected set \(U_{n,i}\)
such that when letting \(F_n = \cup_i U_{n,i}\) be a disjoint union, we have
\begin{equation*}
m(F_n) = m\left(\cup_i U_{n,i}\right) = \sum_i m(U_{n,i}) = 2^{-(n+1)} a
\end{equation*}

\paragraph{Building \(G_n\).}
Assume \(\{F_i\}_{i=1}^{n}\)
and \(\{G_i\}_{i=1}^{n-1}\) are built,
The set
\begin{equation*}
 S_{n-1} = [0,1] \setminus
  \left( \bigcup_{i=1}^n F_i \cup \bigcup_{i=1}^{n-1} G_i\right)
\end{equation*}
consists of countably many intervals.
Within each of these intervals we build a totally disconnected set \(V_{n,i}\)
such that when letting \(G_n = \cup_i V_{n,i}\) be a disjoint union, we have
\begin{equation*}
m(G_n) = m\left(\cup_i V_{n,i}\right) = \sum_i m(U_{n,i}) = 2^{-(n+1)} (1-a).
\end{equation*}

Now $F$ and $G$ are well defined, and \(F\disjunion G = [0,1]\).

Note that the way \(\{F_i\}\) and \(\{G_i\}\) were build,
for each \(x,y\in F\), where \(x<y\), there is some $n$, such that
\(x,y\in \cup_{i\leq n} F_i\) and \(m((x,y) \cap G_i) > 0\).

Similarly,
for each \(x,y\in G\), where \(x<y\), there is some $n$, such that
\(x,y\in \cup_{i\leq n} G_i\) and \(m((x,y) \cap  F_{i+1}) > 0\).

The last observations show
the inequalities \(0 < m(F\cap\, [x,y]\,) < m([x,y])\).
To generalize it, we repeat the process of building $F_i$ for
any interval \(\{[n,n+1]\}_{n\in\Z}\), with \(a = 2^{-|n|}\).
Put \(E = \cup F_i\). Now \(m(E) = \sum_{n\in\Z} = 3 < \infty\)
and the desired inequalities \(0 < m(E\cap\, I\,) < m(I)\)
for and interval $I$ in \R\ is established.




%%%%%%%%%%%%%%
\begin{excopy}
Construct a sequence of continuous function \(f_n\) on \([0,1]\) such that
\(0\leq f_n \leq 1\), such that
\begin{equation*}
 \lim_{n\rightarrow \infty} \int_0^1 f_n(x)dx = 0,
\end{equation*}
but such that the sequence \(\{f_n(x)\}\) converges for no \(x\in[0,1]\).
\end{excopy}

Put \(N_0 =0\) and \(N_k = \sum_{i=1}^{k+1} i = (k+1)(k+2)/2\).
We will define the \(f_n\) functions in $k$-batches.
For each \(k>0\), we  define \(\{f_n:  N_{k=1} < n \leq N_k\}\):
\begin{equation*}
f_n(x) = f_{N_{k-1}+i}(x) = \left\{
 \begin{array}{l@{\quad}c}
   1 - k|x - (i-1)/k| & \textrm{if}\; |x - (i-1)/k|<1/k \\
   0                  & \textrm{Otherwise}
 \end{array}\right.
\end{equation*}
for \(1\leq i \leq N_k\).
We see that \(\int_0^1 f(n) = 1/k\) and so \(\int_0^1 f(n)\to 0\)
but for every $k$-batch, for any \(x\in[0,1]\)
there is some $n$ such that \(N_{k-1}< n \leq N_k\)
and \(f_n(x) \geq 1/2\).


%%%%%%%%%%%%%%
\begin{excopy} % 10
If \label{ex:2:10}
\(\{f_n\}\) is a sequence of continuous functions on \([0,1]\) such that
\(0\leq f_n \leq 1\), and such that
\(f_n(x)\to 0\) as \(n \to \infty\),
for every \(x\in[0,1]\), then
\begin{equation*}
 \lim_{n\rightarrow \infty} \int_0^1 f_n(x)dx = 0,
\end{equation*}
Try to prove this without using any measure theory or any theorem
about Lebesgue integration. (This is to impress you with the power of
the Lebesgue integral. A nice proof was given
\index{Eberlein}
by W.F.~Eberlein in \emph{communications on Pure and Applied Mathematics},
vol.~X, pp.~357-360, 1957.)
\end{excopy}

We will allow ourselves to use some trivial measure concepts
such as the sum of lengths of finite union of intervals.
That is if \(\{I_i\}_{i=1}^n\),
where \(I_i\) are intervals of the form
\((a_i,b_i)\),
\((a_i,b_i]\),
\([a_i,b_i)\) or
\([a_i,b_i]\), and \(I_i \cap I_j = \emptyset\) whenever \(i\neq j\) then
we use
\begin{equation*}
m\left(\Disjunion_{i=1}^n I_i\right) =
\sum_{i=1}^n m(I_i) = \sum_{i=1}^n b_i - a_i.
\end{equation*}

\iffalse
Back to the exercise. Let us define some sequences of subsets of \([0,1]\)
(with nicknames) based on \(\{f_n\}_{i\in\N}\).
\begin{equation*}
\begin{array}{lcl@{\qquad}r}
 U_{n,k} &=&
    \{x\in [0,1]: |f_n(x)| \geq 1/k\}    & \textrm{(Upper)} \\ \\
 T_{n,k} &=&
   \bigcup\limits_{i=n}^\infty U_{i,k}   & \textrm{(Tail)} \\ \\
 R_k     &=&
    \bigcap\limits_{i=1}^\infty T_{i,k}  & \textrm{(Resistance)} \\
\end{array}
\end{equation*}

When looked closely, we see that \(R_k\) consists of all \(x\in[0,1]\)
such that \(|f_n(x)| \geq 1/k\)  for infinitely many $n$.
Thus the assumption that \(f_n(x)\to 0\) for all \(x\in[0,1]\)
means that
\begin{equation*}
\cap_{k\in\N} R_k = \emptyset.
\end{equation*}
\fi

Back to the exercise. Let
\begin{equation*}
 g_n(x) = \sup_{k\geq n} |f_k(x)|
\end{equation*}
Clearly \(\{g_n\}_{n\in\N}\) is a decreasing sequence,
converges pointwise \(g_n(x)\xrightarrow{n\to\infty} 0\) and
\begin{equation*}
 \left|\int_0^1 f_n(x)dx\right|
 \leq  \int_0^1 |f_n(x)|dx \leq  \int_0^1 g_n(x)dx.
\end{equation*}
Hence it is sufficient to show that
\begin{equation} \label{eq:integ:gto0}
\lim_{n\to\infty} \int_0^1 g_n(x)dx = 0.
\end{equation}

Let \(\epsilon>0\) be arbitrary, and define
\begin{equation*}
 L_n = L_{n,\epsilon} = \{x\in[0,1]: |f_n(x) < \epsilon\}.
\end{equation*}
By definition and assumptions, \(\{L_n\}_{n\in\N}\) is an increasing
sequence of open sets with union
\begin{equation} \label{eq:ULn}
\bigcup_{n\in\N} L_n = [0,1].
\end{equation}
Each \(L_n\) is a union of countably many open intervals,
\begin{equation*}
 L_n = \bigcup_{k\in \N} I_{n,k} = \bigcup_{k\in \N} (a_{n,k}, b_{n,k})
 \qquad \textrm{for each}\; n\in\N.
\end{equation*}

The (still limited to intervals) total measure
\begin{equation}
 M_n = M(L_n) =\sum_{k\in \N} m(I_{n,k})
\end{equation}
is clearly an increasing sequence (by simply looking at sub-finite sums).
We will now show that it
converges to the measure of the unit \(m([0,1])\), that is:
\begin{equation} \label{eq:Lsubint:to1}
 \lim_{n\to\infty} \sum_{k\in \N} m(I_{n,k}) = 1.
\end{equation}
By negation, assume
\begin{equation*}
 \lim_{n\to\infty} M_n = \beta < 1.
\end{equation*}

For any sub-intervals \(I,[a,b]\subset [0,1]\)
trivially
\begin{equation*}
m(I\cap[a,b]) = m(I\cap[a,(a+b)/2]) + m(I\cap[(a+b)/2, b]).
\end{equation*}
Similarly
\begin{equation} \label{eq:MLn}
 M(L_n \cap [a,b]) =  M(L_n \cap [a,(a+b)/2]) + M(L_n\cap[(a+b)/2, b])
\end{equation}
Starting with \(a_0=0\), \(b_0=1\), and recursively bisect \([a_i,b_i]\),
we get a decreasing sequence of closed sub-segments with sizes
\(m([a_i,b_i]) = 2^{-i}\),
such that
\begin{equation*}
 M(L_n \cap [a_i,b_i]) \leq 2^{-i}\beta < 2^{-i}
\end{equation*}
for all \(n\in\N\).
This can be done, by choosing the smaller of the two choices
in each bisection step, and using (\ref{eq:MLn}).
Let $b$ be the intersection point \(\{b\} = \cap_{i\in\N} [a_i,b_i]\).
But by (\ref{eq:ULn}), there exists $n$, such that \(b\in L_n\).
Since \(L_n\) is open, there is some $i$ such that \(b\in[a_i,b_i]\subset L_n\).
Now
\begin{equation*}
M(L_n \cap [a_i,b_i]) = M([a_i,b_i]) = 2^{-i}
\end{equation*}
gives a contradiction and thus (\ref{eq:Lsubint:to1}) is true.

Let \(N<\infty\) be such that
\begin{equation*}
M_N = \sum{k\in \N} m(I_{N,k}) > 1 - \epsilon
\end{equation*}
and \(K<\infty\) be such that
\begin{equation*}
M_N = \sum{k=1}^K m(I_{N,k}) > 1 - 2\epsilon.
\end{equation*}
Set
\begin{eqnarray*}
  D & = & \cup{k=1}^K I_{N,k} \\
  U & = & [0,1] \setminus D
\end{eqnarray*}
Both $D$ and $U$ are unions of finite number of intervals.
Since \(m(D) > 1 - 2\epsilon\), the complement has \(m(U) < 2\epsilon\).
Now we compute:
\begin{equation*}
\int_0^1 g_N(x)dx = \int_D g_N(x)dx + \int_U g_N(x)dx
 \leq (1-2\epsilon)\epsilon + 2\epsilon\cdot 1 < 3\epsilon.
\end{equation*}
Since \(\epsilon>0\) was arbitrary and
\(\{g_n\}_{n\in\N}\) is a decreasing sequence,
the convergence of (\ref{eq:integ:gto0}) is shown.


%%%%%%%%%%%%%% 11
\begin{excopy}
Let \(\mu\) be a regular Borel measure on a compact Hausdorff space $X$:
assume \(\mu(X) = 1\). Prove that there is a compact set \(K\subset X\)
\index{carrier}
\index{support!measure}
(the \emph{carrier} or \emph{support} of \(\mu\))
such that \(\mu(K) = 1\) but \(\mu(H)<1\)
for every proper compact subset $H$ of $K$.
\emph{Hint}: Let $K$ be the intersection of all compact \(K_\alpha\) with
\(\mu(K_\alpha)=1\);
show that every open set $V$ which contains $K$ also contains some \(K_\alpha\).
Regularity of \(\mu\) is needed; compare Exercise~18.
Show that \(K^c\) is the largset open set in $X$ whose measure is $0$.
\end{excopy}

If both \(K_1\) and \(K_2\) are compact then
\(K_1 \cap K_2\) is compact. If
\(\mu(K_1) = \mu(K_2) = 1\), then
\(mu(X\setminus K_1) = \mu(X\setminus K_2) = 0\) and so
\begin{equation*}
\mu(K_1 \cap K_2) \geq \mu(X) - (\mu(X\setminus K_1) + \mu(X\setminus K_2) =
        1 - 0 = 1.
\end{equation*}
By induction \(\mu(\cap_{i=1}^n K_i = 1\) for finite intersection.

Let $K$ be the intersection as in the hint.
Let $V$ be an open set such that \(\cap_\alpha K_\alpha \subset V \subset X\).
Now \(V^c \subset \cup_\alpha K_\alpha^c\), this is an open covering
of the compact \(V^c\). Hence there is a sub-finite covering,
that is we have \(\{\alpha_i\}_{i=1}^m\) such that
\(V^c \subset \cup_{i=1}^m K_{\alpha_i}^c\), equivalently
\(L = \cap_{i=1}^m K_{\alpha_i} \subset V\). By our initial remarks,
$L$ is compact and \(\mu(L)=1\) thus \(L=K_\alpha\) for some \(\alpha\).

By regulartiy,
\begin{equation*}
\mu(K) = \inf \{\mu(V): V\;\textrm{is open, and }\; K\subset V\}
       \geq \mu(K_\alpha) = 1.
\end{equation*}
Thus \(\mu(K) = 1\) and by construction, $K$ is minimal compact with
such measure.


%%%%%%%%%%%%%% 12
\begin{excopy}
Show \label{ex:2:12}
that every compact set of \(\R^1\) is the support of a Borel measure.
\end{excopy}

Let \(K\subset\R^1\) be a compact set, with the inherited topology.
For any function \(f\in C_c(K) = C(K)\) we will define an extension
\(\tilde{f} \in C_c(\R)\). By $K$ being bounded, we can pick
\(b_0 < \min(K)\) and \(b_1 > \max(K)\).
Denote the compact \(\tilde{K} = K \cup \{b_0,b_1\}\)
and the complement open set \(G=\R\setminus \tilde{K}\).
For any \(x\in G\) there is a maximal open interval \((a_x,b_x) \subset $G$\)
whose endpoints \(a_x,b_x\in \tilde{K}\). Define:
\begin{equation*}
 \tilde{f}(x) = \left\{\begin{array}{l@{\qquad}l}
                  f(x) & x \in K \\
                  0    & x \leq b_0 \quad \textrm{or} \quad b_1 \leq x \\
                  \left(\tilde{f}(b_x) - \tilde{f}(a_x)\right)
                  \frac{x - a_x}{b_x - a_x}
                  + \tilde{f}(a_x)
                    & x \in (a_x,b_x)\subset G \quad\textrm{and}\quad
                      a_x,b_x \in \tilde{K}
                  \end{array}\right.
\end{equation*}
Note the ``pseudo'' recursive definition is fine, since
only \(\tilde{f}(b_x)\) and  \(\tilde{f}(a_x)\) are used
and for them \(\tilde{f}\) is well defined on previous cases.
The mapping \(f\to\tilde{f}\) is clearly linear mapping of \(C(K) \to C_c(\R)\).
We now define a positive functional on \(C(K)\)
\begin{equation*}
 \Lambda(f) = \int_{\R} \tilde{f}\,dm =  \int_{b_0}^{b_1} \tilde{f}\,dm
\end{equation*}
where $m$ is the regular Lebesgue's measure on \(\R\).

\index{Riesz}
From Riesz Theorem~2.14, there is Borel measure \(\mu\) on $K$
such that for all \(f\in C(K)\)
\begin{equation*}
 \int_{\R^1} f\,d\mu = \Lambda f.
\end{equation*}

We will now show that $K$ is the support of \(\mu\).
By negation, let \(H\subsetneq K\) be a compact such that
 \(\mu(H) = \mu(K)=1\), hence \(\mu(K\setminus H) = 0\).
Pick \(x\in K\setminus H\). By being a compact Hausdorff space,
there exists an open set $V$ such that \(x\notin V\) and \(H\subset V\).
\index{Urysohn's lemma}
By Urysohn's Lemma~2.12, we can find a continuous function \(g\in C(K)\)
such that \(\{x\} \prec g \prec V\).
Now
\begin{equation} \label{eq:Ksupp:mu}
 \Lambda g = \int_K g\,d\mu = \int_H g\,d\mu + \int_{K\setminus H} g\,d\mu
           = \int_H 0\,d\mu + \int_{\emptyset} g\,d\mu = 0+0 = 0.
\end{equation}
But \(\tilde{g} \geq 0\) is continuous, with \(\tilde{g}(x) = g(x) > 0\),
hence
\(\Lambda g = \int_{\R} \tilde{g}dm > 0\) a contradiction
to (\ref{eq:Ksupp:mu}).


%%%%%%%%%%%%%% 13
\begin{excopy}
Is it true that every compact subset of \(\R^1\) is a support of a continuous
functions? If not, can you describe the class of all compact sets in
\(\R^1\) whoch are supports of continuous functions?
Is your description valid in other topological spaces?
\end{excopy}

The first claim is false. Take a any singleton, say \(\supp f = \{a\}\)
then \(f(a) \neq 0\), by continuity, using a positive \(\epsilon < |f(a)|\)
we can find a \(\delta>0\) such that
\begin{equation*}
 \supp f \supset (a-\delta, a + \delta)
         \subset \{x\in\R: |f(x)| > |f(a)| - \epsilon > 0\}.
\end{equation*}


Here is the classification:
\begin{llem}
A compact \(K\in\R\) is a support of some \(f\in C(\R)\)
iff \(K = \overline{\inter{K}}\) (equals the closure of its interior).
\end{llem}
\begin{thmproof}
Let $K$ be  a compact $K$ set.

Assume \(K = \supp f\), for some \(f\in C(\R)\).
For any closed set $K$, (in particular compact),
\(K \subset \overline{\inter{K}}\).
If by negation \(K \subsetneq \overline{\inter{K}}\), then we can find
\(x \in K \setminus \overline{\inter{K}}\).
If \(f(x) \neq 0\) then there is a neighborhood $V$ of $x$
such that \(x\in V \subset \supp f\) which gives the contradiction
\(x\in \inter{K}\). Thus \(f(x) = 0\) but then
\(K = \supp f \subset \overline{\inter{K}}\) gives a contradiction.

Conversely, assume \(K = \overline{\inter{K}}\).
Since \(G = \inter{K} = \cup_{i\in\N} I_i\), is a countable union
of open  intervals \(I_i = (a_i, b_i)\), we denote their
length  \(l_i = b_i - a_i\) and
center
\(c_i = (a_i + b_i)/2\) to define
\begin{equation*}
f(x) = \left\{\begin{array}{l@{\qquad}l}
               l_i/2 - |x - c_i| &  a_i \leq x \leq b_i \\
               0 & x \notin K
              \end{array}\right.
\end{equation*}
Now clearly $f$ is continuous and \(\supp f = K\).
\end{thmproof}

From the proof, we can see that the classification
can extend to any space $X$ where any open set $G$ is a union
of open sets \(\{V_\alpha\}_{\alpha\in I}\) with disjoint closures
such that for each \(V_\alpha\) there is a function \(f\in C(X)\)
such that \(K \prec f \prec V\) for any compact \(K\subset V\).
This is true for \(\R^n\).

%%%%%%%%%%%%%% 14
\begin{excopy}
If $f$ is a Lebesgue measurable complex function on \(\R^1\),
prove that there is a Borel function $g$ on \(\R^1\) such that
\(f=g\)~a.e.[$m$].
\end{excopy}

With the understanding (similar to the definition
\index{Borel}
of \emph{Borel function} --- see:
\cite{RudinRCA80} page~13)
that
\index{Lebesgue!measurable function}
\(f:X\to \C\) is a \emph{Lebesgue measurable function} iff
\(f^{-1}(V)\) is Lebesgue measurable for any open set \(V\subset X\).

We will build a subset \(M \subset \R\) with \(m(M) = 0\)
and define
\begin{equation} \label{eq:g:Borel}
g(x) = \left\{\begin{array}{l@{\qquad}l}
             f(x) & x\notin M \\
             0    & x\in M
             \end{array}\right..
\end{equation}

By Theorem~2.20 for any Lebesgue measurable set $L$, we can find
Borel (\(F_{\sigma}\) set \(B\subset L\) such that \(m(L\setminus B) = 0\).

Enumerate all open rectangles with rational boundaries
\({\cal{R}} = \{R_i\}_{i\in\N}\)
where
\begin{equation*}
 R_i = \{z\in\C : r_0\leq \Re(z) \leq r_1 \;\wedge\;
                  q_0\leq \Im(z) \leq q_1 \;\wedge\;r_0,r_1,q_0,q_1\in\Q\}.
\end{equation*}

Now \(L_i = f^{-1}(R_i)\) is a Lebesgue measurable function.
Let \(B_i\subset L_i\) be a Borel measurable set
(as on the preceding remark) such that and \(m(L_i\setminus B_i) = 0\).
We define \(M = \cup_{i\in\N} (L_i\setminus B_i)\).
By \(\sigma\)-additivity, \(m(M)=0\) as requried.
Now, it is left to show
that $g$ defined in (\ref{eq:g:Borel}) is a Borel measurable function.

Any (origin-less) open set \(G\subset \C\setminus\{0\}\)
can be represented as a countable union of intervals from \(\cal{R}\),
say
\begin{equation*}
G = \cup_{i\in\N} R_{k_i}
\end{equation*}
Now computing the inverse images
\begin{eqnarray*}
 g^{-1}(G)
 &=&  f^{-1}(G) \setminus M \\
 &=& f^{-1}\left(\cup_{i\in\N} R_{k_i}\right) \;\setminus M
     = \bigcup_{i\in\N} f^{-1}( R_{k_i}) \;\setminus M
     = \bigcup_{i\in\N} L_{k_i} \;\setminus M \\
 &=& \left(\bigcup_{i\in\N} B_{k_i}
     \disjunion \left(L_{k_i}\setminus B_{k_i}\right)\right) \;\setminus M \\
 &=& \left(\bigcup_{i\in\N} B_{k_i}
     \,\cup\, \bigcup_{i\in\N}  \left(L_{k_i}\setminus B_{k_i}\right)
     \right) \;\setminus M \\
 &\subset&  \left(\bigcup_{i\in\N} B_{k_i} \cup M'\right) \;\setminus M \\
 &=& \bigcup_{i\in\N} B_{k_i}.
\end{eqnarray*}
Where
\begin{equation*}
 M' = \bigcup_{i\in\N}  \left(L_{k_i}\setminus B_{k_i}\right) \subset M.
\end{equation*}
So \(g^{-1}(G)\) is a Borel set. In particular \(g^{-1}(\C\setminus\{0\})\)
is Borel set, and by \calB\ being \(\sigma\)-algebra,
the complement \(g^{-1}(\{0\})\) is a Borel set as well.


%%%%%%%%%%%%%% 15
\begin{excopy}
It is easy to gueass the limits of
\begin{equation*}
 \int_0^\pi \left(1 - \frac{x}{n}\right)^n e^{x/2}\, dx
 \quad\textrm{and}\quad
 \int_0^\pi \left(1 + \frac{x}{n}\right)^n e^{-2x}\, dx.
\end{equation*}
as \(n\to\infty\). Prove that your guesses are correct.
\end{excopy}

\begin{itemize}

\item
We compute the limit
\begin{equation*}
\lim_{n\to\infty} \left(1 - \frac{x}{n}\right)^n e^{x/2}
=  e^{x/2} \lim_{n\to\infty} \left(1 - \frac{x}{n}\right)^n \\
=  e^{x/2} e^{-x} = e^{-x/2}.
\end{equation*}

The integrand \(\left(1 - \frac{x}{n}\right)^n e^{x/2}\)
is bounded on \([0,\pi]\) by  \(1\cdot e^{\pi/2}\).
Using \(\int e^{-x/2}\,dx = -2e^{-x/2} + c\) and
Lebesgue Dominated Theorem
\begin{equation*}
 \lim_{n\to\infty} \int_0^\pi \left(1 - \frac{x}{n}\right)^n e^{x/2}\, dx =
 \int_0^\pi e^{-x/2}\,dx = -2e^{-\pi/2} - (-2e^{-0/2}) = -2e^{-\pi/2} - 2.
\end{equation*}

\item
We compute the limit
\begin{equation*}
\lim_{n\to\infty} \left(1 + \frac{x}{n}\right)^n e^{-2x}
=  e^{-2x} \lim_{n\to\infty} \left(1 + \frac{x}{n}\right)^n \\
=  e^{-2x} e^x = e^{-x}.
\end{equation*}

The integrand \(\left(1 + \frac{x}{n}\right)^n e^{-2x}\)
is bounded on \([0,\pi]\) by  $2$.
Using \(\int e^{-x}\,dx = -e^{-x} + c\) and
Lebesgue Dominated Theorem
\begin{equation*}
 \lim_{n\to\infty} \int_0^\pi \left(1 + \frac{x}{n}\right)^n e^{-2x}\, dx =
 \int_0^\pi e^{-x}\,dx = -e^{\pi} - (-e^0) = -e^{-\pi/2} + 1.
\end{equation*}

\end{itemize}


%%%%%%%%%%%%%%
\begin{excopy}
If $m$ is a Lebesgue measurable on \(\R^k\), prove that \(m(-E) = m(E)\),
where
\begin{equation*}
 -E  = \{-x: x\in E\},
\end{equation*}
and hence that
\begin{equation*}
 \int_{\R^k} f(x)dx = \int _{\R^k} f(-x)dx =
\end{equation*}
for all \(f\in L^1(\R^k)\).
\end{excopy}

For any $k$-cell $W$, we have
\begin{equation*}
m(W) = \vol(W) = \prod_{i=1}^k (\beta_i - \alpha_i) =
\prod_{i=1}^k ((-\alpha_i) - (-\beta_i)) = \vol(-W) = m(W).
\end{equation*}
Since $E$ is a limit of countable union of $k$-cells
\(m(E) = m(-E)\) follows.
The equality of integration is now simply matter of mapping \(\R^k \to \R^k\)
and a change of variable.

%%%%%%%%%%%%%% 17
\begin{excopy}
Let $X$ be the plane, with the following topology: A set is open
if and only if its intersection with every vertical line is an open subset
of that line,
with respect to the usual topology of \(\R^1\).
Show that this $X$ s a locally compact Hausdorff space.
If \(f\in C_c(X)\), let \seqxn\ be those values of $x$
for which \(f(x,y)\neq 0\) for at least one $y$
(there are only finitely many such $x$!), and define
\begin{equation*}
\Lambda f = \sum_{j=1}^n \int_{-\infty}^\infty f(x_j,y) dy.
\end{equation*}
Let \(\mu\) be the measure associated with this \(\Lambda\) by Theorem~2.14.
If $E$ is the $x$-axis, show that \(\mu(E) = \infty\) although
\(\mu(K) = 0\) for every compact \(K\subset E\).
\end{excopy}

For any two distinct point \(P_i = (x_i,y_i)\in \R^2 = X\) where \(i=1,2\)
we present open sets \(V_1\) and \(V_2\) that separate the points.
If \(x_1\neq x_2\), then we take  \(V_i = \{x_1\}\times\R\),
otherwise \(y_1\neq y_2\) and we take
\begin{equation*}
V_i = \{(x_i,y)\in\R^2: |y-y_i| < |y_1-y_2|/2\}.
\end{equation*}

For Any point \(P=(x,y)\in\R^2=X\) and any neighborhood $V$ of $P$,
we can find \(\epsilon>0\) such that
\begin{equation*}
 G = \{(x,v)\in\R^2: |v-y| < \epsilon\} \subset
\overline{G} = \{(x,v)\in\R^2: |v-y| \leq \epsilon\}  \subset V
\end{equation*}
and clearly \(\overline{G}\) is compact,

Now that we have shown that $X$ is locally compact Hausdorff space,
we proceed. Let \(f\in C_c(X)\).
Let \(W\subset \R\)  be of all
\(x\in W\) such that there exist $y$ such that \(f(x,y)\neq 0\).
If by negation $W$ is inifinite, then
the family of open sets
\begin{equation*}
 \{\,\{x\}\times\R\, : x\in W\}
\end{equation*}
is a covering of \(\supp f\) which has no finite sub-covering.
This contradicts the assumption that \(f\in C_c(\R)\).

By looking at vertical finite segments
\(S = \{(x,y)\in\R^2: x=x_0 \wedge y_0\leq y\leq y_1\}\),
clearly \(\Lambda \chhi_S = y_1 - y_0\) hence \(\mu(S) = y_1-y_0\).
Let $E$ is the $x$-axis. The last assertion is also true for any
non empty sub interval \([a,b]\subset E\).
Any open super set \(V \supset [a,b]\) contains a continuum
of vertical segments and hence \(\mu(V) = \infty\).
By Theorem~2.14(c) \(\mu([a,b] = \infty\).

The inifinite subsets of $E$
are covered by inifinitely many vertical open segments
with no sub-finite sub-covering.
Hence, the compact subsets of $E$ are exactly the finite subsets.
Clearly the measure of a singleton \(\mu(\{(x,y)\}) = 0\),
and thus \(\mu(K)=0\) for any compact \(K\subset E\).



%%%%%%%%%%%%%% 18
\begin{excopy}
This exercise requires more set-theoretic skill than the preceding
ones.  Let $X$ be a well-ordered uncountable set which has a last
element \(\omega_1\), such that every predecessor of \(\omega_1\) has
at most countably many predecessors.
(``Construction'': Take any well-ordered set which has elements with uncountably
many predecessors, and let \(\omega_1\) be the first of these;
\(\omega_1\) is called the first uncountable ordinal.)
For \(\alpha\in X\), let \(P_\alpha\) [\(S_\alpha\)] be the set of all
predecessors (successors) of \(\alpha\),
and call a subset of $X$ open if it is
a \(P_\alpha\) or an \( S_\beta\)
or an  \(P_\alpha \cap S_\beta\)
or a union of such sets.
Prove that $X$ is then a compact Hausdorff space.
(\emph{Hint}: No well-ordered set contains an infinite decreasing sequence.)

Prove that the complement of the point \(\omega_1\) is an open set
\index{sigma-compact@\(\sigma\)-compact}
which is not \(\sigma\)-compact.

Prove that to every \(f\in C(X)\)  there corresponds an \(\alpha \neq \omega_1\)
such that $f$ is constant on \(S_\alpha\).

Prove that the intersection of every countable collection \(\{K_\alpha\}\)
of uncountable compact subsets of $X$ is uncountable.
(\emph{Hint}: Consider limits of increasing countable sequences in $X$ which
intersects each \(K_n\) in infinitely many points.)

Let \frakM\ be the collection of all \(E\subset X\) such that either
\(E\cup \{\omega_1\}\) or
\(E^c\cup \{\omega_1\}\) contains an uncountable compact set; in the first case,
define \(\lambda(E)=1\); in the second case, define \(\lambda(E) = 0\).
Prove that \frakM\ is a \(\sigma\)-algebra which contains all Borel sets
in $X$, that \(\lambda\) is a measure on \frakM\ which is \emph{not}
regular (every neighborhood of \(\omega_1\) has measure $1$), and that
\begin{equation*}
 f(\omega_1) = \int_X f\, d\lambda
\end{equation*}
for every \(f\in C(X)\). Describe the regular \(\mu\) which Theorem~2.14
associates with this linear functional.
\end{excopy}


\paragraph{Hausdorff.} Let \(\alpha<\beta\) be any two points in $X$.
Clearly \(\alpha+1\leq \beta\) and then  \(P_{\alpha+1}\) and \(S_{\alpha}\)
are open sets that separate the two points.

\paragraph{Compactness.} Let \(\{G_i\}_{i\in I}\) be a family of open
sets such that \(X=\cup_{i\in I}G_i\). By negation, assume that
there is no finite sub-covering. We build
\begin{itemize}
\item A sequence of sets
     \(\{G_{i_j}: j\in\{0\}\cup \N\;\textrm{and}\; i_j \in I\}\).
\item A strictly decreasing sequence of points \(\{\alpha_j\}_{i\in\N}\)
\end{itemize}
by induction.
For notational convenience we use ``\(G_j=G_{i_j}\)''
and we denote \(U_n = \cup_{j=0}^n G_j\).
Pick \(G_0\) such that \(\omega_1\in G_0\).

By looking at base neighborhoods, we see that
each non empty open set (except for \(\{1\}\)).
contains a
non empty open set of the form \(S_\alpha\) or \(S_\alpha \cap P_\beta\)

In step $n$, let \(\alpha_n\) be the least point such that
\(S_{\alpha_j} \subset U_{n-1}\).
or \(S_{\alpha_j}\cap P_\beta \subset U_{n-1}\) for some
\(\beta \geq \alpha+1\).
Note that \(\alpha_j \notin \cup_{j<n} G_j\). Now we pick \(G_n\) such that
\(\alpha_n \in G_n\).
By construction, there is a base neighborhood of \(\alpha_1\)
that is contained in \(U_n\). Thus \(\{\alpha_j\}_{i\in\N}\) is strictly
decreasing. Hence has no minimal element in contradiction to $X$ being
well ordered.


\paragraph{Non \(\sigma\)-compactness of \(X\setminus\{\omega_1\}\).}
Let \(Y= X\setminus\{\omega_1\}\). Now
\(\{P_y: y\in Y\}\) is an open covering of $Y$.
By exercise's, assumption \(|P_y| \leq \aleph_0\) for all \(y\in Y\).
If by negation $X$ is \(\sigma\)-compact, then be looking at the sub-compacts
\(X=\cup_{i\in\N} K_n\), the finite sub-covering of each and finally
joining the sub-coverings to a countable sub-covering
\begin{equation*}
 Y = \bigcup_{i\in N} P_{y_i}.
\end{equation*}
But then \(|Y| \leq |\N|\cdot\aleph_0 = \aleph_0\)
contradiction to the fact \(|Y|=|X|>\aleph_0\).

\paragraph{Constant Tail.}
Indeed some ``set-theoretic skill'' is needed.
Here we use some terms and results from \cite{Dug1966}.
\paragraph{Definitions} (\cite{Dug1966} \textsf{II 3.1}).
Let $W$ be a well ordered set
\index{ideal!set}
\index{initial interval}
The set of ideals \(I(W)\),
an initial interval \(W(a)\) for each \(a\in W\))
and the set of initial interval \(J(W)\)
are defined as follows:
\begin{eqnarray*}
I(W) &=& \{S\subset W: \forall x\in S\wedge y < x \Rightarrow y\in S \\
W(a) &=& \{x\in W: x < a \wedge x\neq a\}  \\
J(W) &=& \{W(a): a\in W\}
\end{eqnarray*}

%%%%%%%%%%%%%%%%%%%%%%%%%%%%%%%%
\begin{llem} \label{llem:set:ideals}
\textnormal{(\cite{Dug1966} \textsf{II 3.2(a)})}
Let $W$ be a well ordered set, then \(J(W) = I(W) \setminus \{W\}\).
\end{llem}
\begin{thmproof}
\(J(W) \subset I(W) - \{W\}\): Each initial interval is obviously an ideal
but not the whole $W$.
Conversely, let \(S\in I(W) \setminus \{W\}\) Then
\(U = W\setminus S\neq \emptyset\), so $U$ has a first element $a$.

We now show that \(W(a)=S\).
If \(x\in W(a)\), and if by negation \(x notin S\) then \(x\in U\)
contradiction to \(a=\min(U)\) since \(x<a\).
Otherwise \(x\notin W(a)\), but then \(a\leq x\) and so \(x\notin S\)
for if \(x\in S\) by being ideal we would have \(a\in S\).
\end{thmproof}

%%%%%%%%%%%%%%%%%%%%%%%%%%%%%%%%
\begin{llem} \label{llem:countable:ub}
\textnormal{(\cite{Dug1966} \textsf{II 9.1})}
Each countable subset of the ordinals \([0,\omega_1)\) has an upper bound in
 \([0,\omega_1)\).
\end{llem}
\begin{thmproof}
Let \(L = [0,\omega_1) \) the set of all ordinals less than \(\omega_1\).
Let \(A\subset L\) and let $S$ be the ideal
\begin{equation} \label{eq:UWa}
  S =
  \bigcup_{\alpha\in A} W(\alpha) = \bigcup_{\alpha\in A} \{x\in L: x< \alpha\}
  \subset L.
\end{equation}
Since the cardinality of the ordinal \(\omega_1\) is not equal
to any smaller ordinal, \(|W(\alpha)| \leq \aleph_0\)
for each \(\alpha < \omega_1\).
So the above \(\ref{eq:UWa}\) is a countable union of countable sets.
% Since \(\aleph_0\cdot \aleph_0 = \aleph_0\) using
\begin{equation*}
 |S| = \aleph_0 \cdot  \aleph_0 = \aleph_0 < \aleph_1 = |L|.
\end{equation*}
Consequently $S$ cannot be isomorphic to $L$. By lemma~\ref{llem:set:ideals},
\(S=[0,\beta)\) for some \(\beta < \omega_1\) and \(\beta\) is
the least upper bound of $A$.
\end{thmproof}

The following lemma solves the ``constant tail'' statement.
Note that in \cite{Dug1966} the underlying  space does \emph{not}
contain (``the last'') \(\omega_1\).

\index{Vickery}
\begin{llem} \textnormal{(\cite{Dug1966} \textsf{III 8.4 Ex.7}
                         \textrm{Vickery})}
 \label{llem:Vickery}
With the above interval topology defined above on \([0,\omega_1]\) and
induced on \([0,\omega_1)\),
If \(f\in C([0,\omega_1])\)
then $f$ is continuous on a tail \([\beta,\omega_1]\).
\end{llem}
\begin{thmproof}
We first assert that
\begin{equation} \label{eq:f:omega:Cauchy}
 \forall n\in\N,
 \exists \alpha_n < \omega_1\,
 \forall \xi \in (\alpha_n,\omega_1)\;
         |f(\xi) - f(\alpha_n)| < 1/n.
\end{equation}
Intuitively --- $f$ behaves like a Cauchy sequence.
By negation, assume
\begin{equation*}
 \exists n_0\in\N,
 \forall \alpha < \omega_1\,
 \exists \xi \in (\alpha,\omega_1)\;
         |f(\xi) - f(\alpha)| \geq 1/n_0.
\end{equation*}
Now we build by induction an increasing sequence \(\{\xi_i\}_{i\in \N}\)
such that \(|f(\xi) - f(\alpha)| \geq 1/n_0\).
In the  \((i+1)\)-th step of the induction we look for the first \(\xi_{i+1}\)
within \((\xi_i,\omega_1)\) that satisfies the hypothesis.
By lemma~\ref{llem:countable:ub}, \(\xi_{i+1}\) have a least upper bound
\(\gamma < \omega_1\). But then $f$ would not be continuous at \(\gamma\).
This proves (\ref{eq:f:omega:Cauchy}).


Now let \(\beta\) be an upper bound of \(\{\alpha_i\}_{i\in \N}\).
Again by lemma~\ref{llem:countable:ub} \(\beta<\omega_1\).
Now if \(\zeta\in(\beta,\omega_1)\), then
\begin{equation*}
|f(\zeta) - f(\beta)| \leq
|f(\zeta) - f(\alpha_n)| + |f(f(\beta) - f(\alpha_n)| = 2/n
\end{equation*}
for every $n$, thus \(f(\zeta) = f(\beta)\), for every \(\zeta\) such that
\(\beta < \zeta < \omega_1\).
As for last point, If by negation \(f(\omega_1)\neq f(\beta)\) then we could
find a base neighborhood interval \(I=S_{\delta}\) of \(\omega_1\)
such that for every \(x\in I\), we have
\begin{equation*}
|f(x)-f(\omega_1)| < |f(\omega_1) - f(\beta)| / 2 > 0
\end{equation*}
But then there must be an \(x\in I\setminus\{\omega_1\}\)
for which we know that \(f(x)=f(\beta)\).
Thus $f$ is constant on \(S_\beta\).
\end{thmproof}

An immediate consequence is:
\begin{llem}
With the above interval topology defined above on \([0,\omega_1]\),
If \(f\in C([0,\omega_1])\)
then $f$ is continuous on a tail \([\beta,\omega_1]\).
\end{llem}
\begin{thmproof}
With such \(f\in C([0,\omega_1])\),
put \(\tilde{f} = f_{|[0,\omega_1)}\in C([0,\omega_1))\).
Applying lemma~\ref{llem:Vickery}, we see that \(\tilde{f}\)
is constant on a tail. By continuity \(f(\omega_1)\) has this tail value,
and so $f$ is constant on a tail.
\end{thmproof}


\paragraph{Intersection of Compacts.}
Let \(\{K_n\}_{n\in\N}\) a countable family of uncountable compact sets
in \([0,\omega_1]\). We note that \(\omega_1\in K_n\) for all \(n\in\N\)
since otherwise each of their upper bound would imply countable cardinality.
Let \(K = \bigcap_{n\in\N} K_n\) It is clealy compact.
Let \(K'=K\setminus \{\omega_1\}\).
Assume by negation $K$ that is countable, and so is \(K'\).
Let $b$ be a an upper bound for \(K'\) whose existence
is implied by lemma~\ref{llem:countable:ub}.

We will build an strictly increasing sequence \(\{a_n\}_{n\in\N}\)
in \([0,\omega_1)\). We look at blocks \(B_k\) of indices of increasing size.
Let
\begin{eqnarray*}
b_0 &=& 0 \\
b_k &=& \sum_{i=1}^k i = k(k+1)/2 \qquad \textrm{for}\; k\in\N \\
B_k &=& \{m\in\N: b_{k-1} < m \leq b_k\}
\end{eqnarray*}
Note that \(\N = \disjunion_{k\in\N} B_k\). Hence
for each \(n\in\N\) there are unique \(k(n)\in\N\) such that \(n\in B_{k(n)}\).
We denote the position of $n$ within \(B_k(n)\)
by \(j(n) = n - b_{k(n)-1}\).

We now define the desired sequence by induction.
\begin{eqnarray}
a_1 &=& \min(K_1) \cap [b+1,\omega_1)  \label{eq:a1K1} \\
a_n &=& \min\left(K_{j(n)} \cap [a_{n-1},\omega_1)\right). \notag
\end{eqnarray}

The sequence \(\{a_n\}_{n\in\N}\) intersects each of the
compact sets \(\{K_n\}_{n\in\N}\) in infinitely many points.
Again by lemma~\ref{llem:countable:ub},
The limit \(u = \lim_{n\in\N} a_n < \omega_1\).
This $u$, must be the same limit of all sub-sequences, in particular
intersection of \(\{a_n\}_{n\in\N}\) with \(K_i\). Thus \(u\in K_i\),
hence \(u\in K'\), but by construction (\ref{eq:a1K1})
we have a contradiction \(u>b\).

\paragraph{The \(\lambda\) measure}
Let \frakM\ be defined as in the exercise. Clearly by definition
it is closed under completion, and it contains $X$ and \(\emptyset\).
To show it is a \(\sigma\)-algebra we need to show it is closed
under countable union. Let \(A_n\in\frakM\) for \(n\in\N\) and
let \(A=\cup_{n\in\N}A_n\). If there exists \(n\in\N\) such that
\(A_n\cup\{\omega_1\}\) contains an uncountable compact subset,
then so does $A$.  Otherwise, for every \(n\in\N\) the sets
\(A_n^c\cup\{\omega_1\}\) each contains an uncountable compact subset.
By previous result so is their intersection
\begin{equation*}
 \bigcap_{n\in\N} A_n^c\cup\{\omega_1\} =
 \left(\bigcap_{n\in\N} A_n^c\right)\cup\{\omega_1\} =
 A^c\cup\{\omega_1\}.
\end{equation*}
Hence, in both cases \(A\in\frakM\).

A closed interval \[a,b\] is uncountable iff \(a<b=\omega_1\). So clearly
\frakM\ contains all closed intervals. By completion closure, \frakM\ contains
all open intervals, hence all Borel sets.

\paragraph{The \(\mu\) measure}
Let \(\mu\) be the measure on \([0,\omega_1]\) provided by
\index{Riesz}
Riesz representation theorem. In the constructive proof
(\cite{RudinRCA80} Theorem 2.14, formula (3)) we have the definition
\begin{equation*}
 \mu(E) = \sup\{\mu(K): K\subset E,\, K\, \textrm{compact}\}.
\end{equation*}
Since the singleton \(\{\omega_1\}\) is compact, and for all \(f\in C(X)\)
we actually have \(f(\omega_1)\) as the value of the functional,
we have the following unique measure
\begin{equation*}
 \lambda(E) = \left\{\begin{array}{l@{\qquad}l}
                      1 & \omega_1 \in E \\
                      0 & \omega_1 \notin E
                     \end{array}\right.
\end{equation*}
for all Borael sets $E$.

%%%%%%%%%%%%%% 19
\begin{excopy}
If \label{ex:2:19}
\(\mu\) is an arbitrary possible measure and \(f\in L^1(\mu)\),
prove that  \(\{x: f(x)\neq 0\}\) has \(\sigma\)-finite measure.
\end{excopy}

Say the measure \(\mu\) is over $X$. Let
\begin{eqnarray*}
A_0 &=& \emptyset  \\
A_n &=& \{x\in X: |f(x)| > 1/n\} \setminus A_{n-1}
\end{eqnarray*}
Now
\begin{equation*}
\int |f|\,d\mu
 =    \cup_{n\in\N} \int_{A_n} |f|\,d\mu
 \geq \cup_{n\in\N} \mu(A_n)/n.
\end{equation*}
If \(mu\) were not \(\sigma\)-finite measure, then there would exist
\(n\in\N\) such that \(\mu(A_n)=\infty\).
But then \(\int |f|\,d\mu = \infty\) contrsicting the
assumption \(f\in L^1(\mu)\).


%%%%%%%%%%%%%%
\begin{excopy}
A positive measure \(\mu\) on a set $X$ is called
\index{sigma-finite@\(\sigma\)-finite}
\emph{\(\sigma\)-finite} if $X$ is a countable union of sets \(X_i\)
with \(\mu(X_i) < \infty\). Prove that \(\mu\) is \(\sigma\)-finite
if and only if there exists  \(f\in L^1(\mu)\)
such that \(f(x)>0\) for every \(x\in X\).
\end{excopy}

One direction was shown in the above exercise~\ref{ex:2:19}.
Conversely, say \(\mu\) is a \(\sigma\)-finite measure over $X$.
Hence we have \(X = \disjunion_{n\in\N} X_i\) and \(\mu(X_n)<\infty\).
Define
\begin{equation*}
f(x) = 2^{-n}\,/\,\max(\mu(X_n),1) \qquad \textrm{if}\; x \in X_i.
% \left\{ \begin{array}{l@{\qquad}l}
\end{equation*}
Now clearly
\begin{equation*}
\int_X |f|\,d\mu
 = \sum_{n\in\N} \int_{X_n} |f|\,d\mu
 \leq \sum_{n\in\N} 2^{-n} = 1.
\end{equation*}


%%%%%%%%%%%%%%%
\end{enumerate}
%%%%%%%%%%%%%%%

 % \setcounter{chapter}{2}  % -*- latex -*-
% $Id: rudinrca3.tex,v 1.4 2008/07/19 08:56:55 yotam Exp $

%%%%%%%%%%%%%%%%%%%%%%%%%%%%%%%%%%%%%%%%%%%%%%%%%%%%%%%%%%%%%%%%%%%%%%%%
%%%%%%%%%%%%%%%%%%%%%%%%%%%%%%%%%%%%%%%%%%%%%%%%%%%%%%%%%%%%%%%%%%%%%%%%
%%%%%%%%%%%%%%%%%%%%%%%%%%%%%%%%%%%%%%%%%%%%%%%%%%%%%%%%%%%%%%%%%%%%%%%%
\chapterTypeout{\ensuremath{L^p}-Spaces} % Chapter 3


%%%%%%%%%%%%%%%%%%%%%%%%%%%%%%%%%%%%%%%%%%%%%%%%%%%%%%%%%%%%%%%%%%%%%%%%
%%%%%%%%%%%%%%%%%%%%%%%%%%%%%%%%%%%%%%%%%%%%%%%%%%%%%%%%%%%%%%%%%%%%%%%%
\section{Notes}

\index{Jensen}
In 3.3~Theorem (Jensen's Inequality) to derive the (2) inequality:
\begin{equation*}
 \varphi(s) \geq \varphi(t) + \beta\varphi(s-t) \qquad (a<s<b)
\end{equation*}
one should check the cases: \(s<t\) and \(t<s\) separately.


%%%%%%%%%%%%%%%%%%%%%%%%%%%%%%%%%%%%%%%%%%%%%%%%%%%%%%%%%%%%%%%%%%%%%%%%
%%%%%%%%%%%%%%%%%%%%%%%%%%%%%%%%%%%%%%%%%%%%%%%%%%%%%%%%%%%%%%%%%%%%%%%%
\section{Inequalities}

Here we bring and use results from \cite{Hardy:1952:I}.

%%%%%%%%%%%%%%%%%%%%%%%%%%%%%%%%%%%%%%%%%%%%%%%%%%%%%%%%%%%%%%%%%%%%%%%%
\subsection{Proportional Vectors}

Two vectors \(\mathbf{a} = (a_i)_{i=1}^n\)
and \(\mathbf{b} = (b_i)_{i=1}^n\) are said to be \emph{proportional}
iff there exist a scalar \(\mu\) such that
\(\mu a = b\)  or \( a = \mu b\).

The following trivial lemmas do not need a detailed proof.

\begin{lem}
Any vector is proportional to the zero vector of same dimensionality.
\end{lem}

\begin{lem}
For non zero vectors, the proportionality
is an equivalence relation.
\end{lem}


\begin{lem} \label{lem:prop:det}
The vectors \(\mathbf{a} = (a_i)_{i=1}^n\)
and \(\mathbf{b} = (b_i)_{i=1}^n\) are proportional
iff \(a_i b_j - a_j b_i = 0\) for all \(1\leq i,j \leq n\).
\end{lem}


\begin{lem}
Given a \(m\times n\) matrix \(A = (a)_{ij}\) where \(1\leq i \leq m\)
and \(1\leq j \leq n\).
The $m$ row vectors are proportional to each other
iff
the $n$ column vectors are proportional to each other.
\end{lem}
\begin{thmproof}
Applying Lemma~\ref{lem:prop:det}.
\end{thmproof}



%%%%%%%%%%%%%%%%%%%%%%%%%%%%%%%%%%%%%%%%%%%%%%%%%%%%%%%%%%%%%%%%%%%%%%%%
\subsection{Cauchy Inequality}

Let's prove the Cauchy inequality.
\begin{thm} \label{thm:cauchy}
\begin{equation} \label{eq:cos}
 \left(\sum_{i=1}^n a_i b_i\right)^2 \leq
 \left(\sum_{i=1}^n a_i^2\right)
 \left(\sum_{i=1}^n b_i^2\right)
\end{equation}
and equality happens iff
\(\mathbf{a}=(a_i)_{1\leq i\leq n}\)
and
\(\mathbf{b}=(b_i)_{1\leq i\leq n}\)
are proportional.
\end{thm}
\begin{thmproof}
Looking at the \(n^2\) square terms sum
\begin{equation}
D = \sum_{1\leq i,j \leq n} (a_i b_j - a_j b_i)^2 \geq 0
\end{equation}
carefully, we see that \(D=0\) iff
\(\mathbf{a}\) and \(\mathbf{b}\) are proportional.
Evaluate
\begin{eqnarray*}
D
  &=& \sum_{1\leq i,j \leq n} (a_i b_j - a_j b_i)^2 \\
  &=& \sum_{\ineqjton} (a_i b_j - a_j b_i)^2 \\
  &=& \sum_{\ineqjton} a_i^2 b_j^2 + a_j^2 b_i^2
                             - 2 a_i a_j b_i b_j \\
  &=& 2 \sum_{\ineqjton} a_i^2 b_j^2 - a_i a_j b_i b_j \\
\end{eqnarray*}

With the following definition of \(\Delta\), the inequality (\ref{eq:cos})
is equivalent to \(\Delta\geq 0\), which we will now show.
\begin{eqnarray*}
 \Delta
 &=& \left(\sum_{i=1}^n a_i^2\right) \left(\sum_{i=1}^n b_i^2\right) -
     \left(\sum_{i=1}^n a_i b_i\right)^2 \\
 &=& \left(\sum_{\ineqjton}{a_i^2 b_j^2} + \sum_{i=1}^n{a_i^2 b_i^2} \right)
     -
     \left(
       \sum_{i=1}^n{a_i^2 b_i^2} +
       \sum_{\ineqjton}{a_i a_j b_i b_j}
     \right) \\
 &=& \sum_{\ineqjton}{a_i^2 b_j^2} - \sum_{\ineqjton}{a_i a_j b_i b_j} \\
 &=& D/2
\end{eqnarray*}
\end{thmproof}


%%%%%%%%%%%%%%%%%%%%%%%%%%%%%%%%%%%%%%%%%%%%%%%%%%%%%%%%%%%%%%%%%%%%%%%%
\subsection{Arithmetic and Geometric Means}

Our main result here is showing that the geometric mean is bounded
by the arithmetic mean. The book \cite{Hardy:1952:I} gives several proofs,
here we follow the shortest (but less intuitive).

\paragraph{Definitions:}
Let \seqn{q} satisfy \(0\leq q_i\leq 1\) and
\(\sum_{i=1}^n q_i = 1\) and let \(a=\seqn{a}\) be non negative scalars.
\begin{itemize}
 \item For \(0<r<\infty\) the $r$-mean is
   \begin{equation}
     \frakM_r(a) = \left(\sum_{i=1}^n q_i a_i^r\right)^{1/r}
   \end{equation}
 \item We call \(\frakM_1(a)\) the \emph{arithmetic mean}.
 \item We define the \emph{geometric mean} by
        \(\frakG(a) = \prod_{i=1}^n a_i^{q_i}\).
\end{itemize}

\begin{lem} \label{lem:Mr:Mr2}
Given the assumption of the above definitions,
\begin{equation} \label{eq:Mr:Mr2}
\frakM_r(a) \leq \frakM_{2r}(a)
\end{equation}
where equality holds
iff \(a_i=a_1\) for all \(1\leq i \leq n\).
\end{lem}
\begin{thmproof}
The (\ref{eq:Mr:Mr2}) inequality
is equivalent (after taking power of \(2r\)) to
\begin{equation*}
\left(\sum_{i=1}^n q_i a_i^r\right)^2  \leq \sum_{i=1}^n q_i a_i^{2r}
\end{equation*}
Appling Cauchy's Theorem~\ref{thm:cauchy} substituting \(a_i\) and \(b_i\)
by \(\sqrt{q_i}\) and \(a_i^r\sqrt{q_i}\) gives
\begin{eqnarray*}
\left(\sum_{i=1}^n q_i a_i^r\right)^2
&=& \left(\sum_{i=1}^n \sqrt{q_i} \cdot (a_i^r\sqrt{q_i})\right)^2 \\
&\leq & (\sum_{i=1}^n \sqrt{q_i}^2)
        (\sum_{i=1}^n (a_i^r\sqrt{q_i})^2) \\
&=& 1 \cdot \sum_{i=1}^n q_i a_i^{2r}
\end{eqnarray*}
Now that inequality was shown, note that equality holds iff
\((\sqrt{q_i})_{i=1}^n\) and
\((a_i^r\sqrt{q_i}_{i=1}^n)\)  are proportional which is equivalent
to \(a_i^r\) all be equal. Since \(a_i\geq 0\), this
is equivalent to all \(a_i\) being equal.
\end{thmproof}


Now we will justify the notation \(\frakM_0 = \frakG\).
\begin{lem} \label{lem:meanr0:g}
Given the assumption of the above definitions,
\begin{equation*}
\lim_{r\to 0} \frakM(a) = \frakG(a)\,.
\end{equation*}
\end{lem}
\begin{thmproof}
Given \(r>0\), we compute
\begin{eqnarray}
\frakM_r(a)
 &=& \notag
  \exp\left(\log\left(\biggl(\sum_{i=1}^n q_i a_i^r\biggr)^{1/r}\right)\right) \\
 &=& \notag
  \exp\left(\frac{1}{r}\log\biggl(\sum_{i=1}^n q_i a_i^r\biggr)\right) \\
 &=& \notag
  \exp\left(\frac{1}{r}\log\biggl(\sum_{i=1}^n q_i a_i^r\biggr)\right) \\
 &=& \label{eq:Mr0G:taylor}
  \exp\biggl(\log\bigl(1 + r\sum_{i=1}^n q_i \log(a_i) + O(r^2)\bigr)\big/r\biggr)
\end{eqnarray}

Where the equality in (\ref{eq:Mr0G:taylor}) is the Taylor expansion
of \(\sum_{i=1}^n q_i a_i^r\)
at \(r=0\). Note \(a_i^r = e^{r\log(a_i)}\) and so
\(\frac{d}{dr} q_i a_i^r = q_i r \log(a_i)a_i^r\)

Before taking limit of the above, we concentrate first on subexpression.
Put
\begin{equation*}
 % L_r = \log\bigl(1 + r\sum_{i=1}^n q_i \log(a_i) + O(r^r)\bigr) \bigm/ r
 U(r) = 1 + r\sum_{i=1}^n q_i \log(a_i) + O(r^r)
\end{equation*}
and by l'Hospital's rule
\begin{eqnarray*}
\lim_{r\to 0} \log(U(r))/r
&=&
 \lim_{r\to 0} \frac{d}{dr} \log(U(r)) \\
&=&
 \lim_{r\to 0} \left(\frac{d}{dr} U(r) \right) \bigm/ U(r)  \\
&=&
 \lim_{r\to 0} \left(\sum_{i=1}^n q_i \log(a_i) + \frac{d}{dr}O(r^r)\right) / 1
 \\
&=& \lim_{r\to 0} \left(\sum_{i=1}^n q_i \log(a_i) + O(r)\right) \\
&=& \sum_{i=1}^n q_i \log(a_i)
\end{eqnarray*}


Finally the computation of the desired limit
\begin{eqnarray*}
\lim_{r\to 0} \frakM_r(a)
&=&
 \lim_{r\to 0}
 \exp\biggl(\log\bigl(1 + r\sum_{i=1}^n q_i \log(a_i) +
                      O(r^2)\bigr)\big/r\biggr) \\
&=&  \lim_{r\to 0} \exp(\log(U(r))/r) \\
&=&  \exp\left(\sum_{i=1}^n q_i \log(a_i)\right) \\
&=&  \prod_{i=1}^n a_i^{q_i} \\
&=&  \frakG(a)
\end{eqnarray*}
\end{thmproof}

The goal of the this section is to compare the
geometric mean \(\frakG(a)\) with the arithmetic mean \(\frakM_1(a)\)
\begin{thm} \label{thm:geo:arith}
Given the assumption of the above definitions,
\begin{equation*}
\frakG(a) \leq \frakM_1(a)
\end{equation*}
where equality holds iff
iff \(a_i=a_1\) for all \(1\leq i \leq n\).
\end{thm}
\begin{thmproof}
Utilizing lemma~\ref{lem:Mr:Mr2} and lemma~\ref{lem:meanr0:g},
we compute
\begin{equation*}
 \frakM_1(a) \geq \frakM_{\frac{1}{2}}(a)
 \cdots     \geq \frakM_{2^{-k}}(a)
 \cdots     \geq \lim_{m\to\infty} \frakM_{2^{-m}}(a) = \frakG(a).
\end{equation*}
The inequalities are equalities, as was shown in lemma~\ref{lem:Mr:Mr2},
iff \(a_i\)'s are constant.
\end{thmproof}



%%%%%%%%%%%%%%%%%%%%%%%%%%%%%%%%%%%%%%%%%%%%%%%%%%%%%%%%%%%%%%%%%%%%%%%%
\subsection{Generalizing Cauchy Inequality}

%%%%%%%%%%%%%%%%%%%%%%%%%%%%%%%%
\begin{lem} \label{lem:eqgeom}
Let $n$ and $g=2^m$ be positive integers,
if \(a_{ij}\geq 0\)
for all \(1\leq i \leq n\), \(1\leq j \leq g\),
then
\begin{equation} \label{eq:eqgeom}
 \left(\sum_{i=1}^n \prod_{j=1}^g a_{ij}\right)^g
 \leq
 \prod_{j=1}^g \sum_{i=1}^n a_{ij}^g
\end{equation}
Equality happens iff (the columns)
\(\mathbf{a}_j = (a_{ij})_{i=1}^n\) are proportional, or at least
one of them is all zero.
\end{lem}
\begin{thmproof}
The case where some \(\mathbf{a}_j\) is all zeros is trivial,
so we exclude it from the rest of the proof.
By induction on $m$.
Assume \(m=0\), then \(g=1\) and (\ref{eq:eqgeom}) is a trivial equality.
Also being single dimensional vectors, they are proportional.
The case of \(m=1\), that is \(g=2\) is proved in Theorem~\ref{thm:cauchy}.
Now assume that the lemma holds for \(m=k\geq 1\).
Consider the case of \(m=k+1\)
\begin{eqnarray}
 \left(\sum_{i=1}^{n} \prod_{j=1}^{2^{k+1}} a_{ij}\right)^{2^{k+1}}
 &=& \notag
 \left(\sum_{i=1}^{n}
       \Biggl(
       \biggl(\prod_{j=1}^{2^k} a_{ij}\biggr)
       \biggl(\prod_{j=2^k + 1}^{2^{k+1}} a_{ij}\biggr)
       \Biggr)
 \right)^{2\cdot 2^k} \\
 &\leq& \label{eq:eqgeom:induc1}
 \left(
 \Biggl(\sum_{i=1}^{n}
       \biggl(\prod_{j=1}^{2^k} a_{ij}\biggr)^2
 \Biggr)
 \cdot
 \Biggl(\sum_{i=1}^{n}
       \biggl(\prod_{j=2^k+1}^{2^{k+1}} a_{ij}\biggr)^2
 \Biggr)
 \right)^{2^k} \\
 &=& \notag
 \left(\sum_{i=1}^{n}
       \prod_{j=1}^{2^k} a_{ij}^2
 \right)^{2^k}
 \cdot
 \left(\sum_{i=1}^{n}
       \prod_{j=2^k+1}^{2^{k+1}} a_{ij}^2
 \right) ^{2^k}
 \\
 &\leq& \label{eq:eqgeom:induc2}
 \left(\prod_{j=1}^{2^k} \,\sum_{i=1}^n a_{ij}^{2^{k+1}}\right)
 \cdot
 \left(\prod_{j=2^k+1}^{2^{k+1}} \;\sum_{i=1}^n a_{ij}^{2^{k+1}}\right) \\
 &=& \notag
 \prod_{j=1}^{2^{k+1}} \,\sum_{i=1}^n a_{ij}^{2^{k+1}}
\end{eqnarray}
The inequality (\ref{eq:eqgeom:induc1}) is from Cauchy's Theorem~\ref{thm:cauchy}
and the inequality (\ref{eq:eqgeom:induc2}) by induction.
The inequalities are equality iff \(\mathbf{a}_i\) are proportional.
\end{thmproof}

Now we remove the restriction of $g$ being a power of $2$.
%%%%%%%%%%%%%%%%%%%%%%%%%%%%%%%%
\begin{lem} \label{lem:cauchy:ng}
Let $n$ and $g$ be positive integers,
if \(a_{ij}\geq 0\)
for all \(1\leq i \leq n\), \(1\leq j \leq g\),
then
\begin{equation} \label{eq:eqgeom:n}
 \left(\sum_{i=1}^n \prod_{j=1}^g a_{ij}\right)^g
 \leq
 \prod_{j=1}^g \sum_{i=1}^n a_{ij}^g
\end{equation}
Equality happens iff (the columns)
\(\mathbf{a}_j = (a_{ij})_{i=1}^n\) are proportional, or at least
one of them is all zero.
\end{lem}
\begin{thmproof}
The case where some \(\mathbf{a}_i\) is all zeros is trivial,
so we exclude it from the rest of the proof.

If the vectors are proportional, then there are \((c_i)_{i=1}^n\)
such that \(a_{ij} = c_i a_{1j}\)
for all \(1\leq i \leq n\), \(1\leq j \leq g\). Inthis case
\begin{equation*}
\left(\sum_{i=1}^n \prod_{j=1}^g a_{ij}\right)^g =
\left(\sum_{i=1}^n c_i \prod_{j=1}^g a_{1j}\right)^g =
\left(\left(\sum_{i=1}^n c_i\right) \prod_{j=1}^g a_{1j}\right)^g =
\left(\prod_{j=1}^g \sum_{i=1}^n c_i a_{1j}\right)^g
\end{equation*}


The case where \(g=2^m\) for some integer $m$ was proved
in the previous Lemma.
Let $m$ be a positive integer such that \(2^{m-1} < g < 2^m\).

Put
\begin{equation*}
T_i = \prod_{j=1}^g a_{ij}^{1/{2^m}}
\end{equation*}
and define \(2^m\)-dimensional vectors
\begin{equation*}
b_{ij} = \left\{\begin{array}{ll}
               a_{ij}^{g/{2^m}} \quad & 1 \leq j \leq g\\
               T_i              \quad & g <    j \leq 2^m
               \end{array}\right.
               \qquad \textrm{(for}\quad 1\leq i \leq n, \;
                                         1 \leq j \leq 2^m \textrm{)}
\end{equation*}
Note that
\begin{equation*}
\prod_{j=1}^{2^m} b_{ij}
= \left(\prod_{j=1}^{g} a_{ij}^{g/{2^m}} \right)
  \left(\prod_{i=g+1}^{2^m} T_i \right)
= T_i^g T_i^{2^m-g} = T_i^{2^m}
= \prod_{j=1}^g a_{ij}
\end{equation*}

Applying Lemma~\ref{lem:eqgeom} we have
\begin{equation*}
 \left(\sum_{i=1}^n \prod_{j=1}^{2^m} b_{ij}\right)^{2^m}
 \leq
 \prod_{j=1}^{2^m} \sum_{i=1}^n b_{ij}^{2^m}
\end{equation*}
With this we can compute
\begin{eqnarray}
\left(\sum_{i=1}^n \prod_{j=1}^g a_{ij}\right)^{2^m}
&=&  \left(\sum_{i=1}^n \prod_{j=1}^{2^m} b_{ij}\right)^{2^m} \notag \\
&\leq& \label{eq:eqgeom:useinduc}
    \prod_{j=1}^{2^m} \sum_{i=1}^n b_{ij}^{2^m} \\
&=& \notag
        \left(\prod_{j=1}^{g} \sum_{i=1}^n b_{ij}^{2^m}\right)
        \left(\prod_{j=g+1}^{2^m} \sum_{i=1}^n b_{ij}^{2^m}\right) \\
&=&     \notag
        \left(\prod_{j=1}^{g} \sum_{i=1}^n a_{ij}^g\right)
        \left(\sum_{i=1}^n \prod_{j=1}^g a_{ij} \right)^{2^m - g}
\end{eqnarray}

The inequality (\ref{eq:eqgeom:useinduc})
is by previous lemma and it is equality iff the columns vectors
\(\mathbf{b}_j = (b_{ij})_{i=1}^n\) are proportional.
It is easy to see that the latter condition is equivalent
to the columns vectors \(\mathbf{a}_j\) being proportional.

Now \(F=\sum_{i=1}^n \prod_{j=1}^g a_{ij} = 0\) iff
\(\prod_{j=1}^g a_{ij} = 0\) for all \(1\leq i \leq n\).
In this case (\ref{eq:eqgeom:n}) holds with strict inequality,
since all the columns \(\mathbf{a}_j=0\), as the other case was excluded.

So we now may assume \(\sum_{i=1}^n \prod_{j=1}^g a_{ij} \neq 0\).
Thus from the recent inequality, we can derive (dividing by \(F^{2^m-g}\))
\begin{equation*}
\left(\sum_{i=1}^n \prod_{j=1}^g a_{ij}\right)^g
\leq \left(\prod_{j=1}^{g} \sum_{i=1}^n a_{ij}^g\right)
\end{equation*}
\end{thmproof}

%%%%%%%%%%%%%%%%%%%%%%%%%%%%%%%%%%%%%%%%%%%%%%%%%%%%%%%%%%%%%%%%%%%%%%%%
\subsection{Hold\"er Inequality}

We begin with special cases that will support
the proof of the general Holder Inequality.

\begin{lem} \label{lem:holder:eq}
Let $n$ and $g$ be positive integers,
if \(a_{ij}\geq 0\)
for all \(1\leq i \leq n\), \(1\leq j \leq g\),
then
\begin{equation}
 \sum_{i=1}^n \prod_{j=1}^g a_{ij}^{1/g}
 \leq
 \prod_{j=1}^g \left(\sum_{i=1}^n a_{ij}\right)^{1/g}
\end{equation}
Equality happens iff (the columns)
\(\mathbf{a}_j = (a_{ij})_{i=1}^n\) are proportional, or at least
one of them is all zero.
\end{lem}
\begin{thmproof}
Using Lemma~\ref{lem:cauchy:ng},
but substituting \(a_{ij}\) with \(a_{ij}^{1/g}\)
we have
\begin{equation*}
 \left(\sum_{i=1}^n \prod_{j=1}^g a_{ij}^{1/g}\right)^g
 \leq
 \prod_{j=1}^g \sum_{i=1}^n (a_{ij}^{1/g})^g
\end{equation*}
Equivelantly
\begin{equation*}
 \sum_{i=1}^n \prod_{j=1}^g a_{ij}^{1/g}
 \leq
 \prod_{j=1}^g \left(\sum_{i=1}^n a_{ij}\right)^{1/g}
\end{equation*}
and equality happens as was needed to show, since \(\mathbf{a}_j\)
are proportional iff \(\overline{\mathbf{a}}_j = (a_{ij}^{1/g})_{i=1}^n\)
that were used here are proportional.
\end{thmproof}


Next generalization would be with varying powers.
\begin{lem} \label{lem:holder:rat}
Let $n$ and $g$ be positive integers,
if \(a_{ij}\geq 0\)
for all \(1\leq i \leq n\), \(1\leq j \leq g\),
then
if \((\alpha_j)_{j=1}^g\) are rationals satisfying \(0\leq \alpha_j \leq 1\)
and \(\sum_{j=1}^g \alpha_j = 1\) then
\begin{equation}
 \sum_{i=1}^n \prod_{j=1}^g a_{ij}^{\alpha_j}
 \leq
 \prod_{j=1}^g \left(\sum_{i=1}^n a_{ij}\right)^{\alpha_j}
\end{equation}
Equality happens iff (the columns)
\(\mathbf{a}_j = (a_{ij})_{i=1}^n\) are proportional, or at least
one of them is all zero.
\end{lem}
\begin{thmproof}
Being rationals, we can find integers $M$ and \(p_j\) such that
\(\alpha_j = p_j/M\) for \(1\leq j \leq g\).
By viewing
\begin{equation*}
a_{ij}^{p_j/M} = \left(a_{ij}^{1/M}\right)^{p_j}
\end{equation*}
noting that \(\sum_{j=1}^g p_j = M\),
by converting to equal power (\(1/M\)),
we use Leamm~\ref{lem:holder:eq}
\begin{equation*}
     \sum_{i=1}^n \prod_{j=1}^g a_{ij}^{\alpha_j}
 =   \sum_{i=1}^n \prod_{j=1}^g \left(a_{ij}^{1/M}\right)^{p_j}
\leq \prod_{j=1}^g \left(\left(\sum_{i=1}^n a_{ij}\right)^{1/M}\right)^{p_j}
=    \prod_{j=1}^g \left(\sum_{i=1}^n a_{ij}\right)^{p_j/M}
=    \prod_{j=1}^g \left(\sum_{i=1}^n a_{ij}\right)^{\alpha_j}
\end{equation*}
Again the inequality is an equality, when the same conditions
required in Lemma~\ref{lem:holder:eq} hold.
\end{thmproof}

Finally removing the restriction for \(\alpha_i\in\Q\),
\textbf{Hold\"er}'s inequality,
\begin{llem} \label{lem:holder}
Let $n$ and $g$ be positive integers,
if \(a_{ij}\geq 0\)
for all \(1\leq i \leq n\), \(1\leq j \leq g\),
then
if \((\alpha_j)_{j=1}^g\) are reals satisfying \(0\leq \alpha_j \leq 1\)
and \(\sum_{j=1}^g \alpha_j = 1\) then
\begin{equation} \label{eq:holder}
 \sum_{i=1}^n \prod_{j=1}^g a_{ij}^{\alpha_j}
 \leq
 \prod_{j=1}^g \left(\sum_{i=1}^n a_{ij}\right)^{\alpha_j}
\end{equation}
Equality happens iff (the columns)
\(\mathbf{a}_j = (a_{ij})_{i=1}^n\) are proportional, or at least
one of them is all zero.
\end{llem}
\begin{thmproof}
By applying limit process with rationals converting to reals to
Lemma~\ref{lem:holder:rat} we get the desired result, except for the
(temporary) loss of strict inequality (for the non proportional case).
So for now we know that
\begin{itemize}
 \item  If the vectors \(\mathbf{a}_j\) are proportional
        or at least one of the is zero,
        then equality holds in (\ref{eq:holder}).
 \item  Otherwise, (\ref{eq:holder}) holds.
\end{itemize}
We would now show that in the second case, it is indeed strict inequality.

We now assume that  \(\alpha_j \neq 0\)
for all \(1\leq j \leq g\). Otherwise,
we can simply drop the corresponding $j$-th columns,
without effecting the expression values in (\ref{eq:holder}).
For \(1\leq j \leq g\) we
partition \(\alpha_j = q_j + \beta_j\)
such that both \(q_j, \beta_j > 0\) and \(q_j \in \Q\).
Put,
\(q = \sum_{j=1}^g q_j\) and \(\beta = \sum_{j=1}^g \beta_j\).
Clearly \(q+\beta=1\) and \(q,\beta \in \Q\).
Now we define
\begin{eqnarray*}
Q_i &=& \prod_{j=1}^g a_{ij}^{q_j/q} \\
B_i &=& \prod_{j=1}^g a_{ij}^{\beta_j/\beta}.
\end{eqnarray*}
We know
\begin{eqnarray}
\sum_{i=1}^n Q_i = \sum_{i=1}^n \prod_{j=1}^g a_{ij}^{q_j/q}
  &<& \label{eq:holder:Qlt}
      \prod_{j=1}^g \left(\sum_{i=1}^n  a_{ij}\right)^{q_j/q} \\
\sum_{i=1}^n B_i = \sum_{i=1}^n \prod_{j=1}^g a_{ij}^{\beta_j/\beta}
  &\leq& \label{eq:holder:Bleq}
      \prod_{j=1}^g \left(\sum_{i=1}^n  a_{ij}\right)^{\beta_j/\beta}
\end{eqnarray}
where (\ref{eq:holder:Qlt}) resulted by Lemma~\ref{lem:holder:rat}
while (\ref{eq:holder:Bleq}) was shown in the beginning of the proof.
Finally
\begin{eqnarray*}
 \sum_{i=1}^n \prod_{j=1}^g a_{ij}^{\alpha_j}
 =    \sum_{i=1}^n Q_i^q B_i^\beta
 &\leq& \left(\sum_{i=1}^n Q_i\right)^q \left(\sum_{i=1}^n  B_i\right)^\beta \\
 &<&
    \prod_{j=1}^g
          \left(\sum_{i=1}^n  a_{ij}\right)^{q_j}
          \left(\sum_{i=1}^n  a_{ij}\right)^{\beta_j}
 =   \prod_{j=1}^g
          \left(\sum_{i=1}^n  a_{ij}\right)^{\alpha_j}
\end{eqnarray*}
\end{thmproof}


%%%%%%%%%%%%%%%%%%%%%%%%%%%%%%%%%%%%%%%%%%%%%%%%%%%%%%%%%%%%%%%%%%%%%%%%
\subsection{H\"older's Inequality for Integrals}

\paragraph{Definition.} Two functions $f$, $g$
are said to be \emph{effectively proportional} if
iff there exist a scalar \(\mu\) such that
\(\mu f = g \;\aded\)  or \(f = \mu g\;\aded\).


\begin{llem} \textnormal{(\cite{Hardy:1952:I} \textbf{188})}
\label{llem:hlp:188}
If $k$ is an integer and
\begin{itemize}
 \item \seq{q}{k} positive such that \(\sum_{i=1}^k q_i = 1\).
 \item \seq{f}{k} are measurable functions on \(X\to [0,\infty]\).
 \item \(\mu\) a positive measure on $X$.
\end{itemize}
Then
\begin{equation} \label{eq:HLP:188}
% \left\int_X \prod_{i=1}^k f_i^{q_i}\,d\mu\right. \leq
\int_X \prod_{i=1}^k f_i^{q_i}\,d\mu \leq
\prod_{i=1}^k \left(\int_X f_i\,d\mu\right)^{q_i}
\end{equation}
where equality holds iff \seqn{f} are effectively proportional.
\end{llem}

\begin{thmproof}
Geometric means is less or equation arithmetic mean as shown
in \cite{RudinRCA80} [page 64, (8)]
and also ``here'' in Theorem~\ref{thm:geo:arith}.
Thus
\begin{eqnarray*}
\frac{\int_X \prod_{i=1}^k f_i^{q_i}\,d\mu}{
 \prod_{i=1}^k \left(\int_X f_i\,d\mu\right)^{q_i}}
 &=& \int_X \left(\frac{f_i}{\int_X f_i\,d\mu}\right)^{q_i}\,d\mu \\
 &\leq& \int_X \sum_{i=1}^k \frac{q_i f_i}{\int_X f_i\,d\mu}\,d\mu \\
 &=& 1.
\end{eqnarray*}
and (\ref{eq:HLP:188}) is clear. By local lemma~\ref{lem:holder}
an equality happens iff
the functions \(f_i/{\int_X f_i\,d\mu}\) are effectively proportional,
or equivalently \(f_i\) are.
\end{thmproof}


%%%%%%%%%%%%%%%%%%%%%%%%%%%%%%%%%%%%%%%%%%%%%%%%%%%%%%%%%%%%%%%%%%%%%%%%
\subsection{Jensen's Strict Inequality}

We would now give a variation of Theorem~3.3 (Jensen's Inequality)
\cite{RudinRCA80}. We start with some definition and trivial results.

\textbf{Definition} A real functions \(\varphi\) defined on
a segment \((a,b)\) where \(-\infty\leq a < b \leq \infty\)
is called \emph{strictly convex} if the inequality
\begin{equation} \label{eq:convex:def}
 \varphi\bigl((1-\lambda)x + \lambda y\bigr) <
 (1-\lambda)\varphi(x) + \lambda\varphi(y)
\end{equation}
holds whenever \(a<x<b\), \(a<y<b\),  and \(0<\lambda<1\).

Note the differences with the Definition~3.1 of \emph{convex} function,
both with the strict inequality and avoiding \(\lambda=0,1\) cases.

\iffalse
\begin{llem} \label{lem:convex:stu}
If \(\varphi\) is strictly convex on \((a,b)\)
then for any $s$,$t$,$u$ such that \(s<t<u\)
\begin{equation} \label{eq:convex:stu}
\frac{\varphi(t) - \varphi(s)}{t-s} < \frac{\varphi(u) - \varphi(t)}{u-t}
\end{equation}
\end{llem}
\begin{thmproof}
Using \(\lambda = (t-s)(u-s)\),
we make several simple derivations from (\ref{eq:convex:def})
\begin{eqnarray}
 \varphi(t) &<& \frac{u-t}{u-s}\varphi(s) + \frac{t-s}{u-s}\varphi(t) \notag \\
  (u-t \;+\; t-s)\varphi(t) &<& (u-t)\varphi(s) + (t-s)\varphi(t) \notag \\
  (u-t)\bigl((\varphi(t) - \varphi(s)\bigr)
 &<& \notag
  (t-s)\bigl(\varphi(u) - \varphi(t)\bigr) \\
%%%
  u\varphi(t) - u\varphi(s) - t\varphi(t) + t\varphi(s)
 &<& \notag
  t\varphi(u) - t\varphi(s) - s\varphi(u) + s\varphi(t) \\
%%%
  (u-t)\bigl(\varphi(t) - \varphi(s)\bigr)
&<& \notag
    (t-s)\bigl(\varphi(u) - \varphi(t)\bigr) \\
%%%
\frac{\varphi(t) - \varphi(s)}{t-s} &<& \frac{\varphi(u) - \varphi(t)}{u-t}
\end{eqnarray}
\end{thmproof}
\fi

The following lemma shows that the divided differences of
a convex function on \((a,b))\) viewed as a function of
two variables on \(\{(x,y): a<x<y,\; a<y<b,\; x\neq y\}\)
is strictly increasing in each variable.

\begin{llem} \label{lem:convex:stu}
If \(\varphi\) is strictly convex on \((a,b)\)
then for any $s$,$t$,$u$ such that \(s<t<u\)
\begin{equation} \label{eq:convex:stu}
\frac{\varphi(t) - \varphi(s)}{t-s}
 < \frac{\varphi(u) - \varphi(s)}{u-s}
 < \frac{\varphi(u) - \varphi(t)}{u-t}
\end{equation}
\end{llem}
\begin{thmproof}
To express $t$ as a convex combination
\((1-\lambda)s + \lambda u\)
of $s$ and $u$, we
use \(\lambda = (t-s)(u-s)\) and
 \(1 - \lambda = (u-t)(u-s)\).

We put
\begin{equation*}
T = (1-\lambda)\varphi(s) + \lambda\varphi(u)
  = \frac{u-t}{u-s} \varphi(s) + \frac{t-s}{u-s} \varphi(u).
\end{equation*}
Let's first establish the trivial slope equalities
\begin{equation} \label{eq:convex:slope}
 \frac{T-\varphi(s)}{t-s}
 = \frac{\varphi(u)-\varphi(s)}{u-s}
 = \frac{\varphi(u) - T}{u-t}.
\end{equation}
Left equality:
\begin{eqnarray*}
 \frac{T-\varphi(s)}{t-s}
 &=& \frac{(u-s)(T-\varphi(s))}{(u-s)(t-s)}
 \;=\; \frac{(u-s)T - (u-s)\varphi(s)}{(u-s)(t-s)} \\
 &=& \frac{(u-t)\varphi(s) + (t-s)\varphi(u) - (u-s)\varphi(s)}{
           (u-s)(t-s)} \\
 &=& \frac{(t-s)\varphi(u) + (s-t)\varphi(s)}{(u-s)(t-s)} \\
 &=& \frac{\varphi(u)-\varphi(s)}{u-s}
\end{eqnarray*}
Similarly, the right equality:
\begin{eqnarray*}
 \frac{\varphi(u) - T}{u-t}
 &=& \frac{(u-s)(\varphi(u)-T)}{(u-s)(u-t)}
 \;=\; \frac{(u-s)\varphi(u) - (u-s)T)}{(u-s)(u-t)} \\
 &=& \frac{(u-s)\varphi(u) - (u-t)\varphi(s) - (t-s)\varphi(u)}{
           (u-s)(u-t)} \\
 &=& \frac{(u-t)\varphi(u) - (u-t)\varphi(s)}{(u-s)(u-t)} \\
 &=& \frac{\varphi(u)-\varphi(s)}{u-s}
\end{eqnarray*}
By strict convexirty, \(\varphi(t)<T\) and so we get
\begin{eqnarray*}
\frac{\varphi(t) - \varphi(s)}{t-s}
 &<& \frac{T - \varphi(s)}{t-s} \\
 &=& \frac{\varphi(u) - \varphi(s)}{u-s} \\
 &=& \frac{\varphi(u) - T}{u-t} \\
 &<& \frac{\varphi(u) - \varphi(t)}{u-t}
\end{eqnarray*}
that contains the desired double inequalitiy.
\end{thmproof}

\begin{llem}
If \(\varphi\) is strictly convex, then any value is assumed mostly twice.
\end{llem}
\begin{thmproof}
For any \(x<y<z\) in the domain of $f$, if \(f(x)=f(z)\)
then by strictly convexity \(f(y) < f(x)\).
\end{thmproof}

Here is general simple result (not related to convexity)
\begin{llem} \label{lem:fgz:igz}
If \(f>0\) is a \(\mu\)-measaurable function, on $X$ and \(\mu(X) > 0\)
then
\begin{equation*}
\int_X f\,d\mu > 0.
\end{equation*}
\end{llem}
\begin{thmproof}
Define measurable subsets: \(U_0 = \{x:X: f(x)>1\}\) and for \(j>0\) define
\(U_j = \{x:X: 1/(j+1) < f(x) \leq 1/j\}\).
Clearly \(X = \disjunion_{j=0}^\infty U_j\), and for at least some $j$
we have \(\mu(U_j) > 0\), thus \(\int_X f\,d\mu \geq \mu(U_j)/(j+1) > 0\).
\end{thmproof}

%%%%%%%%%%%%%%%%%%%%%%%%%%%%%%%%%%%%%%%%%%%%%%%%%%%%%%%%%%%%%%%%%%%%%%%%
Now for this section main result. The proof is also a variant
of that of Theorem~3.3 (\cite{RudinRCA80}).
\begin{llem} \label{lem:jensen:strict}
Let \(\mu\) be a positive measure on 
a~\(\sigma\)-algebra \frakM in a set \(\Omega\), so that \(\mu(\Omega)=1\).
If $f$ is a real function in \(L^1(\mu)\),
which is not constant \aded, satisfying
 \(a<f(x)<b\) for all \(x\in \Omega\)
and
if \(\varphi\) is strictly convex on \((a,b)\), then
\begin{equation} \label{eq:jensen:strict}
 \varphi\left(\int_\Omega f\,d\mu\right) <  \int_\Omega (\varphi\circ f)\,d\mu
\end{equation}
\end{llem}
\begin{thmproof}


Put \(t=\int_\Omega f,d\mu\). Then \(a<t<b\).

We split \(\Omega\) to a disjoint union of the following measurable subsets.
\begin{eqnarray*}
\Omega^{-} &=& \{x\in\Omega: f(x) < t\} \\
\Omega^{=} &=& \{x\in\Omega: f(x) = t\} \\
\Omega^{+} &=& \{x\in\Omega: f(x) > t\} \\
\end{eqnarray*}
By the assumption,
\(\mu(\Omega^{-}) > 0\)
or
\(\mu(\Omega^{+}) > 0\).


We look at the inequality of local lemma~\ref{lem:convex:stu}.
Let
\begin{eqnarray*}
 \beta  &\eqdef& \sup_{a<s<t} \frac{\varphi(t) - \varphi(s)}{t-s}\\
 \gamma &\eqdef& \inf_{t<u<b} \frac{\varphi(u) - \varphi(t)}{u-t}.
\end{eqnarray*}
By local lemma~\ref{lem:convex:stu},
For any $s$, $t$ such that \(a<s<t<u<b\) we have
\begin{equation*}
\frac{\varphi(t)-\varphi(s)}{t-s}
 < \beta \leq \gamma < \frac{\varphi(u)-\varphi(t)}{u-t}
\end{equation*}

{\small (Now we could arbitrary proceed using either \(\beta\) or \(\gamma\),
we choose \(\beta\) similar to the proof mention above).}
Thus we have two inequalities
\begin{eqnarray*}
  \varphi(s) &<& \varphi(t) + \beta(t-s)\\
  \varphi(u) &>& \varphi(t) + \beta(u-t)
\end{eqnarray*}
combined to
\begin{equation*}
  \varphi(r) \,>\, \varphi(t) + \beta(r-t)
\end{equation*}
for any $r$ such that \(a<r<b\) and \(r\neq t \).
Hence
\begin{eqnarray*}
  \varphi(f(x)) - \varphi(t) > \beta(f(x)-t)
       & \qquad \textrm{for}\; x\in\Omega^{-}\cup\Omega^{+} \\
  \varphi(f(x)) - \varphi(t) = \beta(f(x)-t)
       & \qquad \textrm{for}\; x\in\Omega^{=}
\end{eqnarray*}
Since \(\varphi\) is continuous, \(\varphi\circ f\) is measurable.
We integrate the above expression by $x$. Using local lemma~\ref{lem:fgz:igz}
gives
% \int_\Omega  \varphi(f(x)) - \varphi(t) + \beta(f(x)-t)\;d\mu
%% \begin{equation*}
%% \int_\Omega (\varphi\circ f)\,d\mu - \varphi\left(\int_\Omega f\,d\mu\right)
%%  > \int_\Omega \beta(f(x)-t)\,d\mu = 0.
%% \end{equation*}
\begin{eqnarray*}
\int_\Omega (\varphi\circ f)\,d\mu - \varphi\left(\int_\Omega f\,d\mu\right)
&=& \int_\Omega (\varphi\circ f)\,d\mu - \varphi(t) \\
&=& \int_\Omega \bigl(\varphi\circ f - \varphi(t)\bigr) \,d\mu \\
&=& \int_{\Omega^=} \bigl(\varphi\circ f- \varphi(t)\bigr) \,d\mu
 +  \int_{\Omega^-\cup\Omega^+} \bigl(\varphi\circ f- \varphi(t)\bigr) \,d\mu \\
&=& \int_{\Omega^=} \beta(f(x) - t)\,d\mu
 +  \int_{\Omega^-\cup\Omega^+} \bigl(\varphi\circ f- \varphi(t)\bigr) \,d\mu \\
&>& \int_{\Omega^=} \beta(f(x) - t)\,d\mu
   +  \int_{\Omega^-\cup\Omega^+} \beta(f(x) - t)\,d\mu \\
&=&  \int_\Omega \beta(f(x) - t)\,d\mu \\
&=& 0.
\end{eqnarray*}

Which gives the desired inequality (\ref{eq:jensen:strict}).
\end{thmproof}




%%%%%%%%%%%%%%%%%%%%%%%%%%%%%%%%%%%%%%%%%%%%%%%%%%%%%%%%%%%%%%%%%%%%%%%%
\subsection{Technical Calculus Results}

Here is a simple technical lemma, using basic calculus.
\begin{llem} \label{lem:fg:bnless}
Let \(f,g:[0,\infty)\to[0,\infty)\) functions, satsisfying
\begin{itemize}
 \item \emph{Boundness}: \(\int_0^M g(x)\,dx < \infty\)  for any \(M<\infty\).
 \item \emph{Boundless}: \(\int_0^\infty g(x)\,dx = \infty\).
 \item \emph{Dominance}:
   For every \(x\in[0,\infty)\), there exists \(\epsilon_x > 0\),
   such that
   \begin{equation*}
    f(x) > (1-\epsilon_x)g(x)
   \end{equation*}
   and \(\lim_{x\to\infty} \epsilon_x = 0\).
\end{itemize}
Then for every \(\eta>0\), there exists \(A<\infty\) such that
\begin{equation*}
 \int_0^a f(x)\,dx > (1 - \eta)\int_0^a g(x)\,dx
\end{equation*}
\end{llem}
\begin{thmproof}
Given (small) \(\eta>0\), let $h$ be such that \(\epsilon_x < \eta/2\)
for every \(x\geq h\). By the first two ssumptions,
there exists \(a<\infty\) such that
\begin{equation*}
 (1-\eta/2)\int_h^a g(x)\,dx > (1-\eta) \int_0^h g(x)\,dx.
\end{equation*}
Now we can derive the desired estimate
\begin{eqnarray*}
 \int_0^a f(x)\,dx
 &>& \int_0^a (1-\epsilon_x)g(x)\,dx  \\
 &=&   \int_0^h (1-\epsilon_x)g(x)\,dx  + \int_h^a (1-\epsilon_x)g(x)\,dx  \\
 &\geq& (1-\epsilon_h)\int_h^a g(x)\,dx  \\
 &\geq& (1-\eta)\int_h^a g(x)\,dx + (\eta/2))\int_h^a g(x)\,dx  \\
 &>& (1 - \eta)\int_0^a g(x)\,dx
\end{eqnarray*}

\end{thmproof}


%%%%%%%%%%%%%%%%%%%%%%%%%%%%%%%%%%%%%%%%%%%%%%%%%%%%%%%%%%%%%%%%%%%%%%%%
\subsection{Partial Sum Near Fraction}

Given sufficiently ``fine'' partition of a number,
we can find a sub-partition of a fraction of the number.
Let's be less general, but more precise.
\begin{llem} \label{lem:near:frac}
% Put \(N=\{k\in \N: 1\leq k \leq n\}\).
Assume \((a_k)_{k\in{\N_n}}\) be a finite sequence of complex numbers
and \(\alpha\in[0,1]\).
For each subset \(G\subset \N_n\) define
\begin{eqnarray*}
S(G) &\eqdef& \sum_{k\in G} a_k \\
D(G) &\eqdef& |S(G) - \alpha S(\N_n)|.
\end{eqnarray*}
If \(|a_k| \leq r > 0\) for all \(k\in \N_n\)
then there exists a subset \(H \subset \N_n\)
such that \(D(H) \leq \sqrt{2}r/2\).
\end{llem}
\textbf{Note.} There is no claim here
that \(\sqrt{2}/2\) is necessarily the best constant.
\newline
\begin{thmproof}
If \(n\leq 1\) or \(S=0\), then we take \(H=\N_n\) and we are done.
So we may assume \(n>1\) and \(S\neq 0\).
There exists \(\theta\in[0,2\pi]\) such that \(S= e^{i\theta}|S|\).
By multiplying all of the \(a_k\) by \(e^{-i\theta}\),
the above assumptions do not change.
Thus, \wlogy\ we may assume that \(0< S \in\R\).
To construct $H$, we define \((h_k)_{k\in\N_n}\),
a permutation of \(\N_n\) by induction. Intuitively,
we keep the partial sums close to the real line.

Let \(h_1=1\) (arbitrary) and assume \(h_k\) were defined for \(k\leq m\).
For \(h_{m+1}\) we need to pick from the remaining
\(R = \N_n\setminus \{h_k: 1\leq k \leq m\}\) indices.
We look at
\begin{eqnarray*}
T(m) &=& \sum_{k\leq m} a_{h_i} \\
I(m) &=& \Im\bigl(T(m)\bigr).
\end{eqnarray*}

Since $S$ is real, $R$ must contain some \(h_{m+1}=k\) for which
\(\Im(a_{h_k})\) and $I$ have opposite signs or both are zero.

By the way \(h_k\) were picked,
clearly \(|I(m)| \leq r\) for all \(1\leq m \leq n\).
Since
\begin{equation*}
\sum_{k=1}^n a_{h_k} = \sum_{k=1}^n a_k = S > 0,
\end{equation*}
there exist a first $j$, such that \(\Re(T(j)) \geq \alpha S\).
We will now show that \(z_0 =  T(j-1)\) or \(z_1 = T(j)\)
satisfy the desired requirements for $H$.
We put \(z_k = v_k + iw_k\) for \(k=0,1\) and we have the following
inequalities:
\begin{gather}
% \begin{align}
v_0 < \alpha S \leq v_1 \notag \\
w_0  \leq 0 \leq w_1 \label{eq:lem:subaver} \\
% \end{align} \\
 |z_1 - z_0| \leq r \notag
\end{gather}
\paragraph{Note.} By the choice of \(h_j\),
the (\ref{eq:lem:subaver}) inequality may actually be reversed.
But the treatment of such case is the same as with the following:
\begin{eqnarray*}
 |\alpha S - z_0|^2 +  |\alpha S - z_1|^2
 &=&
  (\alpha S - v_0)^2 + (\alpha S - v_1)^2  + w_0^2 + w_1^2 \\
 &\leq&
  (v_1 - v_0)^2 + (w_1 - w_0)^2 \\
 &=& |z_1 - z_0|^2 \\
 &\leq& r^2.
\end{eqnarray*}
Thus \(|\alpha S - z_k|^2 \leq r^2/2\) for \(k=0\) or \(k=1\), and so
\(|\alpha S - z_k| = \sqrt{2}r/2\).
\end{thmproof}


%%%%%%%%%%%%%%%%%%%%%%%%%%%%%%%%%%%%%%%%%%%%%%%%%%%%%%%%%%%%%%%%%%%%%%%%
%%%%%%%%%%%%%%%%%%%%%%%%%%%%%%%%%%%%%%%%%%%%%%%%%%%%%%%%%%%%%%%%%%%%%%%%
\section{Equalities}

%%%%%%%%%%%%%%%%%%%%%%%%%%%%%%%%%%%%%%%%%%%%%%%%%%%%%%%%%%%%%%%%%%%%%%%%
\subsection{Sequences Equalities}

\begin{llem} \label{lem:limsup:liminf}
Let \((a_i)_{i\in\N}\) and \((b_i)_{i\in\N}\) be sequences of real numbers.
Put
\begin{equation*}
a^{*} \eqdef \limsup_{n\to\infty} a_n \qquad
b_{*} \eqdef \liminf_{n\to\infty} b_n\,.
\end{equation*}

If
\begin{equation*}
 c \eqdef \lim_{n\to\infty} a_n + b_n
\end{equation*}
exists and \(-\infty < c < \infty\) and \(a^{*} < \infty\)
then
\begin{equation} \label{eq:limsup:liminf}
 c = a^{*} + b_{*}\,.
\end{equation}
\end{llem}
\begin{thmproof}
Take a subsequence \((a_{k(i)})_{i\in\N}\), such that
\begin{equation*}
a^{*} = \lim_{n\to\infty} a_{k(n)}.
\end{equation*}
Now clearly
\begin{equation*}
 c = \lim_{n\to\infty} a_{k(n)} + b_{k(n)} = a^{*} + \lim_{n\to\infty} b_{k(n)}.
\end{equation*}
But since
\begin{equation*}
 b_{*} \leq \lim_{n\to\infty} b_{k(n)}
\end{equation*}
we have
\begin{equation}
 c \geq a^{*} + b_{*}.
\end{equation}
Similarly, we can derive the reversed inequality.
Thus (\ref{eq:limsup:liminf}) follows.
\end{thmproof}


%%%%%%%%%%%%%%%%%%%%%%%%%%%%%%%%%%%%%%%%%%%%%%%%%%%%%%%%%%%%%%%%%%%%%%%%
\subsection{Equality in Minkowski's Inequality}

\index{Minkowski's inequality}

The discussion following the proof of Theorem~3.5 \cite{RudinRCA87}
show the condition under which H\"older's inequality becomes an inequality.
The case for Minkowski's  inequality is left in the etxt as an~excercise.

The next lemma intuitively says that if function
do not share arguments, together they lose their norm.
\begin{llem} \label{eq:fgp:leq:afagp}
Let $f$ and $g$ be complex measurable functions on $X$ and \(1\leq p < \infty\).
Then
\begin{equation} \label{eq:fg:absfg}
\|f+g\|_p = \||f|+|g|\|_p
\end{equation}
iff
\begin{equation} \label{eq:argf:argg}
f(x)g(x) = 0 \qquad\textrm{or}\qquad \Arg(f(x)) = \Arg(g(x))\;\aded
\end{equation}
\end{llem}
\emph{Note:} We assume \(\Arg(z) = 0\) when \(z=0\).\\
\begin{thmproof}
Assume \eqref{eq:argf:argg} holds.
Then there the function \(\theta(x) \eqdef \Arg(f(x))\)
on $X$, satisfies
\begin{eqnarray*}
f(x) &=& e^{i\theta(x)}|f(X)| \;\aded \\
g(x) &=& e^{i\theta(x)}|g(X)| \;\aded.
\end{eqnarray*}
Now
% \begin{eqnarray*}
\[
\|f+g\|_p^p
= \int_X |f(x)+g(x)|^p\,d\mu(x)
= \int_X |e^{i\theta(x)}|\cdot\bigl(|f(x)|+|g(x)|\bigr)^p\,d\mu(x)
= \||f|+|g|\|_p^p.
\]
% \end{eqnarray*}
and so \eqref{eq:fg:absfg} follows.

Conversely, assume \eqref{eq:fg:absfg} and by negation
there is \(E\subset X\) such that \(\mu(E) > 0\) where
\(f(x)g(x)\neq 0\) and
\(\Arg(f(x)) \neq  \Arg(g(x))\) for all \(x\in E\).
Let \(\delta(x) = |\Arg(f(x)) - \Arg(g(x))|\).
Using Pythagoras theorem, for \(x\in E_n\) we have
\begin{eqnarray*}
|f(x) + g(x)|^2
&=&   \bigl(|f(x)| + |g(x)|\cos(\delta(x))\bigr)^2
  + \bigl(|g(x)|\sin(\delta(x)\bigr)^2 \\
&=& (|f(x)| + |g(x)|)^2 + 2\bigl(1-\cos(\delta(x))\bigr)|f(x)g(x)|
\end{eqnarray*}
Define
\[
% E_n = \{x\in E: |f(x)| > 1/n\;\vee\; |g(x)| > 1/n\}.
E_n = \left\{x\in E: |f(x)g(x)|\bigl(1-\cos(\delta(x))\bigr) > 1/n\right\}.
\]
Clearly \(E=\cup_n E_n\), hence there exist some $n$ such that \(\mu(E_n) > 0\).
Hence for \(x\in E_n\) we have
\[
(|f(x)| + |g(x)|)^2 - |f(x) + g(x)|^2  > 2/n.
\]
Hence, dividing by \((|f(x)| + |g(x)|) + |f(x) + g(x)|\) which must be \(>0\)
\[
(|f(x)| + |g(x)|)^2 - |f(x) + g(x)|
  > 2\,\bigm/\,n\bigl((|f(x)| + |g(x)|) + |f(x) + g(x)|\bigr) > 0.
\]
Hence \(\int_{E_n} |f+g|^p < \int_{E_n} (|f|+|g|)^p\).
Finally
\begin{eqnarray*}
\|f+g\|_p^p
&=& \int_X |f+g|^p \\
&=& \int_{X\setminus E_n} |f+g|^p     + \int_{E_n} |f+g|^p \\
&<& \int_{X\setminus E_n} (|f|+|g|)^p + \int_{E_n} (|f|+|g|)^p \\
&=& \| |f| + |g| \|_p^p
\end{eqnarray*}
which contradicts the assumption \eqref{eq:fg:absfg}.
\end{thmproof}


We start with a weak
real version for condition on equality in Minkowski's inequality.
\begin{llem} \label{llem:mink:real:eq}
Let \(f,g\) be real non negative measurable functions on $X$ and \(1<p<\infty\).
If
\begin{equation}
 \| f + g \|_p = \|f\|_p + \|g\|_p
\end{equation}
then there exist
real non-negative constants \(a,b\) not both $0$, such that \(af=bg \;\aded\).
\end{llem}
\begin{thmproof}
If \(\|f\|_p = 0\) or \(\|g\|_p = 0\) the result is trivial.
Say  \(\|f\|_p = 0\), then \(f=0\,\aded\), and we take \(a=1\), \(b=0\).

We now may assume \(\|f\|_p \neq 0\) and \(\|g\|_p \neq 0\).
Denote
\[ a(x) \eqdef f(x) + g(x)\]
and the conjugate exponent $q$ such that \(1/p+1/q=1\).
Now
\begin{eqnarray}
\|f+g\|_p^p
&=& \int_X a^p
= \int_X a\cdot a^{p-1}
=    \int_X f\cdot a^{p-1} +\int_X g\cdot a^{p-1} \notag \\
&\leq& \|f\|_p \|a^{p-1}\|_q + \|g\|_p \|a^{p-1}\|_q
       \label{eq:mink:real:2holder} \\
       % \qquad \textrm{H\"older} \\
&=& \bigl(\|f\|_p + \|g\|_p\bigr)\cdot\|a^{p-1}\|_q \notag
= \bigl(\|f\|_p + \|g\|_p\bigr)\cdot\left(\int_X a^{(p-1)q}\right)^{1/q}
   \notag \\
&=& \bigl(\|f\|_p + \|g\|_p\bigr)\cdot \|a\|_p^{p/q} \notag
\end{eqnarray}
The inequality in \eqref{eq:mink:real:2holder} is by applying H\"older
inequality twice.
Since \(p-p/q=1\) we get, dividing by \(\|a\|_p^{p/q}\) the following
\[
\|a\|_p \leq \|f\|_p + \|g\|_p.
\]
Since it is actually an \emph{equality} by the lemma condition,
the inequality in \eqref{eq:mink:real:2holder}
becomes equalities that the discussion quoted from the text,
shows that the must be real constants \(\alpha_i\) and \(\beta_i\)
for \(i=1,2\) such that
\begin{eqnarray*}
\alpha_1 f^p = \beta_1 a^q \\
\alpha_2 g^p = \beta_2 a^q.
\end{eqnarray*}
These constants are all non zero, and we have
\[( \alpha_1/\beta_1) f^p = \alpha_2/\beta_2) g^p\]
or setting
 \(a = (\alpha_1/\beta_1)^{1/p}\)
and
 \(b_i = (\alpha_2/\beta_2)^{1/p}\)
we get equivalently:
\( a f = bg\) and clearly \(a,b>0\).
\end{thmproof}



Now, the desired generalization.
\begin{llem} \label{llem:minkowski:eq}
Let \(f,g\) be measurable functions on $X$ and \(1<p<\infty\).
Then
\begin{equation} \label{eq:mink:equal}
 \left\| |f| + |g| \right\|_p = \|f\|_p + \|g\|_p
\end{equation}
iff there exist
real non-negative constants \(a,b\) not both $0$, such that \(af=bg \;\aded\).
\end{llem}
\begin{thmproof}
If either \(\|f\|_p=0\) or \(\|g\|_p=0\) the result is trivial.
This we may assume that both \(\|f|_p,\,\|g\|_p>0\).

Assume there exist such constants. \Wlogy, \(a>0\), put \(c=b/a\) and we have
\(f=cg\;\aded\).
Now both sides of \eqref{eq:mink:equal}
equal \((1+c)\|g\|_p\).

Conversely, assume \eqref{eq:mink:equal}.
By Minkowski's inequality
% and local lemma~\ref{eq:fgp:leq:afagp}
we have
\begin{equation} \label{eq:fgp:afagp:apgp}
\|f + g\|_p \leq \||f|+|g|\|_p \leq \|f\|_p + \|g\|_p.
\end{equation}
Our assumption, force the  inequalities in \eqref{eq:fgp:afagp:apgp}
to be equalities. But then
\begin{itemize}
\item
By local lemma~\ref{llem:mink:real:eq}
there exists real non negative constants $a$ and $b$ such that
\(a|f| = b|g|\;\aded\).
\item
By local lemma~\ref{eq:fgp:leq:afagp} \(\Arg(f(x)) = \Arg(g(x))\;\aded\)
\end{itemize}
Combing these conclusion,
show that there exists real non negative constants $a$ and $b$ such that
\(af = bg\;\aded\).
\end{thmproof}


%%%%%%%%%%%%%%%%%%%%%%%%%%%%%%%%%%%%%%%%%%%%%%%%%%%%%%%%%%%%%%%%%%%%%%%%
%%%%%%%%%%%%%%%%%%%%%%%%%%%%%%%%%%%%%%%%%%%%%%%%%%%%%%%%%%%%%%%%%%%%%%%%
\section{Integration Range}

The following result about convexity of the avareges set
will be used in Exercise~19.

\begin{llem} \label{llem:averages:convex}
Let   \((X,\frakM,m)\)
be a measurable space, where \(X\subset \R^n\) and $m$
is the restriction of Lebesgue's measure.
If \(f:X\to\C\) is a $m$-measurable function,
denote the average
\begin{equation*}
a(E) = a_f(E) \eqdef \frac{1}{m(E)} \int_E f\,d\mu
 \qquad \textnormal{where}\quad E\in\frakM \quad\textnormal{and}\quad m(E)>0.
\end{equation*}
Then the set of averages
\begin{equation*}
 % f \eqdef \left\{\, \frac{1}{\mu(E)} \int_E f\,d\mu\,:
 A_f \eqdef \left\{a_f(E): E\in\frakM \wedge \mu(E)>0 \right\}
\end{equation*}
is convex.
\end{llem}
\begin{thmproof}
Let \(\|\cdot\|\) some norm in \(\R^n\) inducing the Euclidean topology
(\(\|\cdot\|_2\) will do). For any generalized scalar \(r\in[0,\infty]\)
let \(B(r) = \{x\in X: \|x\|\leq r\}\).
For any measurable \(E\in\frakM\)
\iffalse
and any scalar \(\alpha\in[0,+\infty]\), let
\begin{equation*}
 E_\alpha \eqdef \{x\in E: \|x\|\leq \alpha\}.
\end{equation*}
\fi
by regularity of $m$, the real function
\(\nu(r) = m(E\cap B(r))\) is continuous. Similarly,
\begin{equation*}
\sigma(r) = \int_{E\cap B(r)} f\,dm
\end{equation*}
is continuous.

Pick arbitrary \(E_0,E_1\in\frakM\) such that \(m(E_0)\neq 0 \neq m(E_1)\).
We will show that the \(\C\)-segment
\begin{equation} \label{eq:llem:av:cvx:0}
[a(E_0),a(E_1)] \subset A_f,
\end{equation}
thus proving the convexity of \(A_f\).
We assume \(a(E_0) \neq a(E_1)\), otherwise the case is trivial.
\Wlogy, we may also assume that
\begin{equation} \label{eq:llem:av:cvx:1}
 a(E_0) = 0 \qquad a(E_1) = 1
\end{equation}
otherwise we may replace $f$ by
\(\bigl(f - a(E_0)\bigr)/\bigl(a(E_1) - a(E_0)\bigr)\).
Thus \eqref{eq:llem:av:cvx:0} is simplified to
\begin{equation} \label{eq:llem:av:cvx:2}
[0,1] \subset A_f.
\end{equation}
\iffalse
\(a(E_1)-a(E_0) = |a(E_1)-a(E_0)|e^{i\theta}\)
for some \(\theta\in[0,2\pi)\)
and we may substitue $f$ with \(e^{-i\theta}f\) and possibly
exchange \(E_0\) with \(E_1\).
\fi

Put \(D_j = E_j \setminus E_{1-j}\) for \(j=0,1\) and \(F=E_0\cap E_1\).
\iffalse
Now
\begin{eqnarray*}
a(E_1)-a(E_0)
 &=&
  \frac{1}{m(D_1) + m(F)} \left(\int_{D_1} f\,dm + \int_F f\,dm\right)
  -
  \frac{1}{m(D_0) + m(F)} \left(\int_{D_0} f\,dm + \int_F f\,dm\right)
\end{eqnarray*}
\fi
For \(j=0,1\) let
\begin{eqnarray*}
 D_j^- &\eqdef& \{x\in D_j: \Im(f(x)) < 0\} \\
 D_j^+ &\eqdef& \{x\in D_j: \Im(f(x)) \leq 0\} \\
\end{eqnarray*}
Note that any of the above 4 mutually disjoint sets may be empty.

We will now ``morph'' \(D_0^- \to D_1^+\) and \(D_0^+ \to D_1^-\).
We pick arbitrary strictly monotonic increasing and decreasing functions
(using \(1/0=+\infty\))
\begin{alignat*}{2}
u : [0,1]  &\nearrow [0,\infty]   &\qquad  v : [0,1] &\searrow [0,\infty] \\
u(\lambda) &= 1/(1-\lambda) - 1   &\qquad  v(\lambda)  &= 1/\lambda - 1
\end{alignat*}
to define the morphing functions:
\begin{alignat*}{2}
\Phi: [0,1] &\to P(D_0^-\cup D_1^+)
          &\quad \Psi: [0,1] &\to P(D_0^+\cup D_1^-) \\
\Phi(s) &=
  \left(D_0^- \cap B(v(s))\right)
  \cup
  \left(D_1^+ \cap B(u(s))\right)
  &\quad
\Psi(s) &=
  \left(D_0^+ \cap B(v(s))\right)
  \cup
  \left(D_1^- \cap B(u(s))\right).
\end{alignat*}
Note that
\begin{alignat*}{2}
\Phi(0) &= D_0^-  &\qquad  \Phi(1) = D_1^+ \\
\Psi(0) &= D_0^+  &\qquad  \Psi(1) = D_1^-
\end{alignat*}

Now define
\begin{eqnarray*}
\varphi : [0,1]\times[0,1] \to &P(E_0\cup E_1) \\
\varphi(s,t) = &\Phi(s) \disjunion F \disjunion \Psi(t).
\end{eqnarray*}
Note
\begin{equation} \label{eq:llem:av:cvx:E01}
\varphi(0,0) = E_0 \qquad \varphi(1,1) = E_1.
\end{equation}

Now we look at the ``morphed average'' continuous function
and its imaginary part.
\begin{gather*}
\Xi : [0,1]\times[0,1] \to \C \\
\Xi(s,t) = a(\varphi(s,t)) \\
\iota(s,t) = \Im\bigl(\Xi(s,t)\bigr).
\end{gather*}

Now \(\iota(s,t)\) is increasing in $s$ and decreasing in $t$
and
\(\iota(0,0) = \iota(1,1) = 0\)
by \eqref{eq:llem:av:cvx:E01} and \eqref{eq:llem:av:cvx:1}.
Thus for any \(s\in[0,1]\) we have
\begin{equation*}
\iota(s,1) \leq 0 \leq \iota(s,0)
\end{equation*}
so by continuity of \(\iota\), we can define
\begin{equation*}
\tau(s) \eqdef \sup\{t \in[0,1]: \iota(s,t) = 0\}
\end{equation*}
which is continuous, again by continuity of \(\iota\).
Now \(\Xi(s,\tau(s))\) is a continuous function of \([0,1]\) on itself
with fixed endpoints.
Thus for any \(\lambda\in[0,1]\)
there exists \(s\in[0,1]\) such that \(\Xi(s,\tau(s)) = \lambda\).
Hence \(a(\varphi(s,\tau(s)) = \lambda\) and \eqref{eq:llem:av:cvx:2} holds.
\end{thmproof}


%%%%%%%%%%%%%%%%%%%%%%%%%%%%%%%%%%%%%%%%%%%%%%%%%%%%%%%%%%%%%%%%%%%%%%%%
%%%%%%%%%%%%%%%%%%%%%%%%%%%%%%%%%%%%%%%%%%%%%%%%%%%%%%%%%%%%%%%%%%%%%%%%
\section{Exercises} % pages 73-78

%%%%%%%%%%%%%%%%%
\begin{enumerate}
%%%%%%%%%%%%%%%%%

%%%%%%%%%%%%%%
\begin{excopy}
Prove that the supremum of any collection of convex functions on \((a,b)\)
\index{convex function}
is convex on \((a,b)\) and that pointwise limit of sequence of convex
functions are convex.
What can you say about upper and lower limits of sequence of convex functions?
\end{excopy}

Let \(\{f_i(x) \}_{i\in I}\) be a collection of convex functions on \((a,b)\).
From now on, for any given \(x_0,x_1\in(a,b)\) and \(\lambda\in([0,1]\)
we define the \(\lambda\)-convex combination
as \(x_\lambda = (1-\lambda)x_0 + \lambda x_1\).


\paragraph{Supremum.}
Let \(s(x) = \sup_{i\in I}f_i(x)\).
By negation let \(x_0,x_1\in(a,b)\) and \(\lambda\in[0,1]\) be such that
\begin{equation*}
s(x_\lambda) >  (1-\lambda)s(x_0) + \lambda s(x_1)
\end{equation*}
Let
\begin{equation*}
 \epsilon =
   s(x_\lambda) -
   \bigl((1-\lambda)s(x_0) + \lambda s(x_1)\bigl) > 0.
\end{equation*}
By definition of \(s(x)\) there exists \(i\in I\) such that
\begin{equation*}
 0 < s(x_\lambda) - f_i(x_\lambda) < \epsilon\,.
\end{equation*}
Therefore,
\begin{equation*}
f_i(x_\lambda)
> s(x_\lambda)
> (1-\lambda)s(x_0) + \lambda s(x_1)
\geq  (1-\lambda)f_i(x_0) + \lambda f_i(x_1)
\end{equation*}
contradiction to \(f_i\) being convex, hence \(s(x)\) is convex.

\paragraph{Pointwise Limit.}
Let \(l(x) = \lim_{i\in \N}f_i(x)\).
By negation let \(x_0,x_1\in(a,b)\) and \(\lambda\in[0,1]\) be such that
\begin{equation*}
l(x_\lambda) >  (1-\lambda)s(x_0) + \lambda s(x_1)
\end{equation*}
Let
\begin{equation*}
 \epsilon =
   l(x_\lambda) -
   \bigl((1-\lambda)l(x_0) + \lambda l(x_1)\bigr) > 0.
\end{equation*}
There exists a sufficient large \(n\in\N\) (maximum of three)
such that for all \(i\geq n\)
\begin{equation*}
 \bigl|l(x_t) - f_i(x_t)\bigr| < \epsilon/2 \qquad
 \textrm{where}\; t\in\{0,\lambda,1\}
\end{equation*}
Now
\begin{eqnarray*}
\Delta(f_n,\lambda)
&=& f_n(x_\lambda) - \bigl((1-\lambda)f_i(x_0) + \lambda f_i(x_1)\bigr) \\
&\geq&
\left(l(x_\lambda) - \bigl((1-\lambda)l(x_0) + \lambda l(x_1)\bigr)\right)
 \; - \\
&&
  \bigl(
     |l(x_\lambda) - f_i(x_\lambda)| +
     (1-\lambda)|l(x_0) - f_i(x_0)|
     \lambda)|l(x_1) - f_i(x_1)|\bigr)\\
&>&
\left(l(x_\lambda) - \bigl((1-\lambda)l(x_0) + \lambda l(x_1)\bigr)\right)
 - (\epsilon/2 + (1-\lambda)\epsilon/2 + \lambda\epsilon/2) \\
&=&
\left(l(x_\lambda) - \bigl((1-\lambda)l(x_0) + \lambda l(x_1)\bigr)\right)
 - \epsilon \\
&=& 0
\end{eqnarray*}
Hence
\begin{equation*}
f_n(x_\lambda) > (1-\lambda)f_i(x_0) + \lambda f_i(x_1)
\end{equation*}
contradiction to \(f_i\) being convex, hence \(l(x)\) is convex.


\paragraph{Upper Limit.} Let \(u(x)\) be the upper limit of
\(\{f_i(x) \}_{i\in \N}\). By definition
\begin{equation*}
u(x) = \limsup_{n\to \infty} f_n(x)
     = \lim_{n\to \infty} \sup_{m\geq n} f_m(x)\,.
\end{equation*}
From the previous (supremum, and pointwise limit) results \(u(x)\) is convex.


\paragraph{Lower Limit.} A different behavior here.
The functions \(f_n(x) = (-1)^n x\) are convex.
But
\begin{equation*}
w(x)
= \liminf_{n\to \infty} f_n(x)
= \min_{n\to \infty}f_n(x)
= \min\bigl(f_0(x),f_1(x)\bigr) = -|x|
\end{equation*}
which is clearly \emph{not} convex.


%%%%%%%%%%%%%%
\begin{excopy}
If \(\varphi\) is a convex on \((a,b)\) and if \(\psi\) is convex and
nondecreasing on the range of \(\varphi\), prove that \(\psi\circ\varphi\)
is convex on \((a,b)\).
For \(\varphi>0\), show that the convexity of \(\log\varphi\) implies
the convexity of \(\varphi\), but not vice versa.
\end{excopy}

Let \(\varphi\) be convex on \((a,b)\) and \(\psi\)  convex and
nondecreasing on the range of \(\varphi\).
For any \(x,y\in(a,b)\) and \(\lambda\in[0,1]\) we have:
\begin{eqnarray}
\psi\circ\varphi(\lambda x + (1-\lambda)y)
 &=& \psi\bigl(\varphi(\lambda x + (1-\lambda)y)\bigr) \notag \\
 &\leq& \psi\bigl(\lambda\varphi(x) + (1-\lambda)\varphi(y)\bigr)
        \label{eq:phiconv:psiinc} \\
 &\leq& \lambda \psi\bigl(\varphi(x)\bigr) +
        (1-\lambda)\psi\bigl(\varphi(y)\bigr)
        \label{eq:psiconv} \\
 &=& \lambda \psi\circ\varphi(x) +  (1-\lambda)\psi\circ\varphi(y). \notag
\end{eqnarray}
The inequality (\ref{eq:phiconv:psiinc}) holds because \(\varphi\) is
convex and \(\psi\) is increasing.
The inequality (\ref{eq:psiconv}) holds because \(\psi\) is convex.

If \(\varphi>0\), and \(\log\varphi\) is convex, then by using \(\psi\)
as the exponential (\(\exp(x)\) in the result just shown, we get
that \(\exp(\log(\varphi)) = \varphi\) is convex.

The converse does not hold. Take the identity \(f(x)=x\), which is
clearly convex, but \(\log = \log\circ f\) is not.


%%%%%%%%%%%%%%
\begin{excopy}
Assume that \(\varphi\) is a continuous real function on \((a,b)\)
such that
\begin{equation*}
 \varphi\left(\frac{x+y}{2}\right)
 \leq \frac{1}{2}\varphi(x) + \frac{1}{2}\varphi(y)
\end{equation*}
for all $x$ and $y$ \(\in (a,b)\). Prove that \(\varphi\) is convex.
(The conclude does \emph{not} follow if continuity is omitted from the
hypotheses.)
\end{excopy}


We first show that for every integers \(n\geq 0\) and
\(0\leq m \leq 2^n > 0\)
and any \(x,y\in (a,b)\)
\begin{equation} \label{eq:ex3.3:m:n}
 \varphi\bigl((2^n-m)x/2^n + my/2^n\bigr) \leq
  \left((2^n-m)\varphi(x) + m\varphi(x)\right)\bigm/2^n
\end{equation}

By denoting
\begin{equation}
C(m,N,x,y) = (N-m)x/N + my/N.
\end{equation}
we actually need to show in (\ref{eq:ex3.3:m:n})
\begin{equation}  \label{eq:ex3.3:m:n:C}
  \varphi(C(m,2^n,x,y)) \leq C(m,2^n,\varphi(x), \varphi(y)).
\end{equation}



By induction on $n$. For \(n=0\), we must have \(m=0\) or \(m=1\)
and the inequality becomes trivial true equality.
Assume that (\ref{eq:ex3.3:m:n}) holds for \(n\leq k\geq 0\), we will show
that it holds also for \(n=k\).
Let $m$ be an integer such that \(0\leq m \leq 2^{k+1}\).
We use the trivial equality of common \(2^x\) denominator.
There are two cases:

 \textbf{Case (i)}:
   If \(2\mid m\), then \(m/2\) is an integer, hence, by induction hypotheses
 \begin{eqnarray*}
  C(m,2^{k+1},x,y)
  &=&     \varphi\bigl(((2^{k+1}-m)x + my)/2^{k+1}\bigr) \\
  &=&     \varphi\bigl(((2^k-m/2)x + my)/2^k\bigr) \\
  &\leq&  \left((2^k-m/2)\varphi(x) + (m/2)\varphi(x)\right)\bigm/2^k \\
  &=&     \left((2^{k+1}-m)\varphi(x) + m\varphi(x)\right)\bigm/2^{k+1} \\
  &=&     C(m,2^{k+1},\varphi(x)) + C(m,2^{k+1},\varphi(y)).
 \end{eqnarray*}

\textbf{Case (ii)}:
Otherwise,
\(2\nmid m\), then \(0<m<2^n\) and \((m\pm 1)/2\) are two integers.
  we use a trivial mid-point equality:
  \begin{equation*}
  C(m,2^{k+1},x,y) = \left( C(m-1,2^{k+1},x,y) + C(m-1,2^{k+1},x,y) \right) / 2
  \end{equation*}
  By initial assumption
  \begin{equation} \label{eq:ex3.3:m:mid}
  \varphi\left(C(m,2^{k+1},x,y)\right)
   \leq
    \left(\varphi\left(C(m-1,2^{k+1},x,y)\right) +
          \varphi\left(C(m+1,2^{k+1},x,y)\right) \right) \bigm / 2.
  \end{equation}

  Now
  \begin{equation*}
   C(m\pm 1,2^{k+1},x,y) =  C((m\pm 1)/2,2^k,x,y).
  \end{equation*}

  Thus (supporting arguments follows),
  \begin{eqnarray}
     \varphi\bigl(C(m,2^{k+1},x,y)\bigr)
    &\leq& \label{eq:ex3.3:m:midC}
          \Bigl(
          \varphi\bigl(C(m-1,2^{k+1},x,y)\bigr) + \\
    && \phantom{\Bigl(}
          \varphi\bigl(C(m+1,2^{k+1},x,y)\bigr) \Bigr) \Bigm/ 2 \notag \\
    &=&
          \Bigl(
          \varphi\bigl(C((m-1)/2,2^k,x,y)\bigr) + \notag \\
    && \phantom{\Bigl(}
          \varphi\bigl(C((m+1)/2,2^k,x,y)\bigr) \Bigr) \Bigm/ 2
          \notag \\
    &\leq & \label{eq:ex3.3:induct:step}
     \Bigl(
             C\bigl((m-1)/2,2^k,\varphi(x),\varphi(y)\bigr) + \\
    &&  \phantom{\Bigl(}
             C\bigl((m+1)/2,2^k,\varphi(x),\varphi(y)\bigr) \Bigr) \Bigm/ 2
             \notag \\
    &\leq &  C\bigl((m-1)/2,2^{k+1},\varphi(x),\varphi(y)\bigr) + \notag \\
    &&       C\bigl((m+1)/2,2^{k+1},\varphi(x),\varphi(y)\bigr) \notag \\
    &= &     \label{eq:ex3.3:endinuct}
             C\bigl(m,2^{k+1},\varphi(x),\varphi(y)\bigr).
  \end{eqnarray}

The (\ref{eq:ex3.3:m:midC}) inequality follows from (\ref{eq:ex3.3:m:mid}).
The (\ref{eq:ex3.3:induct:step}) inequality follows by induction.
The last equality (\ref{eq:ex3.3:endinuct}) follows from the following
two equalities (\ref{eq:ex3.3:m}) and (\ref{eq:ex3.3:m2})
\begin{equation} \label{eq:ex3.3:m}
 \bigl((m-1)/2\bigr)/ 2^{k+1} +
 \bigl((m+1)/2\bigr)/ 2^{k+1} =
  m / 2^{k+1}
\end{equation}

\begin{equation}\label{eq:ex3.3:m2}
 \bigl(2^{k+1}-(m-1)/2\bigr)/ 2^{k+1} +
 \bigl(2^{k+1}-(m+1)/2\bigr)/ 2^{k+1} =
      (2^{k+1}-m) / 2^{k+1}
\end{equation}

Now that (\ref{eq:ex3.3:m:n}) is established, let \(\lambda\in[0,1]\)
and by negation assume that
\begin{equation*}
 \epsilon = \varphi(\lambda x + (1-\lambda)y) -
 \lambda \varphi(x) + (1-\lambda)\varphi(y) > 0.
\end{equation*}
Since \(\varphi\) is continuous on \([a,b]\) it is uniformly continuous
on \([a,b]\). So there exists \(\delta>0\) such that
\(|\varphi(t_1) - \varphi(t_2)| < \epsilon/4\)
whenever \(|t_1-t_2|<\delta\).
Since the set \(\{m/2^n: m,n\in\N\wedge 0\leq m\leq 2^n\}\)
is dense in \([0,1]\), we can find \(n>0\) and $m$ such that
(the first two inequalities are actually equivalent):
\begin{eqnarray*}
|(2^n - m)/2^n - \lambda)| &<& \delta \\
|m/2^n - (1-\lambda)| &<& \delta \\
|\bigl((2^n - m)/2^n - \lambda)\bigr)\varphi(x)| &<& \epsilon/4 \\
|\bigl(m/2^n - (1-\lambda)\bigr)\varphi(y)| &<& \epsilon/4
\end{eqnarray*}

We now get the following contradiction
\begin{eqnarray*}
\varphi(\lambda x + (1-\lambda)y)
&\leq& \varphi(C(m,n,x,y)) + \epsilon/4 + \epsilon/4 \\
&\leq& C(m,n,\varphi(x),\varphi(y)) + \epsilon/2 \\
&\leq& (\lambda\varphi(x) + \epsilon/4) + ((1-\lambda)\varphi(y) + \epsilon/4)
       + \epsilon/2 \\
&=&    \lambda\varphi(x) (1-\lambda)\varphi(y) + \epsilon.
\end{eqnarray*}

Therefore \(\varphi\) is convex.


%%%%%%%%%%%%%% 4
\begin{excopy}
Suppose $f$ is a complex measurable function on $X$, \(\mu\) is a
positive
measure on $X$, and
\begin{equation*}
 \varphi(p) = \int_X |f|^p\,d\mu = \|f\|_p^p \qquad (0<p<\infty).
\end{equation*}
Let \(E = \{p: \varphi(p)<\infty\}\). Assume \(\|f\|_\infty > 0\).
\begin{itemize}
 \itemch{a}
   If \(r<p<s\), \(r\in E\), and \(s\in E\), prove that \(p\in E\).
 \itemch{b}
   Prove that  \(\log\varphi\) is convex in the interior of $E$ and
   that  \(\varphi\) is continuous on $E$.
 \itemch{c}
   By \ich{a}, $E$ is connected. Is $E$ necessarily open? Closed? Can
   $E$ consist of a single point?
   Can $E$ be any connected subset of \((a,\infty)\)
 \itemch{d}
   If \(r<p<s\), prove that \(\|f\|_p \leq \max( \|f\|_r, \|f\|_s)\).
   Show that this implies the inclusion
   \(L^r(\mu) \cap  L^s(\mu) \subset L^p(\mu)\).
 \itemch{e}
   Assume that \(\|f\|_r < \infty\) for some \(r<\infty\) and prove
   that
   \begin{equation*}
     \|f\|_p \to \|f\|_\infty \qquad \textrm{as}\; p\to\infty.
   \end{equation*}
\end{itemize}
\end{excopy}

 Let \(X_0 = \{x\in X: |f(x)|<1\}\)
 and \(X_1 = \{x\in X: |f(x)|\geq 1\}\).

\begin{itemize}
\itemch{a}
 Compute
 \begin{eqnarray*}
 \varphi(p)
 &=& \int_X |f|^p\,d\mu
 = \int_{X_0} |f|^p\,d\mu + \int_{X_1} |f|^p\,d\mu
 \leq \int_{X_0} |f|^r\,d\mu + \int_{X_1} |f|^s\,d\mu \\
 &\leq& \int_{X} |f|^r\,d\mu + \int_{X} |f|^s\,d\mu
 \leq \varphi(r) + \varphi(s) < \infty.
 \end{eqnarray*}


\itemch{b}
  Let \(r < s < t\) be in the interior of $E$.
  Put \(p_1 = (t-s)/(t-r)\)
  and \(p_2 = (s-r)/(t-r)\)
  to form  the convex combination \(s = p_1 r + p_2 t\).

  Now from Lemma~\ref{llem:hlp:188} --- with \(k=2\),
  \(f_1 = |f|^r\) and
  \(f_2 = |f|^t\) we have
  \begin{eqnarray*}
    \varphi(s)
    &=&    \int_X |f|^s\,d\mu \\
    &=&    \int_X f_1^{p_1} f_2^{p_2}\,d\mu \\
    &\leq& \left( \int_X f_1\,d\mu \right)^{p_1}
           \left( \int_X f_2\,d\mu \right)^{p_2} \\
    &=&    \varphi(r)^{p_1} \varphi(t)^{p_2}.
  % \int_X |f|^s\,d\mu
  \end{eqnarray*}
  Now since \(\log\) is monotonic increasing, we get
  \begin{equation*}
  \log\varphi(s) \leq  \log(\varphi(r)^{p_1} \varphi(t)^{p_2} =
           p_1\log\varphi(r) + p_2\log\varphi(t).
  \end{equation*}
  Now that we have shown convexity in $E$,
  by Theorem~3.2 (\cite{RudinRCA80}), \(log(\varphi(x)\) is continuous in
  the interior of $E$. To show continuity on all of $E$,
  let \([a,c] = \overline{E}\) and put \(b=(a+b)/2\).

  If \(a\in E\), then \(b\in E\) and put
  \begin{equation*}
  g_a(x) = \max(|f(x)|^a, |f(x)|^b)
  \end{equation*}
  and clearly \(\int_X g_a(x)\,d\mu < \infty\).
  For \(s\in[a,b]\),  \(g_a\) is a dominated function for \(\varphi(s)\)
  and by Lebesgue's dominated convergence theorem, \(\varphi(s)\)
  is continuous at \(s=a\).

  If \(c\in E\), then \(b\in E\) and put
  \begin{equation*}
  g_c(x) = \max\left(|f(x)|^b, |f(x)|^c\right)
  \end{equation*}
  and clearly \(\int_X g_c(x)\,d\mu < \infty\).
  For \(s\in[b,c]\),  \(g_c\) is a dominated function for \(\varphi(s)\)
  and by Lebesgue's dominated convergence theorem, \(\varphi(s)\)
  is continuous at \(s=c\).

  Thus \(\varphi\) is continuous on $E$.

\itemch{c}
  The answer is ``Yes'' to the last question. That is, $E$
  can be any connected subset of \((0,\infty)\).

  First, the trivial cases:
  \begin{itemize}
  \item \(E=\emptyset\) with \(f:\R\to\R\) defined as \(f(x)=1\).
  \item \(E=\R^+\) with \(f:X\to\R\) defined as \(f(x)=0\)
        for any measurable space $X$.
  \end{itemize}

  Let's generalize the definition of \(\phi\) and $E$.
  For any function \(g:X\to\C\), let
  \begin{eqnarray*}
   \varphi_g(p) &=& \int_X |g|^p\,d\mu = \|g\|_p^p \qquad (0<p<\infty) \\
   E(g) &=& \{p: \varphi_g(p)<\infty\}.
  \end{eqnarray*}

  Say we have functions
 \(\{f_i\}_{i\in\N}\)
  \begin{equation*}
   f_i: X_i \to \R^+ \qquad(i\in\N)
  \end{equation*}
  where \(\R^+ = \{x\in\R: x\geq0\}\).
  % We'll use the abbreviations: \(\varphi_i(p) = \varphi_{f_i}(p)\).

  It is easy to see that \(E(f_1 + f_2) = E(f_1) \cap E(f_2)\).
  Similarly
  \begin{equation*}
  E\left(\sum_{i\in\N} f_i\right) = \bigcap_{i\in\N} E(f_i)\,.
  \end{equation*}
  We can simplify the fucntion summation by assuming that
  their domains are mutually disjoint. Without loss of generality,
  we can thing of disjoint union of domains.

  We will show that for any \(a\in\R^+\),
  we can build functions $g$, such that \(E(g)\)
  is \emph{half open line}, that is
  \(E(g) = \{x\in\R^+: 0<x<a\}\) or
  \(E(g) = \{x\in\R^+: a<x<a\}\).

  It is easy to see that any connected subset of \(\R^+\) can
  be represented as an intersection of countably many half open lines.


  Let \(X=\{x\in\R: x\geq 1\}\) with the normal Lebesgue's measure and let
  \(g_a(x) = x^a\) for some \(a\in\R\). Let's see how \(E(g_a)\) looks.
  If \(a=0\) then \(E(g_0) = \emptyset\).
  Otherwise \(a\neq 0\) and then for
  \begin{equation*}
  E(g_a) = \{p\in \R^+: pa < 1\}
  \end{equation*}
  we have two cases:
  \begin{itemize}
   \item[($-$)] If \(a<0\) then \(E(g_a) = \{p\in\R^+: p > 1/a\}\).
   \item[($+$)] If \(a>0\) then \(E(g_a) = \{p\in\R^+: p < 1/a\}\).
  \end{itemize}
  Thus we can find $a$ for any desired half open line we need.

\itemch{d}
 If either \(\|f\|_r = \infty\) or \(\|f\|_s = \infty\)
 then the inequality trivially follows.  Thus we may assume
 that both \(r,s\in E\).
 By negation, let's assume
 \begin{equation} \label{eq:3:fpfrfs}
 \|f\|_p > \max( \|f\|_r, \|f\|_s).
 \end{equation}
 We want to avoid dealing with an exceptional case, where
 $r$ or $s$ are on the boundary of $E$.
 From \ich{b} since \(\varphi\) is continuous on $E$,
 we can find \(r'\), \(s'\) such that \(r<r'<p<s'<s\)
 and still
 \(\|f\|_p > \max( \|f\|_{r'}, \|f\|_{s'})\).
 Since (By \ich{b} again) \(\log\varphi\) is convex in the interior of $E$,
 and so
 \begin{equation*}
 \log\varphi(p) \leq \bigl(\log\varphi(r') + \log\varphi(s')\bigr)/2
                \leq \max(\log\varphi(r'), \log\varphi(s')\bigr)
 \end{equation*}
 But \(\log\) is monotonically increasing, and so
 \(\varphi(p) \leq \max(\varphi(r'), \varphi(s')\bigr)\)
 contradiction to (\ref{eq:3:fpfrfs}).

 Now \(L^r(\mu) \cap  L^s(\mu) \subset L^p(\mu)\) follows by definitions.

\itemch{e}
 Assume $E$ is bounded by \(M<\infty\). Then
 \begin{equation*}
 \lim_{p\to\infty}\varphi(p) =
 \lim_{M<p\to\infty}\varphi(p) = \infty.
 \end{equation*}

 Otherwise, \(E=\R^+\) (we follow \cite{Hardy:1952:I} \textbf{192}).
 Two cases:
 \begin{itemize}
  \item[\(\bullet\)]
   Suppose \(M = \|f\|_\infty < \infty\).
   Then \(\|f\|_r \leq M\) and for any \(\epsilon>0\)
   we put \(H=\{ x\in X: f(x)> M-\epsilon\}\) with \(\xi=m(H)>0\).
   Then for any \(r\in\R^+\setminus\{0\}\) we have
   \begin{equation*}
   \xi(M-\epsilon)^r \leq \int_H |f|^rd\mu
                     \leq \int_X |f|^rd\mu \leq \|f\|_r^r\,.
   \end{equation*}
   Thus % \(M - \epsilon) \leq \|f\|_r
   \begin{equation*}
   \underline{\lim}_{r\to\infty} \|f\|_r \xi^{-r} \geq M - \epsilon
   \end{equation*}
   and consequently  \(\underline{\lim}_{r\to\infty} \|f\|_r  \geq M\).

  \item[\(\bullet\)]
   Suppose \(\|f\|_\infty = \infty\). Then for any \(M<0\)
   we put \(H=\{ x\in X: f(x)> M\}\) and \(\xi = m(H) > 0\).
   Similarly as the bounded case, we see that
   \(\underline{\lim}_{r\to\infty}  \|f\|_r \geq M\).
   Since $M$ can be arbitrary large,
   \(\underline{\lim}_{r\to\infty}  \|f\|_r = \infty\).
 \end{itemize}
 Thus, in both cases \(\lim_{r\to\infty} \|f\|_r = \|f\|_\infty\).


\end{itemize}

%%%%%%%%%%%%%% 5
\begin{excopy}
Assume, in addition to the hypotheses of Exercise~4, that
\begin{equation*}
 \mu(X) = 1.
\end{equation*}
\begin{itemize}
\itemch{a}
 Prove that \(\|f\|_r \leq \|f\|_s\) if \(0<r<s\leq \infty\).
\itemch{b}
 Under what conditions does it happen that \(0<r<s\leq \infty\)
 and \(\|f\|_r = \|f\|_s < \infty\) ?
\itemch{c}
 Prove that
  \(L^r(\mu) \supset L^s(\mu)\) if \(0<r<s\).
 Under what conditions do these two spaces contain the same functions.
\itemch{d}
 Assume that \(\|f\|_r < \infty\) for some \(r>0\), and prove that
 \begin{equation*}
   \lim_{p\to 0} \|f\|_p = \exp\left\{\int_X \log|f|\,d\mu\right\}
 \end{equation*}
 if \(\exp\{-\infty\}\) is defined to be $0$.
\end{itemize}
\end{excopy}

\begin{itemize}
%
\itemch{a}
If \(\|f\|_s=\infty\) the inequality trivially holds. Otherwise
We use lemma~\ref{lem:holder:eq} to compute, for simplification
assume \(f\geq 0\).
\begin{equation*}
\|f\|_r^r
 = \int_X f^r\,d\mu
 = \int_X (f^s)^{r/s}\cdot \mathbf{1}^{1-r/s}\,d\mu
 \leq \left(\int_X f^s\right)^{r/s} \cdot \left(\int_X \mathbf{1}\right)^{1-r/s}
 = \left(\int_X f^s\right)^{r/s}
\end{equation*}
Taking power of \(1/r\) gives the desired inequality \(\|f\|_r \leq \|f\|_s\).

\itemch{b}
 Following lemma~\ref{lem:holder:eq}, strict inequality happens
 when $f$ is not effectively constant.

\itemch{c}
The inclusion is immediate from \ich{a}. The spaces contain the same
functions (Spaces are ``equal'' except for their norm) iff $X$ has only finite
number of subsets with non zero measure.

If this condition is met. Let \(X=\disjunion_{j=1}^n X_j\) be the partition.
Meaning: \(\mu(X_j)>0\) and if \(A\subset X_j\) is measurable, then
\(\mu(A)=0\) or \(\mu(X\setminus A)=0\).
Then a measurable function \(f_{|X_j}=c_j\)
is constant \aded\ on each \(X_j\).
% assumes finite number of values except for set of measure zero.
It is easy to see now that for such function \(\|f\|_p < \infty\)
if \(c_j<\infty\) for \(1\leq i \leq n\).

Conversely, we can find inifinite enumerable partition
\(X=\disjunion_{j=1}^\infty X_j\), with \(\mu(X_j) = m_j > 0\).
We will construct a function \(f:X\to [0,\infty)\) such that
\(f\in L^r(X)\setminus L^s(X)\).
Since \(\mu(X)=1\), there are infinite subsets \(X_j\) with small measure
as desired.
Before picking values for $f$,
let's assume (or by picking a sub-family) that \(X_j\) are such that
\begin{equation*} \label{eq:LrLs:mj}
m_j \eqdef \mu(X_j) < (2^{-js} j)^{1/(s-r)}.
\end{equation*}
Hence,
\begin{equation*}
\left(\frac{1}{jm_j}\right)^r < \left(\frac{2^{-j}}{m_i}\right)^s.
\end{equation*}
Equivelantly,
\begin{equation}
l_j \eqdef (j m_j)^{-1/s} < \left(2^{-j}/m_i\right)^{1/r} \eqdef u_j.
\end{equation}
Setting ``mid-constants'' \(c_j = (l_j + u_j)/2\) we cann now define
\begin{equation*}
f(x) = \left\{ \begin{array}{ll}
               c_j \qquad & x\in X_j \\
               0          & \textrm{Otherwise}
               \end{array}\right..
\end{equation*}
To show that $f$ is satisfies our goal:
\begin{itemize}

\item[(i)] \(f\in L^r(X)\) \quad:
\begin{equation*}
\|f\|_r^r
 = \sum_{j=1}^\infty m_j c_j^r
 <  \sum_{j=1}^\infty m_j u_j^r
 \leq \sum_{j=1}^\infty m_j \left(\left(2^{-j}/m_i\right)^{1/r}\right)^r
 = \sum_{j=1}^\infty 2^{-j} = 1 < \infty
\end{equation*}

\item[(ii)] \(f\notin L^s(X)\) \quad:
\begin{equation*}
\|f\|_s^s
 = \sum_{j=1}^\infty m_j c_j^s
 >  \sum_{j=1}^\infty m_j l_j^s
 = \sum_{j=1}^\infty m_j \left((j m_j)^{-1/s} \right)^s
 = \sum_{j=1}^\infty 1/j = \infty
\end{equation*}

\end{itemize}

\itemch{d}
 Using \(\exp(x)=(\exp(x^p)^{1/p}\), we have:
 \begin{equation*}
  \exp\left(\int\log(|f|)\,d\mu\right)
   = \left(\exp\bigl(\int\log(|f|^p)\,d\mu\bigr)\right)^{1/p}
   \leq \left(\int|f|^p\,d\mu\right)^{1/p} = \|f\|_p.
 \end{equation*}
 The inequality is derived from comparing geometrical and arithmetical means,
 see (7) in the proof of \index{Jensen} Jensen's theorem
 (\cite{RudinRCA80}  page~64).

 Note that for \(x>0\)
 \begin{equation*}
  \varphi(x) = (t^x-1)/x
  = \log t \frac{e^{x\log t} - e^0}{x\log t - 0}
 \end{equation*}
 is an increasing function by convexity of \(\exp\).
 and so \(\varphi(x)\to \log t\) (decreasing)  as \(x\to 0\)
 by l'Hospital's rule. We also note the simple inequality
 \begin{equation} \label{eq:logt:leq:tm1}
 \log t \leq t-1.
 \end{equation}
 We now compute
 \begin{eqnarray}
  \exp\left(\int \log|f|\,d\mu\right)
  &\leq& \lim_{p\to 0} \|f\|_p          \notag \\
  &=& \overline{\lim_{p\to 0}} \|f\|_p         \notag \\
  &=& \exp\left( \overline{\lim_{p\to 0}}
                 \frac{1}{p}\log\bigl( \int |f|^p\,d\mu\bigr) \right) \notag\\
  &\leq& \exp \left(\overline{\lim_{p\to 0}}
                        \int(|f|^p-1)/p\,d\mu\right) \label{eq:5d:whyleq} \\
  &=&    \exp\left( \int \log|f|\,d\mu\right)        \label{eq:5d:whyeq}
 \end{eqnarray}
 Where (\ref{eq:5d:whyleq}) is by (\ref{eq:logt:leq:tm1})
 and   (\ref{eq:5d:whyeq})  is by Lebesgue's dominated convergence theorem.
 Hence
 \begin{equation*}
  \exp\left\{\int \log|f|\,d\mu\right\} =  \lim_{p\to 0} \|f\|_p
 \end{equation*}
\end{itemize}


%%%%%%%%%%%%%% 6
\begin{excopy}
Let $m$ be Lebesgue measure on \([0,1]\), and define \(\|f\|_p\)
with respect to $m$.
Find all functions \(\Phi\) on \([0,\infty)\) such that the relation
\begin{equation*}
 \Phi(\lim_{p\to 0} \|f\|_p) = \int_0^1(\Phi\circ f)\,dm
\end{equation*}
holds for every bounded, measurable, positive $f$. Show first that
\begin{equation*}
c\Phi(x)+(1-c)\Phi(1) = \Phi(x^c) \qquad (x>0, 0\leq c\leq 1).
\end{equation*}
Compare with Exercise~5\ich{d}.
\end{excopy}

Looking at Exercise~5\ich{d}, we clearly see that \(\log\) can be
one of such \(\Phi\) functions.
Define
\begin{equation*} % \label{eq:fc:chi}
 f(x) = f_c(x) x\cdot\chhi_{(0,c)} + 1\cdot \chhi_{(c,1)}
\end{equation*}
We compute the left side
\begin{equation*}
\Phi\left( \lim_{p\to 0} \|f\|_p\right)
= \Phi\left( \exp \bigl(\int_0^1 \log f\,dm\bigr)\right)
= \Phi\bigl( \exp (c\cdot\log x + (1-c)\log 1)\bigr)
= \Phi(x^c)
\end{equation*}
and the right side
\begin{equation*}
 \int_0^1 \Phi\circ f \,dm
 = \int_0^c \Phi(x)\,dm + \int_c^1 \Phi(x)\,dm
 = c\Phi(x) + (1-c)\Phi(1)
\end{equation*}
So the required equality is shown.

In the given equality
\begin{equation} \label{eq:Phi:lim}
 \Phi(\lim_{p\to 0} \|f\|_p) = \int_0^1(\Phi\circ f)\,dm
\end{equation}
We put
\begin{equation} \label{eq:ex:3.6:psi}
\psi(x) = \Phi(x) - \Phi(1),
\end{equation}
 getting
\begin{equation*}
 \psi(x^c)+\Phi(1)
  = c\bigl(\psi(x)+\Phi(1)\bigr) + (1-c)\bigl(\psi(x)+\Phi(1)\bigr)
\end{equation*}
Hence, when \(0\leq c\leq 1\).
\begin{equation} \label{eq:psi:xc}
\psi(x^c) = c\psi(x).
\end{equation}
If \(c>1\) then \(0\leq 1/c\leq 1\) and similarly
\begin{equation*}
\psi\bigl((x^c)^{1/c}\bigr) = (1/c)\psi(x^c)
\end{equation*}
and (\ref{eq:psi:xc}) holds for all \(c\geq 0\).

Hence for all \(x>0\),
\begin{equation*}
 \psi(x) = \psi\left(e^{\log x}\right) = \psi(e)\log x
\end{equation*}
and using (\ref{eq:ex:3.6:psi}) we see that any such \(\Phi\) must satisfy
\begin{equation*}
\Phi(x) = \psi(e)\log x + \Phi(1).
\end{equation*}
Clearly the converse, is true. That is for any \(a,b\in\R\)
the function \(\Phi(x) = a\log x + b\) satisfies the required condition.


%%%%%%%%%%%%%% 7
\begin{excopy}
For some measures, the relation \(r<s\) implies
\(L^r(\mu) \subset L^s(\mu)\);
for others, the inclusion is reversed;
and there are some for which \(L^r(\mu)\) does not contain \(L^s(\mu)\)
if \(r\neq s\). Give examples of these situations, and find conditions
on \(\mu\) under which these situations will occur.
\end{excopy}

For trivial atomic measure space \(\mu\) with only \(\{\emptyset,X\}\)
in its \salgebra, \(L^p(X,\mu)\) consists of measurable functions $f$
such that \(|f|<\infty \;\aded\).
In particular
\begin{equation*}
L^r(\mu)=L^s(\mu)\supset L^r(\mu)\subset L^s(\mu).
\end{equation*}
In Exercise~5(c) we saw a case (\(\mu(X)<\infty\) where for \(r<s\) we showed
\(L^s(\mu) \subsetneq L^r(\mu)\).
% Put \(t=(r+s)/2\).

Put \(t = (1/r+1/s)/2 > 0\).
Note that \(st>1\) and \(rt<1\) or equivalently \(-st < -1 < -rt\).

Now in \(\R^+\) with Lebesgue measure, the function \(f(x)=x^{-t}\) satisfies
\begin{equation*}
\|f\|_s^s = \int_0^\infty x^{-st},dm < \infty
\end{equation*}
while
\begin{equation*}
\|f\|_r^r = \int_0^\infty x^{-rt},dm > \infty.
\end{equation*}
Thus \(L^r(\mu) \nsupseteq L^s(\mu)\).


%%%%%%%%%%%%%% 8
\begin{excopy}
 If $g$ is a positive function on \((0,1)\) such that
\(g(x)\to\infty\) as \(x\to 0\),
then there is a convex function $h$ on \((0,1)\) such that \(h\leq g\)
and
\(h(x)\to\infty\) as \(x\to 0\).
True or false?
Is the problem changed of \((0,1)\) is replaced by \((0,\infty)\)
and \(x\to 0\) is replaced by \(x\to\infty\).
\end{excopy}

First part is true.
Let \((g(x)\) be as requried. We will now define a decreasing sequence
\seqan\ converging to $0$ by induction.
Let \(a_0=1\) and
let \(a_1>0\) be such that \(g(x)>1\) for all \(x<a_1\).
Now assume \(a_i\) are defined for \(i<k\).
Let \(a_k > 0\) be such that
\begin{itemize}
 \item  \(a_k\leq a_{k-1}/2\)
 \item  \(g(x) > k\) for all \(x < a_k\).
\end{itemize}
We now define $h$ on the segments \(I_i = [a_{i+1},a_{i}]\) by induction.
For \(x\in I_0\), we set \(h(x)=0\).
Assume $h$ was defined for the segments \(I_i\) for \(i<k\).
In particular, \(h(a_k)\) is defined.
For \(x\in I_k\) let \(h(x)\) get the values of the lines passing
through \(L=(0,k)\) and \(R=(a_k,h(a_k))\).
It is easy to see that $h$
\begin{itemize}
 \item is less than $g$.
 \item is continuous.
 \item converges to \(\infty\) as \(x\to 0\)
 \item is decreasing and its absolute slopes values increase,
       and thus it is convex.
\end{itemize}

Second part is false. Simply look at \(g(x)=\log(x)\). Any convex $h$
satisfying the requirements will have some \(0<a<b<\infty\)
with \(f(a)<f(b)\). But then the line passing through
\(A=(a,h(a))\) and
\(B=(b,h(b))\)
would intersect \(\log(x)\) at some point $M$, but then
\(h(M)<\log(x) = h(a) + (h(b)-h(a))(x-a)/(b-a)\)
contradicting $h$'s convexity.

%%%%%%%%%%%%%% 9
\begin{excopy}
Suppose $f$ is Lebesgue measurable on \((0,1)\), and not essentially bounded.
By Exercise~4\ich{e}, \(\|f\|_p\to\infty\) as \(p\to\infty\).
Can \(\|f\|_p\) tend to \(\infty\) arbitrarily slowly?
More precisely, is it true that for any function \(\Phi\) on \((0,\infty)\)
such that \(\Phi(p)\to\infty\) as \(p\to\infty\) one can find an $f$ such that
\(\|f\|_p\to\infty\) as \(p\to\infty\), but
\(\|f\|_p \leq \Phi(p)\) for all sufficiently large $p$?
\end{excopy}

The answer is \emph{yes}. Assume \(\Phi\) is as described above.
We will construct a ``step'' function $f$ as desired.

We can assume that \(\Phi\) is monotonically increasing.
Otherwise, we can simply look at
\begin{equation*}
 \Psi(x) = \inf_{w\geq x}\Phi(w)
\end{equation*}
instead.

For each \(k\in\Z\),
let \(p_k>0\) be such that \(\Phi(p)\geq k\) for all \(p>p_k\),
and define
\begin{equation*}
 \alpha_k
 \eqdef \inf_{p_1\leq p } \bigl(\Phi(p)/k\bigr)^p
 = \inf_{p_1\leq p \leq p_k } \bigl(\Phi(p)/k\bigr)^p.
\end{equation*}
Clearly,
\begin{equation*}
0 < 1/k^{p_k} \leq \alpha_k \leq 1.
\end{equation*}
Now, define ``interval lengths'' \(m_k = 2^{-k}\alpha_k\).
We can easily form disjoint open sub-intervals \(I_k\) of \([0,1]\)
such that \(m(I_k) = m_k\). Finally, we define
\begin{equation*}
f(x) = \sum_{k=1}^\infty \chhi_{I_k}(x) \cdot k =
    \left\{\begin{array}{ll}
             k & \qquad x\in I_k\\
             0 & \qquad \textrm{otherwise}
           \end{array}\right.
\end{equation*}
To show the required inequality, for \(p\geq p_1\)
\begin{equation*}
\|f\|_p^p
  = \sum_{k=1}^\infty m_k k^p
  \leq \sum_{k=1}^\infty 2^{-k} \bigl(\Phi(p)/k\bigr)^p k^p
  = \sum_{k=1}^\infty 2^{-k} \bigl(\Phi(p)\bigr)^p
  = \bigl(\Phi(p)\bigr)^p
\end{equation*}
Hence \(\|f\|_p \leq \Phi(p)\) for \(p\geq p_1\).


%%%%%%%%%%%%%% 10
\begin{excopy}
Suppose \(f_n\in L^p(\mu)\), for \(n=1,2,3,\ldots\), and
\(\|f_n-f\|_p\to 0\) and
\(f_n\to g\) \aded, as \(n\to \infty\).
What relation exists between $f$ and $g$?
\end{excopy}

By Theorem~3.12 \cite{RudinRCA80}, there is a subsequence of \(\{f_n\}\)
that converges pointwise to $f$ \aded. Obviousy this subsequence
converges to $g$ \aded. These two convergences may be missed
in two sets of measure $0$, and so is their union.
Thus \(f=g \aded\).

%%%%%%%%%%%%%% 11
\begin{excopy}
Suppose \(\mu(\Omega)=1\), and suppose $f$ and $g$ are positive measurable
functions on \(\Omega\) such that \(fg\geq 1\). Prove that
\begin{equation*}
 \int_\Omega f\,d\mu \cdot \int_\Omega g\,d\mu \geq 1.
\end{equation*}
\end{excopy}

Clearly \((fg)^{1/2} \geq 1\), hence by H\"older's inequality,
\begin{equation*}
1 \leq \int_\Omega f^{1/2} g^{1/2}\,d\mu
  \leq \left(\int_\Omega f\,d\mu\right)^{1/2}
       \left(\int_\Omega g\,d\mu\right)^{1/2}
\end{equation*}
Taking squares gives the desired inequality.


%%%%%%%%%%%%%% 12
\begin{excopy}
Suppose \(\mu(\Omega)=1\) and \(h:\Omega\to[0,\infty]\) is measurable. If
\begin{equation*}
 A = \int_\Omega h\,d\mu,
\end{equation*}
prove that
\begin{equation*}
 \sqrt{1+A^2} \leq \int_\Omega \sqrt{1+h^2}\,d\mu \leq 1 + A.
\end{equation*}
If \(\mu\) is Lebesgue measure on \([0,1]\) and if $h$ is continuous, \(h=f'\),
the above inequalities have simple geometric interpretation.
From this, conjecture (for general \(\Omega\)) under what conditions on $h$
equality can hold in either of the above inequalities, and prove your
conjecture.
\end{excopy}

We define a function \(\varphi\) and compute:
\begin{eqnarray}
 \varphi(x)  &=& \sqrt{x^2+1}         \label{eq:phi:sqrt:x2} \\
 \varphi'(x) &=& x/\sqrt{x^2+1}       \notag \\
 \varphi''(x) &=& 1/(x^2+1)^{3/2} > 0 \notag
\end{eqnarray}
Thus \(\varphi(x)\) is convex. Applying Jensen Theorem~3.3 to $h$ gives
\begin{equation*}
\sqrt{1+A^2} = \varphi\left(\int_\Omega h\,d\mu\right)
 \leq \int_\Omega (\varphi \circ h)\,d\mu
 = \int_\Omega \sqrt{1+h^2}\,d\mu.
\end{equation*}
the first desired inequality. The second inequality is immediate
by noting that \(\sqrt{1+x^2}\leq 1+x\)  for \(x\geq 0\), then integrating
over \(\Omega\).

For the \(h'=f\) case, here is the geometric interpretation.
The function $f$ is increasing.
The length of the graph curve of \((x,f(x))\) is
\(\int_0^1 \sqrt{1+h^2}\,d\mu\) which is greater or equal
the distance \(\sqrt{1+A^2}=d((0,f(0)),(1,f(1)))\) of its endpoints,
but is less or equal the Manhattan distnace \(1+A\).

The conjecture
\begin{itemize}
 \item[(i)]  \(\sqrt{1+A^2} = \int_\Omega \sqrt{1+h^2}\,d\mu\)
             \,iff\, $h$ is constant \aded.
 \item[(ii)]  \(\int_\Omega \sqrt{1+h^2}\,d\mu = 1 + A\)
             \,iff\, \(h=0\;\aded\).
\end{itemize}

\textbf{Proof.}
\newline
\textbf{(i).} If \(h(x)=c \;\textrm{a.e.}\)
then clearly \(A=c\) and the equality follows. Conversely,
if $h$ is not constant \aded, then
since \(varphi\) (defined in (\ref{eq:phi:sqrt:x2})) is strictly convex,
we can apply local lemma~\ref{lem:jensen:strict}
to get strict inequality
% \begin{equation*}
\(\sqrt{1+A^2} < \int_\Omega \sqrt{1+h^2}\,d\mu\).
% \end{equation*}
\newline
\textbf{(ii).} If \(h(x)=0 \;\textrm{a.e.}\) then clearly \(A=0\)
and the equality follows. Conversely, assume
\(\int_\Omega \sqrt{1+h^2}\,d\mu = 1 + A\).
Hence
\(\int_\Omega 1 + h - \sqrt{1+h^2}\,d\mu = 0\) and since the integrand
is non negative, \(1 + h - \sqrt{1+h^2} = 0\;\aded\), thus \(h=0\;\aded\).

%%%%%%%%%%%%%% 13
\begin{excopy}
Under what conditions on $f$ and $g$ does equality hold in the inclusions
of Theorem~3.8 and~3.9? You may have to treat the cases \(p=1\) and \(p=\infty\)
separately.
\end{excopy}

Call the domain of the functions $X$.
\begin{itemize}
 \item[\textbf{[3.8]}]
  Given conjugate \(1\leq \,p\,,q\,\leq \infty\),
  if \(f\in L^p(\mu)\)
  and \(g\in L^q(\mu)\), then
  \begin{equation} \label{eq:fg1:eq:fpgq}
      \|fg\|_1 = \|f\|_p\cdot\|g\|_q
  \end{equation}
  iff one of the following occurs:
\begin{itemize}
 \itemdim \emph{Normal}: \(1<p,q<\infty\) and
        the functions \(f^p\) and \(g^q\) are effectively proportional
        (see \cite{Hardy:1952:I} \textbf{189.}).
 \itemdim \emph{Supremum}:
            \(p=1\) and \(q=\infty\) and
            \(|g(x)| = \|g\|_\infty\) constant \aded\ on \(\supp f\)
 \itemdim \emph{Supremum}:
            \(q=1\) and \(p=\infty\) and
            \(|f(x)| = \|f\|_\infty\) constant \aded\ on \(\supp g\)
\end{itemize}

  \emph{Normal}: Assume \(1<p,q<\infty\),
  since \(1/p+1/q=1\) we have
  \(q/p = q - 1\).
  If \(f^p\) and \(g^q\) are effectively proportional, by symmetry, we can assume
  there is some scalar $a$ such that \(f^p = a g^q \aded\). Then
  \(|g|^{q/p+1} = |g|^q\). Now compute:
 \begin{eqnarray*}
  \|fg\|_1
    &=& \int |fg|\,d\mu
    = \int (a|g|^q)^{1/p}|g|\,d\mu
    = a^{1/p} \int |g|^{q/p+1}\,d\mu
    = a^{1/p} \int |g|^q\,d\mu \\
    &=& a^{1/p} \left(\int |g|^q\,d\mu\right)^{1/p + 1/q}
    = \left(\int a|g|^q\,d\mu\right)^{1/p}
      \left(\int  |g|^q\,d\mu\right)^{1/q} \\
    &=& \|f\|_p \|g\|_q
 \end{eqnarray*}

 Conversely, assume (\ref{eq:fg1:eq:fpgq}) holds. Then
 \begin{equation*}
 \int \left(|f|^p\right)^{1/p}g|
      \left(|g|^q\right)^{1/q}g|
        \,d\mu
 = \int |fg|\,d\mu
 = \|f\|_p \|g\|_q
 = \int \left(|f|^p\,d\mu\right)^{1/p}
   \int \left(|g|^q\,d\mu\right)^{1/q}.
 \end{equation*}
 By local lemma~\ref{lem:holder}
 with \(k=2\), we conclude that the functions \(|f|^p\) and \(|g|^q\)
 are effectively proportional.

 % \smallskip
 \medskip
 \emph{Supremum}: The two cases are symmetrical.
 Assume \(p=1\) and \(q=\infty\).
 If $g$ is constant \aded\ on \(\supp f\) then
 \begin{equation*}
 \int |fg|\,d\mu = \int_{\esssup f} |fg|\,d\mu
 = \left(\int_{\esssup f} |f|\,d\mu\right)\|g\|_\infty
 = \|f\|_1 \|g\|_\infty
 \end{equation*}
 Conversely, assume (\ref{eq:fg1:eq:fpgq}) holds. By negation
 assume that \(|g(x)|\) is not a constant \(\|g\|_\infty\) on \(E=\esssup f\).
 Then $g$ ``misses'' the supremum on $f$'s support. More precisely,
 there exists $a$ such that \(0\leq a < \|g\|_\infty\)
 such that the set
 \begin{equation*}
  B \eqdef \{x\in \esssup f: |g(x)| < a\} \subset \esssup \subset X
 \end{equation*}
 has \(\mu(B)> 0\). Now
 \begin{eqnarray}
 \int_X |fg|\,d\mu
 &=&    \notag
         \int_{X\setminus B} |fg|\,d\mu + \int_B |fg|\,d\mu \\
 &\leq& \notag
          \|g\|_\infty \int_{X\setminus B} |f|\,d\mu
        + a \int_B |f|\,d\mu \\
 &<&    \label{eq:gB:a}
          \|g\|_\infty \int_{X\setminus B} |f|\,d\mu
        + \|g\|_\infty \int_B |f|\,d\mu \\
 &=&    \notag
        \|f\|_1 \cdot \|g\|_\infty
 \end{eqnarray}
 Where the strict inequality in (\ref{eq:gB:a})
 by local lemma~\ref{lem:fgz:igz} gives a contradiction.

 \item[\textbf{[3.9]}]
  Given \(1\leq p \leq \infty\),
  if \(f,g\in L^p(\mu)\)
  then
  \begin{equation} \label{eq:fg:p:fpgp}
      \|f + g\|_p = \|f\|_p + \|g\|_p
  \end{equation}
  iff one of the following occurs:
   \begin{itemize}
    \itemdim \emph{Trivial}:  Both expressions are zero.
    \itemdim \emph{Manhattan}: \(p=1\) and
             \begin{equation} \label{eq:absfg:absfabsg}
              |f(x)+g(x)| = |f(x)| + g(x)\;\aded
             \end{equation}
    \itemdim \emph{Normal}: \(1< p <\infty\) and the functions
          $f$ and $g$ are effectively proportional
          (see \cite{Hardy:1952:I} \textbf{198.}).
    \itemdim \emph{Supremum}: \(p=\infty\) and for each \(\epsilon>0\)
          the set
          \begin{equation*}
           S_\epsilon =
               \bigl\{x\in X:
              \bigl|\left(\|f\|_\infty + \|g\|_\infty\right) - (f(x)+g(x))\bigr|
               < \epsilon\bigr\}
          \end{equation*}
          has positive measure.
   \end{itemize}
    \emph{Trivial}: Both expression are zero. For the rest of the cases
       (especially \emph{Normal}) we may assume \(\|f+g\|_p > 0\).

    \medskip
    \emph{Manhattan}:
    Assume \(p=1\).
    Clearly  \(|f(x)+g(x)| \leq |f(x)| + g(x)\;\aded\).
    Now \(\ref{eq:fg:p:fpgp}\) is equivalent to
    \begin{equation*}
    \int  |f| + |g| - |f+g|\,d\mu = 0.
    \end{equation*}
    this is equivalent by local lemma~\ref{lem:fgz:igz}
    to~(\ref{eq:absfg:absfabsg}).

    \medskip
    \emph{Normal}: Assume  \(1\leq p<\infty\).
    If $f$ and $g$ are effectively proportional, say \(f=ag\)
    for some scalar \(a>0\), then
    \begin{eqnarray*}
    \|f + g\|_p
    &=& \left(\int (a+1)^p|g|^p\,d\mu\right)^{1/p}
              = (a+1)\left(\int |g|^p\,d\mu\right)^{1/p} \\
       &=&   \left(\int |ag|^p\,d\mu\right)^{1/p}
            + \left(\int |g|^p\,d\mu\right)^{1/p} \\
    &=& \|f\|_p + \|g\|_p
    \end{eqnarray*}
    Conversely, assume (\ref{eq:fg:p:fpgp}) holds.
    Let \(S \eqdef \|f + g\|_p\),
    and $q$ be the conjugate exponent of $p$, that is \(1/p+1/q = 1\)
    or \(p=(p-1)q\).
    We compute
    \begin{eqnarray}
    S^p
    &=& \notag
     \|f + g\|_p^p  \\
    &=& \notag
     \int |f+g|^p\,d\mu \\
    &\leq& \label{eq:ex13:f+g}
           \int |f|\cdot |f+g|^{p-1}\,d\mu
         + \int |g|\cdot |f+g|^{p-1}\,d\mu \\
    &\leq& \label{eq:fg:p-1}
         \left(\int |f|^p\,d\mu\right)^{1/p}
            \left(\int |f+g|^{(p-1)q}\,d\mu\right)^{1/q}
         + \\
    &&   \notag
          \left(\int |g|^p\,d\mu\right)^{1/p}
            \left(\int |f+g|^{(p-1)q}\,d\mu\right)^{1/q} \\
    &=&  \notag
         \left(\|f\|_p + \|g\|_p\right) \|f+g\|_p^{p/q} \\
    &=&  \notag
         \left(\|f\|_p + \|g\|_p\right) \|f+g\|_p^{p-1}
    \end{eqnarray}
    The inequality (\ref{eq:fg:p-1}) is by applying
    Theorem~3.8 (\cite{RudinRCA80}) twice, and it is an equality
    iff both \(f^p\) and \(g^p\) are effectively proportional
    to \((f+g)^{p-1}\). Since \(\|f+g\|_p > 0\), by assumption,
    the above inequalities (divided by \(\|f+g\|_p^{p-1}\))
    are indeed equalities.
    Hence \(|f|^p\) and \(|g|^p\) are effectively proportional
    and so are \(|f|\) and \(|g|\).
    But since we equality also in (\ref{eq:ex13:f+g})
    we have \(|f+g|=|f|+|g| \;\aded\),
    and so $f$ and $g$ are effectively proportional.

    \medskip
    \emph{Supremum}: Assume \(p=\infty\).
    For any \(\mu\)-measurable function \(\varphi\)
    we have \(\|\varphi\|_\infty = \esssup \varphi\).
    Let \(M=\|f + g\|_\infty\) and
        \(N = \|f\|_\infty + \|g\|_\infty\).
    Clearly \(|f(x)+g(x)|\leq M \leq N\) for all \(x\in X\).
    If (\ref{eq:fg:p:fpgp}) holds, then for any \(\epsilon>0\)
    the set \(\{x\in X: |f(x)+g(x)| > M-\epsilon\}\) has non zero measure.
    But in this case, this set is exactly \(S_\epsilon\).
    Conversely, if \(\mu(S_\epsilon)>0\) for and \(\epsilon > 0\)
    then \(\|f+g|_\infty \geq M-\epsilon\) and so we get
    (\ref{eq:fg:p:fpgp}).


\end{itemize}

%%%%%%%%%%%%%% 14
\begin{excopy}
Suppose \(1<p<\infty\), \(f\in L^p=L^p((0,\infty))\),
relative to Lebesgue measure, and
\begin{equation*}
 F(x) = \frac{1}{x} \int_0^x f(t)\,dt \qquad (0<x<\infty)
\end{equation*}
\begin{itemize}
\itemch{a}
 Prove Hardy's inequality
 \begin{equation*}
  \|F\|_p \leq \frac{p}{p-1} \|f\|_p
 \end{equation*}
 which shows that the mapping \(f\to F\) carries \(L^p\) into \(L^p\).
\itemch{b}
 Prove that equality holds only if \(f=0\) a.e.
\itemch{c}
 Prove that the constant \(p/(p-1)\) cannot be replaced by a smaller one.
\itemch{d}
 If \(f>0\) and \(f\in L^1\), prove that \(F\notin L^1\).\newline
\end{itemize}
 \qquad \emph{Suggestions}: \ich{a} Assume first that \(f\geq 0\) and
 \(f\in C_c((0,\infty))\). Integration by parts gives
 \begin{equation*}
   \int_0^\infty F^p(x)\,dx = -p \int_0^\infty F^{p-1}(x)xF'(x)\,dx.
 \end{equation*}
 Note that \(xF' = f - F\), and apply
 \index{Holder@H\"older}
 H\"older's inequality to \(\int F^{p-1}f\).
 Then derive the general case.
 \ich{c}~Take \(f(x)=x^{-1/p}\) on \([1,A]\), \(f(x)=0\) elsewhere,
 for large $A$.
\end{excopy}


% \begin{itemize}

%%%%%%%%%%%%%%%%%%%%%%%%%%%%%%%%
\textbf{\ich{a}.}
Following the suggestion, we first assume \(0\leq f\in C_c((0,\infty))\).
Then
\begin{equation*}
F(x) \eqdef \int_0^x f(t)\,dt
\end{equation*}
and \(F(x)\) are differentiable.
Integration by parts (see theorem~6.22 \cite{RudinPMA85})\\

\begin{quote}
\footnotesize
 Reminder: Assume \(\Phi'=\phi\) and \(\Psi'=\psi\)
 \begin{equation*}
  \int_a^b \Phi(x)\psi(x)\,dx
  = \Phi(b)\Psi(b) - \Phi(a)\Psi(a) - \int_a^b \phi(x)\Psi(x)\,dx.
 \end{equation*}
\end{quote}

gives:
\begin{equation} \label{eq:int0x:Fpx}
 \int_0^x F^p(x)\cdot 1\,dx
 = \bigl(\lim_{x\to\infty} F^p(x)\cdot x\bigr) - F^p(0)\cdot 0
   - \int_0^x pF^{p-1}(x)F'(x)x\,dx .
\end{equation}
Since \(f(x)=0\) when \(x>b\) for some sufficiently large \(b<\infty\),
we can use the boundary  \(M\int_0^\infty |f(x)|\,dx\) to estimate
\begin{equation*}
 F^p(x)\cdot x
 \leq (M/x)^p x = Mx^{-(p-1)}.
\end{equation*}
Thus \(\lim_{x\to\infty} F^p(x)x = 0\) and also \(F(0) = 0\).

By differentiation (see theorem~6.20 \cite{RudinPMA85})
\begin{equation*}
F'(x)
 = \frac{d}{dx}\left(x^{-1}\int_0^x f(t)\,dt\right)
 = -x^{-2}\int_0^x f(t)\,dt + x^{-1}f(x) = (1/x)\bigl(F(x) + f(x)\bigr).
\end{equation*}
Thus
\begin{equation}  \label{eq:ex14:xF}
xF'(x) = f(x) - F(x).
\end{equation}
Collecting these resulted equalities,
we can substitute in~(\ref{eq:int0x:Fpx})
\begin{equation*}
 \int_0^\infty F^p(x)\,dx
  =  - p \int_0^\infty F^{p-1}(x)\bigl(f(x) - F(x)\bigr)\,dx
\end{equation*}
and simplify to
\begin{equation*}
(p-1) \int_0^\infty F^p(x)\,dx =  p \int_0^\infty F^{p-1}(x)f(x)\,dx
\end{equation*}
By H\"older inequality with $p$ and its conjugate exponent of \(q=p/(p-1)\).
\begin{eqnarray}
 \left(\|F\|_p\right)^p
    = \left|\int F^{p-1}f\right| \notag
 &\leq& \|F^{p-1}\|_q \|f\|_p    \label{eq:ex14:a} \\
 &=& \left(\int_0^\infty F^{q(p-1)}(x)\,dx\right)^{(p-1)/p} \|f\|_p \notag \\
 &=& \left(\|F\|_p\right)^{p-1} \|f\|_p \notag
\end{eqnarray}
Combining with the recent simplified equality
\begin{equation} \label{eq:ex:3.14a}
 \|F\|_p \leq (p/p-1) \|f\|_p.
\end{equation}

Theorem~3.14 (\cite{RudinRCA80}) shows that the functions with compact support
are dense in \(L^p((0,\infty\), so there (\ref{eq:ex:3.14a})
holds as well.

%%%%%%%%%%%%%%%%%%%%%%%%%%%%%%%%
\textbf{\ich{b}.}
If \(f=0\) equality clearly holds. Conversely, assume \(f=0\; \aded\)
does not hold, and by negation assume equality holds.

If \(f\geq0\;\aded\) does not hold, take
\(g(x)=|f(x)|\) and \(G(x)=(1/x)\int_0^x g(t)\,dt\).
Then
\begin{equation*}
\|F\|_p\leq \|G\|_p \leq \frac{p}{p-1}\|g\|_p = \frac{p}{p-1}\|f\|_p
\end{equation*}
and so the above inequalities reduce to equalities.

Thus we may assume \(f\geq0\). The inequality (\ref{eq:ex14:a})
must also be an equality, and by previous exercise, extending theorem~3.8
\cite{RudinRCA80} (see \ref{eq:fg1:eq:fpgq}) above),
the functions \((F^{p-1}(x))^q = F^p(x)\) and \(f^p(x)\) must be
effectively proportional and so are \(F(x)\) and \(f(x)\).
Hence there is a scalar \(a\neq 0\) such that
\(aF(x) = f(x)\;\aded\) for \(x\in(0,\infty)\).
Since \(F(x)\) is continuous, we can consider \(f(x)\) to be continuous,
otherwise we can look at \(aF(x)\). Using (\ref{eq:ex14:xF}),
we will solve the following first order ordinary partial equation
\begin{equation*}
\frac{d}{dx} \log(F(x)) = \frac{F'(x)}{F(x)} = \frac{a-1}{x}
\end{equation*}
Integrating gives
\begin{equation*}
 \log(F(x)) = \int_1^x F'(t)/F(t)\;dt + c_0 = (a-1)\int_1^x 1/t\;dt + c_0
 = b\log(x) + c
\end{equation*}
for some constants \(c_0\), \(c_1\).
Hence
\begin{equation*}
F(x) = e^{b\log(x)+c} = e^c x^b
\end{equation*}
But then \(f(x) = ae^c x^b\) is a contradiction to \(f \in L^p((0,\infty))\).

%%%%%%%%%%%%%%%%%%%%%%%%%%%%%%%%
\textbf{\ich{c}.}
Given \(A>1\), let
\begin{equation*}
f(x) = \left\{ \begin{array}{ll}
                 x^{-1/p}   & \quad  x\in[1,A] \\
                 0         & \quad  \textrm{otherwise}.
                \end{array}\right.
\end{equation*}

Now compute
\begin{equation*}
\|f\|_p = \left(\int_1^A (x^{-1/p})^p\,dx\right){1/p}
        = \left(\log(A)\right)^{1/p}
\end{equation*}

For \(1\leq x \leq A\)
\begin{equation*}
F(x)
 = \left(\int_1^x t^{-1/p}\;dt\right)/x
 = \frac{p}{(p-1)x}\bigl(x^{(p-1)/p} - 1\bigr)
\end{equation*}
\begin{equation*}
F^p(x)
 = \left(\frac{p}{p-1}\right)^p \left(\frac{x^{(p-1)/p} - 1}{x}\right)^p
\end{equation*}
We need to estimate the last factor.
For each (large) $x$ We look for \(\epsilon_x\) such that
\begin{equation} \label{eq:ex14:eps}
 \left(\frac{x^{(p-1)/p} - 1}{x}\right)^p > \frac{1-\epsilon_x}{x}
\end{equation}
Isolating
\begin{equation*}
\epsilon_x > 1 - x\left(\frac{x^{(p-1)/p} - 1}{x}\right)^p
\end{equation*}
Using l'Hospital rule;
\begin{equation*}
\lim_{x\to\infty} x\left(\frac{x^{(p-1)/p} - 1}{x}\right)^p
% = \lim_{x\to\infty} \left(\frac{x^{1/p}x^{(p-1)/p} - x^{1/p}}{x}\right)^p
= \lim_{x\to\infty} \left(\frac{x - x^{1/p}}{x}\right)^p
= \lim_{x\to\infty} 1 - x^{1/p-1}/p = 0.
\end{equation*}
So for each $x$ % \(1\leq x \leq A\)
we can find \(\epsilon_x>0\) such that (\ref{eq:ex14:eps}) holds
and \(\lim_{x\to\infty}\epsilon_x = 0\) (with \(A\to\infty\) as well).

Next, we need estimate for integration.
\begin{equation*}
\|F\|_p^p
= \int_0^\infty F^p(x)\,dx
\geq \int_1^A F^p(x)\,dx
\geq \left(\frac{p}{p-1}\right)^p \int_1^A \frac{1-\epsilon_x}{x}\,dx.
\end{equation*}
We will improve the last estimate.

Given \(\eta > 0\)
there is an $h$ such
that \(\epsilon_x < \eta\) for any \(x \geq h\).
By local lemma~\ref{lem:fg:bnless}
(replacing integral domain minimum), there exists \(A<\infty\) such that
\begin{equation*}
   \int_1^A \frac{1-\epsilon_x}{x}\,dx \geq + (1 - \eta) \int_1^A 1/x\,dx
\end{equation*}
Thus, by choosing sufficiently large $A$, we get
\begin{equation*}
\|F\|_p^p
\geq \left(\frac{p}{p-1}\right)^p (1 - \eta) \int_h^A f^p(x)\,dx
\end{equation*}
which shows that \(p/(p-1)\) is the minimal constant,
satisfying (\ref{eq:ex:3.14a}).

%%%%%%%%%%%%%%%%%%%%%%%%%%%%%%%%
\textbf{\ich{d}.}
By local lemma~\ref{lem:fgz:igz}, there exist some \(a<\infty\)
such that
\begin{equation*}
0 < A = \int_0^a f(x)\,dx < \infty
\end{equation*}
But now
\begin{equation*}
\|F\|_1
= \int_0^\infty F(x)\,dx
= \int_0^\infty \left((1/x)\int_0^x f(t)\,dt\right)\,dx
\geq \int_a^\infty A/x\,dx = \infty
\end{equation*}


% \end{itemize}


%%%%%%%%%%%%%% 15
\begin{excopy}
Suppose \(\{a_n\}\) is a sequence of positive numbers. Prove that
\begin{equation*}
 \sum_{N=1}^\infty \left(\frac{1}{N} \sum_{n=1}^N a_n\right)^p
 \leq
 \left(\frac{p}{p-1}\right)^p \sum_{n=1}^\infty a_n^p
\end{equation*}
if \(1<p<\infty\). \emph{Hint}: If \(a_n\geq a_{n+1}\), the result can be made
 to follow from Exercise~14.
 This special case implies the general case.
\end{excopy}

We will first show that the supremum of
\begin{equation*}
 \sup_\pi \sum_{N=1}^\infty \left(\frac{1}{N} \sum_{n=1}^N \pi(a_n)\right)^p
\end{equation*}
where \(\pi\) runs over all permutations of \N, is achieved with
\(\bigl(\pi(a_n)\bigr)_{n\in\N}\) monotonically decreasing.
Assume \((a_n)_{n\in\N}\) is monotonically decreasing, and
\((b_n)_{n\in\N}\) any of its permutations.
Then clearly, for any $N$
\begin{equation*}
 \sum_{n=1}^N a_n > \sum_{n=1}^N b_n
\end{equation*}
and so
\begin{equation*}
 \sum_{N=1}^\infty \left(\frac{1}{N} \sum_{n=1}^N a_n\right)^p
 \geq
 \sum_{N=1}^\infty \left(\frac{1}{N} \sum_{n=1}^N b_n\right)^p.
\end{equation*}

Thus, it is sufficient to prove the desired inequality for
decreasing sequence \((a_n)_{n\in\N}\).
Let's define \(f:(0,\infty)\to (0,\infty)\), by
\(f(x) = a_{\lceil x \rceil}\).
From the trivial equality
\begin{equation*}
 \sum_{n=1}^N a_n = \int_0^N f(x)
\end{equation*}
we can use the previous exercise~14, with the definition of \(F(x)\)
\begin{eqnarray*}
 \sum_{N=1}^\infty \left(\frac{1}{N} \sum_{n=1}^N a_n\right)^p
 &=& \sum_{N=1}^\infty \left(\frac{1}{N} \int_0^N f(x) \right)^p \\
 &=& \sum_{N=1}^\infty F^p(N) \\
 &=& \int_0^\infty F^p(x)\,dx = \|F\|_p^p \\
 &\leq& \left(\frac{p}{p-1} \|f\|_p\right)^p
        = \left(\frac{p}{p-1}\right)^p \int_0^\infty f^p(x)\,dx  \\
 &=&  \left(\frac{p}{p-1}\right)^p \sum_{N=1}^\infty a_n^p\;.
\end{eqnarray*}



%%%%%%%%%%%%%% 16
\begin{excopy}
Prove
\index{Egoroff}
Egoroff's theorem: If \(\mu(X)<\infty\), if \(\{f_n\}\) is a sequence of complex
measurable functions which converges pointwise at every point of $X$,
and if \(\epsilon > 0\), there is a measurable set \(E\subset X\), with
\(\mu(X\setminus E)<\epsilon\), such that \(\{f_n\}\)
converges uniformly on $E$.

(The conclusion is that by redefining the \(f_n\) on a set of arbitrarily small
measure we can convert a pointwise convergent sequence to a uniformly
convergent one: note the similarity with
\index{Lusin}
Lusin theorem.)

\qquad\emph{Hint}: Put
\begin{equation*}
 S(n,k) = \cap_{i,j>n} \left\{x: |f_i(x) - f_j(x)| < \frac{1}{k}\right\},
\end{equation*}
show that \(\mu(S(n,k))\to \mu(X)\) as \(n\to\infty\), for each $k$,
and hence that there is a suitable increasing sequence \(\{n_k\}\) such that
\(E = \cap S(n_k,k)\) has the desired property.

Show that the theorem does not extend to \(\sigma\)-finite spaces.

Show that the theorem does extend (with the same proof)
to the situation in which the sequences \(\{f_n\}\) are replaced by families
\(\{f_t\}\), where $t$ ranges over the positive reals, and the assumption
is that \(f_t(x) \to f(x)\), as \(t\to\infty\), for every \(x\in X\).
\end{excopy}

Let \(f(x) \eqdef \lim_{n\to\infty} f_n(x)\) for every $x$.
Following the hint. Clearly \(S(n,k)\subset S(n+1,k)\) for all $n$, $k$.
Now \((f_n(x))_{n\in\N}\) is a Cauchy sequence for every $x$.
Equivelantly,
for every $x$ and every $k$, we have \(x\in\cap_n S(n,k)\) for some $n$.
Thus
\begin{equation*}
\lim_{n\to\infty}\mu(S(n,k)) = \mu(X) < \infty.
\end{equation*}
Given \(\epsilon > 0\), for every \(k>0\), pick \(n_k\) such that
\begin{equation*}
\mu(S(n,k)) > \mu(X) 2^{-k}\epsilon
\end{equation*}
Now
\begin{equation*}
\mu(E)
 =  \mu\bigl(\cap S(n_k,k)\bigr)
 \geq \mu(X) - \sum_{k=1}^\infty 2^{-k}\epsilon
 = \mu(X) - \epsilon.
\end{equation*}

Here is \(\sigma\)-finite space example,
where the above does not hold.
Let \(f_n:\R\to\R\) defined by
\begin{equation*}
 f_n(x) = \chhi_{[-n,+n]}(x).
\end{equation*}
Clearly, \(f_n(x) \to 1 \) for every \(x\in \R\) but uniformly
only on bounded subsets.

Assume we have a family \((f_t)_{t\in\R^+}\) of functions
such that a limit \(f(x) = \lim_{t\to\infty} f_t(x)\) exists
for every \(x\in X\).
The result does  extend with similar proof.
We can define a sequence \((g_n)_{n\in\N}\)
\begin{equation} \label{eq:ex16:gnf}
 g_n(x) = f(x) + \sup_{t\geq n} |f_t(x) - f(x)|.
\end{equation}
Clearly \(\lim_{n\to\infty} g(x) = f(x)\) pointwise.
Fron what we have shown, for every \(\epsilon > 0\)
we can find a subset \(E\subset X\) such that \(\mu(X\setminus E) < \epsilon\)
and \((g_n)_{n\in\N}\) converges uniformly on $E$.
By the definition (\ref{eq:ex16:gnf})
\((f_n)_{n\in\N}\) converges uniformly on $E$ as well.


%%%%%%%%%%%%%% 17
\begin{excopy}
\begin{itemize}
\itemch{a}
 If \(0<p<\infty\), put \(\gamma_p = \max(1,2^{p-1})\), and show that
 \begin{equation*}
 |\alpha - \beta|^p \leq \gamma_p(|\alpha|^p + |\beta|^p)
 \end{equation*}
 for arbitrary numbers \(\alpha\) and \(\beta\).
\itemch{b}
   Suppose \(\mu\) is a positive measure on $X$, \(0<p<\infty\),
   \(f\in L^p(\mu)\), \(f_n\in L^p(\mu)\), \(f_n(x)\to f(x)\;\textrm{a.e.}\),
   and \(\|f_n\|_p \to \|f\|_p\) as \(n\to\infty\).
   Show that then \(\lim\|f-f_n\|_p = 0\),
   by completing the two proofs that are sketched below.
  \begin{itemize}
   \item[(i)]
    By Egoroff's theorem, \(X=A\cup B\) in such a way that
    \(\int_A|f|^p<\epsilon\), \(\mu(B)<\infty\), and \(f_n\to f\) uniformly
    on $B$, Fatou's lemma applied to \(\int_B|f_n|^p\), leads to
    \begin{equation*}
      \lim\sup \int_A |f_n|^p\,d\mu \leq \epsilon
    \end{equation*}
   \item[(ii)]
    Put \(h_n = \gamma_p(|f|^p + |f_n|^p) - |f-f_n|^p\),
    and use Fatou's lemma as in the proof of Theorem~1.34.
  \end{itemize}
\itemch{c}
 Show that the conclusion of \ich{b} is false if the hypothesis
 \(\|f_n\|_p \to \|f\|_p\) is omitted, even if \(\mu(X)<\infty\).
\end{itemize}
\end{excopy}


\begin{itemize}
\itemch{a}
 Let \(a=|\alpha|\) and \(b=|\beta|\).
 Since \(|\alpha - \beta| \leq a + b\), it is sufficient to show
 \begin{equation} \label{eq:3:17a}
   (a+b)^p \leq \gamma_p(a^p + b^p).
 \end{equation}
 If \(a=0\) (or \(b=0\), by symmetry)
 then (\ref{eq:3:17a}) is trivial. So we can assume \(a\neq 0 \neq b\).
 Put \(x = b/a > 0\).
 We have two cases:

 \paragraph{Case 1.} Assume \(0<p<1\) and \(\gamma_p = 1\).

 We first show that for \(x\geq 0\)
 \begin{equation*}
 (x + 1)^p \leq (x^p + 1)
 \end{equation*}
 or equivalently
 \begin{equation*}
 f(x) = (x^p + 1) - (x + 1)^p \geq 0
 \end{equation*}
 Clearly \(f(0) = 0\).
 Since \(p-1 < 0\) the function \(x\to x^{p-1}\) is decreasing (for \(x\neq 0\)).
 Hence $f$ is increasing for \(x>0\) since
 \begin{equation*}
 f'(x) = p\bigl(x^{p-1} - (x+1)^{p-1}\bigr) > 0
 \end{equation*}

 Now the inequality is derived:
 \begin{equation*}
 (a+b)^p =  (a+xa)^p = (x+1)^p a^p
 \leq a^p(x^p + 1)
 =    a^p + (xa)^p
 =    \gamma_p(a^p + b^p)
 \end{equation*}

 \paragraph{Case 2.} Assume \(p\geq 1\) and \(\gamma=2^{p-1}\).
 We first we want to show (\ref{eq:ex3.17:a}).
 Define a 2-points measurable space: \(X=\{0,1\}\)
 with \(\mu(\{0\}) = \mu(\{1\}) = 1\). Let \(f,g:X\to[0,\infty]\)
 defined by
 \begin{equation*}
  f(0) = x\,, \quad f(1) = 1\,, \qquad g = 1\,.
 \end{equation*}
 Let $q$ be the conjugate exponent of $p$.
 By Minkowski inequality (Theorem~3.5(2) \cite{RudinRCA80}), we have
 \begin{eqnarray*}
   x + 1 = \int_X fg\,d\mu
   &\leq&
        \left(\int_X f^p\,d\mu\right)^{1/p}
        \left(\int_X f^q\,d\mu\right)^{1/q} \\
   &=& \left(x^p+1\right)^{1/p} \left(1^q+q^q\right)^{1/q}
   = 2^{1/q} \left(x^p+1\right)^{1/p}.
 \end{eqnarray*}
 Raising to $p$ power, noting that \(p/q = p-1\) we now have
 \begin{equation} \label{eq:ex3.17:a}
  (x+1)^p \leq 2^{p-1} (x^p+1)
 \end{equation}

 Now we can derive
 \begin{equation*}
 (a+b)^p = (a+xa)^p =  a^p (x+1)^p
  \leq 2^{p-1} a^p(x^p + 1) = \gamma_p(a^p + b^p).
 \end{equation*}


\itemch{b}
 (Directly, \emph{not} following the hint). We use our local variant of Lebesgue's
 dominated theorem~\ref{lem:Lebesgue:domvar} (page \pageref{lem:Lebesgue:domvar}).
 Looking at \(f_n^p\in L^1(\mu)\) playing both roles of
 \(\{f_n\}\) and \(\{g_n\}\) there. Now the final result
 \begin{equation*}
  \lim_{n\to\infty} \|f_n^p - f^p \| = 0.
 \end{equation*}
 follows.


 Now going through the suggested hints
 (though unecessary after result established).
 Take arbitrary \(\epsilon > 0\).
 Let \(K= \{x\in X: f(x)=0\}\) and
 \begin{equation*}
   X_n = \{x\in X: 2^{n-1} \leq |f(x)|^p < 2^n\}
 \end{equation*}
 Clearly \(X = K \disjunion (\Disjunion_{n\in\Z} X_n)\) and
 \begin{equation*}
  \|f\|_p^p = \int_X |f|^p\,d\mu = \sum_{n\in\Z} \int_{X_n} |f|^p\,d\mu < \infty.
 \end{equation*}
 Hence \(\mu(X_n) < \infty\) for all \(n\in\Z\),
 and we can find some \(N<\infty\) such that
 \begin{equation*}
  T = \sum_{|n|>N} \int_{X_n} |f|^p\,d\mu < \epsilon/2.
 \end{equation*}
 We put
 \begin{equation*}
   A' = \Disjunion_{|n|>N} X_n
 \end{equation*}
 and we have \(\int_{A'} |f|^p\,d\mu < \epsilon/2\).
 We will now deal with
 \begin{equation*}
   B' \eqdef X \setminus A' = \Disjunion_{n = -N}^N X_n
 \end{equation*}
 \begin{equation*}
 \|f\|_p^p - T
  = \sum_{|n|\leq N} \int_{X_n} |f|^p\,d\mu
  \geq \sum_{|n|\leq N} 2^{n-1}\mu(X_n)
  \geq 2^{N-1} \sum_{|n|\leq N} \mu(X_n)
  = 2^{N-1} \mu(B').
 \end{equation*}
 We will concentrate now on the \(2N+1\) sets \(\{X_n\}\) for \(|n|<N\).
 Each has a finite \(\mu\) measure.
 Restricting \(\{f_n\}\) and $f$ to these sets and applying Egoroff's theorem
 on each \(X_n\)
 gives a partition \(X_n = A_n \disjunion B_n\)
 such that
 \begin{equation*}
    \int_{A_n} |f|^p\,d\mu < 2^{-2(|n| + 1)}\epsilon
 \end{equation*}
 and \(\{f_n\}\to f\) uniformly on \(B_n\).
 Since \(f_n\) is bounded on \(X_n\),
 \(\{|f_n|^p\}\to |f|^p\) uniformly on \(X_n\) as well,
 and also on any finite sub-union of them
 and in particular on
 \begin{equation*}
  B \eqdef \Disjunion_{n= -N}^N B_n \subset B'.
 \end{equation*}
 We put
 \begin{equation*}
  A = X \setminus B = A' \disjunion \Disjunion_{n= -N}^N A_n\,.
 \end{equation*}

 Combining results on $A$ and $B$ gives
 \begin{eqnarray}
      \int_X |f|^p\,d\mu
  &=& \lim_{n\to\infty} \int_X |f_n|^p\,d\mu \notag \\
  &=& \lim_{n\to\infty} \int_A |f_n|^p\,d\mu + \int_B |f_n|^p\,d\mu \notag \\
  &=& \limsup_{n\to\infty} \int_A |f_n|^p\,d\mu +
      \liminf_{n\to\infty} \int_B |f_n|^p\,d\mu  \label{eq:3.17b:1} \\
  &=& \int_B |f|^p\,d\mu + \limsup_{n\to\infty} \int_A |f_n|^p\,d\mu \notag
 \end{eqnarray}
 where (\ref{eq:3.17b:1}) is by local lemma~\ref{lem:limsup:liminf}.
 Now we get hint-(\emph{i}) estimate:
 \begin{equation*}
  \limsup_{n\to\infty} \int_A |f_n|^p\,d\mu
  = \int_X |f|^p\,d\mu - \int_B |f|^p\,d\mu
  = \int_A |f|^p\,d\mu < \epsilon\,.
 \end{equation*}

 Since \(\mu(B) < \infty\) (finite sum of positive numbers),
 and \(f_n^p\to f^p\) uniformly on $B$, clearly
 \begin{equation} \label{eq:3.17b:3}
  \lim_{n\to\infty} \int_B |f_n^p - f|\,d\mu = 0.
 \end{equation}
 Hence it is sufficient to show that (\ref{eq:3.17b:3}) holds
 also for $A$ instead of $B$, this will make it valid for $X$ as well,
 thus establishing the desired conclusion.

 \begin{eqnarray}
  2\gamma_p \int_A |f|^p\,d\mu
  &=& \notag
   \int_A \gamma_p |f|^p\,d\mu
   +
   \int_A \lim_{n\to\infty} \gamma_p |f_n|^p\,d\mu
   +
   \int_A \lim_{n\to\infty} |f - f_n|^p\,d\mu \\
  &=& \notag
   \int_A \lim_{n\to\infty} \gamma_p(|f|^p + |f_n|^p) - |f - f_n|^p\,d\mu \\
  &=& \notag
   \int_A \liminf_{n\to\infty} \gamma_p(|f|^p + |f_n|^p) - |f - f_n|^p\,d\mu \\
  &\leq& \label{eq:3.17b:fatou}
   \liminf_{n\to\infty} \int_A \gamma_p(|f|^p + |f_n|^p) - |f - f_n|^p\,d\mu \\
  &\leq& \notag
   \limsup_{n\to\infty} \int_A \gamma_p(|f|^p + |f_n|^p)\,d\mu
   +\liminf_{n\to\infty} \left(-\int_A |f - f_n|^p\,d\mu\right) \\
  &=& \notag
   \gamma_p \int_A |f|^p\,d\mu
   + \gamma_p \limsup_{n\to\infty} \int_A |f_n|^p\,d\mu
   - \limsup_{n\to\infty} \int_A |f - f_n|^p\,d\mu \\
 \end{eqnarray}

 Where (\ref{eq:3.17b:fatou}) is by Fatou lemma and \ich{a}.
 Thus
 \begin{equation*}
  \limsup_{n\to\infty} \int_A |f - f_n|^p\,d\mu  + \gamma_p \int_A |f|^p\,d\mu
  \leq \gamma_p \limsup_{n\to\infty} \int_A |f_n|^p\,d\mu.
 \end{equation*}
 Therefore
 \begin{equation*}
  \limsup_{n\to\infty} \int_A |f - f_n|^p\,d\mu  \leq \gamma_p \epsilon.
 \end{equation*}
 This shows (\ref{eq:3.17b:3}) for $A$ replaceing $B$ as needed.


\itemch{c}
 We will construct a counterexample.
 Let \(X=[0,1]\) with Lebesgue measure.
 Define a sequence of functions in \(L^1\)
 \begin{equation*}
   f_n(x) = n(n+1)\chhi_{[1/(n+1),1/n]}(x) \qquad \textrm{for}\; 1\leq n\in\N.
 \end{equation*}
Clearly \(\lim_{n\to\infty} f_n(x) = 0\) for every \(x\in[0,1]\), but
\(\|f_n\|_1 = 1\)

\end{itemize}


%%%%%%%%%%%%%% 18
\begin{excopy}
Let \(\mu\) be a positive measure on $X$. A sequence \(\{f_n\}\) of complex
measurable functions on $X$ is said to \emph{converge in measure}
to the measurable function $f$ if to every \(\epsilon>0\) there corresponds
an $N$ such that
\begin{equation*}
 \mu(\{x: |f_n(x) - f(x)| > \epsilon\}) < \epsilon
\end{equation*}
for all \(n>N\).
(This notion is of importance in probability theory.)
Assume \(\mu(X)<\infty\) and prove the following statements:
\begin{itemize}
 \itemch{a} If \(f_n(x)\to f(x)\;\aded\), then \(f_n\to f\) in measure.
 \itemch{b} If \(f_n\in L^p(\mu)\) and \(\|f_n-f\|_p \to 0\),
            then \(f_n\to f\) in measure; here \(1\leq p \leq \infty\).
 \itemch{c} If  \(f_n\to f\) in measure, then \(\{f_n\}\) has a subsequence
            which converges to $f$ \aded.
\end{itemize}
Investigate the convergences of \ich{a} and \ich{b}.
What happens to \ich{a}, \ich{b}, and \ich{c} if \(\mu(X)=\infty\),
for instance, if \(\mu\) is Lebesgue measure on \(\R^1\)?
\end{excopy}


\begin{itemize}
\itemch{a}
 Given \(\epsilon>0\), for each \(x\in X\) there is a least $m_x$
 such that \(|f_n(x) - f(x)| < \epsilon\) for all \(n \geq m_x\).
 For each \(m\in\N\) define
 \begin{eqnarray*}
   X_m
    &\eqdef& \{x\in X: m = m_x\} \\
    &=& \{x\in X: \forall n\geq m,\, |f_n(x) - f(x)| < \epsilon
        \; \wedge\; (m=1 \,\vee\, \exists n<m,\,  |f_n(x) - f(x)| \geq \epsilon
       \}.
 \end{eqnarray*}
 Clearly \(X=\disjunion_{n\in\N}X_i\), and \(X_i\) are measurable,
 so there is \(N_\epsilon\)  such that
 \begin{equation*}
 \mu\left(X\setminus \Disjunion_{i=1}^{N_\epsilon} X_i\right) < \epsilon.
 \end{equation*}
 This shows that \(f_n\to f\) in measure.
\itemch{b}
 By negation, assume there is \(\epsilon>0\)
 such that for each $N$, there exists \(n>N\)
 such that the subset
 \begin{equation*}
    E \eqdef \{x\in X: |f_n(x) - f(x)| \geq \epsilon\}
 \end{equation*}
 satisfies \(\mu(E) \geq \epsilon\).
 But then
 \begin{equation*}
   \int_X |f_n - f|^p\,d\mu
   \geq \int_E |f_n - f|^p\,d\mu
   \geq \epsilon^p\mu(E) = \epsilon^{p+1}
 \end{equation*}
 Thus
 \begin{equation*}
 \limsup_{n\to\infty} \|f_n-f\|_p =
 \limsup_{n\to\infty} \left(\int_X |f_n - f|^p\,d\mu\right)^{1/p}
 \geq \epsilon^{(p+1)/p} > 0
 \end{equation*}
 contradiction the assumption that \(\|f_n - f\|_p \to 0\).

\itemch{c}
 First let's illustrate the difficulty by constructing a case
 of sequence of functions converging in measure but converging \emph{nowhere}.
 Define for \(n\geq 1\)
 \begin{eqnarray*}
   s_n &=& \sum_{k=1}^n 1/k \\
   a_n &=& s_n - \lfloor s_n \rfloor \\
   b_n &=& \min(1, a_n + 1/(n+1)
 \end{eqnarray*}
 Note that \(a_n \leq b_n = a_{n+1} \leq a_n + 1/n\)
 and \(b_n - a_n \leq 1/(n+1)\).
 The sequence of characteristic functions  \(\chhi_{[a_n,b_n]}\) converges
 to $0$, but converges nowehere.

 Now back to the exercise task. For each \(\epsilon_k = 2^{-k}\)
 pick \(N_k\) such that for any \(n>N_k\)
  \begin{equation*}
   B_k \eqdef \mu(\{x: |f_n(x) - f(x)| > \epsilon_k\}) < \epsilon_k = s^{-k}.
  \end{equation*}
 Now take the subsequence \(\calF = (f_{N_k + 1})_{k\in\N}\).
 We now show that this sequence converges almost everywhere.
 The sequence \calF\ does not converge for \(x\in X\)
 iff $x$ belong to infinitely many subsets \(B_k\).
 That is
 \begin{equation*}
  x\in  = B = \bigcap_{j=1}^\infty \left( \bigcup_{k=j}^\infty B_k \right)\,.
 \end{equation*}
 But clearly \(\mu(B) = 0\).
\end{itemize}

Now assume \(X=\R^1\) with Lebesgue measure.
With \(f_n(x) = \chhi_{[n,n+1]}\), clearly \(f_n(x)\to 0\) for all \(x\in\R\)
but not in measure, thus \ich{a} does not hold.
But both \ich{b} and \ich{c} hold, since in their proofs above
we did not make use of the fact that \(\mu(X)<0\).

%%%%%%%%%%%%%% 19
\begin{excopy}
Define the \emph{essential range} of a function \(f\in L^\infty(\mu)\)
to be the set \(R_f\) consisting of all complex numbers $w$ such that
\begin{equation*}
 \mu(\{x: |f(x) - w | < \epsilon\}) > 0
\end{equation*}
for every \(\epsilon > 0\). Prove that \(R_f\) is compact.
What relation exists between the set \(R_f\) and the number \(\|f\|_\infty\)?

Let \(A_f\) be the set of all averages
\begin{equation*}
 \frac{1}{\mu(E)}\int_E f\,d\mu
\end{equation*}
where \(E\in\frakM\) and \(\mu(E)>0\).
What relations exist between \(A_f\) and \(R_f\)?
Is \(A_f\) always closed?
Are there measures \(\mu\) such that \(A_f\) is convex
for every \(f\in L^\infty(\mu)\)?
Are there measures \(\mu\) such that \(A_f\) fails to be convex for
some \(f\in L^\infty(\mu)\)?

How are these results affected if \(L^\infty(\mu)\) is replaced
by \(L^1(\mu)\), for instance?
\end{excopy}

Since \(f\in L^\infty(\mu)\), the essential range \(R_f\) is bounded in \C.
To show compactness, it is sufficent to show that \(R_f\) is closed.
By negation, let \(z\in \overline{R_f} \setminus R_f\).
For every open ball \(B = B(z,\epsilon)\)
with center $z$ and radii \(\epsilon > 0\)
there is \(w\in B\cap R_f\).
Now we take \(\eta = (\epsilon - |w-z|)/2\) and
\begin{equation*}
 G \eqdef \{x\in X: |f(x) - w| < \eta\} \subset
 H_\epsilon \eqdef \{x\in X: |f(x) - z| < \epsilon\}
\end{equation*}
since for \(|u-w|<\eta\;\Rightarrow\; |u - z|<\epsilon\) for any \(u\in\C\).
Since \(w\in R_f\), we have \(\mu(G) > 0\) but then \(\mu(H_\epsilon) > 0\).
This is contradiction to \(z\notin R_f\) since \(\epsilon\) was arbitrary.

It is easy to see that  \(\|f\|_\infty = \sup\{|z|: z\in R_f\}\).
Each number of \(A_f\) is a convex combination of numbers in \(R_f\).
Formally, \(A_f \subset \conv(R_f)\).

The set \(A_f\) need \emph{not} be closed. For example, let \(X=\N\)
with atomic measure \(\mu(\{n\}) = 1\) for all \(n\in X\),
and let \(f(n) = 1/n\). Clearly \(0\in \overline{A_f}\setminus A_f\).

There \emph{are} measures \(\mu\) for with \(A_f\) is convex
for every \(f\in L^\infty(\mu)\).
For example Lebesgue's measure on subsets of \(\R^n\) is shown
in \loclemma~\ref{llem:averages:convex}.

There \emph{are} measures for which \(A_f\) is not convex.
For example, let \(X=\{0,1\}\) with atomic counting measure \(\mu(A)=|A|\)
for all \(A\subset X\). For the function \(f(x) = x\), clearly
\(R_f=\{0,1\}\) and \(A_f=\{0,1/2,1\}\) which is not convex.

When replacing the space with \(L^1(\mu)\) we may get
a \emph{non} compact \(R_f\).
% \mldots


%%%%%%%%%%%%%%  20
\begin{excopy}
Suppose \(\varphi\) is a real function on \(\R^1\) such that
\begin{equation*}
 \varphi\bigl(\int_0^1 f(x)dx\bigr) \leq \int_0^1  \varphi(f)dx
\end{equation*}
for every real bounded measurable $f$. Prove that \(\varphi\) is then convex.
\end{excopy}

For any \(a,b\in\R\) and \(t\in [0,1]\)
define \(f:[0,1] \to \{a,b\}\) by
\begin{equation*}
 f(x) = \left\{\begin{array}{ll}
               a & \qquad x \in [0,t] \\
               b & \qquad x \in (t,1]
               \end{array}\right.
\end{equation*}
Now the assumed inequality gives
\begin{equation*}
  \varphi\bigl( ta + (1-t)b\bigr)
 =
  \varphi\left(\int_0^1 f(x)dx\right)
 \leq
  \int_0^1  \varphi(f)dx
 =
  t \varphi(a) + (1-t)\varphi(b)
\end{equation*}
which is the definition of convex function.



%%%%%%%%%%%%%% 21
\begin{excopy}
Call a metric space $Y$ a \emph{completion} of a metric space $X$ if $X$ is
dense in $Y$ and $Y$ is complete.
In Sec.~3.15 reference was made to ``the'' completion of a metric space.
State and prove a uniqueness theorem which justifies this terminology.
\end{excopy}

The uniqueness can be expressed by the following
\begin{llem}
Let \(Y_1\) and \(Y_2\) two completions of a metric space $X$.
Then there is a unique isometry (implying one-to-one onto continuous mapping)
\(T:Y_1\to Y_2\) inducing the identity on $X$.
\end{llem}
\begin{thmproof}
Given the \(Y_1\) and \(Y_2\) completions of $X$
define $T$ as the identity on $X$ and extend to \(Y_1\) as follows.
For each \(y_1\in Y_1\setminus X\)
there is Cauchy sequence \(\mathbf{x} = (x_n)_{n\in\N}\) in $X$
that converges to \(y_1\). Now \(\mathbf{x}\) is a Cauchy sequence
in \(Y_2\) and converges to a unique \(y_2\in Y_2\).
Define \(T(y_1) = y_2\). We need to show that the mapping is well defined.
Say \(\mathbf{w} = (w_n)_{n\in\N}\) in $X$ converges to \(y_1\) as well.
Then
\begin{equation*}
 x_1, w_1, x_2, w_2, \ldots x_n, w_n, \ldots
\end{equation*}
converges to \(y_1\) and this is a Cauchy sequence in $X$.
It must converge to a limit in \(Y_2\) which must be \(y_2\),
and so does \(\mathbf{w}\). Thus \(T(y_1)\) is independent on the choice
of converging sequence to \(y_1\).
\end{thmproof}

%%%%%%%%%%%%%% 22
\begin{excopy}
Suppose $X$ is a metric space in which every Cauchy sequence has a convergent
subsequence. Does it follow that $X$ is complete?
(See the proof of Theorem~3.11.)
\end{excopy}

The answer is ``Yes''.

If every Cauchy sequence \(\mathbf{x}\)
has a subsequence that converges to a limit $L$, then it can be easily
shown that  \(\mathbf{x}\) itself converges to $L$.

%%%%%%%%%%%%%% 23
\begin{excopy}
Suppose \(\mu\) us a positive measure on $X$, \(\mu(X)<\infty\),
\(f\in L^\infty(\mu)\), \(\|f\|_\infty > 0\), and
\begin{equation*}
\alpha_n = \int_X |f|^n\,d\mu \qquad (n=1,2,3,\ldots).
\end{equation*}
Prove that
\begin{equation*}
 \lim_{n\to \infty} \frac{\alpha_{n+1}}{\alpha_n} = \|f\|_\infty.
\end{equation*}
\end{excopy}

Put \(M = \|f\|_\infty\) and Define \(g = f/M\).
Now \(\|g\|_\infty = 1\) and
\begin{equation*}
 \frac{\alpha_{n+1}}{\alpha_{n}}
 =
 \frac{\int_X (Mg)^{n+1}\,d\mu}{\int_X (Mg)^{n}\,d\mu}
 =
 M \frac{\int_X g^{n+1}\,d\mu}{\int_X g^{n}\,d\mu}.
\end{equation*}
Hence it is sufficent to show that
\begin{equation} \label{3.23:gn}
 \lim_{n\to\infty} \frac{\int_X g^{n+1}\,d\mu}{\int_X g^{n}\,d\mu} = 1.
\end{equation}

By exercise~4\ich{e},
actually \(\lim_{n\to\infty} \int_X g^n\,d\mu = 1\)
and thus (\ref{3.23:gn}) follows.

%%%%%%%%%%%%%% 24
\begin{excopy}
Suppose \(\mu\) is a positive measure.
\(f\in L^p(\mu)\), \(g\in L^p(\mu)\).
\begin{itemize}
 \itemch{a}
  If \(0<p<1\), prove that
  \begin{equation*} % typo in book (extra '|')
   \int \bigl| |f|^p - |g|^p \bigr|\,d\mu \leq \int |f - g|^p \,d\mu
  \end{equation*}
  and that \(\Delta(f,g) = \int|f-g|^p\,d\mu\)
  defines a metric on \(L^p(\mu)\).
 \itemch{b}
  If \(1 \leq p < \infty\) and \(\|f\|_p\leq R\), \(\|g\|_p\leq R\),
  prove that
  \begin{equation*}
   \int \bigl| |f|^p - |g|^p \bigr|\,d\mu \leq 2pR^{p-1}\|f-g\|_p.
  \end{equation*}
  \emph{Hint}: Prove first, for \(x\geq 0\), \(y\geq 0\), that
  \begin{equation*}
   |x^p - y^p| \leq
   \left\{\begin{array}{ll}
          |x-y|^p                   & \quad \textrm{if}\; 0<p<1, \\
          p|x-y|(x^{p-1} + y^{p-1}) & \quad \textrm{if}\; 1\leq p < \infty.
          \end{array}\right.
  \end{equation*}
\end{itemize}
Note that \ich{a} and \ich{b} establish the continuity of the mapping
\(f\to |f|^p\) that carries \(L^p(\mu)\) into \(L^1(\mu)\).
\end{excopy}

Following the hints.
But symmetry of the expressions with regards to $x$ and $y$,
we can assume \wlogy\ that \(x>y\).

Assume \(0 < p < 1\). If \(y=0\) then
\begin{equation*}
 |x^p - 0^p| = x^p = |x-0|^p.
\end{equation*}
Otherwise, by dividing by \(y^p\) we need to show
\begin{equation} \label{eq:3.24:xp1}
 x^p - 1 \leq (x-1)^p
\end{equation}
or equivalently that
\begin{equation*}
 \varphi(x) = x^p - (x-1)^p - 1 \leq 0
\end{equation*}
for \(x\geq 1\).
Indeed \(\varphi(1) = 0\) and since \(x\to x^{p-1}\) is decreasing
\begin{equation*}
\varphi'(x) = p\bigl(x^{p-1} - (x-1)^{p-1}\bigr) < 0.
\end{equation*}
Hence \(\varphi\) is decreasing for \(x\geq 0\) and (\ref{eq:3.24:xp1}) holds.

Assume \(1\leq p < \infty\). \Wlogy, we can assume \(x\geq y\).
If \(y=0\)
\begin{equation*}
 x^p - 0^p \leq px^p = p(x-0)(x^{p-1} + 0^{p-1})
\end{equation*}
Otherwise, \(y>0\) and
\begin{equation*}
 x^p - y^p \leq p(x-y)(x^{p-1} + y^{p-1})
\end{equation*}
is equivalent --- dividing by \(y^p=yy^{p-1}\) --- to
\begin{equation*}
 (x/y)^p - 1 \leq p\bigl((x/y)-1\bigr)\bigl((x/y)^{p-1} + 1).
\end{equation*}
Hence it is sufficient to show
\begin{equation} \label{eq:3.24:xp2}
 x^p - 1 \leq p(x-1)(x^{p-1} + 1)
\end{equation}
or equivalently,
\begin{equation*}
 \varphi(x) = p(x-1)(x^{p-1} + 1) - (x^p - 1) \geq 0
\end{equation*}
for \(x\geq 1\). Clearly, \(\varphi(1) = 0\).
\begin{eqnarray*}
 \varphi'(x)
 &=& p\bigl( (x^{p-1} + 1) + (x-1)(p-1)x^{p-1} \bigr) - px^{p-1} \\
 &=& p\bigl( (x-1)(p-1)x^{p-1} + 1 \bigr) \\
 &\geq& 0.
\end{eqnarray*}
Thus \(\varphi\) is increasing for \(x\geq 1\) and  (\ref{eq:3.24:xp2}) holds.



\begin{itemize}
\itemch{a}
 The inequality is immediate from the integrands satisfy
 \(\bigl| |f|^p - |g|^p \bigr| \leq |f - g|^p\)
 by the hint's first (\(0<p<1\)) part.

 The function \(\Delta\) indeed is a metric since
 clearly \(\Delta(f,g) = 0\) iff \(f = g\,\aded\).
 and by the inequality we just justified,
 for every \(f,g,h \in L^p(\mu)\)
 \begin{equation*}
   \int \bigl| |f-g|^p - |g-h|^p \bigr|\,d\mu
    \leq \int |(f-g) - (g-h)|^p \,d\mu
    = \int |f - h|^p \,d\mu
 \end{equation*}
 But this shows
 \begin{equation*}
   \Delta(f,g) = \int |f-g|^p\,d\mu
    \leq \int |f - h|^p \,d\mu + \int |g - h|^p \,d\mu =
   \Delta(f,h) +  \Delta(h,g).
 \end{equation*}
 With the triangle shown, \(\Delta\) is a metric.

\itemch{b}
 Using the hint's inequality. Let $q$ be the conjugate exponent of $p$.
 Compute
 \begin{eqnarray}
   \int \bigl| |f|^p - |g|^p \bigr|\,d\mu
  &\leq& \label{eq:3.24b:hint}
   p \int |f-g|\cdot ( |f|^{p-1} + |g|^{p-1})\,d\mu \\
  &\leq& \label{eq:3.24b:mink}
   p \|f-g\|_p \left(\int (|f|^{p-1} + |g|^{p-1})^q\,d\mu\right)^{1/q} \\
  &\leq& \label{eq:3.24b:holder}
   p \|f-g\|_p \left(\int (|f|^{p-1})^q\,d\mu +
                     \int (|g|^{p-1})^q\,d\mu \right)^{1/q} \\
  &=& \label{eq:3.24b:expconj}
   p \|f-g\|_p \left(\int (|f|^p\,d\mu + \int (|g|^p\,d\mu \right)^{1/q} \\
  &=& \notag
   p \|f-g\|_p (\|f\|_p^p + \|g\|_p^p)^{1/q} \\
  &\leq& \notag
   p \|f-g\|_p (2R^p)^{1/q} =  p \|f-g\|_p 2^{1/q}R^{p-1} \\
  &\leq& \label{eq:3.24b:expconjgt1}
   2pR^{p-1}\|f-g\|_p. \notag
 \end{eqnarray}

Where
  (\ref{eq:3.24b:hint}) follows by the hint,
  (\ref{eq:3.24b:mink}) by Minkowski's inequality,
  (\ref{eq:3.24b:holder}) by H\"older's inequality,
  (\ref{eq:3.24b:expconj}) by $q$ being conjugate exponent (\((p-1)q = p\)),
  (\ref{eq:3.24b:expconjgt1}) by \(q\geq 1\).
\end{itemize}

% Added in 3rd edition

\begin{excopy}

Suppose \(\mu\) is a positive measure on $X$ and \(f: X\to(0,\infty)\) satisfies
\(\int_X f\,d\mu=1\).
Prove, for every \(E\subset X\) with \(0 < \mu(E) < \infty\), that
\begin{equation*}
\int_E (\log f)\,d\mu \leq \mu(E) \log \frac{1}{\mu(E)}
\end{equation*}
and, when \(0<p<1\).
\begin{equation*}
\int_E f^p\,d\mu \leq \mu(E)^{1-p}\,.
\end{equation*}
\end{excopy}

\begin{excopy}
If $f$ is a positive measure function on \([0,1]\), which is larger
\begin{equation*}
\int_0^1 f(x)\log f(x)\,dx 
\qquad \textnormal{or} \qquad
\int_0^1 f(s)\,ds \int_0^1 \log f(t)\,dt\; \textnormal{?} 
\end{equation*}
\end{excopy}


%%%%%%%%%%%%%%%
\end{enumerate}
%%%%%%%%%%%%%%%

 % \setcounter{chapter}{3}  % -*- latex -*-
% $Id: rudinrca4.tex,v 1.2 2008/07/19 08:56:55 yotam Exp $

%%%%%%%%%%%%%%%%%%%%%%%%%%%%%%%%%%%%%%%%%%%%%%%%%%%%%%%%%%%%%%%%%%%%%%%%
%%%%%%%%%%%%%%%%%%%%%%%%%%%%%%%%%%%%%%%%%%%%%%%%%%%%%%%%%%%%%%%%%%%%%%%%
%%%%%%%%%%%%%%%%%%%%%%%%%%%%%%%%%%%%%%%%%%%%%%%%%%%%%%%%%%%%%%%%%%%%%%%%
\chapterTypeout{Elementary Hilbert Space Theory}

%%%%%%%%%%%%%%%%%%%%%%%%%%%%%%%%%%%%%%%%%%%%%%%%%%%%%%%%%%%%%%%%%%%%%%%%
%%%%%%%%%%%%%%%%%%%%%%%%%%%%%%%%%%%%%%%%%%%%%%%%%%%%%%%%%%%%%%%%%%%%%%%%
\section{Notes}

%%%%%%%%%%%%%%%%%%%%%%%%%%%%%%%%%%%%%%%%%%%%%%%%%%%%%%%%%%%%%%%%%%%%%%%%
\subsection{Completeness of the Trigonometric System}
\label{sec:comp:trig}

In showing the completeness of the trigonometric system,
on page~95, the text uses the following equality
\begin{equation*}
\frac{c_k}{\pi}\int_0^\pi \left(\frac{1 + \cos t}{2}\right)^k \sin t\,dt =
  \frac{2c_k}{\pi(k+1)}.
\end{equation*}
Let us show it in details. Define
\begin{equation*}
f(t) \eqdef \bigl((1+\cos(t))/2\bigr)^{k+1}.
\end{equation*}
Now
\begin{equation*}
f'(t)
= -\bigl(\sin(t)/2\bigr)\cdot(k+1)\bigl((1+\cos(t))/2\bigr)^k
% = -\bigl((k+1)\sin(t)/2\bigr)\cdot\bigl((1+\cos(t))/2\bigr)^k.
= \bigl(-(k+1)/2\bigr)\cdot\bigl((1+\cos(t))/2\bigr)^k\sin(t).
\end{equation*}
Hence
\begin{equation*}
\int_0^\pi \left(\frac{1 + \cos t}{2}\right)^k \sin t\,dt
 = \bigl(-2/(k+1)\bigr) \bigl(f(\pi) - f(0)\bigr)
 = \bigl(-2/(k+1)\bigr) (0 - 1)
 = 2/(k+1).
\end{equation*}
Which justifies the above quoted equality.


%%%%%%%%%%%%%%%%%%%%%%%%%%%%%%%%%%%%%%%%%%%%%%%%%%%%%%%%%%%%%%%%%%%%%%%%
%%%%%%%%%%%%%%%%%%%%%%%%%%%%%%%%%%%%%%%%%%%%%%%%%%%%%%%%%%%%%%%%%%%%%%%%
\section{Exercises} % pages 97-99

In this set of exercises, $H$ always denotes a Hilbert space.

%%%%%%%%%%%%%%%%%
\begin{enumerate}
%%%%%%%%%%%%%%%%%

%%%%%%%%%%%%%% 1
\begin{excopy}
If $M$ is a closed subspace of $H$, prove that
\(M = (M^\perp)^\perp\).
Is there a similar true statement for subspaces
$M$ which are not necessarily closed?
\end{excopy}

As discussed in section~4.9 \cite{RudinRCA80}, \(A^\perp\) is a closed
subspace for any \emph{subset} \(A\subset H\). Directly by definitions
we also have \(A \subset (A^\perp)^\perp\).
If we re-apply theorem~4.11(2) \cite{RudinRCA80}, with \(M^\perp\)
instead of $M$, we get again unique projections
$P$, $Q$ and \(P^\perp\)  of $H$ onto
$M$, \(M^\perp\) and \((M^\perp)^\perp\) respectably.
That is for any \(x\in H\) we have
\begin{equation*}
 x = Px + Qx = Qx + P^\perp x.
\end{equation*}
Hence \(P=P^\perp\) and so \(M = (M^\perp)^\perp\).
where $P$, $Q$ and \(P^\perp\) are projections of $H$ onto
$M$, \(M^\perp\) and \((M^\perp)^\perp\).

If $M$ is not closed, we cannot ensure the equality, as the following
example show. Let
\begin{equation*}
M \eqdef \{x\in\ell^2: \exists m < 0\,\forall n\geq m\; x(n) = 0\}.
\end{equation*}
Clearly $M$ is a non closed subspace of \(\ell^2\) and we have
\(M^\perp = \{0\}\) and
\(M \subsetneq \ell^2 = (M^\perp)^\perp\).


%%%%%%%%%%%%%% 2
\begin{excopy}
For \(n=1,2,3,\ldots\), let \(\{v_n\}\) be an independent set of vectors in $H$.
Develop a \emph{constructive} process which generates an orthonormal set
\(\{u_n\}\), such that \(u_n\) is a linear combination of \seqn{v}.
Note that this leads to a proof of the existence of a maximal orthonormal
set in a separable Hilbert spaces which makes no apeal to the Hausdorff
maximality principle.
(A space is \emph{separable}
\index{separable}
if it contains a countable dense subset.)
\end{excopy}

This is the
\index{Graham-Schmidt!Orthonormalization}
\index{Orthonormalization!Graham-Schmidt}
\emph{Graham-Schmidt Orthonormalization}.

By induction, Let \(u_1 \eqdef v_1/\|v_1\|\).
Let \(k\geq 1\).
Assume \(\{u_j\}\) defined for all \(1\leq j < k\).
Define
\begin{eqnarray*}
 {u'}_k &\eqdef& v_k - \sum_{j=1}^{k-1} \langle u_j,v_k \rangle \cdot u_j \\
 u_k    &\eqdef& {u'}_k \,/\, \|{u'}_k\|
\end{eqnarray*}


%%%%%%%%%%%%%% 3
\begin{excopy}
Show that \(L^p(T)\) is separable if \(1\leq p < \infty\),
but that \(L^\infty(T)\) is not separable.
\end{excopy}

Assume \(1\leq p < \infty\).
We will show that the set $Q$ of trigonometric polynomial
with (complex) rational coefficients is dense in \(L^p(T)\).
Clearly \(|Q| = \aleph_0\).

Assume \(\epsilon > 0\) and \(f\in L^p(T)\).
By Theorem~3.14 \cite{RudinRCA80} \(C_c(T)\) is dense in \(L^p(T)\).
Since $T$ is compact \(C_c(T)=C(T)\).
Take \(g\in C(T)\) such that \(\|g - f\|_p < \epsilon/3\).

By Theorem~4.25 \cite{RudinRCA80}, the trigonometric polynomial
are dense in C(T) in the \(\|\cdot\|_\infty\) norm.
Take a trigonometric polynomial $p$ such that
\(\|p - g\|_\infty < \epsilon/(6\pi)\), hence
\(\|p - g\|_p < \epsilon/3\).

For each coefficient \(c_j\) of $p$, we can find a sequence of
rationals \(\{q_{jk}\}_{k=1}^\infty\)
such that \(\lim_{k\to\infty} q_{jk} = c_j\).
Note that  the degree $N$ of $p$ is finite (\(c_j = 0\) if \(|j| > N\)).
Put
\begin{equation*}
h_k(t) \eqdef \sum_{j=-N}^N q_{jk} e^{ijt}
\end{equation*}
For each $k$,
we have \(\lim_{k\to\infty} q_{jk} e^{ijt} = c_k e^{ijt}\) uniformly.
Hence  \(\lim_{k\to\infty} h_k(t) = p(t)\) uniformly as well.
Therefore, we can pick some \(q_k \in Q\) such that
\(\|q - p\|_\infty < \epsilon/(6\pi)\), hence.
\(\|q - p\|_p < \epsilon/3\).
Combining the results, \(\|q - f\|_p < \infty\)
and so $Q$ is dense in \(L^P(T)\) which is thus separable.

We will now show that  \(L^\infty(T)\) is not separable.
For each \(r\in[-\pi,\pi)\) put
\begin{equation*}
 u_r \eqdef\chhi_{[-\pi,r]}.
\end{equation*}
Look at the set of \(R \eqdef \{u_r:  r\in[-\pi,\pi)\} \subset L^\infty(T)\).
Clearly \(|R| > \aleph_0\), but
for any two \(-\pi \leq r < s < \pi\) we have \(\|u_r - u_s\| = 1\).
Assume by negation there exists a countable dense set $D$ in \(L^\infty(T)\).
Put \(\epsilon = 1/3\). For each \(u_r\in R\) there exists some \(f_r\in D\)
such that \(\|u_r - f_r\|_\infty < 1/3\). By simple cardinality argument,
there must exist some pair \(r<s\) such that \(f_r = f_s\).
But then
\begin{equation*}
 \|u_r - u_s\|_\infty
 \leq \|u_r - f_r\|_\infty + \|f_r - f_s\|_\infty + \|f_s - u_s\|_\infty
 \leq 1/3 + 0 + 1/3 = 2/3 < 1
\end{equation*}
which us a contradiction.

%%%%%%%%%%%%%% 4
\begin{excopy}
Show that $H$ is separable if and only if $H$ contains a maximal orthonormal
system which is at most countable.
\end{excopy}

Assume $H$ is separable.
Let \(\{v_j\}_{j\in\N}\) a countable set which is dense in $H$.
By induction, we can through out all vectors \(v_k\)
which are depenent ob \(\{v_j\}_{1\leq j<k}\).
We can now use the
Graham-Schmidt!Orthonormalization (Exercise~2 above) to get orthonormal
system whose cardinality is at most \(|\N| = \aleph_0\).

Conversely, assume \(U=\{u_j\}_{j\in J}\) is a maximal orthonormal system
where \(|J| \leq \aleph_0\).
Clearly the set
of all finite linear combinations of $U$ with (complex) rational coefficients
\begin{equation*}
D \eqdef \left\{\sum_{j\in F} (q_j+ir_j)v_j :
    F\subset J \;\wedge\;
   |F|<\infty \;\wedge\;
   q_j, r_j \in \Q\right\}
\end{equation*}
is dense in $H$ and \(|D|=\aleph_0\).


%%%%%%%%%%%%%% 5
\begin{excopy}
If \(M = \{x: Lx = 0\}\), where $L$ is a continuous linear functional on $H$,
prove that \(M^\perp\) is a vector space of dimension $1$ (unless \(M=H\)).
\end{excopy}

Assume \(M\neq H\) and \(v\in H\setminus M\).
If by negation \(\dim(M^\perp)> 1 \) then there are
linearly independent \(v_1,v_2\in \M^\perp\).
Clearly \(v_1\neq 0 \neq v_2\)
and \(L(v_1)\neq 0 \neq L(v_2)\).
Put \(v \eqdef L(v_2)v_1 - L(v_1)v_2 \in M^\perp\).
By the independence of \(v_1,v_2\) we have \(v\neq 0\).
But
\begin{equation*}
L(v) = L\bigl( L(v_2)v_1 - L(v_1)v_2 \bigr) = L(v_2)L(v_1) - L(v_1)L(v_2) = 0.
\end{equation*}
which is a contradiction.


%%%%%%%%%%%%%% 6
\begin{excopy}
Let \(\{u_n\}\) (\(n=1,2,3,\ldots\)) be an orthonormal set in $H$.
Show that this gives an example of a closed and bounded set which is
not compact.
Let $Q$ be the set of all \(x\in H\) of the form
\begin{equation*}
x = \sum_1^\infty c_n u_n
   \qquad \left(\textrm{where} |c_n| \leq \frac{1}{n}\right).
\end{equation*}
Prove that $Q$ is compact. ($Q$ is called the Hilbert cube.)

More generally, let \(\{\delta_n\}\) be a sequence of positive numbers,
and let $S$ be the set of all \(x\in H\) od the form
\begin{equation*}
x = \sum_1^\infty c_n u_n
  \qquad \left(\textrm{where} |c_n| \leq \delta_n\right).
\end{equation*}
Prove that $S$ is compact if and only if \(\sum_1^\infty \delta_n^2 < \infty\).
Prove that $H$ is not locally compact.
\end{excopy}

The set \(U = \{u_n\}\) is clearly bounded by $1$.
It it was not closed and \(x\in\overline{U}\setminus U\) then
there would be a sequence $s$ in $U$ that converges to $x$.
\Wlogy, we can remove repetitions from the $s$.
But then $s$ cannot be a Cauchy sequence since \(\|u_j - u_k\|=2\)
for any \(j<k\).
Now let \(V = \{x\in H: \|x\| < 1/3\}\) be a zero neighborhood.
The family
\begin{equation*}
\{u_n + V\}_{n=1}^\infty
\end{equation*}
is an open covering of $U$ and clearly has no finite sub-covering.
Therefore, $U$ is not compact.

% Assume $G$ is a covering of the Hilbert cube $Q$.
Assume \(\sum_1^\infty \delta_n^2 < \infty\).
For each $n$, let
\begin{eqnarray*}
H_n &\eqdef& \{x\in H: x = \sum_{j=1}^n c_j u_j: |c_j|\}
             = \{x\in H: \forall m > n,\; \langle x,u_j\rangle = 0\}
             \subset S. \\
K_n &\eqdef& S \cap H_n
% K_n &\eqdef& \{\sum_1^n c_j u_j: |c_j| \leq \delta_j\} \subset S.
\end{eqnarray*}
Clearly \(K_n\) is homeomorphic to $n$-dimensional closed box in \(\C^n\)
and thus is compact (See Hiene Borel Theorem~2.41 \cite{RudinPMA85}).
\begin{quotation}
\small
Clearly \(\cup_n K_n \subset S\).
Equality \(\cup_n K_n = S\) holds iff there exist \(m<\infty\)
such that \(\delta_j = 0\) for all \(j>m\).
\end{quotation}

\iffalse
Let $G$ be an open covering of $S$.
For each $n$, the $G$ covering of $S$ is also covering of \(K_n\).
Let \(G_n \subset G\) a finite subcovering of \(K_n\).
We may also assume \(G_n \subset G_{n+1}\)
(otherwise, we simply annex \(G_n\) to \(G_{n+1}\)).
Suppose by negation that $S$ is \emph{not} compact.
Then \(\cup G_n \subsetneq S\) for each $n$.
Pick \(x_n \in S \setminus \cup G_n\).
We will now find a subsequence of \((x_n)_{n\in\N}\) that converges in $S$,
by getting subsequences that converge on projections on \(K_n\)'s.
Moving to double indexing, we put \(x_{0,n} = x_n\).
By induction assume that for $k$ the sequence \((x_{k,n})_{n\in\N}\)
is defined.
Pick a subsequences \((x_{k+1,n})_{n\in\N}\) whose projection on \(K_{k+1}\)
converges. Taking the diagonal subsequences, let \(y_n = x_{n,n}\).
Its projection (\(\langle\cdot,u_k\rangle\))
on \(K_k\) converges for all $k$.
Moreover its limit \(t = \sum_{k=1}^\infty \langle x,u_k \rangle u_k\)
converges in $S$ since \(\|t\| \leq \sum_{k=1}^\infty \delta_k^2\).
Thus \(t\in S\). Pick \(V\in G\) such that \(t\in V\).
There exist \(0<r\in\R\) such that \(B(t,r) \subset V\).
Let $m$ be such that
\begin{equation*}
\sum_{j=m+1}^\infty \delta_j^2 < r
\end{equation*}
Let \(t'\) be the projection of $t$ on \(H_m\).
So now \(t' \in K_n\).
\mldots
\fi

We will show that $S$ is homeomorphic to
the product space \(T = \prod_{k=1}^\infty \{z\in \C: |z|<\delta_k\)
with the weak topology.
By Tychonoff theorem (Appendix A~3 \cite{RudinFA79}) $T$ is compact.
The homeomorphism is given by the identity mapping. We need to show that
it is continuous in both directions.
\begin{itemize}
 \item
 Look at \(\Id: S \to T\). Let \(x\in S\) and note that \(\Id(x) = x\in T\).
 Let $V$ be a base neighborhood of $x$ in $T$. That is for some finite
 subset \(I \subset \N\) we have
 \begin{equation*}
 V = \{v\in T: \forall j\in I,\, |v_j - x_j| < \epsilon_j\}.
 \end{equation*}
 Now take
 \begin{equation*}
 U = \{s\in S: \|s-x\| < \min_{j\in J} \epsilon_j\}
 \end{equation*}
 and clearly \(\Id(U) = U \subset V\).

 \item
 Look at \(\Id: T \to S\).
 Let \(x\in T\) and note that \(\Id(x) = x\in S\).
 Pick a neighborhood
 \begin{equation*}
 V = \{s\in S: \|s-x\| = \epsilon
 \end{equation*}
 of \(x\in S\).
 There exist some \(M<\infty\) such that
\(\sum_{j=m+1}^\infty \delta_j^2 < \epsilon/2\)
 Now pick a base neighborhood
 \begin{equation*}
 U = \left\{v\in T:
     \forall 1\leq j\leq m,\, |v_j - x_j| < 2^{-(j+1)}\epsilon_j\right\}.
 \end{equation*}
 and clearly \(\Id(U) = U \subset V\).
\end{itemize}
Thus the identity is a homeomorphism. Since $T$ is compact so is $S$.


In particular, the Hilbert cube $Q$ is compact.

Conversely, assume that $S$ is compact.
If by negation \(\sum_1^\infty \delta_n^2 = \infty\).
then the family \(V_n \eqdef = \{x\in H: \|x\| < n\}\) of open sets
is a covering pf $S$ but has no finite sub-covering
since for each $n$ we can find some $m$ such that
\(\sum_{j=1}^m \delta_j^2 > n\) and then
\begin{equation*}
w \eqdef \sum_{j=1}^m \delta_j u_j \notin \bigcup_{j=1}^k V_n.
\end{equation*}
Therefore $G$ has no subcovering which is a contradiction.


%%%%%%%%%%%%%% 7
\begin{excopy}
Suppose \(\{a_n\}\) is a sequence of positive numbers
such that \(\sum a_n b_n < \infty \)
whenever \(b_n \geq 0\)
and
  \(\sum b_n^2 < \infty\).
Prove that
  \(\sum a_n^2 < \infty\).

\emph{Suggestion:} If \(sum a_n^2 = \infty\) then there are disjoint
sets \(E_k\) (\(k=1,2,3,\ldots\)) so that
\begin{equation*}
 \sum_{n\in E_k} a_n^2 > 1.
\end{equation*}
Define \(b_n\) so that \(b_n = c_k a_n\) for \(n\in E_k\). For suitably chosen
\(c_k\), \(\sum a_n b_n = \infty\) although \(\sum b_n^2 < \infty\).
\end{excopy}

Following the suggestion. Assume  \(\sum a_n^2 = \infty\)
and sets \(\{E_k\}_{k\in\N}\) such that \(\N = \disjunion E_k\) and
\(\sum_{n\in E_k} a_n^2 > 1\).
Define \(c_k\) so that
\begin{equation*}
\sum_{n\in E_k} (c_k a_n)^2  = c_k^2 \sum_{n\in E_k} a_n^2  = 1/k^2
\end{equation*}
Hence
\begin{equation*}
c_k \eqdef 1 \left/ \left( k \sqrt{\sum_{n\in E_k} a_n^2} \right)\right.
\end{equation*}
Now
\begin{equation*}
\sum_{n\in E_k} a_n b_n = c_k \sum_{n\in E_k} a_n^2
=  c_k \left. \sqrt{\sum_{n\in E_k} a_n^2} \right/ k
= 1/k
\end{equation*}
Hence
\(\sum_{n=1}^\infty b_n^2 < \infty\)
and
\(\sum_{n=1}^\infty a_n b_n = \infty\)
which contradicts the assumption on \(\{a_n\}\)
and so \(\sum a_n^2 < \infty\).


%%%%%%%%%%%%%%  8
\begin{excopy}
If \(H_1\) and \(H_2\) are two Hilbert spaces, prove thatone of them
is isomorphic to a subspace of the other. (Note that every closed subspace
of a Hilbert space is a Hilbert space.)
\end{excopy}

By the statemnt showm in section~4.19 of~\cite{RudinRCA87}, a Hilbert space
is determined by the cardinality of its orthonormal base.
By simply mapping the smaller orthonormal base of the two spaces
to the other we get an isomorphic embedding.

%%%%%%%%%%%%%% 9
\begin{excopy}
If \(A\subset[0,2\pi]\) and $A$ is measurable, prove that
\begin{equation*}
 \lim_{n\to\infty} \int_A \cos nx\,dx
 =
 \lim_{n\to\infty} \int_A \sin nx\,dx
 = 0.
\end{equation*}
\end{excopy}

By regularity of the Lebesgue measure, given \(\epsilon > 0\),
we can find a finite number of intervals \(\{I_k\}_{k=}^n\)
such that
\begin{eqnarray*}
 A &\subset& \bigcup_{k=1}^N I_k \\
 B &\eqdef& A\setminus \bigcup_{k=1}^N I_k \\
 m(B) &<& \epsilon/2.
\end{eqnarray*}
So it is sufficient to show that the above zero limits hold
for $A$ where $A$ is an interval.
The periods of \(\cos nx\) and \(\sin nx\) converge to zero
as \(n\to\infty\). Since  the intergal value
of a periodic function is determined by the ``residue'' interval
whose length is
\begin{equation*}
m(I) - \left\lfloor \frac{n m(I)}{2\pi} \right\rfloor \frac{2\pi}{n}
\leq 2\pi/n
\end{equation*}
and since the functions here are bounded, the limits are zero.

%%%%%%%%%%%%%%
\begin{excopy}
Let \(n_1 < n_2 < n_3 < \cdots\) be positive integers,
and let $E$ be the set of all \(x\in[0,2\pi]\) at which
\(\{\sin n_k x\}\) converges. Prove that \(m(E) = 0\).
\emph{Hint}: \(2\sin^2 \alpha = 1 - \cos 2\alpha\),
so \(\sin n_k x\to \pm 1/\sqrt{2}\;\aded\) on $E$,
by Exercise~9.
\end{excopy}

Assume by negation \(m(E) > 0\).
By previous exercise
\begin{equation*}
   \lim_{k\to\infty} \int_E \cos n_k x\,dm(x)
 = \lim_{k\to\infty} \int_E \sin n_k x\,dm(x) = 0
\end{equation*}
But Lebesgue's dominated convergence theorem
gives
\begin{eqnarray}
 \int_E \lim_{k\to\infty} \cos n_k x\,dm(x)  \label{eq:ex4:10}
      &=& \lim_{k\to\infty} \int_E \cos n_k x\,dm(x) = 0 \\
 \int_E \lim_{k\to\infty} \sin n_k x\,dm(x)  \notag
      &=& \lim_{k\to\infty} \int_E \sin n_k x\,dm(x) = 0
\end{eqnarray}

Let
\begin{eqnarray*}
E_+ &\eqdef& \{x\in E: \lim{k\to\infty} \cos n_k x > 0\} \\
E_0 &\eqdef& \{x\in E: \lim{k\to\infty} \cos n_k x = 0\} \\
E_- &\eqdef& \{x\in E: \lim{k\to\infty} \cos n_k x < 0\}
\end{eqnarray*}
Clearly \(m(E_+) > 0\) iff \(m(E_-) > 0\)
since otherwise \eqref{eq:ex4:10} would not hold.
But if \(m(E_+) > 0\) then we can apply again
the previous exercise, now to \(E_+\) which gives a contradicton
of
\begin{equation*}
\lim_{k\to\infty} \int_{E_+} \cos n_k x\,dm(x) > 0.
\end{equation*}
Thus \(m(E_0) = m(E) > 0\) and  \(\lim_{k\to\infty} \cos n_k x = 0\),
\aded\ on $E$.

The identity \(2\sin^2 \alpha = 1 - \cos 2\alpha\), now implies
\begin{equation*}
\lim_{k\to\infty} \sin n_k x = \pm\sqrt{2}/2
\end{equation*}
Appling previous exercise (again) on the two sets
\begin{equation*}
\{x\in E: \lim_{k\to\infty} \sin n_k x = -\sqrt{2}/2\}
\qquad
\{x\in E: \lim_{k\to\infty} \sin n_k x = \sqrt{2}/2\}
\end{equation*}
gives a contradiction with at least one of them.


%%%%%%%%%%%%%% 11
\begin{excopy}
Find a nonempty closed subset $E$ in \(L^2(T)\) that contains no element
of smallest norm.
\end{excopy}

We will construct \(\{f_n\}\) in \(L^2(T)\)  such that \(\|f_n\| = 1 + 1/n\).
We identify $T$ with \([0,2\pi)\).
We define (almost disjoint) sub-segments \(I_n = [a_{n-1},a_n]\)
of measure \(d_n = 2^{-n}\)
for \(n\geq 1\), by letting \(a_0 = 0\), and \(a_n = a_{n-1} + d_n\).
We define
\begin{equation*}
 f_n(t) = c_n\chhi_{I_n}(t).
\end{equation*}
We set
\begin{equation*}
c_n = (n+1) \bigm/ \left(n \|\chhi_{I_n}\|\right)
\end{equation*}
which ensures that \(\|f_n\| = 1 + 1/n\).
Clearly \(\|f_m - f_n\| > 1\) whenever \(m\neq n\).
The set \(F = \{f_n\}\) has no accumulation points and so $F$ is closed.


%%%%%%%%%%%%%% 11 2nd edition
\item[11b]
\begin{minipage}[t]{.8\textwidth}\footnotesize
[\textbf{Note:} This exercise was removed in the 3rd edition]

Prove that the identify
\begin{equation*}
4\langle x,y \rangle
=
    \|x+y\|^2
 -  \|x-y\|^2
 +  i\|x+iy\|^2
 -  i\|x-iy\|^2
\end{equation*}
is valid every inner product space, and show that it proves the
implication \ich{c}~\(\to\)~\ich{d} of Theorem~4.18.
\smallskip\hrule
\end{minipage}

Compute
\begin{eqnarray*}
 4\langle x,y \rangle
&=&
      2 \langle x,y \rangle
    + 2 \overline{\langle x,y \rangle}
    + 2 \langle x,y \rangle
    - 2 \overline{\langle x,y \rangle}  \\
&=&
      2 \langle x,y \rangle
    + 2 \overline{\langle x,y \rangle}
    + 2i\overline{i} \langle x,y \rangle
    + 2i^2\overline{\langle x,y \rangle}  \\
&=&
    \bigl(2 \langle x,y \rangle + 2 \langle y,x \rangle \bigr)
    + i \bigl(2 \langle x,iy \rangle + 2 \langle iy,x \rangle \bigr) \\
&=&
   \bigl(
      \langle x,x \rangle
    + \langle x,y \rangle
    + \langle y,x \rangle
    + \langle y,y \rangle \bigr)
 \\ &&
  -
   \bigl(
      \langle x,x \rangle
    - \langle x,y \rangle
    - \langle y,x \rangle
    + \langle y,y \rangle \bigr)
 \\
&&
  +
   i\bigl(
      \langle x,x \rangle
    + \langle x,iy \rangle
    + \langle iy,x \rangle
    + \langle iy,iy \rangle \bigr)
 \\ &&
  -
   i\bigl(
      \langle x,x \rangle
    - \langle x,iy \rangle
    - \langle iy,x \rangle
    + \langle iy,iy \rangle \bigr)
 \\
&=&
    \|x+y\|^2
 -  \|x-y\|^2
 +  i\|x+iy\|^2
 -  i\|x-iy\|^2
\end{eqnarray*}



%%%%%%%%%%%%%% 12
\begin{excopy}
The constants \(c_k\) in section~4.24 were shown to be such that
\(k^{-1}c_k\) is bounded.
Estimate the relevant integral more precisely and show that
\begin{equation*}
 0 < \lim_{k\to\infty} k^{-1/2} c_k < \infty.
\end{equation*}
\end{excopy}

Section~4.24 defines (see also \ref{sec:comp:trig}) \(c_k\) such that
\begin{eqnarray*}
Q_k(t) &\eqdef& c_k \left(\frac{1+\cos t}{2}\right)^k \\
\frac{1}{2\pi} \int_{-\pi}^\pi Q_k(t)\,dt &=& 1
\end{eqnarray*}

Thus
\begin{equation*}
c_k
= 2\pi \left/ \int_{-\pi}^\pi \left(\frac{1+\cos t}{2}\right)^k \,dt \right.
= 2^{k+1}\pi \left/ \int_{-\pi}^\pi (1+\cos t)^k \,dt \right..
\end{equation*}
Note that \(((1+\cos t)/2)^k\) is decreasing with $k$ and so a limit exists.

%%%%%%%%%%%%%% 13
\begin{excopy}
Suppose $f$ is a continuous function on \(\R^1\), with period $1$. Prove that
\begin{equation*}
 \lim_{N\to\infty} \frac{1}{N} \sum_{n=1}^N f(n\alpha) = \int_0^1 f(t)\,dt
\end{equation*}
for every irrational number \(\alpha\). \emph{Hint}: Do it first for
\begin{equation*}
 f(t) = \exp(2\pi ikt), \qquad k=0,\pm 1,\pm 2,\ldots
\end{equation*}
\end{excopy}

We use the \emph{binary modulus} notation for real numbers:
\begin{equation} \label{eq:4.13.bmod}
x \bmod 1 \eqdef x - \lfloor x\rfloor.
\end{equation}

Because of being periodic,
\(f(x) = f(x \bmod 1)\) for all \(x\in\R\).
Thus looking at $f$-evaluations of
\(\{n\alpha\}_{n=1}^N\) is equivalent
to looking at $f$-evaluations of
\begin{equation*}
A_N = \{n \alpha \bmod 1\}_{n=1}^N \subset (0,1)
\end{equation*}
Let \(X_N = \{x_n\}_{n=1}^N\) be an increasing re-ordering of \(A_N\).
Since \(\alpha\) is irrational, the sequence \(X_N\)
is strictly increasing. By defining
\begin{eqnarray*}
a_0 &=& 0 \\
a_k &=& (x_{k} + x_{k+1}) / 2 \qquad \textrm{for}\, 1\leq k < N \\
a_N &=& 1 \\
\end{eqnarray*}
We denote
\begin{eqnarray*}
\delta_N &\eqdef& \min_{1\leq n < N} a_{n+1} - a_{n} \\
\Delta_N &\eqdef& \max_{1\leq n < N} a_{n+1} - a_{n}.
\end{eqnarray*}

We have a partition of \([0,1]\).
Since $f$ is continuous, it is uniformly continuous on \([0,1]\).
By Theorem~6.7 of \cite{RudinPMA85}
the expression
\begin{equation*}
\lim_{N\to\infty} \frac{1}{N} \sum_{n=1}^N f(n\alpha)
= \lim_{N\to\infty} \frac{1}{N} \sum_{n=1}^N f(x_n)
\end{equation*}
\index{Riemann-Stieltjes}
becomes the Riemann-Stieltjes integral of $f$ over \([0,1]\)
provided we show that
\begin{equation} \label{eq:ex4.13:Delto:to0}
\lim_{N\to\infty} \Delta_N = 0.
\end{equation}


Clearly
\(\delta_N \leq 1/N\).
Thus there exists some integers \(j,k\in\N\) such that
\begin{equation*}
0 < d \eqdef (j\alpha \bmod 1) - (k\alpha \bmod 1) < 1/N.
\end{equation*}
Let \(M = \lceil 1/d \rceil \cdot \max(j,k\).
Looking at \(X_M\), we can see that it contains
all numbers of the form
\begin{equation*}
\{ md: 1\leq m \leq M\}
=
\{ (mj\alpha \bmod 1) - (mk\alpha \bmod 1): 1\leq m \leq M\}.
\end{equation*}
Hence \(\Delta_m \leq 1/N\) for all \(m\geq M\)
and so \eqref{eq:ex4.13:Delto:to0} is proved.


%%%%%%%%%%%%%% 14
\begin{excopy}
Compute
\begin{equation*}
 \min_{a,b,c} \int_{-1}^1 |x^3 - a - bx -cx^2|^2 \,dx
\end{equation*}
and find
\begin{equation*}
 \max \int_{-1}^1 x^3 g(x)\,dx,
\end{equation*}
where $g$ is subject to the restrictions
\begin{equation*}
  \int_{-1}^1 g(x)\,dx
  = \int_{-1}^1 x g(x)\,dx
  = \int_{-1}^1 x^2 g(x)\,dx
  = 0 ;
  \qquad
 \int_{-1}^1 |g(x)|^2\,dx=1.
\end{equation*}
\end{excopy}

Let's ortho-normalize the base of the $3$-dimensional subspace $S$
spanned by \(\{x^0, x^1, x^2\}\) in \(C[-1,1]\).
\begin{eqnarray*}
f_0(x) &=& \sqrt{2}/2 \\
f_1(x) &=& \sqrt{3/2}\cdot x \\
f_2(x) &=& x^2 - 1/3
\end{eqnarray*}
Note that
\begin{eqnarray*}
\langle f_2, f_0 \rangle
&=& \int_{-1}^1 (t^2 - 1/3)\cdot \sqrt{2}/2\,dt
   = (\sqrt{2}/2)\cdot \left(\left(\int_{-1}^1 t^2\,dt\right) - 2/3\right) \\
&=& (\sqrt{2}/2)\cdot \left(\left( (t^3/3)\bigm|_{-1}^1 \right) - 2/3\right)
= (\sqrt{2}/2)\cdot ( 2/3 - 2/3)
= 0
\end{eqnarray*}

The minimization we are looking for, is actually the \(L^2\) distance
between \(\tau(x) = x^3\) and~$S$.
We will project \(\tau\) on \(\{f_0,f_1,f_2\}\)
\begin{eqnarray*}
\langle \tau, f_0\rangle
  &=& \int_{-1}^1 t^3\cdot \sqrt{2}/2\,dt
      = (\sqrt{2}/2)\left( (1/4)t^4 \bigm|_{-1}^1 \right)
      = (\sqrt{2}/2)\cdot(1/4)\cdot 2
      = \sqrt{2}/4 \\
\langle \tau, f_1\rangle
 &=& \int_{-1}^1 t^3\cdot \sqrt{3/2}t,dt
      = \sqrt{3/2} \left( (1/5)t^5 \bigm|_{-1}^1 \right)
      =  \sqrt{2\cdot3}/5 \\
\langle \tau, f_2\rangle
 &=&  \int_{-1}^1 t^3\cdot (t^2 - 1/3)\,dt
       =  (t^6/6 - t^4/12)\bigm|_{-1}^1
       =  0
\end{eqnarray*}

\iffalse
\begin{eqnarray*}
\langle \tau, f_0\rangle &=&
  \int_{-1}^1 t^3\cdot \sqrt{2}/2\,dt \\
  &=& (\sqrt{2}/2)\left( (1/4)t^4 \bigm|_{-1}^1 \right) \\
  &=& (\sqrt{2}/2)\cdot(1/4)\cdot 2  \\
  &=& \sqrt{2}/4 \\
\langle \tau, f_1\rangle &=&
  \int_{-1}^1 t^3\cdot \sqrt{3/2}t,dt \\
 &=&  \sqrt{3/2} \left( (1/5)t^5 \bigm|_{-1}^1 \right) \\
 &=&  \sqrt{2\cdot3}/5 \\
\langle \tau, f_2\rangle
 &=&  \int_{-1}^1 t^3\cdot (t^2 - 1/3)\,dt \\
 &=&  (t^6/6 - t^4/12)\bigm|_{-1}^1 \\
 &=&  0
\end{eqnarray*}
\fi

Thus the projection is
\begin{equation*}
\tau'(x) \eqdef \sqrt{2}/4 f_0 + \sqrt{2\cdot3}/5 f_1
 = 3 x + 1/4.
\end{equation*}
The square distance is
\begin{equation*}
d^2(\tau,\tau')
 = \int_{-1}^1 t^3 - 3t - 1/4\,dt
 = (t^4/4 - 3t^2/2 - t) \bigm|_{-1}^1
 = 2
\end{equation*}
and so \(d(\tau,\tau') = \sqrt{2}\).

In order to find the maximal distance with $g$
we can simply apply exercise~\ref{ex:4.16}.
The condition on $g$
are exactly as required
there, namely \(g\in S^\perp\) and \(\|g\|=1\).
Thus the maximal distance is \(\sqrt{2}\).

\iffalse
We put
\begin{equation*}
\tilde{g}(x) \eqdef \tau(x) - \tau'(x) = x^3 - 3 x - 1/4
\end{equation*}
then \(\tilde{g} \in S^\perp\).
To normalize, we compute the norm
\begin{eqnarray*}
\|\tilde{g}\|^2
 &=& \int_{-1}^1 (t^3 - 3 t - 1/4)^2\,dt \\
 &=& \int_{-1}^1
      t^6   - 6t^4   - 2t^3  + 9 t^2 + 3t/2   + 1/16 \,dt \\
 &=& (t^7/7 - 6t^5/5 - t^4/2 + 3 t^3 + 3t^2/4 + t/16) \bigm|{-1}^1 \\
 &=& (t^4/2 + 3t^2/4) \bigm|{-1}^1 \\
 &=& 2(1 + 3/4) \\
 &=& 7/2 \\
\end{eqnarray*}
Finally we define $g$
\begin{eqnarray*}
g(x)
  \eqdef \tilde{g}(x)/\|\tilde{g}\|
  = \sqrt{7/2}t^3 - 3\sqrt{7/2}t - \sqrt{2\cdot 7}/8
\end{eqnarray*}
\fi


%%%%%%%%%%%%%% 15
\begin{excopy}
Compute
\begin{equation*}
 \min_{a,b,c} \int_0^\infty |x^3 - a - bx -cx^2|^2e^{-x} \,dx
\end{equation*}
State and solve the corresponding maximum problem,
as in Exercise~14.
\end{excopy}

With a little help from \texttt{http://integrals.wolfram.com}.

Let's ortho-normalize the base of the $3$-dimensional subspace $S$
spanned by \(\{x^0, x^1, x^2\}\) in \(C[0,\infty)\)
where the inner product is defined as
\begin{equation*}
\langle f,g \rangle \eqdef \int_0^\infty f(t)\overline{g(t)}e^{-t}\,dt
\end{equation*}
\index{Graham-Schmidt}
We use the Graham-Schmidt procedure.

\paragraph{Ortho-Normalize \(\phi_0(x)=x^0\)}.
\begin{equation*}
f_0(x) = 1
\end{equation*}

\paragraph{Ortho-Normalize \(\phi_1(x)=x^1\)}.
\begin{equation*}
\langle \phi_1, f_0 \rangle
 = \int_0^\infty t e^{-t}\,dt
 = \left(e{-t}(-1-t)\right)\bigm|_0^\infty
 = 0 - (-1) = 1
\end{equation*}
For
\begin{equation} \label{eq:ex4.15:psi1}
\psi_1 = \phi_1 - 1\cdot f_0
\end{equation}
we compute the norm
\begin{equation*}
\|\psi_1\|_2^2
 = \int_0^\infty \bigl(\psi_1(t)\bigr)^2 e^{-t}\,dt
 = \int_0^\infty (t - 1)^2 e^{-t}\,dt
 =  \left(e{-t}(-1-t^2)\right)\bigm|_0^\infty
 = 0 - 1\cdot (-1) = 1.
\end{equation*}
Hence we use \(\psi_1\) \eqref{eq:ex4.15:psi1} for
\begin{equation*}
f_1(x) = x - 1
\end{equation*}

\paragraph{Ortho-Normalize \(\phi_2(x)=x^2\)}.
\begin{equation*}
\langle \phi_2, f_0 \rangle
 = \int_0^\infty t^2 e^{-t}\,dt
 = \left((-t^2-2t-2)e{-t}\right)\bigm|_0^\infty
 = 2
\end{equation*}

\begin{equation*}
\langle \phi_2, f_1 \rangle
 = \int_0^\infty t^2(t-1) e^{-t}\,dt
 = \left((x^3-4x^2+8x-8) e^{-t}\right)\bigm|_0^\infty
 = 8
\end{equation*}

For
\begin{equation} \label{eq:ex4.15:psi2}
\psi_2 = \phi_2 - 2\cdot f_0 -8f_1.
\end{equation}
we compute the norm
\begin{eqnarray*}
\|\psi_2\|_2^2
 &=& \int_0^\infty \bigl(\psi_2(t)\bigr)^2 e^{-t}\,dt  \\
 &=& \int_0^\infty (t^2 - 2 - 8(t-1))^2 e^{-t}\,dt
      = \left((-t^4+12t^3-40t^2+16t-20)e{-t}\right)\bigm|_0^\infty \\
 &=& 20
\end{eqnarray*}

Hence we use \(\psi_2\) \eqref{eq:ex4.15:psi2} to get
\begin{eqnarray*}
f_2(x)
 &=& \psi_2(x)/\|\psi_2\| \\
 &=& (\sqrt{5}/10)(x^2 -8(x-1) -2) \\
 &=& (\sqrt{5}/10)(x^2 -8x + 6) \\
 &=& (\sqrt{5}/10)\cdot x^2 (-4\sqrt{5}/5)\cdot x^2 + 3\sqrt{5}/5
\end{eqnarray*}


The minimization we are looking for, is actually the \(L^2\) distance
between \(\tau(x) = x^3\) and~$S$.
We will project \(\tau\) on \(\{f_0,f_1,f_2\}\)
\begin{eqnarray*}
\langle \tau, f_0\rangle &=&
 \int_0^\infty t^3 e^{-t} \,dt
    = ((-t^3 -3t^2 -6t -6)e^{-t})\bigm|_0^\infty  \\
 &=& 6 \\
\langle \tau, f_1\rangle
&=& \int_0^\infty t^3(t-1)e^{-t} \,dt
    = ((-t^4 -3t^3 -9t^2 -18t -18)e^{-t})\bigm|_0^\infty  \\
 &=& 18 \\
\langle \tau, f_2\rangle
 &=&  \int_0^\infty t^3\,(\sqrt{5}/10)(t^2 - 8t + 6)\, e^{-t} \,dt \\
 &=&  (\sqrt{5}/10)
      \bigl((-t^5 + 3t^4 + 6t^3 +18t^2 +36t +36)e^{-t}\bigr)\biggm|_0^\infty \\
 &=&  \sqrt{5}\cdot 18/5
\end{eqnarray*}

Thus the projection is
\begin{equation*}
\tau'(x) \eqdef 6 f_0 + 18 f_1 + (\sqrt{5}\cdot 18/5) f_2
\end{equation*}
Let \(\tau = \tau' + \tau''\) directo decomposition.
The square distance is
\begin{eqnarray*}
d^2(\tau,\tau')
&=& \|\tau''\|^2 \\
&=& \|\tau\|^2 - \|tau'\|^2 \\
&=& \|\tau\|^2 - \sum_{j=0}^2 \langle \tau, f_j \rangle^2 \\
&=& \int_0^\infty t^6 e^{-t} \,dt
    - \bigl(6^2 + 18^2 +  (\sqrt{5}\cdot 18/5)^2\bigr) \\
&=& \left((-t^6 -6t^5 - 30t^4 - 120 t^3
          - 360t^2 -720t - 720)e^{-t}\right) \bigm|_0^\infty  \\
& & - (36 + 324 + 324/5) \\
&=& 720 - 360 - 324/5 \\
&=&  1476/5
\end{eqnarray*}



Again we apply exercise~\ref{ex:4.16}, to get
\begin{equation*}
\max \{\langle g, \tau\rangle : g\in S^\perp\;\wedge\; \|g\|=1\}
= d(\tau,\tau')
\end{equation*}



%%%%%%%%%%%%%% 16
\begin{excopy}
If \label{ex:4.16}
\(x_0\in H\) and $M$ is a closed linear subspace  of $H$, prove that
\begin{equation*}
 \min \bigl\{\|x - x_0\|: x\in M\bigr\}
 = \max \left\{|\langle x_0,y\rangle|: y\in M^\perp,\, \|y\|=1\right\}.
\end{equation*}
\end{excopy}

Using projections,
let \(x_0 = \mu + \nu\) be the decomposition,
such that \(\mu\in M\) and \(\nu \in M^\perp\).
By Theorem~4.11\ich{a} (\cite{RudinRCA87}),
\begin{equation*}
d = d(x_0,M) = \min \bigl\{\|x - x_0\|: x\in M\bigr\} = d(x_0,\mu) = \|\nu\|.
\end{equation*}
By Theorem~4.11\ich{d} (\cite{RudinRCA87}),
\(\|x_0\|^2 = \mu^2 + \nu^2\).

Clearly \(\nu = 0\) iff \(x_0 \in M\). In this case \(d=0\)
and for every \(y\in M^\perp\) we have \(\langle x_0, y\rangle = 0\)
and the desired equality holds.

Otherwise, \(\nu \neq 0\).
We pick \(y_1 = \nu/\|\nu\|\) and so
\begin{equation*}
\langle x_0, y_1\rangle
= \langle \mu + \nu, \nu\rangle / \| \nu \|
= (\langle \mu , \nu\rangle +  \langle \nu, \nu\rangle) / \| \nu \|
= 0 + \|\nu\|^2 / \|\nu\|
= \|\nu\|.
\end{equation*}
Hence
\begin{equation} \label{eq:ex4.15:dleq}
d \leq  \max \left\{|\langle x_0,y\rangle|: y\in M^\perp,\, \|y\|=1\right\}.
\end{equation}
Assume \(y\in M^\perp\) such that \(\|y\|=1\).
% and \(\langle x_0,y\rangle > d\).
Let $N$ be the $1$-dimensional subspace spanned by \(\nu\),
and denote its orthogonal subspace in \(M^\perp\) by \(N^\perp\).
Let \(y = v + w\) be the direct decomposition,
such that \(v\in N\) and \(w\in N^\perp \subset M^\perp\).
Thus there exists a scalar \(a\in\C\) such that \(v = ay_1\)
and since
\begin{equation*}
\|v\| + \|w\| = \|y\| = \|y_1\| = 1
\end{equation*}
we have \(|a| \leq 1\).
Now
\begin{eqnarray*}
\langle y, x_0 \rangle
&=& \langle v+w, \mu + \nu \rangle \\
&=&   \langle v, \mu \rangle
    + \langle w, \mu \rangle
    + \langle v, \nu \rangle
    + \langle w, \nu \rangle \\
&=& 0 + 0 + \langle v, \nu \rangle + 0 \\
&=& \langle a y_1, \|\nu\| y_1 \rangle \\
&=& a\|\nu\|
\end{eqnarray*}
Hence \(|\langle y, x_0 \rangle| \leq \|\nu\|\), which shows
the desired reversed inequality of \eqref{eq:ex4.15:dleq}.

\iffalse
Since
\begin{equation*}
  \langle x_0, \nu \rangle
= \langle x_0, \nu \rangle
= \langle x_0, x_0 - \mu \rangle
= \|x_0\|^2 - \langle x_0, \mu \rangle
\end{equation*}
\fi

%%%%%%%%%%%%%% 17
\begin{excopy}
Show that there is a continuous one-to-one mapping \(\gamma\) of \([0,1]\)
into $H$ such that
\(\gamma(b) - \gamma(a)\) is orthogonal to
\(\gamma(d) - \gamma(c)\) whenever
\(0\leq a \leq b \leq c \leq d \leq 1\).
(\(\gamma\) may be called a ``curve with orthogonal increments.'')
\emph{Hint}: Take \(H=L^2\), and constants characteristic functions
 of certain subsets of \([0,1]\).
\end{excopy}

Simply define \(\gamma(t) = {\chhi}_{[0,t]}\).


%%%%%%%%%%%%%% 18 2nd edition
\item[18b]
\begin{minipage}[t]{.8\textwidth}\footnotesize
[\textbf{Note:} This exercise was removed in the 3rd edition]

Give a direct proof of Theorem~4.16, i.e., one which does not
depend on the more general consideration of Sec.~4.15.
\smallskip\hrule
\end{minipage}


%%%%%%%%%%%%%% 18
\begin{excopy}
Define \(u_s(t) = e^{ist}\) for all \(s\in \R^1\).
Let $X$ be the complex vector space of all finite linear combinations
of these functions \(u_s\).
If \(f\in X\) and \(g\in X\), show that
\begin{equation*}
\langle f,g\rangle
= \lim_{A\to\infty} \frac{1}{2A} \int_{-A}^A f(t)\overline{g(t)}\,dt
\end{equation*}
exists. Show that this inner product makes $X$ into a unitary spce
whose completion is a non separable Hint space $H$.
Show that \(\{u_s: s\in\R^1\}\) is a maximal orthonormal set in $H$.
\end{excopy}

% Using the \eqref{eq:4.13.bmod} notation,
For \(b>0\) we use the real modulus notation:
\begin{equation*}
 a \bmod b \eqdef a - \lfloor \frac{a}{b} \rfloor b.
\end{equation*}
It is clear that
\begin{equation*}
\int_{-A}^A u_s(t)\,dt = \int_{-(A \bmod 2\pi/s) }^{A \bmod 2\pi/s} u_s(t)\,dt
\end{equation*}
If \(m, n \in \Z\) then
\begin{equation*}
\langle u_m,u_n\rangle
= \lim_{A\to\infty} \frac{1}{2A} \int_{-A}^A u_m(t)\overline{u_n(t)}\,dt
= \lim_{A\to\infty} \frac{1}{2A} \int_{-A}^A e^{i(m-n)t}\,dt.
\end{equation*}
Hence, if \(m=n\) then
\begin{equation*}
\langle u_m,u_n\rangle
= \lim_{A\to\infty} \frac{1}{2A} \int_{-A}^A e^{i\cdot 0\cdot t}\,dt.
= \lim_{A\to\infty} \frac{1}{2A} \int_{-A}^A 1\,dt
= 1.
\end{equation*}
Otherwise, \(m\neq n\). Put \(d = m - n \neq 0\) and so
\begin{eqnarray*}
|\langle u_m,u_n\rangle|
&=& \left|\lim_{A\to\infty} \frac{1}{2A} \int_{-A}^A u_d(t)\,dt\right| \\
&=&\left|\lim_{A\to\infty}
   \frac{1}{2A} \int_{-(A \bmod 2\pi/d)}^{A \bmod 2\pi/d} u_d(t)\,dt\right| \\
&=& \lim_{A\to\infty} \frac{1}{2A} %
   \left|\int_{-(A \bmod 2\pi/d)}^{A \bmod 2\pi/d} u_d(t)\,dt\right| \\
&\leq&  \lim_{A\to\infty} \frac{1}{2A} \cdot 4\pi/d \\
&=& 0
\end{eqnarray*}
Hence \(\{u_s\}_{s\in\Z}\) is an orthonormal set.




%%%%%%%%%%%%%% 19
\begin{excopy}
Fix a positive integer $N$,
put \(\omega = e^{2\pi i/N}\), prove the orthogonality relations
\begin{equation*}
 \frac{1}{N} \sum_{n=1}^N \omega^{nk} =
 \left\{\begin{array}{ll}
         1 \qquad \textrm{if}\quad k=0\\
         0 \qquad \textrm{if}\quad 1\leq k \leq N - 1
        \end{array}\right.
\end{equation*}
and use them to derive the identities
\begin{equation*}
\langle x, y \rangle = \frac{1}{N} \sum_{n=1}^N \|x + \omega^n y\|^2 \omega^n
\end{equation*}
that hold in every inner product space if \(N\geq 3\). Show also that
\begin{equation*}
\langle x, y \rangle
 = \frac{1}{2\pi} \int_{-\pi}^\pi \|x + e^{i\theta} y\|^2 e^{i\theta}\,d\theta.
\end{equation*}
\end{excopy}

If \(k=0\) then
\begin{equation*}
  \frac{1}{N} \sum_{n=1}^N \omega^{nk}
= \frac{1}{N} \sum_{n=1}^N \omega^{n0}
= \frac{1}{N} N\cdot 1
= 1.
\end{equation*}
Otherwise \( 1\leq k < N\). Put
\begin{equation*}
S_k = \sum_{n=1}^N \omega^{nk}
\end{equation*}
Since \(\Omega^{Nk} = (\Omega^N)^k = 1^k = 1\) we have
\begin{equation*}
\omega^k S_k
= \sum_{n=1}^N \omega^{nk + k}
= \left(\sum_{n=2}^N \omega^{nk}\right) +  \omega^{Nk + k}
= \left(\sum_{n=1}^N \omega^{nk}\right)
= S_k.
\end{equation*}
If by negation, \(S_k\neq 0\) then
 \(\omega^k= 1\) which is a contradiction. Hence
\begin{equation*}
\frac{1}{N}S_k = \frac{1}{N}0 = 0.
\end{equation*}

Assume now that \(N\geq 3\). Compute:
\begin{eqnarray*}
S
&\eqdef&
 \sum_{n=1}^N \|x + \omega^n y\|^2 \omega^n \\
&=&
 \sum_{n=1}^N \langle x + \omega^n y, x + \omega^n y\rangle \omega^n  \\
&=&
    \|x\|^2 \sum_{n=1}^N \omega^n
 +  \sum_{n=1}^N \left\langle \omega^n x,  \omega^{n}y \right\rangle
 +  \sum_{n=1}^N \left\langle \omega^{2n} y, x \right\rangle
 + \|y\|^2 \sum_{n=1}^N  \omega^n \overline{\omega^n} \omega^n \\
&=&
    \|x\|^2 \cdot 0
    + \sum_{n=1}^N \left\langle x, \overline{\omega^n} \omega^{n}y \right\rangle
    + \langle y, x \rangle \sum_{n=1}^N \omega^{2n}
    + \|y\|^2 \sum_{n=1}^N \omega^n \\
&=&
    0 +
    \sum_{n=1}^N \left\langle x, \overline{\omega^n} \omega^{n}y \right\rangle
    + 0 + 0 \\
&=& N \langle x, 1\cdot y \rangle.
\end{eqnarray*}
Hence
\begin{equation*}
\langle x, y \rangle = S/N.
\end{equation*}

Now for the analog integral equality.
\begin{eqnarray*}
I
&\eqdef&
 \int_{-\pi}^\pi \|x + e^{i\theta} y\|^2 e^{i\theta}\,d\theta \\
&=&
 \int_{-\pi}^\pi
    \langle x + e^{i\theta} y, x + e^{i\theta} y \rangle e^{i\theta}\,d\theta \\
&=&
    \|x\|^2 \int_{-\pi}^\pi e^{i\theta}\,d\theta
 +  \int_{-\pi}^\pi
         \left\langle e^{i\theta} x,  \omega^{n} y \right\rangle\,d\theta
 +  \int_{-\pi}^\pi
         \left\langle e^{2i\theta} y, x \right\rangle\,d\theta
 + \|y\|^2 \int_{-\pi}^\pi
         e^{i\theta} \overline{e^{i\theta}} e^{i\theta}\,d\theta \\
&=&
     \|x\|^2 \cdot 0
   + \int_{-\pi}^\pi
      \left\langle x, \overline{e^{i\theta}} e^{i\theta}y
     \right\rangle \,d\theta
    + \langle y, x \rangle \int_{-\pi}^\pi e^{2i\theta}\,d\theta
    + \|y\|^2 \int_{-\pi}^\pi e^{i\theta}\,d\theta \\
&=&
    0 +
    \int_{-\pi}^\pi
      \left\langle x, \overline{e^{i\theta}} e^{i\theta}y \right\rangle
    + \langle x, 0\cdot y\rangle  + 0 \\
&=& 2\pi \langle x, 1\cdot y \rangle.
\end{eqnarray*}
Hence
\begin{equation*}
\langle x, y \rangle = \frac{1}{2\pi}I.
\end{equation*}

\end{enumerate}

 \setcounter{chapter}{4}  % -*- latex -*-
% $Id: rudinrca5.tex,v 1.8 2006/04/30 19:06:58 yotam Exp $


%%%%%%%%%%%%%%%%%%%%%%%%%%%%%%%%%%%%%%%%%%%%%%%%%%%%%%%%%%%%%%%%%%%%%%%%
%%%%%%%%%%%%%%%%%%%%%%%%%%%%%%%%%%%%%%%%%%%%%%%%%%%%%%%%%%%%%%%%%%%%%%%%
%%%%%%%%%%%%%%%%%%%%%%%%%%%%%%%%%%%%%%%%%%%%%%%%%%%%%%%%%%%%%%%%%%%%%%%%
\chapterTypeout{Example of Banach Space Techniques} % 5

%%%%%%%%%%%%%%%%%%%%%%%%%%%%%%%%%%%%%%%%%%%%%%%%%%%%%%%%%%%%%%%%%%%%%%%%
%%%%%%%%%%%%%%%%%%%%%%%%%%%%%%%%%%%%%%%%%%%%%%%%%%%%%%%%%%%%%%%%%%%%%%%%
\section{Notes}

%%%%%%%%%%%%%%%%%%%%%%%%%%%%%%%%%%%%%%%%%%%%%%%%%%%%%%%%%%%%%%%%%%%%%%%%
\subsection{Baire's Category Theorem}

Other applications of Baire's theorem are the following

\begin{llem} \label{lem:count:1cat}
If $A$ is a countable subset of a complete metric space $X$,
then $A$ is of first category.
\end{llem}

\begin{thmproof}
As a set --- every point is a nowhere dense. 
Since  $A$ is a countable union of its elements, it is of first category.
\end{thmproof}

\begin{llem} \label{lem:gdel:2cat}
If $D$ is a dense \(G_\delta\) subset of a complete metric space $X$,
then $D$ is of second category.
\end{llem}

\begin{thmproof}
Let \(D=\cup_{n\in\N}G_n\) where \(G_n\) are open. By assumption
all \(G_n\) are dense. Put \(F_n = X \setminus G_n\), now each \(F_n\) is
closed and nowehere dense, since otherwise \(G_n\) would not be dense.
Thus \(F = \cup_{n\in\N}F_n\) is of first category, 
therefore, since $X$ is of second category, 
so is \(D = X \setminus F\).
\end{thmproof}



%%%%%%%%%%%%%%%%%%%%%%%%%%%%%%%%%%%%%%%%%%%%%%%%%%%%%%%%%%%%%%%%%%%%%%%%
\subsection{Working out Dirichlet's kernel}

\index{Dirichlet's kernel}

Let's verify the equality of \textbf{5.11}(9) of \cite{RudinRCA80}.
First note that
% \begin{equation*}
\(e^{i\theta} = \cos\theta + i\sin\theta\)
% \end{equation*}
and so
\begin{equation*}
e^{i\theta} - e^{-i\theta}
=     \bigl(\cos(\theta) - \cos(-\theta)\bigr)
  +  i\bigl(\sin(\theta) - \sin(-\theta)\bigr)
= 2i\sin\theta.
\end{equation*}
We use the above equality twice, in the following computation
\begin{eqnarray*}
D_n(t)
 &\eqdef& \sum_{k=-n}^n e^{ikt} \\
 &=&      \frac{e^{it/2} - e^{-it/2}}{e^{it/2} - e^{-it/2}}
           \sum_{k=-n}^n e^{ikt} \\
 &=& \left(\sum_{k=-n}^n e^{ikt}(e^{it/2} - e^{-it/2})\right)
     \biggm/
     \bigl(2i\sin(t/2)\bigr)\\
 &=& \bigl(e^{i(n+1/2)t} - e^{-i(n+1/2)t}\bigr) \bigm/
     \bigl(2i\sin(t/2)\bigr) \\
 &=& 2i\sin\bigl((n+1/2)t\bigr) \bigm/
     \bigl(2i\sin(t/2)\bigr) \\
 &=& \sin\bigl((n+1/2)t\bigr) / \sin(t/2). \\
\end{eqnarray*}

%%%%%%%%%%%%%%%%%%%%%%%%%%%%%%%%%%%%%%%%%%%%%%%%%%%%%%%%%%%%%%%%%%%%%%%%
\subsection{Parseval's Identity}
\index{Parseval}

Let's derive an equality in section~5.24. Given 
\begin{equation*}
f(z) = \sum_{n=0}^N b_n(z-z_0)^n
\end{equation*}
we have
\begin{align*}
 \frac{1}{2\pi}
  \int_\pi^\pi \left|f(z_0 + re^{i\theta}\right|^2\,d\theta
&= \frac{1}{2\pi} \int_\pi^\pi 
   \left(\sum_{n=0}^N b_n r^n e^{ni\theta}\right) 
   \cdot
   \overline{\left(\sum_{n=0}^N b_n r^n e^{ni\theta}\right)}
   \,d\theta \\
&= \frac{1}{2\pi} \int_\pi^\pi 
   \left(\sum_{n=0}^N b_n r^n e^{ni\theta}\right) 
   \cdot
   \left(\sum_{n=0}^N \overline{b_n} r^n e^{-ni\theta}\right)
   \,d\theta \\
&=  \frac{1}{2\pi}  \sum_{m,n=0}^N  \int_\pi^\pi 
   b_m \overline{b_n} r^{m+n} e^{(m-n)i\theta}
   \,d\theta \\
&=  \frac{1}{2\pi}  \sum_{m,n=0}^N  
    b_m \overline{b_n} r^{m+n} 
   \int_\pi^\pi e^{(m-n)i\theta}
   \,d\theta \\
&=  \sum_{n=0}^N |b_n|^2 r^{2n} 
\end{align*}


%%%%%%%%%%%%%%%%%%%%%%%%%%%%%%%%%%%%%%%%%%%%%%%%%%%%%%%%%%%%%%%%%%%%%%%%
\subsection{Working out Poisson's kernel}

\index{Poisson's kernel}

Let's verify the equality of \textbf{5.24}(8) of \cite{RudinRCA80}.
\begin{eqnarray*}
1+2\sum_1^\infty \left(ze^{-it}\right)^n
&=& 1 + 2\left(-1 + \sum_0^\infty \left(ze^{-it}\right)^n\right) \\
&=& -1 + 2/\left(1 - ze^{it}\right) \\
&=& \frac{ze^{-it} - 1 + 2}{1 - ze^{-it}} \\
&=& \frac{e^{it} + z}{e^{it} - z} \\
&=& \frac{1 + ze^{-it}}{1 - ze^{-it}} \\
&=& \frac{ (1 + ze^{-it})\overline{(1 - ze^{-it}) } }{
           (1 - ze^{-it})\overline{(1 - ze^{-it}) } } \\
&=& \frac{ 1 + ze^{-it} - \overline{ze^{-it}} - ze^{-it}\overline{ze^{-it}} }{
           |1 - ze^{-it}|^2 } \\
&=& \frac{ 1 + re^{i\theta}e^{-it} - re^{-i\theta}e^{it}
             - re^{i\theta}e^{-it}re^{-i\theta}e^{it} }{
           |1 - ze^{-it}|^2 } \\
&=& \frac{ 1 - r^2 + r\left(e^{i(\theta-t)} - e^{i(t-\theta)}\right)
         }{ |1 - ze^{-it}|^2 } \\
&=& \frac{ 1 - r^2 + 2ir\sin(\theta-t) }{ |1 - ze^{-it}|^2 } \\
\end{eqnarray*}


%%%%%%%%%%%%%%%%%%%%%%%%%%%%%%%%%%%%%%%%%%%%%%%%%%%%%%%%%%%%%%%%%%%%%%%%
%%%%%%%%%%%%%%%%%%%%%%%%%%%%%%%%%%%%%%%%%%%%%%%%%%%%%%%%%%%%%%%%%%%%%%%%
\section{Exercises} % pages 112-115

%%%%%%%%%%%%%%%%%
\begin{enumerate}
%%%%%%%%%%%%%%%%%

%%%%%%%%%%%%%% 1
\begin{excopy}
Let $X$ consist of two points $a$ and $b$,
put \(\mu(\{a\}) = \mu(\{b\}) = \half\),
and let \(L^p(\mu)\) be the resulting \emph{real} \(L^p\)-space.
Identify each real function $f$ on $X$ with the point \((f(a),f(b))\)
in the plane, and skecth the unit balls of \(L^p(\mu)\),
for\(0<p\leq \infty\).
Note that they are convex if and only if \(1\leq p \leq \infty\).
For which $p$ is the unit ball a square? A circle?
If \(\mu(\{a\}) \neq \mu(\{b\})\), how does the situation differ from the
preceding one?
\end{excopy}

The unit balls are \(B_p=\{(x,y)\in\R^2: x^p + y^p \leq 1\}\).
Assume \(1\leq p \leq \infty\) and let \((x_1,y_1),(x_2,y_2)\in B_p\).
By Minkowski's inequality (Theorem~3.5 \cite{RudinRCA87})
\[
\bigl((x_1+x_2)^p/2 + (y_1+y_2)^p/2\bigr)^{1/p}
\leq
   (x_1^p/2 + y_1^p/2)^{1/p}
 + (x_2^p/2 + y_2^p/2)^{1/p}
.\]
This is equivalent to
\(\|(x_1+x_2,y_1+y_2)\|_p \leq \|(x_1,y_1)\|_p + \|(y_2,y_2)\|_p\)
or
\[\left\|\bigl((x_1+x_2)/2,(y_1+y_2)/2\bigr)\right\|_p
 \leq
 \bigl(\|(x_1,y_1)\|_p + \|(y_2,y_2)\|_p\bigr)\,/\,2.
\]
which show convextiy.

If \(0<p<1\), take real \(\alpha\) such that \(\alpha^p = 2\).
Let
\begin{eqnarray*}
(x_1,y_1) &\eqdef& (\alpha, 0) \\
(x_2,y_2) &\eqdef& (0, \alpha)
\end{eqnarray*}
Now clearly
\(\|(x_1,y_1)\|_p =  \|(x_2,y_2)\|_p = 1\)
but
\[
   \|\bigl((x_1+x_2)/2,(y_1+y_2)/2\bigr)\|_p^p
=  (\alpha/2)^p)/2 + (\alpha/2)^p)/2
=  (\alpha/2)^p
=  2^{1-p} > 1.
\]
Hence \(B_p\) is not convex.

When \(p=1\) the unit ball is a square inscribed in the unit circle.
When \(p=2\) the unit ball is the unit circle.
When \(p=\infty\) the unit ball is a square circumscribed around the unit circle.

If \(\mu(\{a\}) \neq \mu(\{b\})\) the squares change to rectangles
and the circle eo ellipse.
% 1-800 340340

%%%%%%%%%%%%%% 2
\begin{excopy}
Prove that the unit ball (open or closed) is convex in every normed linear space.
\end{excopy}

Let $U$ be an open unit ball of a normed linear space
and let \(v_0,v_1\in U\).
For any \(t\in[0,1]\) let \(v_t \eqdef v_0 + t(v_1 - v_0)\).
We need to show that \(v_t\in U\).
We may assume that \(v_0\neq v_1\) and \(0<t<1\), since otherwise
we trivially have \(v_t=v_0\) or \(v_t=v_1\).
By definition of a norm
\[
\|v_t\| = \|(1-t)v_0 + tv_1\|
\leq \|(1-t)v_0\| + \|tv_1\|
=     (1-t)\|v_0\| + t\|v_1\| < (1-t)+t = 1.
\]

If $U$ is a closed unit ball,
we simply change the last strict inequality ($<$) into \(\leq\).


%%%%%%%%%%%%%% 3
\begin{excopy}
If \(1<p<\infty\), prove that the unit ball
of \(L^p(\mu)\) is
\index{strictly convex}
\emph{strictly convex};
this means if
\[ \|f\|_p = \|g\|_p = 1, \qquad f\neq g, \qquad h = \half(f+g), \]
then \(\|h\|_p < 1\).
(Geometrically, the surface of the ball contains no straight lines.)
Show that this fails in every \(L^1(\mu)\), and in every \(C(X)\).
(Ignore trivialities, such as spaces consisting of only one point.)
\end{excopy}

Let $X$ be the functions' domain.
For all \(x\in X\) we have
\[|h(x)| = \half|f(x)+g(x)| \leq \half(|f(x)|+g|(x)|).\]
If there is in equality, then by local lemma~\ref{llem:minkowski:eq},
there are non negative real constants $a$ and $b$ such that
\(af=bg\;\aded\), but since \(\|f\|_p = \|g\|_p\)
we have \(a=b\) and the case is trivial \(f=g\;\aded\).


%%%%%%%%%%%%%%
\begin{excopy}
Let $C$ be the space of all continuous functions on \([0,1]\),
with the supremum norm.
Let $M$ consist of all \(f\in C\) for which
\[ \int_0^\half f(t)\,dt - \int_\half^1 f(t)\,dt = 1. \]
Prove that $M$ is a closed subset of $C$ which contains no element
of minimal norm.
\end{excopy}

If \(f_n\to f\) in the supremum norm, clearly
\[ \lim_{n\to \infty} \int_0^\half f_n(t)\,dt - \int_\half^1 f_n(t)\,dt =
                      \int_0^\half f(t)\,dt - \int_\half^1 f(t)\,dt = 1. \]
Thus $M$ is closed.

We will now show that \(\|f\|_\infty > 1\) for every \(f\in M\).
Let \(f\in C([0,1])\) be such that \(\|f\|_\infty \leq 1\).
and let \(h=f(1/2)\).
Two cases:
\begin{itemize}
\item
If \(|h|<1\) then let \(\delta>0\) be such that \(|f(x)|<(1+|h|)/2\)
for all \(x\in (1/2-\delta,1/2+\delta)\). Now
\begin{eqnarray*}
\left|\int_0^\half f(t)\,dt - \int_\half^1 f(t)\,dt\right|
&=&
\left|\int_0^{1/2-\delta} f(t)\,dt
+ \int_{1/2-\delta}^\half f(t)\,dt
- \int_\half^{\half+\delta} f(t)\,dt
- \int_{\half+\delta}^1 f(t)\,dt\right| \\
&\leq& (1/2-\delta) + (\delta/2 + \delta/2)(1+|h|)/2 + (1/2 - \delta) \\
&=& 1 + 2\bigl((1+|h|)/2 - 1\bigr)\delta \\
&<& 1.
\end{eqnarray*}

\item
If \(|h|=1\), then let \(\delta>0\) be such that \(|f(x)-h| < 1/3\)
when \(x\in (\half-\delta,\half+\delta)\).
Note that if \(|y_i-h|<1/3\) for \(i=1,2\) then \(|y_1-y_2| < 2/3\).
Now
\begin{eqnarray*}
\left|\int_0^\half f(t)\,dt - \int_\half^1 f(t)\,dt\right|
&=&
\left|\int_0^{1/2-\delta} f(t)\,dt
+ \int_{1/2-\delta}^\half f(t)\,dt
- \int_\half^{\half+\delta} f(t)\,dt
- \int_{\half+\delta}^1 f(t)\,dt\right| \\
&\leq& (1/2 - \delta)
       + \int_0^\delta \left(f(t+h-\delta) - f(t+h)\right)dt
       + (1/2 - \delta) \\
&\leq& 1 - 2\delta + (2/3)\delta \\
&<& 1.
\end{eqnarray*}
\end{itemize}
In both case we see that \(f\notin M\).
Next we will show that for any \(\epsilon>0\) there exists \(f\in M\)
such that \(\|f\|_\infty > 1 + \epsilon\).
These results combined, show that $M$ has no minimal normed element.


\[
\left|\int_0^\half f(t)\,dt - \int_\half^1 f(t)\,dt\right|
\leq \int_0^\half |f(t)|\,dt + \int_\half^1 |f(t)|\,dt
= \int_0^1 |f(t)|\,dt \leq \|f\|_\infty
\]
Let \(\epsilon>0\). Define
\[
f(x) = \left\{
       \begin{array}{ll}
       -1 - \epsilon& \qquad 0\leq x \leq 1/2 - \delta \\
        (x-1/2) (1+\epsilon)/\delta
                        & \qquad 1/2 - \delta \leq x \leq 1/2 + \delta \\
       1 + \epsilon  & \qquad 1/2 + \delta \leq x \leq 1
       \end{array}\right.
\]
Clearly \(\|f\|_\infty = 1/2+\epsilon\) and if we pick
\(\delta\) such that
\[2(1+\epsilon)(1/2 - \delta) + 2\delta(1+\epsilon)/2 = 1\]
we can have \(f\in M\).
Simplifying:
\[2(1+\epsilon)(1/2 - \delta) + 2\delta(1+\epsilon)/2
  = (1+\epsilon)(2(1/2 - \delta) + \delta)
  = (1+\epsilon)(1 - \delta)\]
and solving the above,
gives \(\delta = 1 - 1/(1+\epsilon) = \epsilon/(1+\epsilon)\).



%%%%%%%%%%%%%% 5
\begin{excopy}
Let $M$ be the set of all \(f\in L^1([0,1])\), relative to Lebesgue measure,
such that
\[\int_0^1 f(t)\,dt = 1.\]
Show that $M$ is a closed subset of  \(L^1([0,1])\) which contains
infinitely many element of minimal norm.
(Compare this and Exercise~4 with Then~4.10.)
\end{excopy}

Let \(\{f_n\}\) be a sequence in $M$ such that
\(\lim_{n\to\infty}\|f_n - f\|_1 = 0\)
where \(f\in L^1([0,1])\). Now
\begin{eqnarray*}
\|f\|_1 &\leq& \|f_n - f\|_1 + \|f_n\|_1 \\
 \|f_n\|_1 &\leq& \|f_n - f\|_1 + \|f\|_1
\end{eqnarray*}
and so given \(\epsilon > 0\) for sufficiently large $n$
\[ 1 - \epsilon \leq \|f\|_1 \leq 1 + \epsilon\]
thus \(\|f\|_1=1\) and $M$ is closed.



%%%%%%%%%%%%%% 6
\begin{excopy}
Let $f$ be a bounded linear functional on a subspace $M$ of a Hilbert space $H$.
Prove that $f$ has a \emph{unique} norm-preserving extension to a bounded
linear functional on $H$, and that this extension vanishes on \(M^\perp\)
\end{excopy}

The case \(f=0\) is trivial, thus \wlogy by dividing by \(\|f\|\)
we may assume that \(\|f\|=1\).
Clearly for every
\(x=v+w\in H\) where \(v\in M\) and \(w\in M^\perp\)
we can define \(F(x) = f(v)\) which is a well defined extension
and clearly \(\|F\|=1\).

Now we will show that this extension is unique.
Let $F$ be an norm-preserving extension of $f$ on $H$.
By negation, assume that $F$ does not vanish on \(M^\perp\)
we can find \(w\in M^\perp\) such that \(\|w\|=1\)
and \(F(w)=a>0\). Pick and arbitrary \(\epsilon > 0\),
for which pick \(u\in M\) such that \(\|u\|=1\)
and \(h \eqdef f(u) > 1-\epsilon\).
We will contradict the assumtion if we show
that there exist a real $t$ such that
\[\frac{F(u+tw)}{\|u+tw\|} > 1.\]
Equivalently,
\[(h+ta)^2 = (F(u+tw))^2 > \|u+tw\|^2 = t^2 + 1.\]
that is
\[(1-a^2)t^2 - 2hat + (1-h^2) < 0.\]
Since the leaading coefficient \(1-a^2>0\),
by looking at the discriminant, there exist some $T$
satisfying the above inequality(ies), iff
\[4h^2a^2 - 4(1-a^2)(1-h^2) > 0\]
that is if
\begin{equation} \label{eq:ex:5.6:disc}
(h^2a^2)>(1-a^2)(1-h^2).
\end{equation}
Since \(\lim_{\epsilon\to 0} h = 1\),
When \(\epsilon\to 0\), the left side of \eqref{eq:ex:5.6:disc}
converges to \(a^2 > 0\) while zero is the limit of the right side.
Hence such $t$ exists, \(\|F\|>1\) contradicting the assumtion.


%%%%%%%%%%%%%% 7
\begin{excopy}
Construct a bounded linear functional on some subspace of some \(L^1(\mu)\)
which has two (hence inifinitely many) distinct norm-preserving
to \(L^1(\mu)\).
\end{excopy}

Define \(X=\{0,1\}\) and \(\mu(\{0\}) = \mu(\{1\}) = 1/2\)
Let
\[ M \eqdef \{f\in L^1(X,\mu): f(1) = 0\}\]
and the functional on $M$  \(\Lambda f = f(0)/2\).
If \(f\in M\) then \(\|f\|_1 = |f(0)|/2\), and so
\[ \|\Lambda\|
= \sup \{|\Lambda f|: f \in M\;\wedge\;\|f\|=1\}
= \sup \{|f(0)/2|: f \in M\;\wedge\;|f(0)|/2=1\}
= 1. \]

Now we can easily extend \(\Lambda\) to \(L^1(X,\mu)\)
by the following two extensions
\begin{eqnarray*}
\Lambda_1(f) &=& f(0)/2 \\
\Lambda_2(f) &=& f(0)/2 + f(1)/2
\end{eqnarray*}


%%%%%%%%%%%%%% 8
\begin{excopy}
Let $X$ be a normed linear space, and let \(X^*\) be its dual space, as defined
in Sec.~5.21, with the norm
\[\|f\| = \sup \{|f(x)|: \|x\|\leq 1\}.\]
\begin{itemize}
 \itemch{a} Prove that \(X^*\) is a Banach space.
 \itemch{b}
 Prove that the mapping \(f\to f(x)\) is, for each \(x\in X\),
 a bounded linear functional on \(X^*\), of norm \(\|x\|\).
 (This gives a natural imbedding of $X$ on its ``second dual'' \(X^{**}\),
 the dual space of \(X^*\).)
 \itemch{c} Prove that \(\{\|x_n\|\}\) is bounded if \mset{x_n} is a sequence
 in $X$ such that \(\{f(x_n)\}\) is bounded for every \(f\in X^*\).
\end{itemize}
\end{excopy}

\begin{itemize}
 \itemch{a}
 Let \(\Lambda_1,\Lambda_2 \in X^*\) and \(z_1,z_2 \in \C\)
 for any \(x\in X\), by definition, we have
 \[ (z_1\Lambda_1 + z_2\Lambda_2)(x) = z_1\Lambda_1(x) + z_2\Lambda_2)(x)  \]
 and so \(X^*\) is linear vector space over \(\C\).
 Also \(\|z\Lambda(x)\| = |z||\Lambda(x)|\) so to prove \(\|\cdot\|_{X^*}\)
 is a norm, we show the triangle inequality:
 \begin{eqnarray*}
 \|\Lambda_1 + \Lambda_2\| 
 &=& \sup\{|(\Lambda_1 + \Lambda_2)(x)|: x\in X,\;\|x|=1\} \\
 &=& \sup\{|\Lambda_1(x) + \Lambda_2(x)|: x\in X,\;\|x|=1\} \\
 &\leq& \sup\{|\Lambda_1(x)| + |\Lambda_2(x)|: x\in X,\;\|x|=1\} \\
 &\leq&    \sup\{|\Lambda_1(x)|: x\in X,\;\|x|=1\}
         + \sup\{|\Lambda_2(x)|: x\in X,\;\|x|=1\} \\
 &=& \|\Lambda_1\| + \|\Lambda_2\|.
 \end{eqnarray*}
 It is now left to show completeness. 
 Let \mset{\Lambda_n} be a Cauchy sequence in \(X^*\).
 Pick arbitrary \(x\in X\setminus\{0\}\). Now since
 \begin{eqnarray*}
 |\Lambda_m(x) - \Lambda_n(x)|
 = |(\Lambda_m - \Lambda_n)(x)|
 = \|x\|\cdot |(\Lambda_m - \Lambda_n)(x/\|x\|)|
 \leq \|x\|\cdot\|\Lambda_m - \Lambda_n\|
 \end{eqnarray*}
 \mset{\Lambda_n x} is a Cauchy sequence in \(\C\) and converges there,
 and for \(x=0\) as well.
 Thus we can define the limit 
 \[ \Lambda_x \eqdef \lim_{n\to\infty} = \lim_{n\to\infty} \Lambda_n x.\]
 Now for \(x_1,x_2\in X\) and \(z_1,z_2\in \C\)
 \begin{eqnarray*}
 \Lambda (z_1 x_1 + z_2 x_2)
 &=& \lim_{n\to\infty} \Lambda_n (z_1 x_1 + z_2 x_2) \\
 &=& \lim_{n\to\infty} \bigl(z_1\Lambda_n(x_1) + z_2 \Lambda(x_2)\bigr) \\
 &=& \bigl(
             \lim_{n\to\infty} z_1\Lambda_n(x_1)
           + \lim_{n\to\infty} z_2\Lambda_n(x_2) \bigr) \\
 &=& \bigl(  z_1\lim_{n\to\infty} \Lambda_n(x_1)
           + z_2\lim_{n\to\infty} \Lambda_n(x_2) \bigr) \\
 &=& z_1\Lambda(x_1) + z_2\Lambda(x_2).
 \end{eqnarray*}
 To show \(\Lambda\) is bounded, let \(\|x\| \leq 1\) in $X$, now
 \[ |\Lambda(x)| = |\lim_{n\to\infty} \Lambda_n(x)| 
    \leq = \lim_{n\to\infty} \|\Lambda_n\| < \infty\]
 and the limit is independent of $x$.

 Therefore \(X^*\) \emph{is} a banach space.
 
 \itemch{b} Let \(x\in X\). Since
 \[(z_1f_1+z_2f_2)(x) = z_1f_1(x) + z_2f_2(x)
   \qquad f_1,f_2\in X^*\;z_1,z_2\in \C\]
 clearly \(f\to f(x)\) is a linear functional on \(X^*\).
 Define \(x^{**}\in X^{**}\) by \(x^{**}(f) = f(x)\) for all \(f\in X^*\).
 If \(x=0\) then \(\|x^**\|_{X^{**}} = 0\) trivially, for \(x\neq 0\)
 \begin{eqnarray}
 \|x^{**}\|_{x^{**}} 
 &=& \sup\{|f(x)|: \|f\|_{X^*} = 1,\;f\in X^*\} \notag \\
 &=& \sup\bigl\{|f(x)|: \sup\{ |f(w)|: \|w\|=1,\; f\in X^*,\;w\in X\}=1\bigr\} 
     \notag \\
 &\leq& \sup\bigl\{|f(x)|: \sup\{ |f(x/\|x\|)|: f\in X^*\}=1\bigr\} 
        \label{eq:5.8:sup1} \\
 &\leq& \|x\| \label{eq:5.8:sup2}.
 \end{eqnarray}
 We now show the reverese inequality. Define a functional \(f_x\)
 on a 1-dimensional subspace of $X$ generated by $x$, as \(f_x(zx/\|x\|) = z\).
 Clearly \(\|f_x\| = f(x)/\|x\| = 1\)
 and applying Hahn-Banach theorem, we can extend
 \(f_x\) to a functional $f$ on whole $X$, such that 
 \(\|f\|_{X^*} = \|f_x\| = 1\).
 Using it in the supremum of \eqref{eq:5.8:sup1} and \eqref{eq:5.8:sup2} 
 we get an equality and so \(\|x^{**}\|_{X^{**}} = \|x\|\).
 
 \itemch{c}
 By negation, assume \mset{x_n} is unbounded.
 Let $W$ be the subspace of $X$ generated by \mset{x_n}.
 Two cases:

 \paragraph{Case 1.} Assume \(\dim(W) < \infty\) and \seqn{v} its base.
 Thus we have a unique representation 
 \(x = \sum_{i=j}^n a_j(x) v_j\), where \(a_j\in X^*\).
 Since \mset{x_n} is unbounded, there exist some $j$ such that
 \(\{a_j(x_i)\}_{i=1}^\infty\) is not bounded.

 \paragraph{Case 2.} Assume \(\dim(W)\infty\). 
 Let \(W_n\) be the subspace of $W$ generated by \seqn{x}.
 Clearly \(W = \cup_{n=1}^\infty W_n\).
 For convenience, we drop from \seqn{x} any \(x_j\) such that 
 \(x_j \in X_{j-1}\); note that zeros are dropped. 
 This case assumtion, guarantees that the sequence is still infinite.
 The functional $f$ to be constructed, 
 will furnish a contradiction, namely \mset{f(x_n)} unbounded for 
 the original, trivially, as well.

 For each \(w\in W\) there is a unique (finite) representation
 \[ w = \sum_{j=1}^{M(w)} a_j(w) (x_j/\|x_j\|).  \]
 Note that \(a_n(x_n) = \|x_n\|\).
 We will define a sequence of functionals \(f_n: W_n \to \C\).
 \begin{eqnarray*}
 % u_n &=& \overline{a_n(x_n)}/|x_n| \\
 f_n(w) &=& f\left(\sum_{j=1}^{M_w} a_j(w) x_j\right) 
       \eqdef \sum_{j=1}^{M_w} u_j a_j(w) \qquad w \in W_{M_w}
 \end{eqnarray*}
 By construction, 
 \begin{eqnarray*}
  % |u_n| &\in& \{0,1\} \\
  f_n(x_n) &=& \sum_{j=1}^n a_j(x_j) = \|x_n\| \\
  f_n &=& {f_m}_{|W_n}  \qquad \textrm{if}\, n\leq m
 \end{eqnarray*}
 Thus we can define the functional \(f = \cup f_n\), 
 for which \(f(x_n) = \|x_n\|\) for all $n$.
\end{itemize}


%%%%%%%%%%%%%% 9
\begin{excopy}
Let \(c_0\), \(\ell^1\), and \(\ell^\infty\) be Banach spaces consisting
of all complex sequences \(x=\{\xi_i\}\), \(\mbox{i=1,2,3,\ldots,}\) 
defined as follows:
\begin{alignat*}{3}
x&\in\ell^1 & &\quad\textrm{if and only if} 
   \quad && \|x\|_1 = \sum|\xi_i|<\infty.\\
x&\in\ell^\infty && \quad\textrm{if and only if} 
   \quad && \|x\|_\infty = \sup |\xi_i|<\infty.
\end{alignat*}
\(c_0\) is the subspace of \(\ell^\infty\) consisting of all \(x\in\ell^\infty\)
for which \(\xi_i \to 0\) as \(i\to\infty\).

Prove the following statements.
\begin{itemize}
 \itemch{a} If \(y=\{\eta_i\}\in \ell^1\) and \(\Lambda x = \sum \xi_i\eta_i\)
            for every \(x\in c_0\), then \(\Lambda\)
            is a~bounded linear functional on \(c_0\),
            and \(\|\Lambda\| = \|y\|_1\).
            Moreover, every \(\Lambda \in (c_0)^*\) is obtain in this way.
            In brief, \((c_0)^* = \ell^1\).

            (More precisely, these two space are not equal, the preceding
            statement, exhibits as isometric vector space isomorphism
            between them.
 \itemch{b} In the same sense, \((\ell^1)^* = \ell^\infty\).
 \itemch{c} Every \(y\in \ell^1\), induces a bounded linear functional on
            \(\ell^\infty\), as in \ich{a}.
            However, this does \emph{not} give all of \((\ell^\infty)^*\),
            since \((\ell^\infty)^*\) contains nontrivial functionals that
            vanish on all of \(c_0\).
 \itemch{d} \(c_0\) and \(\ell^1\) are separable but \(\ell^\infty\) is not.
\end{itemize}
\end{excopy}

\begin{itemize}
\itemch{a}
Clearly, \(\Lambda\) is linear.
Estimate:
\[ |\Lambda x| 
   = \left|\sum \eta_i\xi_i\right|
   \leq \sum |\eta_i \xi_i|
   \leq \left(\sum |\eta_i|\right) \left(\sup |x_i|\right) 
   = \|y\|_1 \cdot \|x\|\_infty.\]
Thus \(\|\Lambda\| \leq \|y\|_1\). 

For the reverse inequality, we may assume \(y\neq 0\).
let \(\|y\|_1 > \epsilon>0\),
let \(N>0\) be such that \(\sum_{i>N} |\eta_i| < \epsilon\).
% and let $M$ be the number of nonzero components of $y$ upto $N$;
% formally: \(M = |\{i\in\N: 1\leq i \leq N\,\wedge\, \eta_i=0\}|\).
In the next definitions and derivations we freely use \(0/0 = 0\).
Put \(u_i = \overline{\eta_i}/|\eta_i|\).
and define \(x=(\xi_i)_i\) by 
\[
   \xi_i = \left\{\begin{array}{ll}
                   u_i   & \qquad \textrm{if}\;  1\leq i\leq N \\
                   0     & \qquad \textrm{if}\;  i> N
                  \end{array}\right.\]

Now 
\[\Lambda x 
 = \sum \xi_i\eta_i
 = \sum_{i=1}^N \overline{\eta_i}\eta_i \,/\, |\eta_i|
 = \sum_{i=1}^N |\eta_i|
 > \|y\|_1 - \epsilon.\]
Since \(\epsilon\) can be arbitrarily small, \(\|\Lambda\|\geq \|y\|_1\).

Now Pick arbitrary \(f\in c_0^*\). 
Define \mset{e_i} in \(c_0\) as a function \(\N\to \C\)
by \(e_i(n) = delta_{in}\) that is \(e_i(i)=1\) and zero otherwise.
Define \(x=(\xi_i)_{i=1}^\infty\) by \(\xi_i = f(e_i)\)
and \(\Lambda\) as above. 
For each \(j\in \N\), we have
\[\Lambda e_j = \sum_{i=1}^\infty \xi_i e_j(i) = \xi_j.\]
Thus \(\Lambda\) and $f$ agree on the subspace $S$ generated by \mset{e_i}.
It is easy to see that \(c_0 = \overline{S}\) and thus \(\Lambda = f\)
that is every functional on \(c_0\) is given by the above ``producs-sum'' form.

\itemch{b}
 Let \(x=\{\xi_i\}\in \ell^1\) 
 \(y=\{\eta_i\}\in \ell^\infty\) and \(\Lambda x = \sum \eta_i\xi_i\).
 for 
 \[ |\Lambda x| \leq \sum |\eta_i\xi_i| 
    \leq \left(\sup \eta_i\right)\left( \sum |\eta_i\xi_i| \right)
    = \|y\|_\infty \cdot \|x\|_1.\]
 Hence \(\|\Lambda\| \leq \|y\|_\infty\).
 Given \(\epsilon>0\) pick some $J$ such that \(|\eta_j| \geq \|y\|-\epsilon\),
 and let \(e_j\in l_1\) 
 such that \(e_j(n) = 1\) if \(n\neq j\) and \(e_j(j)=1\).
 Now \(\Lambda e_j = \eta_j\) and so \(\|\Lambda\| \geq \|y\|-\epsilon\).
 Since \(\epsilon\) may be arbitrarily small, \(\|\Lambda\| \geq \|y\|_\infty\),
 hence \(\|\Lambda\| = \|y\|_\infty\).

 Now let \(f\in (\ell^1)^*\). Define \(e_j\) as before for all $j$, 
 let \(\eta_j= f(e_j)\) and let \(y=\{\eta_i\}\).
 Since \(\|e_j\|=1\) for all $J$, we have \(|\eta_j| \leq \|f\|\)
 and so \(y\in \ell^\infty\).

\itemch{c}
 Construct a functional \(f\in (\ell^\infty)^*\) as follows.
 It vanishes \(f(x)=0\) for all \(x\in c_0\). 
 Pick \(\mathbf{u}=(u_i)_{i\in\N}\)
 such that \(u_i=1\) for all \(i\in\N\). 
 Clearly \(u\in \ell^\infty\setminus c_0\).
 Now define \(f(u)=1\) and extend $f$ to \(\ell^\infty\) by Hahn-Banach theorem.
 Clearly $f$ is not in the image of \(\ell^1\) embedding in 
 \((\ell^\infty)^{**}\) 
 
\itemch{d}
 The extension field \(\Q(i)\in \C\) of the rational is actually
 \[\Q(i) = \{a+bi:\;a,b\in\Q\}.\]
 Clearly \(|\Q(i)|=\aleph_0\).
 Define the set $E$ of inifinite sequences of \(\Q(i)\) which 
 have only finite nonzero components. Formally:
 \[ E \eqdef \left\{x\in \bigl(Q(i)\bigr)^\N:\; 
                     \exists M,\; \forall n>m\,\Rightarrow x(n)=0\right\}.\]
 Now \(|E|=\aleph_0\) and $E$ is dense in \(c_0\) and \(\ell^1\).
 
 {\small
 \emph{Sketch:} Pick \(v\in c_0\) or \(v \in \ell^1\) and \(\epsilon>0\).
 \(c_0\)-case: \(\exists M, \forall n>m,\, |v_n|<\epsilon/2\). 
 \(\ell^1\)-case: \(\exists M, \sum{j=1}^M |v_j| <\epsilon/2\). 
 Then \(\epsilon/2\)-approximate $v$ trimmed above $M$, by some \(x\in E\).
 Now \(\|x-v\|<\epsilon/2+\epsilon/2=\epsilon\).
 }

 Define a subset
 \[V \eqdef  \{0,1\}^\N \subset \ell^\infty \subset \C^\N.\]
 Note that \(\|x-y\|\in\{0,1\}\) for all \(x,y\in V\).
 Clearly \(|V| = 2^{\aleph_0} > \aleph_0\).
 If by negation \(\ell^\infty\) is separable, let $D$ be a countable dense
 set in  \(\ell^\infty\). 
 Define a mapping \(\nu:V\to D\) by picking 
 for each \(v\in V\) an \(x\in D\) such that \(\|x-v\|<1/2\).
 Existence of \(\nu(v)\) is guaranteed by density of $D$.
 This map must be one-to-one, since if for some \(x\in D\)
 there exist \(v_1,v_2\in V\) such that \(\|x-v_i\|<1/2\) for \(i=1,2\)
 then \(\|v_1-v_2\|<1\) but then \(v_1=v_2\).
 But now a cardinality \(|V|<|D|\) contradiction.
\end{itemize}

%%%%%%%%%%%%%% 10
\begin{excopy}
If \(\sum \alpha_i\xi_i\) converges for every sequence \((\xi_i)\) such that
\(\xi_i \to 0\) as \(i\to\infty\), prove that \(\sum|\alpha_i| < \infty\).
\end{excopy}

This is a consequence of Exercise~9\ich{a} above.

%%%%%%%%%%%%%% 11
\begin{excopy}
For \(0<\alpha\leq 1\),
\index{Lip@\(\Lip\)}
let \(\Lip\alpha\) denote the space of all complex
functions $f$ on \([a,b]\) for which
\[M_f = \sup_{s\neq t} \frac{|f(s)-f(t)|}{|s-t|^\alpha} < \infty.\]
Prove  that \(\Lip\alpha\) is a Banach space, if \(\|f\| = |f(a)| + M_f\);
also if
\[\|f\| = M_f + \sup_x |f(x)|.\]
(The members of \(\Lip\alpha\) are said to satisfy
\index{Lipschitz condition}
a~\emph{Lipschitz condition} of order \(\alpha\).)
\end{excopy}

We first show that \(\|\cdot\|\) is indeed a norm.
\paragraph{Scalar Multiplication}.
Let \(z\in\C\). For all \(f\in \Lip\alpha\)
\[
M_{zf} 
= \sup_{s\neq t} \frac{|(zf)(s)-(zf)(t)|}{|s-t|^\alpha} 
= |z|\sup_{s\neq t} \frac{|f(s)-f(t)|}{|s-t|^\alpha} 
|z| M_f.\]
Hence
\[
\|zf\| = |(zf)(a)| + M_{zf} = |z|\cdot|f(a)| + |z|M_f = |z|\cdot\|f\|.
\]

\paragraph{Sub additivity}. For all \(f,g\in \Lip\alpha\)
\begin{eqnarray*}
\|f+g\| 
&=& |(f+g)(a)| + \sup_{s\neq t} \frac{|(f+g)(s)-(f+g)(t)|}{|s-t|^\alpha} \\
&=& |f(a) + g(a)| + 
  \sup_{s\neq t} \frac{\left|\bigl(f(s)-f(t)\bigr) + 
                             \bigl(g(s)-g(t)\bigr)\right|)}{|s-t|^\alpha} \\
&\leq& |f(a)| + |g(a)| + 
     + \sup_{s\neq t} \frac{|(f(s)-f(t)|}{|s-t|^\alpha} 
     + \sup_{s\neq t} \frac{|(g(s)-g(t)|}{|s-t|^\alpha} \\
&=& \|f\|+\|g\|.
\end{eqnarray*}

Clearly if \(\|f\|=0\) then \(f(a)=0\) and \(f'(t)=0\) for all \(t\in [a,b]\)
and so \(f=0\).

To show that \(\Lip\alpha\) is a Banach space it is left to show completeness.
Let \mset{f_n} be a Cauchy sequence in \(\Lip\alpha\).
Since \(|f_m(a)-f_n(a)| \leq \|f_m - f_n\|\), 
the evaluations \(\{f_n(a)\}\) is a Cauchy sequence in \(\C\).
Similarly for \(t\in(a,b]\)
\[
\frac{|(f_m - f_n)(t) - (f_m - f_n)(a)|}{|s-t|^\alpha} \leq \|f_m - f_n\|,\]
equivalently
\(|(f_m - f_n)(t) - (f_m - f_n)(a)| \leq |s-t|^\alpha \|f_m - f_n\|\)
hence \mset{f_n(t)} is a Cauchy sequence in \(\C\) as well.
Therefore we can define \(f(t) = \lim_{n\to\infty} f_n(t)\).
If by negation \(M_f = \infty\) then for any \(0<M<\infty\) we can find
\(s,t\in[a,b]\) such that \(s\neq t\) and \(|f(s)-f(t)|\geq |s-t|^\alpha M\).
But then we can find some $n$ such that \(f_n\) is sufficiently close 
to $f$ in \(\{s,t\}\) so \(|f_n(s)-f_n(t)|\geq |s-t|^\alpha (M-\epsilon)\).
But then \mset{M_{f_n}} is not bounded, which is a contradiction.


%%%%%%%%%%%%%% 12
\begin{excopy}
Let $K$ be triangle (two-dimensional figure) in the plane,
let $H$ be the set consisting of the vertices of $K$, of the form
\[ f(x,y) = \alpha x + \beta y + \gamma
   \qquad (\alpha, \beta,\, \textrm{and}\, \gamma\; \textrm{real}). \]
Show that to each \((x_0,y_0)\in K\) there corresponds
a unique measure \(\mu\) on $H$ such that
\[ f(x_0,y_0) = \int_H f\,d\mu. \]
(Compare Sec.~5.22.)

Replace $K$ by a square, let $H$ again be the set of its vertices, and let $A$
be as above.
Show that to each point of $K$ there still corresponds a measure on $H$,
with the above property, but that uniqueness is now lost.

Can you extrapolate to a more general theorem?
(Think of other figures, higher dimensional spaces.)
\end{excopy}

Denote the vertices $H$ of the triangle $K$ 
by \(v_i = (x_i,y_i\) for \(i=1,2,3\).
Since any triangle is convex, 
for any \((x_0,y_0)\in K\) there is a convex combination
\[(x_0,y_0) = \sum_{j=1}^3 w_j(x_j,y_j)\]
where the weights \(w_j\in[0,1]\) and \(\sum_{j=1}^3 w_j = 1\).
Put \(\mu(\{v_j\}) = w_j\). 
Now for any \(f(x,y) = \alpha x + \beta y + \gamma\) 
we compute:
\[
f(x_0,y_0)
 = \alpha x_0 + \beta y_0 + \gamma 
 = \sum_{j=1}^3  w_j (\alpha x_j + \beta y_j + \gamma) 
 = \sum_{j=1}^3 f(x_j,y_j) w_j 
 = \int_H^f\,d\mu.
\]

\paragraph{Uniqueness.} Let \(\mu_1\) and \(\mu_2\)
satsisfying the above condition for any $f$ of the above (affine) form.
\Wlogy, we may assume \(\mu_1(\{v_1\}) \neq \mu_2(\{v_1\})\).
We solve the following system of linear equations:
\begin{eqnarray*}
\alpha x_1 + \beta y_1 + \gamma &=& 1 \\
\alpha x_2 + \beta y_2 + \gamma &=& 0 \\
\alpha x_3 + \beta y_3 + \gamma &=& 0
\end{eqnarray*}
Where \(\alpha,\beta,\gamma\in\R\) are the unknowns.
The vertices are not co-linear, and there must exist a solution.
But now,
\[\int_H f\,d\mu_1 = \mu_1(\{v_1\} \neq \mu_2(\{v_1\}) = \int_H f\,d\mu_2\]
is a contradiction.

When $K$ is a squere we do the same, but now there are 
inifnite number of convex combinations of the verices.
Thus the representing measure is not unique in the interior of $K$.

As for higher dimension. In \(\R^n\) for \(n\geq 2\),
any  set $H$ of \(n-1\) points which do not lie in the same hyperplane
generate a compact convex hull $K$.
The condition on $H$ as equivalent of saying that they do not 
simultanously satisfy a single non trivial linear equation in \(\R^n\).
The ``extreme'' vertices of of $K$ are $H$.
Now, for any \(\mathbf{x}\in K\) there exist a unique measure 
\(\mu_{\mathbf{x}}\) on $H$ such that
\[f(\mathbf{x}) 
  = \int_H f\,d\mu_{\mathbf{x}} 
  = \sum_{v\in H} \mu_{\mathbf{x}}(v) \cdot f(v)\,.\]

%%%%%%%%%%%%%% 13
\begin{excopy}
Let \(\{f_n\}\) be a sequence of continuous complex functions on a (nonempty)
complete metric space $X$, such that \(f(x) = \lim f_n(x)\) exists
(as a complex number) for every \(x\in X\).
\begin{itemize}
 \itemch{a} Prove that there is an open set \(V\neq \emptyset\) and a number
            \(M<\infty\) such that \(|f_n(x)| < M\) for all \(x\in V\)
            and \(n=1,2,3,\ldots\).
 \itemch{b} If \(\epsilon>0\), prove that there is an open set \(V\neq\emptyset\)
            and an integer $N$ such that \(|f(x) - f_n(x)| \leq \epsilon\)
            if \(x\in V\) and \(n\geq N\).

\end{itemize}
\emph{Hint for \ich{b}}: For \(N=1,2,3,\ldots\), put
\[ A_N = \{x: |f_m(x)-f_n(x)|\leq \epsilon
\;\textrm{if}\; m\geq N\;\textrm{and}\; n\geq N\}.\]
Since \(X=\cup A_N\), some \(A_N\) has a nonempty interior.
\end{excopy}


\begin{itemize}
\itemch{a}
Let 
\[ B_M \eqdef  \{x\in X: \forall n\in\N,\, |f_n(x)|\leq M\} 
  = \bigcup_{n\in\N} \{x\in X: |f_n(x)|\leq M\}.\]
Clearly \(B_M\) are closed. For any \(x\in X\), 
the sequence \mset{f_n(x)} hence \mset{|f_n(x)|} is bounded.
Therefore, \(X=\cup_{M\in\N} B_M\) and
\index{Baire!category theorem}
by Baire's category theorem~5.6 since $X$ is complete, some \(B_M\) 
must have non empty interior $V$.

\itemch{b}
The sets 
\[ A_N = \{x: |f_m(x)-f_n(x)|\leq \epsilon\;\textrm{if}\; m,n\geq N\}\]
are closed. Similarly, \(X=\cup_{N\in\N} A_N\)
and so again by Baire's theorem, some \(A_N\) has non empty interior $V$.
\end{itemize}

%%%%%%%%%%%%%% 14
\begin{excopy}
Let $C$ be the space of all real continuous functions on \(I=[0,1]\)
with the supremum norm. Let $X$, be the subset of $C$ consisting of those
$f$ for which there exists a \(t\in I\) such that \(|f(s)-f(t)| \leq n|s-t|\)
for all \(s\in I\). Fix $n$ and prove that each open set in $C$ contains
an open set which does not intersect \(X_n\). (Each \(f\in C\) can be uniformly
approximated by a zigzag function $g$ with very large slopes and if
\(\|g-h\|\) is small, \(h\notin X_n\).)
Show that this implies the existence of a dense \(G_\delta\) in $C$ which
consists entirely of nowhere differentiable functions.
\end{excopy}


Fix $n$ and take a base open set in $C$ by picking \(f\in C\) and 
\(\epsilon>0\), and setting
\[V \eqdef \{\phi\in C: \|f-\phi\|_\infty < \epsilon/3\}.\]
In $I$ every continuous functions is uniformly continuous.
Hence there exists \(\delta>0\) such that \(|f(t)-f(s)|<\epsilon\)
whenever \(|s-t|<\delta\). 
Let 
\[s = \lceil 1\bigm/\,\min\bigl(\delta, \epsilon/(3(n+1))\bigr)\rceil\]
and split $I$ by \(x_i = i/(s+1)\) for \(0\leq i \leq s+1\).
Note that \(x_i - x_{i-1} < \delta/(n+1)\).
We will now construct \(g\in V\).
Intuitively, it will have on each subsegment \([x_{i-1},x_i]\) 
a slope of \(\pm(n+1)\), and it will increase or decrease so it follows $f$.
Formally, we define $g$ by induction on \([x_{i-1},x_i]\) for \(0< i \leq s+1\).
On the first subsegment \([x_0,x_1]\) let
\(g(x) = f(0) + (n+1)x\). Assume that $g$ is defined
on \([x_0,x_{i-1}]\), and by induction on \([x_{i-1}, x_i]\) by
\[g(x) = g(x_{i-1}) + \sigma (n+1)(x - x_{i-1})\]
where \(\sigma=1\) if \(g(x_{i-1}) < f(x{i-1})\) and 
\(\sigma= -1\) otherwise. It is easy to see that \(\|g-f\|_\infty < \epsilon/3\)
and that \(g\notin X_n\).
We now need a trivial lemma
\begin{llem}
Let \(f:[x_0,x_1]\to \R\) be defined by \(f(x) = ax+b\). 
For any \(\epsilon>0\)
such that \(\epsilon<|a|\) there exists \(\delta>0\) such that 
if \(h:[x_0,x_1]\to \R\) and \(\|f-h\|_\infty < \delta\)
then 
\begin{equation} \label{eq:ex5.14:lem}
 \max\left( \frac{h(t) - h(x_0)}{t-x_0}, 
              \frac{h(x_1) - h(t)}{x_1 - t} \right) < |a|-\epsilon. 
\end{equation}
for all \(t\in(x_0,x_1)\) we 
\end{llem}
\begin{thmproof}
\iffalse
By defining  \(\tilde{f}:[0,x_1-x_0]\to \R\)
as \(\tilde{f}(x) = |a|x\)
we see that we can assume that \(f(0)=0\) and \(a>0\).
Since we can similarly convert a $h$ 
and get the same ratios of \eqref{eq:ex5.14:lem}.
\fi
\Wlogy, we may assume \(a>0\), the negative case is analogous.
Let \(\delta = \epsilon(x_1-x_0)/4\), and assume \(\|f-h\|_\infty < \delta\).
For any \(t\in (x_0,x_1)\), there are two cases

\paragraph{High Case}: \((x_0+x_1)/2 \leq t < x_1\).
Let us estimate,
\begin{eqnarray*}
\frac{h(t) - h(x_0)}{t-x_0}
&=& \biggl( f(t) + (h(t)-f(t)) - \bigl(f(x_0) + (h(x_0)-f(x_0))\bigr)\biggr)
    \bigm /\,(t - x_0) \\
&\geq& \bigl(f(t) - f(x_0)\bigr) / (t - x_0) 
        - 2\delta / \bigl((x_1 - x_0)/2\bigr) \\
&=& a - 4\delta/(x_1 - x_0) \\
&=& a - \epsilon.
\end{eqnarray*}

\paragraph{Low Case}: \(x_0 < t \leq (x_0+x_1)/2\), similar derivation
as the high-case, but estimating the slope against \((x_0,h(x_0))\).
\end{thmproof}

From this lemma, back to this exercise, we see that we can pick 
some \(\delta>0\) such that for any \(h:I\to\R\) such that
\(\|g-h\|_\infty\), that is a neighborhood of $g$ the zigzag function, 
\(h\notin X_n\). 
Therefore, \[V_n \eqdef \inter{\left(C(I)\setminus X_n\right)}\] is dense.
By the corollary of theorem~5.6 in \cite{RudinRCA87}, 
\(\cap_{n\in\N} V_n\) is a dense \(G_\delta\) set, 
(and in particular non empty) that consists of nowhere differentiable functions.


%%%%%%%%%%%%%% 15
\begin{excopy}Let \(A=(a_{ij})\) be an infinite matrix with complex entries,
wher \(i,j=0,1,2,\ldots\).
$A$ associateswith each
sequence \(\{s_i\}\)
a sequence \(\{\sigma_i\}\), defined by
\[ \sigma_i = \sum_{j=0}^\infty a_{ij} s_j \qquad (i=1,2,3,\ldots), \]
provided that these series converge.

Prove that $A$ transforms every convergent sequence
\(\{s_i\}\)
to a sequence \(\{\sigma_i\}\) which converges to the same limit
if and only if the following conditions are satisfied:
\begin{alignat*}{2}
 \ich{a} & \qquad &\lim_{i\to\infty} a_{ij} &= 0\qquad \textrm{for each}\; j. \\
 \ich{b} & \qquad & \sup_i \sum_{j=0}^\infty |a_{ij}| &< \infty \\
 \ich{c} & \qquad & \lim_{i\to\infty} \sum_{j=0}^\infty a_{ij} &= 1.
\end{alignat*}
The process of passing from
\(\{s_i\}\) to \(\{\sigma_i\}\) is called
\index{summability method}
a~\emph{summability method}. Two examples are
\begin{eqnarray*}
a_{ij} &=&
   \left\{\begin{array}{ll}
          \frac{1}{i+1} & \quad \textrm{if}\; 0\leq j \leq i. \\
          0             & \quad \textrm{if}\; i < j,
          \end{array}\right. \\
\textrm{and}\qquad a_{ij} &=& (1-r_i)r_i^j, \qquad  0<r_i<1,\quad r_i\to 1.
\end{eqnarray*}
Prove that each of these also transforms some divergent sequence \(\{s_i\}\)
(even some unbounded ones) to a convergent sequences \(\{\sigma_i\}\).
\end{excopy}

Following the above notations, we denote the transform as \(\sigma = A(s)\).

\paragraph{Summability implies limits.}
Assume $A$ transforms convergent sequences to convergent sequences.

Assume by negation \ich{a} does not hold for some $j$.
Pick the sequences \mset{s_j}, such that \(s_{i} = \delta_{ij}\).
Clearly \(\lim_{i\to\infty}s_i = 0\), 
but \(\sigma_i = a_{ij}\) which does not converge to $0$, 
and so by contradiction, \ich{a} holds.

We will now show \ich{b}. We first show that 
\begin{equation} \label{eq:ex5.15:finsum}
\sum_{j=0}^\infty |a_{rj}| < \infty.
\end{equation}
for each $r$. By negation, assume \eqref{eq:ex5.15:finsum} does not hold 
for some fixed $r$.
We will construct, in appending steps, 
a~sequence \mset{s_j} such that converges to zero but
\((A(s))_r = \infty\).
We define a sequence of blocks of indices such that 
each block if $A$'s $r$th row is ``not too small''.
Formally, let \(M_0 = 0\) and let \(M_k<\infty\) be the minimal integer
such that \[\sum_{j=M_{k-1}}^{M_k} |a_{rj}| > 1.\]
Clearly \mset{M_k} is an infinitely increasing sequnce.
For \(k\in\N\) let 
\[s_j = e^{i\theta_j}/k \qquad 
 \textrm{where}\;
 M_{k-1} \leq j < M_k
 \;\textrm{and}\;
 \theta_j = -\Arg(a_{rj}).\]
Note that \(\lim_{j\to\infty}s_j = 0\) and \(a_{rj}s_j \geq 0\).
Compute the $r$ component of \(A(s)\):
\begin{eqnarray*}
\sigma_r 
&=& \sum_{j=0}^\infty a_{rj}s_j 
 = \sum_{k=1}^\infty \sum_{j=M_{k-1}}^{M_k} a_{rj}s_j \\
&=& \sum_{k=1}^\infty 
      \left(\sum_{j=M_{k-1}}^{M_k} a_{rj}e^{i\theta_j}\right) \bigm/\,k 
 = \sum_{k=1}^\infty \left(\sum_{j=M_{k-1}}^{M_k} |a_{rj}|\right) \bigm/\,k \\
&\geq& \sum_{k=1}^\infty 1/k \\
&=& \infty.
\end{eqnarray*}
Thus \(\sigma=A(s)\) is not a valid sequence, (limit cannot be defined at all
and thus \eqref{eq:ex5.15:finsum} is true, and we can denote
\[S_r \eqdef \sum{j=0}^\infty |a_{rj}.\]


Now assume by negation \ich{b} does not hold.
We will again construct, in appending steps, 
a~sequence \mset{s_j} that will provide a contradiction.
We will also build increasing sequences, 
\mset{b_k} of column blocks, and \mset{r_k} of rows, such that
\begin{eqnarray}
\sum_{j=0}^{b_k-1} |a_{r_k j}| &\leq& 1 \label{eq:ex5.15:head} \\
\sum_{j=b_k}^{b_{k+1}-1} |a_{r_k j}| &\geq& 2^k \label{eq:ex5.15:mid} \\
\sum_{j=b_{k+1}}^\infty |a_{r_k j}| &\leq& 1 \label{eq:ex5.15:tail}
\end{eqnarray}
for each \(k\in\N\).
Let \(b_0=0\). 
Assume \mset{s_j} is defined for all \(j<b_{k'}\)
By \ich{a}, there exists some \(\rho\) such that 
\(|a_{rj}| < 1/(b_{k'}+1)\)
for all \(r\geq \rho\) and all \(j<b_{k'}\).
To ensure our next row pick, let \(U_k = \max_{r\leq\rho}S_r\).
By our negation hypothesis, there exists \(r_k\) such that 
\[S_{r_k} \geq \max(k^2+2,U_k).\] 
Clearly \(r_k>\rho\), and we can find \(b_{k+1}\) such that 
\[\sum_{j=b_{k+1}}^\infty |a_{rj}| < 1.\]
Now define \(s_j\) for \(b_k \leq j < b_{k+1}\) by
\[s_j = e^{i\theta_j}/k \qquad  \textrm{where}\; \theta_j = -\Arg(a_{r_k j}).\]
By induction we complete the definitions of \mset{s_j} and 
the supporting sequences \mset{b_k} and \mset{r_k}.
Clearly \(\lim_{j\to\infty}s_j = 0\), but
\begin{eqnarray*}
|\lim_{r\to\infty}\sigma_r|
&=& \lim_{r\to\infty} |\sigma_r| 
= \lim_{r\to\infty} \left| \sum_{j=0}^\infty a_{rj}s_j \right| 
= \lim_{k\to\infty} \left| \sum_{j=0}^\infty a_{r_k j}s_j \right| \\
&\geq& \lim_{k\to\infty} 
       \left(
          \left| \sum_{j=b_k}^{b_{k+1}-1} a_{r_k j}s_j \right| 
         - \left| \sum_{j=0}^{b_k-1} a_{r_k j}s_j \right| 
         - \left| \sum_{j=b_{k+1}}^\infty a_{r_k j}s_j \right|
       \right) \\
&=& \lim_{k\to\infty} 
       \left(
          \left| \sum_{j=b_k}^{b_{k+1}-1} |a_{r_k j}| \right| / k
         - \left| \sum_{j=0}^{b_k-1} a_{r_k j}s_j \right| 
         - \left| \sum_{j=b_{k+1}}^\infty a_{r_k j}s_j \right|
       \right) \\
&\geq& \lim_{k\to\infty} 
       \left(
          (k^2+2k) / k
         - \sum_{j=0}^{b_k-1} |a_{r_k j}|
         - \sum_{j=b_{k+1}}^\infty |a_{r_k j}|
       \right) \\
&\geq& \lim_{k\to\infty} k+2 - 1 - 1 \\
&=& \infty.
\end{eqnarray*}
This contradicts the summability, hence \ich{b} is true.

Consider the constant sequences \(s_j=1\) for all \(j\in\N\).
It has \(\lim_{j\to\infty} s_j = 1\) which implies
 \(\lim_{j\to\infty} \sigma_j = 1\) which is actually equivalent
to \ich{c}.

\paragraph{Limits implies summability.}
Conversely, assume conditions \ich{a}, \ich{b}, \ich{c} hold.

\emph{Constant:} If \(s_j=c\) for all \(j\in\N\), 
then by \ich{c} we get \(\lim_{j\to\infty}\sigma_j = c\).
\emph{Vanishing:} Assume \(\lim_{j\to\infty s_j} s_j = 0\).
Pick arbitrary \(\epsilon>0\). 
Let $J$ be such that \(|s_j| < \epsilon/3\) whenever \(j\geq J\).
Put \(h = \max_{1\leq j \leq J} |s_j| + 1\).
By \ich{a} there exists \(\rho_1\) such that \(|a_rj| < \epsilon/(Jh+1)\)
for all \(r\geq\rho_1\) and \(0\leq j \leq J\).
Denote \(T_r = \sum{j=0}^\infty a_{rj}\).
By \ich{c}, pick \(\rho_2\) such that \(|T_r - 1| < \epsilon\)
whenever \(r\geq \rho_2\). 
Note that in this case, 
\[
\left|\left(T_r - \sum_{j=0}^J a_{rj}\right) - 1\right| 
\leq |T_r - 1| + J\epsilon/(Jh+1)
< 2\epsilon.\]
Let \(\rho = \max(\rho_1,\rho_2)\), now for \(r\geq \rho\)
\begin{eqnarray*}
|\sigma_r|
&=& \left|\sum_{j=0}^\infty a_{rj}s_j\right| 
 = \left|\sum_{j=0}^J a_{rj}s_j
        + \sum_{j=J+1}^\infty a_{rj}s_j\right| \\
&\leq&  \left|\sum_{j=0}^J a_{rj}s_j\right| 
      + \left|\sum_{j=J+1}^\infty a_{rj}s_j\right| \\
&\leq&  h\left|\sum_{j=0}^J a_{rj}\right| 
      + \epsilon\left|\sum_{j=J+1}^\infty a_{rj}\right| \\
&\leq&  hJ\epsilon/(Jh+1) + \epsilon(1+2\epsilon) \\
&<& 2\epsilon(\epsilon+1).
\end{eqnarray*}
Hence \(\lim_{r\to\infty} \sigma_r = 0\).

Now for arbitrary converging sequence, let \(\lambda = \lim_{j\to\infty} s_j\).
By the above two case, we have
\begin{eqnarray*}
\lim_{r\to\infty} \sigma_r 
&=& \lim_{r\to\infty} \sum_{j=0^\infty} a_{rj}s_j \\
&=& \lim_{r\to\infty} \sum_{j=0^\infty} a_{rj}(s_j - \lambda + \lambda) \\
&=& \left(\lim_{r\to\infty} \sum_{j=0^\infty} a_{rj}(s_j - \lambda)\right) + 
    \left(\lim_{r\to\infty} \sum_{j=0^\infty} a_{rj} \lambda\right) \\
&=& 0 + \lambda.
\end{eqnarray*}
Thus $A$ satisfies the summability condition.

\paragraph{Specific transformations.}
When 
\begin{eqnarray*}
a_{rj} &=&
   \left\{\begin{array}{ll}
          \frac{1}{r+1} & \quad \textrm{if}\; 0\leq j \leq r. \\
          0             & \quad \textrm{if}\; r < j,
          \end{array}\right.
\end{eqnarray*}
we observe the sequence \mset{s_j} defined by 
% \(s_j = (-1)^j\), that is \((+1,-1,+1,-1,\ldots)\).
% that is \((+1,-1,+1,-1,\ldots)\).
\(s_j = (-1)^j\sqrt{\lfloor j/2\rfloor}\), 
that is \((0,0,+1,-1,+\sqrt{2},-\sqrt{2},\ldots)\).
Clearly does not converge and is unbounded, but since
we observe that 
\[\sum_{j=0}^r s_j 
  = \left\{\begin{array}{ll}
          0 & \qquad r = 1 \bmod 2 \\
          \sqrt{r/2} & \qquad r = 0 \bmod 2 
          \end{array}\right.\]
we see that 
\begin{eqnarray*}
\sigma_r 
&=& \sum_{j=0}^r a_{rj}s_j
 = \left(\sum_{j=0}^r s_j\right)/(r+1) \\
&=& \left\{\begin{array}{ll}
          0 & \qquad r = 1 \bmod 2 \\
          \sqrt{r}/(r+1) & \qquad r = 0 \bmod 2 
          \end{array}\right.
\end{eqnarray*}
Hence \(\lim_{r\to\infty}\sigma_r = 0\).

When \(a_{kj} = (1-r_k)r_k^j\) 
where \(0<r_k<1\) and \(\lim_{k\to\infty} r_k\to 1\),
we pick
\(s_j = (-1)^j\), that is \((+1,-1,+1,-1,\ldots)\).
that is \((+1,-1,+1,-1,\ldots)\).
Clearly does not converge. Compute
\[
\sigma_k
= \sum_{j=0}^r a_{kj}s_j
 =  \sum_{j=0}^r (1-r_k)r_k^j \cdot (-1)^j
 = (1-r_k)\sum_{j=0}^r (-r_k)^j 
 = (1-r_k) \cdot \bigl(1/(1-r_k)\bigr) = 1.\]
In particular,  \(\lim_{k\to\infty}\sigma_k=1\).



%%%%%%%%%%%%%% 16
\begin{excopy}
Suppose $X$ and $Y$ are Banach spaces, and suppose \(\Lambda\)
is a linear mapping of $X$ into $Y$, with the following property:
For every sequence \mset{x_n} in in $X$ for which \(x = \lim x_n\),
and \(y = \lim \Lambda x_n\) exist,
it is true that \(y=\Lambda x\). Prove that \(\Lambda\) is continuous.

This is so called ``closed graph theorem''
\emph{Hint}: Let \(X \oplus Y\) be the set of all ordered pairs
\((x,y)\), \(x\in X\) and \(y\in Y\), with addition and scalar multiplication
defined componentwise.
Prove that \(X\oplus Y\) is a Banach space,
if \(\|x,y)\| = \|x\| + \|y\|\).
The graph $G$ of \(\Lambda\) is a subset of \(X\oplus Y\)
formed by the pairs \((x,\Lambda x)\), \(x\in X\). Note that our hypothesis
says that $G$ is closed; hence $G$ is a banach space.
Note that \((x,\Lambda x) \to x\) is continuous, one-to-one,
and linear and maps $G$ onto $X$.

Observe that there exist \emph{nonlinear} mappings
(of \(\R^1\) onto \(\R^1\), for instance)
whose graph is closed although they are
not continuous: \(f(x)=1/x\) if \(x=0\), \(f(0)=0\).
\end{excopy}

Define projections: 
\begin{alignat*}{2}
p_1 & : G \to X & \qquad p_1(x,\Lambda x) &= x \\
p_2 & : (X,Y) \to X & \qquad p_2(x,y) &= y
\end{alignat*}
These projections are linear and continuous, and \(p_1\) is also one-to-one.
By the open mapping theorem~5.9 (\cite{RudinRCA87}), \(p_1^{-1}\) 
is continuous. But \(\Lambda = p_2 \circ p_1^{-1}\) and thus is continuous.

%%%%%%%%%%%%%% 17
\begin{excopy}
If \(\mu\) is a positive measure, each \(f\in L^\infty(\mu)\) defines
a  multiplication operator \(M_f\)
on \(L^2(\mu)\) into \(L^2(\mu)\) such that \(M_f(g) = fg\).
Prove that \(\M_f\|\leq \|f\|_\infty\).
For which measures \(\mu\) is it true that \(\|M_f\| = \|f\|_\infty\)
for all \(f\in L^\infty(\mu)\)?
For which measures \(f\in L^\infty(\mu)\) does \(M_f\) map
\(L^2(\mu)\) onto \(L^2(\mu)\)?
\end{excopy}

Let \(f\in L^\infty(\mu)\) and \(g \in L^2(\mu)\).
Now
\[ \|M_f(g)\|_2 
   = \left(\int |fg|^2\,d\mu\right)^{1/2}
   \leq \left(\int (\|f\|_\infty |g|)^2\,d\mu\right)^{1/2}
   \leq \|f\|_\infty \left(\int |g|^2\,d\mu\right)^{1/2}.
\]
Hence \(\|M_f\| \leq \|f\|_\infty\).

Now pick some \(f\in L^\infty(X,\mu)\) and an arbitrary \(\epsilon>0\).
By definition, the set
\[ U \eqdef \{x\in X: |f(x)|>\|f\|_\infty - \epsilon\}\]
satisfies \(\mu(U)>0\). Assume that for any such set, 
there exists \(W\subset U\) such that \(\mu(W)<\infty\).
Now we consider \(g=\chi_W\).
Clearly \(\|g\|_2^2 = \mu(W)\). Also
 we see that 
\[
\|M_f(g)\|_2^2
= \int |fg|^2\,d\mu
= \int_W f^2\,d\mu
\geq (\|f\|_\infty - \epsilon)^2 \mu(W).\]
Hence \(\|M_f\| \geq \|f\|_\infty - \epsilon\)
and the equality \(\|M_f\| = \|f\|_\infty\) is established.

Finally, if \(1/f \in L^\infty(\mu)\), then 
\(M_f\) is invertible and \((M_f)^{-1} = M_{1/f}\).
In particular, in this case, \(M_f\) is onto.

%%%%%%%%%%%%%% 18
\begin{excopy}
Suppose \mset{\Lambda_n} is a sequence of bounded linear transformations
from a normed linear space $X$ to a Banach space $Y$,
suppose \(\|\Lambda_n\| \leq M <\infty\) for all $n$, and suppose
there is a dense set \(E\subset X\) such that
\mset{\Lambda_n x} converges for each \(x\in E\).
Prove that \mset{\Lambda_n x} converges for each \(x\in X\).
\end{excopy}

We will show that \mset{\Lambda_n x} is a Cauchy sequence.
Pick some arbitrary \(\epsilon>0\). Pick some \(x'\in E\) such that
\(\|x-x'\|<\epsilon/(3M)\).
Since \mset{\Lambda_n x'} is a Cauchy sequence, there exists
some $N$ such that \(\|\Lambda_m x' - \Lambda_n x'\| < \epsilon/3\)
whenever \(m,n>N\). But also
\begin{eqnarray*}
\|\Lambda_m x - \Lambda_n x\|
&\leq&
 \|\Lambda_m x - \Lambda_m x'\|
 + \|\Lambda_m x' - \Lambda_n x'\|
 + \|\Lambda_n x' - \Lambda_n x\| \\
&\leq& (\|\Lambda_m\| + \|\Lambda_n\|)\cdot\|x-x'\| 
      + \|\Lambda_m x' - \Lambda_n x'\| \\
&\leq& 2M\cdot\epsilon/(3M) + \epsilon/3 = \epsilon.
\end{eqnarray*}
Thus, \mset{\Lambda_n x} is a Cauchy sequence, 
since $Y$ is a Banach space this sequence converges.


%%%%%%%%%%%%%% 19
\begin{excopy}
If \(s_n\) is the $n$th partial sum of the Fourier series of a function
\(f\in C(T)\), prove that \(s_n/\log n \to 0\)
uniformly, as \(n\to \infty\), for each \(f\in C(T)\). That is prove that
\[ \lim_{x\to\infty} \frac{\|s\|_\infty}{\log n} = 0. \]

On the other hand, if \(\lambda_n/\log n \to 0\) prove that there exists an
\(f\in C(T)\) such that the sequence \(\{s_n(f;0)/\lambda_n\}\) is unbounded.
\emph{Hint}: Apply the reasoning of Exercise~18 and that of Sec.~5.11,
with a better estimate of \(\|D_n\|_1\), than used there.
\end{excopy}

We first prove the following Lemmas
(See also Theorems~I-8-1 and II-8-13 in \cite{Zyg:2002}).

\begin{llem} \label{llem:fog:ifoig}
Let $f$,\ and $g$ be integrable functions on each 
subinterval \([a,b']\) such that \(a\leq b' < b\).
Let
\begin{equation*}
F(x) = \int_a^x f(t)\,dt
\qquad
G(x) = \int_a^x g(t)\,dt.
\end{equation*}
Assume that \(g(x)\geq 0\) and that \(\lim_{x\to b} G(x) = \infty\).
If 
\begin{equation*}
\lim_{x\to b} f(x)/g(x) = 0
\end{equation*}
then
\begin{equation*}
\lim_{x\to b} F(x)/G(x) = 0
\end{equation*}
\end{llem}

In $o$-notation: the theorem says:
If \(f(x) = o(g(x))\)
then \(F(x) = o(G(x))\).

\begin{thmproof}
Pick arbitrary \(\epsilon>0\).
Let \(x_0\in[a,b)\) such that 
\(|f(x)/g(x)| < \epsilon/2\) for \(x_0 <x < b\).
Pick \(x_1 \in (x_0,b)\) such that for all \(x\geq x_1\) we have
\begin{equation*}
G(x) > 2\int_a^{X_0} |f(t)|\,dt \bigm/ \epsilon.
\end{equation*}

Thus for \(x\geq x_1\)
\begin{equation*}
|F(x)|
\leq 
   \int_a^{x_0} f(t)\,dt 
 + \int_{x_0}^x f(t)\,dt 
\leq 
   \int_a^{x_0} f(t)\,dt 
 + \epsilon G(x)/2
\leq \epsilon G(x).
\end{equation*}
\end{thmproof}

Clearly similar result and proof holds with 
reversed directions of $a$ and $b$.

The following lemma shows that for continuous functions 
\(\|s_n(f)\|_\infty = o(\log n)\).
\begin{llem} \label{llem:ex:5.19a}
Let \(g\in C(T)\) and \(s_n\) be the $n$-Fourier sum of $g$.
Then 
\begin{equation} \label{eq:ex:5.19}
\lim_{n\to\infty} s_n(f,x) / \log n = 0.
\end{equation}
\end{llem}
\begin{thmproof}
We freely identify $T$ with \(\{x: -\pi \leq x < \pi\}\).
Let \(D_n\) be the Dirichlet's kernel, 
that is \[s_n(f,t) = (D_n * f)(t).\]

Pick arbitrary \(\epsilon>0\).
Since $f$ is uniformly continuous (on $T$), 
there exists \(\delta > 0\)
such that 
\begin{equation} \label{eq:ex5:19:fcont}
|f(x-t) - f(x+t)| < \epsilon
\end{equation}
for all \(t < \delta\) and \(x\in T\).

If \(|t|\leq \pi/2\) then \(2t/\pi \leq \sin t\)
and also
\begin{equation} \label{eq:ex5:19:Dnltt}
|D_n(t)| 
= \left|\frac{\sin\bigl((n+1/2)t\bigr)}{\sin(t/2)}\right|
\leq 1 \bigm/ (\pi t).
\end{equation}

Combining \eqref{eq:ex5:19:fcont} and \eqref{eq:ex5:19:Dnltt}
we get that
\begin{equation*}
\lim_{t\to 0} \bigl(f(x-t) - f(x+t) \bigr) D_n(t)  \bigm/\, (1/t) = 0
\end{equation*}
uniformly for all \(x\in T\).
Using 
\[
\int_{\pi/n}^{\pi} \frac{1}{t}\,dt = \log(\pi) - \log(\pi/n) = \log n.
\]
with local lemma~\ref{llem:fog:ifoig}, we have:
\begin{equation} \label{eq:ex5:19:fDn:olog}
\lim_{n\to \infty} 
 \int_{\pi/n}^{\pi} \bigl(f(x-t) - f(x+t) \bigr) D_n(t)\,dt
   \bigm/\, \log n = 0.
\end{equation}
again, uniformly for all \(x\in T\).

We now estimate \(|s_n(f,x)|\) by integrating two domains.
\begin{eqnarray*}
2\pi |s_n(f,x)|
&=& \left| \int_{-\pi}^{\pi} f(x-t)D_n(t)\,dt \right| \\
&=& 
     \left|\int_0^{\pi/n} \bigl(f(x-t) - f(x+t)\bigr)D_n(t)\,dt\right| 
   + 
     \left|\int_{\pi/n}^{\pi} \bigl(f(x-t) - f(x+t)\bigr)D_n(t)\,dt\right| \\
&\leq&
     (\pi/n) \|f\|_\infty (2n+1)
   + \left|\int_{\pi/n}^{\pi} \bigl(f(x-t) - f(x+t)\bigr)D_n(t)\,dt\right| 
\end{eqnarray*}

Using the fact that \(\lim_{n\to\infty} (2n+1) / (n\log n) = 0\)
and \eqref{eq:ex5:19:fDn:olog} we get the desired \eqref{eq:ex:5.19}.
\end{thmproof}

The following may be viewed as a converse to local lemma~\ref{llem:ex:5.19a}.
It shows that \(\log n\) is the best estimate order for \(\|s_n\|\).
\begin{llem} 
Let \(g\in C(T)\) and \(s_n\) be the $n$-Fourier sum of $g$.
Then 
If \((\lambda_n)_{n\in\N}\) is a sequence of positive numbers
such that \(\lim_{n\to\infty}\lambda_n/\log n = 0\),
then there exists \(f\in C(T)\) such that
\(\{s_n(f;0)/\lambda_n\}_{n\in\N}\) is unbounded.
\end{llem}

\begin{thmproof}
% we may trivially ignore occurrences of \(\lambda_n = 0\).
By Secation 5.11 (\cite{RudinRCA87}) if \(\Lambda_n f = s_n(f;0)\)
then
\begin{equation*}
\| \Lambda_n\| = \|D_n\|_1 
 > \frac{4}{\pi}\sum_{k=1}^\infty \frac{1}{k} = c \log n
\end{equation*}
for some constant \(c>0\).
Hence
\begin{equation*}
\|\Lambda_n\|_\infty /\lambda_n > c \log n / \lambda_n
\end{equation*}
and so 
\begin{equation*}
\lim_{n\to\infty} \|\Lambda_n\|_\infty /\lambda_n 
\geq \lim_{n\to\infty} c \log n / \lambda_n = \infty.
\end{equation*}
and by the uniform bounded principle (theorem~5.8 \cite{RudinRCA87})
there must exists \(f\in C(T)\) such that the sequence
\(\{s_n(f;0)/\lambda_n\}_{n\in\N}\) is unbounded.
\end{thmproof}


\begin{excopy}
{\small [Appears in and refers to second edition].}\newline
Is the lemma of Sec.~4.15 valid on every Banach space?
In every normed linear space?
\end{excopy}

The lemma says:
\begin{quote}
\textsl{
If $V$ is a closed subspace of a Hilbert space $H$,
\(y\in H\),
\(y\notin V\),
and \(V^{*}\) is the space spanned by $V$ and $y$, then 
\(V^{*}\) is closed.
}
\end{quote}
% \newline

The proof is it is in the (old edition) text applies for Banach spaces
as it is. Now we will prove thje following lemma using 
Hahn Banach
\index{Hahn Banach}
theorem~5. \cite{}.

\begin{llem}
If $V$ is a closed subspace of a normed linear space $L$,
\(y\in L\)
and \(V^{*}\) is the space spanned by $V$ and $y$, then 
\(V^{*}\) is closed.
\end{llem}

\begin{thmproof}
We may assume \(y\notin L\) since otherwise the result is trivial.
Now let's define a functional on \(V^{*}\) by
\[ f(v + \lambda y) = \lambda \qquad v\in V,\; \lambda \in\C.\]
By Hahn-Banach theorem we can extend $f$ to all $N$.
Assume \(w\in \overline{V^*}\).
By having a norm, there exists a sequence \((v_n + \lambda_n y)_{n\in\N}\)
such that 
\[\lim_{n\to\infty} v_n + \lambda_n y = w.\]
But then 
\[f(w) = \lim_{n\to\infty} f(v_n + \lambda_n y) 
       = \lim_{n\to\infty} \lambda_n.\] 
Thus \(w = f(y)y + \lim_{n\to\infty} v_n\) and 
% so \(w - f(y)y = \lim_{n\to\infty} v_n\).
since $V$ is closed,
\(w - f(y)y\in V\) and so \(w\in V^*\) and we have shown that \(V^*\) is 
closed.
\end{thmproof}




\end{enumerate}

\nobreak
\begin{enumerate}

\setcounter{enumi}{19}

%%%%%%%%%%%%%% 20
\begin{excopy}
\begin{itemize}

\itemch{a}
Does there exist a sequence of continuous positive functions \(f_n\)
on \(\R^1\) such that \mset{f_n(x)} is unbounded if and only if $x$ is
rational?

\itemch{b}
Replace ``rational'' by irrational in \ich{a} and answer the resulting
question.

\itemch{c}
Replace ``\mset{f_n(x)} is unbounded''
by ``\(f_n(x)\to \infty\) as \(n\to\infty\)''
and answer the resulting analogues of \ich{a} and \ich{b}.
\end{itemize}
\end{excopy}

It is easy to see that for all cases, the existence 
is the same if positive functions are defined on \([0,1]\).
Simply by restricting functions, or uniting them with 
appropriate shifting to maintain continuity.

\begin{itemize}

%%%%%%%%%%
\itemch{a}

Define 
\[G_{m,n} \eqdef \{x\in\R: f_n(x)>m\}.\]

By definition \(\lim_{n\to\infty} f_n(x) = \infty\) iff
\[\forall M\exists N \forall n\geq N\;f_n(x)>M.\] 
Hence the set consists exactly of such points $x$ is
\[L \eqdef \bigcap_M \bigcup_N \bigcap_{n>N} G_{M,n}.\]

By definition \(\limsup_{n\to\infty} f_n(x) = \infty\) iff
\[\forall M\forall N \exists n\geq N\;f_n(x)>M.\]
Hence the set consists exactly of such points $x$ is
\[U \eqdef \bigcap_M \bigcap_N \bigcup_{n>N} G_{M,n}.\]

Thus the subset of \(\R^1\) for which  \mset{f_n(x)} is unbounded
is a \(G_\delta\) set. It cannot be \Q\ since otherwise,
by local lemmas~\ref{lem:count:1cat} and~\ref{lem:gdel:2cat},
\Q\ would be of second category.

%%%%%%%%%%
\itemch{b}

Following \cite{Myerson:1991:FCF}, we will show such sequence.
Interestingly, there it cites \cite{Gelb1996}, Chapter~7 Example~4.

Let \(\{q_j\}_{j\in\N}\) be an enumeration
of all the rationals in \((0,1)\).
Define \(f_n\) as a periodic function of period $1$,
thus we need to define it on \([0,1]\).
Let \(P_n = \{0,1\} \cup \{q_j: j\leq n\}\). 
We firrst define \(f_n\) on \(P_n\).
as follows in \([0,1]\) as follows.
\begin{align*}
f_n(0) = f_n(1) &= 0 \\
\forall j\leq n\quad f_n(q_j) &= j.
\end{align*}

The set \(P_n\) partitions \([0,1]\)
into \(n+1\) sub-segmnets, where we define \(f_n\) to be linear.
It is easy to see that \(f_n\) are continuous, monotonically increasing
and that for all \(x\in\R\)
\begin{equation*}
\lim_{n\to\infty} f_n(x) = 
\left\{\begin{array}{ll}
       j       & x - \lfloor x \rfloor = q_j \\
       \infty  &   x \in \R\setminus \Q
       \end{array}\right.
\end{equation*}



%%%%%%%%%%
\itemch{c}

The analogue for \ich{a} is \emph{false}.
Assume by negation there exists a sequence \(\{f_n\}_{n\in\N}\)
of continuous functions such that 
\begin{equation*}
L \eqdef \{x\in\R: \lim_{n\to\infty} f_n(x) = \infty\} = \Q.
\end{equation*}
Let \(\{q_n\}_{n\in\N}\) be an enumeration of \Q.
We will define a decreasing sequence of closed intervals
\(\{I_n\}_{n\in\N}\) such that 
for all \(n\in\N\)
the following hold:
\begin{align}
m(I_n) &> 0  \label{eq:ex:5.20:c0} \\
I_n &\supset I_{n+1} \label{eq:ex:5.20:c1} \\
q_n &\notin I_n \label{eq:ex:5.20:c2} \\
\forall x \in I_n,\; f_n(x) &\geq n \label{eq:ex:5.20:c3}
\end{align}
Since \(\lim_{n\to\infty} f_n(1/2)=\infty\), 
there exists some \(N_1\) such that 
\(f_n(1/2) > 2\) for all \(n\geq N_1\).
By taking a sufficiently small neighborhood \(V_1\) of \(1/2\)
and picking a closed sub-interval \(I_1 \subset V_1 \setminus \{q_1\}\).
We can ensure that 
\eqref{eq:ex:5.20:c0},
\eqref{eq:ex:5.20:c2} and
\eqref{eq:ex:5.20:c3} hold.

By induction assume that 
\(\{I_j\}_{j=1}^k\) were picked and that
the above 
\eqref{eq:ex:5.20:c0},
\eqref{eq:ex:5.20:c1},
\eqref{eq:ex:5.20:c2} and
\eqref{eq:ex:5.20:c3} hold for \(n< k\).
Since \(m(I_{k-1})>0\) we can
pick some rational \(\alpha \in I_{k-1} \cap \Q\).
Since 
Since \(\lim_{n\to\infty} f_n(\alpha)=\infty\), 
we can find some \(N_k\) such that 
\(f_n(\alpha) > k+1\) for all \(n\geq N_k\).
By taking a sufficiently small neighborhood \(V_k\subset I_{k-1}\) 
of \(\alpha\)
and picking a closed sub-interval \(I_k\subset V_k \setminus \{q_k\}\)
We can ensure that the above 
\eqref{eq:ex:5.20:c0},
\eqref{eq:ex:5.20:c1},
\eqref{eq:ex:5.20:c2} and
\eqref{eq:ex:5.20:c3} hold for \(n=k\) as well.

Since \(\cup_{n\in\N} I_n \neq \emptyset\) 
There exists \(c \in \cup_{n\in\N}\)
and \(c\notin \Q\) by \eqref{eq:ex:5.20:c2}.
Clearly \(\lim_{n\to\infty} f_n(c) = \infty\)
Which gives the contradiction \(c\in L=\Q\).


The analogue for \ich{b} is \emph{true}, since the example shown
for \ich{b} holds for here as well, since
that sequence converges everywhere, either for a real number
or infinity.

\end{itemize}

%%%%%%%%%%%%%% 21
\begin{excopy}
Suppose \(E\subset \R^1\) is measurable, and \(m(E)=0\).
Must there be a translate \(E+x\) of $E$ that does not intersect $E$?
Must there be a homeomorphism $H$ of \(\R^1\) onto \(\R^1\) so that
 \(h(E)\) does not intersect $E$?
\end{excopy}

The answer is no. It is sufficiently to defy the second conjecture.

We use exercise~2 in chapter~2 of \cite{RudinFA79} 
(althogh later in the order of Rudin's text books, but the exercise
does not require further knowledge than we already have).
Let \(I \eqdef [0,1] = E_0\disjunion F_0\) 
be a disjoint union of the unit segment
such that \(m(E_0)=0\) and \(F_0\) is of first category.

Using the translation notation \(E+a = \{x+a: x\in E\}\), define
\[
E = \bigcup_{n\in\Z} E_0 + n \qquad
F = \bigcup_{n\in\Z} F_0 + n \qquad.
\]
Clearly \(\R = E \disjunion F\) where 
such that \(m(E)=0\) and $F$ is of first category.

Assume by negation there is homeomorphism \(T:\R\to\R\) 
such that \(E\cap T(E) = \emptyset\).
But then \(I = F \cup T(F)\) contradiction to the fact that
the unit segment is of second category.


%%%%%%%%%%%%%% 22
\begin{excopy}
Suppose \(f\in C(T)\) and
\index{Lip@\(\Lip\)}
\(f\in \Lip\alpha\) for some \(\alpha > 0\). (See  Exercise~11.)
Prove that the Fourier series of $f$ converges to \(f(x)\),
by completing the following outline:
It is enough to consider the case \(x=0\),
\(f(0)=0\). The difference between the partial sums \(s_n(f;0)\)
and the integrals
\[ \frac{1}{\pi} \int_{-\pi}^\pi f(t)\frac{\sin nt}{t}\,dt \]
tends to $0$ as \(n\to \infty\).
The functions \(f(t)/t\) is in \(L^1(T))\).
\index{Riemann-Lebesgue lemma}
Apply the Riemann-Lebesgue lemma. More careful reasoning shows that the
convergence is actually uniform on $T$.
\end{excopy}

Define \(f_\tau(t) = f(t + \tau)\).
Now 
\begin{align*}
2\pi s_n(f;x)
&= \int_{\pi}^\pi f(t)D_n(x - t)\,dt
 = \int_{\pi}^\pi f(t+x)D_n(x - t + x)\,dt
 = \int_{\pi}^\pi f_x(t)D_n(-t)\,dt \\
& = 2\pi s_n(f_x;0).
\end{align*}
Clearly \(f\in\Lip_\alpha\) iff  \(f_x\in\Lip_\alpha\).
Thus it is sufficient to to show that \(\lim_{n\to\infty} s_n(f;0) = f(0)\).
By looking at \(g(t) = f(t) - f(0)\) we see that 
\(f\in\Lip_\alpha\) iff  \(g\in\Lip_\alpha\)
and \(s_n(f)\) differs from \(s_n(g)\) by the constant \(\hat{f}(0)\).
Thus we may also assume \(f(0)=0\).

Let \(E_n(t) = 2\sin(t)/t\)
and \(H_n(t) = D_n(t) - E_n(t)\).
Applying L'Hospital rules we have
\begin{align*}
\lim_{t\to 0} D_n(t) &= \lim_{t\to 0} \sin((n+1/2)t)/\sin(t/2) = 2n+1\\
\lim_{t\to 0} E_n(t) &= \lim_{t\to 0} 2\sin(nt)/t = 2n\\
\lim_{t\to 0} H_n(t) &= 1
\end{align*}
and also for \(G(t) = 1/\sin t - 1/t\)
\begin{equation*}
\lim_{t\to 0} G(t) % \frac{1}{\sin t} - \frac{1}{t}
= \lim_{t\to 0} \frac{t - \sin t}{t\sin t}
= \lim_{t\to 0} \frac{1 - \cos t}{\sin t + t\cos t} 
= \lim_{t\to 0} \frac{\sin t}{\cos t + \cos t - t\sin t} 
= \frac{0}{1+1+0} = 0.
\end{equation*}
Thus if we define \(G(0)=0\) then $G$ is continuous,
and we have \(U = \sup_{t\in[-\pi,\pi]} |G(t)|+1 < \infty\).
Now  for \(t\in [-\pi/2,\pi/2]\) we can estimate 
\begin{align}
|H_n(2t)|
&\leq \left|\frac{\sin\bigl((2n+1)t\bigr)}{\sin t} 
      - \frac{\sin(2nt)}{t}\right| \notag \\
&\leq \left|\frac{\sin\bigl((2n+1)t\bigr) - \sin(2nt)}{\sin t}\right|
      + \left| \sin(2nt) \left( \frac{1}{\sin t} - \frac{1}{t}\right) \right|
     \notag \\
&\leq \left| \frac{\sin t}{\sin t}\right| 
     + 1\cdot U = U + 2. \label{eq:5.22:sinn1t}
\end{align}
In \eqref{eq:5.22:sinn1t} we used the inequality
\begin{equation*}
\sin\bigl((2n+1)t\bigr) - \sin(2nt) \leq |\sin t|
\end{equation*}
for \(-\pi/2\leq t \leq \pi/2\) derived from
\begin{equation*}
\sin\bigl((2n+1)t\bigr) = \sin(2nt)\cos t + \cos(2nt)\sin t.
\end{equation*}
Note that the bound of \(H_n\) is independent of $n$.

Put \(M=\|f\|_\infty+1\).
Pick arbitrary \(\epsilon>0\), we may assume \(\epsilon < 1\).
Find \(\eta>0\)
such that 
\begin{itemize}
\item \(\eta < \epsilon\).
\item \(|f(t)| < \epsilon/MU\) for all \(t\in[-\eta,+\eta]\).
\end{itemize}

Put 
\(f_d(t) = f(t)/\sin(t/2)\)
and
\(f_e(t) = f(t)/(t/2)\).
Since the three functions in
\(\calF = \{f, f_d, f_e\}\)
are uniformly continuous in \(\T\setminus(-\eta,\eta)\), 
let \(\delta > 0\) be such that \(|g(t_1) - g(t_0)| < \eta\epsilon < \epsilon\)
for \(g\in \calF\)
whenever \(|t_1 - t_0| < \delta\). we may assume \(\delta<\eta\).

Pick \(n_0\in\N\) such that \(2\pi/n_0 < \delta)/MU\) and
take arbitrary \(n\geq n_0\).

Now
\begin{align}
\Delta_n 
&= \left|s_n(f;0) - \frac{1}{\pi}\int_{-\pi}^\pi f(t)\sin nt/t\,dt \right| 
   \notag \\
&= \left|\frac{1}{2\pi} \int_{-\pi}^\pi f(t)H_n(t)\,dt \right| 
 = 
   \frac{1}{2\pi}
   \left|
     \int_{-\eta}^\eta \cdots + \int_{\T\setminus[-\eta,+\eta]} \cdots
   \right| \notag \\
&\leq 
     \frac{1}{2\pi}
     \left(
         2MU\epsilon
       + \left|\int_{-\pi}^{-\eta} f(t)H_n(t),dt\right|
       + \left|\int_\eta^\pi f(t)H_n(t),dt\right|
     \right)  \label{eq:5.22:2int}
\end{align}

We will estimate the last two terms can also be as small as desired.
We will workout the last ($t$-positive) one. % , the other is similar.
For abbreviation, put \(\nu = (n+1/2)\).
The periods of 
\(\sin(\nu t)\) and \(\sin(nt)\) 
are \(\gamma_d=2\pi/\nu = 4\pi/(2n+1)\)
and \(\gamma_e=2\pi/n)\) respectably.
We will separate the integration segments to whole periods as available.
For \(\iota=d,e\) (as symbols), 
find the minimal \(l_\iota\) and maximal \(h_\iota\) such that 
\(\eta\leq l_\iota \gamma_\iota\) and \(h_\iota\gamma_\iota \leq \pi\).
The ``gaps'' size are
\begin{align*}
(l_d\gamma_d - \eta ) + (\pi - h_d\gamma_d) < 2\gamma_d &= 2\pi/(2n+1)\\
(l_e\gamma_e - \eta ) + (\pi - h_e\gamma_e) < 2\gamma_e &= 2\pi/n.
\end{align*}
Now
\begin{eqnarray*}
\left|\int_\eta^\pi f(t)H_n(t),dt\right|
&\leq&
    \left|\int_\eta^{\max(l_d\gamma_d,l_e\gamma_e)} f(t)H_n(t)\,dt\right| \\
& &  + \left|\sum_{k=l_d}^{h_d-1} \int_{k\gamma_d}^{(k+1)\gamma_d} 
           f_d(t)\sin(\nu t)\,dt\right| 
     + \left|\sum_{k=l_e}^{h_e-1} \int_{k\gamma_e}^{(k+1)\gamma_e}
           f_e(t)\sin(nt)\,dt\right| \\
& &  + \left|\int_{\min(l_d\gamma_d,l_e\gamma_e)}^\pi f(t)H_n(t)\,dt\right| \\
&\leq& 2\pi MU/n \\
& &  + \left|\sum_{k=l_d}^{h_d-1} \int_0^{\gamma_d}
           f_d(t+k\gamma_d)\sin\bigl(\nu (t+k\gamma_d)\bigr)\,dt\right| 
\\ & &
     + \left|\sum_{k=l_e}^{h_e-1} \int_0^{\gamma_e}
           f_e(t+k\gamma_e)\sin\bigl(n(t+k\gamma_e)\bigr)\,dt\right| \\
& &  + 2\pi MU/n \\
&\leq& 4\pi MU/n \\
& &  + \left|\sum_{k=l_d}^{h_d-1} \int_0^{\gamma_d/2}
           \bigl(f_d(t+k\gamma_d) - f_d(t+k\gamma_d + \gamma_d/2)\bigr)
           \sin\bigl(\nu (t+k\gamma_d)\bigr)\,dt\right| \\
& &  + \left|\sum_{k=l_e}^{h_e-1} \int_0^{\gamma_e/2}
           \bigl(f_e(t+k\gamma_e) - f_e(t+k\gamma_e + \gamma_e/2) \bigr)
           \sin\bigl(n(t+k\gamma_e)\bigr)\,dt\right| \\
&\leq& \epsilon
       + (h_d - l_d)\gamma_d\epsilon/2
       + (h_e - l_e)\gamma_e\epsilon/2. \\
&\leq& (1+\pi/2+\pi/2)\epsilon.
\end{eqnarray*}

Similar estimation can be done to 
\begin{equation*}
\left|\int_{-\pi}^{-\eta} f(t)H_n(t),dt\right|.
\end{equation*}
Thus the last two terms in \eqref{eq:5.22:2int} can be arbitrarily small,
and so 
\begin{equation*}
\lim_{n\to\infty}
  \left|s_n(f;0) - \frac{1}{\pi}\int_{-\pi}^\pi f(t)\sin nt/t\,dt \right| = 0.
\end{equation*}

Therefore, in order to see that \(\lim_{n\to\infty} s_n(f;0)=0\), 
it is sufficient to show that 
\begin{equation} \label{eq:5.22.suff}
\lim_{t\to 0} \int_{-\pi}^\pi f(t)\sin nt/t\,dt = 0.
\end{equation}

{\small
Note that it is \emph{not} necessarily that \(\lim_{t\to 0}f(t)/t = 0\).
}

Let $K$ be such that \(|f(x)-f(y)|/|x-y|^\alpha\) for all \(x\neq y\).
Then 
\begin{equation*}
|f(t)/t| 
= |f(t)| |t|^{\alpha-1}/|t|^\alpha
\leq K |t|^{\alpha-1}
\end{equation*}
for \(t\neq 0\). Thus
\begin{equation*}
\|f/t\|_1
= \frac{1}{2\pi} \int_{-\pi}^\pi |f(t)/t|\,dt 
\leq \frac{1}{2\pi} \int_{-\pi}^\pi K|t|^{\alpha-1}\,dt 
% = \frac{K}{\alpha\pi}\pi^\alpha.
= K\pi^{\alpha-1}/\alpha < \infty
\end{equation*}
which shows that \(f_e = f(t)/t \in L^1(\T)\).
Now
\begin{equation*}
\int_{-\pi}^\pi f(t)\sin nt/t\,dt 
= (2\pi/2i)\bigl(\hat{f_e}(n) - \hat{f_e}(n)\bigr).
\end{equation*}
By the Riemann-Lebesgue lemma \(\lim_{n\to\infty} \hat{f_e}(n) = 0\).
Hence \eqref{eq:5.22.suff} holds.

All the above bounds and minimal (\(\delta\), \(\eta\)) values
could have been taken for any value other than \(t=0\).
So we could have the limit converge uniformly
also for \(f_{\tau}(t) = f(t+\tau) - f(\tau)\).


\iffalse
Define \(f_\tau(t) = f(t + \tau)\).
Now 
\begin{align*}
2\pi s_n(f;x)
&= \int_{\pi}^\pi f(t)D_n(x - t)\,dt
 = \int_{\pi}^\pi f(t+x)D_n(x - t + x)\,dt
 = \int_{\pi}^\pi f_x(t)D_n(-t)\,dt \\
& = 2\pi s_n(f_x;0).
\end{align*}
Clearly \(f\in\Lip_\alpha\) iff  \(f_x\in\Lip_\alpha\).
Thus it is sufficient to to show that \(\lim_{n\to\infty} s_n(f;0) = f(0)\).
By looking at \(g(t) = f(t) - f(0)\) we see that 
similarly 
Clearly \(f\in\Lip_\alpha\) iff  \(g\in\Lip_\alpha\)
and \(s_n(f)\) differs from \(s_n(g)\) by the constant \(\hat{f}(0)\).
Thus we may also assume \(f(0)=0\).

{\small
Note that it is \emph{not} necessarily that \(\lim_{t\to 0}f(t)/t = 0\).
}

Let $K$ be such that \(|f(x)-f(y)|/|x-y|^\alpha\) for all \(x\neq y\).
Then 
\begin{equation*}
|f(t)/t| 
= |f(t)| |t|^{\alpha-1}/|t|^\alpha
\leq K |t|^{\alpha-1}
\end{equation*}
fot \(t\neq 0\). Thus
\begin{equation*}
\|f/t\|_1
= \frac{1}{2\pi} \int_{-\pi}^\pi |f(t)/t|\,dt 
\leq \frac{1}{2\pi} \int_{-\pi}^\pi K|t|^{\alpha-1}\,dt 
% = \frac{K}{\alpha\pi}\pi^\alpha.
= K\pi^{\alpha-1}/\alpha < \infty
\end{equation*}
which shows that \(f(t)/t \in L^1(\T)\).
\fi


%%%%%%%%%%%%%%%%%
\end{enumerate}

 % \setcounter{chapter}{5}  % -*- latex -*-
% $Id: rudinrca6.tex,v 1.5 2008/07/19 08:56:55 yotam Exp $


%%%%%%%%%%%%%%%%%%%%%%%%%%%%%%%%%%%%%%%%%%%%%%%%%%%%%%%%%%%%%%%%%%%%%%%%
%%%%%%%%%%%%%%%%%%%%%%%%%%%%%%%%%%%%%%%%%%%%%%%%%%%%%%%%%%%%%%%%%%%%%%%%
%%%%%%%%%%%%%%%%%%%%%%%%%%%%%%%%%%%%%%%%%%%%%%%%%%%%%%%%%%%%%%%%%%%%%%%%
\chapterTypeout{Complex Measures} % 6

%%%%%%%%%%%%%%%%%%%%%%%%%%%%%%%%%%%%%%%%%%%%%%%%%%%%%%%%%%%%%%%%%%%%%%%%
%%%%%%%%%%%%%%%%%%%%%%%%%%%%%%%%%%%%%%%%%%%%%%%%%%%%%%%%%%%%%%%%%%%%%%%%
\section{Notes}

Intuitively the following lemma shows that \(f\,d\mu\) is regular.
\begin{llem} \label{llem:fdmu:abscont}
Let $f$ be a \(\mu\)-measurable function on $X$
such that \(\int|f|\,d\mu<\infty\).
For any \(\epsilon>0\) there exists \(\delta>0\) such that
\(\int_E |f|\,d\mu < \epsilon\) whenever \(\mu(E)<\delta\).
\end{llem}
\begin{thmproof}
Define the complex measure \(\lambda(E) = \int_E f\,d\mu\).
Clearly \(\lambda \ll \mu\). Theorem~6.11 gives the desired conclusion.
\end{thmproof}

%%%%%%%%%%%%%%%%%%%%%%%%%%%%%%%%%%%%%%%%%%%%%%%%%%%%%%%%%%%%%%%%%%%%%%%%
%%%%%%%%%%%%%%%%%%%%%%%%%%%%%%%%%%%%%%%%%%%%%%%%%%%%%%%%%%%%%%%%%%%%%%%%
\section{Exercises} % pages 132-134

%%%%%%%%%%%%%%%%%
\begin{enumerate}
%%%%%%%%%%%%%%%%%

%%%%%%%%%%%%%% 01
\begin{excopy}
If \(\mu\) is a complex measure on a \(\sigma\)-algebra \frakM,
and if \(E \in \frakM\), define
\begin{equation*}
\lambda(E) = \sup \sum |\mu(E_i)|,
\end{equation*}
the supremum being taken over all \emph{finite} partitions \(\{E_i\}\) of $E$.
Does it follow that \(\lambda = |\mu|\)?
\end{excopy}

Yes.

Clearly \(\lambda(E) \leq |\mu|(E)\). To show the opposite inequality,
let \(\epsilon > 0\). We can find some countable
partition \(\{E_i\}\) of $E$
such that
\begin{equation*}
 \sum_{i\in\N} |\mu(E_i)| > |\mu|(E) - \epsilon/2.
\end{equation*}
So  we can find some integer \(m<\infty\) such that
\begin{equation*}
 \sum_{i=1}^m |\mu(E_i)| > |\mu|(E) - \epsilon.
\end{equation*}
Hence \(\lambda(E) \geq |\mu|(E)\) and thus
\(\lambda(E) = |\mu|(E)\).


%%%%%%%%%%%%%%
\begin{excopy}
Prove that the example given at the end of Sec.~6.10 has the stated properties.
\end{excopy}

\paragraph{Property 1.}
On \(I=(0,1)\), let \(\mu\) be the Lebesgue measure
and \(\lambda\) the counting measure.
By negation, let \(\lambda = \lambda_s + \lambda_a\) be the
Lebesgue decomposition.
For any \(x\in I\), we have
 \(\lambda(\{x\}) = 1\) and  \(\mu(\{x\}) = 0\)
Hence
\(\lambda_s(\{x\}) = 0\) and  \(\lambda_a(\{x\}) = 1\).
Since \(\mu\perp \lambda_a\) we have \(I = A \disjunion B\)
such that for any Lebesgue measurable $E$, we have
\begin{equation*}
\mu(E) = \mu(E\cap A) \qquad \textrm{and} \qquad
\lambda_a(E) = \lambda_a(E\cap B).
\end{equation*}
Since
\(\lambda_a(\{x\}) = 1\)
for any \(x\in I\), we must have \(B=I\), but then \(\mu=0\)
which is a contradiction.

\paragraph{Property 2.}
Assume by negation \(h\in L^1(\lambda)\) such that \(d\mu = h\,d\lambda\).
If \(h=0\) then \(\mu=0\) so we may assume that \(h(x)\neq 0\)
for some \(x\in I\). But then
\begin{equation*}
0 = \mu(\{x\}) = h(x)\lambda(\{x\}) = h(x) \neq 0.
\end{equation*}
which is again a contradiction.


%%%%%%%%%%%%%%
\begin{excopy}
Prove that the vector space \(M(X)\) of  all complex regular Borel measures
on a locally compact Hausdorff space $X$ is a Banach space if
\(\|\mu\| = |\mu|(X)\).
\emph{Hint}: Compare Exercise~8, Chap.~5.
[That the difference of any two members of \(M(X)\) is in \(M(X)\)
was used in the first
paragraph of the proof of Theorem~6.19; supply a proof of this fact.]
\end{excopy}

Let \(\lambda,\mu \in M(X)\).
For \(E = \disjunion E_j\) use definition to compute
\begin{align}
\lambda(E) - \mu(E) \notag \\
&= \left(\sum_{j\in\N} \lambda(E_j)\right) -
   \left(\sum_{j\in\N} \mu(E_j)\right) \notag \\
&= \sum_{j\in\N} \lambda(E_j) - \mu(E_j) \label{eq:ex6.3} \\
&= \sum_{j\in\N} (\lambda - \mu)(E_j) \notag \\
&= (\lambda - \mu)(E).
\end{align}
The equality in \eqref{eq:ex6.3} holds since the series converge absolutely.

If we define \((a\mu)(E) = a(\mu(E)\) for \(a\in\C\), all vector space
equalities trivally hold. Thus we need only to show completeness.
Let \(\{\mu_j\}_{j\in\N}\) be a Cauchy sequence of measures.
That is for each \(\epsilon>0\) there exists some $N$ such that
\(|\mu_j - \mu_k|(X) < \epsilon\) if \(j,k>N\).
Since
\(|\mu_j - \mu_k|(E) \leq |\mu_j - \mu_k|(X)|\)
for any \(E\in\frakM\), the sequence
\(\{\mu_j(E)\}\) is a Cauchy sequence, and we can define
its limit \(\mu(E) = \lim_{j\to\infty} \mu_j(E)\).
It is easy to see that with \(\mu\), the vector space equalities still hold.
We need to show that \(\mu\in M(X)\).

We have the unique decompositions
\begin{align*}
\mu_j &= \lambda_j^+ - \lambda_j^- + i(\nu_j^+ - \nu_j^-) \\
\mu   &= \lambda^+ - \lambda^- + i(\nu^+ - \nu^-)
\end{align*}
where
\(\lambda_j^+\), \(\lambda_j^-\), \(\nu_j^+\) and \(\nu_j^-\)
are positive measures since
the mappings\(x\mapsto|x|\), \(\Re\) \(\Im\) are continuous,
and
\(\lambda^+\), \(\lambda^-\), \(\nu^+\) and \(\nu^-\)
are non-negative functions on \frakM.

For temporary abberviation, we use
\(\lambda=\lambda^+\) and \(\lambda_j = \lambda_j^+\).

Pick arbitrary \(K<\infty\) so we can estimate
\begin{align*}
\sum_{k\leq K} \lambda(E_k)
&= \sum_{k\leq K} \lim_{j\to\infty} \lambda_j(E_k)
 = \lim_{j\to\infty} \sum_{k\leq K} \lambda_j(E_k)
 = \lim_{j\to\infty} \lambda_j\left(\Disjunion_{k\leq K} E_k\right) \\
&\leq \lim_{j\to\infty} \lambda_j(E) \\
&= \lambda(E).
\end{align*}
Hence
\begin{equation*}
\sum_{k\in\N} \lambda(E_k) \leq \lambda(E).
\end{equation*}
Looking at \(f_j = \lambda_j(E_k)\) as function on \(k\in\N\)
with at the coundting measure on \N, we utilize
Lebesgue's dominated convergence theorem
in the following \eqref{eq:ex:6.3}.

Now
If \(E = \disjunion_{j\in\N} E_i\) all in \frakM, then
\begin{align}
\lambda(E)
&= \lim_{j\to\infty} \lambda_j(E)
 = \lim_{j\to\infty} \lambda_j\left(\Disjunion_{K\in\N} E_k\right) \notag \\
&= \lim_{j\to\infty} \sum_{k\in\N} \lambda_j(E_k) \notag \\
&= \sum_{k\in\N} \lim_{j\to\infty} \lambda_j(E_k)  \label{eq:ex:6.3} \\
&= \sum_{k\in\N} \lambda(E_k) \notag
\end{align}

Similarly, we can derive the equlaities for
\(\lambda^-\), \(\nu^+\) and \(\nu^-\). Then we sum them to get
\begin{equation*}
\mu(E) = \sum_{k\in\N}  \mu(E_k).
\end{equation*}


%%%%%%%%%%%%%%
\begin{excopy}
Suppose \(1 \leq p \leq \infty\), and $q$  is the exponent conjugate to $p$.
Suppose \(\mu\) is a positive \(\sigma\)-finite measure and $g$ is a measurable
function such that \(fg\in L^1(\mu)\) for every
\(f\in L^p(\mu)\). Prove that \(g\in L^q(\mu)\).
\end{excopy}

\iffalse
% By H\"older's inequality \(\|fg\|_1 \leq \|f\|_p \|g\|_q\).
Clearly \(f \mapsto \int fg\;d\mu\) is a linear functional on \(L^p(\mu)\).
Let \(X=\disjunion_{j\in\N} X_j\) a decomposition of the space
such that \(\mu(X_j)<\infty\). We set \(S_n = \disjunion_{j\leq n} X_j\)
and let \(g_n = g_{|S_n}\).
Using Exercise~2.5\ich{a} we have
\begin{equation*}
\|g_n\|_q = \|g_n\|_\infty \leq \|g\|_\infty.
\end{equation*}

Clearly \(g\in L^q(\mu)\) iff \(|g| L^q(\mu)\).
Since \(|\mu|(X)<\infty\), we have the constant \(1\in L^q\)
for all \(q\in[1,\infty]\). Hence also
\(g\in L^q(\mu)\) iff \(g\pm 1\in L^q(\mu)\).
Let \(X_n=\{x\in X: n-1 \leq |g(x)| < n\}\).

Assume first \(1<p,q<\infty\), and by negation, assume
\begin{equation} \label{eq:ex6.4:neg}
g\notin L^q(\mu).
\end{equation}
Hence we can have the estimatation.
\begin{equation*}
\sum_{n\in \N} n^q \mu(X_n)
 \geq \sum_{n\in \N} \int_{X_n} |g|^q\,d\mu \\
 = \int_X |g|^q\,d\mu
 = \|g\|q^q = \infty.
\infty
\end{equation*}

Define the function
\begin{equation*}
h(x) =
\left\{
 \begin{array}{ll}
 0 & \quad x\in X_0\\
 n^{q/p} & \quad x\in X_n \;\textrm{and}\; n>0
 \end{array}
\right.
\end{equation*}

Now
\begin{equation*}
\|h\|_p^p
= \int_X h^p\,d\mu
= \sum_{n=0}^\infty \int_{X_n} h^p\,d\mu
= \sum_{n\in\N}  \left(n^(q/p)\right)^p\,d\mu
= \sum_{n\in\N}  n^q\,d\mu
\end{equation*}
Hence \(\||g|-1\|_q = \infty\),

\fi

For each $n$ define
\begin{equation*}
g_n(x) =
\left\{
 \begin{array}{ll}
 g(x) & \quad |g(x)| \leq n\\
 ng(x)/|g(x)| & \quad |g(x)| > n
 \end{array}
\right.
\end{equation*}
Clearly \(\lim_{n\to\infty}g_n(x) = g(x)\;\aded\)\,.
Now we can define functionals \(\Lambda_n\in (L^p(\mu))^*\) by
\begin{equation*}
\Lambda_n(f) = \int_X fg_n\,d\mu.
\end{equation*}
With the estimatation
\begin{equation*}
|\lambda_n{f}|
\leq \int_X |fg_n|\,d\mu
\leq \int_X |fg|\,d\mu
< \infty
\end{equation*}
we use the
\index{Banach-Steinhaus}
\index{Steinhaus}
Banach-Steinhaus Theorem~5.8 (\cite{RudinRCA87})
to get a unifrom bound \(\|\Lambda_n\|\leq M\).
Now for each \(f\in L^p(\mu)\) we have
\begin{equation*}
\lim_{n\to\infty} \int_X fg_n\,d\mu = \int_X fg\,d\mu
\end{equation*}
and if we define \(\Lambda(f) = \int_X fg\,d\mu\) we
have \(\|\Lambda\| \leq M\).


Applying the uniquness part of Theorem~6.16 \cite{RudinRCA87}
we get the desired result for \(p<\infty\).

Now assume \(p=\infty\). Obviously the constant function
\(1\in L^\infty(\mu)\) and by assumption
\(|\int_X 1\cdot g\,d\mu|<\infty\). Hence \(f\in L^1{\mu}\).


%%%%%%%%%%%%%% 05
\begin{excopy}
Suppose $X$ consists of two points $a$ and $b$; define
\(\mu(\{a\}) = 1\),
\(\mu(\{b\}) = \mu(X) = \infty\), and
\(\mu(\emptyset) = 0\).
Is it true for this \(\mu\) ,
that \(L^\infty(\mu)\) is the dual space of \(L^1(\mu)\)?
\end{excopy}

No. Clearly
\begin{equation*}
L^1(\mu) = \{f\in \C^X: f(b)=0\}.
\end{equation*}
Now consider \(g_1,g_2\in L^\infty(\mu)\) defined as
\(g_1=1\), \(g_2(a)=1\) and \(g_2(b)=0\). Both induces the same
functional in \(\left(L^1(\mu)\right)^*\).


%%%%%%%%%%%%%%
\begin{excopy}
Suppose \(1 < p < \infty\), and prove that
\(L^q(\mu)\) is the dual space of \(L^p(\mu)\)
even if \(\mu\) is not \(\sigma\)-finite.
(as usual \(1/p+1/q=1\).)
\end{excopy}

For positive  \(\sigma\)-finite measure the claim has been proved.
Assume now that \((X,\frakM,\mu)\) is a complex measure space.
By Theorem~6.12 (\cite{RudinRCA87}) there exists a measurable
function $h$ such that \(\forall x\in X, |h(x)|=1\) and
\(d\mu = h\,d|\mu|\).
We note, that as \emph{sets of functions} we trivially have
\(L^p(\mu)=L^p(|\mu|)\) and \(L^q(\mu)=L^q(|\mu|)\).
The vector space operations are identical.
Also the topolgies are the same, since the norms are invariant,
that is
\begin{equation*}
\|f\|_{p,\mu}
= \left(\int |f|^p\,d\mu\right)^{1/p}
= \left(\int |f|^p\,d|\mu|\right)^{1/p}
\|f\|_{p,|\mu|}.
\end{equation*}

Hence, given a \(\Lambda\in (L^p(\mu))^*\)
it may be viewed as a functional on \(L^p(|\mu|)\).
From results on positive measures we have \(g\in L^q(|\mu|)\)
such that
\begin{equation*}
\Lambda(f) = \int_X fg\,d|\mu|
\end{equation*}
for all \(f\in L^p(\mu)\).
But then
\begin{equation*}
\Lambda(f) = \int_X f(g/h)\,d\mu
\end{equation*}
and clearly \(g/h\in L^q(\mu)\).
The uniqness of \(g/h)\) is proved in same way
as in the proof of Theorem~6.16 (\cite{RudinRCA87}).


%%%%%%%%%%%%%%
\begin{excopy}
Suppose \(\mu\) is a complex Borel measure on \([0,2\pi]\)
(or on the unit circle \T),
and define the Fourier coefficients of \(\mu\) by
\begin{equation*}
\widehat{\mu}(n) = \int e^{-int}\,d\mu(t) \qquad (n=0,\pm 1,\pm 2, \ldots).
\end{equation*}
Assume that \(\widehat{\mu}(n) \to 0 \) as \(n\to +\infty\)
and prove that then
\(\widehat{\mu}(n) \to 0 \) as \(n\to -\infty\).

\emph{Hint}: The assumption also holds with \(f\,d\mu\) in  place of \(d\mu\)
if $f$ is any trigonometric polynomial, hence if $f$ is continuous,
hence if $f$ is any bounded Borel function,
hence if \(d\mu\) is replace by \(d|\mu|\).
\end{excopy}

Following the hint. For each \(m\in\Z\) we have
\begin{equation*}
\lim_{n\to+\infty} \int e^{-int}e^{imt}\,d\mu(t)
= \lim_{n\to+\infty} \int e^{-i(n-m)t}\,d\mu(t)
= \lim_{n\to+\infty} \int e^{-int}\,d\mu(t)
= 0.
\end{equation*}
If for \(j=1,2\) we have
\begin{equation*}
\lim_{n\to+\infty} \int e^{-int}f_j(t)\,d\mu(t) = 0
\end{equation*}
Then
\begin{equation*}
\lim_{n\to+\infty} \int e^{-int}(f_1+f_2)(t)\,d\mu(t) = 0.
\end{equation*}
Thus we have similar convergence \(\lim_{n\to+\infty}\widehat{(f\mu)}(n)\)
for trigonometric polynomials $f$.
Similarly this holds for all continuous $f$, in particular for
the unique $h$ such that \(|\mu| = h\mu\) and \(|h(t)|=1\)
Now
\begin{align}
\widehat{\mu}(-n)
&= \int e^{int}\,d\mu(t)
 = \int e^{int}/h(t)\,d|\mu(t)|
 = \overline{\int \overline{e^{int}}/\overline{h(t)}\,d|\mu(t)|}
 = \overline{\int e^{-int}/\overline{h(t)}\,d|\mu(t)|} \notag \\
&= \overline{\int e^{-int}\left(h(t)/\overline{h(t)}\right)\,d\mu(t)}
   \label{eq:ex6.7}.
\end{align}
As before, the convergence to zero also holds for
\(h(t)/\overline{h(t)}\) and thus the values \eqref{eq:ex6.7}
converge to zero.


%%%%%%%%%%%%%%
\begin{excopy}
In the terminology of Exercise~7, find all \(\mu\) such that
\(\hat{\mu}\) is periodic, with respect to $k$
[This means that \(\hat{\mu}(n+k) = \hat{\mu}(n)\) for all integers $n$;
of course $k$ is also assumed to be an integer.]
\end{excopy}

Equivalently, we can look for all measurable functions $f$, such that
\begin{equation*}
\int_\T e^{-int}f(t)\,dm(t)
\end{equation*}
is periodic in $n$.


\paragraph{Example:} Given $k$, for each \(j\in\N_k\) let
\(K = \{e^{2j\pi i/k}: 0\leq j < k\}\) and set
\(\mu(\{P\}) = 1/k\) for each \(P\in K\)
and \(\mu(\T\setminus K) = 0\). Complete \(\mu\) to a measure,
then
\begin{equation*}
\widehat{\mu}(n) = \left\{
 \begin{array}{ll}
 1 & \quad \textrm{if}\; n = 0 \bmod k \\
 0 & \quad \textrm{if}\; n \neq 0 \bmod k
 \end{array}
 \right.
\end{equation*}
If \(\mu\ll m\) then \(d\mu = h\,dm\) for some \(h\in L^1(\T)\)
but \(\widehat{h}(n)\) cannot be periodic (unless \(h=0\)),
since \(\lim_{n\to\infty}\widehat{h}(n) = 0\).


%%%%%%%%%%%%%%
\begin{excopy}
Suppose that \(\{g_n\}\) is a sequence of positive continuous functions on
\(I=[0,1]\), that \(\mu\) is a positive Borel measure  on $I$, and that
\begin{itemize}
\itemch{i} \(\lim_{n\to\infty} g_n(x) = 0\quad \aded [m]\).
\itemch{ii} \(\int_I g_n\,dm = 1\) for all $n$,
\itemch{iii} \(\lim_{n\to\infty} \int_I fg_n\,dm = \int_I f\,d\mu\)
              for every \(f\in C(I)\).
\end{itemize}
Does it follow that \(\mu\perp m\)?
\end{excopy}

Yes. We will show that there exists a Borel subset \(D\subset I\) such that
\(m(D)=0\) and \(\mu(E) = \mu(E\cap D)\) for each Borel set $E$.

Consider the ``bad set''
\begin{equation*}
 B = \{x\in I: \limsup g_n(x) > 0\}.
\end{equation*}
Since \(m(B) = 0\) by \ich{i} we can redefine \(g_n(x)=0\)
for each \(x\in B\).
This redefinition removes the \aded\ restriction from \ich{i}, but
does not effect
the other assumptions \ich{ii}, \ich{iii} and not the desired conclusion.

Pick some \(k\in\N\).
\index{Egoroff}
By Egoroff's Theorem (Exercise~3.16 \cite{RudinRCA87})
there exists a measurable \(D_n \subset I\) such that
\(m(D_k) < 1/k\) and such that
\(\{g_n\}\) converges \emph{uniformly} to $0$ on \(C_k\)
where \(E_k = I \setminus D_k\).
We will show that \(\mu(E_k) = 0\).
\iffalse
Hence
\begin{equation*}
\lim_{n\to\infty} \int_{E_k} g_n\,dm
\leq \bigl(1 - m(D_k)\bigr) \lim_{n\to\infty} \sup_{x\in E_k} g_n(x)
= 0.
\end{equation*}
\fi
Given \(\epsilon>0\), we
apply Lusin Theorem~2.24 (\cite{RudinRCA87}) to the function
\(\chhi_{E_k}\) to get a function \(f_k\in C(I)\)
such that \(\|f_k\|_\infty = 1\) and
\begin{equation*}
m\bigl(\{x\in I: f(x) \neq \chhi_{E_k}(x)\}\bigr) < \epsilon.
\end{equation*}
\begin{align*}
\mu(E_k)
&= \int_I \chhi{E_k}\,d\mu \\
&\leq \int_I f_k\,d\mu + \epsilon
\end{align*}




%%%%%%%%%%%%%% 10
\begin{excopy}
Let \((X,\frakM,\mu)\) be a positive measure space. Call a set
\(\Phi \subset L^1(\mu)\)
\index{uniformly  integrable}
\emph{uniformly  integrable}
if to each \(\epsilon < 0\) corresponds \(\delta>0\) such that
\begin{equation*}
\left|\int_E f\,f\mu\right| < \epsilon
\end{equation*}
whenever \(f\in\Phi\) and \(\mu(E) < \delta\).
\begin{itemize}

\itemch{a}
Prove that every finite subset of \(L^1(\mu)\) is uniformly integrable.

\itemch{b}
Prove the following convergence theorem of \index{Vitali} Vitali:

\textsl{
If (i) \(\mu(X)<\infty\), (ii) \(\{f_n\}\) is uniformly integrable,
(iii) \(f_n(x)\to f(x)\;\aded\) as \(n\to\infty\),
and (iv) \(|f(x)|<\infty\;\aded\), then \(f\in L^1(\mu)\) and
\begin{equation*}
\lim_{n\to\infty} \int_X|f_n - f|\,d\mu = 0.
\end{equation*}
}
\emph{Suggestion}: Use \index{Egoroff} Egoroff's theorem

\itemch{c}
Show that \ich{b} fails if \(\mu\) is a Lebesgue measure on
\((-\infty,\infty)\), even if \(\{\|f_n\|_1\}\) is assumed to be bounded.
Hypothesis (i) can therefore  not be omitted in \ich{b}.

\itemch{d}
Show that the hypothesis (iv)  is redundant in \ich{b} for some \(\mu\)
(for instance, for Lebesgue measure on a bounded interval), but that there are
finite measures for which the omission of~(iv)
would make \ich{b} false.

\itemch{e}
Show that Vitali's theorem implies Lebesgue's dominated convergence theorem,
for finite measure space. Construct an example in which Vitali's theorem
applies although the hypothesis of
Lebesgue's theorem does not hold.

\itemch{f}
Construct a sequence \(\{f_n\}\), say on \([0,1]\), so that
\(f_n(x)\to 0\) for every $x$,
\(\int f_n\to 0\) but \(\{f_n\}\) 
is not uniformly integrable (with respect to Lebesgue's measure).

\itemch{g}
However, the following converse of Vitali's theorem is true:

\textsl{
If \(\mu(X)<\infty\), \(f_n\in L^1(\mu)\), and
\begin{equation*}
\lim_{n\to\infty} \int_E f_n\,d\mu
\end{equation*}
exists for every \(E\in \frakM\), then \(\{f_n\}\) is uniformly integrable.
}

Prove this by completing the following outline.

Define \(\rho(A,B) = \int |\chhi_A - \chhi_B|\,d\mu\).
Then \((\frakM,\rho)\) is a complete metric space
(modulo sets of measure $0$), and \(E\mapsto \int_E f_n\,d\mu\) is
a continuous for each $n$.
If \(\epsilon > 0\), there exist \(E_0\), \(\delta\), $N$
(Exercise~13, Chap~5) so that
\begin{equation} \label{eq:ex:6.10}
\left| \int_E (f_n - f_N)\,d\mu\right| < \epsilon
\qquad \textrm{if}\qquad
\rho(E, E_0) < \delta, \qquad n > N.
\end{equation}
If \(\mu(A)<\delta\), \eqref{eq:ex:6.10} holds with \(B = E_0 - A\)
and \(C = E_0 \cup A\)
in place of $E$.
Thus \eqref{eq:ex:6.10} holds with $A$ in place of $E$ and \(2\epsilon\)
in place of \(\epsilon\).
Now apply \ich{a} to \(\{\seq{f}{N}\}\): There exists \(\delta'>0\)
such that
\begin{equation*}
\left| \int_A f_n\,d\mu \right| < 3\epsilon
\qquad \textrm{if}\qquad
\mu(A) < \delta', \qquad n=1,2,3,\ldots.
\end{equation*}
\end{itemize}
\end{excopy}

We will need the following result.
\begin{llem}
If
\(\Phi=\{f_j: j\in J\}\) is uniformly  integrable,
then
\(\{|f_j|: j\in J\}\) is uniformly  integrable.
\end{llem} \label{llem:abs:unifinteg}
\begin{thmproof}
Define the complex plane quartans
\begin{align*}
\C_0 &= \{z\in\C: \Re(z)>0 \;\wedge\; \Im(z)\geq 0\} \\
\C_j &= \{e^{j\pi i/2}z: z\in \C_0 \} \quad j=1,2,3.
\end{align*}
Now
\begin{equation*}
\C = \{0\} \disjunion \Disjunion_{j=0}^3 \C_j.
\end{equation*}
Pick arbitrary \(\epsilon> 0\). Let \(\delta>0\) such that 
\(|\int_E f\,d\mu|<\epsilon\) for all \(f\in\Phi\) 
whenever \(\mu(E)<\delta\).
Take such $E$, and define \(E_j = E\cap \C_j\) for \(j=0,1,2,3\).
For each such $j$
\begin{align*}
\int_{E_j} |f|\,d\mu
&= \int_{E_j} |\Re(f) + \Im(f)|\,d\mu \\
&\leq  \int_{E_j} |\Re(f)|\,d\mu + \int_{E_j} |\Im(f)|\,d\mu \\
&=  \left|\int_{E_j} \Re(f)\,d\mu\right| 
  + \left|\int_{E_j} \Im(f)\,d\mu\right| 
 =  \left|\Re\left(\int_{E_j} f\,d\mu\right)\right| 
  + \left|\Im\left(\int_{E_j} f\,d\mu\right)\right| \\
&\leq  2\max\left(\left|\Re\left(\int_{E_j} f\,d\mu\right)\right|,
                  \left|\Im\left(\int_{E_j} f\,d\mu\right)\right|\right) \\
&\leq  \sqrt{2}\left|\int_{E_j} f\,d\mu\right|.
\end{align*}

Hence
\begin{equation*}
\int_E |f|\,d\mu
=    \sum_{j=0}^3 \int_{E_j} |f|\,d\mu
\leq \sqrt{2}\left|\sum_{j=0}^3 \int_{E_j} f\,d\mu\right| 
<    \sqrt{2}\epsilon
\end{equation*}
and thus \(\Phi\) is uniformly integrable.
\end{thmproof}



\begin{itemize}

\itemch{a}
By Theorems~1.29 and~6.11 \cite{RudinRCA87} $f$ is uniformly integrable
(by itself) if \(f\in L^1(\mu)\).
For finite set \(\{f_j\}_{j=1}^n\), for each \(\epsilon>0\)
we pick \(\delta = \min\{\delta_j: 1\leq j \leq n\}\)
where \(\delta_j\) corresponds to \(f_j\)


\itemch{b}
By removing a subset of measure zero, we may assume
\(f_n(x) \to f(x)\) for all \(x\in X\).

Pick arbitrary \(\epsilon>0\).
By Egoroff's Theorem (Exercise~3.16)  
and by being uniform integrable, 
we can find some some \(\delta>0\) and a set \(E\subset X\) such that 
\begin{enumerate}
\itemch{i} \(\mu(X\setminus E)<\delta\) .
\itemch{ii} On \(X\setminus E\) the convergence \(f_n \to f\) is uniform.
\itemch{iii} \(\int_E |f_n|\,d\mu < \epsilon\) for all $n$, 
      see local lemma~\ref{llem:abs:unifinteg}.
\end{enumerate}
By Fatou's Lemma (Theorem~1.28 \cite{RudinRCA87}) we have
\begin{equation*}
\int_E |f|\,d\mu 
= \int_E |\lim_{n\to\infty}f_n|\,d\mu 
= \int_E |\liminf_{n\to\infty}f_n|\,d\mu 
\leq \liminf_{n\to\infty} \int_E |f_n|\,d\mu < \sqrt{2}\epsilon
\end{equation*}
Pick $m$ such that \(|f_n(x)-f(x)| < \epsilon\)
for all \(n\geq m\) and all \(x\in X\setminus E\).
Thus
\begin{equation*}
\left|\int_X f\,d\mu\right|
\leq
  \left|\int_{X\setminus E} f\,d\mu\right|
+ \left|\int_E f\,d\mu\right|
\leq
  \int_{X\setminus E} (|f_n| + \epsilon)\,d\mu + \sqrt{2}\epsilon < \infty.
\end{equation*}
Thus \(f\in L^1(\mu)\).

Now
\begin{equation*}
\int_X |f-f_n|\,d\mu
= \int_{X\setminus E} |f-f_n|\,d\mu + \int_E |f-f_n|\,d\mu 
% \leq \mu(X)\epsilon + 2\sqrt{2}\epsilon
\leq (\mu(X) + 2\sqrt{2})\epsilon
\end{equation*}
and the desired convergence holds.

\itemch{c}
For a counterexample, simple take \(f_n(x) = \chhi_{[0,n]}\)
that are uniformly integrable.
Clearly \(\lim_{n\to\infty} f_n = 1 \notin L^1(\R,m)\).

\itemch{d}
For Lebesgue measure $m$ on \([0,1]\), say we dropy the 
\ich{iv} requirement.

If we take a singleton space \(X=\{x\}\) with \(\mu(\{x\}=1\)
and \(f_n(x)=n\), then the set is trivially uniformly integrable,
by ``suggesting'' \(\delta=1/2<1\).
Clearly \(f(x)=\infty\) and \(\int_X|f_n-f|\,d\mu=\infty\).

\itemch{e}
Given the assumptions of Lebesgue's dominated convergence theorem ---
\(f_n\to f\) and \(|f_n|\leq g\in L^1(\mu)\).
we Actually need to show that \(\Phi=\{f_n\}_{n\in\N}\) 
is uniformly integrable.
By \ich{a}, the function \(\{g\}\) is uniformly integrable.
Given \(\epsilon>0\) there exists some \(\delta>0\) 
such that \(|\int_E g\,d\mu|<\epsilon\) whenever \(\mu(E)<\delta\).
For such $E$ and for any $n$, by the dominating condition, 
\(|\int_E |f_n|\,d\mu|<\epsilon\) as well, thus
\(\Phi\) is uniformly integrable.

We now show an
example that satisfies Vitali's lemma assumptions but not
Lebesgue's theorem. For all \(n\in\N\)
let \(f_n: \R\to\R\) defined by 
\begin{align*}
a_0 &= 0 \\
a_n &= \sum_{j=1}^n 1/n \\
f_n &= \chhi_{[a_n,a_{n+1}]}.
\end{align*}
Clearly \(\lim_{n\to\infty}=0\) 
and also \(\lim_{n\to\infty} \|f_n\|_1 = 0\)
but there is no dominating function for \(\{f_n\}_{n\in\N}\)
in \(L^1(\R,m)\).


\itemch{f}
% Define \(f_n = n\chhi_{(0,1/n^2]}\).
Define
\begin{equation*}
f_n(x) = \left\{
\begin{array}{ll}
n^2        & \qquad 0 < x \leq 1/2n \\
-n^2       & \qquad 1/2n < x < 1/n \\
0          & \qquad x=0\;\vee\; x \geq 1/n
\end{array}\right.
\end{equation*}
Clearly \(\lim_{n\to\infty} f_n(x) = 0\) for all \(x\in[0,1]\)
and \(\int_{[0,1]} f_n\,dm = 0\) but
\begin{equation*}
\int_{[0,1/2n]} f_n\,dm = n/2.
\end{equation*}




\itemch{g}
The triangle inequality \(\rho(A,B)\leq \rho(A,C) + \rho(B,C)\)
is trivial. To show completeness we follow \cite{Oxtoby1980} chapter~10.
Say \(\{E_j\}_{j\in\N}\) a Cauchy sequence.
For each $j$ there is \(n_j > n_{j-1}\) such that \(\rho(E_m,E_n)<2^{-j}\)
whenever \(m,n\geq j\). Let \(F_j=E_{n_j}\), 
we have \(\rho(F_j,F_k)<2^{-j}\) for all \(k>j\). Define
\begin{equation*}
H_j = \bigcap_{k=j}^\infty F_k \qquad 
E = \bigcup_{j=1}^\infty H_j.
\end{equation*}
The set $E$ consists of points that belong to all 
but a finite number of sets \(\{f_j\}_{j\in\N}\).
If \(x\in E\vartriangle  F_j\) then there are two cases
\begin{itemize}
\itemch{i} 
\(x\in  E\setminus F_j\) then we look for the first $k$
such that \(x\in F_{j+k}\).
\itemch{ii} 
\(x\in  F_j\setminus E\) then we look for the first $k$
such that \(x\notin F_{j+k}\).
\end{itemize}
In both cases \(x\in F_{j+k-1} \vartriangle F_{j+k}\).

For each  \(x\in H_j\vartriangle F_j = F_j \setminus H_j\)
by looking at the first $k$ such that \(x\notin F_{j+k}\)
we see that
\begin{equation*}
x \in F_{j+k-1} \vartriangle F_{j+k}.
\end{equation*}

Using the simple rule
\((A\vartriangle B) \cup (A\vartriangle B) \subset B\vartriangle C \),
now clearly
\begin{equation*}
E\vartriangle  F_j
\subset (E\vartriangle H_j)\cup (H_j\vartriangle F_j)
\subset \bigcup_{k=1}^\infty F_{j+k-1} \vartriangle F_{j+k}.
\end{equation*}
Consequently
\begin{equation*}
m(E\vartriangle  F_j)
\leq \sum_{k=1}^\infty m(F_{j+k-1} \vartriangle F_{j+k})
\leq \sum_{k=1}^\infty 2^{-(j+k)} = 2^{1-j}.
\end{equation*}
For any \(n\geq n_j\) we have
\begin{align*}
\rho(E,E_n) 
&= m\bigl((E\vartriangle F_j) \vartriangle
          (E_{n_j}\vartriangle E_n)\bigr) \\
&\leq m(E\vartriangle F_j) + m(E_{n_j}\vartriangle E_n) 
< 2^{1-j}+2^{-j}.
\end{align*}

Thus completeness of the metric space $S$ of measurable sets 
modulu sets of zero-measure with the metric \(\rho\) was shown.

The continuity of \(E\mapsto \int_E f_n\,d\mu\) is immediate
by \(f_n\in L^1(\mu)\).
By Exercise~5.13(b) there exists $N$ and  an open set 
\(V = \{E\in S: \rho(E_0,E) < \delta)\}\)
of \(E_0\) such that \eqref{eq:ex:6.10} holds for any \(E\in V\) and \(n>N\).
If \(\mu(A)<\delta\) then \(B = E_0 \setminus A\in V\)
and \(C = E_0 \cup A\in V\).
Thus
\begin{equation*}
\left|\int_A f_n\right| 
= \left|\int_{E_0\setminus A} f_n + \int_{E_0\cup A} f_n\right| 
\leq \left|\int_{E_0\setminus A} f_n\right| + \left|\int_{E_0\cup A} f_n\right| 
< 2\epsilon.
\end{equation*}

We pick some \(0<\delta'<\delta\) such that
\(|\int_E f_j\,d\mu| < \epsilon\) whenever \(1\leq j \leq N\).
Now if \(\mu(A)<\delta'\) then
for any \(n>N\) we have
\begin{equation*}
\left|\int_A f_n\,d\mu\right|
\leq \left|\int_A f_N\,d\mu\right| + \left|\int_A (f_n - f_N)\,d\mu\right|
\leq \epsilon + 2\epsilon < 3\epsilon
\end{equation*}
If \(n\leq N\) then \(|\int_A f_n\,d\mu| < \epsilon < 3\epsilon\) trivially.
Thus
\begin{equation*}
\left|\int_A f_n\,d\mu\right| < 3\epsilon
\end{equation*}
for any \(n\in\N\) and $A$ such that \(\mu(A) < \delta'\).
Thus \(\{f_n\}_{n\in\N}\) are uniformly integrable, and by Vitali's lemma
the desired result follows.
\end{itemize}


%%%%%%%%%%%%%%
\begin{excopy}
Suppose \(\mu\) is a positive measure on $X$, \(\mu(X)<\infty\),
\(f_n \in L^1(\mu)\) for \(n=1,2,3,\ldots\),
\(f_n(x)\to f(x)\;\aded\),
and there exists \(p>1\) and \(C<\infty\) such that
\(\int_X |f_n|^p\,d\mu<C\) for all $n$. Prove that
\begin{equation*}
\lim_{n\to\infty} \int_X |f - f_n|\,d\mu = 0.
\end{equation*}
\emph{Hint}: \(\{f_n\}\) is uniformly integrable.
\end{excopy}

By negation there exists \(\epsilon>0\) and 
a sequences \(\{\delta_n\}_{n\in\N}\) and sets \(\{A_n\}_{n\in\N}\)
such that 
\begin{align*}
\mu(A_n) &< \delta_n = \epsilon/n \\
\left|\int_{A_n} f_n\,d\mu \right| &\geq \epsilon.
\end{align*}

Let $q$ be the exponent conjugate of $q$.
By H\"older inequality
\begin{equation*}
\left|\int_{A_n} f_n\,d\mu \right|
\leq \int_{A_n} |f_n|\,d\mu 
\leq 
\iffalse
     \left(\int_{A_n} |f_n|^p\,d\mu\right)^{1/p}
     \left(\int_{A_n} 1^q\,d\mu\right)^{1/q}
= 
\fi
\left(\int_{A_n} |f_n|^p\,d\mu\right)^{1/p} \bigl(\mu(A_n)\bigr)^{1/q}
\end{equation*}
Hence
\begin{equation*}
\int_{A_n} |f_n|^p\,d\mu
\geq \left( 
      \left|\int_{A_n} f_n\,d\mu \right| \bigm/ \bigl(\mu(A_n)\bigr)^{1/q}
     \right)^p
\geq \bigl( \epsilon / (\epsilon/n)^{1/q} \bigr)^p 
= (n\epsilon^{1-1/q})^p
= n^p\epsilon.
\end{equation*}
Which gives the contradiction
\begin{equation*}
\int_X |f_n|^p\,d\mu \geq \int_{A_n} |f_n|^p\,d\mu \geq n^p\epsilon > C
\end{equation*}
For sufficiently large $n$.


%%%%%%%%%%%%%%
\begin{excopy}
Let \frakM\ be the collection of all sets $E$ in the unit interval \([0,1]\) such that either $E$ or its complement is at most countable.
Let \(\mu\) be the counting measure on this \(\sigma\)-algebra \frakM.
If \(g(x) = x\) for \(0\leq x \leq 1\),
show that $g$ is not \frakM-measurable, although the mapping
\begin{equation*}
f \to \sum xf(x) = \int fg\,d\mu
\end{equation*}
makes sense for every \(f\in L^1(\mu)\) and defines a bounded linear functional
on \(L^1(\mu)\). Thus \((L^1)^* \neq L^\infty\) in this situation.
\end{excopy}

The set \(g^{-1}([0,1/2]) = [0,1/2]\) is not measurable, hence $g$ is not.
Indeed this measurable space is not \(\sigma\)-finite.

%%%%%%%%%%%%%%
\begin{excopy}
Let \(L^\infty = L^\infty(m) \), where $m$ is a Lebesgue measure on
\(I=[0,1]\). Show that there is a bounded linear functional \(\Lambda \neq 0\)
on \(L^\infty\) that is $0$ on \(C(I)\), and therefore there is no
\(g\in L^1(m)\) that satisfies
\(\Lambda f = \int_I fg\,dm\) for every \(f\in L^\infty\).
Thus \((L^\infty)^* \neq L^1\).
\end{excopy}

\(C(I)\) is a closed subspace of \((L^\infty)\).
By Hahn Banach Theorem
\index{Hahn Banach}
Theorem~5.16 \cite{RudinRCA87} 
and its consequence Theorem~5.19 there is a functional 
\(\Lambda\in (L^\infty)^*\) such that 
\(\Lambda f = 0\) for all \(f\in C(I)\) but 
\(\Lambda(\chhi_{[0,1/2]}) \neq 0\). Viewing \(C(I)\subset L^2\)
as a subspace of Hilbert space, if \(\Lambda\) may be represented
by inner multiplication by the zero function, but it surely cannot
represent \(\Lambda\) for the \(\chhi{[0,1/2]}\) case.


%%%%%%%%%%%%%%%%%
\end{enumerate}

 % \setcounter{chapter}{6}  % -*- latex -*-
% $Id: rudinrca7.tex,v 1.15 2006/09/05 20:39:02 yotam Exp $


%%%%%%%%%%%%%%%%%%%%%%%%%%%%%%%%%%%%%%%%%%%%%%%%%%%%%%%%%%%%%%%%%%%%%%%%
%%%%%%%%%%%%%%%%%%%%%%%%%%%%%%%%%%%%%%%%%%%%%%%%%%%%%%%%%%%%%%%%%%%%%%%%
%%%%%%%%%%%%%%%%%%%%%%%%%%%%%%%%%%%%%%%%%%%%%%%%%%%%%%%%%%%%%%%%%%%%%%%%
\chapterTypeout{Differentiation} % 7

%%%%%%%%%%%%%%%%%%%%%%%%%%%%%%%%%%%%%%%%%%%%%%%%%%%%%%%%%%%%%%%%%%%%%%%%
%%%%%%%%%%%%%%%%%%%%%%%%%%%%%%%%%%%%%%%%%%%%%%%%%%%%%%%%%%%%%%%%%%%%%%%%
\section{Notes}

%%%%%%%%%%%%%%%%%%%%%%%%%%%%%%%%%%%%%%%%%%%%%%%%%%%%%%%%%%%%%%%%%%%%%%%%
\subsection{Notations}

\paragraph{Null Set.} We use the term
\emph{nullset}
\index{nullset}
of \cite{Oxtoby1980}
for a set of measure zero.

We denote the \emph{length} of an interval $I$ by \(\ell(I)\).

%%%%%%%%%%%%%%%%%%%%%%%%%%%%%%%%%%%%%%%%%%%%%%%%%%%%%%%%%%%%%%%%%%%%%%%%
\subsection{Outer Measure}

In Rudin's treatment \cite{RudinRCA87}, measure theory is developed
without relying in the notion of outer measure.
In contrast, Royden gives a the following definition
(\cite{Royden} Chapter~3, Section~2 page~56).

\paragraph{Definition.}
\index{outer measure}
\index{measure!outer}
The \index{outer measure} of a set \(A\subset \R\) is
\begin{equation*}
m^*(A) = \inf_{A\subset \cup I_n} \ell(I_n).
\end{equation*}
When can generalize this to sets in \(\R^n\) and also show that
\begin{equation*}
m^*(A) = \inf \{m(G): A\subset G \wedge G \;\mathrm{is\ open}\}.
\end{equation*}

%%%%%%%%%%%%%%%%%%%%%%%%%%%%%%%%%%%%%%%%%%%%%%%%%%%%%%%%%%%%%%%%%%%%%%%%
\subsection{Lemma of Vitali}

We bring the 
Lemma of Vitali
\index{Vitali!of Vitali}
from \cite{Royden}.

\begin{llem} \label{lem:vitali}
Let $E$ be a set of finite measure and \frakI\ a set of intervals
that covers $E$ in the sense of Vitali, that is
\begin{equation*}
\forall x\in E\, \forall \epsilon>0\;
 \exists I\in\frakI\, x\in I \wedge \ell(I) < \epsilon.
\end{equation*}
Then given \(\epsilon>0\) there exists a finite subset of intervals 
\(\{I_n\}_{n=1}^N \subset \frakI\) such that 
\begin{equation}
m\left(E \setminus \cup_{n=1}^N I_n\right) < \epsilon>0.
\end{equation}
\end{llem}

\textbf{Note} Originally, the lemma deals with \emph{outer} measure,
but here we use less generalized version.

\begin{thmproof}
\Wlogy\ we may assume that \(\{\ell(I): I\in\frakI\}\) is bounded,
otherwise we pick some finite measure open set \(O\supset E\) 
and intersect each interval with this opens set, producing new interval(s)
that still satisfy the requirements.
Form a sequence \(\{I_j\}_{j\in\N}\) by induction. Say 
\(\{I_j\}_{j=1}^n\) were picked. Define
\begin{align*}
U_n &= \cup_{j=1}^n \overline{I_j} \\
k_n &= \sup \{\ell(I): I\in\frakI \wedge I \cap U_n = \emptyset\}.
\end{align*}
If \(E\subset U_n\) we are done, otherwise \(0< k_n < \infty\).
Pick \(I_{n+1}\) such that \(\ell(I_{n+1}) > k_n/2\).
Clearly \(\lim_{n\to\infty} \ell(I_n) = 0\), hence there exists $N$
such that
\begin{equation*}
\sum_{n=N+1}^\infty \ell(I_n) < \epsilon/5.
\end{equation*}
Let \(x\in R\) and pick \(I\in\frakI\) such that \(I\cap U_N=\emptyset\).
Let $n$ be the smallest integer such that \(I\cap I_n \neq \emptyset\).
Clearly \(n>N\) 
and by our method of forming the sequences 
\(\ell(I) \leq k_{n-1} \leq 2(I_n)\). If \(c_n\) is the center of \(I_n\) then
\begin{equation*}
|x-c_n| \leq \ell(I) + \ell(I_n)/2 \leq \frac{5}{2} \ell(I_n).
\end{equation*}
Define  $5$-time expansions of \(I_n\)
\begin{equation*}
J_n = \left[c_n - \frac{5}{2} \ell(I_n),\, c_n + \frac{5}{2} \ell(I_n)\right]
\end{equation*}
and now  \(x\in J_n\) and so
\begin{equation*}
R \subset \cup_{n=N+1}^\infty J_n
\end{equation*}
Hence
\begin{equation*}
m(R) 
\leq \sum_{n=N+1}^\infty \ell(J_n)
= 5\sum_{n=N+1}^\infty \ell(I_n)
< \epsilon.
\end{equation*}
\end{thmproof}


%%%%%%%%%%%%%%%%%%%%%%%%%%%%%%%%%%%%%%%%%%%%%%%%%%%%%%%%%%%%%%%%%%%%%%%%
\subsection{Lebesgue Differentiation Theorem}

Here is a theorem given in \cite{Royden} Chapter~5, Section~1, page~100,
Theorem~3.

\begin{llem} \label{lem:leb:diff}
If \(f:[a,b]\to\R\) be a non decreasing function
then $f$ is differentiable almost everywhere.
The derivative \(f'\) is measurable and 
\begin{equation*}
\int_a^b f'(x)\,dx \leq f(b) - f(a). \label{eq:lem:leb:diff}
\end{equation*}
In particular, \(f'\in L^1([a,b],m)\).
\end{llem}
\begin{thmproof}
Let \(D\subset [a,b]\) be the set of points where $f$ is discontinuous.
Since $f$ is non decreasing we now that $D$ is countable.

Define
\begin{equation} \label{eq:7:DfMDfm}
D^+f(x) = \varlimsup_{h\to 0} \frac{f(x+h)-f(x)}{h}
\qquad
D^-f(x) = \varliminf_{h\to 0} \frac{f(x+h)-f(x)}{h}
\end{equation}
where \(x,x+h \in [a,b]\) of course.

We would like to be able to use limits such that that $h$ run over
some countable set for all \(x\in[a,b]\).
\begin{align} 
\Delta^+f(x) &\eqdef \label{eq:7:DfMQ} 
   \varlimsup_{\stackrel{h\to 0}{h\in\Q}} \frac{f(x+h)-f(x)}{h} \\
\Delta^-f(x) &\eqdef \label{eq:7:DfmQ}
   \varliminf_{\stackrel{h\to 0}{h\in\Q}} \frac{f(x+h)-f(x)}{h}
\end{align}

\paragraph{Claim:} 
\begin{align}
D^+f(x) &= \Delta^+f(x)  \label{eq:DfM:eqQ} \\
D^-f(x) &= \Delta^-f(x)  \label{eq:Dfm:eqQ}
\end{align}
We will show \eqref{eq:DfM:eqQ} and \eqref{eq:Dfm:eqQ} 
will follow from \(\varliminf f = -\varlimsup -f\).
Clearly \(\Delta^+f(x) \leq D^+f(x) \) since \(\Q\subset\R\).
For this claim, fix \(x\in[a,b]\) and let \(u \eqdef D^+f(x)\).
Pick an arbitrarily small \(\delta>0\).
There is \(h\neq 0\) such that \(|h|<\delta\)
\begin{equation*}
\bigl(f(x+h) - f(x)\bigr)/h > u - \epsilon \qquad x+h\in(a,b).
\end{equation*}
Let us assume that \(h>0\). The negative case can be similarly treated.
Put
\begin{equation*}
Q(x,h) = \{q\in\Q: q\geq h \wedge x+q\in[a,b]\}
\end{equation*}
Clearly \(Q(x,h)\neq \emptyset\) and 
 \(f(x+h)\leq f(x+q)\) for each \(q\in Q(x,h)\).
Thus
\begin{equation*}
\varlimsup_{\stackrel{q\to h}{q\in\Q(x,h)}}  \frac{f(x+q) - f(x)}{q}
\geq \varlimsup_{\stackrel{q\to h}{q\in\Q(x,h)}}  \frac{f(x+h) - f(x)}{q}
= \varlimsup_{\stackrel{q\to h}{q\in\Q(x,h)}}  \frac{f(x+h) - f(x)}{h}
= u.
\end{equation*}
Hence \(\Delta^+f(x) \geq D^+f(x)\) for any \(x\in[a,b]\)
and the claim \eqref{eq:DfM:eqQ} is shown.

Define the ``bad'' sets
\begin{align}
E       &\eqdef \left\{x\in[a,b]: D^-f(x) < D^+f(x)\right\} \notag \\
E_{u,v} &\eqdef \left\{x\in[a,b]: D^-f(x) < v < u < D^+f(x)\right\} 
        \label{eq:7:Euv}
\end{align}
and clearly
\begin{equation} \label{eq:ex7.14:Eu}
E = \bigcup_{u,v\in\Q} E_{u,v}
\end{equation}
Since 
\begin{align*}
\Delta^+f(x) &= \inf_{r\in\Q^+} 
    \sup\left\{ \frac{f(x+q)-f(x)}{q} :
                q\in\Q \swedge 0<|q|<r \swedge x+q\in[a,b] \right\}
\\
\Delta^-f(x) &= \sup_{r\in\Q^+} 
    \inf\left\{ \frac{f(x+q)-f(x)}{q} :
                q\in\Q \swedge 0<|q|<r \swedge x+q\in[a,b] \right\}
\end{align*}
the sets $E$ and \(E_{u,v}\) are measurable! They could be defined
by countable intersection and unions of open sets.
Noting that inverse image of intervals via $f$ are intervals since
$f$ is non decreasing.
This was 
the whole reason for the claim, thus avoiding the usage of outer measure
and using a simplified lemma (\ref{lem:vitali}) of Vitali.

We note that 
$f$ is differentiable on \([a,b]\setminus E\) 
and obviously \(E_{u,v} = \emptyset\) if \(u\geq v\).
Since the above \eqref{eq:ex7.14:Eu} is a countable union 
it will suffice to show that \(m(E_{u,v}) = 0\) whenever \(u < v\).
Fix some \(u,v\in\R\) such that \(v<u\).
Assume by negation 
\begin{equation} \label{eq:7:vit:sgt0}
s \eqdef m(E_{u,v}) > 0.
\end{equation}
Pick arbitrary \(\epsilon>0\) and take an open \(G \supset E_{u,v}\)
such that \(m(G) < s + \epsilon\)
For each \(x\in E_{u,v}\) we can find some $h$ such that
the interval \([x,x+h]\) or  \([x+h,x]\) is contained in $G$ and 
\begin{equation} \label{eq:7:vit:ltv}
\bigl(f(x+h) - f(x)\bigr)/h < v.
\end{equation}
By local lemma (Vitali)~\ref{lem:vitali} 
there is a finite set of disjoint open intervals 
\(\{I_j\}_{j=1}^N = \{(a_j,b_j)\}_{j=1}^N\) (where \(a_j<b_j\))
that 
cover a subset \(A = \cap_{j=1}^N I_j \cap  E_{u,v}\) 
and \(m(A) > s - \epsilon\). Summing \eqref{eq:7:vit:ltv} we get
\begin{equation} \label{eq:7:vit:ltvs}
\sum_{j=1}^N f(b_j) - f(a_j) 
< v\sum_{j=1}^N b_j - a_j < v\,m(G)  <  v(s+\epsilon).
\end{equation}
Now for each \(y\in A\) there is an interval
 \([y,y+k]\) or  \([y+k,y]\) is contained in some \(I_j\) such that 
\begin{equation} \label{eq:7:vit:gtu}
\bigl(f(y+k) - f(y)\bigr)/k > u.
\end{equation}
Applying the same lemma again, there are disjoint intervals
\(\{J_j\}_{j=1}^M = \{(c_j,d_j)\}_{j=1}^M\) (where \(c_j<d_j\))
that 
cover a subset \(B = \cap_{j=1}^M J_j \cap  A\) 
such that \(m(B) > s-2\epsilon\). 
Similar summation of~\eqref{eq:7:vit:gtu} gives
\begin{equation} \label{eq:7:vit:gtus}
\sum_{j=1}^M f(d_j) - f(c_j) > u \sum_{j=1}^M d_j - c_j >  u(s - 2\epsilon).
\end{equation}
For the segment \(I_n\), since $f$ is nondecreasing we have
\begin{equation*}
\sum_{\stackrel{1\leq j \leq M}{J_j \subset I_n}} f(d_j) - f(c_j) 
\leq f(b_n) - f(a_n).
\end{equation*}
Summing over all \(\{I_j\}_{j=1}^N\) gives
\begin{equation} \label{eq:vit:dcltba}
\sum_{j=1}^M f(d_j) - f(c_j) \leq \sum_{j=1}^N f(b_j) - f(a_j) 
\end{equation}
Combining 
\eqref{eq:7:vit:ltvs},
\eqref{eq:7:vit:gtus} and
\eqref{eq:vit:dcltba}
gives
\begin{equation*}
u(s - 2\epsilon) \leq v(s+\epsilon).
\end{equation*}
Since \(\epsilon\) is arbitrarily small, we have
\(us \leq vs\) and by \eqref{eq:7:Euv} \(s=0\) contradiction 
to~\eqref{eq:7:vit:sgt0}. Thus \(m(E)=0\) and $f$ is differentiable
almost everywhere.

To show the final part of this differentiation theorem, let 
\begin{equation*}
g_n(x) \eqdef n\bigl(f(x+1/n) - f(x)\bigr) 
                        \qquad \forall x>b,\; f(x)\eqdef f(b).
\end{equation*}
Clearly \(\lim_{n\to\infty} g_n(x) = f'(x)\) hence it is measurable.
Now by Fatou's lemma
\index{Fatou}
\begin{align*}
\int_a^b f'(x)\,dx
&\leq \varliminf_{n\to\infty} \int_a^b g_n(x)\,dx  
= \varliminf_{n\to\infty} n \int_a^b \bigl(f(x+1/n) - f(x)\bigr) \,dx  \\
&= \varliminf_{n\to\infty} 
   \left(n \int_b^{b+1/n} f(x)\,dx  - \int_a^{a+1/n} f(x)\,dx  \right)
 = \varliminf_{n\to\infty} 
   \left(f(b)  - n\int_a^{a+1/n} f(x)\,dx  \right) \\
& \leq f(b) - f(a).
\end{align*}
Hence \eqref{eq:lem:leb:diff} holds.
\end{thmproof}



%%%%%%%%%%%%%%%%%%%%%%%%%%%%%%%%%%%%%%%%%%%%%%%%%%%%%%%%%%%%%%%%%%%%%%%%
\subsection{Absolute Continuity and Bounded Variation.}

Trvial lemma showing that absolute continuity is
stronger than having bounded variation.

\begin{llem}
If \(f:[a,b]\to\C\) is an absolute continuous function,
then it has bounded variation.
\end{llem}
\begin{thmproof}
Pick \(\epsilon=1\) then there exists \(\delta>0\)
such that 
\(\sum_{j=1}^n |f(b_k) - f(a_k)| < \epsilon=1\)
whenever
\(\sum_{j=1}^n b_k - a_k| < \delta\)
where \(a \leq a_k \leq b_k \leq b\).

Now given an arbitrary partition \(a = t_0 < t_1 < \cdots < t_m = b\)
we can extended it to 
Now given a partition \(a = u_0 < u_1 < \cdots < u_n = b\)
such that 
\(u_k - u_{k-1} < \delta/2\) for all \(k\in\N_n\) and 
for any \(j\in\N_m\) there exists \(k\in\N_n\) 
such that \(t_j = u_k\).
We split the partitions to chunks bounded by \(\delta/2\) by defining
\begin{equation*}
c(k) = \min \{k\in\Z^+: u_k - a \geq n\delta/2\}.
\qquad 0\leq n \leq \lceil 2(b-a)/\delta \rceil.
\end{equation*}
Clearly \(u_{c(n)} - u_{c(n-1)} < \delta/2\).
Now
\begin{equation*}
\sum_{j=1}^m |f(t_j) - f(t_{j-1})|
\leq \sum_{j=1}^n |f(u_j) - f(u_{j-1})|
= \sum_k \sum_{j=c(k) + 1}^{c(k+1)} |f(u_j) - f(u_{j-1})|
\leq \lceil 2(b-a)/\delta \rceil\cdot 1.
\end{equation*}
Thus $f$ has bounded variation.
\end{thmproof}



%%%%%%%%%%%%%%%%%%%%%%%%%%%%%%%%%%%%%%%%%%%%%%%%%%%%%%%%%%%%%%%%%%%%%%%%
\subsection{Detailed Reference.}

In the \textsc{Notes and Comments} appendix for this chapter,
there is a reference for elementary proof of almost everywhere
differentiability of monotone function.
The full reference is:\\
\emph{A Geometric Proof of the Lebesgue Differentiation Theorem},
\textbf{Donald Austin},
Proceedings of the American Mathematical Society, 
Vol.~16, No.~2 (Apr.,~1965), pp.~ 220--221.


%%%%%%%%%%%%%%%%%%%%%%%%%%%%%%%%%%%%%%%%%%%%%%%%%%%%%%%%%%%%%%%%%%%%%%%%
%%%%%%%%%%%%%%%%%%%%%%%%%%%%%%%%%%%%%%%%%%%%%%%%%%%%%%%%%%%%%%%%%%%%%%%%
\section{Exercises} % pages 156-159

%%%%%%%%%%%%%%%%%
\begin{enumerate}
%%%%%%%%%%%%%%%%%


%%%%%%%%%%%%%% 1
\begin{excopy}
Show that \(|f(x)| \leq (Mf)(x)\) at every Lebesgue point of $f$ if 
\(f\in L^1(\R^k)\).
\end{excopy}

Say $x$ is a Lebesgue point
\index{Lebesgue point}
then
\begin{equation*}
\lim_{r\to\infty} \frac{1}{m(B_r)} \int_{B(x,r)} |f(y)-f(x)|\,dm(y) = 0.
\end{equation*}
Pick \(\epsilon>0\) and \(r>0\) such that
\begin{equation*}
D = \frac{1}{m(B_r)} \int_{B(x,r)} |f(y)-f(x)|\,dm(y) < \epsilon.
\end{equation*}
Hence
\begin{align*}
D 
&\geq \frac{1}{m(B_r)} \left( \int_{B(x,r)} |f(x)|\,dm(y) - 
                             \int_{B(x,r)} |f(y)|\,dm(y) \right) \\
&=  |f(x)| - \frac{1}{m(B_r)} \int_{B(x,r)} |f(y)|\,dm(y) 
\end{align*}
Combing the above, we get
\begin{equation*}
|f(x)| \leq \frac{1}{m(B_r)} \int_{B(x,r)} |f(y)|\,dm(y) + \epsilon.
\end{equation*}
Since \(\epsilon>0\) was arbitrary, we actually have 
the desired
\begin{equation*}
|f(x)| \leq \sup_{r>0} \frac{1}{m(B_r)} \int_{B(x,r)} |f|\,dm
\end{equation*}
inequality.


%%%%%%%%%%%%%%
\begin{excopy}
For \(\delta>0\), let \(I(\delta)\) be the 
segment \((-\delta,\delta) \subset \R^1\). Given \(\alpha\) and \(\beta\),
\(0\leq\alpha\leq \beta\leq 1\),
construct a measurable set \(E\subset \R^1\) so that the upper and lower
limits of 
\begin{equation*}
\frac{m\bigl(E\cap I(\delta)\bigr)}{2\delta}
\end{equation*}
are \(\beta\) and \(\alpha\) respectively, as \(\delta\rightarrow 0\).

(Compare this with Section~7.12.)
\end{excopy}

Let \(0< s,u < 1\). We will soon determine their actual values.
Define % the decreasing sequence \(s_n = s^n\) for \(n\geq 0\) and
the sequence of clopen intervals 
\begin{equation*}
I_n = (s^n,s^{n-1}] \qquad n \geq 0.
\end{equation*}
Clearly \((0,1] = \disjunion_{n\geq 0} I_n\).
Define ``inter-points'' and open sub-intervals.
\begin{align*}
t_n &= s^n + u\cdot m(I_n) \\
U_n &= (s^n, t_n) \subsetneq I_n \qquad n \geq 0.
\end{align*}
and let \(U = \cup_n U_n\), put \(U^- = \{x\in\R: -x \in U\)
and finally \(E = U \disjunion U^-\).
Clearly 
\begin{equation*}
\frac{m\bigl(E\cap I(\delta)\bigr)}{2\delta} 
= m\bigl(U\cap I(\delta)\bigr) / \delta.
\end{equation*}
Thus it is sufficiently to look at the limits of 
\(m(U\cap [0,\delta]) / \delta\).
For any \(\delta\) looking at the 
the subsequence of \(\{U_n\}_{n\geq 0}\) below \(\delta\)
we can see that
\begin{equation*}
\liminf_{\delta\to 0} m(U\cap [0,\delta]) / \delta = u.
\end{equation*}
Similarly, if we look at the ``shifted segments'' \(S_n = (t_{n+1},t_n)\)
we can see that
\begin{equation*}
\limsup_{\delta\to 0} m(U\cap [0,\delta]) / \delta 
= u / (s(1-u) + u) = u / (s + (1-s)u) > u.
\end{equation*}
Trivial solving give the desired values \(u=\alpha\)
and for
\begin{equation*}
u / (s(1-u) + u) = \beta
\end{equation*}
we have
\begin{equation*}
s 
= (u / \beta - u) / (1-u)
= (\alpha / \beta - \alpha) / (1-\alpha).
\end{equation*}

In Section~7.12 we see that the limit exists and must be $0$ or $1$
almost everywhere. 
So a case like constructed above may happen only on a nullset (measure zero). 

%%%%%%%%%%%%%%
\begin{excopy}
Suppose that $E$ is a measurable set of real numbers with arbitrarily
small periods. Explicitly,
this means that there are positive numbers \(p_i\), converging to $0$
as \(i\to \infty\), so that 
\begin{equation*}
E + p_i = E \qquad (i=1,2,3,\ldots).
\end{equation*}

Prove that then either $E$ or its complement has measure $0$.

\emph{Hint:} Pick \(\alpha\in\R^1\), 
put \(F(x) = m(E\cap[\alpha,x])\) for \(x>\alpha\), show that
\begin{equation*}
F(x + p_i) - F(x - p_i) = F(y + p_i) - F(y - p_i)
\end{equation*}
if \(\alpha + p_i < x < y\). 
What does this imply about \(F(x)\) if \(m(E)>0\)?
\end{excopy}

Let \(n = \lfloor (y-x)/2p_i \rfloor\).
Then \(x - p_i < (y-2np_i) - p_i < (y-2np_i) + p_i\).
Now
\begin{eqnarray*}
F(x + p_i) - F(x - p_i) 
&=& m(E \cap [x-p_i, x+p_i]) \\
&=& m(E \cap ([x-p_i, (y-2np_i) - p_i] \cup [(y-2np_i) - p_i, x+p_i]) \\
&=& m(E \cap ([x-p_i, (y-2np_i) - p_i]) \\
& & + m(E \cap [(y-2np_i) - p_i, x+p_i]) \\
&=& m((E - 2p_i) \cap ([x-p_i, (y-2np_i) - p_i]) \\
& & + m(E \cup [(y-2np_i) - p_i, x+p_i]) \\
&=& m(E \cap [(y-2np_i) - p_i, x+p_i]) \\
& & + m(E \cap ([x+p_i, (y-2np_i) + p_i]) \\
&=& m(E \cap [(y-2np_i) - p_i, (y-2np_i) + p_i]) \\
&=& m(E \cap [y - p_i, y + p_i]) \\
&=& F(y + p_i) - F(y - p_i)
\end{eqnarray*}

By Theorem~7.18 \cite{RudinRCA87}, \(F(x)\) is differentiable
almost everywhere. By the above equality, its value is constant
simply by looking at 
\begin{equation*}
F'(x) = \lim_{j\to\infty} \frac{F(x+p_j) - F(x-p_j)}{2p_j}.
\end{equation*}
This value is actually the 
metric density
\index{metric density}
of $E$ and by the discussion in Section~7.12 \cite{RudinRCA87}
this constant must be $0$ or $1$. If \(m(E)>0\) it must be $1$
and thus \(F(x) = x-\alpha\) for \(x>\alpha\).
Since \(\alpha\) was arbitrary, for any interval $I$ we have
\(m(E\cap I) = m(I)\) and so \(m(\R\setminus E) = 0\).


%%%%%%%%%%%%%%
\begin{excopy}
Call $t$ a 
\emph{period}
\index{period}
of the function $f$ on \(\R^1\) if \(f(x+t)=f(x)\) for all \(x\in\R^1\).
Suppose $f$ is a real Lebesgue measurable function with periods $s$ and $t$
whose quotient is irrational.
Prove that there is a constant $c$ such that \(f(x)=c\;\aded\),
but that $f$ need not be constant.

\emph{Hint}: Apply Exercise~3 to the sets \(\{f>\lambda\}\).
\end{excopy}

We construct a sequence \(\{p_j\}_{j\in\N}\) such that 
\(\lim_{j\to\infty} p_j = 0\) and \(p_j > 0\) are periods of $f$.
Assume \(s>t\). Let (Similar to Euclid GCD algorithm)
\begin{align*}
p_1 &= s \\
p_2 &= t \\
p_n &= p_{n-2} - p_{n-1}\left\lfloor \frac{p_{n-2}}{p_{n-1}}\right\rfloor
       < p_{n-1}.
\end{align*}
Assume by negation that  $f$ is not constant almost everywhere.
We can find some \(\lambda\) such that 
\(\{f\leq\lambda\}\) and its complement \(\{f>\lambda\}\)
each is not a nullset.
The sequence \(\{p_j\}_{j\in\N}\) is also periods of these sets.
But by previous exercise, one of these two sets must be a nullset.


%%%%%%%%%%%%%% 5
\begin{excopy}
If \(A \subset \R^1\) and \(B \subset \R^1\), 
define \(A+B = \{a+b: a\in A, b\in B\}\).
Suppose \(m(A)>0\) and \(m(B)>0\).
Prove that \(A+B\) contains a segment, by completing the following outline:

There are points \(a_0\) and \(b_0\) where $A$ and $B$ have 
\index{metric density}
metric density~$1$.
Choose a small \(\delta>0\).
Put \(c_0 = a_0 + b_0\).
For each \(\epsilon\), positive or negative, define \(B_\epsilon\) 
to be the set of all \(c_0 + \epsilon - b\) for which 
\(b\in B\) and \(|b-b_0| < \delta\).
Then \(B_\epsilon \subset  (a_0 + \epsilon - \delta, a_0 + \epsilon + \delta)\).
If \(\delta\) was well chosen and \(|\epsilon|\) is sufficiently small,
it follows that $A$ intersects \(B_\epsilon\), so that 
\(A+B \supset (c_0 - \epsilon_0, c_0 + \epsilon_0)\) 
for some \(\epsilon_0 > 0\).

Let $C$ be 
\index{Cantor}
Cantor's ``middle thirds'' set and show that \(C+C\) is an interval,
although \(m(C)=0\).

(See also Exercise~19, Chap. 9.)
\end{excopy}

In this exercise solution, whenever we use $X$, we mean that
the expression or statement holds both for $A$ and for $B$.
\Wlogy, we may assume \(a_0=b_0=0\),
thus we have \(c_0=0\).
Take \(\delta>0\) such that 
\begin{align*}
m\bigl(X\cap (-h,h)\bigr)/2h &> 2/3 \\ 
\end{align*}
for any \(0<h<\delta\).
Pick \(\epsilon_0 = \delta/3\).
For \(\epsilon < \epsilon_0\), we have
\begin{equation*}
X_\epsilon = \epsilon - \bigl(X \cap (-\delta,+\delta)\bigr)
\subset (\epsilon - \delta, \epsilon + \delta).
\end{equation*}
Clearly \(m(X_\epsilon) = m(X\cap(-\delta,\delta))\).
Now
\begin{equation*}
\bigl(X\cap(\epsilon-\delta,\epsilon+\delta)\bigr)
\setminus
\bigl(X\cap(-\delta,\delta)\bigr)
\supset 
(\epsilon-\delta,\epsilon+\delta) \setminus  (-\delta,\delta)
\supset (\delta,\delta + \epsilon)
\end{equation*}
therefore,
\begin{align*}
   m\bigl(X \cap (\epsilon-\delta, \epsilon+\delta)\bigr)
&\geq  m\bigl(X \cap (-\delta, +\delta)\bigr) - \epsilon.
\end{align*}
Thus
\begin{align*}
m\bigl(A \cap (\epsilon-\delta, \epsilon+\delta)\bigr) + m(B_\epsilon)
&\geq m\bigl(A \cap (-\delta, +\delta)\bigr) - \epsilon 
      + m\bigl(B\cap(-\delta,\delta)\bigr) \\
&\geq \bigl((2/3)\cdot2\delta - \epsilon\bigr) + (2/3)\cdot2\delta \\
&> 2\bigl((2/3)\cdot2\delta - \epsilon\bigr)
 > 2(4\delta/3 - \epsilon_0) = 2\delta.
\end{align*}
and so \(A\cap B_\epsilon \neq \emptyset\).

Equivalently, for all \(\epsilon\in(-\epsilon_0,+\epsilon_0)\), 
there are \(a\in A\) and \(b\in B\) such that \(a=\epsilon -b\),
that shows that \(\epsilon\in A+B\) and actually
\((-\epsilon,\epsilon) \subset A+B\).
 
\paragraph{Cantor Set.} 
We will show that \(C+C = [0,2]\).
Any arbitrary \(x\in[0,2]\) can be represented as
\begin{equation*}
x = \sum_{n=0}^\infty t_n 3^{-n}
\end{equation*}
where \(t_0\in\{0,1\}\) and \(t_n\in\{0,1,2\}\) for \(n\geq 1\).
We will construct \(a,b\in C\) such that \(x=a+b\).
Put \(c_{-1} = 0\), using
\begin{equation*}
v_n = t_n + 3c_{n-1}
\end{equation*}
we define \(a_n\),\(b_n\) and \(c_n\) 
by induction for \(n\geq 0\) by the following 6 cases
\begin{equation*}
(a_n,b_n,c_n) = 
\left
\{\begin{array}{ll}
(0,0,0) \quad & v_n = 0 \\
(0,0,1) \quad & v_n = 1 \\
(2,0,0) \quad & v_n = 2 \\
(2,0,1) \quad & v_n = 3 \\
(2,2,0) \quad & v_n = 4 \\
(2,2,1) \quad & v_n = 5
\end{array}\right.
\end{equation*}
Note that \(a_0 = b_0 = 0\), so  if we put
\begin{equation*}
a = \sum_{n=0}^\infty a_n 3^{-n} \in C
\qquad
b = \sum_{n=0}^\infty b_n 3^{-n} \in C
\end{equation*}
we get the desired \(x = a + b \in C + C\) .
 

%%%%%%%%%%%%%%
\begin{excopy}
Suppose $G$ is a subgroup of \(\R^1\) (relative to addition),
\(G\neq \R^1\), and $G$ is Lebesgue measurable. Prove that then \(m(G)=0\).

\emph{Hint}: Use Exercise~5.
\end{excopy}

Assume by negation \(m(G)>0\). By previous exercise \(G+G\) contains a segment.
But \(G+G=G\) since it is a group. Thus $G$ contains an interval 
\([a,b]\in G\) such that \(a<b\).
Pick arbitrary \(x\in \R\). Define
\begin{align*}
d &= b - a > 0\\
n &= \lfloor x / d \rfloor \in \Z \\
r &= x - nd \in [a,b) \subset G
\end{align*}
Now 
\begin{align*}
x = n(b-a) + r \in G
\end{align*}
and so we get the \(\R \subset G\) contradiction.


%%%%%%%%%%%%%%
\begin{excopy}
Construct a continuous monotonic function $f$ on \(\R^1\) so that $f$ is not
constant on any segment although \(f'(x)=0\;\aded\)
\end{excopy}

The function constructed in Section~7.16 Example~\ich{b} satisfies
the requirements.

%%%%%%%%%%%%%% 8
\begin{excopy}
Let \(V=(a,b)\) be a bounded segment in \(\R^1\).
Choose segments \(W_n\subset V\) in  such a way that their union $W$ is dense 
in $V$ and the set \(K = V \setminus W\) has \(m(K)>0\).
Choose continuous function \(\varphi_n\) so that 
\(\varphi_n(x)=0\) outside \(W_n\), \(0<\varphi_n(x)<2^{-n}\) in \(W_n\).
Put \(\varphi = \sum \varphi_n\) and define
\begin{equation*}
T(x) = \int_a^x \varphi(t)\,dt \qquad (a<x<b).
\end{equation*}
Prove the following statements:
\begin{itemize}
\itemch{a} $T$ satisfies the hypothesis of Theorem~7.26, with \(X=V\).
\itemch{b} $T$ is continuous, \(T'(x)=0\) on $K$, \(m(T(K)) = 0\).
\itemch{c} If $E$ is a nonmeasurable subset of $K$ 
           (See Theorem~2.22) and \(A=T(E)\), then \(\chhi_A\) 
           is Lebesgue measurable but \(\chhi_A \circ T\) is not.
\itemch{d} The functions \(\varphi_n\) can be so chosen that the resulting $T$
           is an \emph{infinitely differentiable} homeomorphism of $V$ onto some
           segment in \(\R^1\) and \ich{c} still holds.

\end{itemize}
\end{excopy}

Let \(\{q_j\}_{j\in\N}\) be a sequence consisting of all of \(\Q\cap V\).
Put \(d=b-a\) and let \(W_1 = (a_1,b_1)\) be an open segment such that
\(q_1 \in W_1\subset V\) and \(d_1 = b_1 - a_1 < 2^{-1}d\).
By induction, assume \(W_j\) where chosen for \(j<n\).
Pick the first \(q_k\) in our sequence such that 
\(q_k \notin \cup_{j<n} W_j\) and choose \(W_n = (a_n,b_n)\) such that
\begin{align*}
q_n &\in W_n \subset V \setminus \left(\cup_{j<n} W_j\right) \\
d_n &= b_n - a_n < 2^{-n} d
\end{align*}
Clearly the chosen intervals \(\{W_j\}_{j\in\N}\) are disjoint
and satisfy the requirements.

We now use Exercise~7.1 in \cite{RudinPMA85}
and  Example~3.11 (page 40) in \cite{Gelb1996}.
For each interval \(W_n = (a_n,b_n)\) we define 
\begin{equation*}
g_n(x) = \left\{\begin{array}{ll}
c_n\exp\left((a-b) \,/\, (x-a_n)^2(b_n-x)^2\right) \qquad & a_n < x < b_n \\
0  & \mathrm{otherwise}.
\end{array}
\right.
\end{equation*}
and choose \(c_n\) such that \(\varphi_n(x) \leq  2^{-n+1 } < 2^{-n}\).
Actually 
\begin{equation*}
c_n =   2^{-n+1 } \exp\left((b-a) \,/\, (x-a_n)^2(b_n-x)^2\right).
\end{equation*}

Note that if
\begin{equation*}
g_a(x) = 
\left\{
\begin{array}{ll}
e^{-1/(x-a_n)^2} \quad & a_n < x  \\
0  & \mathrm{otherwise}.
\end{array}\right.
\qquad
g_b(x) = 
\left\{
\begin{array}{ll}
e^{-1/(b_n-x)^2} \quad &  x < b_n \\
0  & \mathrm{otherwise}.
\end{array}\right.
\end{equation*}
Then \(g_a^{(k)}(a_n) = g_b^{(k)}(b_n) = 0\) for all \(k\in\N\)
and for \(x\in (a_n,b_n)\) we have
\begin{align*}
g_a(x)g_b(x) 
&= e^{-1/(x-a)^2} e^{-1/(b-x)^2} 
 = \exp\left(- 1/(x-a)^2 - 1/(b-x)^2\right) \\
&= \exp\left(\frac{(x-b) - (x-a)}{(x-a)^2 (b-x)^2}\right)
 = \exp\left(\frac{a-b}{(x-a)^2 (b-x)^2}\right) \\
&= \varphi_n(x)
\end{align*}
Each \(\varphi_n\) is continuous, and since their support sets
are disjoint, \(\varphi\) is also continuous.
Now that $T$ is defined we prove the above.
\begin{itemize}

\itemch{a}

Now we show that $T$
satisfies Theorem~7.26's requirements.
\begin{itemize}

\itemch{i} 
\(X=V\) is a union of open intervals and so $V$ is open.
The continuity of $T$ follows from continuity of \(\varphi\).

\itemch{ii}
$X$ is open and so measurable.
Let \(x_1,x_2\in V=(a,b)\) and  \(x_1 < x_2\).
There must be some \(W_n = (a_n,b_n) \subsetneq (x_1,x_2)\).
Since 
\begin{equation*}
\int_{W_n} \varphi_n(t)\,dt > 0
\end{equation*}
We have \(T(x_1) < T(x_2)\) and so $T$ is injective.
The fact that \(\varphi\) is continuous
also implies that $T$ is differentiable on $X$.
and \(T'(x)=\varphi(x)\) for all \(x\in X\).

\itemch{iii}
Trivially, \(m(T(V-X))=0\) since \(V-X=\emptyset\).
\end{itemize}

\itemch{b}
As previously shown in \ich{a}-\ich{ii}, 
\(T'(x)=\varphi(x)\) for all \(x\in X\),
in particular \(T(x)=0\) for \(x\in K\).

\itemch{c}
Let \(E\subset K\) be a non measurable set.
then \(A=T(E)\subset T(K)\) and clearly Lebesgue measurable
since \(m(T(K)) = 0\).

\itemch{d}
We already took this infinitely differentiability into account
in our construction of \(\{\varphi_n\}_{n\in\N}\).
\end{itemize}

%%%%%%%%%%%%%%
\begin{excopy}
Suppose \(0<\alpha < 1\). Pick $t$ so that \(t^\alpha = 2\). Then
\(t>2\) and the construction of Example~\ich{b} in Sec.~7.16 can be
carried out with \(\delta_n = (2/t)^n\). Show that the resulting function $f$
\index{Lip@\(\Lip\)}
\index{Lipschitz condition}
belongs to \(\Lip \alpha\) on \([0,1]\).
\end{excopy}

For each \(n\in\N\) it is easy to see that
\begin{equation*}
\sup_{0\leq x <y\leq1} \frac{|f(y)-f(x)|}{y-x}
= \frac{f\bigl((2/t)^n/2^n\bigr) - f(0)}{
        \left((2/t)^n/2^n - 0\right)^\alpha}
= \frac{2^{-n}}{\left((2/t)^n/2^n\right)^\alpha}
= \frac{2^{-n}}{t^{-n\alpha}}
= (t^\alpha/2)^n.
\end{equation*}
Since \((t^\alpha/2)=1\) we have \(f_n\in\Lip\alpha\)
which is a Banach space and closed. Hence \(f\in\Lip\alpha\).


%%%%%%%%%%%%%%
\begin{excopy}
If 
\index{Lip@\(\Lip\)}
\index{Lipschitz condition}
\(f\in \Lip 1\) on \([a,b]\), prove that $f$ is absolutely continuous
and that \(f'\in L^\infty\).
\end{excopy}

Pick arbitrary \(\epsilon>0\), choose some \(\delta < \epsilon\).
For any set of disjoint intervals \(\{(\alpha_j,\beta_j)\}_{j\in\N}\)
such that \(\sum_j (\beta_j - \alpha_j) < \delta\) we have
\begin{equation*}
\sum_j |f(\beta_j) - f(\alpha_j)|
\leq \sum_j 1\cdot|\beta_j - \alpha_j|
= \sum_j \beta_j - \alpha_j < \delta < \epsilon
\end{equation*}
Thus $f$ is absolutely continuous.
By Theorem~7.18 $f$ is differentiable \aded\ and 
by \(f\in\Lip 1\) we have \(\|f'\|_{\infty} \leq 1\) and thus
\(f'\in L^\infty\).


%%%%%%%%%%%%%% 11
\begin{excopy}
Assume that \(1<p<\infty\), $f$ is absolutely continuous on \([a,b]\),
\(f'\in L^p\), and \(\alpha = 1/q\) where $q$ is the exponent conjugate to $p$.
Prove that \(f\in \Lip\alpha\).
\end{excopy}

{\small (From Curtis T.~McMullen's\\
\texttt{\scriptsize
www.math.harvard.edu/{\~{}}ctm/home/text/class/harvard/212a/03/html/home/course/course.pdf})}

If \(a\leq x<y\leq b\) then
\begin{equation*}
|f(y)-f(x)| 
\leq \int_a^b |f'(t)| \cdot \chhi_{[x,y]}\,dt
\leq \|f'\|_p \cdot \|\chhi_{[x,y]}\|_q 
=    \|f'\|_p \cdot(y-x)^{1/q}.
\end{equation*}
Hence 
\begin{equation*}
\frac{|f(y)-f(x)|}{(y-x)^\alpha} < \|f'\|_p < \infty.
\end{equation*}
and \(f\in \Lip_\alpha\).





%%%%%%%%%%%%%% 12
\begin{excopy}
Suppose 
\label{ex:7.12}
\(\varphi: [a,b]\to\R^1\) is nondecreasing.
\begin{itemize}

\itemch{a} 
Show that there is a left-continuous nondecreasing $f$ on \([a,b]\),
so that \(\{f\neq\varphi\}\) is at most countable. 
[Left-continuous means: if \(a<x\leq b\) and \(\epsilon > 0\), then there is 
a \(\delta>0\) so that \(|f(x) - f(x-t)|<\epsilon\) whenever \(0<t<\delta\).]

\itemch{b}
Imitate the proof of Theorem~7.18 to show that there is a positive Borel measure
\(\mu\) on \([a,b]\) for which
\begin{equation*}
f(x) - f(a) = \mu([a,x)) \qquad (a\leq x \leq b).
\end{equation*}

\itemch{c}
Deduce from \ich{b} that \(f'(x)\) exists \aded \([m]\), 
that \(f'\in L^1(m)\), and that
\begin{equation*}
f(x) - f(a) = \int_a^b f'(t)\,dt + s(x) \qquad (a\leq x \leq b)
\end{equation*}
where $s$ is a nondecreasing and \(s'(x)=0\;\aded[m]\).

\itemch{d}
Show that \(\mu\perp m\) if and only if \(f'(x)=0\;\aded[m]\),
and that \(\mu \ll m\) if and only if $f$ is AC on \([a,b]\).

\itemch{e}
Prove that \(\varphi'(x) = f'(x)\;\aded[m]\).
\end{itemize}
\end{excopy}

\begin{itemize}
\itemch{a}
If \(x\in(a,b)\) is a point of discontinuity than
\begin{equation*}
\sup_{t<x} \varphi(t) < \inf_{t>x} \varphi(t) 
\end{equation*}
There could be at most countable number of such discontinuity points.
Otherwise we have an uncountable sum of positive numbers
which must be less than \(b-a\).
Now we define \(f(x) = \varphi(a)\) and for \(x\in(a,b)\)
\begin{equation} \label{eq:ex7.12e:f}
f(x) = \sup_{a\leq t < x} \varphi(t).
\end{equation}
which is clearly left-continuous and differs from \(\varphi\)
only where the latter is discontinuous.

\itemch{b}
Define \((g(x) = f(x) + x\), clearly $g$ is monotonic, one to one and 
left-continuous. Let \(V\subset (a,b)\) some open set.
We will show that \(g(V)\) is a Borel set.
Recalling that a monotonic function can have at most countable
number of points of discontinuity.
We can represent $V$ is a countable disjoint union of intervals
\(V = \disjunion_{j\in\N} I_j\) such that each \(I_j\)
is open, closed or half open, and $g$ is continuous on it.
Clearly, \(g(I_j)\) is an interval --- open, closed or half open.
Thus \(g(V)\) is a Borel set. By being one-to-one, 
\begin{equation*}
g\bigl((a,b)\setminus V\bigr) = \bigl(g(a),g(b)\bigr) \setminus g(V) 
\end{equation*}
and so \(g(C)\) is a Borel set for any closed set $C$
and consequently 
\(g(E)\) is a Borel set for any Borel set $E$.

Now we can define for each Borel set $E$
\begin{align*}
\nu(E) &= m\bigl(g(E)\bigr) \\
\mu(E) &= \nu(E) - m(E).
\end{align*}
If follows that for each \(x\in(a,b)\)
\begin{align*}
g(x) - g(a) &= \nu([a,x]) \\
f(x) - f(a) &= \nu([a,x]) - (x-a) = \mu([a,x)).
\end{align*}

\itemch{c}
By Lebesgue differentiation theorem (local lemma~\ref{lem:leb:diff})
\(f'\) exists almost everywhere and \(f'\in L^1([a,b],m)\). Put 
\begin{equation} \label{eq:ex7.12:sx}
s(x) = f(x) - f(a) - \int_a^x f'(t)\,dm(t).
\end{equation}
By Theorem~7.11 \cite{RudinPMA85} 
\begin{equation*}
\frac{d}{dx} \int_a^x f'(t)\,dm(t) = f'(x) \quad\aded
\end{equation*}
Hence \(s'(x) = 0\; \aded\).


\itemch{d}
Let 
\begin{equation*}
\mu = \mu_a + mu_s \qquad  d\mu = h\,dm + d\mu_s
\end{equation*}
be the unique and positive Lebesgue decomposition 
of Radon-Nikodym Theorem~6.10 \cite{RudinRCA87}. 
By Theorem~7.14 \cite{RudinRCA87}.
\begin{equation*}
\lim_{t\to 0+} \frac{\mu\bigl([x+t)\bigr)}{m\bigl([x+t)\bigr)}
= \lim_{t\to 0+} \frac{\mu\bigl([x+t)\bigr)}{t}
= h(x) \quad \aded[m].
\end{equation*}
But
\begin{equation*}
\lim_{t\to 0+} \frac{\mu\bigl([x+t)\bigr)}{t} = f'(x) \quad \aded[m].
\end{equation*}
So by the the uniqueness of $h$, we have \(f' = h\;\aded[m]\).
Hence, by \eqref{eq:ex7.12:sx} we have \(\mu_s([a,x)] = s(x)\).

\paragraph{Equivalence 1.}
Assume \(\mu\perp m\), then \(\mu_a = h\,dm \perp m\).
Since $h$ is positive \(h = 0\;\aded[m]\) and so \(f'=0\;\aded[m]\).
Conversely, assume  \(f'=0\;\aded[m]\). Then \(\mu_a = h\,dm = 0\)
and thus \(\mu_a \perp m\) trivially.
Since \(\mu_s \perp m\) so is \(\mu\).

\paragraph{Equivalence 2.}
Assume  \(\mu \ll m\). Then $f$ maps nullsets to nullsets and 
by Theorem~7.18 \cite{RudinRCA87} $f$ is absolutely continuous.
Conversely, assume $f$ is absolutely continuous.
then by Theorem~7.18 \cite{RudinRCA87} 
\begin{equation*}
f(x) - f(a) = \int_a^x f'(t)\,dt  \qquad a\leq x \leq b
\end{equation*}
Thus \(\mu_s = 0\) and so \(\mu = \mu_a\) and \(\mu \ll m\).

\itemch{e}
We know that by construction \eqref{eq:ex7.12e:f}
 \(f \leq \varphi\) 
and \(f = \varphi\;\aded[m]\) and both are nondecreasing.
Pick \(x\in[a,b]\) such that \(f(x) = \varphi(x)\) and \(f'(x)\) exists.
Hence if \(h>0\) and \(x\pm h \in [a,b]\) then
\begin{equation*}
     \frac{\varphi(x-h) - \varphi(x)}{-h} 
\leq f'(x) 
\leq \frac{\varphi(x+h) - \varphi(x)}{h}.
\end{equation*}
Therefore
\begin{align*}
     \varliminf_{h\to x^-} \frac{\varphi(x+h) - \varphi(x)}{h}
&\leq \varlimsup_{h\to x^-} \frac{\varphi(x+h) - \varphi(x)}{h} \\
&\leq f'(x) \\
&\leq \varliminf_{h\to x^+} \frac{\varphi(x+h) - \varphi(x)}{h}
\leq \varlimsup_{h\to x^+} \frac{\varphi(x+h) - \varphi(x)}{h}.
\end{align*}

We will show 
\begin{equation} \label{eq:ex7.12e:desired}
\varlimsup_{h\to x^+} \frac{\varphi(x+h) - \varphi(x)}{h} = f'(x).
\end{equation}
The analog equality
\begin{equation*}
\varliminf_{h\to x^-} \frac{\varphi(x+h) - \varphi(x)}{h}  = f'(x)
\end{equation*}
can be shown with similar arguments. Thus the above inequalities
are actually equalities. Hence \(\varphi'(x) = f'(x)\).

By definition of \(\varliminf\), we have
a strictly decreasing sequence 
\(\{h_j\}_{j\in\N}\) 
of positive numbers
such that
\begin{align*}
\lim_{j\to\infty} h_j &= 0 \\
\lim_{j\to\infty} \frac{\varphi(x+h_j) - \varphi(x)}{h_j}
&=\varlimsup_{h\to x^+} \frac{\varphi(x+h) - \varphi(x)}{h}
\end{align*}
By the construction of $f$ in \eqref{eq:ex7.12e:f} 
the ``modified'' set \(S=\{x\in[a,b]: f(x)\neq \varphi(x)\}\)
is countable.
We also note that for \(\varphi(x_1) \leq f(x_2)\)
whenever \(a\leq x_1\leq x_2 \leq b\).
Thus in each segment \(H_n = [x + h_{n}, x + h_{n-1}]\)
we can find a sequence \(G_n = \{g_j\}_{j\in\N}\) such that
\begin{align*}
f(g_j) &= \varphi(g_j) \qquad (\mathrm{since}\;g_j\notin S) \\
\lim_{j\to\infty} g_j &= x + h_{n} 
\end{align*}
Hence
\begin{equation*}
\lim_{j\to\infty} f(g_j) \geq \varphi(x + h_{n})
\end{equation*}
and therefore
\begin{equation*}
\lim_{j\to\infty} \frac{\varphi(g_j) - \varphi(x)}{g_j - x}
 \geq \lim_{j\to\infty} \frac{f(g_j) - f(x)}{g_j - x}
 \geq \frac{\varphi(x+h_{n}) - \varphi(x)}{h_{n}}.
\end{equation*}
By a diagonal process we can pick 
a sequence \(\{x_j\}_{j\in\N}\) such that for all $n$
\begin{align*}
x_n &\in G_n \subset H_n \\
\lim_{n\to\infty} \frac{f(x_n) - f(x)}{x_n - x}
&\geq  \frac{\varphi(x+h_{n}) - \varphi(x)}{h_{n}} + \frac{1}{n}
\end{align*}
Clearly 
\begin{align*}
\lim_{n\to\infty} x_n &= x \\
f'(x) &= \lim_{n\to\infty} \frac{f(x_n) - f(x)}{x_n - x}
 \geq  \lim_{n\to\infty} \frac{\varphi(x+h_{n}) - \varphi(x)}{h_{n}}.
\end{align*}
Hence \eqref{eq:ex7.12e:desired} is true.

\end{itemize}


%%%%%%%%%%%%%% 13
\begin{excopy}
Let \(BV\) be the class of all $f$ on \([a,b]\) that have bounded variation
on \([a,b]\), as defined after Theorem~7.19. Prove the following statements.
\begin{itemize}
\itemch{a} 
Every monotonic bounded function on \([a,b]\) is in \(BV\).

\itemch{b}
If \(f \in BV\) is real, there exists bounded monotonic functions
\(f_1\) and \(f_2\) so that \(f=f_1-f_2\).\\
\emph{Hint}: Imitate the proof of Theorem~7.19.

\itemch{c}
If \(f\in BV\) is left-continuous then \(f_1\) and \(f_2\) 
can be chosen in \ich{b} so as to be also left-continuous.

\itemch{d}
If \(f\in BV\) is left-continuous then there is a Borel measure \(\mu\) 
on \([a,b]\) that satisfies
\begin{equation*}
f(x) - f(a) = \mu([a,x)) \qquad (a\leq x \leq b);
\end{equation*}
\(\mu\ll m\) if and only if $f$ is AC on \([a,b]\).

\itemch{e}
Every \(f\in BV\) is differentiable \(\aded[m]\), and \(f'\in L^1(m)\).
\end{itemize}
\end{excopy}

\begin{itemize}
\itemch{a}
Let $f$ be monotonic on \([a,b]\).
If 
\begin{equation*}
a\leq t_0 < t_1 < \cdots < t_N\leq b
\end{equation*}
then clearly 
\begin{equation*}
\sum_{j=1}^N |f(t_j) - f(t_{j-1})|
= \left| \sum_{j=1}^N \bigl(f(t_j) - f(t_{j-1})\bigr) \,\right|
= |f(t_N)-f(t_0)|.
\end{equation*}
Therefore,
\begin{equation*}
\sup_{a=t_0 < t_1 < \cdots < t_N=b} \sum_{j=1}^N |f(t_j) - f(t_{j-1})|
= |f(b)-f(a)|
\end{equation*}

\itemch{b}
By Theorem~7.19 \cite{RudinRCA87}
if
\begin{equation}  \label{eq:Fx:ex:7.13b}
F(x) = 
\sup_{a=t_0 < t_1 < \cdots < t_N=x} \sum_{j=1}^N |f(t_j) - f(t_{j-1})|
= |f(b)-f(a)|
\end{equation}
then \(F\pm f\) are nondecreasing and continuous.
Now
\begin{equation*}
f_1 = (F+f)/2 \qquad f_2 = -(F-f)/2
\end{equation*}
satisfy the requirements.

\itemch{c}
Assume $f$ is left-continuous, then $F$ defined in \eqref{eq:Fx:ex:7.13b}
is left-continuous as well. To show this pick \(x\in[a,b]\) and \(\epsilon>0\).
Take \(\delta>0\) such that \((a\leq x-\delta\) and 
\begin{equation*}
\sum_{j=1}^n |f(b_j) - f(a_j)| < \epsilon
\end{equation*}
whenever 
\begin{equation*}
\sum_{j=1}^n b_j - a_j < \delta \qquad  \forall j\, a\leq a_j\leq b_j\leq b
\end{equation*}
Now we can see that 
\begin{equation*}
F(x-\delta) + \epsilon \geq F(x).
\end{equation*}
Thus our construction satisfy the left-continuous requirement.

\itemch{d}
Assume such required Borel measure \(\mu\) exists.
Now $f$ maps sets of measure $0$ to sets of measure $0$ since 
\(\mu \ll m\). By Theorem~7.18 \cite{RudinRCA87} $f$ is absolutely continuous.

Conversely, assume $f$ is absolutely continuous.
By Theorem~7.20 \cite{RudinRCA87} \(f'\in L^1(m)\) and we can define
\begin{equation*}
\mu(E) = \int_E f'(x)\,dm(x)
\end{equation*}
for each Borel set $E$. Now \(\mu\) is a Borel measure and \(\mu \ll m\).

\itemch{e}
By \ich{b} we can represent \(f=f_1-f_2\) where \(f_1,f_2\) are monotonic.
As we saw in Exercise~12\ich{a} these functions of at most countable
number of points where they are not continuous.
By Exercise~\ref{ex:7.12}\ich{c}
\(f_j'\) exist \aded\ and \(f_j'\in L^1(m)\) for \(j=1,2\)
and so is \(f' = f_1' - f_2'\).

\paragraph{Note:} Here we \emph{cannot} show that 
\(f(x)-f(a) = \int_a^x f't)\,dt\). Absolute continuity is missing for this.
\end{itemize}

%%%%%%%%%%%%%% 
\begin{excopy}
Show that the product of two absolutely continuous functions on \([a,b]\)
is absolutely continuous. Use this to derive a theorem about integration
by parts.
\end{excopy}

Let \(f_1,f_2\) be two absolutely continuous functions and \(\epsilon>0\).
Thus there is a \(\delta>0\) such that 
\begin{equation*}
\sum_{j=1}^N |f(t_j) - f(t_{j-1})|<\epsilon
\end{equation*}
if 
\begin{equation} \label{eq:ex7.14}
\sum_{j=1}^N |t_j - t_{j-1}|<\delta.
\end{equation}
Put \(f = f_1\cdot f_2\) and 
for abbreviation put \(a_k = f_1(a_k)\) and \(b_k = f_2(a_k)\). 
Using:
\begin{align*}
|f(t_j) - f(t_{j-1})|
&= |(f_1\cdot f_2)(t_j) - (f_1\cdot f_2)(t_{j-1})|
 = |a_j b_j - a_{j-1}b_{j-1}| \\
&= \left|
    \bigl(a_{j-1} + (a_j - a_{j-1})\bigr)
    \bigl(b_{j-1} + (b_j - b_{j-1})\bigr) - a_{j-1}b_{j-1}
   \right| \\
&= \bigl|
      (a_j - a_{j-1})b_{j-1} 
    + (b_j - b_{j-1})a_{j-1} 
    + (a_j - a_{j-1}) (b_j - b_{j-1}) 
   \bigr| \\
&\leq 
    |a_j - a_{j-1}|\cdot \|f_2\|_\infty
    + |b_j - b_{j-1}|\cdot \|f_1\|_\infty
    + |a_j - a_{j-1}|\cdot |b_j - b_{j-1}| \\
&\leq 
    |a_j - a_{j-1}|\cdot 3\|f_2\|_\infty
    + |b_j - b_{j-1}|\cdot \|f_1\|_\infty
\end{align*}
Assuming \eqref{eq:ex7.14} we have:
\begin{align*}
\sum_{j=1}^N |f(t_j) - f(t_{j-1})|
&\leq
    3\|f_2\|_\infty \sum_{j=1}^N  |a_j - a_{j-1}|
  + \|f_1\|_\infty \sum_{j=1}^N  |b_j - b_{j-1}|  \\
&\leq \left(\|f_1\|_\infty + 3\|f_2\|_\infty \right) \epsilon
\end{align*}
Thus $f$ is absolutely continuous as well.

Now assume \(u,v: [a,b]\to\C\) are absolutely continuous functions.
By the above result, they are simultaneously differentiable almost everywhere.
Hence by Leibnitz's rule
\begin{equation*}
\frac{d\bigl(f(x)g(x)\bigr)}{dx} = f'(x)g(x) + f(x)g'(x) \; \aded
\end{equation*}
Consequently
\begin{equation*}
f(b) - f(a) =
\int_a^b \frac{d\bigl(f(x)g(x)\bigr)}{dx}\,dx 
= \int_a^b f'(x)g(x)\,dx + \int_a^b f(x)g'(x)\,dx.
\end{equation*}



%%%%%%%%%%%%%% 15
\begin{excopy}
Construct a monotonic functions $f$ on \(\R^1\) so that \(f'(x)\) exists
(finitely) for every \(x\in \R^1\), but \(f'\) is not a continuous function.
\end{excopy}

Based on \cite{Gelb1996} Chapter~3 counterexample~2 page~36. Let
\begin{equation*}
f(x) = \left\{
\begin{array}{ll}
x^2\sin(1/x) - 2x^2 + x \quad & \mathrm{if}\; x < 0 \\
0            \quad & \mathrm{if}\; x =  0 \\
x^2\sin(1/x) + 2x^2 + x \quad & \mathrm{if}\; x > 0
\end{array}
\right.
\end{equation*}
Now
\begin{equation*}
f'(x) = \left\{
\begin{array}{ll}
2\bigl(x\sin(1/x) + 2|x|\bigr) + \bigl(2-\cos(1/x)\bigr)
    \quad & \mathrm{if}\; x\neq 0 \\
1   \quad & \mathrm{if}\; x=0 \\
\end{array}
\right.
\end{equation*}
Clearly \(f'>0\) but is not continuous in $0$ since \(\cos(1/x)\) is not.


%%%%%%%%%%%%%%
\begin{excopy}
Suppose \(E\subset [a,b]\), \(m(E)=0\).
Construct an absolutely continuous monotonic function $f$ on \([a,b]\)
so that \(f'(x)=\infty\) at every \(x\in E\).

\emph{Hint}: \(E \subset \cap V_n\), \(V_n\) open, \(m(V_n)<2^{-n}\).
Consider the sum of characteristic functions of these sets.
\end{excopy}

Following the hint.
Let \(V_n\) as suggested, but also require \(V_{n+1}\subset V_n\).
This could be easily done, by possibly replacing
\begin{equation*}
V_n \leftarrow \cap_{k\leq n} V_n
\end{equation*}
Define
\begin{equation*}
g(x) = \sum_{n\in\N} \chhi_{V_n}
\end{equation*}
Thus \(g(x)=0\) iff \(x\in E\) and \(g\in L^1([a,b])\).
Now define
\begin{equation*}
f(x) = \int_a^x g(t)\,dt.
\end{equation*}
By Exercise~10\ich{a} of Chapter~6 \cite{RudinRCA87}
$g$ is uniformly integrable (by itself).
Thus $f$ is absolutely continuous.
Now let \(x\in E\) and \(M\in\N\).
There is some \(\delta>0\) such that
\((x-\delta,x+\delta) \subset V_M\) and 
\begin{equation*}
\frac{f(x+h) - f(x)}{h} 
= \frac{1}{h}\int_x^{x+h} g(t)\,dt
\geq \frac{1}{h}\int_x^{x+h} \sum_{n=1}^M \chhi_{V_n}\,dt
= Mh/h = M.
\end{equation*}
Hence \(f'(x)=\infty\).



 By Theorem~7.19
$f$ is differentiable \aded\ and
\begin{equation*}
f(x) = \int_a^x f'(t)\,dt \qquad \forall x\in[a,b]
\end{equation*}


%%%%%%%%%%%%%% 17
\begin{excopy}
Suppose \(\{\mu_n\}\) is a sequence of positive Borel measures on \(\R^k\) and
\begin{equation*}
\mu(E) = \sum_{n=1}^\infty \mu_n(E).
\end{equation*}
Assume \(\mu(\R^k)<\infty\). Show that \(\mu\) is a Borel measure.
What is the relation between Lebesgue decomposition of the \(\mu_n\) 
and that of \(\mu\)?

Prove that 
\begin{equation*}
(D\mu)(x) = \sum_{n=1}^\infty (D\mu_n)(x) \quad \aded[m].
\end{equation*}
Derive corresponding theorems for sequences \(\{f_n\}\) 
of positive nondecreasing functions on \(\R^1\) and their sums \(f=\sum f_n\).
\end{excopy}

Let \(\mu_n = \mu_{n,a} + \mu_{n,s}\) be the Lebesgue decomposition.
Clearly 
\(\sum_{n=1}^\infty \mu_{n,a}\)
and
\(\sum_{n=1}^\infty \mu_{n,s}\)
are positive Borel measures and
\begin{equation*}
\sum_{n=1}^\infty \mu_{n,a} \ll m 
\qquad
\sum_{n=1}^\infty \mu_{n,s} \perp m.
\end{equation*}
By uniqueness, these make the Lebesgue decomposition of \(\mu\). 

By Theorem~7.14 \cite{RudinRCA87} and the above observation, if 
\(d\mu = f\,dm + d\mu_s\) and
\(d\mu_n = f_n\,dm + d\mu_{n,s}\)
then 
\begin{equation*}
(D\mu)(x) = f(x) = \sum_{n=1}^\infty f_n(x) = \sum_{n=1}^\infty (D\mu_n)(x) 
\quad\aded[m].
\end{equation*}

Say \(\{f_n\}_n\in\N\) are positive nondecreasing functions on \(\R^1\)
such that 
\begin{equation*}
f = \sum_{n=1}^\infty f_n \in L^\infty(\R,m).
\end{equation*}
Then
\begin{equation*}
f'(x) = \sum_{n=1}^\infty f_n'(x) \qquad \aded[m].
\end{equation*}
This can be shown by using Exercise~12 and replacing $a$ and \(f(a)\)
by \(-\infty\) and 
\begin{equation*}
\lim_{t\to-\infty} f(t) = \inf_{t\in\R} f(t).
\end{equation*}


%%%%%%%%%%%%%% 18
\begin{excopy}
Let \(\varphi_0(t) = 1\) on \([0,1)\), \(\varphi_0(t) = -1\)
on \([1,2)\), extended \(\varphi_0\) to \(\R^1\) so as to have a period $2$,
and define \(\varphi_n(t) = \varphi_0(2^n t)\), \(n=1,2,3,\ldots\)

\textsl{
Assume that \(\sum |c_n|^2 < \infty\) and prove that the series
\begin{equation} \label{eq:ex7.18}
\sum_{n=1}^\infty c_n\varphi_n(t)
\end{equation}
converges then for almost every $t$.
}

Probabilistic Interpretation: The series \(\sum(\pm c_n)\) converges
with probability $1$.\\
\emph{Suggestion}: \(\{\varphi_n\}\) is orthonormal on \([0,1]\), 
hence \eqref{eq:ex7.18} is the Fourier series of some \(f\in L^2\).
If \(a = j\cdot 2^{-N}\), \(b = (j+1)\cdot 2^{-N}\), \(a<t<b\),
and \(s_N = c_1\varphi_1 + \cdots + c_N\varphi_N\), then, for \(n>N\),
\begin{equation*}
s_N(t) = \frac{1}{b-a} \int_a^b s_N\,dm = \frac{1}{b-a} \int_a^b s_n\,dm,
\end{equation*}
and the last integral converges to \(\int_a^b f\,dm\), as \(n\to\infty\).
Show that \eqref{eq:ex7.18} at almost every Lebesgue point of $f$.
\end{excopy}

If \(m<n\) then within any segment \([j 2^{-n},  (j+1) 2^{-n}]\) clearly
\(\varphi_m\) is constant, while \(\varphi_n\) equally alternates signs
in \(2^{n-m}\) subsegments. Thus \(\varphi_m \perp \varphi_n\).
This argument also shows that the above equality of the integrals holds.

The existence of $f$ is implied by the completeness of 
the Hilbert space \(L^2([0,1],m)\).

Let $x$ be a Lebesgue point
\index{Lebesgue point}
of $f$ such that \(x\in B\) --- the countable \(B = \{j2^{-n}: j,n\in\Z^+\}\). 
We will define
nicely shrinking sets
\index{nicely shrinking sets}
for $x$ with \(\alpha = 1/2\). 
Let \(r_n=2^{-n}\) 
and pick \(E_n = [jr_n/2, (j+1)r_n/2]\subset [0,1]\)
such that \(x\in E_n\). 
We have 
\(E_n \subset B(x,r_n)\) and
\begin{equation*}
m(E_n) = r_n/2 = \alpha r_n = m\bigl(B(x,r_n)\bigr).
\end{equation*}
Now by the above observation and Theorem~7.10 \cite{RudinRCA87}
\begin{equation*}
f(x) 
= \lim_{n\to\infty} \frac{1}{m(E_n)} \int_{E_n} f\,dm
= \lim_{n\to\infty} s_n(x).
\end{equation*}
Thus the series converges for all Lebesgue points except for at most 
a countable set.


%%%%%%%%%%%%%% 19
\begin{excopy}
Suppose $f$ is continuous on \(\R^1\), \(f(x)>0\) if \(0 < x < 1\), \(f(x)= 0\)
otherwise. Define
\begin{equation*}
  h_c(x) = \sup \{n^c f(nx): n = 1,2,3,\ldots\}.
\end{equation*}
Prove that
\begin{itemize}
\itemch{a} \(h_c\) is in \(L^1(\R^1)\) if \(0<c<1\),
\itemch{b} \(h_1\) is in weak \(L^1\) but not in \(L^1(\R^1)\),
\itemch{c}  \(h_c\) is not weak \(L^1\) if \(c>1\).
\end{itemize}
\end{excopy}

\textbf{Remimder:} A function $f$ is 
\emph{weak} \(L^1\)\
\index{weak L1@weak \(L^1\)}
if 
\begin{equation*}
\lambda\cdot m\left( \{x: |f(x)| > \lambda\}\right)
\end{equation*}
is a bounded function of \(\lambda\).

For abbreviation we put \(M = \|f\|_\infty\).
For any $c$ we have.
\begin{align*}
\int_{\R^1} |h_c(t)|\,dt
&=    \int_0^1 h_c(t)\,dt
 =    \sum_{m=1}^\infty \int_{1/(m+1)}^{1/m} h_c(t)\,dt \\
&=    \sum_{m=1}^\infty \int_{1/(m+1)}^{1/m} 
      \bigl(\sup_{n\in\N} n^c f(nt)\bigr)\,dt 
 =    \sum_{m=1}^\infty \int_{1/(m+1)}^{1/m} 
      \bigl(\sup_{1\leq n\leq m+1} n^c f(nt)\bigr)\,dt
\end{align*}
and also put \(I_m=[1/(m+1),1/m]\)
the mid-third \(K= [1/2,2/3]\) and using continuity, let 
\begin{equation*}
T = \inf\{f(x): x\in K\} = \min\{f(x): x\in K\} > 0.
\end{equation*}

\begin{itemize} 
\itemch{a}
If \(0<c<1\) then
\begin{align*}
\int_{\R^1} |h_c(t)|\,dt
&\leq \sum_{m=1}^\infty \int_{1/(m+1)}^{1/m} (m+1)^c M\,dt \\
&=    M \sum_{m=1}^\infty (m+1)^c \left(\frac{1}{m+1} - \frac{1}{m}\right) 
 =    M \sum_{m=1}^\infty (m+1)^c \frac{1}{m(m+1)} \\
&=    M \sum_{m=1}^\infty m^{-1}(m+1)^{c-1} \\
&\leq M \sum_{m=1}^\infty (m+1)^{c-2} \\
&\leq M \int_1^\infty x^{c-2}\,dx 
=     \frac{M}{c-1}x^{c-1}\bigm|_1^\infty = M(1-c) < \infty.
\end{align*}
Hence \(h_c\in L^1(\R)\).

\itemch{b}
clearly \(m\cdot I_{2m} \subset K\) for 
any \(m\geq 1\). Thus we have the following  ``generous'' estimation:
%  (m/2) [1/(m+1),1/m]  \subset [1/3,2/3]
\begin{align*}
\int_{\R^1} |h_1(t)|\,dt
&\geq \sum_{m=1}^\infty \int_{I_m} h_1(t)\,dt 
 \geq \sum_{m=1}^\infty \int_{I_{2m}} h_1(t)\,dt 
 \geq \sum_{m=1}^\infty \int_{I_{2m}} 
      \bigl(\sup_{1\leq n\leq 2m+1} n^1 f(nt)\bigr)\,dt \\
&\geq \sum_{m=1}^\infty \int_{I_{2m}} m T 
 =    T \sum_{m=1}^\infty m \ell(I_{2m})  
 =    T \sum_{m=1}^\infty m \frac{1}{2m(2m+1)} \\
&=    (T/2) \sum_{m=1}^\infty  \frac{1}{2m+1}
 \geq (T/2) \sum_{m=1}^\infty  \frac{1}{2m+2}
 =    (T/4) \sum_{m=2}^\infty  \frac{1}{m} = \infty.
\end{align*}
Hence \(h_1 \notin L^1(\R)\).

To show that \(h_1\) is weak \(L^1\) we first assume that \(\lambda < M\).
Then
\begin{equation*}
\lambda \cdot m\bigl( \{x\in\R: |h_1(x)| > \lambda\}\bigl)
\leq M \cdot \ell([0,1]) = M.
\end{equation*}
Now assume that \(\lambda \geq M\). If \(|h_1(x)|> \lambda\)
then \(n\cdot f(nx) > \lambda\) for some $n$. 
Hence \(nM > \lambda\) and  \(x < 1/n\).
Thus \(n \geq \lfloor \lambda/M \rfloor \geq 1\) and 
\begin{equation*}
0 < x < 1/\lfloor \lambda/M \rfloor.
\end{equation*}
Since
\begin{equation*}
\lambda < \left(\lfloor \lambda/M\rfloor + 1\right)M 
        \leq 2 \lfloor \lambda/M\rfloor M 
\end{equation*}
we can estimate
\begin{align*}
\lambda \cdot m\bigl( \{x\in\R: |h_1(x)| > \lambda\}\bigl) 
&<  \lambda /\lfloor \lambda/M \rfloor < 2M.
\end{align*}
In both cases the expression is bounded for any \(\lambda\) and 
\(h_1\) is weak \(L^1\).

\itemch{c}
By checking odd and even cases \(\lfloor n/2\rfloor I_n \subset K\)
for all \(n\geq 2\). 
It is sufficient to consider \(\lambda \geq T\).
If 
\begin{equation*}
n \geq 2\left\lceil \left(\frac{\lambda}{T}\right)^{1/c} \right\rceil
  \geq        \left(\frac{\lambda}{T}\right)^{1/c} 
\end{equation*}
then
\begin{equation*}
n^c T > \lambda.
\end{equation*}
Therefore if \(x \in I_{2n+1} \cup I_{2n}\) then \(nx \in K\) 
and \(h_c(x) \geq \lambda\), consequently
\begin{equation*}
(0,1/2n] \subset \{x\in\R: h_c(x)> \lambda\}.
\end{equation*}
Now since \(\lambda/T \geq 1\) we see that
\begin{align*}
\lambda \cdot m\bigl( \{x\in\R: |h_1(x)| > \lambda\}\bigl) 
&\geq \lambda \cdot \ell\bigl((0,1/n]\bigr) \\
&\geq \lambda \bigm/ 
      \left\lceil (\lambda/T)^{1/c} \right\rceil 
 \geq \lambda \bigm/ 2 (\lambda/T)^{1/c} \\
&= \left(T^{-c}/2\right) \lambda^{1-1/c}
\end{align*}
is clearly unbounded for \(\lambda\).
Hence \(h_c\) is not weak \(L^1(\R)\).
\end{itemize}


%%%%%%%%%%%%%% 20
\begin{excopy}
\begin{itemize}
\itemch{a}
For any set \(E\subset \R^2\), the boundary \(\partial E\) of $E$ is,
by definition the closure of $E$ minus the interior of $E$. Show that
$E$ is Lebesgue measurable whenever \(m(\partial E) = 0\).

\itemch{b}
Suppose that $E$ is the union of a (\emph{possibly uncountable}) collection
of \emph{closed} discs in \(\R^2\) whose radii are at least $1$.
Use \ich{a} to show that $E$ is Lebesgue measurable.

\itemch{c} 
Show that the conclusion of \ich{b} is true even when the
radii are unrestricted.

\itemch{d}
Show that some unions of closed discs of radius $1$
are not Borel sets (See Sec.~2.21.)

\itemch{e}
Can discs be replaced by triangles, rectangles, arbitrary polygons, etc.,
in all this?
What is the relevant geometric property?
\end{itemize}
\end{excopy}

\begin{itemize}

\itemch{a}
For every set $E$ we have
\begin{equation*}
E = \inter{A} \disjunion (E \cap \partial E).
\end{equation*}
Now if \(m(\partial E) = 0\)
then also \(m(E \cap \partial E) = 0\) and thus $E$ is measurable
as a union of two measurable sets.


\itemch{b}

{\small Using a hint of Dmitry Ryabogin
see 
\linebreak[1]
\texttt{http://www.math.ksu.edu/\~{}ryabs/tar13.pdf}
\linebreak[1]
or
\texttt{http://www.math.ksu.edu/\~{}ryabs/tar13.dvi}{}.}

Let \(\{D_j\}_{j\in J}\) be a family of closed discs
with radii \(\geq 1\). 
Let \(U = \cup_{j\in J} D_j\) and \(V = \cup_{j\in J} \inter{D_j}\).
Let $Y$ be the boundary of the $U$. Since $Y$ is closed, it is measurable.
Assume by negation that \(m(Y)>0\). 
By Lebesgue's density theorem~7.7 \cite{RudinRCA87}
there are points (actually almost all)
\(b\in Y\) such that their denisty is $1$.
Pick arbitrary \(b\in Y\) and let \(\iota = 1/4\)
(Actually and value \(0 < \iota < 1/2\) will do).
We will show that the density 
\begin{equation*}
\lim_{r\to 0^+} \frac{m\bigl(Y\cap B(b,r)\bigr)}{m(B_r)} \leq 1 - \iota < 1
\end{equation*}
which will provide the desired contradiction.
Let \(0<r<\iota\) and \(0<\epsilon<r\).
Since $b$ is a boundary point,
there exist some disc with radius \(\rho\geq 1\)
whose center $c$ such that \(\|b-c\| < \rho + \epsilon\).
\Wlogy, assume \(b=(0,0)\) and \(c=(\rho + \epsilon, 0)\).
Let $S$ be the intersection
of the discs \(B(b,r)\) and \(B(c,\rho + \epsilon)\).
These boundaries of these discs intersects in two points
\((r\cos\alpha, \pm r \sin\alpha)\). Denote the following points
\begin{alignat*}{2}
Q &= (r\cos\alpha, r \sin\alpha) & & \qquad\textrm{Positive intersection} \\
P &= (r\cos\alpha, 0)            & & \qquad\textrm{Projection of\ }\; Q \\
A &= (\epsilon, 0)               & & \qquad\textrm{Nearest point of}\; 
                                     B(c,\rho+\epsilon)\;\textrm{to}\; b \\
R &= (r, 0)
                                 & & \qquad\textrm{Nearest point of}\; 
                                     B(b,r)\;\textrm{to}\; c \\
\end{alignat*} 
Now we estimate the area which is the measure of the intersection
\begin{align*}
m\bigl(B(b,r) \cap B(c,\rho + \epsilon)\bigr)
&= 2m\bigl(B(b,r) \cap B(c,\rho + \epsilon) 
           \cap \{(x,y)\in\R^2: y\geq 0\} \bigr) \\
&\geq 2 m\bigl(\vartriangle(A,R,Q)\bigr) 
 \geq 2 m\bigl(\vartriangle(b,R,Q)\bigr) 
 \geq 2 m\bigl(\vartriangle(P,R,Q)\bigr) \\
&= 2 \frac{\|P-R\|\cdot \|Q-P\|}{2}
 = r(1-\cos\alpha)\cdot r\sin\alpha.
\end{align*}

To use this estimatation, we need to see how \(\alpha\)
that clearly satisfies \(0<\alpha < \pi/2\)
depends on $r$, \(\rho\) and \(\epsilon\).
Equating the square distance of \(\overline{Qc}\) by Pythagoras
\begin{equation*}
\rho^2 = \|P-c\|^2 + \|Q-P\|^2 
= \bigl((\rho+\epsilon)-r\cos\alpha\bigr)^2 +(r\sin\alpha)^2
= (\rho+\epsilon)^2-2r(\rho+\epsilon)\cos + r^2
\end{equation*}
hence
\begin{equation*}
\cos\alpha = \frac{r^2 + (\rho+\epsilon)^2 - \rho^2}{2r(\rho+\epsilon)} > 0
\end{equation*}
and so
\begin{equation*}
\lim_{\epsilon\to 0^+} \cos\alpha = r/2\rho \leq r/2.
\end{equation*}
For each (sufficiently small) \(r>0\), We can pick 
a disc \(D_{j(r)} = D_j\) of the family with distance of 
\(\epsilon\geq 0\) to $x$ 
(radius \(\rho\) and center $c$ such that \(\|c-x\| = \rho+\epsilon\))
such that \(\alpha_r = \alpha > \pi/2 - r\).

We start estimating the density of $Y$ in $x$. 
We note that \(U \subset \inter{V} \subset Y\),
hence \(U \cap Y = \emptyset\).
\begin{equation} \label{eq:ex7.20:DYx}
D_Y(x) = 
\lim_{r\to0+} \frac{m\bigl( B(x,r) \cap Y\bigr)}{m(B_r)}
\leq
1 - \lim_{r\to0+} \frac{m\bigl( B(x,r) V\bigr)}{m(B_r)}\;.
\end{equation}
Focusing on the last limit 
\begin{align*}
\lim_{r\to 0+} \frac{m\bigl( B(x,r) \cap V\bigr)}{m(B_r)}
&\geq \lim_{r\to 0+} \frac{m\bigl( B(x,r) \cap D_{j(r)}\bigr)}{m(B_r)} 
 \geq \lim_{r\to 0+} \frac{r^2(1-\cos\alpha_r)\sin\alpha_r}{\pi r^2} \\
&\geq \lim_{r\to 0+} (1-\cos\alpha_r)\sin\alpha_r / \pi = 1/\pi > 0.
\end{align*}
Returning to \eqref{eq:ex7.20:DYx} we see that 
\begin{equation*}
D_Y(x) \leq 1 - 1/\pi < 1.
\end{equation*}
which is the the desired contradiction.


\itemch{c}
The statement of \ich{b} can 
be trivially to any lower bound \(r\geq 0\) instead of $1$ for the radii.
So now if \calF\ is any collection if closed circles,
let \(\calF(r)\) be the sub-collection consisting of closed circles 
of \calF\ whose radii is at least $r$.
Now since
\begin{equation*}
\calF = \cup_{n=1}^\infty \calF(1 - 1/n)
\end{equation*}
we have
\begin{equation*}
\bigcup_{C\in\calF} C
= \cup_{n=1}^\infty \left(\bigcup_{C\in \calF(1 - 1/n)} C\right).
\end{equation*}
Thus any union of closed circles, is a countable union of 
Lebesgue measurable sets
and so the union is Lebesgue measurable as well.

\itemch{d}
Pick a non measurable set \(A\subset \R^1\). Now let
\begin{equation*}
U 
= \cup_{a\in A} B\bigl((a,1),1\bigr)
= \cup_{a\in A} \{(x,y)\in\R^2: (x-a)^2+(y-1)^2 \leq 1\}.
\end{equation*}
Assume by negation that $U$ is a Borel set. Then so would be
\begin{equation*}
A \times \{0\} = U \cap \left(\R\times \{0\}\right).
\end{equation*}
Now $A$ is generated by a countable sequence of countable unions
and countable intersections of the 
\(\sigma\)-algebra base of open sets. Restricting these open sets to \(\R^1\)
results with open sets in \(\R^1\). Producing the same countable processes
in \(\R^1\) now generates $A$ that must be Borel, which is a contradiction.


\itemch{e}
Instead of disc, the required property is that the geometric shape
would have a lower bound on the internal angles. Not ``too accute'' angle.
Thus perfect polygons would work.
\end{itemize}

%%%%%%%%%%%%%% 21
\begin{excopy}
  If $f$ is a real function on \([0,1]\) and
\begin{equation*}
  \gamma(t) = t + if(t)
\end{equation*}
the length of the graph of $f$ is, by definition, the total variation
of \(\gamma\) on \([0,1]\). Show that this length is finite if and
only if \(f\in BV\).  (See Exercise~13.)  Show that it is equal to
\begin{equation*}
  \int_0^1 \sqrt{1 + [f'(t)]^2}\,dt
\end{equation*}
if $f$ is absolutely continuous.
\end{excopy}


We will deal with more general case of a curve, 
following Theorem~(8.4)(iii) of \cite{Saks37}.

\begin{llem}
Let \(\phi:[0,1]\to \R^2\) be a continuous curve given by 
\(\phi(t) = \bigl(x(t),y(t)\bigr)\).
Let
\begin{equation*}
d(t_0,t_1) = \|\phi(t_1) - \phi(t_0)\| = 
\sqrt{\bigl(x(t_1) - x(t_0)\bigr)^2 + \bigl(y(t_1) - y(t_0)\bigr)^2}
\qquad (0\leq t_0 \leq t_1 \leq 1)
\end{equation*}
the distance of the curve \(\phi\) on the parameter range \([t_0,t_1]\).
Let 
\begin{equation*}
S(\phi,\alpha;t) = \sup_{\alpha = a_0 < a_1 < \cdots < a_n = t}
                   \sum_{j=1}^n d(a_j,a_{j-1})
\qquad (0\leq \alpha \leq t \leq 1)
\end{equation*}
be the length of the curve on the parameter range \([\alpha,t]\).
\begin{itemize}
\itemch{i} The length $S$ of the curve is finite iff \(x(t)\) and \(y(t)\)
           have bounded variation.
\itemch{ii} If  \(x(t)\) and \(y(t)\) are absolutely continuous, then
\begin{equation} \label{eq:7.20:Seq}
S(\phi,0;1) = \int_0^1 \sqrt{\bigl(x'(t)\bigr)^2 + \bigl(y'(t)\bigr)^2}.
\end{equation}
\end{itemize}
\end{llem}
\begin{thmproof}
Assume the length is finite.
Let \(0=a_0 < a_1 < \cdots a_n = 1\) be any partition, then
\begin{equation*}
\sum_{j=1}^n |x(a_j) - x(a_{j-1})|
\leq \sum_{j=1}^n d(a_j,a_{j-1}) \leq S(\phi,0;1) < \infty
\end{equation*}
and so \(x(t)\) has bounded variation.
Simiar estimation can be done with \(y(t)\) that  has bounded variation as well.

Conversely, assume  \(x(t)\) and \(y(t)\) have  bounded variation.
Again \(0=a_0 < a_1 < \cdots a_n = 1\) be any partition, then
\begin{equation*}
\sum_{j=1}^n d(a_j,a_{j-1}) 
\leq \sum_{j=1}^n |x(a_j) - x(a_{j-1})| + |y(a_j) - y(a_{j-1})| < \infty
\end{equation*}
and the curve has finite length and \ich{i} is proved.

Now assume that  \(x(t)\) and \(y(t)\) are absolutely continuous.


We first show that so is \(S(\phi,0,t)\).
Let \(\epsilon>0\) and \(\delta\) be such that 
\(\sum_{k=1}^n |x(b_j) - x(a_j)| < \epsilon\) 
and
\(\sum_{k=1}^n |y(b_j) - y(a_j)| < \epsilon\) 
whenever \(\sum_{k=1}^n b_j - a_j < \delta\) where \(0 \leq a_j < b_j \leq 1\).
Now \(S(t) = S(\phi,0;t)\) is clearly monotonically increasing, 
and we have
\begin{eqnarray*}
\sum_{j=1}^n S(b_j) - S(a_j) 
&=&
\sum_{j=1}^n \sup_{a_j = t_0 < t_1 < \cdots < t_n = b_j} 
             \sum_{k=1}^n  d(t_{k-1},t_k) \\
&\leq&
\sum_{j=1}^n \sup_{a_j = t_0 < t_1 < \cdots < t_n = b_j} 
             \sum_{k=1}^n  |x(k) - x(t_{k-1})| + |y(k) - y(t_{k-1})|   \\
&\leq&
\sum_{j=1}^n 
\left(\sup_{a_j = t_0 < t_1 < \cdots < t_n = b_j} 
             \sum_{k=1}^n  |x(k) - x(t_{k-1})| \right) \\
&& +\;
\left(\sup_{a_j = t_0 < t_1 < \cdots < t_n = b_j} 
             \sum_{k=1}^n  |y(k) - y(t_{k-1})| \right)
\\
&<& 2\epsilon
\end{eqnarray*}

Thus \(S(t)\) is absolutely continuous as well.
By Theorem~7.18 \cite{RudinRCA87} \(x(t)\), \(y(t)\) and \(S(\phi,0;t)\)
are differentiable almost everywhere and consequently
are differentiable \emph{simultaneously} almost everywhere.
Let \(J\subset[0,1]\)
the set of points all three functions are differentiable
and \(t\in J\).
We have
\begin{equation*}
S(t+h) -S(t) 
\geq d(t+h,t) 
= \sqrt{\bigl(x(t+h) - x(t)\bigr)^2 + \bigl(y(t+h) - y(t)\bigr)^2}
\end{equation*}
for any \(h>0\). Hence
\begin{align*}
S'(t)
&\geq \lim_{h\to 0^+}  
   \sqrt{\bigl(x(t+h) - x(t)\bigr)^2 + \bigl(y(t+h) - y(t)\bigr)^2} \bigm/ h \\
&= \lim_{h\to 0^+}  
   \sqrt{\left(\bigl(x(t+h) - x(t)\bigr)/h\right)^2 + 
         \left(\bigl(y(t+h) - y(t)\bigr)/h\right)^2}  \\
&= \sqrt{\bigl(x'(t)\bigr)^2 + \bigl(y'(t)\bigr)^2}.
\end{align*}
We will show the reversed inequality holds almost everywhere.
For any interval \(I=(\alpha,\beta)\subset[0,1]\) we use the notations
\begin{align*}
x(I) &= |x(\beta) - x(\alpha)| \\
y(I) &= |y(\beta) - y(\alpha)| \\
S(I) &= S(\phi,\alpha;\beta). 
\end{align*}
Define the ``bad set''
\begin{equation*}
A = \left\{t\in J: 
        S'(t) > \left(\bigl(x'(t)\bigr)^2 + \bigl(y'(t)\bigr)^2\right)^{1/2}
        \right\}. 
\end{equation*}
and its subsets
\begin{align}
A_n = \bigl\{ & t\in J:   \notag \\
      & \forall I=(\alpha,\beta)\ni t,\; 0<m(I) < 1/n  \notag \\
      & \Rightarrow  \label{eq:7.21:An}
        S(I)/m(I)
        \geq \left(
                \bigl(x(I)/m(I)\bigr)^2 + \bigl(y(I)/m(I)\bigr)^2
             \right)^{1/2} + 1/n
        \bigr\}
\qquad (n\in\N).
\end{align}
(Notice that \(g'(t) = \lim_{m(I)\to 0+} g(I)/m(I)\)
if $g$ is differentiable at $t$.)
Clearly \(A = \cup_{n\in\N} A_n\).

Fix $n$ and pick arbitrary \(\epsilon>0\).
There exist a ``sufficiently rich'' partition 
\begin{equation*}
0 = t_0 < t_1 < \cdots t_p = 1
\end{equation*}
putting \(J_k = [t_{k-1},t_k]\),
such that 
\(m(J_k) = t_k - t_{k-1} < 1/n\) for \(k\in\N_p\) and 
\begin{equation} \label{eq:7.21:Sleq}
S([0,1]) = \sum_{k=1}^p S(J_k) \leq \sum_{k=1}^p d(t_{k-1},t_k) + \epsilon
\end{equation}
On the other hand, by \eqref{eq:7.21:An} we have
\begin{equation} \label{eq:7.21:Sgeq}
S(J_k) \geq d(t_{k-1},t_k) + m(J_k)/n
\end{equation}
whenever \(j_k \cap A_n \neq \emptyset\). 
By \eqref{eq:7.21:Sleq} \eqref{eq:7.21:Sgeq} we have
\newcommand{\sumJkAn}{
            \sum_{\stackrel{1\leq k \leq p}{J_k\cap A_n \neq \emptyset}}}
\begin{equation*}
m(A_n) 
\leq \sumJkAn m(J_k) 
\leq n \sumJkAn S(J_k) - d(t_{k-1},t_k) 
\leq n\epsilon.
\end{equation*}
Since \(\epsilon\) was arbitrarily picked, \(m(A_n) = 0\) 
hence \(m(A) = 0\) and the desired reversed inequality holds.
Therefore
\begin{equation} \label{eq:ex7.21:Sdif}
S'(t) = \sqrt{\bigl(x'(t)\bigr)^2 + \bigl(y'(t)\bigr)^2}
\end{equation}
which completes the proof.
\end{thmproof}

For this specifc exercise, we note that 
in a curve of a function graph
\(x(t)=t\) and \(x'(t)=1\).
Now the lemma provides the solution.


%%%%%%%%%%%%%% 22
\begin{excopy}
\begin{itemize}
\itemch{a}
Assume that both $f$ and  its maximal function \(Mf\) are in \(L^1(\R^k)\).
Prove that then \(f(x)=0\;\aded[m]\).
\emph{Hint}: To every other \(f\in L^1(\R^k)\) corresponds a constant
\(c=c(f)>0\) such that
\begin{equation*}
  (Mf)(x) \geq |x|^{-k}
\end{equation*}
whenever \(|x|\) is sufficiently large.

\itemch{b}
Define \(f(x) = x^{-1}(\log x)^{-2}\) if \(0<x<\frac{1}{2}\),
\(f(x)=0\) on the rest of \(\R^1\).
Then \(f\in L^1(\R^1)\).
Show that
\begin{equation*}
  (Mf)(x) \geq |2x \log(2x)|^{-1} \qquad (0<x<1/4)
\end{equation*}
so that \(\int_0^1 (Mf)(x)\,dx = \infty\).
\end{itemize}
\end{excopy}

\begin{itemize}

\itemch{a}
Assume by negation \(\|f\|_1\neq 0\).
Then there exist and integer \(a\in\N\) such that 
\[\alpha = \int_{B(0,a)} |f(x)|\,dm(x) > 0.\]
Assume \(x\in\R^k\) and \(\|x\|\geq a\). 
We note that \(m(B_r) = c_k \pi r^k\) 
(where \(c_k = \pi^{k/2}/\Gamma(n/2+1)\)).
Now
\begin{align*}
(Mf)(x)
&= \sup_{r>0} \frac{1}{m(B_r)} \int_{B(x,r)} |f(t)|\,dt \\
&\geq \frac{1}{m(B_{|x|+a})} \int_{B(x,|x|+a)} |f(t)|\,dt 
 \geq \frac{1}{m(B_{|x|+a})} \int_{B(0,a)} |f(t)|\,dt \\
&= \alpha / m(B_{|x|+a}).
\end{align*}
Define and estimate for each integer \(n>a\)
\begin{align*}
S_n
&= \int_{B(0,n)} (Mf)(x)\,dm(x) \\
&\geq \sum_{r=a+1}^n \int_{B(0,r)\setminus B(0,r-1) } (Mf)(x)\,dm(x) 
 \geq \sum_{r=a+1}^n 
     \int_{B(0,r)\setminus B(0,r-1) } \alpha / m(B_{|x|+a}) \,dm(x) \\
&\geq \sum_{r=a+1}^n 
     m\bigl(B(0,r)\setminus B(0,r-1)\bigr) \alpha / m(B_{2r}) \\
&= \sum_{r=a+1}^n \alpha c_k\bigl(r^k - (r-1)^k\bigr) 
                  \bigm/ \bigl( c_k (r+a)^k \bigr) 
 = \frac{\alpha}{2^k} 
   \sum_{r=a+1}^n \bigl(r^k - (r-1)^k\bigr) \bigm/ r^k \\
&\geq \frac{\alpha}{2^k} 
      \sum_{r=a+1}^n \frac{1}{r}.
\end{align*}
Since \(\lim_{n\to\infty} S_n = \infty\) we get the
contradiction \(Mf \notin L^1(\R^k)\).

\itemch{b}
Put \(F(x) = -1/\log(x)\). Now
\begin{equation*}
F'(x) = (-1/x) \cdot \left(-\bigl(\log(x)\bigr)^{-2}\right) = f(x)
\end{equation*}
when \(0<x<1/2\).
Hence
\begin{equation*}
\|f\|_1 
= \lim_{h\to 0^+} \int_h^{1/2} f(x)\,dx
= -1/\log(1/2) - \lim_{h\to 0^+}  1/\log(h)
= -1/\log(1/2) 
< \infty.
\end{equation*}
Thus \(f\in L^1(\R^1)\).
But if  \(0<x<1/4\) then
\begin{equation*}
(Mf)(x) 
= \sup_{r} \frac{1}{2r} \int_{x-r}^{x+r} f(t)\,dt
\geq \frac{1}{2x} \int_0^{2x} f(t)\,dt
= \bigl(F(2x) - F(0)\bigr)/2x 
= |2x\log(2x)|^{-1} 
\end{equation*}
Hence
\begin{align*}
\int_0^1 (Mf)(x)\,dx
&\geq \int_0^{1/4} (Mf)(x)\,dx \\
&\geq \int_0^{1/4} \frac{1}{2x} \cdot \frac{-1}{\log(2x)} \,dx \\
&= \frac{-1}{2}\left.\left(\log(|\log(2x)|)\right)\right|_0^{1/4} 
 = \frac{-1}{2} \bigl(\log(1/2) - \lim_{x\to +\infty}\log(x)\bigr) \\
&= \infty.
\end{align*}
Indeed \(Mf \notin L^1([0,1])\).
\end{itemize}


%%%%%%%%%%%%%% 23
\begin{excopy}
The definition of Lebesgue points, as made in Sec.~7.6, applies to individual
integrable functions, not to the equivalence classes discussed in Sec.~3.10.
However, if \(F\in L^1(\R^k)\) is one of these equivalence classes, one may call
a point \(x\in \R^k\)
a \emph{Lebesgue point of}
\index{Lebesgue point}
$F$ if there is a complex number, let us call it \((SF)(x)\) to be $0$
at those points \(x\in \R^k\) that are not Lebesgue points of $F$.

Prove the following statement: If \(f\in F\), and $x$ is a Lebesgue
points of $f$, then $x$ is also a Lebesgue point of $F$, 
and \(f(x)(SF)(x)\).  Hence \(SF\in F\).

Thus $S$ ``selects'' a member of $F$ that has a \emph{maximal} set
of Lebesgue points.
\end{excopy}

Assume $x$ is
\index{Lebesgue point}
a~Lebesgue point of $f$.
The set $E$ of non Lebesgue points of $f$ has measure zero by 
Theorem~7.7 \cite{RudinRCA87}. Thus
changing $f$ on $E$ does not effect the values of 
\begin{equation*}
A(x) \frac{1}{m(B_r)} \int_{B(x,r} |f(y) - f(x)|\,dm(y)
\end{equation*}
except possibly for \(x\in E\).
Hence Lebesgue points of $f$ are Lebesgue points of $F$ as well.

%%%%%%%%%%%%%%%%%
\end{enumerate}

 % \setcounter{chapter}{7}  % -*- latex -*-

%%%%%%%%%%%%%%%%%%%%%%%%%%%%%%%%%%%%%%%%%%%%%%%%%%%%%%%%%%%%%%%%%%%%%%%%
%%%%%%%%%%%%%%%%%%%%%%%%%%%%%%%%%%%%%%%%%%%%%%%%%%%%%%%%%%%%%%%%%%%%%%%%
%%%%%%%%%%%%%%%%%%%%%%%%%%%%%%%%%%%%%%%%%%%%%%%%%%%%%%%%%%%%%%%%%%%%%%%%
\chapterTypeout{Integration on Product Spaces} % 8


%%%%%%%%%%%%%%%%%%%%%%%%%%%%%%%%%%%%%%%%%%%%%%%%%%%%%%%%%%%%%%%%%%%%%%%%
%%%%%%%%%%%%%%%%%%%%%%%%%%%%%%%%%%%%%%%%%%%%%%%%%%%%%%%%%%%%%%%%%%%%%%%%
\section{Notes}

%%%%%%%%%%%%%%%%%%%%%%%%%%%%%%%%%%%%%%%%%%%%%%%%%%%%%%%%%%%%%%%%%%%%%%%%
\subsection{Product of Complex Measures} \label{subsec:prod:complex:measures}

Following Theorem~8.6 that assumes the measures are \(\sigma\)-finite
which also implies that thy are positive, there is 
a definition of products \(\mu\times\lambda\) with the similar assumptions.
We need to generalize the notion for complex measures as well.
Given a measure \(\mu\) we can define the following decomposition
based on 
\index{Jordan Decomposition}
Jordan Decomposition.
\begin{alignat*}{2}
\mu_{r} &= \Re(\mu) \qquad& \mu_{i} &= \Im(\mu) \\
\mu_{r+} &= (|\mu_r|+\mu_r)/2  \qquad& \mu_{i+} &= (|\mu_i|+\mu_i)/2 \\
\mu_{r-} &= (|\mu_r|-\mu_r)/2  \qquad& \mu_{i-} &= (|\mu_i|-\mu_i)/2
\end{alignat*}
Now by distributive law for measures we get the desired expected generalization
of the definition of product space to complex (finite!) measures.


%%%%%%%%%%%%%%%%%%%%%%%%%%%%%%%%%%%%%%%%%%%%%%%%%%%%%%%%%%%%%%%%%%%%%%%%
\subsection{Young's Inequality} \label{subsec:young:ineq}

We bring here a proof of 
\index{Young!inequality}
Young's inequality based on lecture notes from \cite{Viaclovsky:18125:lec20}.

\begin{lthm}
Let \(p, q, r \in[1,\infty]\) such that
\begin{equation*}
\frac{1}{r} = \frac{1}{p} + \frac{1}{q} - 1
\end{equation*}
if \(f \in L^p(\R)\) and \(g\in L^q(\R)\), 
then \(f \ast g\) exists \aded\ and \(f\ast g\in L^r(\R)\). 
Moreover,
\begin{equation*}
\|f\ast g\|_r \leq \|f\|_p \cdot \|g\|_q\,. \label{eq:young:ineq}
\end{equation*}
\end{lthm}


\begin{thmproof}
If \(f=0\,\aded\) or \(g=0\,\aded\) then the inequality is trivial.
Without loss of generality, let \(\|f\|_p = \|g\|_q = 1\)
Since otherwise we can look at 
\(f/\|f\|_p\) and \(g/\|g\|_q\) instead. The general
case follows from the nonnegative
case, so assume \(f, g \geq 0\). 
Let \(p'\) and \(q'\) be the exponential conjugate of $p$ and $q$ respectably.
Using \(1/r+1/q'+1/p'=1\) and 
applying H\"older’s inequality Theorem~3.5,
\begin{align*}
(f\ast g)(x)
&= \int_{\R}\left(f(y)^{p/r}g(x-y)^{q/r}\right)f(y)^{1-p/r}g(x-y)^{1-q/r}\,dy \\
&\leq 
  \left(\int_{\R} f(y)^pg(x-y)^q\,dy\right)^{1/r}
     \left(\int_{\R} f(y)^{(1-p/r)q'}\,dy\right)^{1/q'}
        \left(\int_{\R} f(x-y)^{(1-q/r)p'}\,dy\right)^{1/p'}\,.
\end{align*}
Since 
\begin{equation*}
(1-p/r)q' = p \qquad\textnormal{and}\qquad (1-q/r)p' =q
\end{equation*}
we have
\begin{equation*}
(f\ast g)(x) 
\leq \left(\int_{\R} f(y)^pg(x-y)^q\,dy\right)^{1/r}\cdot1\cdot1.
\end{equation*}
Hence
\begin{equation*}
(f\ast g)^r(x) \leq \int_{\R} f(y)^pg(x-y)^q\,dy
\end{equation*}
That is \((f\ast g)^r \leq f^p \ast g^q\).
Now
\begin{align}
\|f\ast g\|_r^r
&= \int_{\R} (f\ast g)^r(x)\,dx \notag \\
&\leq  \int_{\R} (f^p \ast g^q)(x)\,dx = \|(^p \ast g^q\|_1 \notag \\
&\leq \|f^p\|_1 \|g^q\|_1 \label{eq:young:conv1} \\
&=  \|f\|_p^p \cdot \|g\|_q^q = 1. \notag
\end{align}
The inequality~\eqref{eq:young:conv1} is given by Theorem~8.14.
Thus desired \eqref{eq:young:ineq} was shown.
\end{thmproof}

%%%%%%%%%%%%%%%%%%%%%%%%%%%%%%%%%%%%%%%%%%%%%%%%%%%%%%%%%%%%%%%%%%%%%%%%
%%%%%%%%%%%%%%%%%%%%%%%%%%%%%%%%%%%%%%%%%%%%%%%%%%%%%%%%%%%%%%%%%%%%%%%%
\section{Exercises} % pages 174-177

%%%%%%%%%%%%%%%%%
\begin{enumerate}
%%%%%%%%%%%%%%%%%


%%%%%%%%%%%%%% 1
\begin{excopy}
Find a monotone class \frakM\ in \(\R^1\) which is not a \salgebra,
even though \(\R^1 \in \frakM\) and \(\R^1\setminus A \in \frakM\)
for every \(A\in \frakM\).
\end{excopy}

Take
\begin{equation*}
\frakM = \bigl\{\emptyset, \R^1, \; 
                \{0\}, \R^1\setminus\{0\}, \;
                \{1\}, \R^1\setminus\{1\}  \bigr\}.
\end{equation*}
Note that it is a monotone class but \(\{0,1\}\notin \frakM\).


%%%%%%%%%%%%%% 2
\begin{excopy}
Suppose $f$ is a Lebesgue measurable nonnegative real function on \(\R^1\)
and \(A(f)\) is the 
\emph{ordinate set}
\index{ordinate set}
of $f$. This is the set of all points \((x,y)\in\R^2\) for which \(0<y<f(x)\).
\begin{itemize}
\itemch{a} Is it true that \(A(f)\) is Lebesgue measurable, 
           in the two dimensional sense?
\itemch{b} If the answer to \ich{a} is affirmative, 
           is the integration of $f$ over \(\R^1\) 
           equal to the measure of \(A(f)\)?
\itemch{c} Is the graph of $f$ a measurable subsection of \(\R^2\)?
\itemch{d} If the answer to \ich{c} is affirmative, 
           is the measurable of the graph equal to zero?
\end{itemize}
\end{excopy}

\begin{itemize}
\itemch{a}
Yes.
Define simple functions
\begin{equation*}
f_n(x) = \lfloor nf(x)\rfloor / n \qquad (n\geq 1).
\end{equation*}
Clearly \(\lim_{n\to\infty} f_n = f\).
Since \(A(f_n)\) is a countable union of rectangles, it is measurable.
So is \(A(f) = \cup_n A(f_n)\).

\itemch{b}
Yes. Using the notations if \ich{a}, we have 
\begin{equation*}
m_2\bigl(A(f_n)\bigr) 
= \sum_{k=0}^\infty (k/n)\cdot m\bigl(\{x\in\R^1: f_n(x) = k/n\}\bigr)
= \int_{\R^1} f_n(t)\,dm(t).
\end{equation*}
The desired equality is derived 
by Lebesgue monotone convergence theorem.

\itemch{c}
Yes, see \ich{d}.

\itemch{d}
Yes.
Pick \(\epsilon>0\). Define
\begin{equation*}
s(x) = \epsilon \cdot 2 ^{-\lfloor x \rfloor} > 0
\end{equation*}
Clearly 
\begin{equation*}
\int_{\R} s(x)\,dx = 2 \epsilon \sum_{n=0}^\infty 2^{-n} = 4\epsilon.
\end{equation*}
Now 
\begin{equation*}
G 
= \{(x,f(x)): x\in\R\} 
\subset \{(x,y)\in\R^2: f(x)-s(x) < y < f(x)+s(x)\}.
\end{equation*}
and 
\begin{align*}
m(G) 
&\leq m\bigl(\{(x,y)\in\R^2: f(x)-s(x) < y < f(x)+s(x)\}\bigr) \\
&=  m\bigl(\{(x,y)\in\R^2: 0 < y < 2s(x)\}\bigr)
= 2\int_R s(x)\,dx = 8\epsilon.
\end{align*}
Since \(\epsilon\) is arbitrary, we have \(m(G)=0\).
\end{itemize}


%%%%%%%%%%%%%% 3
\begin{excopy}
Find an example of a positive continuous function $f$
in the open unit square in \(\R^2\),
whose integral (relative to Lebesgue measure)
is finite but such that \(\varphi(x)\) 
(in the notation of Theorem~8.8)
is infinite for some \(x \in (0,1)\).
\end{excopy}

Let \(U = (0,1)^2 \subset \R^2\) be the open unit square
and \(C = (1/2,0)\) a point on its boundary.
For \(n\geq 1\) define the open triangle
\begin{equation*}
T_n = \triangle\bigl(C-(1/n, 0),\; C+(1/n, 0),\; C+(0, 1/n)\bigr).
\end{equation*}
Clearly \(T_n \supset T_{n+1}\) is a decreasing sequence, 
\(m(T_n) = 1/n^2\) and \(\cap_n T_n = \emptyset\).
By  Urysohn's Lemma~2.12 \cite{RudinRCA87}, 
\index{Urysohn's lemma}
there exists
\(f_n:U\to[0,1]\) such that 
\begin{equation*}
U \setminus T_n \prec f_n \prec T_{n+1}.
\end{equation*}
Let \(f = \sum_{n=1}^\infty f_n\).
Now
\begin{equation*}
\int_U f(x,y)\,dm(x,y)
= \sum_{n=1}^\infty \int_U f_n(x,y)\,dm(x,y)
\leq \sum_{n=1}^\infty \int_U \chhi_{T_n}(x,y)\,dm(x,y)
= \sum_{n=1}^\infty 1/n^2
< \infty.
\end{equation*}
But for \(x=1/2\)
\begin{align*}
\int_0^1 f_x(y)\,dy 
&= \int_0^1 f(1/2,y)\,dy
= \sum_{n=1}^\infty \int_0^1 f(1/2,y)\,dy \\
&\geq \sum_{n=1}^\infty \int_0^1 \chhi_{T_{n+1}}(1/2,y)\,dy
= \sum_{n=1}^\infty 1/(n+1)
= \infty.
\end{align*}
 

%%%%%%%%%%%%%% 4
\begin{excopy}
Suppose \(1\leq p \leq \infty\), \(f\in L^1(\R)\) and \(g\in L^p(\R)\).
\begin{itemize}
\itemch{a}
Imitate the proof of Theorem~8.14 to show that the integral defining
\((f\ast g)(x)\) exists for almost all $x$, that \(f\ast g \in L^p(\R)\),
and that 
\begin{equation*}
\| f \ast g \|_p \leq \|f\|_1 \|g\|_p.
\end{equation*}

\itemch{b}
Show that equality can hold in \ich{a} if \(p=1\) and if \(p=\infty\),
and find the conditions under which this happens.

\itemch{c}
Assume \(1 < p < \infty\) and equality holds in \ich{a}.
Show that then either 
\(f=0\;\aded\) or 
\(g=0\;\aded\)

\itemch{d}
Assume \(1\leq p \leq \infty\), \(\epsilon>0\), and show that there exists
\(f\in L^1(\R^1)\) and \(g\in L^p(\R^1)\) such that 
\begin{equation*}
\| f \ast g \|_p > (1-\epsilon) \|f\|_1 \|g\|_p.
\end{equation*}
\end{itemize}
\end{excopy}

For a generalization with 
\index{Young}
Young inequalities, see \cite{EdwFA} Theorem~9.5.1 page~655.

\begin{itemize}
\itemch{a}
Let's first assume \(p=\infty\), then
\begin{align*}
\|f \ast g\|_\infty
&= \esssup_{x\in\R} \left| \int_{\R} f(y)g(x-y)\,dy\right|
\leq \esssup_{x\in\R}  \int_{\R} |f(y)g(x-y)|\,dy \\
&\leq \|g\|_\infty \esssup_{x\in\R} \int_{\R} |f(y)|\,dy
 = \|f\|_1 \|g\|_\infty
\end{align*}

So now we may assume \(p<\infty\). We use 
% generalization of
 H\"older's inequality 
\index{Holder@H\"older}
Theorem~3.5 \cite{RudinRCA87}
% (see local lemma~\ref{llem:hlp:188}).
Let \(q = p / (p - 1)\) be the exponent conjugate.
\begin{align}
|(f \ast g)(t)|
&\leq \int_{\R} |f(t-s)g(s)|\,ds \notag \\
&= \int_{\R} 
    \left(|f(t-s)|^{1/p} |g(s)|\right) 
    \left(|f(t-s)|^{1/q}\right) 
    \,ds \notag \\
&\leq \label{eq:ex8.4a:holder}
      \left( \int_{\R} \left(|f(t-s)|^{1/p} |g(s)|\right)^p\,ds \right)^{1/p}
      \left( \int_{\R} \left(|f(t-s)|^{1/q}\right)^q \,ds \right)^{1/q} \\
&=   \left( \int_{\R} |f(t-s)| \cdot |g(s)|^p\,ds \right)^{1/p}
      \left( \int_{\R} |f(t-s)| \,ds \right)^{1/q} \notag \\
&=   \left( \int_{\R} |f(t-s)| \cdot |g(s)|^p\,ds \right)^{1/p} \|f\|_1^{1/q}
     \notag
\end{align}
The above inequality holds for almost all $t$. Hence
\begin{align}
\|f \ast g\|_p^p
&\leq \|f\|_1^{p/q} \notag
      \int_\R 
        \left(\int_{\R} |f(t-s)| \cdot |g(s)|^p\,ds \right)^{(1/p)p}\,dt \\
&= \|f\|_1^{p/q} \notag
   \int_\R \int_{\R} |f(t-s)| \cdot |g(s)|^p\,ds\,dt \\
&= \|f\|_1^{p/q} \label{eq:8.4:fub}
   \int_\R \int_{\R} |f(t-s)| \cdot |g(s)|^p\,dt\,ds \\
&= \|f\|_1^{p/q} \notag
   \int_\R |g(s)|^p \int_{\R} |f(t-s)| \,dt\,ds \\
&= \|f\|_1^{p/q} \cdot \|g\|_p^p \cdot \|f\|_1  \notag
 = \|f\|_1^{p/(p/(p-1)) + 1}  \|g\|_p^p \\
&=  \|f\|_1^p \cdot \|g\|_p^p  \notag
\end{align}
The \eqref{eq:8.4:fub} equality is by Fubini Theorem~8.8 \cite{RudinRCA87}.
Thus \(\|f \ast g\|_p =  \|f\|_1 \|g\|_p\) as desired.

\itemch{b}
\emph{Note:} This is a result of 
\index{Riesz-Thorin}
\index{Thorin}
Riesz-Thorin theorem.

% Clearly if \(f=0\,\aded\) or \(g=0\,\aded\) then equality holds.
Let \(h = f\ast g\).
By the proof of Theorem~8.14, when \(p=1\), there is an equality iff
\begin{equation*}
|h(x)| 
= \left|\int_{-\infty}^\infty f(x-y)g(y)\,dy\right|
= \int_{-\infty}^\infty |f(x-y)g(y)|\,dy
\end{equation*}
This happens iff \(f(x-y)g(y)\) 
has the same argument almost everywhere on~$x$ and~$y$.
Equivalently, since \(x-y\) covers all of \(\R\), iff
$f$ and $g$ wach has constant argument \aded.

When \(p=\infty\) then equality holds 
if for any \(\epsilon>0\) there exists some \(x\in\R\) 
and \(\theta\in[0,2\pi]\)
such that 
\begin{equation*}
\int_{\R}\left| f(y)g(x-y)- e^{i\theta}|f(y)|\cdot\|g\|_\infty\right|\,dx 
< \epsilon.
\end{equation*}


\itemch{c}
Assme \(1<p<\infty\).
If equality holds, then equality must hold for almost all \(t\in\R\) in 
\eqref{eq:ex8.4a:holder}.
By Local~Lemma~\ref{llem:hlp:188} (H\"older), both expressions:
\begin{equation*}
|f(t-s)|^{1/p} |g(s)|  \qquad  |f(t-s)|^{1/q}
\end{equation*}
must be effectively proportional for all \(t\in\R\).
Equivalently, 
\begin{equation*}
|g(s)|  \qquad  |f(t-s)|^{1/q-1/p}
\end{equation*}
must be effectively proportional. 
Hence \(f=0\,\aded\) or \(g=0\,\aded\).

\itemch{d}
Let \(f_n(x) = n\chhi_{[-1/n,+1/n]}/2\) and \(g = \chhi_{[0,1]}\).
Clearly \(\|f_n\|_1=1\) and
\(\|g\|_p = 1\). Now 
\begin{align*}
(f_n\ast g)(x) 
&= \int_{-\infty}^\infty f_n(x-t)g(t)\,dt
 = \int_0^1 f_n(x-t)\,dt
 = (n/2) \int_0^1 \chhi_{[-1/n,+1/n]}(x-t)\,dt \\
&= \left\{\begin{array}{ll}
    n(x+1/n)/2 \qquad &  -1/n \leq x \leq 1/n \\
    1    \qquad    1/n \leq x \leq 1-1/n \\
    n(x-(1-1/n))/2 \qquad &  -1/n \leq x \leq 1/n \\
    0 \qquad              & \textnormal{otherwise}
   \end{array}\right.
\end{align*}
By Lebesgue convergence theorems
\(\lim_{n\to\infty} \|f_n\ast g\|_p = 1\).
Hence for any \(\epsilon>0\) we can find some $n$ such that 
\begin{equation*}
\| f_n \ast g \|_p > 
(1-\epsilon) =
(1-\epsilon) \|f\|_1 \|g\|_p.
\end{equation*}
\end{itemize}


%%%%%%%%%%%%%% 5
\begin{excopy}
Let $M$ be the Banach space of all complex Borel measures on \(\R^1\).
The norm in $M$ is \(\|\mu\| = |\mu|(\R^1)\).
Associate to each Borel set \(E \subset \R^1\) the set
\begin{equation*}
E_2 = \{ (x,y): x+y \in E\} \subset \R^2.
\end{equation*}
If \(\mu\) and \(\lambda \in M\) define their convolution \(\mu \ast \lambda\)
to be the set function given by
\begin{equation*}
(\mu \ast \lambda)(E) = (\mu \times \lambda)(E_2)
\end{equation*}
for every Borel set \(E\subset \R^1\);
\(\mu\times\lambda\) is as in Definition~8.7.
\begin{itemize}
\itemch{a}
Prove that \(\mu \ast \lambda \in M\) and that 
\(\|\mu \ast \lambda\| \leq \|\mu\| \, \|\lambda \|\).

\itemch{b}
Prove that \(\mu \ast \lambda \) is the unique \(\nu\in M\) such that 
\begin{equation*}
\int f\,d\nu = \int\int f(x+y)\,d\mu(x)\,d\nu(y)
\end{equation*}
for every \(f\in C_0(\R^1)\). (all integrals extend over \(\R^1\).)

\itemch{c}
Prove that convolution in $M$ is commutative, associative, and distributive
with respect to addition.

\itemch{d}
Prove the formula
\begin{equation*}
(\mu \ast \lambda)(E) = \int \mu(E-t)\,d\lambda(t)
\end{equation*}
for every \(\mu\) and \(\lambda \in M\) and every Borel set $E$. Here
\begin{equation*}
E - t = \{x - t: x\in E\}.
\end{equation*}

\itemch{e}
Define \(\mu\) to be 
\emph{discrete} 
\index{discrete!measure} 
if \(\mu\) is concentrated on a countable set;
define \(\mu\) to be 
\emph{continuous}
\index{continuous!measure}
if \(\mu(\{x\}) = 0\) for every \(x\in\R^1\);
let $m$ be Lebesgue measure on \(\R^1\)
(note that \(m\notin M\)).
Prove that  \(\mu \ast \lambda\) is discrete if both \(\mu\) and \(\lambda\)
are discrete, that \(\mu \ast \lambda\) is continuous if \(\mu\) is continuous
and \(\lambda\in M\), and that \(\mu \ast \lambda \ll m \) if \(\mu \ll m\).

\itemch{f}
Assume
\(d\mu = f\,dm\), 
\(d\lambda = g\,dm\),
\(f\in L^1(\R^1)\), 
\(g\in L^1(\R^1)\), 
and prove that \\
\(d(\mu\ast\lambda) = (f\ast g)\,dm\).

\itemch{g}
Properties \ich{a} and \ich{c} show that the Banach space $M$ is what one calls
\index{commutative!Banach algebra}
\emph{commutative Banach algebra}.
Show that \ich{e} and \ich{f} imply that the set of all discrete measures in $M$
is a subalgebra of $M$,
that the continuous measures form an ideal in $M$, and that the absolutely
continuous measures (relative to $M$) form an ideal in $M$ which is isomorphic 
(as an algebra) to \(L^1(\R^1)\).

\itemch{h}
Show that $M$ has a unit, i.e., show that there exists a \(\delta\in M\)
such that \(\delta \ast \mu = \mu\) for all \(\mu \in M\).

\itemch{i}
Only two properties of \(\R^1\) have been used in this discussion: \(\R^1\)
is a commutative group (under addition), and there exists a translation
invariant Borel measure $M$ on \(\R^1\) which is not identically $0$ and 
which is finite on all compact subsets of \(\R^1\).
Show that the same results hold if \(\R^1\)
is replaced by \(\R^k\) or by $T$ (the unit circle) or by \(T^k\)
(the $K$-dimensional torus, the cartesian product of $k$
copies of $T$),
as soon as the definitions are properly formulated.
\end{itemize}
\end{excopy}

\emph{Note:} The definition of \(\mu\times\lambda\) where 
\(\mu\) and \(\lambda\) are complex measures requires a generalization
of the definition done in section~8.7 of \cite{RudinRCA87},
see \ref{subsec:prod:complex:measures} above.

Utilizing the notations of Theorem~8.6 gives:
\begin{align*}
(E_2)_t &= \{u\in\R: (t,u)\in E_2\} = \{u\in\R: t+u\in E\} 
    = \{x-t: x\in E\} = E-t \\
(E_2)^u &= \{t\in\R: (t,u)\in E_2\} = \{t\in\R: t+u\in E\} 
    = \{y-u: y\in E\} = E-u.
\end{align*}

\begin{itemize}
\itemch{a}
Let \(\frakM\) be the set of all Borel sets \(E\subset \R\)
such that \(E_2\) is a Borel set in \(\R^2\).

% Assume \(E\subset\R\) is a Borel set. We need to whow that 
% \(E_2\) is a Borel set in \(\R^2\).x
% We will do this in steps.

% First we assume that 
If $E$ is open, then it is trivial to see that~\(E_2\) is open. 
If \(E\in\frakM\) then 
\begin{equation*}
C_2 
= \{(x,y)\in\R^2: x+y \in \R\setminus E\}
= \R^2 \setminus \{(x,y)\in\R^2: x+y \in E\}
= \R^2 \setminus E_2\,,
\end{equation*}
hence \((\R\setminus E)\in\frakM\).

Let \(E = \cup_{n\in\N}B_n\) a Partition of Borel sets
and assume \(B_n\in\frakM\) for all \(n\in\N\), then
\begin{equation*}
E_2 
= \{(x,y)\in\R^2: x+y \in \cup_{n\in\N}B_n\}
= \bigcup_{n\in\N}B_n \{(x,y)\in\R^2: x+y \in B_n\}
\end{equation*}
hence \(E\in\frakM\).
Therefore \frakM\ is a \salgebra\ that contains the open sets,
and so it contains all the Borel sets.


Now we have to show that \(\mu\ast\lambda\) is a measure.
Let \(E=\disjunion_{n\in\N} B_n\) a partition of Borel sets.
\begin{align*}
(\mu\ast\lambda)(E)
&= (\mu\times\lambda)\left(\{(x,y)\in\R^2: x+y\in\disjunion_{n\in\N} B_n\}\right)
  \\
&= \sum_{n\in\N}(\mu\times\lambda)\left(\{(x,y)\in\R^2: x+y \in B_n\}\right) \\
&= \sum_{n\in\N}(\mu\ast\lambda)(B_n)
\end{align*}

We note that if \(A,B\subset\R\) are disjoint, then
\(A_2,B_2\subset\R^2\) are disjoint as well. Hence
\begin{align*}
\left|\mu\ast\lambda\right|
&= |\mu\ast\lambda|(\R)
 = \sup_{\R=\disjunion_{n\in\N} F_n} 
   \left(\sum_{n\in\N} \left| (\mu\ast\lambda) (F_n)\right|\right)
 = \sup_{\R=\disjunion_{n\in\N} F_n} 
   \left(\sum_{n\in\N} \left| (\mu\times\lambda) (F_n)_2\right|\right)
   \\
&\leq (|\mu|\times|\lambda|)(\R^2) 
 = \|\mu\|\cdot\|\lambda\|.
\end{align*}
The last equality is a trivial applying of 
\index{Fubini}
Fubini's Theorem~8.8.

\itemch{b}
The Borel measure is determined by its value on open sets.
In the case of \(\R\) therefore, it is determined by the value on intervals.
But for any interval $I$ we can aproximate \(\chhi_I\) by functions
from \(C_0(\R)\). Thus the uniqueness follows.

\itemch{d}
\emph{Note:} before \ich{c} since we will need this one there.
First we have the set manipulation
\begin{equation*}
(E_2)^y 
= \{x\in\R: (x,y)\in E_2\}
= \{x\in\R: x+y\in E\}
= \{t-y\in\R: t\in E\}
= E - y.
\end{equation*}
Now
\begin{equation*}
(\mu \ast \lambda)(E) 
= (\mu\times\lambda)(E_2) 
= \int_{\R} \mu\left(E_2)^t\right)\,d\lambda(t)
= \int_{\R} \mu(E-t)\,d\lambda(t)
\end{equation*}

\itemch{c}
Let \(\lambda,\mu,\nu\in M\) and a Borel set \(E\subset\R\),
and \(E_2\) defined as above.

\paragraph{Convolution is commutative.}
\begin{equation*}
(\lambda\ast \mu)(E)
= (\lambda\times\mu)(E)
= (\mu\times\lambda)(E)
= (\mu\ast\lambda)(E)
\end{equation*}

We note the set equality:
\begin{gather*}
(E^y)_2 = (E-y)_2 
 = \{(t,u)\in\R^2: t+u\in E-y\}
 = \{(t,u)\in\R^2: t+u+y\in E\} \\
%
\begin{align*}
\bigl((E-y)_2\bigr)^s 
&= \{r\in\R: (r,s)\in (E-y)_2\}
 = \{r\in\R: r+s\in E-y\} \\
&= \{r\in\R: r+s+y\in E\}
 = E-(y+s)
\end{align*}
\end{gather*}

\paragraph{Convolution is associative.}
Using Theorem~8.6 and its notations
\begin{align*}
\bigl((\lambda\ast \mu)\ast \nu\bigr)(E)
&=\bigl((\lambda\ast \mu)\times \nu\bigr)(E_2)
 % = \int_{\R} \nu\bigl((E_2)_x\bigr)\,d(\lambda\ast \mu)(x) \\
 = \int_{\R} (\lambda\ast \mu)\bigl((E_2)^y\bigr)\,d\nu(y) \\
&= \int_{\R} (\lambda\ast \mu)(E-y)\,d\nu(y) 
 = \int_{\R} (\lambda\times \mu)\bigl((E-y)_2\bigr)\,d\nu(y) \\
&= \int_{\R} 
   \left(
      \int_{\R}\lambda\left(\bigl((E-y)_2\bigr)^s\right)\,d\mu(s)
   \right)\,d\nu(y) \\
&= \int_{\R} \left(\int_{\R}\lambda(E-y-s)\,d\mu(s)\right)\,d\nu(y)
\end{align*}

Summerizing the above
\begin{equation} \label{eq:measure:conv:assoc}
\bigl((\lambda\ast \mu)\ast \nu\bigr)(E)
= \int_{\R} \left(\int_{\R}\lambda(E-y-s)\,d\mu(s)\right)\,d\nu(y)
\end{equation}

We now use commutativity and \eqref{eq:measure:conv:assoc} twice:
\begin{equation*}
\bigl((\lambda\ast \mu)\ast \nu\bigr)(E)
\bigl((\mu\ast \lambda)\ast \nu\bigr)(E)
= \int_{\R} \left(\int_{\R}\mu(E-y-s)\,d\lambda(s)\right)\,d\nu(y)
\end{equation*}
and
\begin{equation*}
\bigl(\lambda\ast (\mu\ast \nu)\bigr)(E)
\bigl((\mu\ast \nu)\ast \lambda\bigr)(E)
= \int_{\R} \left(\int_{\R}\mu(E-y-s)\,d\nu(s)\right)\,d\lambda(y)
\end{equation*}
By Fubini's Theorem~8.8 the last integrals of the last two equalities
are equal
\begin{equation*}
  \int_{\R} \left(\int_{\R}\mu(E-y-s)\,d\lambda(s)\right)\,d\nu(y)
= \int_{\R} \left(\int_{\R}\mu(E-y-s)\,d\nu(s)\right)\,d\lambda(y)
\end{equation*}
hence,
\begin{equation*}
  \bigl((\lambda\ast \mu)\ast \nu\bigr)(E) 
= \bigl(\lambda\ast (\mu\ast \nu)\bigr)(E).
\end{equation*}

\paragraph{Convolution is distributive.}
\begin{align*}
\bigl((\lambda+\mu)\ast \nu)(E)
&= \int_{\R} (\lambda+\mu)(E-y)\,d\nu(y)
     \int_{\R} \lambda(E-y)\,d\nu(y)
   + \int_{\R} \mu(E-y)\,d\nu(y) \\
&=  (\lambda+\mu)\ast \nu)(E)
  + (\mu+\mu)\ast \nu)(E)
\end{align*}
The distribution of other side follows by commutativity.

\itemch{e}
\textbf{Discreteness.}
Assume both \(\mu\) and \(\lambda\) are discrete.
Let \(A,B\subset\R\) the countable sets on which the measures are concentrated
respectably. We freely use the notations like \(\mu(\{x\})=\mu(x)\).
Now
\begin{align*}
(\lambda\ast\mu)(E)
&= \int_{\R} \lambda(E-y)\,d\mu(y)
 = \sum_{y\in B} \lambda(E-y)\mu(y)
 = \sum_{y\in B} \left(\sum_{x\in A\cap (E-y)}\lambda(x)\right)\mu(y) \\
&= \sum{x\in A}\sum_{y\in B} \chhi_E(x+y)\lambda(x)\cdot\mu(y)
\end{align*}
Therefore \(\lambda\ast\mu\) is concentrated in \(\{x+y: (x,y)\in A\times B\}\).
and 
\begin{equation*}
(\lambda\ast\mu)(w) = \sum_{\stackrel{(x,y)\in  A\times B}{x+y=w}} \lambda(x)\cdot\mu(y).
\end{equation*}


\textbf{Continuity.}
Assume \(\mu\) is continuous. Pick arbitrary \(x\in\R\).
\begin{equation*}
(\mu\ast\lambda)(x) 
= \int_{\R} \mu(x-y)\,d\lambda(y) = 
= \int_{\R} 0\,d\lambda(y) = 0.
\end{equation*}
Hence \(\mu\ast\lambda\) is continuous.

\textbf{Absolute Continuity.}
Assume \(\mu \ll m\) and \(m(E)=0\). 
Clearly \(\mu(E-y)=0\) for each \(y\in\R\) since \(m(E-y)=0\)
by construction of the Lebesgue measure. Now
\begin{equation*}
(\mu\ast \lambda)(E) 
= \int_{\R} \lambda(E-y)\,d\mu(y)
= \int_{\R} 0\,d\mu(y) = 0
\end{equation*}

\itemch{f}
Take a Borel set $E$. Using the fact that $m$ is a measure that 
is invariant under translation and 
\index{Fubini}
Fubini's Theorem~8.8
\begin{align*}
(\mu\ast\lambda)(E)
&= \int_{\R} \chhi_E\,d(\mu\ast\lambda)
 = \int_{\R} \mu(E-y)\,d\lambda(y)
 = \int_{\R} \left(\int_{E-y}\!f(x)\,dm(x)\right)g(y)\,dm(y) \\
&= \int_{\R} \left(\int_E f(x-y)\,dm(x)\right)g(y)\,dm(y) \\
&= \int_{\R} \left(\int_{\R} \chhi_E(x) f(x-y)g(y)\,dm(x)\right)\,dm(y) \\
&= \int_{\R} \left(\int_{\R} \chhi_E(x) f(x-y)g(y)\,dm(y)\right)\,dm(x) \\
&= \int_{\R} \chhi_E(x) \left(\int_{\R} f(x-y)g(y)\,dm(y)\right)\,dm(x) 
 = \int_{\R} \chhi_E(x) (f\ast g)(x)\,dm(x) \\
&= \int_{\R} \chhi_E(x)\,d(f\ast g)(x)m(x)
\end{align*}


\itemch{g}
Discrete measure are subalgebra by \ich{e} and \ich{f}.
Closure under addition and scalar multiplication is trivial.

Let $C$ be the family of continuous measures.
Again, closure under addition and scalar multiplication is trivial.
By \ich{e} and \ich{f} we have shown that $C$ is an ideal.

Let $A$ be the family of absolutely continuous measures with respect to $m$.
Again, closure under addition and scalar multiplication is trivial.
If \(\mu\ll m\) then by 
\index{Radon-Nikodym}
Radon-Nikodym Theorem~6.10 trivially extended to complex measures,
there exists $m$-measurable $f$
such that \(f\,dm = d\mu\). With this,\ich{e} and \ich{f} $A$ is an ideal
and the mapping \(\mu \to f\) is an isomorphism.


\itemch{h}
Put \(\delta(E) = \chhi_E(0)\). Verification is trivial.

\itemch{i}
For \(E\subset \T^k\) we need to modify the definition 
\begin{equation*}
E_2 = \left\{(u_1,u_2)\in \left(\T^k\right)^2: \exists u\in E,\;
         \forall j\in\N_k,\; (z_1)j\cdot(z_2)_j = u_j\right\}
\end{equation*}
Now all arguments can be easily applied to \(\R^k\) and \(\T^k\).
\end{itemize}


%%%%%%%%%%%%%% 6
\begin{excopy}
(Polar coordinates in \(\R^k\).)
Let \(S_{k-1}\) be the unit sphere in \(\R^k\), 
i.e., the set of all \(u \in \R^k\) whose distance from the origin $0$ is $1$.
Show that every \(x\in \R^k\), except for \(x=0\), has a unique representation
of the form \(x=ru\), where $r$ is a positive real number and \(u\in S_{k-1}\).
Thus \(\R^k\setminus\{0\}\) may be regarded as the cartesian product
\((0,\infty)\times S_{k-1}\).

Let \(m_k\) be the Lebesgue measure on \(\R^k\), and define a measure
\(\sigma_{k-1}\) on \(S_{k-1}\) as follows:
If \(A \subset S_{k-1}\) and $A$ is a Borel set, let \(\overline{A}\)
be the set of all points \(ru\), where \(0 < r < 1\)
and \(u\in A\), and define
\begin{equation*}
\sigma_{k-1}(A) = k \cdot m_k(\overline{A}).
\end{equation*}
Prove that the formula
\begin{equation*}
\int_{R^k} f\,dm_k 
= \int_0^\infty r^{k-1}\,dr \int_{S_{k-1}} f(ru)\,d\sigma_{k-1}(u)
\end{equation*}
is valid for every nonnegative Borel function $f$ on \(\R^{k}\).
Check that this coincides with familiar results 
when \(k=2\)
and when \(k=3\).

\emph{Suggestion}: If \(0 < r_1 < r_2\) and if $A$ 
is an open subset of \(S_{k-1}\), let $E$ be the set of all \(ru\) with
\(r_1 < r < r_2\), \(u\in A\),
and verify that the formula holds for the characteristic function of $E$.
Pass from these to characteristic functions of Borel sets in \(\R^k\).
\end{excopy}

Pick $E$ as suggested. Now
\begin{align*}
\int_{R^k} \chhi_E\,dm_k 
&= m_k(E) 
 = m_k\left(\{ru\in \R^k: u\in A\;\wedge\; r_1<r<r_2\}\right) \\
&=   m_k\left(\{ru\in \R^k: u\in A\;\wedge\; r<r_2\}\right)
   - m_k\left(\{ru\in \R^k: u\in A\;\wedge\; r\leq r_1\}\right) \\
&= m_k(\overline{A})\left(r_2^k - r_1^k\right)
 = k\int_{r_1}^{r_2} r^{k-1}m(\overline{A})\,dr \\
&= \int_{r_1}^{r_2} r^{k-1}\sigma_{k-1}(A)\,dr
 = \int_{r_1}^{r_2} r^{k-1}\left(\int_A 1\,d\sigma_{k-1}(u)\right)\,dr \\
&= \int_0^\infty r^{k-1}\,dr \int_{S_{k-1}} \chhi_E(ru)\,d\sigma_{k-1}(u)
\end{align*}
Thus the desired equality holds for \(f=\chhi_E\).

Now every Borel function \(f\geq 0\) can be monotonically approximated
by simple functions of the forum
\begin{equation*}
s(x) = \sum_{j\in\N} a_j\chhi_{E_j}
\end{equation*}
where \(a_j\geq 0\) and \(E_j\) are sets of the suggested type.
Note: For such sets, \(A_j\) sets can be picked so they are intersection
of \(B_k(u;\epsilon)\) with \(u\in S_{k-1}\) and \(\epsilon>0\).
By Lebesgue's monotone~ convergence Theorem~1.34
the equality holds for non-negative Borel functions.


%%%%%%%%%%%%%% 7
\begin{excopy}
Suppose \((X,\calG, \mu)\)
and \((Y,\calF, \lambda)\)
are finite measure spaces, and suppose \(\psi\) is a measure on
\(\calG\times\calF\) such that 
\begin{equation*}
\psi(A\times B) = \mu(A) \mu(B)
\end{equation*}
whenever \(A\in\calG\) and \(B\in\calF\).
Prove that then \(\psi(E) = (\mu\times\lambda)(E)\) 
for every \(E\in \calG\times\calF\).
\end{excopy}

Let \calE\ be the family if sets $E$ for which the desired equality holds.
Clearly all the elementary sets are in \calE\ and it is monotonic.
By Theorem~8.3  \(\calG\times\calF \subset \calE\).


%%%%%%%%%%%%%% 8
\begin{excopy}
\begin{itemize}
\itemch{a}
Suppose $g$ is a real function on \(\R^k\) such that each section \(f_x\)
is a Borel measurable and each section \(f^y\) is continuous.
Prove that $f$ is Borel measurable on \(\R^2\).
Note the contrast between this and Example~8.9(c).
\itemch{b}
Suppose $g$ is a real function on \(\R^k\) which is continuous in each 
of the $k$ variables separately.
More explicitly, for every choice of \(x_2,\ldots\,x_n\), the mapping
\(x_1 \to g(\seqn{x})\) is continuous, etc.
Prove that $g$ is a Borel function.
\end{itemize}
\emph{Hint}: If \((i-1)/n = a_{i-1} \leq x \leq a_i = i/n\), put
\begin{equation*}
f_n(x,y) = 
\frac{a_i - x}{a_i - a_{i-1}} f(a_{i-1}, y) 
+
\frac{x - a_{i-1}}{a_i - a_{i-1}} f(a_i, y).
\end{equation*}
\end{excopy}

\begin{itemize}
\itemch{a}
We first look at the function \(g_2:\R^2\to\R^2\) defined by
\(g_2(x,y) = (x,f_a(y))\).
Pick a base open set 
\begin{equation*}
G = (u-\delta_x,u+\delta_x)\times(v-\delta_y,u+\delta_y) \subset \R^2.
\end{equation*}
Now
\begin{equation*}
g_2^{-1}(G) = (u-\delta_x,u+\delta_x)\times f_a^{-1}(v-\delta_y,u+\delta_y).
\end{equation*}
which is a cartesian product of two Borel sets, hence
\(g_2^{-1}(G)\) is a Borel set and \(g_2\) is a Borel function.
By Theorem~1.12\ich{d} \(xg_2(x,y) = x\cdot f_a(x,y)\) are Borel
functions for all \(a\in\R\).
Hence the functions \(f_n\) (defined in the hint) are Borel functions.

Clearly for \(\lim_n f_n(x,y) = f(x,y)\) for all \((x,y)\in\R^2\).
If \(G\subset \R^2\) is any open set then
\begin{equation*}
f^{-1}(G) = \bigcup_{m\in\N} \left(\bigcap_{n\geq m} f_n^{-1}(G)\right)
\end{equation*}
Since each \(f_n^{-1}(G)\) is a Borel set, 
by being a \salgebra\ \(f^{-1}(G)\) is a Borel set. 
Therefore $f$ is Borel measurable.

\itemch{b}

Just for comparison with continuity, recall the following example:
\begin{equation*}
f(x,y) = \left\{
\begin{array}{ll}
0 & \textnormal{if}\; x= y = 0 \\
\frac{xy}{x^2+y^2}  \quad & \textnormal{otherwise}
\end{array}\right.
\end{equation*}
Now 
\begin{align*}
\lim_{x\to 0} f(x,0) &= 0 \\
\lim_{y\to 0} f(0,y) &= 0 \\
\lim_{x\to 0} f(x,ax) &= \frac{a}{1+a^2} 
   = \left(\frac{1}{a}\right)\,
     \bigm/\,
     \left(1+\left(\frac{1}{a}\right)^2\right)
     \qquad (a\neq 0)
\end{align*}

By induction. If \(k=1\) then trivially $g$ is continuous,
and every continuous function is also a Borel function.
Now assume that \(k=n\) and that the claim holds for all \(k<n\).
For each fixed \(x_n\) the function \(\tilde{g}:\R^{n-1}\to\R\) defined by 
\begin{equation*}
\tilde{g}(x_1,x_2,\ldots,x_{n-1}) = g(x_1,x_2,\ldots,x_{n-1},x_n)
\end{equation*}
is Borel.
Similar to previous item, 
for each \((x_1,x_2,\ldots,x_{n-1})\in\R^{n-1}\) and 
\begin{equation*}
x_n \in [a_{i-1},a_i] = [(i-1)/n, i/n]
\end{equation*}
we define
\begin{equation*}
f_n(x_1,\ldots,x_n) = 
\frac{a_i - x_n}{a_i - a_{i-1}} g(x_1,\ldots,x_{n-1},a_{i-1}) 
+
\frac{x_n - a_{i-1}}{a_i - a_{i-1}} f(x_1,\ldots,x_{n-1},a_i).
\end{equation*}
Clearly \(\lim_{n\to\infty} f_n = f\) pointwise, and by similar argumnets 
the limit of Borel function is Borel as well. Thus $f$ is a Borel function.
\end{itemize}


%%%%%%%%%%%%%% 9
\begin{excopy}
Suppose $E$ is a dense set in \(\R^1\) and $f$ is a real function on \(\R^2\)
such that 
\ich{a} \(f_x\) is Lebesgue measurable for each \(x\in E\) and 
\ich{b} \(f^y\) is continuous for almost all  \(y\in \R^1\).
Prove that $f$ is a Lebesgue measurable on \(\R^2\).
\end{excopy}

Enumerate $E$ as \(\{a_j\}_{j\in\N}\) such that for each $n$ 
there exists some $n$ such that 
for each \(x\in[-n,n]\) we can there exists \(j,k\in\N_m\)
such that \(a_j\leq x \leq a_k\) and \(a_k-a_j < 1/n\).
(Note that \(2n^2 \leq m < \infty)\).) Let
\(K_n = \left[\min(\{a_j: j\leq n\}), \max(\{a_j: j\leq n\})\right]\)

Let $D$ be the set of all \(y\in\R\) such that \(f^y\) is not continuous.
Clearly \(m(D)=0\) and also 
\(m_2\left(\{(x,y)\in\R^2: y\in D\}\right) = m_2(\R\times D) = 0\).

Now define 
\begin{equation*}
f_n(x,y) = \left\{
\begin{array}{ll}
0 & y \in D \;\vee\; x \notin K_n \\
\frac{a_i - x}{a_i - a_{i-1}} f(a_{i-1}, y) 
+
\frac{x - a_{i-1}}{a_i - a_{i-1}} f(a_i, y) & \textnormal{otherwise}.
\end{array}
\right.
\end{equation*}
Clearly \(f_n\) are Lebesgue measurable, and
\(\lim_{n\to\infty} f_n = f \aded\) Thus $f$ is Lebesgue measurable.


%%%%%%%%%%%%%% 10
\begin{excopy}
Suppose $f$ is a real function on \(\R^2\),
\(f_x\) is a Lebesgue measurable for each $x$,
and \(f^y\) is continuous for each $y$.
Suppose \(g: \R^1\to \R^1\) is Lebesgue measurable,
and put \mbox{\(h(y) = f(g(y),y)\)},
Prove that $h$ is Lebesgue measurable on \(\R^1\).

\emph{Hint}: Define \(f_n\) as in Exercise~8, put \(h_n(y) = f_n(g(y),y)\).
show that each \(h_n\) is measurable and that \(h_n(y) \to h(y)\).
\end{excopy}

Following the hint. For each \(y\in\R\) and \(n\in\N\)
find 
\begin{equation*}
a_{i-1} = (i-1)/n \leq g(y) \leq i/n = a_i
\end{equation*}
and define
\begin{equation*}
h_n(y) 
= f_n(g(y),y) 
= n\left(a_i-g(y)\right)\cdot f(a_{i-1},y) + n(g(y)-a_{i-1})\cdot f(a_i,y).
\end{equation*}
By Theorem~1.9\ich{c} \(h_n\) is Lebesgue measurable.
Fix \(y\in\R\) since \(a_{i}-a_{i-1} = 1/n\) and \(f^y\) is a continuous section
we have \(\lim_{n\to\infty}h_n(y) = h(y)\).
By Corollary~\ich{a} of Theorem~1.14 $h$ is Lebesgue measurable.


%%%%%%%%%%%%%% 11
\begin{excopy}
Let \(\calB_k\) be the \salgebra\ of all Borel sets in \(\R^k\).
Prove that \(\calB_{m+n} = \calB_m \times \calB_n\).
This is relevant in Theorem~8.14.
\end{excopy}

% It is trivial to see that \(\calB_{m+n} \supseteq \calB_m \times \calB_n\).
We start with a topological lemma.
\begin{llem} \label{lem:open-countable}
If \(G\in\R^n\) is open, then it is a union of a countable collection
of $n$-cubes.
\end{llem}
\begin{thmproof}
Let \(D=\Q^n\cap G\) a countable dense set in $G$.
For each \(x\in D\) let \(C_x\) be the maximal open cube whose center is $x$
and \(C_w\subset G\). Clearly \(\cup_{w\in D} C_w \subset G\).
Assume by negation \(y\in G \setminus \cup_{w\in D} C_w\).
Let \(\delta>0\) be such that 
\begin{equation*}
C_y(\delta) := \{x\in \R^n: \|x-y\|_1<\delta\} \subset G.
\end{equation*}
Pick some \(x\in C_y(\delta/2)\cap G\), then 
\(y\in C_x\), with the contradiction \(y\in \cup_{w\in D} C_w\).
\end{thmproof}

By the Lemma every open set \(G\in \R^{m+n}\) is a countable union
of open ``rectangles'' \(X\times Y\subset \R^m\times\R^n\) 
where \(X\in\R^m\) and \(Y\in\R^n\) are open sets.
Hence \(G\in \calB_m \times \calB_n\).
But \(\calB_{m+n}\) is a minimal \salgebra\ that contains all open sets
in \(\R^{m+n}\) and so \(\calB_{m+n} \subset \calB_m \times \calB_n\).

To show the opposite inclusion it suffices to show that if
an arbitrary elementary set
\(E = E_m\times E_n \in \calB_m \times \calB_n\)
then \(E\in \calB_{m+n}\).
But since \(E_m\in \calB_m\) then by minimality of the 
Borel \salgebra\ \(\calB_m\) we also have (its inverse projection)
\(E_m\times\R^n\in\calB_{m+n}\).
Similarly \(R^m\times E_n\in\calB_{m+n}\). Now
\begin{equation*}
E = E_m\times\R^n \;\cap\;R^m\times E_n \in \calB_{m+n}.
\end{equation*}


%%%%%%%%%%%%%% 12
\begin{excopy}
Use Fubini's
\index{Fubini}
Theorem~8.8 and the relation
\begin{equation*}
\frac{1}{x} = \int_0^\infty e^{-xt}\,dt \qquad (x>0)
\end{equation*}
to prove that 
\begin{equation*}
\lim_{A\to\infty} \int_0^A \frac{\sin x}{x}\,dx = \frac{\pi}{2}.
\end{equation*}
\end{excopy}

\begin{align*}
\lim_{A\to\infty}\int_0^A \frac{\sin x}{x}\,dx 
&= \int_0^\infty \sin x\left( \int_0^\infty e^{-xt}\,dt\right)\,dx 
 = \int_0^\infty \left(\int_0^\infty e^{-xt}\frac{e^{ix}-e^{-ix}}{2i}\,dt\right)dx \\
&= \frac{1}{2i}\int_0^\infty \left(\int_0^\infty e^{-xt}(e^{ix}-e^{-ix})\,dt\right)dx
   \\
&= \frac{1}{2i}\int_0^\infty 
        \left(\int_0^\infty e^{(-t+i)x}-e^{-(t+i)x}\,dx\right)dt 
   \qquad \textnormal{(Fubini)}
   \\
&= \frac{1}{2i}\int_0^\infty \left(
       \frac{e^{(-t+i)x}}{-t+i} - \frac{e^{-(t+i)x}}{-t-i}
    \right)\biggm|_{x=0}^\infty \,dt \\
&= \frac{1}{2i}\int_0^\infty 
        \left(0 - \frac{1}{-t + i}\right)
        - \left(0 - \frac{1}{-t - i}\right)\,dt \\
&= \frac{1}{2i}\int_0^\infty 
        \frac{1}{-t - i} - \frac{1}{-t + i}\,dt
 = \frac{1}{2i}\int_0^\infty \frac{2i}{t^2 + 1}\,dt
 = \int_0^\infty \frac{1}{t^2 + 1}\,dt \\
&= \left(\tan^{-1}(t)\right)\bigm|_{t=0}^\infty
 = \tan^{-1}(\infty) - \tan^{-1}(0) = \pi/2
\end{align*}


%%%%%%%%%%%%%% 13
\begin{excopy}
If \(\mu\) is a complex measure on a \salgebra\ \frakM, show that every 
set \(E\in\frakM\) has a subset $A$ for which 
\begin{equation*}
|\mu(A)| \geq \frac{1}{\pi} |\mu|(E).
\end{equation*}
\emph{Suggestion}: There is a measurable real function \(\theta\) so that
\(d\mu = e^{i\theta}\,d|\mu|\).
Let \(A_\alpha\) be the subset of $E$ where \(\cos(\theta - \alpha)>0\), 
show that
\begin{equation*}
\Re[e^{-i\alpha}\mu(A_\alpha)] = \int_E \cos^+(\theta - \alpha)\,d|\mu|.
\end{equation*}
and integrate with respect to \(\alpha\) (as in Lemma~6.3).

Show by an example, that \(1/\pi\) is the best constant in this inequality.
\end{excopy}

Compute
\begin{align*}
e^{-i\alpha} \mu(A_\alpha)
&= e^{-i\alpha}\int_{A_\alpha} 1\,d\mu
 = \int_{A_\alpha} e^{i(\theta - \alpha)}\,d|\mu|
 = \int_{A_\alpha} \cos(\theta-\alpha)\,d|\mu| 
   + i\int_{A_\alpha} \sin(\theta - \alpha)\,d|\mu|.
\end{align*}
Hence
\begin{equation*}
\Re[e^{-i\alpha}\mu(A_\alpha)]
= \int_{A_\alpha} \cos(\theta-\alpha)\,d|\mu| 
= \int_E \cos^+(\theta-\alpha)\,d|\mu|.
\end{equation*}

Integrate by \(\alpha\)
\begin{align*}
\int_0^{2\pi} \Re[e^{-i\alpha}\mu(A_\alpha)]\,d\alpha
&= \int_0^{2\pi} 
   \left(\int_E \cos^+(\theta(x)-\alpha)\,d|\mu|(x)\right)\,d\alpha  \\
&= \int_E \left(
      \int_0^{2\pi} \cos^+(\theta(x)-\alpha)\,d\alpha
          \right)\,d|\mu|(x)  \qquad\textnormal{(Fubini)} \\
&= \int_E \left(
      \int_0^{2\pi} \cos^+(-\alpha)\,d\alpha \right)\,d|\mu|(x) 
 = \int_E 2\,d|\mu|(x) 
 = 2|\mu|(E).
\end{align*}

Assume by negation
\(|\mu(A_\alpha)| <  |\mu|(E)/\pi\) 
for all \(\alpha\in[0,2\pi]\). Then
\begin{align*}
\left|\int_0^{2\pi} \Re[e^{-i\alpha}\mu(A_\alpha)]\,d\alpha\right|
&\leq \int_0^{2\pi} \left|\Re[e^{-i\alpha}\mu(A_\alpha)]\right|\,d\alpha
 \leq \int_0^{2\pi} |\mu(A_\alpha)|\,d\alpha \\
&< \int_0^{2\pi} |\mu|(E)/\pi\,d\alpha
  = 2|\mu|(E)
\end{align*}
that contradicts the previous established equality.
Hence there exists some \(\alpha\)
such that \(|\mu(A_\alpha)| \geq |\mu|(E)/\pi\).

\paragraph{Best Constant.}
Consider the unit circle \(\gamma:[0,2\pi]\to\C\) 
defined by \(\gamma(t) = e^{it}\) with the complex measure
\(d\mu = \lambda'(t)\,dm(t)\).
Clearly \(|\mu|(\gamma^*) = 2\pi\).

For any \(E\subset \gamma^*\) and any \(u\in\C\) such that \(|u|=1\)
we put \(E_u = \{uz: z\in E\}\).
It is easy to see that \(E_u\subset \gamma^*\) and that 
\(|\mu(E)| = |\mu(E_u)|\).

Take 
\begin{equation*}
A = \{\gamma(t): t\in[0,2\pi] \wedge\;\Re(\gamma'(t)) \geq 0\}
  = \{z\in\gamma^{*}: \Im(z) \leq 0\}
\end{equation*}
Now
\begin{equation*}
\mu(A) = \int_{\pi}^{2\pi} = 1\cdot(\gamma^{it})'\,dt
 = \gamma(2\pi) - \gamma(\pi) = 2 = |\mu|(\gamma^*)/\pi.
\end{equation*}

We will show that $A$ consist of the positive part of \(\mu\)
and \(\gamma^*\setminus A\) of the negative.
Let \(P \subset A\) and \(N \subset \gamma^*\setminus A\), then
using the definition of $A$ we have
\begin{equation*}
\Re(\mu(P)) 
= \int_P \Re\left((\gamma^{it})'\right)\,dt
= \int_P \left|\Re\left((\gamma^{it})'\right)\right|\,dt
\geq 0.
\end{equation*}
and
\begin{equation*}
\Re(\mu(N)) 
= \int_N \Re\left((\gamma^{it})'\right)\,dt
= \int_N -\left|\Re\left((\gamma^{it})'\right)\right|\,dt
\leq 0.
\end{equation*}

Assume by negation there exists some \(E\subset\gamma^*\) such that 
\(|\mu(E)| > 2\). \Wlogy\ (or by replacing $E$ by some \(E_u\)
we may assume that \(\mu(E) > 0\) is real.
Now we get the following contradiction
\begin{align*}
\mu(E) 
&= \Re(\mu(E\cap A)) + \Re(\mu(E\setminus A))
 \leq \Re(\mu(E\cap A))
 = \Re(\mu(E\cap A)) \\
&\leq \Re(\mu(E\cap A)) + \Re(\mu(A\cap E))
 = \Re(\mu(A)) 
 = \mu(A)
 = 2.
\end{align*}

%%%%%%%%%%%%%% 14
\begin{excopy}
Complete the following proof of Hardy's
\index{Hardy} 
inequality
(Chap.~3 Exercise~14).
Suppose \(f\geq 0\) on \((0,\infty)\), \(f\in L^p\), \(1<p<\infty\), and
\begin{equation*}
F(x) = \frac{1}{x}\int_0^x f(t)\,dt.
\end{equation*}
Write \(xF(x) = \int_0^x f(t)t^\alpha t^{-\alpha}\,dt\),
where \(0<\alpha<1/q\), use H\"older's inequality 
to get an upper bound for
\(F(x)^p\), and integrate to obtain
\begin{equation*}
\int_0^\infty F^p(x)\,dx 
\leq 
(1-\alpha q)^{1-p} (\alpha p)^{-1} \int_0^\infty f^p(t)\,dt.
\end{equation*}
Show that the best choice of \(\alpha\) yields
\begin{equation*}
\int_0^\infty F^p(x)\,dx 
\leq 
\left(\frac{p}{p-1}\right)^{p} \int_0^\infty f^p(t)\,dt.
\end{equation*}
\end{excopy}

In \cite{Garling2007} Section~7.3,
given \(\mu\)-measurable \(f:\Omega\to\C\)
\index{maximal function!Muirhead}
\index{Muirhead!maximal function}
Muirhead's maximal function is introduced, 
\begin{equation*}
f^\dagger(t) = \sup\left\{\frac{1}{t}\int_E|f|\,d\mu: \mu(E)\leq t\right\}
\qquad \textnormal{where}\; 0 < t < \mu(\Omega).
\end{equation*}

The treatment there (\cite{Garling2007}) uses
the following notation
\begin{equation*}
\lambda_f(t) = \mu(f > t) \qquad (t\geq 0)
\end{equation*}
for the distribution function.


See \cite{Garling2007} Theorem~8.1.1 and its Corollary.

Using H\"older's inequality:
\begin{equation*}
\int_0^x f(t)\,dt
= \int_0^x f(t)t^\alpha t^{-\alpha}\,dt
\leq \left(\int_0^x \left(f(t)t^\alpha\right)^p\,dt\right)^{1/p}
      \cdot \left(\int_0^x t^{-\alpha q}\,dt\right)^{1/q}
\end{equation*}
Simplifying the last term
\begin{equation*}
\left(\int_0^x t^{-\alpha q}\,dt\right)^{1/q}
= \left(\frac{x^{1-\alpha q}}{1-\alpha q}\right)^{1/q}.
\end{equation*}
Combining with the inequality gives
\begin{align*}
F(x)^p
&\leq \left(x^{-1} \left(\frac{x^{1-\alpha q}}{1-\alpha q}\right)^{1/q}\right)^p
      \left(\int_0^x \left(f(t)t^\alpha\right)^p\,dt\right)
 = (1-\alpha q)^{-p/q} x^{-\alpha p - 1}
   \left(\int_0^x \left(f(t)t^\alpha\right)^p\,dt\right) \\
&= (1-\alpha q)^{1-p} x^{-\alpha p - 1}
   \left(\int_0^x \left(f(t)t^\alpha\right)^p\,dt\right).
\end{align*}
The power of $x$ was simplified
by \((-1 + (1-\alpha q)\frac{1}{q})p = -p + p/q - \alpha p = -\alpha p - 1\).

We now integrate using 
\index{Fubini}
Fubini's Theorem~8.8 on \(\{(x,t)\in\R^2: 0\leq t \leq x\}\)
\begin{align*}
\int_0^\infty F^p(x)\,dx 
&\leq \int_0^\infty
  \left((1-\alpha q)^{1-p} x^{-\alpha p - 1}
   \left(\int_0^x \left(f(t)t^\alpha\right)^p\,dt\right)\right)\,dx
   \\
&= (1-\alpha q)^{1-p}
   \int_0^\infty
   \left( x^{-\alpha p - 1}
   \left(\int_0^x \left(f(t)t^\alpha\right)^p\,dt\right)\right)\,dx
   \\
&= (1-\alpha q)^{1-p}
   \int_0^\infty 
     \left(\int_t^\infty x^{-\alpha p - 1}
                 \left(f(t)t^\alpha\right)^p\,dx\right)\,dt
   \\
&= (1-\alpha q)^{1-p}
   \int_0^\infty 
     \left(\left(f(t)t^\alpha\right)^p
       \int_t^\infty x^{-\alpha p - 1}
                 \,dx\right)\,dt.
\end{align*}
The inner integral is simplified as follows:
\begin{equation*}
\int_t^\infty x^{-\alpha p - 1}
= -\frac{1}{\alpha p}\left(x^{-\alpha p}\right)\biggm|_{x=t}^\infty
= -(\alpha p)^{-1}\left(0 - t^{-\alpha p}\right)/p = (\alpha p)^{-1} t^{-\alpha p}.
\end{equation*}
Back to previous integration
\begin{align*}
\int_0^\infty F^p(x)\,dx 
 &\leq (1-\alpha q)^{1-p}
   \int_0^\infty \left(f(t)t^\alpha\right)^p (\alpha p)^{-1} t^{-\alpha p}/p\,dt
 \\
 &= (1-\alpha q)^{1-p} (\alpha p)^{-1} \int_0^\infty \bigl(f(t)\bigr)^p\,dt
\end{align*}

\paragraph{Best Constant.}
We want to minimize
\begin{equation*}
b(\alpha) = (1-\alpha q)^{1-p} (\alpha p)^{-1}
\end{equation*}
where \(0 < \alpha < 1/q\)
This is equivalent to maximize  
\begin{equation*}
c(\alpha) = (1-\alpha q)^{p-1} \alpha p.
\end{equation*}
where \(0 < \alpha q < 1\).
Clearly in this domain \(c(\alpha) > 0\). We differentiate and equate to zero
\begin{gather*}
c'(\alpha) =  (1 - \alpha q)^{p-1} - \alpha q (p-1)(1 - \alpha q)^{p-2} = 0 \\
1 - \alpha q = \alpha q(p-1) \\
\alpha = \frac{1}{pq}\,.
\end{gather*}
Hence the best constant is
\begin{align*}
b(\alpha) 
&= (1-\alpha q)^{1-p} (\alpha p)^{-1}
= \left(1-\frac{1}{p}\right)^{1-p} \left(\frac{1}{q}\right)^{-1}
= \left(\frac{1}{q}\right)^{1-p} \left(\frac{1}{q}\right)^{-1}
= \left(\frac{1}{q}\right)^{-p}
\\
&= \left(1-\frac{1}{p}\right)^{-p}
= \left(\frac{p-1}{p}\right)^{-p}
= \left(\frac{p}{p-1}\right)^{p}\,.
\end{align*}

%%%%%%%%%%%%%% 15
\begin{excopy}
Put \(\varphi(t) = 1 - \cos t\) if \(0\leq t \leq 2\pi\), 
\(\varphi(t) = 0\) for all other real $t$.
For \(-\infty < x < \infty\), define
\begin{equation*}
f(x) = 1, 
\qquad g(x) = \varphi'(x),
\qquad h(x) = \int_{-\infty}^x \varphi(t)\,dt.
\end{equation*}
Verify the following statements about convolutions of these functions:
\begin{itemize}
\item[(i)] \((f\ast g)(x) = 0\) for all $x$.
\item[(ii)] \((g\ast h)(x) = (\varphi \ast \varphi)(x) > 0\) on \((0,4\pi)\).
\item[(iii)] Therefore \((f\ast g)\ast h = 0\), 
             whereas \(f\ast (g\ast h)\) is a positive constant.
\end{itemize}
But convolution is supposedly associative, by Fubini's Theorem~8.8
\index{Fubini}
(Exercise~5(c)). What went wrong?
\end{excopy}

First compute the functions
\begin{equation*}
g(x) = \left\{\begin{array}{ll}
0            & x < 0 \\
\sin x \quad & 0 < x < 2\pi \\
0            & x > 2\pi
\end{array}
\right.
\qquad
h(x) = \left\{\begin{array}{ll}
0                & x \leq 0 \\
x - \sin x \quad & 0 \leq x \leq 2\pi \\
2\pi             & x \geq 2\pi
\end{array}
\right.
\end{equation*}

\begin{itemize}
\item[(i)]
Compute:
\begin{equation*}
(f\ast g)(x) 
= \int_\R f(x-t)g(t)\,dt
= \int_\R g(t)\,dt
= \int_0^{2\pi} \sin(t)\,dt 
= 0
\end{equation*}

\item[(ii)]
We note that \(0 < x-t < 2\pi\) iff  \(x-2\pi < t < x\).
Compute:
\begin{align*}
(g\ast h)(x)
&= \int_\R g(x-t)h(t)\,dt
 = \int_0^{2\pi} g(x-t)\cdot (t - \sin t)\,dt
% = \int_\R g(t)h(x-t)\,dt
% = \int_0^{2\pi} \sin(t)\cdot h(x-t)\,dt
 \\
&= \left\{\begin{array}{ll}
   \int_{\max(x-2\pi,0)}^{\min(x,2\pi)} \sin (x-t)\cdot (t - \sin t)\,dt 
        \quad & x \in [0,4\pi] \\
   0  & x \notin [0,4\pi]
   \end{array}\right.
\end{align*}
We have 
\begin{itemize}
\item[\(\circ\)] \(\max(x-2\pi,0) < \min(x,2\pi)\) when \(0<x<4\pi\),
\item[\(\circ\)] \(t-\sin t > 0\) for all \(t>0\),
\item[\(\circ\)] \(\sin(x-t)>0\) when \(0 < x-t < 2\pi\).
\end{itemize}
Hence \((g\ast h)(x)>0\) when \(0<x<4\pi\).

\item[(iii)]
Clearly by linearity of convolution.

\end{itemize}
The associativity holds in Exercise~5\ich{c} for complex measures, 
meaning finite measures on~\R. Viewing these as functions, 
the associativity holds for functions in \(L^1(\R)\) but clearly
here \(f\notin L^1(\R)\).

%%%%%%%%%%%%%% 16
\begin{excopy}
Prove the following analogue of Minkowski's inequality
\index{Minkowski}
, for \(f\geq 0\):
\begin{equation*}
\left\{\int \left[\int f(x,y)\,d\lambda(y)\right]^p\,d\mu(x)\right\}^{1/p}
\leq
\int \left[\int f^p(x,y)\,d\mu(x) \right]^{1/p}\,d\lambda(y).
\end{equation*}
Supply the required hypothesis.
(Many further developments of this theme may be found in [9].)
\end{excopy}

See also Theorem~2.4 in \cite{LiebLoss200104}.

Let \((X,\mu)\) and \(Y,\lambda\) be \(\sigma\)-finite measurable spaces
and \(f:X\times Y\to  \R^\oplus\) measurable function.

\emph{Note:} If \(Y = \{0,1\}\) and \(\lambda\) is the counting measure,
then we get the known Minkowski's inequality
of Theorem~3.5(2). We follow similar idea as in the proof there.

We may assume that $f$ and \(\supp f\) are bounded
and generalize the result by applying
Lebesgue's monotone convergence Theorem~1.34.

Denote the two sides
and the inner integral of the left side of the desired inequality as
\begin{align*}
L &= \left(\int_X 
        \left(\int_Y f(x,y)\,d\lambda(y)\right)^p d\mu(x)\right)^{1/p} \\
R &= \int_Y\left(\int_X \left(f(x,y)\right)^pd\mu(x) \right)^{1/p} d\lambda(y) \\
\psi(x) &= \int_Y f(x,y)\,d\lambda(y).
\end{align*}
Fix \(v\in Y\), the 
\index{H\"older}
H\"older inequality of of Theorem~3.5(1) gives
\begin{equation*}
\int_X f(x,v)\left(\psi(x)\right)^{p-1} d\mu(x)
\leq \left(\int_X \left(f(x,v)\right)^p\,d\mu(x)\right)^{1/p} 
     \left(\int_X \left(\psi(x)\right)^{(p-1)q} d\mu(x)\right)^{1/q}
\end{equation*}
Using Fubini's Theorem~8.8 and \(p = (p-1)q\), we integrate both sides
\begin{align}
A
&= \int_Y 
    \left(\int_X f(x,v)\left(\psi(x)\right)^{p-1} d\mu(x)\right)\,d\lambda(v)
 = \int_X 
    \left(\int_Y f(x,v)\left(\psi(x)\right)^{p-1} d\lambda(v)\right)\,d\mu(x)
   \notag \\
&= \int_X 
    \left(\psi(x)\right)^{p-1} \left(\int_Y f(x,v)\,d\lambda(v)\right)\,d\mu(x)
 = \int_X \left(\psi(x)\right)^p d\mu(x) = L^p
   \notag \\
&\leq  \label{eq:minkowski:gen1}
    \left(
      \int_Y
        \left(\int_X \left(f(x,v)\right)^p\,d\mu(x)\right)^{1/p} 
      d\lambda(v)
    \right)
    \left(\int_X \left(\psi(x)\right)^p d\mu(x)\right)^{1/q} 
  \\
&= \left(\int_X \left(\psi(x)\right)^p d\mu(x)\right)^{1/q} \cdot R.
  \notag
\end{align}

We now have \(L = A^{1/p}\) and establish an inequality
\(A \leq A^{1/q} R\).
Three cases regarding $A$ are to be considered.

\paragraph{Case 1 (Zero).} 
If \(A=0\) then \(L = A^{1/p}=0\) and the desired inequality is trivial.  

\paragraph{Case 2 (Infinity).}
The case of \(A=\infty\) is impossible because of
our assumption that \(\|f\|_\infty<\infty\) 
and \((\mu\times\lambda)(\supp f) < \infty\).

\paragraph{Case 3 (Normal).}
If \(0 < A < \infty\) then we can safely divide the inequality
we got by \(A^{1/q}\) and since \(1-1/q=1/p\) we get the desired inequality,
\(A^{1/p} \leq R\) or explicitly
\begin{equation*}
 \left(\int_X 
    \left(\int_Y f(x,y)\,d\lambda(y)\right)^p d\mu(x)\right)^{1/p} 
\leq
 \int_Y\left(\int_X \left(f(x,y)\right)^pd\mu(x) \right)^{1/p} d\lambda(y).
\end{equation*}


%%%%%%%%%%%%%%%%%
\end{enumerate}

 % \setcounter{chapter}{8}  %%%%%%%%%%%%%%%%%%%%%%%%%%%%%%%%%%%%%%%%%%%%%%%%%%%%%%%%%%%%%%%%%%%%%%%%
%%%%%%%%%%%%%%%%%%%%%%%%%%%%%%%%%%%%%%%%%%%%%%%%%%%%%%%%%%%%%%%%%%%%%%%%
%%%%%%%%%%%%%%%%%%%%%%%%%%%%%%%%%%%%%%%%%%%%%%%%%%%%%%%%%%%%%%%%%%%%%%%%
\chapterTypeout{Fourier Transforms}

%%%%%%%%%%%%%%%%%%%%%%%%%%%%%%%%%%%%%%%%%%%%%%%%%%%%%%%%%%%%%%%%%%%%%%%%
%%%%%%%%%%%%%%%%%%%%%%%%%%%%%%%%%%%%%%%%%%%%%%%%%%%%%%%%%%%%%%%%%%%%%%%%
\section{Comments and Clarifications}

%%%%%%%%%%%%%%%%%%%%%%%%%%%%%%%%%%%%%%%%%%%%%%%%%%%%%%%%%%%%%%%%%%%%%%%%
\subsection{Convolution} 

\begin{llem} \label{lem:conv:commut}
Convolution is commutative. If \(f,g\in L^1\) then
\(f\ast g = g\ast f\).
\end{llem}
\begin{thmproof}
\begin{align*}
(f\ast g)(x)
&= \int_\R f(x-y)g(y)\,dm(y)
 = \int_\R f(t)g(x-t)(dx/dt)\,dm(t)
 = \int_\R g(x-t)f(t)\,dm(t) \\
&= (g \ast f)(x)
\end{align*}
\end{thmproof}


%%%%%%%%%%%%%%%%%%%%%%%%%%%%%%%%%%%%%%%%%%%%%%%%%%%%%%%%%%%%%%%%%%%%%%%%
\subsection{Transform Formulas} \label{subsec:xform:formulas}


\newcommand{\sqdivtpi}{\ensuremath{\frac{1}{\sqrt{2\pi}}}}
In Theorem~9.2.
Suppose \(f\in \Lp1\), and \(\alpha,\lambda \in\R\).
\begin{itemize}

 \itemch{a} If \(g(x)=f(x)e^{i\alpha x}\) then
    \begin{eqnarray*}
      \Hat{g}(t)
            & = & \intR g(x)e^{-ixt}\,dm(x) \\
            & = & \intR f(x)e^{i\alpha x}e^{-ixt}\,dm(x) \\
            & = & \intR f(x)e^{-i(t-\alpha)x}\,dm(x) \\
            & = & \Hat{f}(t-\alpha).
    \end{eqnarray*}

  \itemch{b} If \(g(x)=f(x-\alpha)\) then
    \begin{eqnarray*}
      \Hat{g}(t)
            & = & \intR g(x)e^{-ixt}\,dm(x) \\
            & = & \intR f(x-\alpha)e^{-i((x-\alpha)+\alpha)t}\,dm(x) \\
            & = & e^{-i\alpha t}\intR f(x-\alpha)e^{-i(x-\alpha)t}\,dm(x) \\
            & = & \Hat{f}(t-\alpha)e^{-i\alpha t}.
    \end{eqnarray*}

  \itemch{d} If \(g(x)=\overline{f(-x)}\) then
    \begin{eqnarray*}
      \Hat{g}(t)
            & = & \intR g(x)e^{-ixt}\,dm(x) \\
            & = & \intR \overline{f(-x)}e^{-ixt}\,dm(x) \\
            & = & \intR \overline{f(-x)e^{-i(-x)t}}\,dm(x) \\
            & = & \overline{\intR f(-x)e^{-i(-x)t}\,dm(x)} \\
            & = & \overline{\Hat{f}(t)}.
    \end{eqnarray*}

  \itemch{e}
    Given \(g(x) = f(x/\lambda)\) and \(\lambda>0\)
    we use the substitution \(y=x/\lambda\) and \(dx/dy=\lambda\)
    \begin{align*}
    \widehat{g}(x)
     &= \int_{-\infty}^\infty g(x)e^{-itx}\,dm(x)
     = \int_{-\infty}^\infty f(x/\lambda)e^{-itx}\,dm(x) 
     = \int_{-\infty}^\infty f(y)e^{- it \lambda y}\lambda\,dm(yx) \\
     &= \lambda \widehat{f}(\lambda t)
    \end{align*}


\end{itemize}


%%%%%%%%%%%%%%%%%%%%%%%%%%%%%%%%%%%%%%%%%%%%%%%%%%%%%%%%%%%%%%%%%%%%%%%%
\subsection{Auxiliary Functions}

In Section~9.7, the following functions are defined:
\begin{gather}
H(t) = e^{-|t|} \\
h_\lambda(x) = \int_{-\infty}^\infty H(\lambda t)e^{itx}\,dm(t) 
 \qquad (\lambda > 0)
\end{gather}
Then text says:
\begin{quote}
A simple computation gives
\begin{equation}
 \tag{Rudin(3)}
h_\lambda(x) = \sqrt{\frac{2}{\pi}} \frac{\lambda}{\lambda^2 + x^2}
\end{equation}
\end{quote}
Let's compute it in details, in two ways.
\begin{itemize}

\item

Assuming \(\lambda>0\):
\begin{eqnarray*}
h_\lambda(x)
  & = & \intR H(\lambda t)e^{itx}dm(t) =
  % & = &
        \intR e^{-|\lambda t|}e^{itx}dm(t) \\
  & = & \int_{-\infty}^0 e^{(ix+\lambda)t}dm(t) +
        \int_0^\infty e^{(ix-\lambda)t}dm(t) \\
  & = & \left.\left(\sqdivtpi\frac{1}{ix+\lambda}e^{(ix+\lambda)t}
        \right)\right|_{-\infty}^0 +
        \left.\left(\sqdivtpi\frac{1}{ix-\lambda}e^{(ix-\lambda)t}
        \right)\right|_0^\infty \\
  & = & \sqdivtpi\frac{1}{ix+\lambda} - 0 +
        0 - \sqdivtpi\frac{1}{ix-\lambda} \\
  & = & \sqdivtpi\,\frac{-2\lambda}{(ix+\lambda)(ix-\lambda)} \\
  & = & \sqrt{\frac{2}{\pi}}\frac{\lambda}{\lambda^2+x^2}\>.
\end{eqnarray*}

\item
We use the indefinite integral formula 
(See \cite{Apostol1961}~6.17, Exercise~20)
\begin{equation*}
\int e^{ax}\cos(bx)\,dx = \frac{e^{ax}(a\cos(bx) + b\sin(bx))}{a^2+b^2} + C.
\end{equation*}
\iffalse
We use the indefinite integral formula
\begin{equation*}
\int 1/(a+x^2)\,dx = \arctan\left(x/\sqrt{a}\,\right)/\sqrt{a}
\end{equation*}
\fi
\iffalse
which can be verified by
\begin{equation*}
1=\frac{d}{dx}\tan\bigl(\arctan(\x)\bigr)
= \frac{d}{dx}\arctan(x) \frac{d}{dx}\tan\bigl(\arctan(\x)\bigr)
\end{equation*}
\fi

Now
\begin{align*}
h_\lambda(x) 
&= \int_{-\infty}^\infty H(\lambda t)e^{itx}\,dm(t)
  =  \int_{-\infty}^0 e^{-\lambda t} e^{itx}\,dm(t) 
   + \int_0^\infty e^{-\lambda t} e^{itx}\,dm(t)  \\
&= \int_0^\infty e^{-\lambda t} (e^{itx} + e^{-itx})\,dm(t)
 = \int_0^\infty e^{-\lambda t} 2\cos(tx)\,dm(t)  \\
&= \frac{2}{\sqrt{2\pi}}
   \left.\frac{e^{-\lambda t}(-\lambda \cos(tx) + x\sin(tx))}{\lambda^2+x^2} 
   \,\right|_{t=0}^{t=\infty} 
 = \sqrt{\frac{2}{\pi}}\cdot\frac{\lambda}{\lambda^2+x^2}
\end{align*}
\end{itemize}

Following in the text is the equality
\begin{equation} \tag{Rudin(4)}
\intR h_\lambda(x)dm(x) = 1
\end{equation}
that we will now verify.

From the derivative equation \(\arctan'(x) = 1/(1+x^2)\) we have
the indefinite integral:
\[\int \frac{1}{\lambda^2+x^2}dx = \arctan'(x/\lambda)/\lambda.\]
To get Equation~(3) there:
\begin{eqnarray*}
\intR h_\lambda(x)dm(x)
 & = & \sqrt{\frac{2}{\pi}}\sqdivtpi \intR \frac{\lambda}{\lambda^2+x^2}dx \\
 & = & (1/\pi)
       \left.\bigl(\arctan(x/\lambda)\bigr)\right|_{-\infty}^{\infty} \\
 & = & \frac{1}{\pi}\left(\frac{\pi}{2} - \frac{-\pi}{2}\right) = 1.
\end{eqnarray*}

%%%%%%%%%%%%%%%%%%%%%%%%%%%%%%%%%%%%%%%%%%%%%%%%%%%%%%%%%%%%%%%%%%%%%%%%
\subsection{Equality in Theorem 9.9}

The proof of Theorem~9.9 uses the equality
\begin{equation*}
h_\lambda(y) = \lambda^{-1}h_1\left(\frac{y}{\lambda}\right)\,.
\end{equation*}
Working it out:
\begin{equation*}
h_1(y/\lambda) 
= \sqrt{\frac{2}{\pi}}\frac{1}{1^2+(y/\lambda)^2}
= \sqrt{\frac{2}{\pi}}\frac{\lambda^2}{\lambda^2+y^2}
= \lambda h_\lambda(y).
\end{equation*}


%%%%%%%%%%%%%%%%%%%%%%%%%%%%%%%%%%%%%%%%%%%%%%%%%%%%%%%%%%%%%%%%%%%%%%%%
%%%%%%%%%%%%%%%%%%%%%%%%%%%%%%%%%%%%%%%%%%%%%%%%%%%%%%%%%%%%%%%%%%%%%%%%
\section{Additional Results}

%%%%%%%%%%%%%%%%%%%%%%%%%%%%%%%%%%%%%%%%%%%%%%%%%%%%%%%%%%%%%%%%%%%%%%%%
\subsection{Generalization of Theorem~9.2} \label{sec:gen:thm9.2}

Theorem~9.2 has several items, each of which gives some equality
under some conditions. Here we generalize item~\ich{f}.

\begin{llem} \label{lem:g-eq-xnf}
Assume \(f\in L^1\).
If \(g(x) = x^nf(x)\) and \(g\in L^1\) then
\(\hat{f}\) is $n$-times differentiable and
\begin{equation}
\hat{g}(t) = i^nD^n\hat{f}(t). \label{eq:gtDn}
\end{equation}
\end{llem}
\begin{thmproof}
The case \(n=1\) was proved in Theorem~9.2\ich{f}.
Assume that \eqref{eq:gtDn} holds for \(1\leq n < k\).
Put \(h(x) = x^kf(x) = xg(x)\). By induction
\begin{equation*}
\hat{h}(t) 
= i\hat{g}'(t) 
= iD^1\left(i^{k-1}D^{k-1}\hat{f}\right)(t)
= i^kD^k\hat{f}(t).
\end{equation*}
\end{thmproof}


%%%%%%%%%%%%%%%%%%%%%%%%%%%%%%%%%%%%%%%%%%%%%%%%%%%%%%%%%%%%%%%%%%%%%%%%
\subsection{Continuity with Domain Transformations}

Theorem~9.5 deals with continuity of changing a function
by shifting its \(\R^1\) domain.
Let's generalize it to other kind of transformations.

\begin{llem} \label{lem:9.5:gen}
Let \((X,d,\mu)\) be a metric and positive measurable space such
that \(\mu(A)<\infty\) whenever \(A\subset X\) is compact
and let \(f\in L^p(X)\).
For each \(\epsilon>0\) there exists \(\delta > 0\) such that if 
\(T:\,X\to X\) is continuous and \(d(T(x),x)<\delta\) for all \(x\in X\)
then \(\|f - f\circ T\|_p < \epsilon\).
\end{llem}
\begin{thmproof}
By theorem~3.14 there exists \(g\in C_c(X)\) such that 
\(\|f-g\|_p < \epsilon/3\). Put \(K=\supp g\).
Let \(\delta>0\)
be such that \(|g(x_1)-g(x_2)|<\epsilon/(3\mu(K))\) 
whenever \(|x_1-x_2|<\delta\).
\end{thmproof}


%%%%%%%%%%%%%%%%%%%%%%%%%%%%%%%%%%%%%%%%%%%%%%%%%%%%%%%%%%%%%%%%%%%%%%%%
\subsection{Inversion --- 4 Steps to Identity}

The Inversion~Theorem~9.11 actually goes only half way.
The following lemma completes it.

\begin{llem}
Denote \(\calF(f) = \hat{f}\).
If \(f\in L^2\) then
\begin{equation} \label{eq:F4}
\calF^4(f) = \calF(\calF(\calF(\calF(f)))) = f \;\aded
\end{equation}
\end{llem}
Notes
\begin{itemize}
\item This can be summarized by \(\calF^4 = \Id\).
\item By Theorem~9.14, if \(\hat{f}\in L^1\) then
\begin{equation*}
f(x) = \int_{\infty}^\infty \hat{f}(t) e^{ixt}\,dm(t)\;\aded
\end{equation*}
\end{itemize}
\begin{thmproof}
By Plancherel Theorem~9.13\ich{d}
\begin{equation*}
f(x) = \lim_{A\to\infty}\int_{-A}^A \hat{f}(t) e^{ixt}\,dm(t) \;\aded
\end{equation*}
Therefore, with
accepting the limit as the defined of the Fourier transform in \(L^2\) 
\begin{equation*}
f(-x) = \lim_{A\to\infty}\int_{-A}^A \hat{f}(t) e^{-ixt}\,dm(t) = \calF^2(f)(x)
\;\aded
\end{equation*}
Applying this twice gives the desired \eqref{eq:F4}.
\end{thmproof}

Intuitively $f$ vanishes at infinitely if \(f\in L^1\),
but it is not difficult to construct a counterexample.
The intuition is true with additional requirement.

\begin{llem} \label{lem:fdf-L1-then-f-inC0}
If \(f\in L^1(\R)\)  and differentable and \(f'\in L^1\) then
\(f\in C_0\).
\end{llem}
\begin{thmproof}
Let 
\(l = \varliminf_{x\to\infty}|f(x)|\)
and
\(h = \varlimsup_{x\to\infty}|f(x)|\).
We need to show that \(l=h=0\).
Similarly we can do the same with 
opposite direction \(\lim_{x\to-\infty}\) limits 
and consequently \(f\in C_0\).

First we will show that \(l=0\). If by negation \(l>0\), 
then \(|f(x)|\geq l\) for all \(x\geq M\) for some \(M<\infty\)
and thus 
\begin{equation*}
\int_{-\infty}^\infty |f(x)|\,dm(x)
\geq \int_M^\infty|f(x)|\,dm(x) 
\geq \int_M^\infty l\,dm(x) = \infty.
\end{equation*}
A contradiction to \(f\in L^1\). 
Hence \(\varliminf_{x\to\pm\infty}|f(x)| = 0\).

Now assume by negation \(h>0\).
Since \(f'\in L^1\) there exist some \(M<\infty\) such that
\begin{equation} \label{eq:intfx-lt4}
\int_M^\infty |f(x)|\,dm(x) < h/4.
\end{equation}
But we can find some \(x_l,x_h>M\) such that
\begin{align*}
|f'(x_l)| &< h/3 \\
|f'(x_h)| &> 2h/3
\end{align*}
and \wlogy\ assume that \(x_l < x_h\). Now
\begin{equation*}
\int_M^\infty |f(x)|\,dm(x) 
\geq \int_{x_l}^{x_h} |f'(x)|\,dm(x) 
\geq \left|\int_{x_l}^{x_h} f'(x)\,dm(x)\right|
= f(x_h) - f(x_l)
> h/3
\end{equation*}
which contradicts \eqref{eq:intfx-lt4}.
\end{thmproof}

%%%%%%%%%%%%%%%%%%%%%%%%%%%%%%%%%%%%%%%%%%%%%%%%%%%%%%%%%%%%%%%%%%%%%%%%
\subsection{Back from Derivative to Multiplication}

The following lemma is an inverse variant of Theorem~9.2\ich{f},
that we have generalized before (\ref{sec:gen:thm9.2}).
\begin{llem} \label{lem:fourierdif}
Let $f$ be a differentable function on \(\R\).
Put \(g(x)=f'(x)\).
If \(f,g\in L^1\) then
\begin{equation} \label{eq:fourierdif}
\hat{g}(t) = it\hat{f}(t) \qquad (t\in\R).
\end{equation}
\end{llem}
\begin{thmproof}
By local lemma~\ref{lem:fdf-L1-then-f-inC0} \(f\in C_0\).
Integrating by parts
\begin{equation*}
\hat{g}(t) 
= \int_{-\infty}^\infty f'(x)e^{-ixt}\,dm(x)
= \frac{1}{\sqrt{2\pi}}f(x)e^{-ixt}\bigm|_{-\infty}^\infty 
  - (-it)\int_{-\infty}^\infty f(x)e^{-ixt}\,dm(x)
= 0 - 0 + it\hat{f}(t).
\end{equation*}
\end{thmproof}

Let's extend this formula
\begin{llem} \label{lem:fourierdifn}
Let $f$ be an $n$-times differentable function on \(\R\).
Put \(g(x)=D^n(f)(x)\).
If \(D^k(f)\in L^1\) for \(0\leq k \leq n\) then
\begin{equation} \label{eq:Fourierdifn}
\hat{g}(t) = \widehat{D^n(f)}(t) = (it)^n\hat{f}(t) \qquad (t\in\R).
\end{equation}
\end{llem}
\begin{thmproof}
The previous locall lemma \ref{lem:fourierdif}
establishes \eqref{eq:Fourierdifn} for \(n=1\).
Assume by induction that \eqref{eq:Fourierdifn} holds for \(1\leq n < k\).
Now
\begin{equation*}
\widehat{D^k(f)}(t) 
= \widehat{D(D^{k-1}(f))}(t) 
= it \widehat{D^{k-1}(f)}(t) 
= it \cdot (it)^{k-1}\hat{f}(t)
= (it)^k\hat{f}(t).
\end{equation*}
\end{thmproof}

%%%%%%%%%%%%%%%%%%%%%%%%%%%%%%%%%%%%%%%%%%%%%%%%%%%%%%%%%%%%%%%%%%%%%%%%
%%%%%%%%%%%%%%%%%%%%%%%%%%%%%%%%%%%%%%%%%%%%%%%%%%%%%%%%%%%%%%%%%%%%%%%%
\section{Exercises} % pages 193-195

%%%%%%%%%%%%%%%%%%%%%%%%%%%%%%%%%%%%%%%%%%%%%%%%%%%%%%%%%%%%%%%%%%%%%%%%
\section{Local Lemmas} 

\begin{llem}
If \(m\in\Z\) and
\begin{equation*}
  f(x) = x^m e^{-x^2}
\end{equation*}
then
\begin{equation}
  \lim_{x\to\infty} f(x) = 0. \label{eq:lim:xm:ex2:eq0}
\end{equation}
\end{llem}
\begin{thmproof}
Since \(f(-x) = \pm f(x)\) we can restrict our attention to \(0<x\to+\infty\).
By induction on~$m$. Clearly \eqref{eq:lim:xm:ex2:eq0} holds for \(m \leq 0\).
Assume it holds for all \(m < k\).
Now by L'Hospital rules and induction assumption
\begin{equation*}
\lim_{x\to\infty} f(x) 
= \lim_{x\to\infty} \frac{mx^{m-1}}{2x e^{x^2}}
= (m/2) \lim_{x\to\infty} x^{m-2} e^{-x^2} = 0.
\end{equation*}
\end{thmproof}


\begin{llem} \label{llem:9.calF}
Let \calF\ be a family of functions of the following form
\begin{equation*}
f(x) = \sum_{j=1}^n a_jx^{m_j}e^{-x^2}
\end{equation*}
where \(a_j\in\C\) and \(m\in\Z^+\).
If \(f \in \calF\) then \(f'\in\calF\).
\end{llem}
\begin{thmproof}
Differentiate \(f(x) = x^me^{-x^2}\) gives
\begin{equation*}
f'(x) = mx^{m-1} e^{-x^2} -2x^{m+1}e^{-x^2} \in \calF.
\end{equation*}
Since differentiation is linear and \calF\ is a linear space,
and since \(f'\in \calF\) for a $f$ in a base, \(f'\in \calF\) also for
all \(f\in \calF\).
\end{thmproof}


\begin{llem} \label{lem:fxfL1}
Assume $f$ is measurable and \(x^nf(x)\) are bounded for \(n=1,2\).
Then \(f\in L^1\).
\end{llem}
\begin{thmproof}
For \(n=0,1,2\) let \(A_{1n}\) be such that 
\(|x^nf(x)| \leq A_{1n}\) (See Exercise~9.\ref{ex:Amn})
\begin{equation*}
\|f\|_1 
= \int |f|
= \int_{-1}^1 |f(t)|\,dt + \int_{|t|>1} |f(t)|\,dt
\leq 2A_{10} + \int_{|t|>1} A_{12}t^{-2}\,dt
= 2(A_{10} + A_{12}).
\end{equation*}
Hence \(f\in L^1\).


\begin{llem} \label{lem:fxfL2}
Assume $f$ is measurable and \(x^nf(x)\) are bounded for \(n=0,1\).
Then \(f\in L^1\).
\end{llem}
Let
\begin{align*}
L &= \{x\in\R\setminus[-1,1]: |f(x)| < 1\} \\
H &= \{x\in\R\setminus[-1,1]: |f(x)| \geq 1\}
\end{align*}
\begin{align*}
\|f\|_2^2 
&= \int_{\R} |f|^2
= \int_{-1}^1 |f|^2 + \int_L |f|^2 + \int_H |f|^2 \\
&\leq 2A_{10}^2 + \int_L |f| + \int_H (A_{11}/t)^2\,dt \\
&\leq 2A_{10}^2 + \|f\|_1 + 2A_{11}^2.
\end{align*}
Hence \(f\in L^2\).
\end{thmproof}


\begin{llem} \label{lem:fxfL12}
Assume $f$ is measurable and \(x^nf(x)\) are bounded for \(0 \leq n \leq 2\).
Then
 \(f\in L^1\cap L^2\).
and
 \(\hat{f}\in L^2\cap C_0\).
\end{llem}
\begin{thmproof}
Combining the results of previous local lemmas 
\ref{lem:fxfL1} and \ref{lem:fxfL2}
we have \(f\in L^1\cap L^2\).
By 
Theorem~9.6 and
\index{Plancherel}
Plancherel Theorem~9.13 \(\hat{f}\in L^2\cap C_0\).
\end{thmproof}

\paragraph{Derivative of rational polynomial.}

\begin{llem} \label{lem:ratpoly:vansihderivf}
Let \(f(x) = p(x)/q(x)\) where \(p(x)\) and \(q(x)\) are 
real polynomials defined on~\(\R\) and \(q(x)\neq 0\) for all \(x\in\R\)
and \(\deg(p) < \deg(q)\).
Then 
\begin{itemize}
\itemch{a} \(f\in C^\infty(\R)\).

\itemch{b} The derivatives are rational polynomials of the form
\begin{equation} \label{eq:ratpoly:vansihderivf:denom}
f^{(n)}(x) = \frac{s_n(x)}{\bigl(q(x)\bigr)^{2^n}}
\end{equation}
for some polynomial \(s_n(x)\) such that 
\(\deg(s_n(x)) < \deg(q(x))^{2^n} = 2^n\deg(q(x))\).


\itemch{c} All derivatives vanish at infinity
\begin{equation} \label{eq:ratpoly:vansihderivf}
\lim_{x\to\pm\infty} f^{(n)}(x) = 0 \qquad (n \in\Z^+).
\end{equation}

\end{itemize}
\end{llem}
\begin{thmproof}
Clearly \(\deg(q) \geq 1\) and \(\lim_{x\to\pm\infty}q(x) = \pm\infty\).
Assume first \(n=0\) 
then \eqref{eq:ratpoly:vansihderivf:denom} is trivial 
and \eqref{eq:ratpoly:vansihderivf}
follows by applying applying L'Hospital rule \(\deg(p)\)-times.

By induction, 
assume \eqref{eq:ratpoly:vansihderivf:denom} 
and \eqref{eq:ratpoly:vansihderivf} holds for \(n=k\).
Now
\begin{equation*}
f^{(k+1)}(x) 
= \left(\frac{s_k(x)}{\bigl(q(x)\bigr)^{2^k}}\right)'
=   \frac{{s_k}'(x)}{\bigl(q(x)\bigr)^{2^k}} 
  - \frac{{s_k}(x) \bigl(q(x)\bigr)^{2^k}}{%
          \left(\bigl(q(x)\bigr)^{2^k}\right)^2} 
= \frac{s_{k+1}(x)}{\bigl(q(x)\bigr)^{2^{k+1}}}
\end{equation*}
where \(s_{k+1}(x) = (q(x))^{2^k}({s_k}'(x) - s_k(x))\).
Clearly
\begin{equation*}
\deg(s_{k+1}(x)) < \deg\left(\bigl(q(x)\bigr)^{2^{k+1}}\right)
\end{equation*}
hence \eqref{eq:ratpoly:vansihderivf:denom} 
and \eqref{eq:ratpoly:vansihderivf} holds for \(n=k+1\) as well.
\end{thmproof}

%%%%%%%%%%%%%%%%%%%%%%%%%%%%%%%%%%%%%%%%%%%%%%%%%%%%%%%%%%%%%%%%%%%%%%%%
\section{The Exercises} % pages 193-195


%%%%%%%%%%%%%%%%%
\begin{enumerate}
%%%%%%%%%%%%%%%%%


%%%%%%%%%%%%%% 1
\begin{excopy}
Suppose \(f \in L^1\), \(f>0\).
Prove that \(|\hat{f}(y)| < \hat{f}(0)\) for every \(y\neq 0\).
\end{excopy}

We have
\begin{equation*}
\hat{f}(0)
= \int_{-\infty}^\infty f(t)e^{-i0t}\,dm(t)
= \int_{-\infty}^\infty f(t)\,dm(t)
= \|f\|_1 \geq 0.
\end{equation*}
Looking at the mutually disjoint \(G_n = f^{-1}\bigl([n-1,n)\bigr)\)
it is easy to see that \(\|f\|_1 > 0\). Actually the same reasoning
shows that
\begin{equation*}
\int_a^b f(t)\,dm(t) > 0
\end{equation*}
Whenever \(a<b\), simply by looking at \(G_n \cap [a,b]\).


For any \(y\in\R\)
\begin{equation*}
|\hat{f}(y)|
= \left| \int_{-\infty}^\infty f(t)e^{-iyt}\,dm(t) \right|
\leq  \int_{-\infty}^\infty |f(t)e^{-iyt}|\,dm(t)
= \int_{-\infty}^\infty f(t)\,dm(t)
= \hat{f}(0).
\end{equation*}
Assume \(y\neq 0\),
we still need to show strict inequality.
Let us have the following simple lemma
\begin{llem} \label{lem:aubuLYab}
If \(a,b>0\) and \(|u_1|=|u_2|=1\) and \(u_1 \neq u_2\),
then
\begin{equation} \label{eq:ex9.1:llem}
a + b > |u_1a + u_2 b|
\end{equation}
\end{llem}
\begin{thmproof}
Since
\begin{equation*}
|u_1a + u_2 b| = |u_1a/u_2 + b|
\end{equation*}
We may assume \(u_2=1\) and \(u = u_1 = e^{i\theta}\)
with \(\theta \neq 0 \mod 2\pi\) that is \(\cos\theta < 1\).
Thus

\begin{equation*}
|ua + b|^2
= (a\sin\theta)^2 + (a\cos\theta) + b)^2
= a^2 + 2ab\cos\theta + b^2
\end{equation*}
\begin{equation*}
(a+b)^2 - |ua + b|^2 = 2ab(1-\cos\theta) > 0.
\end{equation*}
Hence \((a+b)^2 > |ua + b|^2\) which gives \eqref{eq:ex9.1:llem}.
\end{thmproof}

Back to the exercise.
\begin{align*}
\hat{f}(0) - |\hat{f}(y)|
&=
   \int_{-\infty}^\infty f(t)\,dm(t)
   - \left| \int_{-\infty}^\infty f(t)e^{-iyt}\,dm(t) \right| \\
&\geq \int_0^\infty
           \bigl(f(t) + f(-t)\bigr) -
           \left| f(t)e^{-iyt} + f(-t)e^{iyt}\right| \,dm(t)
\end{align*}
By the lemma~\ref{lem:aubuLYab}, the last integrand is positive \aded,
thus \(\hat{f}(0) - |\hat{f}(y)| > 0\)
which gives the desired result.



\iffalse
Pick arbitrary \(\epsilon\in(0,\|f\|_1/2)\).
By Theorem~3.14 \cite{RudinRCA87} there exists
a~continuous function $g$ such that \(\|g-f\|_1 < \epsilon\).
We may assume that \(g\geq 0\), otherwise we pick \(|g|\).
By continuity and choice of \(\epsilon\) there exists \(\xi\)
such that \(g(\xi)>0\) and \(a,\delta>0\) such that
\(g(x)\geq a\) whenever \(x\in[\xi-\delta,\xi+\delta]\).
\fi

%%%%%%%%%%%%%% 2
\begin{excopy}
Compute the Fourier transform of the characteristic function
of an interval. For \(n=1,2,3,\ldots\), let
\(g_n\) be the characteristic if \([-n,n]\),
let $h$ be the characteristic
function of \([-1,1]\) and compute \(g_n \ast h\) explicitly.
(The grpah is piecewise linear).
Show that \(g_n \ast h\) is the Fourier transform
of a~function \(f_n\in L^1\); except for a multiplicative constant,
\begin{equation*}
f_n(x) = \frac{\sin x\,\sin nx}{x^2}
\end{equation*}
Show that \(\|f_n\|_1 \to \infty\) and conclude that the mapping
\(f \to \hat{f}\) maps \(L^1\) into a \emph{proper} subset of \(C_0\).
Show however, that the range of this mapping is dense in \(C_0\).
\end{excopy}

Let \(f=\chhi_{[a,b]}\) then
\begin{align*}
\hat{f}(x)
 &= \int_{-\infty}^\infty f(x)\cdot e^{-ixt}\,dm(t)
  = \int_a^b e^{-ixt}\,dm(t)
  = (e^{-ixb} - e^{-ixa})/\bigl(-ix\sqrt{2\pi}\,\bigr) \\
 &= i(e^{-ixb} - e^{-ixa})/\bigl(\sqrt{2\pi}x\bigr).
\end{align*}
Hence 
\begin{equation*}
\hat{g_n}(x) 
 = i(e^{-inx} - e^{inx})/\bigl(\sqrt{2\pi}x\bigr)
 = -i^2((e^{inx} - e^{-inx})/i)\bigl(\sqrt{2\pi}x\bigr)
 = \frac{1}{\sqrt{2\pi}}\sin(nx)/x.
\end{equation*}
Similarly,
\begin{equation} \label{eq:fourier:chi}
\widehat{\chhi_{[-\lambda,\lambda]}}(x) = \frac{1}{\sqrt{2\pi}}\sin(\lambda x)/x
\end{equation}
for any \(\lambda\in\R^{+}\).


Now we compute the convolution.
\begin{align*}
(g_n \ast h)(x)
 &= \int_{\infty}^\infty g_n(x-t)h(t)\,dm(t) \\
 &= \int_{-1}^1 g_n(x-t)\,dm(t)
  = \left\{
    \begin{array}{ll}
    2              & |x| \leq n - 1 \\
    n+1-|x| \qquad & n-1 \leq |x| \leq n+1 \\
    0 \qquad       & |x| \geq n + 1
    \end{array}\right.
\end{align*}

Put \(\varphi_n = g_n \ast h\) and compute:
\begin{align*}
\hat{\varphi_n}(t) 
 &= \int_{-\infty}^\infty \varphi_n(x)e^{-itx}\,dm(x)
 =  \int_{-\infty}^\infty 
         \left(\int_{-\infty}^\infty g_n(x-y)h(y)\,dm(y)\right)
         e^{-itx}\,dm(x) \\
 &= \int_{-\infty}^\infty 
          \left(\int_{-\infty}^\infty g_n(x-y)e^{-it(x-y)}\,dm(x)\right)
         h(y)e^{-ity}\,dm(y) \\
 &= \int_{-\infty}^\infty 
          \left(\int_{-\infty}^\infty g_n(x)e^{-itx}\,dm(x)\right)
         h(y)e^{-ity}\,dm(y) \\
 &= \int_{-\infty}^\infty \hat{g_n}(x) h(y)e^{-ity}\,dm(y) 
  = \hat{g_n}(x)  \int_{-\infty}^\infty h(y)e^{-ity}\,dm(y) \\
 &= \hat{g_n}(x) \hat{h}(x) = \beta_n \frac{\sin(x)\sin(nx)}{x^2}
\end{align*}
for some \(\beta_n\).

By the inversion formula (Theorem~9.11 \cite{RudinRCA87})
and the fact that \(f_n\) is an even function
\begin{align*}
\varphi_n(t) 
&= \int_{\infty}^\infty \hat{\varphi_n}(x) e^{itx}\,dm(x)
 = \int_{\infty}^\infty \hat{g_n}(x)\hat{h}(x) e^{itx}\,dm(x) \\
&= \beta_n \int_{\infty}^\infty f_n(x) e^{itx}\,dm(x) 
 = \beta_n \int_{\infty}^\infty f_n(x) e^{-itx}\,dm(x) \\
&= \beta_n \hat{f_n}(t).
\end{align*}


We will now show
\begin{equation} \label{eq:ex9.1:fn1:inf}
\lim_{n\to\infty}\|f_n\|_1 = \infty.
\end{equation}
If \(0<x\leq\pi/2\) then \(\sin x/x \geq 2/\pi > 1/2\).
\begin{align*}
\|f_n\|_1
&= \int_{-\infty}^\infty \left|\frac{\sin x\,\sin nx}{x^2}\right|\,dm(x)
 \geq \int_0^1 (\sin x)\cdot|\sin nx|/x^2\,dm(x) \\
&\geq \int_0^1 (|\sin nx|/x)(\sin x/x)\,dm(x)
 \geq (1/2)\int_0^1 (|\sin nx|/x)\,dm(x)
\end{align*}
We will soon estimate the last integral.
The function \(\sin nx\) has \(\lfloor n/2\pi \rfloor\)
complete periods within \([0,1]\).
Since
\begin{equation*}
\sin \pi/3 = \sin 2\pi/3 = -\sin 4\pi/3 = -\sin 5\pi/3 = 1/2,
\end{equation*}
in each period $w$,
the lengths total of the two sub-intervals where \(|\sin nx| \geq 1/2\)
is \(w/3\). The total lengths of these sub-intervals in \([0,1]\)
is at least
\begin{equation*}
(2\pi/n) \lfloor n/2\pi \rfloor / 3 \geq 1/4
\end{equation*}
for \(n>2\pi\).
Hence
\begin{align*}
\int_0^1 (|\sin nx|/x)\,dm(x)
&\geq \sum_{k=0}^{\lfloor n/2\pi \rfloor}
  \int_{2\pi k/n}^{2\pi(k+1)/n} |\sin nx|/x\,dm(x) \\
&\geq (1/4)\sum_{k=1}^{\lfloor n/2\pi \rfloor}
         (1/2)\cdot (1/k)
 \geq (1/8)\sum_{k=1}^{\lfloor n/7\rfloor} 1/k
\end{align*}
which clearly converges to \(+\infty\) as \(n\to\infty\)
and \eqref{eq:ex9.1:fn1:inf} is shown.


By Theorem~9.6 (\cite{RudinRCA87}) the Fourier transform 
is a continuous mapping of \(L^1\) to \(C_0\).
If by negation (similar to Theorem~5.15 \cite{RudinRCA87})
the mapping were \emph{onto} then by Theorem~5.10 (\cite{RudinRCA87})
there would exist \(\delta>0\) such that 
\begin{equation*}
 \delta \|f_n\|_1 \leq \|\hat{f_n}\|_\infty = 1
\end{equation*}
for all $n$, which is a contradiction to the shown \(\lim_{n\to\infty}\|f_n\|_1 = \infty\).

%%%%%%%%%%%%%% 3
\begin{excopy}
Find
\begin{equation*}
\lim_{A\to\infty} \int_{-A}^A \frac{\sin \lambda t}{t}e^{itx}\,dt
  \qquad (\infty < x < \infty)
\end{equation*}
where \(\lambda\) is a positive constant.
\end{excopy}

Using \eqref{eq:fourier:chi} of previous exercise, 
and Theorem~9.13\ich{d} (\cite{RudinRCA87})
\begin{equation*}
\lim_{A\to\infty} \int_{-A}^A \frac{\sin \lambda t}{t}e^{itx}\,dt
= \sqrt{2\pi} \chhi_{[-\lambda,\lambda]}.
\end{equation*}

%%%%%%%%%%%%%% 4
\begin{excopy}
Give examples of \(f\in L^2\) such that \(f\notin L^1\)
but \(\hat{f}\in L^1\). Under what circumstances can this happen?
\end{excopy}

The function in previous exercise, namely \(f(x) = \sin(\lambda t)/t\) 
is an example. 
The question is --- for which \(f\in L^2\setminus L^1\)
we have \(\hat{f}\in L^1\)\,?

%%%%%%%%%%%%%% 5
\begin{excopy}
If \(f\in L^1\) and \(\int|t\hat{f}(t)|\,dm(t) < \infty\),
prove that $f$ coincides \aded\ with a differentiable function
whose derivative is
\begin{equation*}
i \int_{-\infty}^\infty t\hat{f}(t)e^{ixt}\,dm(t).
\end{equation*}
\end{excopy}

By Theorem~9.6 \(\hat{f}\in L^1\), hence
\begin{equation*}
\int_{-\infty}^\infty |\hat{f}(t)|\,dm(t)
\leq \int_{-1}^1 |\hat{f}(t)|\,dm(t) 
     + \int_{\R\setminus[-1,1]} |t\hat{f}(t)|\,dm(t)
< \infty.
\end{equation*}
and \(\hat{f}\in L^1\).

By the inversion Theorem~9.11, we can define
\begin{equation*}
g(x) = \int_{-\infty}^\infty \hat{f}(t)e^{ixt}\,dm(t)
\end{equation*}
and \(f(x)=g(x)\,\aded\)

Differentiate
\begin{align}
g'(x) 
&= \lim_{h\to 0} (g(x+h)-g(x))/h \notag \\
&= \lim_{h\to 0} 
    \left(
     \int_{-\infty}^\infty \hat{f}(t)(e^{i(x+h)t} - e^{ixt})\,dm(t)
    \right) \,\bigm/\, h \notag \\
&= \lim_{h\to 0} 
    \left(
     \int_{-\infty}^\infty \hat{f}(t)(e^{i(x+h)t} - e^{ixt})/h\,dm(t)
    \right) \notag \\
&= 
     \int_{-\infty}^\infty \hat{f}(t)
    \left(
          \lim_{h\to 0} (e^{i(x+h)t} - e^{ixt})/h
    \right) 
    \,dm(t)    \label{eq:ex9.5:lbgdom} \\
&= i\,\int_{-\infty}^\infty t\hat{f}(t)e^{ixt}\,dm(t) \notag
\end{align}

The \eqref{eq:ex9.5:lbgdom} equality is justified by the same argument
of section~9.3(\emph{a}) and the fact that for sufficiently small $h$
we have \(|(e^{i(x+h)t} - e^{ixt})/h|<2\) together with \(\hat{f}\in L^1\).

%%%%%%%%%%%%%% 6
\begin{excopy}
Suppose  \(f\in L^1\), $f$ is differentiable almost everywhere,
and \(f'\in L^1\). Does it follow that the Fourier transform
of \(f'\) is \(ti\hat{f}(t)\)?
\end{excopy}

Consider
\begin{equation*}
f(x) = \left\{
  \begin{array}{ll}
  e^{-x} & x \geq 0\\
  0     & x < 0
  \end{array}
  \right.
\end{equation*}
and \(f'(x) = -f(x)\,\aded\) Now
\begin{equation*}
\hat{f}(t) 
 = \int_0^\infty e^{-x}e^{-ixt}\,dm(x)
 = \int_0^\infty e^{-x(1+it)}\,dm(x)
 = \frac{-1}{1+it}\,e^{-x(1+it)}\bigm|_0^\infty
 = 1/(1+it)
\end{equation*}
Clearly the conjecture here fails.


%%%%%%%%%%%%%% 7
\begin{excopy} 
Let 
\label{ex:Amn}
$S$ be the class of all functions $f$ on \(\R^1\) which have
the following property:
$f$ is infinitely differentiable,
and there are numbers \(A_{m n}(f)<\infty\),
for $m$ and \(n=0,1,2,\ldots\), such that
\begin{equation*}
\left| x^nD^m f(x)\right| \leq A_{m n}(f) \qquad (x\in\R^1).
\end{equation*}
Here $D$ is the ordinary differentiable operator.

Prove that the Fourier transform maps $S$ onto $S$.
Find examples of members of $S$.
\end{excopy}

\textbf{Note:} This $S$ space is called
\index{Schwartz space}
Schwartz space in other texts.

Assume \(f\in S\).
In order to show that \(\hat{f}\in S\) we need to show
\(\hat{f}\) is infinitely differentiable and 
that \(A_{mn}(\hat{f})<0\) exist for all \(m,n\in\Z^+\).

\begin{equation*}
\hat{f}(t) = \int_{\R} f(x)e^{-ixt}\,dm(x)
\end{equation*}
Now
\begin{align*}
|\hat{f}(t)| 
 &\leq \int_{\R} |f(x)|\,dm(x)
  = \int_{-1}^1 |f(x)|\,dm(x) + \int_{|x|>1} |f(x)|\,dm(x) \\
 & \leq \int_{-1}^1 A_{00}(f)\,dm(x) + 2\int_1^\infty A_{02}(f)x^{-2}\,dm(x)
  = 2(A_{00}(f) + A_{02}(f)) /\sqrt{2\pi}.
\end{align*}
Hence 
\begin{equation} \label{eq:A00}
A_{00}(\hat{f}) \leq 2(A_{00}(f) + A_{02}(f)) /\sqrt{2\pi}\,.
\end{equation}

If \(g(x) = x^nf(x)\) then clearly \(g\in S\).
By local lemma~\ref{lem:g-eq-xnf}
\(\hat{g}(t) = i^nD^n\hat{f}(t)\) and so by applying the preceding result
\begin{equation*}
|D^n\hat{f}(x)| = |\hat{g}(t)| \leq 2(A_{00}(g) + A_{02}(g)) /\sqrt{2\pi}.
\end{equation*}


If \(g(x) = (D^nf)(x)\) then clearly \(g\in S\).
By local lemma~\ref{lem:fourierdifn}
\(\hat{g}(t) = (it)^n\hat{f}(t)\).

Let \(f\in S\).
By local lemma~\ref{lem:g-eq-xnf}, \(\hat{f}\) is infinitely differentable.
Let \(g(t) = t^m(D^n(\hat{f}))(t)\).
Put \(g_d(t) = (D^n(\hat{f}))(t)\)
and \(h(x) = x^nf(x) \in S\)
By the same local lemma~\ref{lem:g-eq-xnf}
\begin{equation*}
  D^n\hat{f}(t) = (-i)^n\hat{h}(t)
\end{equation*}
Applying \eqref{eq:A00} we have 
\begin{equation*}
A_{n0}(\hat{g}) \leq A_{00}(\hat{h}) < \infty
\end{equation*}

By local lemma~\ref{lem:fourierdifn} applied to \(t^mg_d(t)\) we get
\begin{equation*}
t^m(D^n(\hat{f}))(t) = (-i)^m\widehat{D^m(h)}(t)
\end{equation*}
It is easy to see that \(D^m(h)\in S\) since $S$ is closed under addition.
Thus 
\begin{equation*}
A_{nm}(\hat{f}) = A_{n0}(\widehat{D^mh} < 0
\end{equation*}
and therefore \(\hat{f}\in S\).

\iffalse
If \(f\in C^1(\R)\) and its derivative is bounded then 
for each \(\epsilon>0\) 
we have \(|f(s)-f(t)|<\epsilon\) whenever 
\(|s-t|<\delta=\epsilon / (\|f'\|_\infty+1)\), 
that is $f$ is uniformly continuous.
Hence any member of $S$ is uniformly continuous.

Also if \(f\in S\) and \(m\in\{0,1\}\) then
\begin{equation*}
\int |f| = \int x^2|f|/x^2 \leq A_{02}x^{-2} < \infty
\end{equation*}
and so \(S\subset L^1(\R)\).
Almost directly by definition, if \(f\in S\) and \(g(x) = x^kf(x)\) 
for some \(k\geq 0\)
then \(g\in S\) as well.

Let \(f\in S\), and let \(g(x) = ixf(x)\).
With the above and theorem~9.2\ich{f} \({\hat{f}}' = \hat{g}\).
\fi % false

As an example,
members of \calF\ defined in local lemma~\ref{llem:9.calF} are in $S$.


%%%%%%%%%%%%%% 8
\begin{excopy}
If $p$ and $q$ are conjugate exponents, 
\(f\in L^p\), \(g\in L^q\) and \(h = f\ast g\),
prove that $h$ is uniformly continuous. If also \(1<p<\infty\), then
\(h\in C_0\): show that this fails for some \(f\in L^1\), \(g\in L^\infty\).
\end{excopy}

Along the lines of the proof of Theorem~9.5.\\
Fix \(\epsilon > 0\). There is a continuous \(\phi\) (Theorem~3.14) such that
\begin{align*}
\supp(\phi) &\subset [-A,A] \\
\|f-\phi\|_p &< \epsilon\,.
\end{align*}
\Wlogy, we can enlarge $A$ such that 
\begin{equation*}
\|f\chhi_{\R\setminus[-A,A]}\|_p
= \left(\int_{\R\setminus[-A,A]}\|_p |f(x)|^p\,dm(x)\right)^{1/p} < \epsilon.
\end{equation*}
By uniform continuity of \(\phi\) there exists \(\delta\in (0, A)\)
such that 
\begin{equation*}
 |\phi(s) - \phi(t)| < A^{-1/p}\epsilon\,.
\end{equation*}
We set 
\(f_a(x) = f(a-x)\) and
\(\phi_a(x) = \phi(a-x)\).
Now
\begin{align*}
|h(s)-h(t)|
&= |(f\ast g)(s) - (f\ast g)(t)| \\
&= \left|\int f(s-x)g(x)\,dm(x) - \int (f(t-x)g(x)\,dm(x) \right| \\
&= \left|\int (f(s-x) - f(t-x))g(x)\,dm(x) \right| \\
&\leq \int |(f(s-x) - f(t-x))g(x)|\,dm(x)  \\
&= \int \left|f(s-x) - f(t-x)\right|\cdot|g(x)|\,dm(x) \\
&\leq \|f_s - f_t\|_p \cdot \|g\|_q \\
&\leq \left(
           \|f_s - \phi_s\|_p + \|\phi_s - \phi_t\|_p +\|\phi_t - f_t\|_p 
     \right) \cdot \|g\|_q \\
&= \left(2\|f - \phi\|_p + \|\phi_s - \phi_t\|_p \right) \cdot \|g\|_q \\
&\leq (2\epsilon + (2A (A^{-1/p}\epsilon)^p)^{1/p}) \cdot \|g\|_q \\
&= 4\epsilon \|g\|_q\,.
\end{align*}
Therefore $h$ is uniformly continuous.

As a counterexample, let \(f = \chhi_{[0,1]} \in L^1\)
and \(g = 1 \in L^\infty\). Then also \(h = f \ast g = 1 \notin C_0\).


%%%%%%%%%%%%%% 9
\begin{excopy}
Suppose \(1\leq p < \infty\), \(f\in L^p\), and
\begin{equation*}
g(x) = \int_x^{x+1} f(t)\,dt.
\end{equation*}
Prove that \(g\in C_0\). What can you say about $g$ if \(f\in L^\infty\).
\end{excopy}

Let \(f\in L^p\). Pick some \(\epsilon>0\), then there exist some \(M<\infty\)
such that \(\int_{|t|>M} |f(t)|\,dt < \epsilon\).
Obviously 
\begin{equation*}
\left|\int_x^{x+1}f(t)\,dt\right|
\leq \int_x^{x+1}|f(t)|\,dt
\leq \int_{|t|>M+1}|f(t)|\,dt < \infty.
\end{equation*}
Hence \(g\in C_0\).

If \(f\in L^\infty\) then $g$ is not necessarily in \(C_0\).
For example, if \(f=1\) then \(g=f\in L^\infty \setminus C_0\).


%%%%%%%%%%%%%% 10
\begin{excopy}
Let \(C^\infty\) be the class of all infinitely differentiable complex functions
on \(\R^1\), and let \(C_c^\infty\) consist of all \(g\in C^\infty\) 
whose supports are compact.
Show that  \(C_c^\infty\) does not consist of $0$ alone.

Let \(L_{\textrm{loc}}^1\) be the class 
of all $f$ which belong to \(L^1\) locally;
that is, \(f\in L_{\textrm{loc}}^1\) provided that $f$ is measurable
and \(\int_I |f|<\infty\) for every bounded interval $I$.

If \(f\in L_{\textrm{loc}}^1\) and \(g\in C_c^\infty\), prove that 
\(f\ast g \in C^\infty\).

Prove that there are sequences \(\{g_n\}\) in \(C_c^\infty\), such that
\begin{equation*}
\|f\ast g_n - f\|_1 \to 0
\end{equation*}
as \(n\to \infty\), for every \(f\in L^1\).
(Compare Theorem~9.10.)
Prove that  \(\{g_n\}\) can also be so chosen that 
\((f\ast g_n)(x) \to f(x)\,\aded\), for every \(f\in L_{\textrm{loc}}^1\);
in fact for suitable \(\{g_n\}\) the convergence occurs at every point
$x$ at which $f$ is derivative of its indefinite integral.

Prove that \((f \ast h_\lambda)(x) \to f(x) \,\aded\) if \(f\in L^1\), 
as \(\lambda\to 0\), and that \(f\ast h_\lambda \in C^\infty\),
although \(h_\lambda\) does not have compact support.
(\(h_\lambda\) is defined in Sec~9.7.)
\end{excopy} 

\paragraph{Example of non trivial \(C_c^\infty\) function.}
Define
\begin{equation*}
v(x) = \left\{
\begin{array}{ll}
e^{-1/x^2}e^{-1/(x-1)^2} \qquad &x\in(0,1) \\
0 & x \notin (0,1)
\end{array}
\right.
\end{equation*}
It can be shown that \(D^n(v)(0) = D^n(v)(1) = 0\) and so 
\(0\neq v \in C_c^\infty\).

\paragraph{Convolution resulting in \(\C^\infty\).}
Assume \(f\in L_{\textrm{loc}}^1\) and \(g\in C_c^\infty\)
and put \(h = f\ast g\). We may assume that \(\supp g \subset [-A,A]\)
for some \(A < \infty\).
By definition, local lemma~\ref{lem:conv:commut}
Using the substitution \(t=x-y\) we have
\begin{equation*}
h(x) = \int_{-A}^A f(x-y)g(y)\,dm(y) = \int_{-A-x}^{A-x} g(x-y)f(y)\,dm(y)
\end{equation*}
Note that the fact that the support of $g$ is bounded
ensures that the integrals above are finite because 
$f$ is locally in \(L^1\).

The difference ratio of $g$ is bounded since 
\begin{equation} \label{eq:gCcdinf:difrat}
\left|\frac{g(x-y)-g(x+s-y)}{s}\right| \leq \|g'\|_\infty < \infty.
\end{equation}
The last inequality holds becuase \(g'\in C_c^\infty\).

The difference ratio, for \(s>0\)
\begin{align*}
d(x,s) 
&= (h(x+s)-h(x))/s \\
&= \frac{1}{s}\int_{-A-x-s}^{A-x+s} (g(x-y)-g(x+s-y))f(y)\,dm(y) \\
&= \int_{-A-x-s}^{A-x+s} \frac{g(x-y)-g(x+s-y)}{s}f(y)\,dm(y) 
\end{align*}
The inequality \eqref{eq:gCcdinf:difrat} 
together with the discussion in section~9.3\ich{a}
allow us to use Lebesgue's dominated convergence theorem~1.34.
\begin{align*}
h'(x)
&= \lim_{s\to 0}d(x,s) \\
&= \lim_{s\to 0} \frac{1}{s}\int_{-A-x-s}^{A-x+s} (g(x-y)-g(x+s-y))f(y)\,dm(y) \\
&= \int_{-A-x-s}^{A-x+s} \lim_{s\to 0} \frac{g(x-y)-g(x+s-y)}{s}f(y)\,dm(y) \\
&= \int_{-A-x-s}^{A-x+s} \lim_{s\to 0} \frac{g(x-y)-g(x+s-y)}{s}f(y)\,dm(y) \\
&= \int_{-\infty}^\infty g'(x-y)f(y)\,dm(y) \\
&= (f \ast g')(x).
\end{align*}
Therefore, \(f \ast g\) is differentiable, and 
since \(g'\in C_c^\infty\), by induction  
\(f \ast D^kg\) is differentiable
and \(D(f \ast D^kg) = D^{k+1}g\) for all \(k\in\Z^+\) and so 
\(f \ast g \in C^\infty\).


\paragraph{Convolution approximation to identity in \(L^1\).}
Define
\begin{equation}
g_n(x) = \left\{%
\begin{array}{ll}
a_n\exp(-1/(x+1/n)^2)\exp(-1/(x-1/n)^2) \qquad & x\in (-1/n,1/n) \\
0                                       \qquad & x\notin (-1/n,1/n)
\end{array}\right.
\end{equation}
where \(a_n\) is defined such that \(\|g_n\|_1 = 1\).
We will now show that 
\begin{equation} \label{eq:limCc-ast-gn}
\lim_{n\to\infty} \|\varphi\ast g_n - \varphi\|_1 = 0
\qquad \forall \varphi \in C_c(\R)
\end{equation}

% Let \(f\in L^1(\R)\) and 
Let \(\varphi\in C_c(\R)\) and pick some \(\epsilon > 0\).
Since its support is bounded, \(\varphi\) is \emph{uniformly} continuous.
Thus we can find some \(m<\infty\) such that
\(|\varphi(t) - \varphi(s)|<\epsilon/A\) whenever \(|t-s|<2/m\).
For any \(n>m\)
\begin{align*}
\|\varphi \ast g_n - \varphi\|_1
&= \left| \int_{-\infty}^\infty
        \left( \int_{-\infty}^\infty 
          \varphi(x-y) \cdot g_n(y)\,dm(y) - \varphi(x)
        \right)\,dm(x) \right| \\
&\leq \int_{-\infty}^\infty
        \left| \int_{-\infty}^\infty 
          \varphi(x-y) \cdot g_n(y)\,dm(y) - \varphi(x)
        \right|\,dm(x) \\
&=    \int_{-\infty}^\infty
        \left| \int_{-\infty}^\infty 
          \bigl(\varphi(x-y) - \varphi(x)\bigr)\cdot g_n(y)\,dm(y)
        \right|\,dm(x) \\
&=    \int_{-\infty}^\infty
        \left| \int_{-1/n}^{1/n}
          \bigl(\varphi(x-y) - \varphi(x)\bigr)\cdot g_n(y)\,dm(y)
        \right|\,dm(x) \\
&\leq  \int_{-\infty}^\infty
         \left( \int_{-1/n}^{1/n}
          \bigl|\varphi(x-y) - \varphi(x)\bigr|\cdot g_n(y)\,dm(y)
         \right)\,dm(x) \\
&=    \int_{-1/n}^{1/n}
         \left( \int_{-\infty}^\infty
          \bigl|\varphi(x-y) - \varphi(x)\bigr|\cdot g_n(y)\,dm(x)
         \right)\,dm(y) \\
&=    \int_{-1/n}^{1/n}
         \left( \int_{-A}^{A+y}
          \bigl|\varphi(x-y) - \varphi(x)\bigr|\cdot g_n(y)\,dm(x)
         \right)\,dm(y) \\
&\leq  \int_{-1/n}^{1/n}
         \left( \int_{-A}^{A+y}
          (\epsilon/A)\cdot g_n(y)\,dm(x)
         \right)\,dm(y) \\
&\leq    (\epsilon/A) \int_{-1/n}^{1/n}(A+y)g_n(y)\,dm(y) \\
&=    (\epsilon/A)\left((2A/n) + \int_{-1/n}^{1/n} y g_n(y)\,dm(y)\right) \\
&\leq \epsilon(2/n + 1).
\end{align*}
Thus the claim \eqref{eq:limCc-ast-gn} holds.

By theorem~3.14 we can find \(\varphi\in C_c(\R)\)
such that
\(\|f-\varphi\|_1 < \epsilon\).
\iffalse
Find \(0<A<\infty\) such that \(\int_{|x|>A}|f(x)|\,dx < \epsilon\)
and \(supp \varphi \subset [-A,A]\).
By Lusin's theorem~2.24 we can approximate \(f\cdot\chhi_{[-A,A]}\)
with a function \(\varphi\in C_c(\R)\) such that
\begin{equation*}
m\left(\{x\in\R: f(x) \neq \varphi(x)\}\right) < \epsilon\,/
\end{equation*}
\fi

Now
\begin{align*}
\|f\ast g_n - f\|_1
&\leq 
    \|f\ast g_n - \varphi \ast g_n\|_1
  + \|\varphi \ast g_n - \varphi\|_1
  + \|\varphi - f\|_1 \\
&\leq \|(f-\varphi)\ast g_n\|_1 + \|\varphi \ast g_n - \varphi\|_1 + \epsilon \\
&\leq \epsilon\cdot 1 +  \|\varphi \ast g_n - \varphi\|_1 + \epsilon
\end{align*}
With \eqref{eq:limCc-ast-gn}
\begin{equation*}
\lim_{n\to\infty} \|f\ast g_n - f\|_1 = 0
\end{equation*}

\paragraph{Convolution approximation to identity almost everywhere.}
Pick an arbitrary finite interval \(I\subset \R\).
We know that \(\|f\chhi_I\|_1 < \infty\).
Let \(X_n = \{X\in I: n - 1 \leq |f(x)| < n\}\), 
clearly \(I=\disjunion_{n\in\N} X_n\).
Now
\begin{equation*}
\sum_{n\in\N} (n-1)\cdot m(X_n) 
 \leq \|f\cdot\chhi_I\|_1
 < \infty.
\end{equation*}
and the last inequality is because \(f\in L_{\textrm{loc}}^1\).
 % \leq \sum_{n\in\N} n\cdot m(X_n)\,.
We can find some \(N>2\) such that  
\begin{equation*}
\sum_{n>N} m(X_n) < \sum_{n>N} (n-1)\cdot m(X_n) < \epsilon\,.
\end{equation*}
Pick \(\delta = \epsilon/(N \max(m(I),1))\) and now 
\begin{equation*}
\int_E |f(x)|\,dm(x) < 2\epsilon
\end{equation*}
for any \(E\subset I\) such that \(m(E) < \delta\).
\index{Lusin}
By Lusin theorem~2.24, we can find a \(\psi\in C_c(\R)\)
such that  \(|\psi(x)|\leq |f(x)|\) for all \(x\in\R\)
and if \(B = \{x\in\R: f(x)\neq \psi(x)\}\)
then \(m(B) < \epsilon / \|f\chhi_I\|_1\). Now for any \(x\in I \setminus B\)
\begin{align*}
|(f\ast g_n)(x) - f(x)|
&\leq  |(f \ast g_n)(x) - (\psi \ast g_n)(x)|
     + |(\psi \ast g_n)(x) - \psi(x)|
     + |\psi(x) - f(x)| \\
&\leq \left\|f-\psi\right\|_1 \cdot \left\|g_n\right\|_1 
      + |(\psi \ast g_n)(x) - \psi(x)|
      + 0 \\
&= \int_B|f(t) - \psi(t)|\,dm(t) + |(\psi \ast g_n)(x) - \psi(x)| \\
&\leq 2\epsilon + |(\psi \ast g_n)(x) - \psi(x)|.
\end{align*}
Using \eqref{eq:limCc-ast-gn} again, shows that 
\begin{equation*}
\lim_{n\to\infty}|(f\ast g_n)(x) - f(x)| = 0 \qquad \forall x\in I\setminus B.
\end{equation*}
Since \(\epsilon = m(B)\) was arbitrary the limit actually holds
for $x$ almost everywhere in $I$, and since~$I$ was arbitrarily
chosen, consequently the limit holds for almost everywhere in~\(\R\).

\paragraph{Convergence with auxilary function.}
In the proof of the inversion theorem~9.11 it was shown that 
\begin{equation*}
\lim_{n\to\infty}(f\ast h_{\lambda_n})(x) = f(x)\quad\aded
\end{equation*}
for any seqence \(\{\lambda_n\}\) such that 
\(\lim_{n\to\infty} \lambda_n = 0\)
\emph{without} the requirement of \(\hat{f}\in L^1(\R)\).
Again with the argument of section~9.3\ich{a}
\begin{equation*}
\lim_{\lambda\to 0}(f\ast h_\lambda)(x) = f(x)\quad\aded
\end{equation*}

\paragraph{Convolution result in \(C^\infty\)}
Let \(f\in L^1(\R)\) and let \(g(x) = (f\ast h_\lambda)(x)\). Now
\begin{align*}
\bigl(g(x+s)-g(x)\bigr) \bigm/ s
&= \frac{1}{s} 
   \int_{-\infty}^\infty \bigl(f(x-y) - f(x+s-y)\bigr)h_\lambda(y)\,dm(s) \\
&= \frac{1}{s}\left(
   \int_{-\infty}^\infty f(x-y) h_\lambda(y)\,dm(s) -
   \int_{-\infty}^\infty f(x-y)h_\lambda(y-s)\,dm(s)
   \right) \\
&= -\int_{-\infty}^\infty f(x-y)\frac{h_\lambda(s) - h_\lambda(y-s)}{s}\,dm(s)
\end{align*}
The limit exists
\begin{equation*}
g'(x) = -(f \ast {h_\lambda}')(x)\,.
\end{equation*}
Since \(h_\lambda\) satisfies the condition 
of local lemma~\ref{lem:ratpoly:vansihderivf} we have
the derivatives \({h_\lambda}^{(n)}\) bounded and vanish at infinity
for all $n$. Thus we can reapply the above manipulation, to show that 
\begin{equation*}
(f\ast h_\lambda)^{(n)} = (f\ast (h_\lambda)^{(n)})
\end{equation*}
in particular \(f\ast h_\lambda \in C^\infty(\R)\).

%%%%%%%%%%%%%% 11
\begin{excopy}
Find conditions of $f$ and/or \(\widehat{f}\) which ensure the correctness of 
the following formal argument: If
\begin{equation*}
 \varphi(t) = \frac{1}{2\pi} \int_{-\infty}^\infty f(x)e^{-itx}\,dx
\end{equation*}
and
\begin{equation*}
 F(x) = \sum_{k = -\infty}^\infty f(x + 2k\pi)
\end{equation*}
then $F$ is periodic with period \(2\pi\), the $n$th Fourier coefficient of $F$
is \(\varphi(n)\), hence
\(F(x) = \sum \varphi(n)e^{inx}\). In particular,
\begin{equation*}
\sum_{k = -\infty}^\infty f(2k\pi) = \sum_{n = -\infty}^\infty \varphi(n).
\end{equation*}

More generally
\begin{equation} \label{eq:ex9.11}
\sum_{k = -\infty}^\infty f(k\beta) 
= \alpha \sum_{n = -\infty}^\infty \varphi(n\alpha).
\qquad \textnormal{if}\; \alpha>0, \beta>0,\, \alpha\beta = 2\pi.
\end{equation}
What does \eqref{eq:ex9.11} say about the limit, as \(\alpha\to 0\),
of the right-hand side 
(for ``nice'' functions, of course)?
Is this in agreement with the inversion theorem?
\\
\index{Poisson summation formula}
[\eqref{eq:ex9.11} is known as the Poisson summation formula.]
\end{excopy}

(See also Watkins (quoting Bump) \cite{Watkins:fnleqn}.)

Note the use of different factors, and thus
\begin{equation*}
\varphi(t) = \frac{1}{\sqrt{2\pi}} \widehat{f}(t).
\end{equation*}
% http://www.secamlocal.ex.ac.uk/people/staff/mrwatkin/zeta/bump-fnleqn.ps


We assume that $f$ is 
\index{Schwartz}
a~Schwartz function (see exercise\ref{ex:Amn}). 
That is $f$ is infinitely differentiable
and \(x^nD^m f(x)\) are bounded for any \(m,n\in\Z^+\).
Let \(A_{mn}\) be such that \(|x^nD^m f(x)| < A_{mn}\).
We have 
\begin{equation*}
\left|(x+ka)^2f(x+ka)\right| \leq A_{12}
\end{equation*}
hence
\begin{equation*}
\sum_{k = -\infty}^\infty |f(x + ka)|
\leq A_{12} \sum_{k = \infty}^\infty|x+ka|^{-2} < \infty.
\end{equation*}

\iffalse
Pick arbitrary \(x\in\R\) and \(a>0\).
We will now show that \(\sum_{k = -\infty}^\infty f(x + ka)\) 
absolutely converges.
Suffices to show that 
\begin{equation}
\sum_{k \geq k_0} |f(x + ka)| < \infty
\end{equation}
for some \(k_0\). We pick \(k_0\) such that \(x+k_0a > 1\).

Clearly the set \(\{k\in\Z: |x+ka|\leq 1\}\) is finite, and so
\begin{equation*}
A := \left|\sum_{\overset{k\in\Z}{|x+ka|\leq 1}} f(x + ka)\right| < \infty.
\end{equation*}

\begin{align*}
\left|\sum_{k = -\infty}^\infty f(x + ka)\right|
&\leq A + \sum_{\overset{k\in\Z}{|x+ka|>1}} |f(x + ka)| \\
&\leq A + \sum_{\overset{k\in\Z}{|x+ka|>1}} |(x+ka)\cdot f(x + ka)| 
\end{align*}
\fi 

By exercise~\ref{ex:Amn}
\begin{itemize}

\item Clearly \(f\in L^1\).

\item 
The Fourier transform \(\widehat{f}\) is a~Schwartz function as well.
Hence \(\widehat{f}\in L^1\)

\item The series
\begin{gather*}
\sum_{k = -\infty}^\infty f(x + 2k\pi) \\
\sum_{n = -\infty}^\infty \varphi(n) 
\end{gather*}
absolutely converge.
\end{itemize}

As defined above, \(F(x)\) converges absolutely and so its 
derivatives. Clearly it has a period of \(2\pi\) and
thus \(F\in C^\infty(\T)\). Also
\begin{align*}
\int_{-\pi}^\pi F(x)\,dx 
&= \int_{-\pi}^\pi \left(\sum_{k = -\infty}^\infty f(x + 2k\pi)\right)\,dx
   = \sum_{k = -\infty}^\infty \int_{-\pi}^\pi f(x + 2k\pi)\,dx \\
&= \int_{-\infty}^\infty f(x + 2k\pi)\,dx < \infty
\end{align*}
The fact that \(F\in L^1(\T)\) can also be derived 
by the fact that it is continuous on the compcat \T\ and thus bounded.
This argument also shows \(F\in L^p(\T)\) for all \(1\leq p \leq \infty\).

According to the analysis of section~4.26, we define the Fourier coefficient
\begin{equation*}
\widehat{F}(n) = \frac{1}{2\pi} \int_{-\pi}^\pi F(t)e^{-int}\,dt\,.
\end{equation*}
Hence
\begin{align*}
\widehat{F}(n) 
&= \frac{1}{2\pi} \int_{-\pi}^\pi 
     \left(\sum_{k = -\infty}^\infty f(t + 2k\pi)\right)e^{-int}\,dt
 = \frac{1}{2\pi} \sum_{k = -\infty}^\infty
     \int_{-\pi}^\pi f(t + 2k\pi)e^{-int}\,dt \\
&= \frac{1}{2\pi} \int_{-\infty}^\infty f(t)e^{-int}\,dt 
 = \varphi(n)\,.
\end{align*}

Since \(F\in L^2(\T) \subset L^1(\T)\) we have
\begin{equation*}
F(x) = \sum_{n\in\Z} \widehat{F}(n)e^{inx} = \sum_{n\in\Z} \varphi(n)e^{inx}
\end{equation*}
The desired equality above is given by evaluating \(F(0)\).

Now for \(\beta > 0\) define
\begin{align*}
f_\beta(x) &= f(\beta x/(2\pi))\\
\varphi_\beta(t) 
  &= \frac{1}{2\pi} \int_{-\infty}^\infty f_\beta(x)e^{-itx}\,dx
   = \frac{1}{\sqrt{2\pi}} \widehat{f_\beta}(t)
\end{align*}
By Theorem~9.2\ich{d} 
\begin{equation*}
\widehat{f_\beta}(x) = (2\pi/\beta)\widehat{f}(2\pi x/\beta)\,.
\end{equation*}
Clearly \(f_\beta\) is as Schwartz function. By what was shown
and putting \(\alpha = 2\pi/\beta\) we have
\begin{align*}
\sum_{k = -\infty}^\infty f(k\beta) 
&= \sum_{k = -\infty}^\infty f_\beta(2k\pi) 
 = \sum_{n = -\infty}^\infty \varphi_\beta(n)
 = \sum_{n = -\infty}^\infty \frac{1}{\sqrt{2\pi}} \widehat{f_\beta}(n) \\
&= \sum_{n = -\infty}^\infty 
   \frac{1}{\sqrt{2\pi}} (2\pi/\beta)\widehat{f}\bigl((2\pi/\beta)n\bigr)
 = (2\pi/\beta) \sum_{n = -\infty}^\infty 
   \varphi\bigl((2\pi/\beta)t\bigr) \\
&= \alpha \sum_{n = -\infty}^\infty  \varphi(\alpha n)
\end{align*}

Looking at the limit
\begin{equation*}
\lim_{\alpha\to 0} \alpha \sum_{n = -\infty}^\infty \varphi(n\alpha)
= \lim_{\beta\to \infty} \sum_{k = -\infty}^\infty f(k\beta) 
= f(0)
\end{equation*}

%%%%%%%%%%%%%% 12
\begin{excopy}
Take \(f(x) = e^{-|x|}\) in exercise~11 and derive the identity
\begin{equation*}
\frac{e^{2\pi\alpha} + 1}{e^{2\pi\alpha} - 1}
= \frac{1}{\pi} \sum_{n=-\infty}^\infty \frac{\alpha}{\alpha^2 + n^2}.
\end{equation*}
\end{excopy}

Using this $f$ (and \(1/\alpha\) instead of \(\alpha\))
\begin{align}
\sum_{k = -\infty}^\infty f(2k\pi\alpha) 
&= \sum_{k = -\infty}^\infty e^{-|2k\pi\alpha|}
= 1 + 2\sum_{k = 1}^\infty \left(e^{-2\pi\alpha}\right)^k
= 1 + 2\frac{e^{-2\pi\alpha}}{1 - e^{-2\pi\alpha}}
= \frac{e^{-2\pi\alpha} + 1}{1 - e^{-2\pi\alpha}} \notag \\
&= \frac{e^{2\pi\alpha} + 1}{e^{-2\pi\alpha}-1}  \label{eq:sum:fkpia}
\end{align}

As for \(\varphi\)
\begin{align}
2\pi\cdot\varphi(n/\alpha)
&= \int_{-\infty}^\infty e^{-|x|}e^{-i(n/\alpha)x}\,dx
 =   \int_{-\infty}^0 e^{x-i(n/\alpha)x}\,dx
   + \int_0^\infty e^{-x-i(n/\alpha)x}\,dx \notag \\
&=  \frac{1}{1-in\alpha}\left.\left(e^{x-i(n/\alpha)x}\right)\right|_{-\infty}^0
  + \frac{1}{-1-in\alpha}\left.\left(e^{-x-i(n/\alpha)x}\right)\right|_0^\infty 
     \notag \\
&=  \frac{1}{1-in/\alpha} - \frac{1}{-1-in/\alpha}
 =  \frac{(1 + in/\alpha) + (1 - in/\alpha)}{1^2 - (in/\alpha)^2} \notag \\
&= \frac{2\alpha^2}{n^2+\alpha^2}  \label{eq:2pi:varphi:na}
\end{align}

Using the \eqref{eq:ex9.11} of previous exercise 
(with \(1/\alpha\) replacing \(\alpha\)) 
and the above
\eqref{eq:sum:fkpia},
\eqref{eq:2pi:varphi:na} 
we get
\begin{align*} 
\frac{e^{2\pi\alpha} + 1}{e^{2\pi\alpha} - 1}
&= \sum_{k = -\infty}^\infty f(2k\pi\alpha) 
= \frac{1}{\alpha} \sum_{n = -\infty}^\infty \varphi(n/\alpha)
= \frac{1}{\alpha} 
     \sum_{n = -\infty}^\infty 
        \frac{1}{2\pi}\cdot \frac{2\alpha^2}{n^2+\alpha^2} \\
&= \frac{1}{\pi} \sum_{n = -\infty}^\infty  \frac{\alpha}{n^2+\alpha^2}\,.
\end{align*}


%%%%%%%%%%%%%% 13
\begin{excopy}
If \(0 < c < \infty\), define \(f_c(x) = \exp(-cx^2)\).
\begin{itemize}
\itemch{a} 
  Compute \(\widehat{f_c}\). \emph{Hint}: If \(\varphi = \widehat{f_c}\),
    an integration by parts gives \(2c\varphi'(t) + t\varphi(t) = 0\).
\itemch{b}
  Show that there is one (and only one) $c$ for which \(\widehat{f_c} = f_c\).
\itemch{c}
  Show that \(f_a \ast f_b = \gamma f_c\); find \(\gamma\) and $c$ 
  explicitly in terms of $a$ and $b$.
\itemch{d}
  Take \(f = f_c\) in Exercise~11. What is the resulting identity?
\end{itemize} 
\end{excopy}

\begin{itemize}
\itemch{a} 
Put \(f_c(x) = \exp(-cx^2)\) and \(\varphi = \widehat{f_c}\).
We will first compute \(\varphi(0)\).

The following equality is shown 
in \cite{RudinPMA85}~Section~8.31.
\begin{equation*}
\int_{-\infty}^\infty e^{-x^2}\,dx = \sqrt{\pi}.
\end{equation*}
By substituting \(y=\sqrt{c}x\) using \(dx/dy=1/\sqrt{a}\) we get
\begin{equation*}
\int_{-\infty}^\infty e^{-cx^2}\,dx 
= \int_{-\infty}^\infty e^{-y^2}/sqrt{c}\,dy = \sqrt{\pi/c}.
\end{equation*}
Hence
\begin{equation*}
\varphi(0)
= \frac{1}{\sqrt{2\pi}} \int_{-\infty}^\infty e^{-cx^2}e^{-i0x}\,dx \\
= \frac{1}{\sqrt{2\pi}} \cdot \sqrt{\pi/c} 
= 1 / \sqrt{2c}.
\end{equation*}

For any \(t\in\R\)
\begin{align*}
\varphi(t) 
&= \frac{1}{\sqrt{2\pi}} \int_{-\infty}^\infty e^{-cx^2}e^{-itx}\,dx \\
&= \frac{1}{\sqrt{2\pi}} 
  \left.\left(e^{-cx^2}\cdot
        \left(\frac{-1}{it}\right)e^{-itx}
  \right)\right|_{-\infty}^\infty
  - \frac{1}{\sqrt{2\pi}} \int_{-\infty}^\infty 
        \frac{2cx}{it}\cdot  e^{-cx^2}e^{-itx}\,dx \\
&= \frac{2ci}{\sqrt{2\pi}t} \int_{-\infty}^\infty x e^{-cx^2-itx}\,dx
\end{align*}

Its derivative
\begin{equation*}
\varphi'(t) 
= \frac{1}{\sqrt{2\pi}} \int_{-\infty}^\infty (-ix)e^{-cx^2}e^{-itx}\,dx
= \frac{-i}{\sqrt{2\pi}} \int_{-\infty}^\infty xe^{-cx^2}e^{-itx}\,dx
\end{equation*}
Combining the above equalities gives
\begin{equation*}
\frac{t}{2c}\varphi(t)
= \frac{i}{\sqrt{2\pi}} \int_{-\infty}^\infty xe^{-cx^2}e^{-itx}\,dx
= -\varphi'(t) 
\end{equation*}
and so we get the hint which is a homogeneous differential equation
\begin{equation*}
2c\varphi'(t) + t\varphi(t) = 0
% \varphi'(t) = \frac{-t}{2c}\varphi(t).
\end{equation*}
whose solution is
\begin{equation*}
\varphi(t) 
= \varphi(0)\exp\left(-\int_0^t\frac{t}{2c}\,dt\right)
= \frac{1}{\sqrt{2c}}\exp\left(\frac{-t^2}{4c}\right).
\end{equation*}

\itemch{b}
By previous item,
\(c=\half\) is the unique $c$ such that \(f_c = \widehat{f_c}\).

\itemch{c}
Let \(g = f_a \ast f_b\). By Theorem~9.2\ich{c} and previous item
\begin{align*}
\widehat{g}(t) 
= \widehat{f_a}(t)\widehat{f_b}(t)
&= \frac{1}{\sqrt{2a}}\exp\left(\frac{-t^2}{4a}\right) \cdot
  \frac{1}{\sqrt{2b}}\exp\left(\frac{-t^2}{4b}\right)
= \frac{1}{2\sqrt{ab}} \exp\left(\frac{-t^2}{4a}+\frac{-t^2}{4b}\right) \\
&= \frac{1}{2\sqrt{ab}} \exp\left(\frac{-(a+b)t^2}{4ab}\right).
\end{align*}
Thus if \(g = \gamma f_c\) then by looking at \(\widehat{f_c}\) 
we must have
\(c = (a+b)/(4ab)\) and 
\begin{equation*}
\gamma / \sqrt{2c} = 1/(2\sqrt{ab}).
\end{equation*}
Hence
\begin{equation*}
\gamma 
= \frac{\sqrt{2c}}{2\sqrt{ab}}
= \frac{\sqrt{2(a+b)}}{2\sqrt{ab(4ab)}}
= \frac{\sqrt{2(a+b)}}{4ab}
\end{equation*}
 
\itemch{d}
By \cite{RudinPMA85}~Section~8.21
\begin{equation*}
\int_{-\infty}^\infty e^{-x^2}\,dx = \sqrt{\pi}.
\end{equation*}
Hence, using the substitution \(y^2=cx^2\) and \(dx/dy=\sqrt{1/c}\) 
we have
\begin{equation*}
  \int_{-\infty}^\infty e^{-cx^2}\,dx 
= \int_{-\infty}^\infty e^{-y^2}\sqrt{1/c}\;dy = \sqrt{\pi/c}
\end{equation*}

Now look at a Fourier transform (without a constant factor).
Given the substitution \(x = (1/\sqrt{c})y - it/(2c)\) we can 
have a ``sqaure -- linear-free'' expression
\begin{equation*}
-cx^2 -itx 
= -c\left(\frac{1}{\sqrt{c}}y - \frac{it}{2c}\right)^2
  -it\left(\frac{1}{\sqrt{c}}y - \frac{it}{2c}\right)
= -y^2 - \frac{t^2}{4c}
\end{equation*}
and \(dx/dy = 1/\sqrt{c}\). Now
\begin{equation*}
\int_{-\infty}^\infty e^{-cx^2}e^{-itx}\,dx
= \int_{-\infty}^\infty e^{-y^2 - t^2/(4c)}\frac{1}{\sqrt{c}}\,dy
= \frac{e^{-t^2/(4c)}}{\sqrt{c}}
   \int_{-\infty}^\infty e^{-y^2}\,dy
= \sqrt{\frac{\pi}{c}}e^{-t^2/(4c)}
\end{equation*}
Hence if \(f_c(x) = \exp(-cx^2)\) then
\begin{equation} \label{eq:gaussian:fourier}
\widehat{f_c}(t) 
= \frac{1}{\sqrt{2\pi}} \sqrt{\frac{\pi}{c}}e^{-t^2/(4c)}
= \frac{1}{\sqrt{2c}} e^{-t^2/(4c)}
\end{equation}

We note that 
in the exercise~11
\(\varphi\) 
is defined with another constant, namely
\(\varphi = (1/\sqrt{2\pi})\widehat{f}\).
Thus 
\begin{equation*}
\alpha \sum_{n = -\infty}^\infty \varphi(n\alpha)
= \frac{\alpha}{\sqrt{2\pi}} \sum_{n = -\infty}^\infty 
  \frac{1}{\sqrt{2c}} e^{-(n\alpha)^2/(4c)}
= \frac{\alpha}{2\sqrt{\pi c}} \sum_{n = -\infty}^\infty  e^{-(n\alpha)^2/(4c)}
\end{equation*}

Now the equality
\begin{equation} 
\sum_{k = -\infty}^\infty f(k\beta) 
= \alpha \sum_{n = -\infty}^\infty \varphi(n\alpha).
\qquad \textnormal{if}\; \alpha>0, \beta>0,\, \alpha\beta = 2\pi.
\end{equation}
of exercise~11 with \(f=f_c\) becomes
\begin{equation*}
\sum_{k = -\infty}^\infty e^{-c(k\beta)^2}
= \frac{\alpha}{2\sqrt{\pi c}} \sum_{n = -\infty}^\infty  e^{-(n\alpha)^2/(4c)}
\end{equation*}
With \(\alpha=1\) and \(\beta=2\pi\) it becomes
\begin{equation*}
\sum_{k = -\infty}^\infty e^{-c(2\pi k)^2}
= \frac{1}{2\sqrt{\pi c}} \sum_{n = -\infty}^\infty  e^{-n^2/(4c)}
\end{equation*}
\end{itemize}


%%%%%%%%%%%%%% 14
\begin{excopy}
The Fourier transform can be defined for \(f\in L^1(\R^k)\) by
\begin{equation*}
\Hat{f}(y) = \int_{\R^k} f(x)e^{-ix\cdot y}\,dm_k(x)\qquad (y\in\R^k),
\end{equation*}
where \((x\cdot y) = \sum \xi_j \eta_j\) if 
\(x= (\seq{\xi}{k})\),
\(y= (\seq{\eta}{k})\)
and \(m_k\) is Lebesgue measure on \(\R^k\),
divided by \((2\pi)^{k/2}\) for convenience. Prove the inversion theorem and
\index{Plancherel theorem}
the Plancherel theorem in this context, as well as the analogue of Theorem~9.23.
\end{excopy}

This is actually redoing most of the chapter~(9) generalizing 
the domain of functions from real line to \(\R^k\).
This approach is taken in \cite{LiebLoss200104} Chapter~5 from the start.

Let's proceed, we put in brackets the sections theorems numbers
being generalized.

We generalize the definition [9.1(4)] of 
\index{convolution}
convolution.
Let \(f,g\in L^1(\R^k)\), then
\begin{equation*}
(f \ast g)(x) = \int_{\R^k} f(x-y)g(y)\,dm_k(y).
\end{equation*}

The generalization of Theorem~9.2 items: 
\ich{a},
\ich{b}
\ich{c},
\ich{d}
follows easily.

The generalization of Theorem~9.5 is trivial.
We carry over the definition
\begin{equation*}
f_y(x) = f(x-y) \qquad (x,y\in \R^k).
\end{equation*}
% We can also consider multi-valued (in \(\C^n\)) functions.
The change needed in the proof worth noting is to take 
$k$-power of \(\epsilon\).

Generalization of Theorem~9.6.
\begin{llem} \label{lem:9.6:kdim}
If \(f\in L^1(\R^k)\), then \(\widehat{f}\in C_0(\R^k)\) and 
\begin{equation*}
\|\widehat{f}\|_\infty \leq \|f\|_1.
\end{equation*}
\end{llem}
\begin{thmproof}
The inequality is obvious from the definition of the transform
and noting that \(|e^{-i(t\cdot x)}|=1\).

If \(\lim_{n\to 0}t_n=t \in \R^k\), then
\begin{equation*}
\left|\widehat{f}(t_n) - \widehat{f}(t)\right|
\leq \int_{\R^k} 
     |f(x)|\cdot\left|e^{-i(t_n\cdot x)} - e^{-i(t\cdot x)}\right|\,dm(x).
\end{equation*}
The integrand is bounded by \(2|f(x)|\) and vanishes as \(n\to\infty\).
Hence \(\lim_{n\to\infty} \widehat{f}(t_n)=\widehat{f}(t)\), 
by the dominated convergence theorem~1.34. Thus \(\widehat{f}\) 
is continuous.

% By definitions and using \(e^{\pi i}=-1\)
% Let \(u=(1,1,\ldots,1)\in\R^k\), hence 
For each \(t\in\R^k\setminus\{0\}\) we can pick ``an inverse''
\(\tau(t)\in\R^k\) such that \(t\cdot\tau_t = 1\) in the following way.
Let \(m\in \N_k\) be the minimal such that
\(|t_m|=\max\{|t_j|: 1\leq j \leq k\}\)
and define
\begin{equation*}
\tau(t)_j = \left\{%
\begin{array}{ll}
1/t_m \quad& \textrm{if}\; j=m \\
0          & \textrm{if}\; j\neq m \\
\end{array}\right.
\end{equation*}

Note that if \(\lim_n \|t_n\|_p=\infty\) then 
\begin{equation} \label{eq:taut:to0}
\lim_n \|\tau(t_n)\|_p = 0  \qquad (1\leq p \leq \infty)
\end{equation}

Now
\begin{align*}
\widehat{f}(t) 
&= \int_{\R^k} f(x)e^{-i(t\cdot x)}\,dm_k(x)
 = -\int_{\R^k} f(x)e^{-i(t\cdot (x + \pi\tau(t))}\,dm_k(x) \\
&= -\int_{\R^k} f(x - \pi\tau(t))e^{-i(t\cdot x)}\,dm_k(x).
\end{align*}
Hence 
\begin{equation*}
2\widehat{f}(t) 
= \int_{\R^k} \bigl(f(x) - f(x - \pi\tau)\bigr) e^{-i(t\cdot x)}\,dm_k(x)
\end{equation*}
so that
\begin{equation*}
2\|\widehat{f}(t)\| \leq \|f - f_{\pi\tau(t)}\|_1.
\end{equation*}
By previous local lemma~\ref{lem:9.6:kdim} and \eqref{eq:taut:to0}
\begin{equation*}
\lim_{\|t\|\to\infty} \|f - f_{\pi\tau(t)}\|_1 
= \lim_{\|\tau(t)\|\to 0} \|f - f_{\pi\tau(t)}\|_1 = 0.
\end{equation*}
Hence \(\widehat{f}\in C_0(\R^k)\).
\end{thmproof}


Define [9.7]
\begin{align*}
H(t) &= e^{-\sum_{1\leq j\leq k}|t_j|} \qquad t \in \R^k \\
h_\lambda(x) &= \int_{\R^k} H(\lambda t) e^{i(t\cdot x)}\,dm(t) 
 \qquad \lambda > 0, \; x\in\R^k
\end{align*}

Compute \(h_\lambda(x)\) using the identity established in the 
$1$-dimensioal case in the text.
\begin{align*}
h_\lambda(x) 
&= \int_{\R^k} e^{-\lambda\sum_{1\leq j\leq k}|t_j|} 
               e^{i\sum_{1\leq j\leq k} t_jx_j}\,dm(t) \\
&= 
    \int_{-\infty}^\infty
    \int_{-\infty}^\infty
    \cdots
    \int_{-\infty}^\infty
      e^{\sum_{1\leq j\leq k}-\lambda|t_j| + i t_jx_j}
    \,dm(t_k) 
     \cdots
    \,dm(t_2) 
    \,dm(t_1) 
    \\
&= 
    \int_{-\infty}^\infty
    e^{-\lambda|t_1|+it_1x_1}
    \int_{-\infty}^\infty
    e^{-\lambda|t_2|+it_2x_2}
    \cdots
    \int_{-\infty}^\infty
      e^{-\lambda|t_k| + t_kx_k}
    \,dm(t_k) 
     \cdots
    \,dm(t_2) 
    \,dm(t_1) 
    \\
&= \prod_{j=1}^k \sqrt{\frac{2}{\pi}}\frac{\lambda}{\lambda^2+x_j^2} \\
&= \left(\frac{2}{\pi}\right)^{k/2} \lambda^k 
   \prod_{j=1}^k \frac{1}{\lambda^2+x_j^2}
\end{align*}

The intergal is similarly (using Fubini's theorem~8.8) computed
\begin{align}
\int_{\R^k}h_\lambda(x)\,dm(x)
&=  \int_{-\infty}^\infty
    \int_{-\infty}^\infty
    \cdots
    \int_{-\infty}^\infty
     \left(\frac{2}{\pi}\right)^{k/2} \lambda^k 
       \prod_{j=1}^k \frac{1}{\lambda^2+x_j^2}
    \,dm(t_k) 
     \cdots
    \,dm(t_2) 
    \,dm(t_1) 
    \notag \\
&=  \int_{-\infty}^\infty
      \sqrt{\frac{2}{\pi}}\frac{\lambda}{\lambda^2+x_1^2}
    \cdots
    \int_{-\infty}^\infty
      \sqrt{\frac{2}{\pi}}\frac{\lambda}{\lambda^2+x_k^2}
    \,dm(t_k) 
     \cdots
    \,dm(t_1) 
    \notag \\
&= 1^k = 1. \label{eq:kdim:9.7(4)}
\end{align}
Note that \(0<H(t)\leq 1\) and \(\lim_{\lambda\to 0} H(\lambda t) = 1\).

The next lemma will give us a convolution [(9.8)] equality.
\begin{llem} \label{lem:9.8:kdim}
If \(f\in L^1(\R^k)\), then
\begin{equation} \label{eq:9.8:kdim}
(f\ast h_\lambda)(x) 
= \int_{\R^k} H(\lambda t) \widehat{f}(t)e^{i(x\cdot t)}\,dm(t).
\end{equation}
\end{llem}
\begin{thmproof}
Using Fubini's theorem~8.8
\begin{align}
(f\ast h_\lambda)(x) 
&= \int_{\R^k} f(x-y)
     \left(\int_{\R^k} H(\lambda t)e^{i(t\cdot y)}\,dm(t)\right)\,dm(y) 
     \notag \\
&= \int_{\R^k} H(\lambda t)
     \left(\int_{\R^k} f(x-y)e^{i(t\cdot y)}\,dm(y)\right)\,dm(t)
     \notag \\
&= \int_{\R^k} H(\lambda t)
     \left(\int_{\R^k} f(y)e^{i(t\cdot (x-y))}\,dm(y)\right)\,dm(t)
     \label{eq:kdim:9.8}
     \notag \\
&= \int_{\R^k} H(\lambda t)e^{i(t\cdot x)}
     \left(\int_{\R^k} f(y)e^{-i(t\cdot y)}\,dm(y)\right)\,dm(t)
     \notag \\
&= \int_{\R^k} H(\lambda t)e^{i(t\cdot x)}\widehat{f}(t)\,dm(t).
\end{align}
In \eqref{eq:kdim:9.8} we actually changed \((x-y)\)
by a ``new'' variable $y$.
\end{thmproof}

Generalizing [9.9]
\begin{llem} \label{lem:9.9:kdim}
If \(g\in L^\infty(\R^k)\) and $g$ is continuous at a point $x$, then
\begin{equation}
\lim_{\lambda\to 0} (g \ast h_\lambda)(x) = g(x). \label{eq:9.9:kdim}
\end{equation}
\end{llem}
\begin{thmproof}
We will integrate using the following
\begin{align*}
h_\lambda(y)
&= \left(\frac{2}{\pi}\right)^{k/2} \lambda^k
   \prod_{j=1}^k \frac{1}{\lambda^2+y_j^2}
  \\
&= \left(\frac{2}{\pi}\right)^{k/2} \frac{\lambda^k}{\lambda^{2k}}
   \prod_{j=1}^k \frac{1}{1/\lambda^2}\frac{1}{\lambda^2+y_j^2}
=  \left(\frac{2}{\pi}\right)^{k/2} 
   \prod_{j=1}^k \frac{1}{1+(y_j/\lambda)^2}
 \\
&= h_1(y/\lambda) / \lambda^k
\end{align*}
On account of \eqref{eq:kdim:9.7(4)}, we have
\begin{align*}
\lim_{\lambda\to 0} (g \ast h_\lambda)(x) 
&= \int_{\R^k} \bigl(g(x-y) - g(x)\bigr)h_\lambda(y)\,dm(y) \\
&= \int_{\R^k} \bigl(g(x-y) - g(x)\bigr)\lambda^{-k}h_1(y/\lambda)\,dm(y) \\
&= \int_{\R^k} \bigl(g(x-\lambda s) - g(x)\bigr)
               \lambda^{-k}h_1(s) \prod_{j=1}^k \frac{dy_j}{ds_j}\,dm(s) \\
&= \int_{\R^k} \bigl(g(x-\lambda s) - g(x)\bigr)h_1(s) \,dm(s) 
\end{align*}
Looking at the last integrand
\begin{align*}
\left|\bigl(g(x-\lambda s) - g(x)\bigr)h_1(s)\right| 
     &\leq 2\|g\|_\infty h_1(s) \\
\forall s\in\R^k,\quad 
  \lim_{\lambda\to 0} \bigl(g(x-\lambda s) - g(x)\bigr)h_1(s) &= 0.
\end{align*}
Hence \eqref{eq:9.9:kdim} follows from
the dominated convergence theorem~1.34.
\end{thmproof}

Generalizing the approximation via convolution [9.10]
\begin{llem} \label{lem:9.10:kdim}
If \(1\leq p < \infty\) and \(f\in L^P(\R^k)\), then
\begin{equation}
 \lim_{\lambda\to 0} \|f\ast h_\lambda - f\|_p = 0. \label{eq:9.10:kdim} 
\end{equation}
\end{llem}
\begin{thmproof}
Since \(h_\lambda L^(\R^k)\), where \(1/p+1/q=1\), 
\(f\ast h_\lambda)(x)\) is defined for every \(x\in\R^k\).
By~\eqref{eq:kdim:9.7(4)} we have
\begin{equation*}
(f\ast h_\lambda)(x) - f(x) 
= \int_{\R^k} \bigl(f(x-y)-f(x)\bigr)h_\lambda(y)\,dm(y)
\end{equation*}
and 
\index{Holder@H\"older}
H\"older inequality (theorem~3.3) gives
\begin{align*}
|(f\ast h_\lambda)(x) - f(x)|
&\leq \int_{\R^k} \bigl|f(x-y)-f(x)\bigr|h_\lambda(y)\,dm(y) \\
&=    \int_{\R^k} \left(\bigl|f(x-y)-f(x)\bigr|^p 
           h_\lambda^{1/p}(y)\right) h_\lambda^{1/q}(y)
          \,dm(y) \\
&\leq 
  \left( \int_{\R^k} \bigl|f(x-y)-f(x)\bigr|^p
        h_\lambda(y) \,dm(y) \right)^{1/p} 
  \left(  \int_{\R^k} \left(h_{1/q}\lambda(y)\right)^q\,dm(y)\right)^{1/q} \\
&= \left( \int_{\R^k} \bigl|f(x-y)-f(x)\bigr|^p h_\lambda(y)
                     \,dm(y) \right)^{1/p}
\end{align*}
The last equality holds becuse of \eqref{eq:kdim:9.7(4)}.
Taking power over the above, gives
\begin{equation*}
|(f\ast h_\lambda)(x) - f(x)|^p
\leq \int_{\R^k} \bigl|f(x-y)-f(x)\bigr|^p h_\lambda(y) \,dm(y) 
\end{equation*}
Integrating with respect to $x$ and apply using Fubini's theorem~8.8:
\begin{align}
\| f\ast h_\lambda - f\|_p^p 
&\leq \int_{\R^k} \left(\bigl|f(x-y)-f(x)\bigr|^p h_\lambda(y) \,dm(y)
                 \right)\,dm(x) \notag \\
&= \int_{\R^k} \left(\bigl|f(x-y)-f(x)\bigr|^p \,dm(x)
                 \right) h_\lambda(y) \,dm(y) \notag \\
&= \int_{\R^k} \|f_y - f\|_p^p h_\lambda(y) \,dm(y). \label{eq:fyfpp:hl}
\end{align}
If \(g(y) = \|f_y - f\|_p^p\), then $g$ is bounded and continuous,
by a trivial $k$-dimensioal generalization of theorem~9.5, 
and \(g(0)=0\). Hence the expression in \eqref{eq:fyfpp:hl}
tends to $0$ as \(\lambda\to 0\), by local lemma~\ref{lem:9.9:kdim}
\end{thmproof}

Now we arrive to the generalization of the inversion theorem [9.11].
\begin{llem} \label{lem:9.11:kdim}
If \(f,\widehat{f}\in L^1(\R^k)\) and if 
\begin{equation*}
g(x) = \int_{\R^k} \widehat{f}(t)e^{ixt}\,dm(t)
\end{equation*}
then \(g\in C_0\) and \(f(x)=g(x)\;\aded\)
\end{llem}
\begin{thmproof}
We use the result of local lemma~\ref{lem:9.8:kdim} 
The integrands on the right side of \eqref{eq:9.8:kdim} are bounded
by \(|\widehat{f}(t)\), and since \(\lim_{\lambda\to 0} H(\lambda t) = 0\),
the right side of \eqref{eq:9.8:kdim} converges to \(g(x)\)
for every \(x\in\R^k\), by the dominated convergence theorem~1.34.

By combining local lemma~\ref{lem:9.10:kdim} and theorem~3.12
there is a sequence \(\{\lambda_n\}\) 
such that \(\lim_{n\to\infty}\lambda_n=0\) and 
\begin{equation*}
\lim_{n\to\infty} (f\ast h_{\lambda_n})(x) = f(x)\quad\aded
\end{equation*}
Hence \(f(x)=g(x)\;\aded\)
and \(g\in C_0(\R^k)\) by local lemma~\ref{lem:9.6:kdim}.
\end{thmproof}

\index{Plancherel}
\paragraph{Generalization of Plancherel Theorem [9.13].}
\begin{llem}
Once can associate to each \(f\in L^2(\R^k)\) a function
\(\widehat{f}\in L^2(\R^k)\) so that the following properties hold:
\begin{itemize}

\itemch{a}
If \(f \in L^1(\R^k)\cap L^2(\R^k)\), 
then \(\widehat{f}\) is the previously defined Fourier transform of $f$.

\itemch{b}
For every \(f \in L^2(\R^k)\), \(\|\widehat{f}\|_2 = \|f\|_2\).

\itemch{c}
The mapping \(f \to \widehat{f}\) is a 
\index{Hilbert}
Hilbert space isomorphism of \(L^2(\R^k)\) onto \(L^2(\R^k)\)

\itemch{d}
The following symmetric relation exists between $f$ and \(\widehat{f}\):
If
\begin{equation*}
\varphi_A(t) \int_{[-A,A]^k} f(x)e^{-i(x\cdot t)}\,dm(x)
\qquad \textrm{and} \qquad
\psi_A(t) \int_{[-A,A]^k} \widehat{f}(x)e^{-i(x\cdot t)}\,dm(x),
\end{equation*}
then
\begin{equation*}
\lim_{A\to+\infty} \|\varphi_A - \widehat{f}\|_2 = 0 \\
\qquad \textrm{and} \qquad
\lim_{A\to+\infty} \|\psi_A - f\|_2 = 0
\end{equation*}
\end{itemize}
\end{llem}
\begin{thmproof}
Our objective is the relation
\begin{equation} \label{eq:plancherel:iso:kdim}
\|\widehat{f}\|_2 = \|f\|_2 \qquad (f\in L^1(\R^k)\cap L^2(\R^k)).
\end{equation}
We fix \(f\in L^1(\R^k)\cap L^2(\R^k)\), 
put \(\widetilde{f}(x) = \overline{f(-x)}\),
(clearly \(\widetilde{f} \in L^1(\R^k)\cap L^2(\R^k)\))
and define \(g = f \ast \widetilde{f}\). Then
\begin{equation*}
g(x) 
= \int_{\R^k} f(x-y)\overline{f(-y)}\,dm(y)
= \int_{\R^k} f(x+y)\overline{f(y)}\,dm(y)
= \langle f_x, f \rangle,
\end{equation*}
where the inner product is taken in \(L^2(\R^k)\).

\begin{quotation}
Note that \(\widetilde{f}\) ``negates the direction'' of the variable,
while it gets ``negated back'' in the convolution $g$.
Thus, in the integration is with respect to variables going
in the ``same direction''
\end{quotation}

By generalization (mentioned above) of theorem~9.5 
\(x\to f_x\) is a continuous mapping, and the continuity of 
the inner product, $g$ is a continuous function.
\index{Schwartz}
Schwartz inequality shows that
\begin{equation*}
|g(x)| \leq \|f_x\|_2\cdot \|f_x\|_2 =  \|f\|_2^2
\end{equation*}
so that $g$ is bounded. Also \(g\in L^1(\R^k)\) since
\(f,\widetilde{f}\in L^1(\R^k)\).
Hence we map apply local lemma~\ref{lem:9.8:kdim} 
\begin{equation} \label{eq:gconvh0}
(g\ast h_\lambda)(0) 
= \int_{\R^k} H(\lambda t) \widehat{g}(t)\,dm(t).
\end{equation}
Since $g$ is continuous and bounded, local lemma~\ref{lem:9.9:kdim}
show sthat
\begin{equation} \label{eq:lim-gconvh0}
\lim_{\lambda\to 0} (g \ast h_\lambda)(0) = g(0) = \|f\|_2^2.
\end{equation} 


Theorem~9.2\ich{c}+\ich{d} 
(trivially generalized to \(\R^k\)) shows that 
\begin{equation*}
\widehat{g} 
= \widehat{f} \cdot\,\widehat{\tilde{f}}
= \widehat{f} \cdot\,\overline{\widehat{f}}
= |\widehat{f}|^2
\end{equation*}
and since \(H(\lambda t)\)
increases to $1$ as \(\lambda\to 0\), 
the monotone convergence theorem~1.26 gives
\begin{equation} \label{eq:limint:Hg}
\lim_{\lambda\to 0} \int_{\R^k} H(\lambda t)\widehat{g}\,dm(t)
= \int_{\R^k} \left| \widehat{f}(t)\right|\,dm(t).
\end{equation}

Now
\eqref{eq:gconvh0},
\eqref{eq:lim-gconvh0} and
\eqref{eq:limint:Hg}
shows that \(\widehat{f}\in L^2(\R^k)\) and that
the \eqref{eq:plancherel:iso:kdim} holds.
The part of the work for \(L^1(\R^k)\cap L^2(\R^k)\)) is complete.


Let $Y$ be the space of all Fourier transforms \(\widehat{f}\)
of functions \(f\in L^1(\R^k)\cap L^2(\R^k)\).
By \eqref{eq:plancherel:iso:kdim}, \(Y\subset L^2(\R^k)\).
We claim that $Y$ is dense in \(L^2(\R^k)\),
that is \(Y^\perp = \{0\}\).

The functions 
\begin{equation*}
f_{\alpha,\lambda}(x) = e^{i(\alpha\cdot x)}H(\lambda x)
\qquad \forall \alpha\in\R^k,\; \forall\lambda>0
\end{equation*}
are in \(L^1(\R^k)\cap L^2(\R^k)\). Their Fourier transforms
\begin{equation*}
\widehat{f_{\alpha,\lambda}}(t)
= \int_{\R^k} = e^{i(\alpha\cdot (t-y))}H(\lambda (t-y))\,dm(y)
= \int_{\R^k} = e^{i(\alpha\cdot x)}H(\lambda x)\,dm(x)
= h_\lambda(\alpha-t)
\end{equation*}
are therefore in $Y$. If \(w\in L^2(\R^k)\cap Y^\perp\), it follows that
\begin{equation*}
(h_\lambda \ast \overline{w})(\alpha)
= \int_{\R^k} h_{\lambda}(\alpha-t)\overline{w(t)}\,dm(t) = 0
\qquad \forall \alpha\in\R^k.
\end{equation*}
hence \(w=0\), by local lemma~\ref{lem:9.10:kdim} and therefore
$Y$ is dense in \(L^2(\R^k)\).

We temporary notate \(\widehat{f}\) by \(\Phi f\).
Collecting our result so far, shows that 
\begin{equation*}
\Phi: L^1(\R^k)\cap L^2(\R^k) \longrightarrow Y
\end{equation*}
is an \(L^2(\R^k)\)-isometry whose domain domain and range
are both dense subspaces of \(L^2(\R^k)\).
There is a unique continuous extension of \(\Phi\) from 
the whole \(L^2(\R^k)\) as a domain, and thus \ich{b} holds.

Since \(L^2(\R^k)\) is a complete metric space (as a Hilbert space),
by Lemma~4.16 this extsnion of \(\Phi\) is \emph{onto}
\(L^2(\R^k)\).
In order to show that this extended \(\Phi\) is a
\emph{Hilbert} space isomorphism, we need to show that 
the inner product is maintained. But the inner product is determined
by the norm:
\begin{equation*}
\langle v,w\rangle = 
\left(
\|v+w\|_2^2
- \|v-w\|_2^2
+ i\|v+iw\|_2^2
- i\|v-iw\|_2^2
\right) \bigm/ 4.
\end{equation*}
Hence by \ich{b} 
\begin{equation*}
\forall f,g\in L^2(\R^k),\quad 
\langle f,g\rangle = \langle \widehat{f},\widehat{g}\rangle.
\end{equation*}
and thus \ich{c} holds.

To prove \ich{d}, let 
\begin{equation*}
k_A = \chhi_{[-A,A]^k}.
\end{equation*}
Then \(k_Af \in L^1(\R^k)\cap L^2(\R^k)\) and by definitions
\begin{equation*}
\varphi_A = \widehat{k_A f}.
\end{equation*}
Since \(\lim_{A\to\infty} \|f - k_A f\|_2 = 0\), 
it follows from \ich{b} that 
\begin{equation*}
\lim_{A\to\infty} \|\widehat{f} - \varphi_A\|_2
= \lim_{A\to\infty} \|\widehat{f - k_A f}\|_2 = 0.
\end{equation*}
Similarly 
\begin{equation*}
\lim_{A\to\infty} \|f - \psi_A\|_2
= \lim_{A\to\infty} \|\Phi^{-1}(f - k_A f)\|_2 = 0.
= \lim_{A\to\infty} \|f - k_A f\|_2 = 0.
\end{equation*}
\end{thmproof}

\paragraph{Complex Homomorphisms of \(L^1(\R^k)\).}
Now we generalize theorem~9.23.
\begin{llem}
To every complex homomorphism \(\varphi\) on \(L^1(\R^k)\setminus\{0\}\)
such that
\begin{equation} \label{eq:9.23:kdim:lem}
\varphi(f\ast g) = \varphi(f)\varphi(g) \qquad (\forall f,g\in(L^1(\R^k))
\end{equation}
there corresponds a unique \(t\in\R^k\) such that
\begin{equation} \label{eq:9.23:kdim:beta}
\forall f\in L^1(\R^k),\quad  \varphi(f) = \widehat{f}(t).
\end{equation}
\end{llem}
\begin{thmproof}
By theorem~6.16, there exists a unique
\(\beta \in L^\infty(\R^k)\) such that 
\begin{equation} \label{eq:9.23:kdim:eu:beta}
\varphi(f) = \int_{\R^k} f(x)\beta(x)\,dm(x) \qquad (\forall f\in L^1(\R^k)).
\end{equation}
Looking at \eqref{eq:9.23:kdim:lem}'s left side
\begin{align}
\varphi(f \ast g)
&= \int_{\R^k} (f\ast g)(x)\cdot\beta(x)\,dm(x) \notag \\
&= \int_{\R^k} \beta(x)
     \left(\int_{\R^k} f(x-y)g(y)\,dm(y)\right)\,dm(x) \notag \\
&= \int_{\R^k} g(y)
     \left(\int_{\R^k} f_y(x)\beta(x)\,dm(x)\right)\,dm(y) \notag \\
&= \int_{\R^k} g(y)\varphi(f_y)\,dm(y). \label{eq:9.23:kdim:left}
\end{align}
Looking at the right side
\begin{equation}  \label{eq:9.23:kdim:right}
\varphi(f)\varphi(g) = \varphi(f)\int_{\R^k} g(y)\cdot\beta(y)\,dm(y).
\end{equation}
Combining
\eqref{eq:9.23:kdim:left} and
\eqref{eq:9.23:kdim:right} gives
\begin{equation} \label{eq:9.23:kdim:lr}
\int_{\R^k} g(y)\varphi(f_y)\,dm(y) 
= \varphi(f)\int_{\R^k} g(y)\cdot\beta(y)\,dm(y)
\end{equation}

% Fix \(f\in L^1(\R^k)\) such that \(\varphi(f)\neq 0\). 
Since  \eqref{eq:9.23:kdim:lr} holds for any \(g\in L^1(\R^k)\)
by the uniqueness of \(\beta\) it is wasy to show that
\begin{equation} \label{eq:vfby=vfy}
\varphi(f)\beta(y) = \varphi(f_y) \; \aded(y)
\end{equation}
\begin{quotation}
In the text \cite{RudinRCA87} the above \eqref{eq:vfby=vfy}
is established after fixing $f$ so \(\varphi(f)\neq 0\).
We do not need it so soon, thus it can be used for any \(f\in L^1(\R^k)\).
\end{quotation}
But \(y\to f_y\) is a continuous mapping
(generalizationof theorem~9.5) and \(\varphi\) 
is continuous on  \(L^1(\R^k)\).
Hence \eqref{eq:vfby=vfy}'s right side is continuous
function of \(y\in\R^k)\).
By
picking some \(f\in L^1(\R^k)\) such that \(\varphi(f)\neq 0\). 
and redefining 
\begin{equation*}
\beta(y) = \varphi(f_y) / \varphi(f)
\end{equation*}
then \(\beta\) may get changed on set of measure~0 at most, 
thus \eqref{eq:9.23:kdim:beta} still holds. Now \(\beta\) is continuous
and now \eqref{eq:vfby=vfy} holds for \emph{all} \(y\in\R^k\).
By replacing $y$ by \(x+y\) and $f$ by \(f_x\)
in \eqref{eq:vfby=vfy}, we obtain
\begin{equation*}
\varphi(f)\beta(x+y)
= \varphi(f_{x+y})
= \varphi((f_x)_y)
= \varphi(f_x)\beta(y) 
= \varphi(f)\beta(x)\beta(y) .
\end{equation*}
By picking (again) $f$ such that \(\varphi(f)\neq 0\)
we get
\begin{equation} \label{eq:9.23:kdim:beta:hom}
\forall x,y\in\R^k,\quad \beta(x+y) = \beta(x)\beta(y).
\end{equation}
Since \(\beta\) is not identically zero, 
\eqref{eq:9.23:kdim:beta:hom}implies \(\beta(0)=1\)
and the continuity of \(\beta\) shows hat the is a \(\delta>0\)
such that
\begin{equation} % \label{eq:9.23:kdim:beta:delta}
\int_{[0,\delta]^k}\beta(y)\,dy = c \neq 0.
\end{equation}
Then
\begin{equation} \label{eq:9.23:kdim:beta:delta}
c\beta(x)
= \int_{[0,\delta]^k}\beta(x)\beta(y)\,dy
= \int_{[0,\delta]^k}\beta(x + y)\,dy
= \int_{[x,x+\delta]^k}\beta(y)\,dy.
\end{equation}
Since \(\beta\) is continuous, the last integral is a differentiable
function of $x$ in each axis.
hence \eqref{eq:9.23:kdim:beta:delta} shows that \(\beta\)
is differentiable.
Differentiating \eqref{eq:9.23:kdim:beta:hom}
with respect to each axis of $y$, then put \(y=0\);
the result is
\begin{equation} \label{eq:9.23:kdim:diffeq}
\frac{\partial\beta(x)}{\partial x_j} = A_j\beta(x), 
\qquad A_j = \frac{d\beta(0)}{dx_j}
\qquad (1\leq j \leq k)
\end{equation}
Hence the partial derivatives of \(\beta(x)e^{-A_jx_j}\) are
\begin{align*}
\frac{\partial}{\partial x_j} \left(\beta(x)e^{-A_jx_j}\right)
&= \left(\frac{\partial}{\partial x_j} \beta(x)\right)e^{-A_jx_j}
   + -A_j\beta(x)e^{-A_jx_j} \\
&= \left(\left(\frac{\partial}{\partial x_j} \beta(x) 
              \right) - A_j\beta(x)\right)e^{-A_jx_j}
= 0
\end{align*}
for \(1\leq j \leq k\). Similarly the partial derivatives of
\begin{equation*}
\beta(x)\prod_{j=1}^ke^{-A_jx_j}
= \beta(x)e^{-\sum_{j=1}^k A_jx_j} = \beta(x)e^{-A\cdot x}
\end{equation*}
are
\begin{align*}
\frac{\partial}{\partial x_j} \left(\beta(x)e^{-A\cdot x}\right)
&= \left(\frac{\partial}{\partial x_j} \beta(x)\right)e^{-A\cdot x})
   -A_j \beta(x)e^{-A\cdot x} \\
&= \left(\left(\frac{\partial}{\partial x_j} \beta(x)\right)
        -A_j \beta(x)\right)e^{-A\cdot x}
= 0.
\end{align*}
Hence \(\beta(x)e^{-A\cdot x}\)
is constant in respect of each of \(\{x_j\}_{j=1}^k\).
Since \(\beta(0)=1\) we obtain
\begin{equation*}
\beta(x) = e^{A\cdot x} \qquad (A\in\C^k,\; x\in\R^k).
\end{equation*}
Since \(\beta\in L^\infty(\R^k)\) it is bounded (also continuous)
and by looking at each axis separately, we see that
\(\forall j\in\N_k,\;\Re(A_i)=0\). Thus $A$ consists
of pure imaginary numbers, and there exist \(t\in R^k\)
such that 
\begin{equation*}
\beta(x) = e^{-it\cdot x}.
\end{equation*}
By looking back  at \eqref{eq:9.23:kdim:eu:beta} we get the desired
\begin{equation*}
\varphi(f) = \int_{\R^k}f(x)e^{-it\cdot x}\,dm(x) = \widehat{f}(t).
\end{equation*}
evaluation of a Fourier transform.
\end{thmproof}


%%%%%%%%%%%%%% 15
\begin{excopy}
If \(f\in L^1(\R^k)\), $A$ is a linear operator on \(\R^k\), 
and \(g(x) = f(Ax)\),
how is \(\Hat{g}\) related to \(\Hat{f}\)?
If $f$ is invariant under rotations, i.e., if \(f(x)\) depends only 
on the euclidean distance of $x$ from the origin, prove that the same 
is true for \(\Hat{f}\).
\end{excopy}

\begin{quote}
This is a generalization of theorem~9.2\ich{e}.
But then we should  either
\textbf{(i)} think of $A$ as \(1/\lambda\),
or 
\textbf{(ii)} 
think of $A$ as \(\lambda\) \emph{and} swap between $f$ and $g$.
\end{quote}

If \(\rank{A} < \dim(A) = k\) then the mapping is not onto 
and its image \((A(\R^k)\) is a subspace of dimension \(<k\)
so \(m(A(\R^k)) = 0\) in \(\R^k\).
Thus $g$ depends on values of $f$ \emph{only} on a set of measure~$0$.
So we cannot get dependency of \(\widehat{g}\) on \(\widehat{f}\)

Now we may assume that $A$ is regular, that is, it is invertible.
We need to use some change of variable technique.
We can use either theorem~10.9 of \cite{RudinPMA85}
or theorem theorem~7.26 of our current~\cite{RudinRCA87}.
The former deals with \emph{continuous} functions
while the latter with \emph{measurable} functions but only 
of \emph{positive} values. We will use the first option.

To use theorem~10.9 of \cite{RudinPMA85}, 
let us first assume that \(f\in C_c(\R^k)\).
We have the inverse \(A^{-1}\).
Note that in our Euclidean real case, $A$ can be represented 
as an \(k\times k\) matrix $A$.
\index{Jacobian}
The Jacobian matrices  of the \(\R^k\)-mappings are
\(J_A=A\) and \(J_{A^{-1}}=A^{-1}\), the equalities hold
since $A$ is linear (so we often drop the '$J$' symbol), 
and their determinants 
which are constant since $A$ is linear and satisfy
\begin{equation*}
\left|J_{A^{-1}}\right|\cdot\left|J_A\right| = |A^{-1}|\cdot |A| = 1.
\end{equation*}
We note that there exists a linear mapping \(A^*\)
such that 
\(\langle Av,w\rangle =  \langle v, A^*w\rangle\)
for all \(v,w\in\R^k\).
See \cite{Lang94} chapter~\textsf{XIII} sections~\S5 and~\S7.

Since \(g(x) = f(Ax)\), equivalently we have \(g(A^{-1}x) = f(x)\).
\begin{align}
\widehat{f}(t) 
&= \int_{\R^k} g\left(A^{-1}y\right)e^{-i(t\cdot y)}\,dm(y) \notag \\
&= \int_{\R^k} g\bigl(A^{-1}A(x)\bigr)e^{-i(t\cdot Ax)} |J_A|\,dm(x) 
   \label{eq:fourier:jacobian:c} \\
&= |A|\int_{\R^k} g(x)e^{-i(A^*t\cdot x)} \,dm(x) 
   \label{eq:fourier:innerprod} \\
&= |A|\widehat{g}(A^*t).
\end{align}
In \eqref{eq:fourier:jacobian:c} the substitution \(y=Ax\) is used.
See also \cite{EdwFA}~5.15.4.

We also note that \(A^* = A^T\) the transpose. To see this let
\begin{equation*}
v = (\seq{v}{k}) \qquad w = (\seq{w}{k})
\end{equation*}
be arbitrary vectors in \(\R^k\). Now 
\begin{alignat*}{2}
(Av)_j &= \sum_{m=1}^k A_{j,m}v_m
\qquad
  & \langle Av,w \rangle 
     &= \sum_{j=1}^k (Av)_jw_j 
     =\sum_{j=1}^k  \left(\sum_{m=1}^k A_{j,m}v_m\right)w_j \\
(A^Tw)_j &= \sum_{m=1}^k A_{m,j}w_m
\qquad
  & \langle v,A^Tw \rangle 
     &= \sum_{j=1}^k v_jA^Tw_j 
     =\sum_{j=1}^k v_j \sum_{m=1}^k A_{m,j}w_m \\
\end{alignat*}
Looking at the double-sums , each with \(k^2\) terms, 
we see that they are just permutations of each other.
Hence 
\begin{equation*}
\langle Av,w\rangle = \langle v, A^*w\rangle
\end{equation*}
Hence \(A^* = A^T\) as matrices.
See also \cite{Herstein1975}~Theorem~6.10.2.

Now since \(C_c(\R^k)\) are dense in \(L^p(\R^k)\)
and we can converge to and \(f\in L^p(\R^k)\)
by functions \(g\in C_c(\R^k)\) 
such that \(|g(x)|\leq |f(x)|\,\aded(x)\), 
by Lebesgue dominated convergence theorem~1.34
\begin{equation}  \label{eq:fourier:jacobian}
\widehat{f}(t) = |A|\,\widehat{g}(A^*t) \qquad (\;g(x)=f(Ax)\;)
\end{equation}
or equivalently
\begin{equation}  \label{eq:fourier:jacobian:alt}
\widehat{g}(t) = |A|^{-1}\,\widehat{f}\left(\left(A^*\right)^{-1}t\right) 
\qquad (\;g(x)=f(Ax)\;)
\end{equation}
for every \(f\in L^1(\R^k)\).

\paragraph{Dependence on Euclidean distance.}
Now assume that \(f(x)\) depends on \(\|x\|_2\).
Then for any rotations mapping $A$, we have \(f(x) = f(Ax)\).
We also note that if $A$ is a rotation, then \(A^T = A^{-1}\) 
and \(|A|=1\) and  all the eigenvalues have absolute value~$1$.
See \cite{Herstein1975}~Lemma~6.10.5.
By what we have shown, 
\begin{equation*}
\widehat{f}(x) = |A|\,\widehat{f}(A^*x) = \widehat{f}(A^Tx)
\end{equation*}
for all \(A^T\) such that $A$ is a rotation.
The mapping \(A\to A^T=A^{-1}\) is (bijection) 
\emph{onto} automorphism of the group of rotations.
Hence \(\widehat{f}\) similarly depends only on the Euclidean distance.

% theorem 10.9 \cite{RudinPMA85}

%%%%%%%%%%%%%% 16
\begin{excopy}
The
\index{Laplacian}
\emph{Laplacian} 
of a function $f$ on \(\R^k\) is 
\begin{equation*}
\Delta f = \sum_{j=1}^k \frac{\partial^2f}{\partial x_j^2},
\end{equation*}
provided that the partial derivatives exist. What is the relation between 
\(\Hat{f}\) and \(\Hat{g}\) if \(g = \Delta f\)
and all necessary integrability conditions are satisfied?
It is clear that the Laplacian commutes with translations.
Prove that it also commutes with rotations, i.e. that 
\begin{equation*}
 \Delta(f \circ A) =  \Delta(f) \circ A
\end{equation*}
whenever $f$ has continuous second derivatives 
and $A$ is a rotation of \(\R^k\).
(Show that it is enough to do this under the additional assumption 
that $f$ has compact support.)
\end{excopy}

We first compute the 
\index{Laplacian}
Laplacian of the exponential 
function we convolute with in the Fourier transform.
\begin{align*}
\Delta e^{-i(x\cdot t)}
&= \sum_{j=1}^k \frac{\partial^2}{\partial x_j^2} e^{-i(x\cdot t)} 
 = \sum_{j=1}^k \frac{\partial}{\partial x_j} (-it_j) e^{-i(x\cdot t)} 
 = \sum_{j=1}^k (-it_j) \frac{\partial}{\partial x_j} e^{-i(x\cdot t)} \\
&= \sum_{j=1}^k (-it_j)^2 e^{-i(x\cdot t)} 
 = e^{-i(x\cdot t)} \sum_{j=1}^k -t_j^2  \\
&= -|t|^2e^{-i(x\cdot t)}
\end{align*}
Now assume \(f\in C_c^2(\R^k\)). 
That is, $f$ is sufficiently differentiable 
and it has (and so have its  derivatives) a compact support. Then
using the inversion theorem (local lemma~\ref{lem:9.11:kdim}) we get
\begin{equation*}  \label{eq:laplacian:fourier:inv}
\Delta f(x)
 = \Delta\int_{\R^k} \widehat{f}(t) e^{i(x\cdot t)}\,dm(t) 
 = \int_{\R^k} \widehat{f}(t) \Delta e^{i(x\cdot t)}\,dm(t) 
 = \int_{\R^k} \left(-|t|^2\right)\widehat{f}(t) e^{i(x\cdot t)}\,dm(t)
\end{equation*}
The inversion theorem also implies uniqueness of the Fourier transform.
That is if \(\widehat{g} = \widehat{h}\) then \(g=h\)
or similarly, more suitable to our case, if
\begin{equation}
\int_{\R^k}\widehat{g}(t)e^{i(x\cdot t)}\,dm(x) =
\int_{\R^k}\widehat{h}(t)e^{i(x\cdot t)}\,dm(x)
\end{equation}
then \(\widehat{g}(t)=\widehat{h}(t)\;\aded(t)\).
Hence by \eqref{eq:laplacian:fourier:inv} we have
\begin{equation} \label{eq:laplacian:fourier}
\widehat{\Delta f}(t) = -|t|^2\widehat{f}(t).
\end{equation}
By utilizing convergence theorems, \eqref{eq:laplacian:fourier}
can applied to every \(f\in L^1(\R^k)\cap C^2(\R^k)\),
that is to smooth functions not necessarily with compact support.

\paragraph{Commuting with rotation.}
Let us first explore the expression \(\Delta(f \circ A)\)
directly without referring to the Fourier transform.
We will do it merely to get some intuition for 
the complexity involved --- or later the saved complexity.
Let $A$ be a rotation operator. Since $A$ is linear
\begin{equation*}
 \frac{\partial^2}{\partial x_j^2}A(x) = 0 \qquad (\forall j \in N_k).
\end{equation*}
Temporarily fix $j$ to compute partial derivatives.
\begin{equation*}
\frac{\partial}{\partial x_j} (f \circ A)(x)
= \frac{\partial}{\partial x_j} (f \circ A)(x) 
= \left(\frac{\partial}{\partial x_j}(f \circ A)(x)\right)
   \cdot
   \frac{\partial A(x)}{\partial x_j}
\end{equation*}
We proceed to second derivative
\begin{align*}
\frac{\partial^2}{\partial x_j^2} (f \circ A)(x)
&=   \frac{\partial}{\partial x_j} 
     \left(
      \left(\frac{\partial}{\partial x_j}(f \circ A)(x)\right)
      \cdot
      \frac{\partial A(x)}{\partial x_j}
     \right) \\
&=    \frac{\partial^2 (f \circ A)(x)}{\partial x_j^2}
      \cdot
      \frac{\partial A(x)}{\partial x_j} 
      +
       \frac{\partial (f \circ A)(x)}{\partial x_j}
       \cdot
       \frac{\partial^2 A(x)}{\partial x_j^2}
       \\
&=    \frac{\partial^2 (f \circ A)(x)}{\partial x_j^2}
      \cdot
      \frac{\partial A(x)}{\partial x_j} 
\end{align*}

Back to the Laplacian
\begin{equation*}
\Delta(f \circ A)(x) 
= \sum_{j=1}^k 
     \frac{\partial^2}{\partial x_j^2} (f \circ A)(x) 
= \sum_{j=1}^k 
      \frac{\partial^2 (f \circ A)(x)}{\partial x_j^2}
      \cdot
      \frac{\partial A(x)}{\partial x_j} 
\end{equation*}

Now looking at Fourier transforms.
We will use the that \(|A|=1\) when $A$ is a rotation.
First transform the left side of the desired equality.
By \eqref{eq:laplacian:fourier} and \eqref{eq:fourier:jacobian:alt}
\begin{equation} \label{eq:fourier:laplacian:rot:left}
 \widehat{\Delta(f \circ A)}(t) 
 = -|t|^2 \widehat{(f \circ A)}(t)
 = -|t|^2 |A|^{-1} \widehat{f}\left(\left(A^*\right)^{-1}t\right)
 = -|t|^2 \widehat{f}\left(\left(A^*\right)^{-1}t\right)
\end{equation}
Transform the right side of the desired equality using
\eqref{eq:fourier:jacobian} 
and again  \eqref{eq:laplacian:fourier} 
 with \eqref{eq:fourier:jacobian:alt}
\begin{align}
 \widehat{\bigl((\Delta f) \circ A\bigr)}(t) 
&= |A|^{-1}\,\widehat{(\Delta f)}\left(\left(A^*\right)^{-1}t\right) 
 = |A|^{-1}
   \cdot 
   \left(-\left|\left(A^*\right)^{-1}t\right|^2\right)
   \cdot
   \widehat{f}\left(\left(A^*\right)^{-1}t\right) \notag 
   \\
   \label{eq:fourier:laplacian:rot:right}
&=  -|t|^2\cdot \widehat{f}\left(\left(A^*\right)^{-1}t\right) 
\end{align}
As we did in previous exercise, 
the implied uniqueness by the inversion theorems applied to 
\eqref{eq:fourier:laplacian:rot:left} and
\eqref{eq:fourier:laplacian:rot:right}
gives the desired equality
\begin{equation*}
 \Delta(f \circ A) = \Delta(f) \circ A\,.
\end{equation*}

\paragraph{Compact support.} Any function $f$ in \(L^1(\R^k)\)
and in particular any sufficiently smooth function in \(L^1(\R^k)\)
can be approximated in \(L^1(\R^k)\) by function \(g\in C_c^2(\R^k)\).
We can also ensure that \(|g(x)|\leq |f(x)|\;\aded\).
Hence 
the arguments above could be applied only to functions with compact support
and later by Lebesgue's dominated convergence theorem~1.34
be generalized to \(L^1(\R^k)\cap C^2(\R^k)\).


%%%%%%%%%%%%%% 17
\begin{excopy}
Show that every Lebesgue measurable character of \(\R^1\) is continuous.
Do the same for \(\R^k\).
(adapt part of the proof of Theorem~9.23.)
Compare with Exercise~18.
\end{excopy}

Let \(\varphi\) be a character of \(\R^k\).
Then \(\varphi\chhi_{\restriction[0,1]}\in L^1(\R)\)
and by theorem~7.11 there exists (almost everywhere) \(b\in(0,1)\) such that
\begin{equation*}
c := \int_0^b \varphi(y)\,dy \neq 0.
\end{equation*}
Now for any \(x\in\R\)
\begin{equation*}
c \varphi(x) 
= \int_0^b \varphi(y)\varphi(x)\,dx = 
= \int_0^b \varphi(y + x)\,dx = 
= \int_x^{b+x} \varphi(y + x)\,dx
\end{equation*}
Hence 
\begin{equation*}
\varphi(x) = \frac{1}{c} \int_x^{b+x} \varphi(y + x)\,dx
\end{equation*}
is continuous.

\begin{quote}
The generalization of continuity cannot be derived
by separate variable continuity, as the example
\(f(0,0)=0\) and otherwise \(f(x,y)=xy/(x^2+y^2)\) shows
(see \cite{Gelb1996}~Chapter~9).
\end{quote}

Now assume \(\varphi\) be a character of \(\R^k\).
Clearly \(\varphi(0)=1\).
For every \(a\in \R^k\) and \(j\in\N_k\) we can define,
by binding to ``\(a \setminus a_j\)'',
a character on \(\R^1\) by
\begin{equation*}
\varphi_{a,j}(t) 
= \varphi\bigl((a_1,\ldots,a_{j-1},t,a_{j+1},\ldots,a_k)\bigr).
\end{equation*}
Hence \(\varphi\) is continuous as a function of each axis.
Let \(\epsilon>0\) and pick \(\delta>0\) such that
for all \(j\in \N_k\), if \(|t|<\delta\) then
\begin{equation*}
\left|\varphi\left(
   \overbrace{0,\ldots,0}^{j-1\;\textrm{times}},
   t,
   \overbrace{0,\ldots,0}^{n-j\;\textrm{times}}\right) - 1
\right| < \epsilon.
\end{equation*}
Now if \(\|x-a\|_2 < \delta\) then
\begin{align*}
|\varphi(x) - \varphi(a)|
&\leq \sum_{j=1}^k 
  \left|
   \varphi(x_1,\ldots        ,x_j,a_{j+1},\ldots,a_k)
   -
   \varphi(x_1,\ldots,x_{j-1},a_j,\ldots,a_k) \right| \\
&= \sum_{j=1}^k 
  \left|\left((\varphi\left(
           \overbrace{0,\ldots,0}^{j-1\;\textrm{times}},
           \,x_j - a_j,\,
           \overbrace{0,\ldots,0}^{n-j\;\textrm{times}}\right)
           - 1
        \right)
   \varphi(x_1,\ldots,x_{j-1},a_j,\ldots,a_k)\right| \\
&\leq k\epsilon.
\end{align*}
Hence \(\varphi\) continuous at $a$ and thus continuous in \(\R^k\).

%%%%%%%%%%%%%% 18
\begin{excopy}
Show (with the aid of the Hausdorff maximality theorem) that there exist real
\emph{discontinuous} functions $f$ on \(\R^1\) such that 
\begin{equation} \label{eq:ex9.18}
f(x + y) = f(x) + f(y)
\end{equation}
for all $x$ and \(y\in \R^1\).

Show that if \eqref{eq:ex9.18} holds and $f$ is Lebesgue measurable then
$f$ is continuous.

Show that if \eqref{eq:ex9.18} holds and the graph of $f$ is
not dense in the plane, then $f$ is continuous.

Find all continuous functions which satisfy \eqref{eq:ex9.18}
\end{excopy}

From \eqref{eq:ex9.18} it is easy that
\begin{equation} \label{eq:ex9.18:Q}
\forall x\in\R,\;\forall q\in\Q:\quad f(qx) = q\cdot f(x).
\end{equation}

\paragraph{Discontinuous example.}
Take 
\index{Hamel}
Hamel base $H$ of \(\R\) over \(\Q\) starting with \(1,\sqrt{2}\in H\).
Define 
\(f(1)=1\) and \(f(h)=0\) for all \(h\in H \setminus\{1\}\).
Clearly $f$ can be extended to a span of any finite sub-base.
By Zorn Lemma $f$ can be extended to the whole \(\R\)
and obviously $f$ is discontinuous while its additive rule holds.

\paragraph{Measurable Character.}
Assume that $f$ is measurable.
Define
\begin{equation*}
\varphi(x) = \exp\left(i\Re\bigl(f(x)\bigr)\right).
\end{equation*}
It is easy to see that \(\varphi\) is a measurable character. 
By previous exercise
\(\varphi\) is continuous and there exist 
some (unique) \(t\in\R\) such that \(\varphi(x) = \exp(itx)\)
for all \(x\in \R\). Hence
\begin{equation*}
\Re\bigl(f(x)\bigr) = tx + 2\pi k_x \qquad (k_x \in \Z).
\end{equation*}
for all \(x\in\R\).
Assume by negation that \(k_w\neq 0\) for some \(w\in\R\).
Let 
\begin{equation} \label{eq:ex9.18:Z}
d = \frac{\Re\bigl(f(w)\bigr) - tw}{2\pi} \in \Z
\end{equation}
By \eqref{eq:ex9.18:Q} we have
\begin{equation*}
\Re\bigl(f(w/(2d))\bigr) = \Re\bigl(f(w)\bigr) \,\bigm/\, (2d)
\end{equation*}
and so
\begin{equation*}
\frac{\Re\bigl(f(w/(2d))\bigr) - tw/(2d)}{2\pi}
= \frac{\Re\bigl(f(w))\bigr) - tw}{4\pi d}
= \half \notin \Z
\end{equation*}
which is a contradiction to \eqref{eq:ex9.18:Z}.
Therefore \(\Re(f(x)) = tx\) and so \(\Re\circ f\) is continuous.

Similarly we can show that \(\Im\circ f\) is continuous, 
therefore $f$ is continuous.


\paragraph{Dense Graph}
If $f$ is \emph{not} continuous, then by what we just saw it is not linear
and we can find
\begin{equation*}
v_1 = \bigl(x_1,f(x_1)\bigr)
\qquad
v_2 = \bigl(x_2,f(x_2)\bigr)
\end{equation*}
such that \(\{v_1,v_2\}\) are linearly independent 
in the vector space \(\R^2\) over \R\ and so they span it.
Pick \((x,y)\in\R^2\), and let \(a_1,a_2\in\R\) be such that
\((x,y)=a_1v_1+a_2v_2\).
We can find two rational sequences \(\{q_{jk}\}_{k=1}^\infty\)
such that \(\lim_{k\to\infty} q_{jk} = a_j\) for \(j=1,2\).
Hence
\begin{equation*}
\lim_{k\to\infty} q_{1k}v_1 + q_{2k}v_2 = a_1v_1+a_2v_2 = (x,y).
\end{equation*}
Since \(q_{1k}v_1 + q_{2k}v_2\) are in the graph of $f$
it is dense in \(\R^2\).

\paragraph{Description of continuous functions.}
If $f$ is continuous, then since \(f(q) = q\cdot f(1)\) for all \(q\in\Q\)
by continuity we also have \(f(x) = x\cdot f(1)\) for all \(x\in\R\).
Hence the continuous functions that satisfy \eqref{eq:ex9.18}
are exactly the linear functions.


%%%%%%%%%%%%%% 19
\begin{excopy}
Suppose $A$ and $B$ are measurable subsets of \(\R^1\), 
having finite positive measure.
Show that the convolution \(\chhi_A \ast \chhi_B\) is continuous 
and not identically zero. Use this to prove that \(A+B\) contains a segment.

(A different proof was suggested in Exercise~5, Chap.~7.)
\end{excopy}

Put \(h = \chhi_A \ast \chhi_B\).
Fix \(\epsilon>0\) and by theorem~3.14, pick \(g\in C_c(\R)\)
such that \(\|\chhi_A - g\|_1 < \epsilon/m(B)\).
% Define \(g_a(x) = g(x+a)\) and cleary \(g_a\) are uniformly continuous.
Cleary $g$ is uniformly continuous.
Pick \(\delta>0\) such that
\(|g(s)-g(t)| < \epsilon/m(B)\) whenever \(|s-t|<\delta\).
If \(|s-t|<\epsilon/m(B)\) then
\begin{align*}
h(s) - h(t)
&= (\chhi_A \ast \chhi_B)(s) - (\chhi_A \ast \chhi_B)(t)  \\
&= \int \chhi_A(s-x)\chhi_B(x)\,dm(x) -
   \int \chhi_A(t-x)\chhi_B(x)\,dm(x) \\
&= \int \bigl(\chhi_A(s-x) - \chhi_A(t-x)\bigr)\chhi_B(x)\,dm(x) \\
&= \int_B \chhi_A(s-x) - \chhi_A(t-x)\,dm(x)
\end{align*}
Now
\begin{eqnarray*}
|h(s) - h(t)|
&=& \left| \int_B \chhi_A(s-x) - \chhi_A(t-x)\,dm(x) \right| \\
&\leq& \int_B |\chhi_A(s-x) - \chhi_A(t-x)|\,dm(x) \\
&\leq& 
   \int_B |\chhi_A(s-x) - g(s-x)|\,dm(x) +
   \int_B |g(s-x) - g(t-x)|\,dm(x) + \\
&& \int_B |g(t-x) - \chhi_A(t-x)|\,dm(x) \\
&\leq& 3\epsilon.
\end{eqnarray*}
Hence $h$ is (uniformly!) continuous.

%%%%%%%%%%%%%%%%%
\end{enumerate}

 \setcounter{chapter}{9}  %%%%%%%%%%%%%%%%%%%%%%%%%%%%%%%%%%%%%%%%%%%%%%%%%%%%%%%%%%%%%%%%%%%%%%%%
%%%%%%%%%%%%%%%%%%%%%%%%%%%%%%%%%%%%%%%%%%%%%%%%%%%%%%%%%%%%%%%%%%%%%%%%
%%%%%%%%%%%%%%%%%%%%%%%%%%%%%%%%%%%%%%%%%%%%%%%%%%%%%%%%%%%%%%%%%%%%%%%%
%chapter 10
\chapterTypeout{Elementary Properties of Holomorphic Functions}

\newcommand{\itwopi}{\frac{1}{2\pi}}
\newcommand{\itwopii}{\frac{1}{2\pi i}}

%%%%%%%%%%%%%%%%%%%%%%%%%%%%%%%%%%%%%%%%%%%%%%%%%%%%%%%%%%%%%%%%%%%%%%%%
%%%%%%%%%%%%%%%%%%%%%%%%%%%%%%%%%%%%%%%%%%%%%%%%%%%%%%%%%%%%%%%%%%%%%%%%
\section{Notes}

%%%%%%%%%%%%%%%%%%%%%%%%%%%%%%%%%%%%%%%%%%%%%%%%%%%%%%%%%%%%%%%%%%%%%%%%
\subsection{Theoerm 10.6 --- Power Series is Holomorphic}

In the proof of theorem~10.6 the following equality
\begin{equation} \label{eq:thm:10.6}
\left[ \frac{z^n - w^n}{z-w} - nw^{n-1} \right]
= (z-w)\sum_{k=1}^{n-1} kw^{k-1} z^{n-k-1}
\end{equation}
is used for \(n\geq 2\) and \(z\neq w\). Let's work it out.
In order to compute the fraction we start with
\begin{align*}
(z-w)\sum_{k=0}^{n-1} w^k z^{n-k-1}
&=  \left(\sum_{k=0}^{n-1} w^k z^{n-k}\right)
  - \left(\sum_{k=0}^{n-1} w^{k+1} z^{n-k-1}\right) \\
&=  \left(\sum_{k=0}^{n-1} w^k z^{n-k}\right)
  - \left(\sum_{k=1}^{n} w^{k} z^{n-k}\right) \\
&= z^n + \left(\sum_{k=0}^{n-1} \left(w^k z^{n-k} - w^k z^{n-k}\right)\right) - w^n\\
&= z^n - w^n
\end{align*}
Hence
\begin{equation} \label{eq:thm:10.6:frac}
\frac{z^n - w^n}{z-w} = \sum_{k=0}^{n-1} w^k z^{n-k-1}
\end{equation}

Finally,
\begin{align*}
(z-w)\sum_{k=1}^{n-1} kw^{k-1} z^{n-k-1}
&= \left(\sum_{k=1}^{n-1} kw^{k-1} z^{n-k}\right) -
   \left(\sum_{k=1}^{n-1} kw^{k} z^{n-k-1}\right) \\
&= \left(\sum_{k=0}^{n-2} (k+1)w^{k} z^{n-k-1}\right) -
   \left(\sum_{k=1}^{n-1} kw^{k} z^{n-k-1}\right) \\
&= (0-1)w^0z^{n-1} +
   \left(\sum_{k=1}^{n-2} (k_1-k)w^{k} z^{n-k-1}\right) -
   (n-1)w^{n-1}z^0 \\
&= z^{n-1} + \left(\sum_{k=1}^{n-2} w^kz^{n-k-1}\right) + w^{n-1} - nw^{n-1} \\
&= \left(\sum_{k=0}^{n-1} w^kz^{n-k-1}\right) - nw^{n-1} \\
&= \frac{z^n - w^n}{z-w} - nw^{n-1}
\end{align*}
We used \eqref{eq:thm:10.6:frac} in the last equality,
hence \eqref{eq:thm:10.6} holds.


%%%%%%%%%%%%%%%%%%%%%%%%%%%%%%%%%%%%%%%%%%%%%%%%%%%%%%%%%%%%%%%%%%%%%%%%
\subsection{Theoerm 10.7 --- Geometric Equality}

The proof of Theoerm 10.7 uses the following equality
\begin{equation*}
\sum_{n=0}^\infty \frac{(z - a)^n}{\left(\varphi(\zeta) - a\right)^{n+1}}
  = \frac{1}{\varphi(\zeta) - a}.
\end{equation*}
Let us derive it.
\begin{align*}
\sum_{n=0}^\infty \frac{(z - a)^n}{\left(\varphi(\zeta) - a\right)^{n+1}}
 &= \frac{1}{\varphi(\zeta) - a}
  \sum_{n=0}^\infty \left(\frac{z - a}{\varphi(\zeta) - a}\right)^n
  = \frac{1}{\varphi(\zeta) - a}
    \cdot
    \frac{1}{1 - \frac{z - a}{\varphi(\zeta) - a}} \\
 &= \frac{1}{\varphi(\zeta) - a) - (z - a)}
  = \frac{1}{\varphi(\zeta) - a}.
\end{align*}

%%%%%%%%%%%%%%%%%%%%%%%%%%%%%%%%%%%%%%%%%%%%%%%%%%%%%%%%%%%%%%%%%%%%%%%%
\subsection{Theoerm 10.10 --- Winding Number Equality}

The proof of Theoerm~10.0 defines
\begin{equation*}
 \varphi(t) = \exp\left\{\int_\alpha^t \frac{\gamma'(s)}{\gamma(s) - z}ds\right\}
\end{equation*}
and later claims that the derivative of \(\varphi/(\gamma - z)\)
is zero in \(V = [\alpha, \beta] \setminus S\) where $S$ is where \(\gamma\)
is not differentiable.
Let us show it in $V$.
Put
\begin{equation*}
h(t) = \int_\alpha^t \frac{\gamma'(s)}{\gamma(s) - z}ds
\end{equation*}
so \(\varphi(t) = \exp(h(t))\) and
\(h'(t) = \frac{\gamma'(t)}{\gamma(t) - z}\)
and \(\varphi'(t) = h'(t)\varphi(t)\).\\
Now \(\frac{d}{dt} \frac{\varphi(t)}{\gamma(t) - z} = N/D\)
where \(D=(\gamma(t) - z)^2\) and
\begin{equation*}
  N = \varphi'(t)(\gamma(t) - z) - \varphi(t)\gamma'(t) 
   = h'(t)e^{h(t)}(\gamma(t) - z) - e^{h(t)}\gamma'(t) 
    = e^{h(t)}\gamma'(t) - e^{h(t)}\gamma'(t) = 0.
\end{equation*}



%%%%%%%%%%%%%%%%%%%%%%%%%%%%%%%%%%%%%%%%%%%%%%%%%%%%%%%%%%%%%%%%%%%%%%%%
\subsection{Theoerm 10.15 --- Cauchy Fromula in Convex Region}

Let us work out the last derivation in the proof of theoerm~10.15.
\begin{equation*}
0 = \itwopii\int_\gamma g(\xi)\,d\xi
= \itwopii\int_\gamma \frac{f(\xi)-f(z)}{\xi - z}\,d\xi
\end{equation*}
Hence
\begin{equation*}
\itwopii\int_\gamma \frac{f(\xi)}{\xi - z}\,d\xi
= \frac{f(z)}{2\pi i}\int_\gamma \frac{d\xi}{\xi - z}\,d\xi
= f(z)\cdot\Ind_\gamma(z)
\end{equation*}


%%%%%%%%%%%%%%%%%%%%%%%%%%%%%%%%%%%%%%%%%%%%%%%%%%%%%%%%%%%%%%%%%%%%%%%%
\subsection{Theoerm 10.25 --- Inequality}

Let us show the inequality used in the proof of theoerm~10.25.

Given a polynomial \(P(x) = \sum_{j=0}^n a_nz^n\) where \(a_n=1\)
and
\begin{equation} \label{eq:thm10.25:r}
r > 1 + 2|a_0| + \sum_{j=1}^{n-1} |a_j|
\end{equation}
we need to show that
\begin{equation} \label{eq:thm10.25}
\left| P(re^{i\theta})\right| > |P(0)| = |a_0|.
\end{equation}
By \eqref{eq:thm10.25:r} we have
\begin{equation*}
r^n
> r^n + 2|a_0|r^n + \sum_{j=1}^{n-1} |a_j|r^n
> 2|a_0| + \sum_{j=1}^{n-1} |a_j|r^j
\end{equation*}
In the last inequality we used the fact that \(r>1\).
Hence
\begin{equation*}
r^n - \sum_{j=0}^{n-1} |a_j|r^n > |a_0|.
\end{equation*}
Now clearly
\begin{equation*}
\left| P(re^{i\theta})\right|
= \left| r^ne^{ni\theta} + \sum{j=1}^{n-1} a_j r^j e^{ji\theta}\right|
\geq r^n - \sum{j=1}^{n-1} |a_j| r^j > |a_0| = |P(0)|.
\end{equation*}
Hence \eqref{eq:thm10.25} is true.

%%%%%%%%%%%%%%%%%%%%%%%%%%%%%%%%%%%%%%%%%%%%%%%%%%%%%%%%%%%%%%%%%%%%%%%%
\subsection{Theoerm 10.26 --- Deriving the Estimation}

Let's finalize the derivation in the proof of theoerm~10.26.
By Theoerm~10.22 for
\begin{equation*}
f(z) = \sum_{n=0}^\infty c_n(z-a)^n
\end{equation*}
we have
\begin{equation*}
\sum_{n=0}^\infty |c_n|^2 r^{2n}
= \frac{1}{2\pi}\int_0^{2\pi} \left|f(a+re^{it})\right|^2\,d\theta \leq M^2
\end{equation*}
for any \(r<R\). Hence \(|c_n|r^n < M\) for all \(r<R\) and for all $n$.
Thus \(|c_n| < M/R^n\) for all $n$. Now clearly
\begin{equation*}
f^{(n)}(a) = n!c_n
\end{equation*}
and so
\begin{equation*}
\left|f^{(n)}(a)\right| = n!|c_n| \leq n!M/R^n.
\end{equation*}


%%%%%%%%%%%%%%%%%%%%%%%%%%%%%%%%%%%%%%%%%%%%%%%%%%%%%%%%%%%%%%%%%%%%%%%%
\subsection{Theoerm 10.28 --- Compact Argument}

In the proof of theoerm~10.28, we are given a compact \(K \subset \Omega\)
and the proof claims that there exists \(r>0\) such that
\begin{equation*}
E := \bigcup_{z\in K} \overline{D}(z;r)
\end{equation*}
is a compact subset of \(\Omega\).
Here is the justification.

Take the distance 
\begin{equation*}
d=d(K,\Omega^c)=\inf\{d(z,w):z\in K, w\in \C\setminus\Omega\}.
\end{equation*}
By compactness argument and the fat that \(\C\) is a Hausdorff space, \(d>0\).
Put \(r=d/2\) and
for each \(z\in K\) we pick \(r_z>0\) such that \(D(z,r) \subset \Omega\).
By compactness of $K$, there is a finite subcover
\begin{equation*}
E = \subset \bigcup_{z\in I} D(z,r) \subset \Omega.
\end{equation*}
where \(I\subset K\) is finite.


%%%%%%%%%%%%%%%%%%%%%%%%%%%%%%%%%%%%%%%%%%%%%%%%%%%%%%%%%%%%%%%%%%%%%%%%
\subsection{Lemma 10.29 --- Working out an Integral}

In the proof of Lemma~10.29.
we use the formula for integration oevr path devleopped
in section~10.8, namely for a smooth \(\gamma:[\alpha,\beta]\to\C\) path
\begin{equation*}
\int_\gamma f(x)\,dz = \int_\alpha^\beta f(\bigl(\gamma(t)\bigr)\gamma'(t)\,dt.
\end{equation*}

with \(\zeta(t) = (1-t)z+tw\) we can compute
\begin{align*}
f(w) - f(z)
&= f(\zeta(1)) - f(\zeta(0)) \\
&= \int_{\zeta([0,1])} f'(\xi)\,d\xi
 = \int_{\zeta([0,1])} f'(\zeta(t))\zeta'(t)\,dt
 = \int_{\zeta([0,1])} f'(\zeta(t))(w-z)\,dt \\
&= (w-z)\int_{\zeta([0,1])} f'(\zeta(t))\,dt
 = (w-z)\int_{\zeta([0,1])} f'(\zeta(t))\,dt
\end{align*}
Hence
\begin{equation*}
\int_0^1 \left[f'(\zeta(t)) - f'(a)\right]\,dt
= \int_0^1 f'(\zeta(t))\,dt - f'(a)
= \bigl((f(w) - f(z)\bigr)/(w-z) - f'(a).
\end{equation*}

It is tempting to try prroving this lemma directly without integration.
For this we need to have a local bound on \(f'\) which is given
by the fact that \(f^{(2)}\) exists and is continuous.
The idea is to show continuity at \((0,0)\) and then for sufficiently small
\(r>0\) and \(z,w\in D'(0,r)\) estimate
\begin{equation*}
\left|\frac{f(z)-f(w)}{z-w} - f'(0)\right|
\leq \left|\frac{f(z)-f(w)}{z-w} - f'(w)\right| + |f'(w) - f(0)|
\end{equation*}
and show that it can be small as desired.

%%%%%%%%%%%%%%%%%%%%%%%%%%%%%%%%%%%%%%%%%%%%%%%%%%%%%%%%%%%%%%%%%%%%%%%%
\subsection{Lemma 10.30 --- Adding Details}

The proof of theoerm~10.30 has several arguments that may use
clarifying details.

\paragraph{Setting $c$.} We can set
\begin{equation*}
c := \left|r\varphi'(z_0)\right|/5
\end{equation*}

\paragraph{Onto neighborhood.}
We pick \(\lambda\) such that
\begin{equation} \label{eq:thm10.30:lambda}
|\lambda - \varphi(a)| < c.
\end{equation}
Then after we show
\begin{equation*}
\min_\theta \left| \lambda - \varphi(a + re^{i\theta})\right|
\geq
  \left(\min_\theta \left| \varphi(a + re^{i\theta}) - \varphi(a)\right|\right)
  - |\varphi(a) - \lambda|
> 2c - c = c
\end{equation*}
the proof utilizes corollary of theoerm~10.24. Let's show how.
Set \(\psi(z) = \lambda - \varphi(z)\).
The corollary says that if
\(\psi(z)\) has no zeros in \(D(a;r)\) then
\begin{equation*}
|\psi(a)| = |\lambda - \varphi(a)|
\geq \min_\theta \left| \lambda - \varphi(a + re^{i\theta})\right|.
\end{equation*}
But the last inequality contradicts \eqref{eq:thm10.30:lambda}, hence
\(\psi\) has a zero in \(D(a;r)\).


%%%%%%%%%%%%%%%%%%%%%%%%%%%%%%%%%%%%%%%%%%%%%%%%%%%%%%%%%%%%%%%%%%%%%%%%
\subsection{Lemma 10.32 --- Differentiation}

Given \(g'/g = h'\) for some \(h\in H(\Omega)\).
Put \(\eta = g\cdot\exp(-h)\). Now
\begin{equation*}
\eta'
= g'\cdot\exp(-h) + g(-h)'\cdot\exp(-h)
= g'\cdot\exp(-h) + g(-g'/g)'\cdot\exp(-h)
= (g'-g')\cdot\exp(-h)
= 0.
\end{equation*}

%%%%%%%%%%%%%%%%%%%%%%%%%%%%%%%%%%%%%%%%%%%%%%%%%%%%%%%%%%%%%%%%%%%%%%%%
\subsection{Lemma 10.32 --- Details}

The term
\index{deleted neighborhood}
\index{deleted!neighborhood}
\emph{deleted neighborhood} of \(z_0\) means \(V\setminus\{z_0\}\)
where $V$ is a neighborhood of \(z_0\).

Tracing the proof. Assuming by negation \(f'(z_0) = 0\).
By theoerm~10.32 for a neighborhood $V$ of \(z_0\) we have
\begin{equation*}
f(z) = f(z_0) + \bigl(\varphi(z)\bigr)^m
\qquad \textnormal{where}\;\varphi\in H(V),\; m > 1.
\end{equation*}
and \(\varphi\) is \emph{onto} some \(D(0;r)\).
So if we look at inverse image
\begin{equation*}
A := \{\varphi^{-1}(\rho e^{2\pi k/m}\} \qquad \textnormal{for}\; \rho < r.
\end{equation*}
Now \(|A|=m\) and \(f(A)\) has one value, thus $f$ is $m$-to-$1$.
This contradiction shows that \(f'(z_0) \neq 0\).

%%%%%%%%%%%%%%%%%%%%%%%%%%%%%%%%%%%%%%%%%%%%%%%%%%%%%%%%%%%%%%%%%%%%%%%%
\subsection{Theoerm 10.37 --- Details}

There is a typo in the end of the stated Theorem.
Instead of
\begin{quote}
\textsl{\ldots\ if \(x\in D_-\) \ldots}
\end{quote}
It should be
\begin{quote}
\textsl{\ldots\ if \(z\in D_-\) \ldots}
\end{quote}

There are illustrations of the paths defined in the proof 
in \figurename{\ref{fig:10-37}}.
\begin{figure}[ht]
 % \centering
 % \captionsetup[subfloat]{nearskip=-3pt}
%
\subfloat[\(C(s)\) and \(\gamma(s)\)]{%
\begin{minipage}[b]{0.4\textwidth}
\centering
\input{10-37-a}
\end{minipage}
}
%
\hspace{0.1\textwidth}
%
\subfloat[\(f(s)\)]{%
\begin{minipage}[b]{0.4\textwidth}
\centering
\input{10-37-b}
\end{minipage}
}
%
\\[20pt]%
%
\subfloat[\(g(s)\)]{%
\begin{minipage}[b]{0.4\textwidth}
\centering
\input{10-37-c}
\end{minipage}
}
%
\hspace{0.1\textwidth}
%
\subfloat[\(h(s)\)]{%
\begin{minipage}[b]{0.4\textwidth}
\centering
\input{10-37-d}
\end{minipage}
}
%
\caption{Paths in the Proof of Theorem 10.37}
\label{fig:10-37}
\end{figure}


%%%%%%%%%%%%%%%%%%%%%%%%%%%%%%%%%%%%%%%%%%%%%%%%%%%%%%%%%%%%%%%%%%%%%%%%
\subsection{Lemma 10.39 --- Details}

Given \(\gamma = (\gamma_1 - \alpha) / (\gamma_0 - \alpha)\)
perfrom the computation:
\begin{equation*}
\frac{\gamma'}{\gamma}
= \frac{\gamma_1'(\gamma_0-\alpha) - (\gamma_1-\alpha)\gamma_0'}{
               (\gamma_0 - \alpha)^2}
   \cdot \frac{\gamma_0 - \alpha}{\gamma_1 - \alpha}
= \frac{\gamma_1'(\gamma_0-\alpha) - (\gamma_1-\alpha)\gamma_0'}{
               (\gamma_1 - \alpha)(\gamma_0 - \alpha)}
= \frac{\gamma_1'}{\gamma_1-\alpha} - \frac{\gamma_0'}{\gamma_0-\alpha}.
\end{equation*}

%%%%%%%%%%%%%%%%%%%%%%%%%%%%%%%%%%%%%%%%%%%%%%%%%%%%%%%%%%%%%%%%%%%%%%%%
%%%%%%%%%%%%%%%%%%%%%%%%%%%%%%%%%%%%%%%%%%%%%%%%%%%%%%%%%%%%%%%%%%%%%%%%
\section{Edition 2 (Old) Exercises} % pages 193-195

Some exercises in edition~2 do not appear or are different than in edition~3.
We bring some of them here

%%%%%%%%%%%%%%%%%
\begin{enumerate}
%%%%%%%%%%%%%%%%%

\iffalse
%%%%%%%%%%%%%% 1
\begin{excopy}
   If $A$ and $B$ are disjoint subsets of the plane, if $A$ is compact,
   and if $B$ is closed, then there exists
   a~\(\delta > 0\) such that \(|\alpha-\beta| \geq 0\) for all
   \(\alpha \in A\) and \(\beta \in B\). Prove this with an arbitrary
   metric space in place of the plane.
\end{excopy}
  Let $d$ be the metric. Define
  \[G_n = \{x \in A: d(x,B) > 1/n\}\]
  for all \(n>0\). It is clear that for all \(x\in A\) \(d(x,B)>0\)
  and so \(\cup G_n = A\). Since A is compact there exists $m$ such that
  \(\cup_{n=1}^m G_n = A\).
  Hence, for all \(x\in A\) \(d(x,B)>1/m\)
  Put \(\delta=1/m\) that satsify the requirement.
\fi

\setcounter{enumi}{1}
%%%%%%%%%%%%%% 2
\begin{excopy}
    At the end of section~10.8 occurs the definition of the length of a path
    \(\gamma\) as
    \[\int_\alpha^\beta|{\gamma^\prime}(t)|dt.\]
    Does this agree with the definitions given in Exercise~10, Chapter 8?
    Length of a graph of~$f$ is
       \[f_s(1) + \int_0^1 \sqrt{1 + |f^\prime(t)|^2}dt.\]
\end{excopy}

\text{Note:} This is of course referring to exercise in
edition~2 (\cite{RudinRCA80}).
In Edition~3, it appears in chapter~7, exercise~21.
Actually in this exercise we have shown in \eqref{eq:ex7.21:Sdif}
what we need here. There we used the result for a specific case
of \(x(t)=t\).

%%%%%%%%%%%%%%
\end{enumerate}


%%%%%%%%%%%%%%%%%%%%%%%%%%%%%%%%%%%%%%%%%%%%%%%%%%%%%%%%%%%%%%%%%%%%%%%%
%%%%%%%%%%%%%%%%%%%%%%%%%%%%%%%%%%%%%%%%%%%%%%%%%%%%%%%%%%%%%%%%%%%%%%%%
\section{The Exercises} % pages 227-230

Exercises of \emph{current} 3rd edition.

%%%%%%%%%%%%%%%%%
\begin{enumerate}
%%%%%%%%%%%%%%%%%

%%%%%%%%%%%%%% 01
\begin{excopy}
The following fact was tacitly used in this chapter:
If $A$ and $B$ are disjoint subsets of the plane,
if $A$ is compat and $B$ is closed, then there exists a \(\lambda > 0\)
such that \(|\alpha - \beta| \geq \lambda\) for all
\(\alpha \in A\) and \(\beta \in B\).
Prove this, with an arbitrary metric space in place of the plane.
\end{excopy}

Let \((X,d)\) be a matric space and $A$ and $B$ as described in $X$.
For each \(x\in A\) pick some \(\lambda_x>0\) such that
\(D(x;\lambda_x) \cap B = \emptyset\). Such \(\delta_x\) always exists
since otherwise \(x \in \overline{B}=B\).

Clearly \(\{D(x;\lambda_x)\}_{x\in A}\) is an open covering of the compact $A$,
hence there is a finite set \(F\subset X\) such that
\begin{equation*}
 X \subset \bigcup_{x\in F} D(x;\lambda_x).
\end{equation*}
Hence we can pick a \(\delta = \min(\{\delta_x: x\in F\}) > 0\)
that satisfies the requirement.


%%%%%%%%%%%%%% 02
\begin{excopy}
Suppose that $f$ is an entire function,
and that in every  power series
\begin{equation*}
f(z) = \sum_{n=0}^\infty c_n(z-a)^n
\end{equation*}
at least one coefficient is $0$. Prove that $f$ is a polynomial.
\\ \emph{Hint:} \(n!\,c_n = f^{(n)}(a)\).
\end{excopy}

For each \(a\in D(0,1)\) let \(n_a\) be the minimal \(n\in\Z^+\) such that
\(c_n=0\) for the power series representation around $a$.
By cardinality argument \(\|D(0,1)|=2^{\aleph_0} > \aleph_0 = \Z^+\)
there is some \(m\in\Z^+\) and infinite subset \(A\subset D(0,1)\)
such that \(m=n_a\) for each \(a\in A\).
But then \(f^{(m)}\) has infinite zeros in \(D(0,1)\)
and hence \(f^{(m)}(z) = 0\) for all \(z\in\C\),
hence $f$ is a polynomial of degree  \(<m\).

%%%%%%%%%%%%%% 03
\begin{excopy}
If $f$ and $g$ are entire function, and \(|f(z)| \leq |g(z)|\)
for every $z$. What conclusions can you draw?
\end{excopy}

\textbf{Claim:}
Under these conditions, \(f(z) = ag(z)\) for all \(z\in\C\) for some
\(a\in\C\) such that \(a\leq 1\).

If \(g = 0\) then so is \(f = 0\) and we are done.

We claim that \(q(z) = f(z)/g(z)\) for \(z\notin Z(g)\)
can be always be extended to an entire function.

So we may now assume that $g$ is not constantly zero, and by theoerm~10.18
\(Z(g)\) has no limit point.
Assume \(g(w) = 0\)
then bt theoerm~10.18 we have the following representations
\begin{equation*}
f(z) = (z-w)^m\tilde{f}(z)
\qquad
g(z) = (z-w)^n\tilde{g}(z)
\end{equation*}
and \(\tilde{f}(w) \neq 0 \neq \tilde{f}(w)\).
If by negation \(m < n\) then for suffuciently small \(\delta>0\)
\begin{equation*}
\delta^{n-m} < \bigl|\ \tilde{f}(z+\delta) / \tilde{g}(z+\delta)\bigr|
\end{equation*}
but then \(|f(z+\delta)| > |g(z+\delta)|\) which is a contradiction.
Hence by theoerm~10.20 \(q=f/g\) has a removable singularity at $z$
and
\begin{equation*}
q(z) = \left\{\begin{array}{ll}
\tilde{f}(z)/\tilde{g}(z) \qquad& m=n \\
0 & m > n
\end{array}\right.
\end{equation*}

Similarly $q$ can be defined for all \(Z(f)\),
hence we can view it as an entire function. By the given condition,
$q$ is bounded. By
\index{Lioville}
Lioville's theoerm~10.23 $q$ is constant. Hence \(f(z)= q(0) g(z)\).


%%%%%%%%%%%%%% 04
\begin{excopy}
Suppose that $f$ is an entire function, and
\begin{equation*}
|f(x)| \leq A + B|z|^k
\end{equation*}
for all $z$, where $A$, $B$, and $k$ are positive numbers.
Prove that $f$ must be a polynomial.
\end{excopy}

Let \(f(z) = \sum c_nz^n\). By
\index{Lioville}
theoerm~10.22 for any \(r\geq 0\) we have
\begin{equation*}
\sum_{n=0}^\infty |c_n|^2 r^{2n}
= \frac{1}{2\pi}\int_{-\pi}^\pi \left| f(re^{i\theta})\right|^2\,d\theta
\leq \frac{1}{2\pi} (2\pi)\cdot(A+Br^k)^2
= A^2+2ABr^k + B^2r^{2k}
\end{equation*}
Clearly \(c_n = 0\) whenever \(n > 2k\), otherwise the above inequality
would fail. Hence $f$ is a polynomial.

%%%%%%%%%%%%%% 05
\begin{excopy} 
Suppose 
\label{ex:fn:uniform}
\(\{f_n\}\) is a uniformly bounded sequence of holomorphic function
in \(\Omega\) such that  \(\{f_n(z)\}\) converges for every \(z\in \Omega\).
Prove that the convergence is uniform on every compact subset of \(\Omega\).
\\ \emph{Hint:} Apply the dominated convergence theorem to the Cauchy formula
for \(f_n - f_m\).
\end{excopy}

Define the pointwise limit \(f(z) = \lim_{n\to\infty} f_n(z)\).
Since \(\{f_n\}\) is a uniformly bounded, we have
\begin{equation*}
M := \sup_n \|f_n\|_\infty < \infty.
\end{equation*}

Let us temporarily restrict the function to some disc
\(\overline{D}(a;r/2)\subset D(a;r) \subset\Omega\).
Let \(\gamma(t) = a+re^{it}\), hence \(\gamma^* = \partial(D(a;r))\).
By the Cauchy formula
\begin{equation*}
\left| \frac{f_n(w)}{w - z}\right| \leq 2M/r
\qquad (z\in \overline{D}(a;r/2),\, w \in \gamma^*)
\end{equation*}
For abbreviation, let \(w_t = a+re^{it}\).
Now for \(z\in \overline{D}(a;r/2)\) we have
\begin{align*}
|f_m(z) - f_n(z) 
&\leq \frac{1}{2\pi} \int_{\gamma^*} \frac{|f_m(w)-f_n(w)}{|w-z|}\,d|w|
 \leq \frac{1}{\pi r}  \int_0^{2\pi} |f_m(w_t)-f_n(w_t)|\,dt \\
& \leq \frac{1}{\pi r}  
       \left(\int_0^{2\pi} |f_m(w_t)-f(w_t)|\,dt +
             \int_0^{2\pi} |f_n(w_t)-f(w_t)|\,dt\right)
\end{align*}
The last integrands are each dominated by \(2M\).
By Lebesgue's dominated convergence theorem~1.34,
for each \(\epsilon>0\) there exists $N$ such that 
the last 2-intergrals expression is \(<\epsilon\)
if \(m,n\geq N\), for \emph{any} \(z\in \overline{D}(a;r/2)\).
Hence \(\{f_n(z)\}\) converges uniformly on \(\overline{D}(a;r/2)\).

Any compact \(K\subset\Omega\) can be covered by finite set
of such \(\{\overline{D}(a_j;r/2): j\in J\}\) closed discs
with \(|J|<\infty\) and \(D(a_j;r)\subset \Omega\).

Returning to the original definitions of \(\{f_n\}\) on \(\Omega\),
the  \(\{f_n(z)\}\) converges uniformly on each closed disc, 
and therefore the sequence converges uniformly on a finite union of them.
Hence  \(\{f_n(z)\}\) converges uniformly on $K$.

%%%%%%%%%%%%%% 06
\begin{excopy}
There is a region \(\Omega\) that \(\exp(\Omega) = D(1;1)\).
Show that \(\exp\) is one-to-one in \(\Omega\),
but that there are many such \(\Omega\). Fix one, and define
\(\log z\), for \(|z-1|<1\), to be that \(w\in\Omega\) for which \(e^w = z\).
Prove that \(\log'(z) = 1/z\). Find the coefficients \(a_n\) in
\begin{equation*}
\frac{1}{z} = \sum_{n=0}^\infty a_n(z-1)^n
\end{equation*}
and hence the coefficients \(c_n\) in the expansion
\begin{equation*}
\log z = \sum_{n=0}^\infty c_n(z-1)^n.
\end{equation*}
In what other discs can this be done?
\end{excopy}

For each \(z\in D(1;1)\) there exists a unique \(\theta\in (-\pi,\pi)\) 
and \(r>0\) such that \(z=re^{i\theta}\).
Let 
\begin{equation*}
\Omega = \left\{\log(r) + i\theta:  
           re^{i\theta} \in D(1;1) \;\wedge\; \theta\in (-\pi,\pi)\right\}.
\end{equation*}
Clearly \(\Omega_n = \{z + 2\pi ni: z\in \Omega\}\)
can be a region as required for all \(n\in\Z\).

Now with the above definion of \(\log\), we have
\begin{equation*}
(\exp\circ\log)(z) = \Id_\Omega(z) \qquad (z\in\Omega).
\end{equation*}
Hence for \(z\in\Omega\) we have 
\begin{equation*}
1 
= \log'(z)\cdot (\exp'\circ \log)(z)
= \log'(z)\cdot (\exp\circ \log)(z) 
= \log'(z)\cdot z.
\end{equation*}
Thus \(\log'(z) = 1/z\) for \(z\in \Omega\).

\paragraph{Computing coefficients.}
Differentiation of \(1/z\) gives
\begin{equation*}
\frac{d^n(z^{-1})}{dz^n} = 
% z^{-1}, -z^{-2}, -2z^{-3}, 6z^{-4}, -24z^{-5}
 (-1)^n \cdot n!\cdot z^{-n-1}
\end{equation*}
Hence, the coefficients of the power series of \(1/z\) around $1$ are
\begin{equation*}
a_n = \frac{d^n(z^{-1})}{dz^n}(z=1)/n!
=  (-1)^n \cdot n!\cdot 1^{-n-1} / n! = (-1)^n
\end{equation*}

The coefficients of the ``primitive'' \(\log(z)\) are:

This can be done on any open disc that does contains the zero.
\begin{align*}
c_0 &= 0 \\
c_n &= -(-1)^n/n \qquad (n > 0)
\end{align*}
That is \((c_n)=0,1,-1/2,1/3,-1/4,\ldots\).

%%%%%%%%%%%%%% 07
\begin{excopy}
If \(f\in H(\Omega)\), the Cauchy formula for the derivatives of $f$,
\begin{equation*}
f^{(n)}(z)
= \frac{n!}{2\pi i} \int_\Gamma \frac{f(\xi)}{(\xi - z)^{n+1}}\,d\xi
\qquad (n=1,2,3,\ldots)
\end{equation*}
is valid under conditions on $z$ and \(\Gamma\).
State these and prove the formula.
\end{excopy}

Sufficent conditions are:
\begin{itemize}
\item \(\Gamma\) is a cycle in \(\Omega\).
\item \(\Ind_\Gamma(\alpha)=0\) for every \(\alpha\in \C\setminus\Omega\).
\item \(z\notin \Gamma^*\).
\item \(\Ind_\Gamma(z)=1\).
\end{itemize}
By Cauchy's formula theorem~10.15 we have
\begin{equation*}
f(z) = \itwopii\int_\Gamma \frac{f(w)}{w-z}\,dw.
\end{equation*}
This shows the desired formula holds for \(n=0\).
By induction assumes that it holds for \(n=k\), that is
\begin{equation*}
f^{(k)}(z)
= \frac{k!}{2\pi i} \int_\Gamma \frac{f(w)}{(w - z)^{k+1}}\,dw
\end{equation*}
Now
\begin{align}
f^{(k+1)}(z)
&= \lim_{h\to 0} \left(f^{(k)}(z+h)- f^{(k)}(z+h)\right)/h \notag \\
&= \lim_{h\to 0} \frac{k!}{2\pi i} 
   \left(
    \int_\Gamma 
    f(w)
    \left(
       \frac{1}{(w - (z+h))^{k+1}} - \frac{1}{(w - z)^{k+1}}
    \right)\,dw  
    \right) \bigm/ h
    \notag \\
&= \frac{k!}{2\pi i} \int_\Gamma \lim_{h\to 0} 
    f(w)
    \left(
       \frac{1}{(w - (z+h))^{k+1}} - \frac{1}{(w - z)^{k+1}}
    \right)\bigm/h\,dw  \label{eq:cauchy:diflim} \\
&= \frac{k!}{2\pi i} \int_\Gamma 
    f(w) \frac{d\,\left((w-z)^{-(k+1)}\right)}{dz}\,dw  \notag \\
&= \frac{k!}{2\pi i} \int_\Gamma f(w)(k+1)(w-z)^{-(k+2)}\,dw \notag \\
&= \frac{(k+1)!}{2\pi i} \int_\Gamma \frac{f(w)}{(w - z)^{k+2}}\,dw \notag
\end{align}
The justification of \eqref{eq:cauchy:diflim} is based on the fact
that \(\Gamma^*\) is compact it is it sufficient to use
sequences as limits and fintally utilize exercise~\ref{ex:fn:uniform} above.

%%%%%%%%%%%%%% 08
\begin{excopy}
Suppose $P$ and $Q$ are polynomials, the  degree of $Q$ exceeds that of $P$
by at least $2$,
and the rational function \(R = P/Q\) has no pole on the real axis.
Prove that the integral of $R$ over \((-\infty,\infty)\)
is \(2\pi i\) times the sum of the residues of $R$ in the upper half plane.
[Replace the integral over \((-A,A)\)  by one over a suitable semicircle,
and apply the residue theorem.] What is the analogous statement for the lower
half plane? Use this method to compute
\begin{equation*}
 \int_{-\infty}^\infty \frac{x^2}{1+x^4}\,dx.
\end{equation*}
\end{excopy}

Let \(a>0\) be such that all the roots of $Q$ are in \(D(0;a)\).
For each \(A>a\) consider the closed path 
\(\gamma_A\: [-A, A+\pi \to\C\) that is defined as follows
\begin{equation*}
\Gamma_A(t) = \left\{%
\begin{array}{ll}
t & t \leq A \\
Ae^{i(t-A)} \quad & t \geq A
\end{array}\right.
\end{equation*}
Now define the upper arc \(U_A=\{z\in\C: |z|=A \wedge \Im(z)>0\}\) and
\begin{equation*}
I(A) 
= \int_U P(z)/Q(z)\,dz 
= \int_0^\pi P(e^{iAt})/Q(e^{iAt})\frac{dz}{dt}\,dt
= iA\int_0^\pi P(e^{iAt})e^{iAt}/Q(e^{iAt})\frac{dz}{dt}\,dt
\end{equation*}
Hence \(|I(A)| \leq \pi A \sup_{z\in A} |P(e^{iAt})/Q(e^{iAt})|\).
Since \(\deg(P) \leq \deg(Q)+2\) we have
\begin{equation*}
\lim_{A\to\infty} I(A) = 0.
\end{equation*}
\index{residue theorem}
The set of poles of $R$ is exactly the set \(Z_Q\) of zeros of $Q$.
We define 
\begin{equation*}
Z_{Q^+} = \{z\in Z_Q: \Im(z)>0\}
\qquad
Z_{Q^-} = \{z\in Z_Q: \Im(z)<0\}
\end{equation*}
By the residue theorem~10.42.
\begin{align*}
\int_{-\infty}^\infty R(x)\,dx 
&= \lim_{A\to\infty} 
   \left(\int_{\Gamma_A}  R(z)\,dz - \int_{U_A} P(z)/Q(z)\,dz\right) \\
&= \lim_{A\to\infty} \int_{\Gamma_A}  R(z)\,dz 
   - \lim_{A\to\infty} \int_{U_A} P(z)/Q(z)\,dz 
 = \lim_{A\to\infty} \int_{\Gamma_A}  R(z)\,dz \\
&= 2\pi i \sum_{z\in Z_{Q^+}} \Res(R;z).
\end{align*}

The analogous statement for \(Z_{Q^-}\) is
\begin{equation*}
\int_{-\infty}^{\infty} R(x)\,dx 
= -\int_{\infty}^{-\infty} R(x)\,dx 
= 2\pi i \sum_{z\in Z_{Q^-}} \Res(R;z).
\end{equation*}
Hence if all the zeros if \(Q(z)\) are on one side, then the integral
over the real line is zero. 
Note that if all the coefficients of \(Q(z)\) are real,
then \(Q(z)=0\) iff \(Q(\overline{z})=0\).

The set of zeros of \(x^4+1\) is \(\{e^{(2k+1)\pi i/4}: k=0,1,2,3\}\).
Hence the zeros above the real line are \(\{e^{\pi i/4}, e^{3\pi i/4}\}\)
or \(\{q,qi\}\) where \(q=e^{\pi i/4}=(1+i)\sqrt{2}/2\).
Now \(Q(z)=\prod_{k=0^3} (z-qi^k)\) hence
\begin{equation*}
\Res(R;q_j) = q^2/ \prod_{0\leq k\leq3 \wedge k\neq j} (z-qi^k) \qquad (j=0,1,2,3)
\end{equation*}
Applying it gives
\begin{align*}
 \int_{-\infty}^\infty \frac{x^2}{1+x^4}\,dx
 &= 2\pi i \sum_{z\in Z_Q} \Res(R;z)
  = 2\pi i \left( \Res(R; e^{\pi i/4}) + \Res(R; e^{\pi i/4})\right) \\
 &= 2\pi i \left( q^2/\bigl((q-qi)(q+q)(q+qi)\bigr) 
                 -q^2/\bigl((qi-q)(qi+q)(qi+qi)\bigr)\right) \\
 &= (2\pi i/q) \cdot
    \left(\frac{1}{(1-i)2(1+i)} - \frac{1}{(-1+i)(1+i)2i}\right) \\
 &= (\pi i/q)  \left(\frac{1}{2}  - \frac{1}{-2i}\right) 
  = \pi q (1-i)/2 
  = \sqrt{2}\pi(1+i)(1-i)/4 \\
 &= \sqrt{2}\pi/2 \simeq 2.2214414690791831
\end{align*}


%%%%%%%%%%%%%% 09
\begin{excopy}
Compute \(\int_{-\infty}^\infty e^{-itx}/(1+x^2)\,dx\) for real $t$,
by methods described on Exercise~8.
Check your answer against the inversion theorem for Fourier transforms.
\end{excopy}

Put \(f_t(z) = e^{-itz}/(1+z^2)\). 
Assume first that \(t\leq 0\). 
If \(\Im(z)\geq 0\) then \(\Re(-itz) \leq 0\) and so \(|e^{-itz}|\geq 1\).
% Thus \(|e^{-itz}/(1+z^2)| \leq 1/z^2\).
Consider the 
 upper plane half circle with radius \(r>1\) defined by:
\begin{equation*}
D_r = \{z\in\C: |z|\geq r \wedge \Im z \geq 0\}.
\end{equation*}
Integration over its boundary gives:
\begin{equation*}
\int_{\partial D_r} e^{-itz}/(1+z^2)\,dz 
= (2\pi i) \Res(e^{-itz}/(1+z^2); i)
= (2\pi i) e^{-iti}/(i-(-i)) = \pi e^t
\end{equation*}
On the arc part of \(D_r\) 
we have \(|e^{-itz}/(1+z^2)| \leq |1/z^2| = 1/r^2\).
Hence
\begin{equation*}
\int_{-\infty}^\infty f_t(x)\,dx
= \lim_{r\to\infty} \int_{\partial D_r} f_t(z)\,dz 
= \Res(e^{-itz}/(1+z^2); i) 
= \pi e^t
\end{equation*}
Since \(f_t(x) = f_{-t}(-x)\) for any \(t\in\R\) we have
\(\int_{-\infty}^\infty f_t(x)\,dx = \int_{-\infty}^\infty f_{-t}(x)\,dx\).
Therefore
\begin{equation*}
\int_{-\infty}^\infty e^{-itx}/(1+x^2)\,dx = \pi e^{-|t|} \qquad(t\in\R).
\end{equation*}

\paragraph{Fourier Inversion.}
\begin{align*}
\int_{-\infty}^\infty e^{itx} e^{-|t|}\,dt
&= \int_{-\infty}^0 e^{itx} e^{t}\,dt  + \int_0^\infty e^{itx} e^{-t}\,dt
 = \int_{-\infty}^0 e^{(1+ix)t}\,dt  + \int_0^\infty e^{(-1+ix)t}\,dt \\
&=   \left(e^{(1+ix)t}/(1+ix)\right)\biggm|_{t=-\infty}^0
   + \left(e^{(-1+ix)t}/(-1+ix)\right)\biggm|_{t=0}^\infty \\
&= e^{(1+ix)0}/(1+ix) - e^{(-1+ix)0}/(-1+ix) 
 = 1/(1+ix) - 1/(-1+ix) \\
&= 2/(x^2+1)
\end{align*}
Using \(\pi e^{-|t|}\) instead of \(e^{-|t|}\) 
and factoring with \(1/\sqrt{2\pi}\) provides 
the expected equality with the original \(1/(1+x^2)\) function.



%%%%%%%%%%%%%% 10
\begin{excopy}
Let \(\gamma\) be a poitively oriented unit circle, and compute
\begin{equation*}
\itwopii \int_\gamma \frac{e^z - e^{-z}}{z^4}\,dz
\end{equation*}
\end{excopy}

Using the residue theorem~10.42 and Taylor expansion
\begin{align*}
\itwopii \int_\gamma \frac{e^z - e^{-z}}{z^4}\,dz
&= \itwopii \int_\gamma z^{-4}\left(\sum_{n=0}^\infty (z^n - (-z)^n)/n!\right)\,dz \\
&= \itwopii \int_\gamma 2z^{-4}\left(\sum_{k=0}^\infty z^{2k+1}/(2k+1)!\right)\,dz \\
&= 2\cdot\Res\left(\sum_{k=0}^\infty z^{2k-3}/(2k+1)!;\,0\right)
 = 2\cdot\Res\left(z^{2\cdot 1-3}/(2\cdot 1+1)!;\,0\right) = \\
&= 2/3! = 1/3
\end{align*}


%%%%%%%%%%%%%% 11
\begin{excopy}
Suppose \(\alpha\) is a complex number, \(|\alpha|\neq 1\), and compute
\begin{equation*}
\int_0^{2\pi} \frac{d\theta}{1 - 2\alpha \cos\theta + \alpha^2}
\end{equation*}
by integrating \((z-\alpha)^{-1}(z-1/\alpha)^{-1}\) over the unit circle.
\end{excopy}

Putting \(z=e^{i\theta}\) (or \(\theta = -i\log(z)\))
then \(dz/d\theta = iz\) and \(d\theta/dz = -i/z\)
and when \(|z|=1\)
then \(\bar{z}=1/z\) and \(\cos \theta = (z+1/z)/2\).
Hence
\begin{align*}
\int_0^{2\pi} \frac{d\theta}{1 - 2\alpha \cos\theta + \alpha^2}
&= -i \int_{|z|=1} \bigl(1 - \alpha(z+1/z) + \alpha^2\bigr)^{-1} \bigm/ z\,dz \\
&= -i \int_{|z|=1} \bigl(z - \alpha(z^2+1) + z\alpha^2\bigr)^{-1}\,dz \\
&= (i/\alpha)  \int_{|z|=1} \left(z^2 - (\alpha+1/\alpha)z+1\right)^{-1}\,dz \\
&= (i/\alpha)  \int_{|z|=1} (z-\alpha)^{-1}(z-1/\alpha)^{-1}\,dz \\
&= (2\pi i^2/\alpha) \cdot \left\{%
   \begin{array}{ll}
   (1/\alpha - \alpha) \quad & \alpha > 1 \\
   (\alpha - 1/\alpha) \quad & \alpha < 1 \\
   \end{array}\right. \\
&= 2\pi / |\alpha^2 - 1|
\end{align*}


%%%%%%%%%%%%%% 12
\begin{excopy}
Compute
\begin{equation*}
\int_{-\infty}^\infty \left(\frac{\sin x}{x}\right)^2 e^{itx}\,dx
\qquad (\textnormal{for real }\; t).
\end{equation*}
\end{excopy}

Using Python and MatPlotLib, I have a conjecture, that the integral equals:
\begin{equation*}
\left\{\begin{array}{ll} %
\pi(2-|t|)/4 \qquad & |t| \leq 2 \\
0 & |t| \geq 2
\end{array}\right.
\end{equation*}

We follow the ideas of section~10.44.

The function \((\sin z/z)^2 e^{itz}\) has a removable singularity at \(z=0\)
and can be defined there as $1$ and be considered an entire function.
Since \(2i \sin z = e^{iz} - e^{-iz}\) we have
\begin{equation*}
-4\sin^2 z = \left(e^{iz} - e^{-iz}\right)^2 = e^{2iz} + e^{-2iz} - 2\,.
\end{equation*}
Thus
\begin{equation*}
(\sin z/z)^2 e^{itz} = -e^{i(t+2)z}/4z^2 - e^{i(t-2)z}/4z^2 + e^{itz}/2z^2\,.
\end{equation*}

Consider the path \(\Gamma_A\) from \(-A\) to \(-1\)
then lower unit circle and finally from $1$ to $A$.
Put
\begin{align*}
% \frac{1}{\pi}\varphi_A(s) = \itwopii \int_{\Gamma_A} \frac{e^{isz}}{z}\,dz\,.
\psi_s(z) &= \frac{e^{isz}}{z^2} \\
\varphi_A(s) 
  &= \int_{\Gamma_A} \frac{e^{isz}}{z^2}\,dz
  = \int_{\Gamma_A} \psi_s(z)\,dz\,.
  % = \int_{-A}^A \psi_s(x)\,dx\,. Not equal 
\end{align*}
Now
\begin{equation} \label{eq:10.12:sin2phis}
\int_{-A}^A \left(\frac{\sin x}{x}\right)^2 e^{itx}\,dx
% = -\varphi_A(t+2)/4 - \varphi_A(t-2)/4 + (1/2)\int_{\Gamma_A} z^{-2}\,dz
= \varphi_A(t)/2 - \bigl(\varphi_A(t+2)/4 - \varphi_A(t-2)\bigr)/4 
\end{equation}

We complete \(\Gamma_A\) into a closed path in two ways.
\begin{itemize}
\item With the lower half circle of radius $A$. Inside this closed path, 
      \(\psi_s(z)\) is analytic.
\item With the upper half circle of radius $A$. Inside this closed path, 
      \(\psi_A(z)\) has one pole at \(z=0\)
      with residue \(is\) (consider \(((isz)^1/1!)/z^2 = is/z\)).
\end{itemize}

For both path closing options, we use the substitution \(z = Ae^{i\theta}\)
and \(dz/d\theta = iAe^{i\theta}\)

Using the first choice of path closing gives:
\begin{align}
\varphi_A(s) 
&= \int_{|z|=A \wedge \Im(z)<0} \psi_s(z)\,dz
 = \int_{-\pi}^0 \psi_s(Ae^{i\theta}) \cdot iAe^{i\theta} \,d\theta \notag \\
&= (i/A) \int_{-\pi}^0 \exp(isAe^{i\theta})/e^{i\theta}\,d\theta.
   \label{eq:ex10.12:lowcirc}
\end{align}
Using the second choice with the residue in the pole gives:
\begin{align}
\varphi_A(s) 
&= 2\pi i\cdot \Res(\psi_s;0) - \int_{|z|=A \wedge \Im(z)>0} \psi_s(z)\,dz
 = -2\pi s - \int_0^\pi iAe^{i\theta} \psi_s(Ae^{i\theta})\,d\theta \notag \\
&= -2\pi s - (i/A) \int_{-\pi}^0 \exp(isAe^{i\theta})/e^{i\theta}\,d\theta.
   \label{eq:ex10.12:upcirc}
\end{align}
Since
\begin{equation*}
\left|\exp(isAe^{i\theta})\right| = \exp\bigl(-As\sin(\theta)\bigr),
\end{equation*}
and that is \(<1\) if $s$ and \(\sin(\theta)\) of the same sign.
Hence as \(A\to\infty\)
the integral of \eqref{eq:ex10.12:lowcirc} when \(s<0\)
and the integral of \eqref{eq:ex10.12:upcirc} when \(s>0\)
converges to~$0$.
Therefore
\begin{equation}
\lim_{A\to\infty} \varphi_A(s) = 
 \left\{\begin{array}{ll}%
 -2\pi s \qquad & \textnormal{if\ }\; s > 0\\
 0       \qquad & \textnormal{if\ }\; s \leq 0
 \end{array}
 \right. \label{eq:10.12:varphiA}
\end{equation}

\iffalse
We compute the rational integral term of \eqref{eq:10.12:sin2phis}.
Within the real axis
\begin{equation*}
\int_{\R\setminus(-1,1)} \frac{dx}{x^2} 
 = 2\int_1^{\infty} x^{-2}\,dx 
 = 2\left(-x^{-1}\right)\bigm|_1^\infty = 2\bigl(0-(-1)\bigr)=2
 \end{equation*}
and within the arc part
\begin{align*}
\int_{-\pi}^0 e^{-2i\theta}\frac{dz}{d\theta}\,d\theta 
&= i \int_{-\pi}^0 e^{-2i\theta}\cdot e^{i\theta}\,d\theta 
 = i \int_{-\pi}^0 e^{-i\theta}\,d\theta
 = i\left.\left(\frac{1}{-i}e^{-i\theta}\right)\right|_{-\pi}^0 \\
&= -\left.e^{-i\theta}\right|_{-\pi}^0 = % BADBAD 1 - \bigl(-(-1)\bigr) = 0.
   -\bigl(1 - (-1)\bigr) = -2
\end{align*}
Adding to:
\begin{equation}
\int_{\Gamma_A} \frac{dz}{z^2}
 = \int_{\R\setminus(-1,1)} \frac{dx}{x^2} 
   + \int_{-\pi}^0 \exp(isAe^{i\theta})e^{i\theta}\,d\theta \\
 = 2 - 2 = 0 \label{eq:10.12:intrat}
\end{equation}
\fi % false

Applying \eqref{eq:10.12:varphiA} % and \eqref{eq:10.12:intrat} 
to \eqref{eq:10.12:sin2phis} gives
\begin{align*}
\int_{-\infty}^\infty (\sin x/x)^2 e^{itx}\,dx
&= \lim_{A\to\infty} 
   \left(\varphi_A(t) -\bigl(\varphi_A(t+2) + \varphi_A(t-2)\bigr)/4)
   \right)\\
&= \left\{\begin{array}{ll}%
   0 + 0 + 0  \qquad & t \leq -2 \\
   0 + 2\pi(t+2)/4 + 0 \qquad & -2 \leq t \leq 0 \\
   -2\pi t/2 + 2\pi(t+2)/4 + 0 \qquad & 0 \leq t \leq 2 \\
   -2\pi t/2 + 2\pi(t+2)/4 + 2\pi(t-2)/4 \qquad & t \geq 2
   \end{array}\right. \\
&= \left\{\begin{array}{ll}%
   0  \qquad & t \leq -2 \\
   \pi t/2+\pi \qquad & -2 \leq t \leq 0 \\
   -\pi t/2 + \pi \qquad & 0 \leq t \leq 2 \\
   0 \qquad & t \geq 2
   \end{array}\right. \\
&= \max\left(0, \pi-|\pi t/2|\right)
\end{align*}

%%%%%%%%%%%%%% 13
\begin{excopy}
Compute
\begin{equation*}
\int_0^\infty \frac{dx}{1 + x^n} \qquad (n=2,3,4,\ldots).
\end{equation*}
[For even $n$, the method of Exercise~8 can be used.
However, a different path can be chosen, which simplifies the computation
and which also works for odd $n$:
from $0$ to $R$ to \(R = \exp(2\pi i/n)\) to $0$.]
\\
\phantom{AAAA}\emph{Answer:} \((\pi/n)\sin(\pi/n)\).
\end{excopy}

Put 
\begin{equation*}
f(z) = \frac{1}{1 + z^n}.
\end{equation*}
Assume first that n is even (\(n/2\in\Z\)) and \(n\geq 2\).
The roots of \(z^n+1\) are \mbox{\(\{e^{\pi i(2k+1)/n}\!: k\in \Z_n\}\)}.
Put \(\alpha_k = e^{\pi i(2k+1)/n}\) for \(k\in\Z_n\).
The roots in the upper place are
 \(\{\alpha_k: k\in \Z_{n/2}\}\).
Using the result of exercise~8 above and the fact that the integrand is 
an even function, we have
\begin{align*}
\int_0^\infty \frac{dx}{1 + x^n}
&= \half\cdot \int_{-\infty}^\infty \left(1 + x^n\right)^{-1}\,dx
 = \pi i \cdot \sum_{k\in\Z_{n/2}} 
        \Res\left( (1 + z^n)^{-1}; \alpha_k \right) \\
&= \pi i \cdot \sum_{k\in\Z_{n/2}}\; 
      \prod_{j\in \Z_n\setminus\{k\}} \left(\alpha_k - \alpha_j)\right)^{-1}
\end{align*}

\iffalse
Let's compute the residue of the ``first'' pole.
\begin{align*}
\Res(f, \alpha_0) 
&= \prod_{j\in \Z_n\setminus\{0\}} (\alpha_0 - \alpha_j)^{-1}
 = \lim_{z\to\alpha_0} \frac{z-\alpha_0}{\prod_{j\in \Z_n} (z - \alpha_j)}
 = \lim_{z\to\alpha_0} \frac{z-\alpha_0}{z^n+1} 
 = \lim_{z\to\alpha_0}\frac{1}{nz^{n-1}} \\
&= e^{\pi i/n}/n
\end{align*}
\fi 

Let's compute the residue of a pole.
\begin{align*}
\Res(f, \alpha_k) 
&= \prod_{j\in \Z_n\setminus\{k\}} (\alpha_k - \alpha_j)^{-1}
 = \lim_{z\to\alpha_k} \frac{z-\alpha_k}{\prod_{j\in \Z_n} (z - \alpha_j)}
 = \lim_{z\to\alpha_k} \frac{z-\alpha_k}{z^n+1} 
 = \lim_{z\to\alpha_k}\frac{1}{nz^{n-1}} \\
&= \alpha_k^{-(n-1)}/n
 %= \left(\alpha_k/\alpha_k^n\right)\bigm/n
 = \alpha_k^{-n}\alpha_k/n = -e^{(2k+1)\pi i/n}/n
% = e^{((2k+1)\pi i/n)(1-n))}/n
% = e^{(2k+1)\pi i(1-n)/n)}/n
\end{align*}
In particular 
\begin{equation*}
\Res(f, \alpha_0) = -\alpha_0/n0) = -e^{\pi i /n}/n.
\end{equation*}

Using the hint, We consider the closed path \(\Gamma_R\)
which is the sum of
\begin{equation*}
\begin{array}{ll}
\gamma_1:[0,R]\to\C & \gamma_1(t) = t \\
\gamma_2:[0,2\pi/n]\to\C \qquad& \gamma_2(t) = e^{it} \\
\gamma_3:[0,R]\to\C & \gamma_3(t) = e^{2\pi i/n}(R-t) 
\end{array}
\end{equation*}

Note that if \(z\in \gamma_3^*\) then 
\(z^n = |z|^n\) and so \((1+z^n)^{-1} = (1+|z|^n)^{-1}\).
For any \(R>1\) we have
\begin{equation*} 
\int_{\Gamma_R} \frac{dz}{1+z^n} = 2\pi i \Res(f;\alpha_0) 
= - 2\pi i e^{-\pi i/n}/n.
\end{equation*}
Now
\begin{align*}
\lim_{R\to\infty} \int_{\Gamma_R} \frac{dz}{1+z^n}
&= \lim_{R\to\infty} 
   \left( \int_{\gamma_1}\cdots +\int_{\gamma_2}\cdots +\int_{\gamma_3}\cdots \right)
 = \lim_{R\to\infty} \left( \int_{\gamma_1}\cdots +\int_{\gamma_3}\cdots \right) \\
&= \lim_{R\to\infty} \left( 
        \int_0^R (x^n+1)^{-1}\,dx +
        \int_0^R \left(\bigl((R-x)e^{2\pi i /n}\bigr)^n+1\right)^{-1}
                 \frac{dz}{dx}
                 \,dx\right) \\
&= (1 - e^{2\pi i /n}) \lim_{R\to\infty}  \int_0^R (x^n+1)^{-1}\,dx
\end{align*}
Hence 
\begin{align*}
 \int_0^\infty (x^n+1)^{-1}\,dx 
 &= \frac{2\pi i \left(-e^{\pi i/n}/n\right)}{1 - e^{2\pi i /n}}
  = \frac{-2\pi i }{ne^{-\pi i/n}(1 - e^{2\pi i /n})}
  = \frac{2\pi i }{n(e^{\pi i/n} - e^{-\pi i /n})} \\
 &= \frac{\pi}{n(e^{\pi i/n} - e^{-\pi i /n})/(2i)}
  = \frac{\pi}{n\cdot\sin(\pi/n)} \\
 &= (\pi/n)\bigm/\sin(\pi/n).
\end{align*}


%%%%%%%%%%%%%% 14
\begin{excopy}
Suppose \(\Omega_1\) and \(\Omega_2\) are plane regions, $f$ and $g$ are
nonconstant functions defined on
\(\Omega_1\) and \(\Omega_2\), respectively,
and \(f(\Omega_1) \subset \Omega_2\).
Put \(h = g\circ f\).
If $f$ and $g$ are holomorphic, we know that $h$ is  holomorphic.
Suppose we know that $f$ and $h$ are holomorphic.
Can we conclude anything about $g$?
What if we know that $g$ and $h$ are holomorphic?
\end{excopy}

In the second case we \emph{cannot} know much about $g$.
Take \(f(z)=0\) and \(g(z)=|z|\).
Clearly \(h=0\) and $g$ is not holomorphic.
A More interesting question would be if we also assume that 
$F$ is \emph{onto} \(\Omega_2\).


In the second case we \emph{cannot} know much about $f$.
For example, take \(g=0\) and \(h=0\) constant functions, 
and the equality holds for any \(f \in \Omega_2^{\Omega_1}\).

%%%%%%%%%%%%%% 15
\begin{excopy}
Suppose \(\Omega\) is a region, \(\varphi \in H(\Omega)\),
\(\varphi'\) has no zero in \(\Omega\),
\(f\in H(\varphi(\Omega))\),
\(g = f\circ \varphi\), \(z_0\in\Omega\), and
\(w_0 = \varphi(z_0)\).
Prove that if $f$ has a zero of order $m$ at \(w_0\),
then $g$ also has a zero of order $m$ at \(z_0\).
How is this modified if \(\varphi'\) has a zero of order $k$ at \(z_0\)?
\end{excopy}

Clearly \(g'(z_0) = f'(w_0)\varphi'(z_0)\).
Then  both \(g'\) and \(f'\) 
has zero of order \(m-1\) in \(w_0\) and \(z_0\) respectively.
Hence  $f$ has a zero of order~$m$ at~\(w_0\).
It is now easy to see that the zero order of~\(\varphi\) at~\(z_0\)
adds to the order $g$ at~\(z_0\).

%%%%%%%%%%%%%% 16
\begin{excopy}
Suppose \(\mu\) is a complex measure on a measure space~$X$,
\(\Omega\)~is an open set in the plane,
\(\varphi\)~is a bounded function on \(\Omega\times X\)
such that \(\varphi(z,t)\) is a measurable function of $t$,
for each \(z \in \Omega\), and \(\varphi(z,t)\) is holomorphic in \(\Omega\),
for each \(t\in X\). Define
\begin{equation*}
f(x) = \int_X \varphi(z,t)\,d\mu(t)
\end{equation*}
for \(z\in\Omega\). Prove that \(f\in H(\Omega)\).
\emph{Hint:} Show that to every compact \(K \subset \Omega\) there corresponds
a constant \(M<\infty\) such that
\begin{equation*}
\left| \frac{\varphi(z,t) - \varphi(z_0,t)}{z - z_0}\right| < M
\qquad (z \;\textnormal{ and }\; z_0\in K,\; t\in X).
\end{equation*}
\end{excopy}

Obviously, in the above formula we assume \(z\neq z_0\).
Let $U$ be some upper bound of \(|\varphi|\).
For all~\(t\in X\),
denote the holomorphic functions \(\varphi_t(z) = \varphi(z,t)\)
and the ratios
\begin{equation*}
\Delta_t(z_1,z_0) = \frac{\varphi_t(z_1) - \varphi_t(z_0)}{z_1 - z_0}
\qquad (\textnormal{for all distinct}\; z_0,z_1\in\Omega).
\end{equation*}
Recall that \(D'(a;r) = \{z: 0 < |z-a|<r\}\).

Let \(K\subset \Omega\) be a compact set.
Pick some arbitrary \(w\in K\).
Then we can find some \(r>0\) such that \(\overline{D(w;r)}\subset \Omega\).
We pick \(v_0\in D(w,r/2)\) and
we will estimate \(|\Delta_t(z,v_0)|\) for all \(z\in\Omega\). 

\paragraph{Local case.}
Clearly \(D(v_0;r/2)\subset D(w;r)\subset\Omega\) and
\(|{\varphi_t}'(v_0)| \leq 2U/r\) for all \(t\in X\) by Theorem~10.26.
% This shows that \(|\Delta_t(w,v_0)| \leq 2U/r\) 
for all \(t\in X\).

Pick \(v_1\in D'(v_0;r/2)\). 
Use the path segment 
\(\gamma(\tau) = v_0 + \tau(v_1-v_0)\) with \(\tau\in[0,1]\).
Then
\begin{equation*}
\varphi_t(v_1) 
= \varphi_t(v_0) + \int_\gamma {\varphi_t}'(z)\,dz
= \varphi_t(v_0) + \int_0^1 {\varphi_t}'(\gamma(\tau))\gamma'(\tau)\,d\tau
\end{equation*}
Hence
\begin{equation*}
\left|\varphi_t(v_1) - \varphi_t(v_0)\right|
\leq \int_0^1 \left|{\varphi_t}'(\gamma(\tau))\gamma'(\tau)\right|\,d\tau
\leq 2|v_1-v_0|U/r.
\end{equation*}
Subsequently
\begin{equation} \label{eq:ex10.16:estimate:in}
|\Delta_t(v_1,v_0)| \leq 2U/r 
\qquad (\textnormal{for all} v_0\in D(w;r/2),\;v_1 \in D(v_0;r/2)\,).
\end{equation}

\paragraph{Outside case.}
When \(v_1\in\Omega \setminus D(v_0;r/2)\)
\begin{equation} \label{eq:ex10.16:estimate:out}
|\Delta_t(v_1,v_0)| 
\leq \frac{|\varphi_t(w) - \varphi_t(v)|}{|r/2|} \leq 4U/r.
\end{equation}

To combine the case, define
\begin{equation*}
\Omega_M := 
\left\{z\in\Omega: 
  \forall t\in X, \forall \zeta\in\Omega\setminus\{z\},\; |\Delta_t(z,\zeta)|<M
\right\}.
\end{equation*}

From \eqref{eq:ex10.16:estimate:in} and \eqref{eq:ex10.16:estimate:out}
we know that there exists \(M = M_w\) (for example \(M_w = 4U/r + 1\))
such that 
\begin{equation*}
D(w;r/2) \subset \Omega_M
\end{equation*}

Since $w$ was arbitrarily chosen, \(\Omega \subset \cup_{M\in\N}\Omega_M\)
and since $K$ is compact we can find some \(M<0\) such that 
\(|\Delta_t(z_1,z_0)|<M\) for all \(t\in X\) and all distinct \(z_0,z_1\in K\).
Thus establishing the hint.

In order for $f$ to be differentiable at $z$, it is sufficient that
for any sequence \(\{z_n\}_{n\in\N}\) there exists a limit 
\begin{equation*}
f'(z) = \lim_{n\to\infty} \frac{f(z_n) - f(z)}{z_n - z} 
 \qquad (\text{where}\; z_n \neq z)
\end{equation*}
Now
\begin{equation*}
\frac{f(z_n) - f(z)}{z_n - z} 
= \frac{\int_X \varphi(z_n,t)\,d\mu - \int_X \varphi(z,t)\,d\mu}{z_n - z} \\
= \int_X \frac{\varphi(z_n,t) -  \varphi(z,t)}{z_n - z}\,d\mu
\end{equation*}

There exists some \(\delta>0\) 
such that \({D(z,\delta)}\subset\Omega\).
For sufficient large $m$, for all \(n>m\) 
we have \(z_n\in \overline{D(z,\delta/2)}\) and by its compatcness
and the established hint, \(\Delta_t(z_n,z)\) are bounded
and we can apply 
\index{Lebesgue}
Lebesgue's Dominated Theorem~1.34 that here gives 
\begin{equation*}
f'(z) 
= \lim_{n\to\infty} \int_X \Delta_t(z_n,t)\,d\mu
= \int_X \lim_{n\to\infty} \Delta_t(z_n,t)\,d\mu
= \int_X \varphi_t'(z)\,d\mu.
\end{equation*}

%%%%%%%%%%%%%% 17
\begin{excopy}
Determine the regions in which the following functions are defined
and holomorphic:
\begin{equation*}
f(z) = \int_0^1 \frac{dt}{1+tz}, \qquad
g(z) = \int_0^\infty \frac{e^{tz}}{1+t^2}\,dt, \qquad
h(z) = \int_{-1}^1 \frac{e^{tz}}{1+t^2}\,dt\,.
\end{equation*}
\emph{Hint}: either use Exercise~16, or combine
\index{Morera}
Morera's theorem with
\index{Fubini}
Fubini's.
\end{excopy}

\begin{itemize}

\item The function \(f(z)\) is holomorphic on any region \(\Omega\) for which 
\begin{equation*}
\{1/(1+tz): t\in[0,1]\;\wedge\; z\in\Omega\}
\end{equation*}
is bounded. This happens when there exists \(r > 0\) such that 
\begin{equation*}
\Omega \subset \{w\in\C: \forall z\in (-\infty,-1],\, |w-z|>r\}.
\end{equation*}

\item
The function \(g(z)\) is holomorphic on any region \(\Omega\) 
such that \(\Re(z)\leq 0\) for all \(z\in\Omega\).
Assume \(\Omega\) satisfies this condition.
Define
\begin{equation*}
g_n(z) = \int_0^n \frac{e^{tz}}{1+t^2}\,dt, \qquad
\end{equation*}
By previous exercise, \(g_n\) are holomorphic on \(\Omega\).
\index{Lebesgue}
By Lebesgue's Dominated Theorem~1.34 
\(\lim_{n\to\infty} g_n(z) = g(z)\) and it is easy to see that 
the convergence is uniform on any bounded subset of \(\Omega\),
in particular on triangle boudnaries. Thus the condition
\index{Morera}
of Morera's Theorem~10.17 holds, hence \(g\in H(\Omega)\). 

\item The function \(h(z)\) is holomorphic on any region \(\Omega\) for which 
\begin{equation*}
\{\exp(tz): t\in[-1,1] \;\wedge\; z\in\Omega\}
\end{equation*}
is bounded. This happens when there exists some \(M<\infty\) such that 
\begin{equation*}
\forall w\in\Omega,\;\Re(w) \leq M.
\end{equation*}

\end{itemize}

%%%%%%%%%%%%%% 18
\begin{excopy}
Suppose \(f\in H(\Omega)\),
\(\overline{D}(a;r)\subset \Omega\),
\(\gamma\) is a positively oriented circle with center at $a$ and radius $r$,
and $f$ has no zero on \(\gamma^*\). For \(p=0\), the integral
\begin{equation*}
\itwopii \int_\gamma \frac{f'(z)}{f(z)} z^p\,dz
\end{equation*}
is equal to the number of zeros of $f$ in \(D(a;r)\).
What is the value of this integral (in terms of zeros of $f$)
for \(p=1,2,3,\ldots\)?
What is the answer if \(z^p\) is replaced by any \(\varphi\in H(\Omega)\)?
\end{excopy}

For \(p = 0\) the claim is shown in Theorem~10.43\ich{a}.
Otherwise the number it still gives the number of zeros,
except that if $f$ has zero of order $k$ in \(z=0\) then 
the number is reduced by \(\min(p, k)\).
Similarly of \(\varphi\) has zeros of order \(p_j\) on \(z_j\)
where $f$ has zeros of order \(k_j\)
then the number is reduced by \(\sum_j \min(p_j,k_j)\).


%%%%%%%%%%%%%% 19
\begin{excopy}
Suppose \(f\in H(U)\), \(g\in H(U)\),
and neither $f$ nor $g$ has a zero in $U$. If
\begin{equation*}
\frac{f'}{f}\left(\frac{1}{n}\right) =
\frac{g'}{g}\left(\frac{1}{n}\right)
\qquad (n=1,2,3,\ldots)
\end{equation*}
find another simple relation between $f$ and $g$.
\end{excopy}

Both \((f'/f), (g'/g)\in H(U)\).
By the Corollary to Theorem~10.18 we have \(f'/f = g'/g\) in \(H(U)\).
Using line segments from $0$ to \(z=te^{i\theta}\in U\) 
as the path: \(\gamma(\tau) = \tau\cdot e^{i\theta}\) with \(0\leq \tau\leq t\), 
Define
\begin{equation*}
F(z) 
= \int_0^t (f'/f)(\tau e^{i\theta})\cdot e^{i\theta}\,d\tau
= e^{i\theta} \int_0^t \frac{d(\log\circ f)}{d\tau}(\tau e^{i\theta})\,d\tau
\end{equation*}
Similarly, we define \(G(z)\) by integrating \(g/g'\).
Clearly $F$ and $G$ differ by a constant, say \(c=F(z)-G(z)\). Hence
\begin{equation*}
(\log\circ f)(z) = (\log\circ g)(z) + c
\end{equation*}
By taking exponents, we get the relation
\begin{equation*}
f(z) = e^c \cdot g(z).
\end{equation*}

%%%%%%%%%%%%%% 20
\begin{excopy}
Suppose \(\Omega\) is a region,
\(f_n\in H(\Omega)\) for \(n=1,2,3,\ldots\),
none of the functions \(f_n\) has a zero in \(\Omega\),
and \(f_n\) converges to $f$ uniformly on compact subsets of \(\Omega\).
Prove that either $f$ has no zero in \(\Omega\) or \(f(z)=0\)
for all \(z\in\Omega\).
\end{excopy}

Consider \(N = f^{-1}(0) \subset\Omega\).
If \(N=\emptyset\) or \(N = \Omega\) we are done.

By negation assume \(\emptyset \subsetneq N \subsetneq \Omega\).
By Theorem~10.18 there exists an isolated zero.
Hence we have 
\(B(z;r) \subset \Omega\setminus N\) for some \(z\in N\) and \(r>0\).
We look at the circle path \(\gamma\) whose image \(\gamma^*\) 
is the boundary of \(\overline{B(z;r/2)}\). 
By Theorem~10.28 the derivatives 
\({f_n}\) converge uniformly to \(f'\) on \(\gamma*\).
But now Theorem~10.43\ich{a} gives the following
\begin{equation*}
0 = \lim_{n\to\infty} N_{f_n} = N_f = 1
\end{equation*}
contradiction.


%%%%%%%%%%%%%% 21
\begin{excopy}
Suppose \(f\in H(\Omega)\), \(\Omega\) contains the closed unit disc, and
\(|f(z)| < 1\) if \(|z|=1\).
How many fixed points must $f$ have in the disc?
That is, how many solutions does the equation \(f(z)=z\) have there?
\end{excopy}

By looking at \(h(z) = z/2\) we can see that the minimal number of
solutions for such $f$ cannot exceed $1$.

Let \(U_0 = \{z\in\C: |z|\leq 1\}\) be the unit circle.
Define by induction \(U_k = f(U_{k-1})\) for \(k\geq 1\).
Also by induction we can see that \(U_k \subset U_{k-1}\).
These are compact sets and thus have non empty intersection
\(X = \cap_{k\in\Z^+} U_k\). Clearly \(f(X) = X\).

\index{maximum modulus}
By the maximum modulus Theorem~10.24
\index{Cauchy's estimates}
and Cauchy's estimates Theorem~10.26, we have \(|f'(z)|<1\) for all \(z\in U\).
Moreover, \(s = \max_{z\in U}|f'(z)|<1\).
Hence the diameter of the compact sets \(\{U_k\}_{k\in\Z^+}\) is decreasing,
such that \(\diam(U_k) \leq s^k\). Hence $X$ must be a singleton.

%%%%%%%%%%%%%% 22
\begin{excopy}
Suppose \(f\in H(\Omega)\), \(\Omega\) contains the closed unit disc,
\(|f(z)| > 2\) if \(|z|=1\), and \(f(0)=1\).
Must $f$ have a zero in the unit disc?
\end{excopy}

If by negation there was no zero, then \(g = 1/f \in H(\Omega)\)
and 
\begin{equation*}
|g(0)| = 1 > 1/2 > \max\{|f(z)|: |z|=1\}
\end{equation*}
that contradicts the
\index{Maximum Modulus}
Maximum Modulus Theorem~10.24.

%%%%%%%%%%%%%% 23
\begin{excopy}
Suppose \(P_n(z) = 1 + z/1! + \cdots + z^n/n!\),
\(Q_n(z) = P_n(z) - 1\), where \(n=1,2,3,\ldots\),
% none of the functions
What can you say about the location of the zeros
of \(P_n\) and \(Q_n\) for large $n$?
Be as specific as you can.
\end{excopy}

The polynomials \(P_n\) have zeros that converge to infinity.
More accurately, for each \(R<0\) there exists some \(m<\infty\)
such that \(P_n\) has \emph{no} zeros in \(B(0;R)\) for all \(n>m\).
Otherwise, there would be an increasing sub-sequence of indices 
\(\{s_n\}_{n\in\N}\) and zeros \(\{z_n\}_{n\in\N}\) in \(B(0;R)\) 
such that \(P_{s_n}(z_n) = 0\) and \(\lim_{n\to\infty} z_n = z\in B(0;R)\).
Now \(\lim_{n\to\infty} P_{s_n}(z) = \exp(z)\) uniformly on \(B(0;R)\)
which leads to the \(\exp(z) = 0\) contradiction.

In addition to \(Q_n(0)=0\) for all \(n\in\N\),
the polynomials \(Q_n\) have zeros that converge to \(\{2\pi i k: \;k\in\Z\}\).
But the other roots diverge. More accurately,
for all \(R<0\) and \(r>0\)  there exists some \(m<\infty\)
such that \(Q_n\) has \emph{no} zeros in
\begin{equation*}
\{z\in\C:\; |z|\leq R \;\wedge\; \Re(z)\geq r\}
\end{equation*}
for all \(n>m\).


%%%%%%%%%%%%%% 24
\begin{excopy}
Prove the following form of
\index{Rouche}
Rouche's theorem:
Let \(\Omega\) be the interior of a compact set $K$ in the plane.
Suppose $f$ and $g$ are continuous on $K$ and holomorphic in \(\Omega\),
and \(|f(z)-g(z)|<|f(z)|\) for all \(z\in K\setminus \Omega\).
Then $f$ and $g$ have the same number of zeros in \(\Omega\).
\end{excopy}

Let $D$ be a connected component of  \(\Omega\).
We will show that $f$ and $g$ have the same number of zeros in $D$.
If by negation $f$ has have infinite number of zeros in any such $D$,
then then \(f_{\restriction D}=0\) and by continuity
 \(f_{\restriction \partial D}\) but this contradicts the assumption 
that \(|f(z)-g(z)|<|f(z)|\) for all \(z\in K\setminus \Omega\).
Thus the number of zeros of $f$ in $D$ is finite.
Once we establish that this number of zeros of $g$ in $D$ is the same,
we can sum up the zeros in all such components and get the desired
result. Let \(N = \{z\in D:\; f(z) = 0\}\).

The boundary \(\partial D \subset K\). For any \(\epsilon>0\),
by introducing sufficiently small squares,
we can find a path \(\gamma_\epsilon\) that is ``\(\epsilon\)-near'' the boundary.
More accurately, \(d(z,K)<\epsilon\) for all \(z\in \gamma_\epsilon^*\).

Since the number of zeros of $f$ in $D$ is finite, we can pick 
\(\epsilon < d(N,\partial D)\).
Hence \(f(z)\neq 0\) for all \(z\in \gamma_\epsilon^*\).
Since $f$ and $g$ are continuous, we can further require \(\epsilon\)
to sufficiently small such that the inequality \(|f(z)-g(z)|<|f(z)|\)
also holds for all \(z\in\gamma_\epsilon^*\).
The needed step to complete is to apply Theorem~10.43\ich{b}
which shows that the numbers of zeros of $f$ and $g$ in $D$ are the same.


%%%%%%%%%%%%%% 25
\begin{excopy}
Let $A$ be the
\index{annulus}
annulus \(\{x: r_1 < |z| < r_2\}\), where \(r_1\) and \(r_2\) are given
positive numbers.
\begin{itemize}

\itemch{a} Show that the Cauchy formula
\begin{equation*}
f(z) = \itwopii \left(\int_{\gamma_1} + \int_{\gamma_2}\right)
       \frac{f(\zeta)}{\zeta - z}\,d\zeta
\end{equation*}
is valid under the following conditions: \(f\in H(A)\),
\begin{equation*}
r_1 + \epsilon < |z| < r_2 - \epsilon,
\end{equation*}
and
\begin{equation*}
\gamma_1(t) = (r_1+\epsilon)e^{-it},
\qquad
\gamma_2(t) = (r_2-\epsilon)e^{it},
\qquad
(0\leq t \leq 2\pi).
\end{equation*}

\itemch{b} Show by means of \ich{a} that every \(f\in H(A)\)
can be decmposed info a sum \(f=f_1+f_2\), when \(f_1\) is holomorphic
outside \(\overline{D}(0;r_1)\)
and  \(f_2 \in H(D(0;r_1))\).
The decomposition is unique if we require that 
\(f_1(z)\to 0\) as \(|z|\to\infty\).

\itemch{c} Use this decomposition to associate with each \(f\in H(A)\) its
so-called
\index{Laurent series}
``Laurent series''
\begin{equation*}
\sum_{-\infty}^\infty c_n z^n
\end{equation*}
which converges to $f$ in $A$. Show that there is only one such series for 
each $f$. Show that it converges to $f$ uniformly on compact subsets of $A$.

\itemch{d} If \(f\in H(A)\) and $f$ is bounded in $A$, show that the components
\(f_1\) and \(f_2\) are also bounded.

\itemch{e} How much of the foregoing can you extend to the case \(r_1=0\)
(or \(r_2=\infty\), or both)?

\itemch{f} How much of the foregoing can you extend  to region bounded 
by finitely many (more than two) cycles?
\end{itemize}
\end{excopy}

See \cite{Gamelin2003} Chapter~VI, Section~1.
\begin{itemize}

\itemch{a}
This result follows by applying \index{Cauchy} Cauchy's Theorem~10.35
for \(\Gamma = \gamma_1 + \gamma_2\).

\itemch{b}
We define
\begin{align*}
f_1(z) &= \itwopii \int_{\gamma_1} \frac{f(\zeta)}{\zeta - z}\,d\zeta \\
f_2(z) &= \itwopii \int_{\gamma_2} \frac{f(\zeta)}{\zeta - z}\,d\zeta\,.
\end{align*}
Clearly \(f = f_1 +f_2\) and \(\lim_{z\to\infty} f_1(z) = 0\).
Say  \(f = g_1 +g_2\) and \(\lim_{z\to\infty} g_1(z) = 0\).
Then \(f_1 - g_1 = g_2 - f_2\) in $A$.
The function \(f_1 - g_1\) is holomorphic outside of \(\gamma_1\)
and the function \(g_2 - f_2\) is holomorphic inside \(\gamma_2\).
Since they agree on the intersection, we have an entire function $h$
\begin{align*}
h(z) &= g_2(z) - f_2(z) \qquad (|z| < r_1) \\
h(z) &= f_1(z) - g_1(z) \qquad (|z| < r_2)
\end{align*}
But \(\lim_{z\to\infty} h(z) = 0\), hence \(h(z)=0\) for all \(z\in\C\)
and the uniqueness follows.

\itemch{c}
We have
\begin{equation*}
f_2(z) = \sum_{n=0}^\infty c_n z^n
\end{equation*}
We can compute the coefficients \(c_n\) using Theorem~10.7
\begin{equation*}
c_n = \itwopii \int_{\gamma_2} \frac{f(\zeta)}{\zeta^{n+1}} \qquad (n\geq 0).
\end{equation*}

Since \(\lim_{z\to\infty} f_1(z) = 0\), we can put \(z=1/w\) and define
\begin{equation*}
g_1(w) = f(1/w) = f(z) \qquad (|z|>r  \Leftrightarrow |w|<1/r)
\end{equation*}
and with \(g_1(0)=0\), we have \(g_1 \in H(\{w\in\C: |w|<1/r\})\).
Hence it as a power series 
\begin{align*}
g_1(w) &= \sum_{n=1}^\infty b_n w^n \\
f_1(z) &= g_1(1/z) = \sum_{n-1}^\infty b_n z^{-n}
\end{align*}
By putting \(c_{-n} = b_n\) we get the desired representation for $f$.

For each \(n\in\Z\), we consider \(f(z)/z^{n+1}\). Now for any sub-annulus of $A$
we can find sufficiently small \(\epsilon>0\) such that it is contained
between the circles of \(\gamma_1^*\) and  \(\gamma_2^*\) and we can compute
\begin{equation*}
\int_{\gamma_1^*} \frac{f(z)\,dz}{z^{n+1}}
 = \int_{\gamma_1^*} \frac{1}{z^{n+1}}\sum_{n\in\Z} c_k z^k\,dz
 = \sum_{n\in\Z} c_k \int_{\gamma_1^*} z^{k-n-1}\,dz
 = 2\pi i c_{n}
\end{equation*}
Since the last integral vanishes whenever \(k-n-1\neq -1\).
Hence
\begin{equation*}
c_n = \itwopii \int_{\gamma_1^*} \frac{f(z)\,dz}{z^{n+1}}.
\end{equation*}

By the arguments of section~10.5, 
the power series of
\begin{equation*}
f_2(z) = \sum_{n=0}^\infty c_n z^n
\end{equation*}
converges absolutely in \(\overline{D(0;\rho)}\) for every \(\rho<r_2\).
Similarly by looking on \(g_1\), the power series of 
\begin{equation*}
f_1(z) = \sum_{n=-1}^{-\infty} c_n z^n
\end{equation*}
converges absolutely in \(\{z\in\C: |z|\geq \rho\}\) for every \(\rho> r_1\).
Now for every compact \(K\subset A\) we can find 
\(\rho_1\) and \(\rho_2\) such that
\begin{equation*}
K \subset \{z\in\C: \rho_1 \leq |z| \leq \rho_2\}
\qquad \textnormal{and}\qquad 
r_1 < \rho_1 < \rho_2 < r_2\
\end{equation*}
and clearlt the whole power series converges absolutely in $K$.x

If by negation there was another power series, 
then by  positive and negative powers split, we would get 
another decomposition for $f$ that contradicts the uniqueness established 
in~\ich{b}.

\itemch{d}
Put \(r_3 = (r_1+r_2)/2\). 
Clearly \(f_2\) is bounded in~\(B_2 = \overline{D(0;r_3)}\)
and  \(f_1\) is bounded in~\(B_1 = \C\setminus D(0;r_3)\).
If \(f = f_1+f_f2\) is bounded in $A$, then 
\(f_1\) must be bounded in \(A\setminus B_2\) and
\(f_2\) must be bounded in \(A\setminus B_1\).
Consequently, \(f_1\) and \(f_2\) are bounded on~$A$.

\itemch{e}
The case \(r_1=0\) and \(r_2=1\) applies for the function 
\begin{equation*}
f(z) = \sum_{n\geq -1} z^n = 1/z + 1/(1-z)
\end{equation*}

The cases of \(r_2=\infty\) requires different interpration of ``\(\epsilon\)''
for selecting \(\gamma_2\). We can use the following condition;
for any \(M<\infty\), let \(\gamma_2(t) = Me^{it}\) for \(0\leq t\leq 2\pi\).
Function that can be applied for such case have essential singularity 
in infinity, for example \(f(z) = \sin(z)\).
This can be combined with \(r_1=0\) as well (add \(1/z\)).

\itemch{f}
With $n$ circles all centered at the origin, there are \(n-1\) annulus regions.
All the above applies to each such annulus.

\end{itemize}

%%%%%%%%%%%%%% 26
\begin{excopy}
It is required to extend the function 
\begin{equation*}
\frac{1}{1-z^2} + \frac{1}{3-z}
\end{equation*}
in the series of the form \(\sum_{-\infty}^\infty c_n z^n\).

How many such expansions are there?
In which region is each of them valid?
Find the coefficients \(c_n\) explicitly for each of these expansions.
\end{excopy}

We use the identity:
\begin{equation*}
f(z) = \frac{1}{1-z^2} + \frac{1}{3-z}
= \frac{1}{2}\left(\frac{1}{z-1} - \frac{1}{z+1}\right) + \frac{1}{3-z}
\end{equation*}

Consider the following case:
\begin{alignat*}{2}
\frac{1}{z-1} &= \sum_{n=0}^\infty z^n && \qquad (|z|<1) \\
\frac{1}{z+1} &= \sum_{n=0}^\infty (-1)^n z^n  && \qquad (|z|<1) \\
\frac{1}{3-z} &= \sum_{n=0}^\infty 3^{-(n+1)} z^n  && \qquad (|z|<3) \\
\frac{1}{z-1} 
 &= \frac{1}{z(1-(1/z))} = 
  \sum_{n<0} z^n && \qquad (|z|>1) \\
\frac{1}{z+1} 
 &= \frac{1}{z(1-(-1/z))} 
 = \sum_{n<0} (-1)^{n+1} z^n && \qquad (|z|>1) \\
\frac{1}{3-z} 
  &= \frac{-1}{z} \cdot \frac{1}{1-3/z} 
  = \sum_{n<0}^\infty (-1)^{n+1}\cdot 3^{n} z^n  && \qquad (|z|>3)
\end{alignat*}

Gathering the results:
\begin{alignat*}{2}
f(z) &= \sum_{n=-1}^{-\infty} 3^{-(n+1)} z^n + (2/3)z^0 + \sum_{n=1}^\infty z^n  
  & \qquad & (|z|<1) \\
f(z) &= \sum_{n=-1}^{-\infty} \left((-1)^{n+1} + 1\right)z^n + 
        \sum_{n=0}^\infty 3^{-(n+1)} z^n 
     & \qquad & (1<|z|<3) \\
f(z) &= \sum_{n=-1}^{-\infty} \left(1 - (-1)^{n}(1+3^n) \right)z^n
     & \qquad & (|z|>3)
\end{alignat*}


%%%%%%%%%%%%%% 27
\begin{excopy}
Suppose \(\Omega\) is a horizontal strip determined by the inequalities
\(a<y<b\), say.
Suppose \(f\in H(\Omega)\) and \(f(z) = f(z+1)\) for all \(z\in\Omega\).
Prove that $f$ has a Fourier expansion in \(\Omega\),
\begin{equation*}
f(z) = \sum_{-\infty}^\infty c_n e^{2\pi inz},
\end{equation*}
which converges uniformly in \(\{z: a+\epsilon \leq y \leq b - \epsilon\}\),
for every \(\epsilon > 0\).
\emph{Hint:} The map \(z\to e^{2\pi i z}\) converts $f$ to a function in an 
annulus.

Find the integral formula by means on which the coefficients \(c_n\) can be 
computed from $f$.
\end{excopy}

Denote \(z=x+iy\) with \(x,y\in\R\). The map
\begin{equation*}
\varphi(z) = e^{2\pi i z} = e^{2\pi i (x+iy)} = e^{-2\pi y}\cdot e^{2\pi i x}
\end{equation*}
maps the stripe \(\Omega = \{x+iy\in\C: x\in\R \wedge a<y<b\}\)
onto the annulus
\begin{equation*}
A = \{z\in\C: e^{-2\pi b} < |z| < e^{-2\pi a}\}.
\end{equation*}
The mapping is not injective, but if \(\varphi(x_1+iy_1) = \varphi(x_2+iy_2)\)
where \(x_i,y_i\in\R\) then \(y_1=y_2\) and \(x_2-x_1\in\Z\).
Hence we can define \(g:A\to\C\) by \(g(w) = f(z)\) where \(w=\varphi(z)\)
which is well defined since \(f(z)=f(z+1)\) for all \(z\in\Omega\).

But the mapping holomorphic and 
is locally injective, that is for each \(z\in\Omega\)
there is a neighborhood \(V_z\subset\Omega\) such that \(\varphi\)
is one-to-one in \(V_z\).
Hence \(g\in H(A)\) by Theorem~1.30.

By previous exercise there exists a sequence \((c_n)_{n\in\Z}\)
of complex numbers such that 
\begin{equation*}
g(w) = \sum_{n\in\Z} c_n w^n \qquad (w\in A).
\end{equation*}
But
\begin{equation*}
c_nw^n = c_n  \left(e^{2\pi i z}\right)^n = e^{2\pi i nz}
\end{equation*}
substituting the last expression in the previous power-series gives
the desired \index{Fourier} Fourier expansion.


%%%%%%%%%%%%%% 28
\begin{excopy}
Suppose \(\Gamma\) is a closed curve in the plane, with parameter interval
\([0,2\pi]\).
Take \mbox{\(\alpha \notin \Gamma^*\).}
Approximate \(\Gamma\) uniformly by trigonometric polynomials \(\Gamma_n\).
Show that \(\Ind_{\Gamma_m}(\alpha) = \Ind_{\Gamma_n}(\alpha)\) if $m$ and $n$
are sufficiently large. Define this common value to be \(\Ind_{\Gamma}(\alpha)\).
Prove that the result does not depend on the choice of \(\{\Gamma_n\}\);
prove that Lemma~10.39 is now true for closed curves, and use this to give 
a~different proof of Theorem~1.40.
\end{excopy}

By Theorem~4.25 we for each $n$, we can find a trigonometric polynomial
\(P_n:\T\to\C\) such that \(\max\{|\Gamma(t)-P_n(t)|: t\in\T\}< 1/n\).
The continuous image \(\Gamma^*\) is compact, hence 
if \(\alpha\notin \Gamma^*\) then \(d(\alpha,\Gamma^*)>0\).
Hence, we can find some \(k<\infty\) such that \(d(\alpha,\Gamma^*)>1/k\).
Now by Lemma~10.39
\begin{equation} \label{eq:ex10.29:Pmn}
\Ind_{P_m}(\alpha) = \Ind_{P_n}(\alpha) 
\end{equation}
for all \(m, n \geq 2k\).
Hence 
\begin{equation*}
\Ind_\Gamma(\alpha) := \lim_{n\to\infty}{P_m}(\alpha)
\end{equation*}
is well defined and is independent of the choice 
of the trigonometric polynomials, since with any replacment of them,
Lemma~10.39 could still be applied and \eqref{eq:ex10.29:Pmn} will hold.

Now we can simplify the proof of Theorem~10.40.
Instead of introducing the paths 
\begin{equation*}
\gamma_k(s) 
 = H\left(\frac{i}{n},\frac{k}{n}\right)(ns + 1 -i)
 + H\left(\frac{i-1}{n},\frac{k}{n}\right)(i - ns)
\qquad (i\in\N_n, \; k\in \Z_n, \; i-1\leq ns\leq i)
\end{equation*}
we can use the curves (now, not necessarily paths)
\begin{equation*}
\gamma_k(s) = H(s,k/n) \qquad (k\in \Z_n)
\end{equation*}
and proceed with the original proof.


%%%%%%%%%%%%%% 29
\begin{excopy}
Define 
\begin{equation*}
f(z) = \frac{1}{\pi} \int_0^1 r\,dr \int_{-\pi}^\pi \frac{d\theta}{re^{i\theta}+z}.
\end{equation*}
Show that \(f(z) = \overline{z}\) if \(|z|<1\) and that 
\(f(z) = 1/z\) if \(|z|\geq 1\).

Thus $f$ is not holomorphic in the unit disc, 
although the integrand is a holomorphic function of~$z$. 
Note the contrast between this, on the one hand, and Theorem~10.7 
and Exercise~16 on the other.

\emph{Suggestion:} Compute the inner integral separately 
for \(r<|z|\) and for \(r>|z|\).
\end{excopy}

We first compute the inner intergral. We define
\begin{equation*}
h(r,z) = \int_{-\pi}^\pi \frac{d\theta}{re^{i\theta}+z}.
\end{equation*}
\paragraph{Assuming \(|r|<z\).} In this case \(z\neq 0\) and we have
\begin{equation*}
\frac{1}{re^{i\theta}+z}
= \frac{1}{z}\cdot\frac{1}{1-((-r/z)e^{i\theta})}
= \frac{1}{z}\sum_{n=0}^\infty \left(-\frac{r}{z}\right)^n e^{in\theta}.
\end{equation*}
Hence
\begin{align*}
% \int_{-\pi}^\pi \frac{d\theta}{re^{i\theta}+z}
h(r,z)
&= \frac{1}{z} \int_{-\pi}^\pi
    \left( \sum_{n=0}^\infty \left(-\frac{r}{z}\right)^n e^{in\theta} 
    \right)\,d\theta
 = \frac{1}{z} \sum_{n=0}^\infty \left(-\frac{r}{z}\right)^n 
     \int_{-\pi}^\pi e^{in\theta} \,d\theta
 = \frac{1}{z} \left(-\frac{r}{z}\right)^0 \cdot 2\pi \\
&= 2\pi/z
\end{align*}


\paragraph{Assuming \(|r|>z\).} In this case we have
\begin{equation*}
\frac{1}{re^{i\theta}+z}
= \frac{1}{re^{i\theta}}\cdot\frac{1}{1-(-z/r)e^{-i\theta}}
= \frac{1}{re^{i\theta}}\sum_{n=0}^\infty (-z/r)^n e^{-in\theta}.
\end{equation*}
Hence
\begin{equation*}
% \int_{-\pi}^\pi \frac{d\theta}{re^{i\theta}+z}
h(r,z)
 = \int_{-\pi}^\pi \frac{1}{re^{i\theta}}
    \left(\sum_{n=0}^\infty \left(\frac{-z}{r}\right)^n e^{-in\theta}\right) d\theta 
 = \sum_{n=0}^\infty \frac{(-z)^n}{r^{n+1}}
    \int_{-\pi}^\pi e^{-i(n+1)\theta}\,d\theta 
 = 0.
\end{equation*}

Back to the outer integral, we again have two cases.
\paragraph{Assume \(|z|\leq 1\).}
\begin{align*}
f(z) 
&= \frac{1}{\pi} \left(
   \int_0^{|z|} r\cdot h(r,z)\,dr + \int_{|z|}^1 r\cdot h(r,z)\,dr \right) 
 = \frac{1}{\pi} \left(
      \int_0^{|z|} \frac{2\pi r}{z}\,dr +  0\right) \\
&= (2/z) \left.\left(r^2/2\right)\right|_0^{|z|}
 = |z|^2/z
 = \overline{z}\,.
\end{align*}

\paragraph{Assume \(|z|\geq 1\).}
\begin{align*}
f(z) 
&= \frac{1}{\pi} \int_0^1 r\cdot h(r,z)\,dr 
 = \frac{1}{\pi} \int_0^1 (2\pi r/z) \,dr 
 = (1/z)\left.\left(r^2\right)\right|_{0}^1
 = 1/z\,.
\end{align*}


%%%%%%%%%%%%%% 30
\begin{excopy}
Let \(\Omega\) be the plane minus two points, and show that some closed paths
\(\Gamma\) in \(\Omega\) satisfy assumtion~(1) of Theorem~10.35 without being
\index{null-homotopic}
\index{null!homotopic}
null-homotopic in~\(\Omega\).
\end{excopy}

Intuitively, make a double ``eight''-pattern, where each ``eight''
is in different direction.
Say the two deleted points are $1$ and $3$.
Let's for the path as the a polyline passing thru the following
points in the given order:
{\newcommand{\qtoq}{\;\to\;}
\begin{align*}
2 \qtoq 3-i \qtoq 4 \qtoq 3+i &\qtoq 2 \qtoq 1-i \qtoq 0 \qtoq 1+i \qtoq \\
2 \qtoq 3+i \qtoq 4 \qtoq 3-i &\qtoq 2 \qtoq 1+i \qtoq 0 \qtoq 1-i \qtoq 2.
\end{align*}
}
Of course, there are similar paths that are nicer
in the sense that they cross themselves only 
in a finite number of points~($5$?).

%%%%%%%%%%%%%%%%%
\end{enumerate}



 \setcounter{chapter}{10}  %%%%%%%%%%%%%%%%%%%%%%%%%%%%%%%%%%%%%%%%%%%%%%%%%%%%%%%%%%%%%%%%%%%%%%%%
%%%%%%%%%%%%%%%%%%%%%%%%%%%%%%%%%%%%%%%%%%%%%%%%%%%%%%%%%%%%%%%%%%%%%%%%
%%%%%%%%%%%%%%%%%%%%%%%%%%%%%%%%%%%%%%%%%%%%%%%%%%%%%%%%%%%%%%%%%%%%%%%%
%chapter 11
\chapterTypeout{Harmonic Functions}

%%%%%%%%%%%%%%%%%%%%%%%%%%%%%%%%%%%%%%%%%%%%%%%%%%%%%%%%%%%%%%%%%%%%%%%%
%%%%%%%%%%%%%%%%%%%%%%%%%%%%%%%%%%%%%%%%%%%%%%%%%%%%%%%%%%%%%%%%%%%%%%%%
\section{Notes}

%%%%%%%%%%%%%%%%%%%%%%%%%%%%%%%%%%%%%%%%%%%%%%%%%%%%%%%%%%%%%%%%%%%%%%%%
\index{Laplacian}
\subsection{The Laplacian}

\newcommand*{\partialby}[1]{\frac{\partial}{\partial #1}}
\newcommand*{\fracpart}[2]{\frac{\partial #1}{\partial #2}}
\newcommand*{\dpartial}[2]{\frac{\partial^2 #1}{\partial #2^2}}
\newcommand*{\px}{\partialby x}
\newcommand*{\py}{\partialby y}

Assuming \(f = u + iv\) and \(f_{xy} = f_{yx}\) we have
\begin{align*}
4\partial\tilde{\partial}f 
 &= 4\partial\left(\half\left(\px + i\py\right)(u + iv)\right)
  = 2\partial\bigl(u_x + iv_x + i(u_y + iv_y)\bigr) \\
 &= \left(\px - i\py\right)\bigl(u_x - v_y + i(v_x + u_y)\bigr) \\
 &= u_{xx} - v_{yx}  + i(v_{xx} + u_{yx}) 
    -i\bigl(u_{xy} - v_{yy} + i(v_{xy} + u_{yy})\bigr) \\
 &= (u + iv)_{xx} + (u + iv)_{yy} = \Delta f.
\end{align*}

\subsubsection{Polar Coordinates}

Using
\begin{equation*}
x = r\cos\theta \qquad
y = r\sin\theta \qquad
r = \sqrt{x^2+y^2} \qquad
\tan \theta = y/x
\end{equation*}
Asssume \(u(x,y)\) is sufficiently differentiable with continuous
partial derivatives.

Using the chain rule
\begin{equation*}
\fracpart{u}{x} 
  = \cos \theta \fracpart{u}{r} 
    - \frac{\sin \theta}{r}\fracpart{u}{\theta} 
  \qquad
\fracpart{u}{y} 
  = \sin \theta \fracpart{u}{r} 
    + \frac{\cos \theta}{r}\fracpart{u}{\theta}
\end{equation*}

Continuing
\begin{align*}
\dpartial{u}{x}
 &= \cos^2\theta \dpartial{u}{r}
    - \frac{2\sin\theta \cos\theta}{r} 
      \frac{\partial^2 u}{\partial r\,\partial \theta}
    + \frac{\sin^2 \theta}{r^2} \dpartial{u}{\theta}
    + \frac{\sin^2 \theta}{r} \fracpart{u}{r}
    + \frac{2\sin\theta \cos\theta}{r^2}\fracpart{u}{\theta} \\
\dpartial{u}{y}
 &= \sin^2\theta \dpartial{u}{r}
    + \frac{2\sin\theta \cos\theta}{r} 
      \frac{\partial^2 u}{\partial r\,\partial \theta}
    + \frac{\cos^2 \theta}{r^2} \dpartial{u}{\theta}
    + \frac{\cos^2 \theta}{r} \fracpart{u}{r}
    - \frac{2\sin\theta \cos\theta}{r^2}\fracpart{u}{\theta}
\end{align*}
Adding the above gives
\begin{equation*}
\Delta u 
= \dpartial{u}{x} + \dpartial{u}{y}
 = \dpartial{u}{r} + \frac{1}{r}\fracpart{u}{r} 
    + \frac{1}{r^2}\dpartial{u}{\theta}
 = \frac{1}{r}\partialby{r}\left(r\fracpart{u}{r}\right)
    + \frac{1}{r^2}\dpartial{u}{\theta}
\end{equation*}


%%%%%%%%%%%%%%%%%%%%%%%%%%%%%%%%%%%%%%%%%%%%%%%%%%%%%%%%%%%%%%%%%%%%%%%%
\subsection{Proof of Theorem~11.9}

The proof of Theorem~11.9 refers to Theorem~10.7
for showing that $f$ is holomorphic. But instead, it should 
refer to Exercise~16 of Chapter~10 (\ref{ex:10.16}).

%%%%%%%%%%%%%%%%%%%%%%%%%%%%%%%%%%%%%%%%%%%%%%%%%%%%%%%%%%%%%%%%%%%%%%%%
\subsection{Harnack's Theorem}

\index{Harnack}
Let's look a the proof of Harnack's Theorem~11.11.
The first double inequality has a minor error.
The middle expression is missing square sign in the denominator.
It should be
\begin{equation*}
\frac{R^2 - r^2}{R^2 - 2rR\cos(\theta - t) + r^{\mathbf{2}}}
\end{equation*}

% The proof of Harnack's Theorem~11.11, specifically 
The second
double-inequality makes use of \textbf{11.10}(1).

\iffalse
of the following observation.

Say $u$ is a real harmonic function, so \(u(z) = \Re(f(z)\) for some
\(f\in H(\Omega)\). 
Let \(\gamma(t) = a + Re^{it}\)
and \(\Gamma = \{\gamma(t): -\pi \leq t < \pi\}\). Now
\(\gamma`(t) = Rie^{it}\) and
\begin{align*}
u(a) 
 &= \Re(f(a))
 = 
   \Re\left(
     \dtwopii
     \int_{\Gamma} \frac{f(a+z)}{(a+z)-a}\,dz
   \right) 
 \\
 &= \frac{1}{2\pi} 
   \Re\left(
    \frac{1}{i}
     \int_{-\pi}^\pi \frac{f(a+\gamma(t))}{Re^{it}}\cdot Rie^{it}\,dt
   \right).
 \\
 &= \frac{1}{2\pi} 
   \Re\left(
     \int_{-\pi}^\pi f(a+\gamma(t))\,dt
   \right).
\end{align*}
\fi


%%%%%%%%%%%%%%%%%%%%%%%%%%%%%%%%%%%%%%%%%%%%%%%%%%%%%%%%%%%%%%%%%%%%%%%%
%%%%%%%%%%%%%%%%%%%%%%%%%%%%%%%%%%%%%%%%%%%%%%%%%%%%%%%%%%%%%%%%%%%%%%%%
\section{The Exercises} % pages 249-252

%%%%%%%%%%%%%%%%%
\begin{enumerate}
%%%%%%%%%%%%%%%%%

%%%%%%%%%%%%%% 
\begin{excopy}
Suppose $u$ and $v$ are real harmonic functions in a plane regular \(\Omega\).
Under what conditions is \(uv\) harmonic?
(Note that the answer depends strongly on the fact that the question
is one about \emph{real} functions.)
Show that \(u^2\) cannot be harmonic in \(\Omega\), unless $u$ is constant.
For which \(f \in H(\Omega)\) is \(|f|^2\) harmonic?
\end{excopy}

Compute
\begin{equation*}
(uv)_{xx} 
= \left((uv)_x\right)_x
= \left(u_xv + uv_x\right)_x
= u_{xx}v + 2u_xv_x + uv_{xx}
\end{equation*}

Hence
\begin{align*}
\Delta(uv) 
 &= u_{xx}v + 2u_xv_x + uv_{xx} + u_{yy}v + 2u_yv_y + uv_{yy} \\
 &= \Delta(u)v + u\Delta(v) + 2(u_xv_x + u_yv_y).
\end{align*}

So \(uv\) is harmonic when \(u_xv_x + u_yv_y = 0\) in \(\Omega\).
Clearly \(\Delta(u^2) = 2(u_x^2 + u_y^2) \geq 0\)
and equality holds iff \(u_x = u_y = 0\).

%%%%%%%%%%%%%% 
\begin{excopy}
Suppose $f$ a complex function in a region \(\Omega\) and both
$f$ and \(f^2\) are hannonic in \(\Omega\). Prove that
either $f$ or \(\overline{f}\) are holnmorphic in \(\Omega\).
\end{excopy}

By the previous exercise, \((f_x)^2 + (f_y)^2 = 0\).
Regardless whether \(f_x = f_y = 0\) holds or not, we have
\begin{equation*}
f_x = \pm i\,f_y.
\end{equation*}
By continuity, only one variant of \(\pm\) holds for all \(\Omega\).
Put \(f = u + iv\) with real valued functions $u$ and $v$.
Now
\begin{equation*}
u_x + iv_x = -v_y + iu_y
\qquad\textnormal{or}\qquad
u_x + iv_x = v_y - iu_y.
\end{equation*}
Hence the Cauchy-Riemann equation holds for \(\overline{f}\) or $f$.

%%%%%%%%%%%%%% 3
\begin{excopy}
If $u$ is a harmonic function in a region \(\Omega\),
what can you say about the set of points at which the
gradient of $u$ is $0$? (Thus is the set which \(u_x = u_y = 0\).)
\end{excopy}

% From Section~1.11, every harmonic function has continuous partial derivative
% of all orders.
We will show that
the vanishing set $K$ is either the whole region where $u$ is constant
in each connected component of \(\Omega\),
or it consists of just isolated points.

By Section~11.10, every real harmonic function is locally the real part
of holomorphic function.

Assume $u$ is harmnonic in a connected region
(\wlogy, it is \(\Omega\) and the set
\begin{equation*}
 K = \{z\in\Omega: u_x(z) = u_y(z) = 0\}
\end{equation*}
has accumulation point \(z_0 \in \Omega\).
Note that both \(\Re(u)\) and \(\Im(u)\) are harmnonic.
By the previous, remark there exist a neighborhood \(V\subset\Omega\)
and holomorphic functions $f$ and $G$ defined on $V\ni z_0$ such that
\begin{equation*}
 u(z) = \Re(f(z)) + i\Re(g(z)) \qquad (z\in V).
\end{equation*}
Now for \(z\in V\)
\begin{align*}
u_x(z) &= \Re(f_x(z)) + i\Re(g_x(z)) = (\Re(f))_x(z) + i(\Re(g))_x(z) \\
u_y(z) &= \Re(f_y(z)) + i\Re(g_y(z)) = (\Re(f))_y(z) + i(\Re(g))_y(z)
\end{align*}
and Cauchy-Riemann equations show that 
\begin{alignat*}{2}
\Re(u)_x &= \Re(f)_x = \Im(f)_y &\qquad \Im(u)_x &= \Re(g)_x = \Im(g)_y \\
\Re(u)_y &= \Re(f)_y = -\Im(f)_x &\qquad \Im(u)_y &= \Re(g)_y = -\Im(g)_x 
\end{alignat*}
For \(z\in K\), the above functions vanish, hence
\begin{align*}
f'(z) = g'(z) = 0 \qquad (z\in K).
\end{align*}
By Theorem~10.18 \(f'(z)=g'(z) = 0\) for all \(z\in V\)
and consequently for all \(z\in \Omega\).
Hence $f$ and $g$ are constant functions and so is $u$.

%%%%%%%%%%%%%% 4
\begin{excopy}
Prove that every partial derivative of every harmonic function is harmonic.

Verify, by direct computation, that \(P_r(\theta - t)\) is, for each fixed $t$,
a harmonic function of \(re^{i\theta}\).
Deduce (without referring to holomorphic functions) that the Poisson integral
\(P[d\mu]\) of every finite
Borel measure \(\mu\) on $T$ is harmonic in $U$, by showing that every partial
derivative of \(P[d\mu]\) is equal to
the integral of the corresponding partial derivative of the kernel.
\end{excopy}

See also \cite{Lang199304} Chapter~\textsf{VIII} Section~\S3 Example~3.

Let
\begin{equation*}
f(z) = \frac{e^{it} + z}{e^{it} - z} = \frac{u + x+iy}{u - x - iy}.
\end{equation*}

First order differentiation:
\begin{align*}
\partialby{x}f(z) 
&= \partialby{x} \frac{u + x+iy}{u - x - iy}
 = \frac{(u - x - iy) + (u + x+iy)}{(u-z)^2}
 = \frac{2u}{(u-z)^2}
 \\
\partialby{y}f(z) 
&= \partialby{y} \frac{u + x+iy}{u - x - iy}
 = \frac{i(u - x - iy) + i(u + x+iy)}{(u-z)^2}
 = \frac{2iu}{(u-z)^2}
\end{align*}

Second order differentiation:
\begin{align*}
\dpartial{}{x}f(z) 
 &= \partialby{x}\frac{2u}{(u-x-iy)^2}
  = \frac{-2u(2x+2iy-2u)}{(u-z)^2}
  = 4u\frac{(-x-iy+u)}{(u-z)^2}
 \\
\dpartial{}{y}f(z) 
 &= \partialby{y}\frac{2iu}{(u-x-iy)^2}
  = \frac{-2iu(2ix-2y-2iu)}{(u-z)^2}
  = 4u\frac{(x+iy-u)}{(u-z)^2}
\end{align*}

By summing the above, and noting that the partial differentiation
commutes with the \(\Re\) operator,
we get 
\begin{align*}
\Delta P_r(\theta - t)
= \left(\dpartial{}{x}+\dpartial{}{y}\right) \Re(f(z))
= \Re\left(\left(\dpartial{}{x}+\dpartial{}{y}\right) f(z)\right)
= \Re(0) = 0.
\end{align*}

Let us explicitly define 
\begin{equation*}
P[d\mu](re^{i\theta}) = \dtwopii \int_{-\pi}^\pi P_r(\theta - t)\,d\mu(t).
\end{equation*}

Put \(m(t) = \mu(\{x: -\pi < x < t\})\).
We can find 4 increasing functions \(a_j(t)\) such that
\begin{equation*}
m(t) = (a_1(t) - a_2(t)) + i\left(a_3(t) - a_4(t)\right).
\end{equation*}
Applying Theorem~9.42 of \cite{RudinPMA85} separately to each \(a_i\)
and summing we can generalize that theorem so we can use \(m(t)\)
in place of \(a(t)\) in that theorem.
Thus

\iffalse
Let $f$ be a harmonic function.
Since its real and imaginary parts are real parts of holomorphic functions
(by Section~11.10), $f$ itself has partial derivatives of all orders.
By Theorem~9.41 \cite{RudinPMA85} we can change the order
of partial derivative axes. Hence
\begin{align*}
\Delta(f_x)
&= (f_x)_{xx} + (f_x)_{yy}
 = f_{xxx} + f_{xyy}
 = f_{xxx} + f_{yyx}
 = \left(f_{xx} + f_{yy}\right)_x
 = \left(\Delta(f)\right)_x
 = 0.
\end{align*}
Similarly we agve \(\Delta(f_y) = 0\).

Putting \(z = re^{i\theta} = x + iy\), we have:
\begin{equation*}
\renewcommand{\currentprefix}{ex11.4}
P(\theta -t) = \Re\left((e^{it}+z)/(e^{it}-z)\right)
\end{equation*}
Thus \(P(z) \in H(U)\).

Let \(e^{it} = x_t + iy_t\), then
\begin{align*}
 P_r(\theta -x)
&= \Re\left((x_t + iy_t + x+iy)/\left(x_t + iy_t -(x+iy)\right)\right) \\
&= \Re\left(((x_t+x) + i(y_t + y))/\left((x_t-x) + i(y_t - y)\right)\right) \\
&= \Re\left(\left((x_t+x) + i(y_t + y)\right)
           \cdot\left((x_t-x) - i(y_t - y)\right)
         / \left((x_t-x)^2 + (y_t - y)^2\right)\right) \\
&= \Re\frac{(x_t^2 - x^2) + (y_t^2 - y^2)
           + i\left((x_t-x)(y_t + y) + (x_t+x)(y_t - y)\right)}{
             (x_t-x)^2 + (y_t - y)^2} \\
&= (x_t^2 - x^2 + y_t^2 - y^2) / \left((x_t-x)^2 + (y_t - y)^2\right)
\end{align*}

Thus
\begin{align*}
\px P_r(\theta -x)
=& \px \left((x_t^2 - x^2) - (y_t^2 - y^2)\right)
   \bigm/ \left((x_t-x)^2 + (y_t - y)^2\right) \\
=&  \left(-2x \left((x_t-x)^2 + (y_t - y)^2\right)
    - \left((x_t^2 - x^2) - (y_t^2 - y^2)\right)(2x -2)\right)
    \\
 & \bigm/
     \left((x_t-x)^2 + (y_t - y)^2\right)^2 \\
=& - (4x + 2)\left((x_t-x)^2 + (y_t - y)^2\right)
     \bigm/ \left((x_t-x)^2 + (y_t - y)^2\right)^2 \\
\end{align*}
\fi

\iffalse
Looking at a kernel $k$, we freely use \(k(z) = k(r,\theta)\).
Asuming the given conditions, we want to show
\begin{equation} \locallabel{eq:need:lim}
\frac{\partial}{\partial x}\left( \int_{-\pi}^\pi k(r,\theta -t)\,d\mu(t)\right)
= 
\int_{-\pi}^\pi \frac{\partial}{\partial x}\left( k(r,\theta -t)\right)\,d\mu(t)
\end{equation}

It is sufficient to show that for each nonzero real sequence \(\{h_n\}_{n\in\N}\)
such that \mbox{\(\lim_{n\to\infty} h_n = 0\)}.
Using 
\begin{equation*}
z + h_n = r_n e^{i\theta_n}
\end{equation*}
we need to show
\begin{multline} \locallabel{need:limn}
\lim_{n\to\infty}\frac{1}{h_n}
 \left( \int_{-\pi}^\pi 
  \bigl(k(r,\theta -t)-k(r_n, \theta_n-t)\bigr)\,d\mu(t)\right)
 \\
 =
\int_{-\pi}^\pi 
  \left( 
    \lim_{n\to\infty}\frac{1}{h_n}
      \bigl(k(r,\theta -t)-k(r_n, \theta_n-t)\bigr)
  \right)
  \,d\mu(t)
\end{multline}

% Consider Theorem 9.42 in Rudin's PMA

% Consider cases:  z=0,   z\neq 0
Two cases. \\
\textbf{Case~1.} \(z\in\R\). \\
Then \(\theta = \theta_n = 0\). Now \localeqref{need:limn} becomes
\begin{equation} \locallabel{eq:need:lim}
\frac{\partial}{\partial x}\left( \int_{-\pi}^\pi k(r, -t)\,d\mu(t)\right)
= 
\int_{-\pi}^\pi \frac{\partial}{\partial x}\left( k(r, -t)\right)\,d\mu(t)
\end{equation}
\fi

\iffalse
\textbf{Case~1.} \(z=0\). \\
Then \(r=0\), \(\theta = \theta_n = 0\), \(h_n = r_n\)
and \eqref{eq:11.4:need:limn} becomes
\begin{equation*}
\lim_{n\to\infty}\frac{1}{h_n}
  \left( \int_{-\pi}^\pi k(0)-k(h_n, -t)\bigr)\,d\mu(t)\right)
 =
\int_{-\pi}^\pi 
  \left( \lim_{n\to\infty}\frac{1}{h_n} \bigl(k(0)-k(h_n, -t)\bigr)  
  \right)  \,d\mu(t)
\end{equation*}

The ``directonal'' derivative
\begin{equation*}
 \lim_{n\to\infty}\frac{1}{h_n} \bigl(k(0)-k(h_n, -t)\bigr)
 = e^{i\Arg(t)}\cdot k'(0).
\end{equation*}

\textbf{Case~2.} \(z\neq 0\).
\fi

\unfinished

%%%%%%%%%%%%%% 6
\begin{excopy}
Suppose \(f \in H(\Omega)\) and $f$ has no zero in \(\Omega\).
Prove that \(\log|f|\) is harmonic in \(\Omega\), by computing its
Laplacian. Is there an easier way?
\end{excopy}

\begin{align*}
\log(|f|)
 &= \log\left( \left(\left((f+\overline{f})/2\right)^2 
    + \left((\overline{f} - f)/2\right)^2\right)^{\half}\right)
 = \log\left( \left(\sqrt{2}/2\right)\left(f^2 
    + {\overline{f}}^2\right)^\half\right) \\
 &= \half \log\left(f^2 + {\overline{f}}^2\right) + \log(\sqrt{2}/2)
\end{align*}

\unfinished

%%%%%%%%%%%%%% 
\begin{excopy}
Suppose \(f \in H(U)\), where $U$ is the open unit disc, $f$ is one-to-one in
$U$, \(\Omega = f(U)\), and \(f(z) = \sum c_n z^n\).
Prove that the area of \(\Omega\) is 
\begin{equation*}
\pi \sum_{n=1}^\infty n |c_n|^2.
\end{equation*}

Hint: The Jacobian of $f$ is \(|f'|^2\).
\end{excopy}


%%%%%%%%%%%%%% 
\begin{excopy}
\ich{a} If \(f \in H(\Omega)\), \(f(z) \neq 0\) for \(z \in \Omega\),
and \(=\infty \alpha < \infty\), prove that
\begin{equation*}
\Delta(|f|^\alpha) = \alpha^2 |f|^{\alpha-2} |f'|^2,
\end{equation*}
by proving the formula
\begin{equation*}
\partial\overline{\partial}(\psi \circ (f\overline{f})) 
 = (\varphi \circ |f|^2)\cdot|f'|^2,
\end{equation*}
in which \(\psi\) twice differentiable on \((0, \infty)\) and
\begin{equation*}
\varphi(t) = t \psi''(t) + \psi'(t).
\end{equation*}

\ich{b}
Assume \(f \in H(\Omega)m\) and \(\Phi\) is a complex function with domain
\(f(\Omega)\), which has continuous
second-order partial derivatives. Prove that
\begin{equation*}
 \Delta[\Phi \circ f] = [(\Delta \Phi) \circ f] \cdot|f'|^2.
\end{equation*}
Show that this specializes to the result of \ich{a} 
if \(\Phi(w) = \Phi(|w|)\).
\end{excopy}


%%%%%%%%%%%%%% 8 
\begin{excopy}
Suppose \(\Omega\) is a region, \(f_n\in H(\Omega)\) for \(n=1,2,3,\ldots\),
\(u_n\) is the real part off \(f_n\), \(\{u_n\}\) converges 
uniformly on compact subsets of \(\Omega\), and \(\{f_n(z)\}\) converges for
 at least one \(z \in \Omega\). Prove that then \(\{f_n\}\)
converges uniformly on compact subsets of \(\Omega\).
\end{excopy}


%%%%%%%%%%%%%% 
\begin{excopy}
Suppose $u$ is a Lebesgue measurable function in a region \(\Omega\), and $u$ 
is locally in \(L^1\). This means that
the integral of \(|u|\) over any compact subset of \(\Omega\) is finite.
 Prove that $u$ is harmonic if it satisfies the
following form of the mean value property:
\begin{equation*}
u(a) = \frac{1}{\pi r^2} \iint\limits_{D(a;r)}  u(x, y)\,dx\,dy
\end{equation*}
whenever \(\overline{D}(a;r) \subset \Omega\).
\end{excopy}


%%%%%%%%%%%%%% 10
\begin{excopy}
Suppose \(I=[a,b]\) is an interval on the real axis, 
\(\varphi\) is a continuous function on $I$, and
\begin{equation*}
f(z) = \dtwopii \int_a^b \frac{\varphi{t}}{t - z}\,dt \qquad (z \notin I).
\end{equation*}
Show that
\begin{equation*}
\lim_{\epsilon\to 0}[f(x+i\epsilon) - f(x-i\epsilon)] \qquad (\epsilon > 0)
\end{equation*}
exists for every real $x$, and find it in terms of \(\varphi\).

How is the result affected if we assume merely that \(\varphi \in L^1\)?
What happens then at points $x$ at
which \(\varphi\) has right- and left-hand limits?
\end{excopy}


%%%%%%%%%%%%%% 
\begin{excopy}
Suppose that \(I=[a,b]\), \(\Omega\) is a region, \(I \subset \Omega\),
$f$ is continuous in \(\Omega\), and \(f \in H(\Omega - I)\). Prove that
actually \(f \in H(\Omega)\).

Replace $I$ by some other sets for which the same conclusion can be drawn.
\end{excopy}


%%%%%%%%%%%%%% 
\begin{excopy}
\index{Harnack} (Harnack‘s Inequalities) 
Suppose \(\Omega\) is a region, $K$ is a compact subset of \(\Omega\),
\(z_0 \in \Omega\), Prove that
there exist positive numbers \(\alpha\) and \(\beta\) 
(depending on \(z_0\), $K$, and \(\Omega\)) such that
\begin{equation*}
\alpha u(z_0) \leq u(z) \leq \beta u(z_0)
\end{equation*}
for every positive harmonic function $u$ in \(\Omega\) and for all \(z \in K\).

If \(\{u_n\}\) is a sequence of positive harmonic functions in \(\Omega\)
 and if \(u_n(z_0)\to 0\), describe the behavior
of \(\{u_n\}\) in the rest of \(\Omega\). Do the same if \(u_n(z_0)\to \infty\).
 Show that the assumed positivity of \(\{u_n\}\) is
essential for these results.
\end{excopy}


%%%%%%%%%%%%%% 13
\begin{excopy}
Suppose $u$ is a positive harmonic function in $U$ and \(u(0) = 1\).
How large can \(u(\half)\) be? How small?
Get the best possible bounds.
\end{excopy}


%%%%%%%%%%%%%% 
\begin{excopy}
For which pairs of lines \(L_1\), \(L_2\) do there exist real functions,
hamonic in the whole plane, that are
$0$ at all points of \(L_1 \cup L_2\) without vanishing identically?
\end{excopy}


%%%%%%%%%%%%%% 
\begin{excopy}
suppose $u$ is a positive harmonic function in $U$, 
and \(u(re^{i\theta}) \to 0\) as \(r\to 1\), for every \(e^{i\theta} \neq 1\).
Prove
that there is a constant $c$ such that
\begin{equation*}
u(re^{i\theta}) = cP_r(\theta).
\end{equation*}
\end{excopy}


%%%%%%%%%%%%%% 16 
\begin{excopy}
Here is an example of a harmonic function in $U$ which is not identically $0$ but all of whose radial
limits are $0$:
\begin{equation*}
u(z) = \Im\left[\left(\frac{1+2}{1-z}\right)^2\right].
\end{equation*}/
Prove that this $u$ is not the Poisson integral of any measure on $T$ 
and that it is not the difference of
two positive harmonic functions in $U$.
\end{excopy}


%%%%%%%%%%%%%% 
\begin{excopy}
Let \(\Phi\) be the set of all positive harmonic functions $u$ in $U$
 such that \(u(0) = 1\). Show that \(\Phi\) is 
a~convex set and find the extreme points of \(\Phi\). (A point $x$ in a convex
 set \(\Phi\) is called an extreme point of
\(\Phi\) if $x$ lies on no segment both of whose end points lie in \(\Phi\)
 and are different from $x$.) \emph{Hint}: If $C$ is the
convex set whose members are the positive Borel measures on $T$,
 of total variation $1$, show that the
extreme points of $C$ are precisely those \(\mu \in C\)
 whose supports consist of only one point of $T$.
\end{excopy}


%%%%%%%%%%%%%% 
\begin{excopy} 18
Let \(X^*\) be the dual space of the Banach space $X$. 
A sequence \(\{\Lambda_n\}\) in \(X^*\) is said to converge weakly
to \(\Lambda \in X^*\) if \(\Lambda_n x \to \Lambda x\) as \(n \to \infty\),
 for every \(x \in X\). Note that \(\Lambda_n \to \Lambda\) weakly whenever
\(\Lambda_n \to \Lambda\) in the
norm of \(X^*\). (See Exercise~8, Chap.~5.) The converse need not be true.
 For example, the functionals
\(f\to \hat{f}(n)\) on \(L^2(T)\) tend to $0$ weakly (by the Bessel inequality),
 but each of these functionals has norm $1$.
Prove that \(\{\| \Lambda_n\|\}\) must be bounded if \(\{\Lambda_n\}\)
 converges weakly.
\end{excopy}


%%%%%%%%%%%%%% 19
\begin{excopy}
\ich{a} Show that \(\delta P_r(\delta) > 1\) if \(\delta = 1 - r\).\\
\ich{b} If \(\mu \geq 0\), \(u = P[d\mu]\), and \(I_\delta \subset T\)
is the are with center $1$ and length \(2\delta\), show that
\begin{equation*}
\mu(I_\delta) \leq \delta\mu(1 - \delta)
\end{equation*}
and that therefore
\begin{equation*}
(M\mu)(1) \leq \pi(M_{\textnormal{rad}}\,u)(1).
\end{equation*}
\ich{c} If, furthermore, \(\mu \perp m\), show that
\begin{equation*}
u(re^{i\theta}) \to \infty \qquad \aded\,[\mu].
\end{equation*}
\emph{Hint}: Use Theorem~7.15.
\end{excopy}


%%%%%%%%%%%%%% 20
\begin{excopy}
Suppose \(E \subset T\), \(m(E) = 0\).
 Prove that there is an \(f \in H^\infty\), with \(f(0) = 1\), that has
\begin{equation*}
\lim_{r\to 1} f(re^{i\theta}) = 0
\end{equation*}
at every \(e^{i\theta} \in E\).

\emph{Suggestion}: Find a lower semicontinuous 
 \(psi \in L^1(T)\), \(\psi > 0\), \(\psi = +\infty\) at every point of $E$.
 There
is a holomorphic $g$ whose real part is \(P[\psi]\). Let \(f= 1/g\).
\end{excopy}


%%%%%%%%%%%%%% 21
\begin{excopy}
Define \(f \in H(U)\) and \(g \in H(U)\) by 
\(f(z) = \exp \{(1 + z)/(1 - z)\}\), \(g(z) = (1 - z) \exp \{-f(z)\}\). Prove
that
\begin{equation*}
g^*(e^{i\theta}) = \lim_{r\to 1} g(re^{i\theta})
\end{equation*}
exists at every \(e^{i\theta} \in T\), that \(g^* \in C(T)\),
 but that $g$ is not in \(H^\infty\).

\emph{Suggestion}: Fix $s$, put
\begin{equation*}
 z_t = \frac{t + is - 1}{t + is + 1} \qquad (0 < t < \infty).
\end{equation*}
For certain values of $s$, \(|g(z_t)| \to \infty\) as \(t \to \infty\).

\end{excopy}


%%%%%%%%%%%%%% 22
\begin{excopy}
Suppose $u$ is harmonic in $U$, and 
\(\{u_r: 0 \leq r < 1\}\) is a uniformly integrable subset of 
\(L^1(T)\). (See
Exercise~10, Chap.~6.) Modify the proof of Theorem~11.30 to show that
\(u = P[f]\) for some \(f \in L^1(T)\).

\end{excopy}


%%%%%%%%%%%%%% 23
\begin{excopy}
Put \(\theta_n = 2^{-n}\) and define
\begin{equation*}
u(z) = \sum_{n=1}^\infty n^{-2}\{P(z,e^{i\theta_n}) - P(z,e^{-i\theta_n})\},
\end{equation*}
for \(z \in U\). Show that $u$ is the Poisson integral of a measure on $T$,
 that \(u(x) = 0\) if \(-1 < x < 1\), but
that
\begin{equation*}
u(1 — \epsilon + i\epsilon)
\end{equation*}
is unbounded, as \(\epsilon\) decreases to $0$. 
(Thus $u$ has a radial limit, but no nontangential limit, at $1$.)

\emph{Hint}: lf \(\epsilon = \sin \theta\) is small and 
\(z = 1 — \epsilon + i\epsilon\), then
\begin{equation*}
 P(z,e^{i\theta}) - P(z,e^{-i\theta}) > 1/\epsilon.
\end{equation*}
\end{excopy}


%%%%%%%%%%%%%% 24
\begin{excopy}
Let \(D_n(t)\) be the 
\index{Dirichlet} Dirichlet kernel, as in Sec.~5.11, define the 
\index{Fejer@Fej\'er} Fej\'er 
kernel by
\begin{equation*}
K_N = \frac{1}{N+1} (D_0 + D_1 + \cdots + D_n),
\end{equation*}
put \(L_N(t) = \min(N, \pi^2/Nt^2)\). Prove that
\begin{equation*}
K_{N_1}(t) = \frac{1}{N} \cdot \frac{1 - \cos Nt}{1 - \cos t} \leq L_N(t)
\end{equation*}
and that \(\int_R L_N\,d\sigma \leq 2\).

Use this to prove that the arithmetic means
\begin{equation*}
\sigma_N = \frac{S_0 + S_1 + \cdots + S_N}{N + 1}
\end{equation*}
of the partial sums \(s_n\) of the Fourier series of a functionf
\(f \in L^1(T)\) converge to \(f(e^{i\theta})\) at every Lebesgue
point of $f$ (Show that \(\sup |\sigma_N|\) is dominated by \(Mf\),
 then proceed as in the proof of Theorem~11.23.)
\end{excopy}

%%%%%%%%%%%%%% 25
\begin{excopy}
If \(1 \leq p \leq \infty\) and \(f \in L^1(\R^1)\), prove that
 \((f * h_\lambda)(x)\) is a harmonic function of \(x + i/\lambda\) in the upper
half plane. 
(\(h_\lambda\) is defined in Sec.~9.7; 
it is the Poisson kernel for the half plane.)
\end{excopy}

%%%%%%%%%%%%%%%%%
\end{enumerate}



\fi



\iffalse
\setcounter{chapter}{10}
%%%%%%%%%%%%%%%%%%%%%%%%%%%%%%%%%%%%%%%%%%%%%%%%%%%%%%%%%%%%%%%%%%%%%%%%
%%%%%%%%%%%%%%%%%%%%%%%%%%%%%%%%%%%%%%%%%%%%%%%%%%%%%%%%%%%%%%%%%%%%%%%%
%%%%%%%%%%%%%%%%%%%%%%%%%%%%%%%%%%%%%%%%%%%%%%%%%%%%%%%%%%%%%%%%%%%%%%%%
%chapter 11
\chapterTypeout{Harmonic Functions}

%%%%%%%%%%%%%%%%%%%%%%%%%%%%%%%%%%%%%%%%%%%%%%%%%%%%%%%%%%%%%%%%%%%%%%%%
%%%%%%%%%%%%%%%%%%%%%%%%%%%%%%%%%%%%%%%%%%%%%%%%%%%%%%%%%%%%%%%%%%%%%%%%
\section{Notes}

%%%%%%%%%%%%%%%%%%%%%%%%%%%%%%%%%%%%%%%%%%%%%%%%%%%%%%%%%%%%%%%%%%%%%%%%
\index{Laplacian}
\subsection{The Laplacian}

\newcommand*{\partialby}[1]{\frac{\partial}{\partial #1}}
\newcommand*{\fracpart}[2]{\frac{\partial #1}{\partial #2}}
\newcommand*{\dpartial}[2]{\frac{\partial^2 #1}{\partial #2^2}}
\newcommand*{\px}{\partialby x}
\newcommand*{\py}{\partialby y}

Assuming \(f = u + iv\) and \(f_{xy} = f_{yx}\) we have
\begin{align*}
4\partial\tilde{\partial}f 
 &= 4\partial\left(\half\left(\px + i\py\right)(u + iv)\right)
  = 2\partial\bigl(u_x + iv_x + i(u_y + iv_y)\bigr) \\
 &= \left(\px - i\py\right)\bigl(u_x - v_y + i(v_x + u_y)\bigr) \\
 &= u_{xx} - v_{yx}  + i(v_{xx} + u_{yx}) 
    -i\bigl(u_{xy} - v_{yy} + i(v_{xy} + u_{yy})\bigr) \\
 &= (u + iv)_{xx} + (u + iv)_{yy} = \Delta f.
\end{align*}

\subsubsection{Polar Coordinates}

Using
\begin{equation*}
x = r\cos\theta \qquad
y = r\sin\theta \qquad
r = \sqrt{x^2+y^2} \qquad
\tan \theta = y/x
\end{equation*}
Asssume \(u(x,y)\) is sufficiently differentiable with continuous
partial derivatives.

Using the chain rule
\begin{equation*}
\fracpart{u}{x} 
  = \cos \theta \fracpart{u}{r} 
    - \frac{\sin \theta}{r}\fracpart{u}{\theta} 
  \qquad
\fracpart{u}{y} 
  = \sin \theta \fracpart{u}{r} 
    + \frac{\cos \theta}{r}\fracpart{u}{\theta}
\end{equation*}

Continuing
\begin{align*}
\dpartial{u}{x}
 &= \cos^2\theta \dpartial{u}{r}
    - \frac{2\sin\theta \cos\theta}{r} 
      \frac{\partial^2 u}{\partial r\,\partial \theta}
    + \frac{\sin^2 \theta}{r^2} \dpartial{u}{\theta}
    + \frac{\sin^2 \theta}{r} \fracpart{u}{r}
    + \frac{2\sin\theta \cos\theta}{r^2}\fracpart{u}{\theta} \\
\dpartial{u}{y}
 &= \sin^2\theta \dpartial{u}{r}
    + \frac{2\sin\theta \cos\theta}{r} 
      \frac{\partial^2 u}{\partial r\,\partial \theta}
    + \frac{\cos^2 \theta}{r^2} \dpartial{u}{\theta}
    + \frac{\cos^2 \theta}{r} \fracpart{u}{r}
    - \frac{2\sin\theta \cos\theta}{r^2}\fracpart{u}{\theta}
\end{align*}
Adding the above gives
\begin{equation*}
\Delta u 
= \dpartial{u}{x} + \dpartial{u}{y}
 = \dpartial{u}{r} + \frac{1}{r}\fracpart{u}{r} 
    + \frac{1}{r^2}\dpartial{u}{\theta}
 = \frac{1}{r}\partialby{r}\left(r\fracpart{u}{r}\right)
    + \frac{1}{r^2}\dpartial{u}{\theta}
\end{equation*}


%%%%%%%%%%%%%%%%%%%%%%%%%%%%%%%%%%%%%%%%%%%%%%%%%%%%%%%%%%%%%%%%%%%%%%%%
\subsection{Proof of Theorem~11.9}

The proof of Theorem~11.9 refers to Theorem~10.7
for showing that $f$ is holomorphic. But instead, it should 
refer to Exercise~16 of Chapter~10 (\ref{ex:10.16}).

%%%%%%%%%%%%%%%%%%%%%%%%%%%%%%%%%%%%%%%%%%%%%%%%%%%%%%%%%%%%%%%%%%%%%%%%
\subsection{Harnack's Theorem}

\index{Harnack}
Let's look a the proof of Harnack's Theorem~11.11.
The first double inequality has a minor error.
The middle expression is missing square sign in the denominator.
It should be
\begin{equation*}
\frac{R^2 - r^2}{R^2 - 2rR\cos(\theta - t) + r^{\mathbf{2}}}
\end{equation*}

% The proof of Harnack's Theorem~11.11, specifically 
The second
double-inequality makes use of \textbf{11.10}(1).

\iffalse
of the following observation.

Say $u$ is a real harmonic function, so \(u(z) = \Re(f(z)\) for some
\(f\in H(\Omega)\). 
Let \(\gamma(t) = a + Re^{it}\)
and \(\Gamma = \{\gamma(t): -\pi \leq t < \pi\}\). Now
\(\gamma`(t) = Rie^{it}\) and
\begin{align*}
u(a) 
 &= \Re(f(a))
 = 
   \Re\left(
     \dtwopii
     \int_{\Gamma} \frac{f(a+z)}{(a+z)-a}\,dz
   \right) 
 \\
 &= \frac{1}{2\pi} 
   \Re\left(
    \frac{1}{i}
     \int_{-\pi}^\pi \frac{f(a+\gamma(t))}{Re^{it}}\cdot Rie^{it}\,dt
   \right).
 \\
 &= \frac{1}{2\pi} 
   \Re\left(
     \int_{-\pi}^\pi f(a+\gamma(t))\,dt
   \right).
\end{align*}
\fi


%%%%%%%%%%%%%%%%%%%%%%%%%%%%%%%%%%%%%%%%%%%%%%%%%%%%%%%%%%%%%%%%%%%%%%%%
%%%%%%%%%%%%%%%%%%%%%%%%%%%%%%%%%%%%%%%%%%%%%%%%%%%%%%%%%%%%%%%%%%%%%%%%
\section{The Exercises} % pages 249-252

%%%%%%%%%%%%%%%%%
\begin{enumerate}
%%%%%%%%%%%%%%%%%

%%%%%%%%%%%%%% 
\begin{excopy}
Suppose $u$ and $v$ are real harmonic functions in a plane regular \(\Omega\).
Under what conditions is \(uv\) harmonic?
(Note that the answer depends strongly on the fact that the question
is one about \emph{real} functions.)
Show that \(u^2\) cannot be harmonic in \(\Omega\), unless $u$ is constant.
For which \(f \in H(\Omega)\) is \(|f|^2\) harmonic?
\end{excopy}

Compute
\begin{equation*}
(uv)_{xx} 
= \left((uv)_x\right)_x
= \left(u_xv + uv_x\right)_x
= u_{xx}v + 2u_xv_x + uv_{xx}
\end{equation*}

Hence
\begin{align*}
\Delta(uv) 
 &= u_{xx}v + 2u_xv_x + uv_{xx} + u_{yy}v + 2u_yv_y + uv_{yy} \\
 &= \Delta(u)v + u\Delta(v) + 2(u_xv_x + u_yv_y).
\end{align*}

So \(uv\) is harmonic when \(u_xv_x + u_yv_y = 0\) in \(\Omega\).
Clearly \(\Delta(u^2) = 2(u_x^2 + u_y^2) \geq 0\)
and equality holds iff \(u_x = u_y = 0\).

%%%%%%%%%%%%%% 
\begin{excopy}
Suppose $f$ a complex function in a region \(\Omega\) and both
$f$ and \(f^2\) are hannonic in \(\Omega\). Prove that
either $f$ or \(\overline{f}\) are holnmorphic in \(\Omega\).
\end{excopy}

By the previous exercise, \((f_x)^2 + (f_y)^2 = 0\).
Regardless whether \(f_x = f_y = 0\) holds or not, we have
\begin{equation*}
f_x = \pm i\,f_y.
\end{equation*}
By continuity, only one variant of \(\pm\) holds for all \(\Omega\).
Put \(f = u + iv\) with real valued functions $u$ and $v$.
Now
\begin{equation*}
u_x + iv_x = -v_y + iu_y
\qquad\textnormal{or}\qquad
u_x + iv_x = v_y - iu_y.
\end{equation*}
Hence the Cauchy-Riemann equation holds for \(\overline{f}\) or $f$.

%%%%%%%%%%%%%% 3
\begin{excopy}
If $u$ is a harmonic function in a region \(\Omega\),
what can you say about the set of points at which the
gradient of $u$ is $0$? (Thus is the set which \(u_x = u_y = 0\).)
\end{excopy}

% From Section~1.11, every harmonic function has continuous partial derivative
% of all orders.
We will show that
the vanishing set $K$ is either the whole region where $u$ is constant
in each connected component of \(\Omega\),
or it consists of just isolated points.

By Section~11.10, every real harmonic function is locally the real part
of holomorphic function.

Assume $u$ is harmnonic in a connected region
(\wlogy, it is \(\Omega\) and the set
\begin{equation*}
 K = \{z\in\Omega: u_x(z) = u_y(z) = 0\}
\end{equation*}
has accumulation point \(z_0 \in \Omega\).
Note that both \(\Re(u)\) and \(\Im(u)\) are harmnonic.
By the previous, remark there exist a neighborhood \(V\subset\Omega\)
and holomorphic functions $f$ and $G$ defined on $V\ni z_0$ such that
\begin{equation*}
 u(z) = \Re(f(z)) + i\Re(g(z)) \qquad (z\in V).
\end{equation*}
Now for \(z\in V\)
\begin{align*}
u_x(z) &= \Re(f_x(z)) + i\Re(g_x(z)) = (\Re(f))_x(z) + i(\Re(g))_x(z) \\
u_y(z) &= \Re(f_y(z)) + i\Re(g_y(z)) = (\Re(f))_y(z) + i(\Re(g))_y(z)
\end{align*}
and Cauchy-Riemann equations show that 
\begin{alignat*}{2}
\Re(u)_x &= \Re(f)_x = \Im(f)_y &\qquad \Im(u)_x &= \Re(g)_x = \Im(g)_y \\
\Re(u)_y &= \Re(f)_y = -\Im(f)_x &\qquad \Im(u)_y &= \Re(g)_y = -\Im(g)_x 
\end{alignat*}
For \(z\in K\), the above functions vanish, hence
\begin{align*}
f'(z) = g'(z) = 0 \qquad (z\in K).
\end{align*}
By Theorem~10.18 \(f'(z)=g'(z) = 0\) for all \(z\in V\)
and consequently for all \(z\in \Omega\).
Hence $f$ and $g$ are constant functions and so is $u$.

%%%%%%%%%%%%%% 4
\begin{excopy}
Prove that every partial derivative of every harmonic function is harmonic.

Verify, by direct computation, that \(P_r(\theta - t)\) is, for each fixed $t$,
a harmonic function of \(re^{i\theta}\).
Deduce (without referring to holomorphic functions) that the Poisson integral
\(P[d\mu]\) of every finite
Borel measure \(\mu\) on $T$ is harmonic in $U$, by showing that every partial
derivative of \(P[d\mu]\) is equal to
the integral of the corresponding partial derivative of the kernel.
\end{excopy}

See also \cite{Lang199304} Chapter~\textsf{VIII} Section~\S3 Example~3.

Let
\begin{equation*}
f(z) = \frac{e^{it} + z}{e^{it} - z} = \frac{u + x+iy}{u - x - iy}.
\end{equation*}

First order differentiation:
\begin{align*}
\partialby{x}f(z) 
&= \partialby{x} \frac{u + x+iy}{u - x - iy}
 = \frac{(u - x - iy) + (u + x+iy)}{(u-z)^2}
 = \frac{2u}{(u-z)^2}
 \\
\partialby{y}f(z) 
&= \partialby{y} \frac{u + x+iy}{u - x - iy}
 = \frac{i(u - x - iy) + i(u + x+iy)}{(u-z)^2}
 = \frac{2iu}{(u-z)^2}
\end{align*}

Second order differentiation:
\begin{align*}
\dpartial{}{x}f(z) 
 &= \partialby{x}\frac{2u}{(u-x-iy)^2}
  = \frac{-2u(2x+2iy-2u)}{(u-z)^2}
  = 4u\frac{(-x-iy+u)}{(u-z)^2}
 \\
\dpartial{}{y}f(z) 
 &= \partialby{y}\frac{2iu}{(u-x-iy)^2}
  = \frac{-2iu(2ix-2y-2iu)}{(u-z)^2}
  = 4u\frac{(x+iy-u)}{(u-z)^2}
\end{align*}

By summing the above, and noting that the partial differentiation
commutes with the \(\Re\) operator,
we get 
\begin{align*}
\Delta P_r(\theta - t)
= \left(\dpartial{}{x}+\dpartial{}{y}\right) \Re(f(z))
= \Re\left(\left(\dpartial{}{x}+\dpartial{}{y}\right) f(z)\right)
= \Re(0) = 0.
\end{align*}

Let us explicitly define 
\begin{equation*}
P[d\mu](re^{i\theta}) = \dtwopii \int_{-\pi}^\pi P_r(\theta - t)\,d\mu(t).
\end{equation*}

Put \(m(t) = \mu(\{x: -\pi < x < t\})\).
We can find 4 increasing functions \(a_j(t)\) such that
\begin{equation*}
m(t) = (a_1(t) - a_2(t)) + i\left(a_3(t) - a_4(t)\right).
\end{equation*}
Applying Theorem~9.42 of \cite{RudinPMA85} separately to each \(a_i\)
and summing we can generalize that theorem so we can use \(m(t)\)
in place of \(a(t)\) in that theorem.
Thus

\iffalse
Let $f$ be a harmonic function.
Since its real and imaginary parts are real parts of holomorphic functions
(by Section~11.10), $f$ itself has partial derivatives of all orders.
By Theorem~9.41 \cite{RudinPMA85} we can change the order
of partial derivative axes. Hence
\begin{align*}
\Delta(f_x)
&= (f_x)_{xx} + (f_x)_{yy}
 = f_{xxx} + f_{xyy}
 = f_{xxx} + f_{yyx}
 = \left(f_{xx} + f_{yy}\right)_x
 = \left(\Delta(f)\right)_x
 = 0.
\end{align*}
Similarly we agve \(\Delta(f_y) = 0\).

Putting \(z = re^{i\theta} = x + iy\), we have:
\begin{equation*}
\renewcommand{\currentprefix}{ex11.4}
P(\theta -t) = \Re\left((e^{it}+z)/(e^{it}-z)\right)
\end{equation*}
Thus \(P(z) \in H(U)\).

Let \(e^{it} = x_t + iy_t\), then
\begin{align*}
 P_r(\theta -x)
&= \Re\left((x_t + iy_t + x+iy)/\left(x_t + iy_t -(x+iy)\right)\right) \\
&= \Re\left(((x_t+x) + i(y_t + y))/\left((x_t-x) + i(y_t - y)\right)\right) \\
&= \Re\left(\left((x_t+x) + i(y_t + y)\right)
           \cdot\left((x_t-x) - i(y_t - y)\right)
         / \left((x_t-x)^2 + (y_t - y)^2\right)\right) \\
&= \Re\frac{(x_t^2 - x^2) + (y_t^2 - y^2)
           + i\left((x_t-x)(y_t + y) + (x_t+x)(y_t - y)\right)}{
             (x_t-x)^2 + (y_t - y)^2} \\
&= (x_t^2 - x^2 + y_t^2 - y^2) / \left((x_t-x)^2 + (y_t - y)^2\right)
\end{align*}

Thus
\begin{align*}
\px P_r(\theta -x)
=& \px \left((x_t^2 - x^2) - (y_t^2 - y^2)\right)
   \bigm/ \left((x_t-x)^2 + (y_t - y)^2\right) \\
=&  \left(-2x \left((x_t-x)^2 + (y_t - y)^2\right)
    - \left((x_t^2 - x^2) - (y_t^2 - y^2)\right)(2x -2)\right)
    \\
 & \bigm/
     \left((x_t-x)^2 + (y_t - y)^2\right)^2 \\
=& - (4x + 2)\left((x_t-x)^2 + (y_t - y)^2\right)
     \bigm/ \left((x_t-x)^2 + (y_t - y)^2\right)^2 \\
\end{align*}
\fi

\iffalse
Looking at a kernel $k$, we freely use \(k(z) = k(r,\theta)\).
Asuming the given conditions, we want to show
\begin{equation} \locallabel{eq:need:lim}
\frac{\partial}{\partial x}\left( \int_{-\pi}^\pi k(r,\theta -t)\,d\mu(t)\right)
= 
\int_{-\pi}^\pi \frac{\partial}{\partial x}\left( k(r,\theta -t)\right)\,d\mu(t)
\end{equation}

It is sufficient to show that for each nonzero real sequence \(\{h_n\}_{n\in\N}\)
such that \mbox{\(\lim_{n\to\infty} h_n = 0\)}.
Using 
\begin{equation*}
z + h_n = r_n e^{i\theta_n}
\end{equation*}
we need to show
\begin{multline} \locallabel{need:limn}
\lim_{n\to\infty}\frac{1}{h_n}
 \left( \int_{-\pi}^\pi 
  \bigl(k(r,\theta -t)-k(r_n, \theta_n-t)\bigr)\,d\mu(t)\right)
 \\
 =
\int_{-\pi}^\pi 
  \left( 
    \lim_{n\to\infty}\frac{1}{h_n}
      \bigl(k(r,\theta -t)-k(r_n, \theta_n-t)\bigr)
  \right)
  \,d\mu(t)
\end{multline}

% Consider Theorem 9.42 in Rudin's PMA

% Consider cases:  z=0,   z\neq 0
Two cases. \\
\textbf{Case~1.} \(z\in\R\). \\
Then \(\theta = \theta_n = 0\). Now \localeqref{need:limn} becomes
\begin{equation} \locallabel{eq:need:lim}
\frac{\partial}{\partial x}\left( \int_{-\pi}^\pi k(r, -t)\,d\mu(t)\right)
= 
\int_{-\pi}^\pi \frac{\partial}{\partial x}\left( k(r, -t)\right)\,d\mu(t)
\end{equation}
\fi

\iffalse
\textbf{Case~1.} \(z=0\). \\
Then \(r=0\), \(\theta = \theta_n = 0\), \(h_n = r_n\)
and \eqref{eq:11.4:need:limn} becomes
\begin{equation*}
\lim_{n\to\infty}\frac{1}{h_n}
  \left( \int_{-\pi}^\pi k(0)-k(h_n, -t)\bigr)\,d\mu(t)\right)
 =
\int_{-\pi}^\pi 
  \left( \lim_{n\to\infty}\frac{1}{h_n} \bigl(k(0)-k(h_n, -t)\bigr)  
  \right)  \,d\mu(t)
\end{equation*}

The ``directonal'' derivative
\begin{equation*}
 \lim_{n\to\infty}\frac{1}{h_n} \bigl(k(0)-k(h_n, -t)\bigr)
 = e^{i\Arg(t)}\cdot k'(0).
\end{equation*}

\textbf{Case~2.} \(z\neq 0\).
\fi

\unfinished

%%%%%%%%%%%%%% 6
\begin{excopy}
Suppose \(f \in H(\Omega)\) and $f$ has no zero in \(\Omega\).
Prove that \(\log|f|\) is harmonic in \(\Omega\), by computing its
Laplacian. Is there an easier way?
\end{excopy}

\begin{align*}
\log(|f|)
 &= \log\left( \left(\left((f+\overline{f})/2\right)^2 
    + \left((\overline{f} - f)/2\right)^2\right)^{\half}\right)
 = \log\left( \left(\sqrt{2}/2\right)\left(f^2 
    + {\overline{f}}^2\right)^\half\right) \\
 &= \half \log\left(f^2 + {\overline{f}}^2\right) + \log(\sqrt{2}/2)
\end{align*}

\unfinished

%%%%%%%%%%%%%% 
\begin{excopy}
Suppose \(f \in H(U)\), where $U$ is the open unit disc, $f$ is one-to-one in
$U$, \(\Omega = f(U)\), and \(f(z) = \sum c_n z^n\).
Prove that the area of \(\Omega\) is 
\begin{equation*}
\pi \sum_{n=1}^\infty n |c_n|^2.
\end{equation*}

Hint: The Jacobian of $f$ is \(|f'|^2\).
\end{excopy}


%%%%%%%%%%%%%% 
\begin{excopy}
\ich{a} If \(f \in H(\Omega)\), \(f(z) \neq 0\) for \(z \in \Omega\),
and \(=\infty \alpha < \infty\), prove that
\begin{equation*}
\Delta(|f|^\alpha) = \alpha^2 |f|^{\alpha-2} |f'|^2,
\end{equation*}
by proving the formula
\begin{equation*}
\partial\overline{\partial}(\psi \circ (f\overline{f})) 
 = (\varphi \circ |f|^2)\cdot|f'|^2,
\end{equation*}
in which \(\psi\) twice differentiable on \((0, \infty)\) and
\begin{equation*}
\varphi(t) = t \psi''(t) + \psi'(t).
\end{equation*}

\ich{b}
Assume \(f \in H(\Omega)m\) and \(\Phi\) is a complex function with domain
\(f(\Omega)\), which has continuous
second-order partial derivatives. Prove that
\begin{equation*}
 \Delta[\Phi \circ f] = [(\Delta \Phi) \circ f] \cdot|f'|^2.
\end{equation*}
Show that this specializes to the result of \ich{a} 
if \(\Phi(w) = \Phi(|w|)\).
\end{excopy}


%%%%%%%%%%%%%% 8 
\begin{excopy}
Suppose \(\Omega\) is a region, \(f_n\in H(\Omega)\) for \(n=1,2,3,\ldots\),
\(u_n\) is the real part off \(f_n\), \(\{u_n\}\) converges 
uniformly on compact subsets of \(\Omega\), and \(\{f_n(z)\}\) converges for
 at least one \(z \in \Omega\). Prove that then \(\{f_n\}\)
converges uniformly on compact subsets of \(\Omega\).
\end{excopy}


%%%%%%%%%%%%%% 
\begin{excopy}
Suppose $u$ is a Lebesgue measurable function in a region \(\Omega\), and $u$ 
is locally in \(L^1\). This means that
the integral of \(|u|\) over any compact subset of \(\Omega\) is finite.
 Prove that $u$ is harmonic if it satisfies the
following form of the mean value property:
\begin{equation*}
u(a) = \frac{1}{\pi r^2} \iint\limits_{D(a;r)}  u(x, y)\,dx\,dy
\end{equation*}
whenever \(\overline{D}(a;r) \subset \Omega\).
\end{excopy}


%%%%%%%%%%%%%% 10
\begin{excopy}
Suppose \(I=[a,b]\) is an interval on the real axis, 
\(\varphi\) is a continuous function on $I$, and
\begin{equation*}
f(z) = \dtwopii \int_a^b \frac{\varphi{t}}{t - z}\,dt \qquad (z \notin I).
\end{equation*}
Show that
\begin{equation*}
\lim_{\epsilon\to 0}[f(x+i\epsilon) - f(x-i\epsilon)] \qquad (\epsilon > 0)
\end{equation*}
exists for every real $x$, and find it in terms of \(\varphi\).

How is the result affected if we assume merely that \(\varphi \in L^1\)?
What happens then at points $x$ at
which \(\varphi\) has right- and left-hand limits?
\end{excopy}


%%%%%%%%%%%%%% 
\begin{excopy}
Suppose that \(I=[a,b]\), \(\Omega\) is a region, \(I \subset \Omega\),
$f$ is continuous in \(\Omega\), and \(f \in H(\Omega - I)\). Prove that
actually \(f \in H(\Omega)\).

Replace $I$ by some other sets for which the same conclusion can be drawn.
\end{excopy}


%%%%%%%%%%%%%% 
\begin{excopy}
\index{Harnack} (Harnack‘s Inequalities) 
Suppose \(\Omega\) is a region, $K$ is a compact subset of \(\Omega\),
\(z_0 \in \Omega\), Prove that
there exist positive numbers \(\alpha\) and \(\beta\) 
(depending on \(z_0\), $K$, and \(\Omega\)) such that
\begin{equation*}
\alpha u(z_0) \leq u(z) \leq \beta u(z_0)
\end{equation*}
for every positive harmonic function $u$ in \(\Omega\) and for all \(z \in K\).

If \(\{u_n\}\) is a sequence of positive harmonic functions in \(\Omega\)
 and if \(u_n(z_0)\to 0\), describe the behavior
of \(\{u_n\}\) in the rest of \(\Omega\). Do the same if \(u_n(z_0)\to \infty\).
 Show that the assumed positivity of \(\{u_n\}\) is
essential for these results.
\end{excopy}


%%%%%%%%%%%%%% 13
\begin{excopy}
Suppose $u$ is a positive harmonic function in $U$ and \(u(0) = 1\).
How large can \(u(\half)\) be? How small?
Get the best possible bounds.
\end{excopy}


%%%%%%%%%%%%%% 
\begin{excopy}
For which pairs of lines \(L_1\), \(L_2\) do there exist real functions,
hamonic in the whole plane, that are
$0$ at all points of \(L_1 \cup L_2\) without vanishing identically?
\end{excopy}


%%%%%%%%%%%%%% 
\begin{excopy}
suppose $u$ is a positive harmonic function in $U$, 
and \(u(re^{i\theta}) \to 0\) as \(r\to 1\), for every \(e^{i\theta} \neq 1\).
Prove
that there is a constant $c$ such that
\begin{equation*}
u(re^{i\theta}) = cP_r(\theta).
\end{equation*}
\end{excopy}


%%%%%%%%%%%%%% 16 
\begin{excopy}
Here is an example of a harmonic function in $U$ which is not identically $0$ but all of whose radial
limits are $0$:
\begin{equation*}
u(z) = \Im\left[\left(\frac{1+2}{1-z}\right)^2\right].
\end{equation*}/
Prove that this $u$ is not the Poisson integral of any measure on $T$ 
and that it is not the difference of
two positive harmonic functions in $U$.
\end{excopy}


%%%%%%%%%%%%%% 
\begin{excopy}
Let \(\Phi\) be the set of all positive harmonic functions $u$ in $U$
 such that \(u(0) = 1\). Show that \(\Phi\) is 
a~convex set and find the extreme points of \(\Phi\). (A point $x$ in a convex
 set \(\Phi\) is called an extreme point of
\(\Phi\) if $x$ lies on no segment both of whose end points lie in \(\Phi\)
 and are different from $x$.) \emph{Hint}: If $C$ is the
convex set whose members are the positive Borel measures on $T$,
 of total variation $1$, show that the
extreme points of $C$ are precisely those \(\mu \in C\)
 whose supports consist of only one point of $T$.
\end{excopy}


%%%%%%%%%%%%%% 
\begin{excopy} 18
Let \(X^*\) be the dual space of the Banach space $X$. 
A sequence \(\{\Lambda_n\}\) in \(X^*\) is said to converge weakly
to \(\Lambda \in X^*\) if \(\Lambda_n x \to \Lambda x\) as \(n \to \infty\),
 for every \(x \in X\). Note that \(\Lambda_n \to \Lambda\) weakly whenever
\(\Lambda_n \to \Lambda\) in the
norm of \(X^*\). (See Exercise~8, Chap.~5.) The converse need not be true.
 For example, the functionals
\(f\to \hat{f}(n)\) on \(L^2(T)\) tend to $0$ weakly (by the Bessel inequality),
 but each of these functionals has norm $1$.
Prove that \(\{\| \Lambda_n\|\}\) must be bounded if \(\{\Lambda_n\}\)
 converges weakly.
\end{excopy}


%%%%%%%%%%%%%% 19
\begin{excopy}
\ich{a} Show that \(\delta P_r(\delta) > 1\) if \(\delta = 1 - r\).\\
\ich{b} If \(\mu \geq 0\), \(u = P[d\mu]\), and \(I_\delta \subset T\)
is the are with center $1$ and length \(2\delta\), show that
\begin{equation*}
\mu(I_\delta) \leq \delta\mu(1 - \delta)
\end{equation*}
and that therefore
\begin{equation*}
(M\mu)(1) \leq \pi(M_{\textnormal{rad}}\,u)(1).
\end{equation*}
\ich{c} If, furthermore, \(\mu \perp m\), show that
\begin{equation*}
u(re^{i\theta}) \to \infty \qquad \aded\,[\mu].
\end{equation*}
\emph{Hint}: Use Theorem~7.15.
\end{excopy}


%%%%%%%%%%%%%% 20
\begin{excopy}
Suppose \(E \subset T\), \(m(E) = 0\).
 Prove that there is an \(f \in H^\infty\), with \(f(0) = 1\), that has
\begin{equation*}
\lim_{r\to 1} f(re^{i\theta}) = 0
\end{equation*}
at every \(e^{i\theta} \in E\).

\emph{Suggestion}: Find a lower semicontinuous 
 \(psi \in L^1(T)\), \(\psi > 0\), \(\psi = +\infty\) at every point of $E$.
 There
is a holomorphic $g$ whose real part is \(P[\psi]\). Let \(f= 1/g\).
\end{excopy}


%%%%%%%%%%%%%% 21
\begin{excopy}
Define \(f \in H(U)\) and \(g \in H(U)\) by 
\(f(z) = \exp \{(1 + z)/(1 - z)\}\), \(g(z) = (1 - z) \exp \{-f(z)\}\). Prove
that
\begin{equation*}
g^*(e^{i\theta}) = \lim_{r\to 1} g(re^{i\theta})
\end{equation*}
exists at every \(e^{i\theta} \in T\), that \(g^* \in C(T)\),
 but that $g$ is not in \(H^\infty\).

\emph{Suggestion}: Fix $s$, put
\begin{equation*}
 z_t = \frac{t + is - 1}{t + is + 1} \qquad (0 < t < \infty).
\end{equation*}
For certain values of $s$, \(|g(z_t)| \to \infty\) as \(t \to \infty\).

\end{excopy}


%%%%%%%%%%%%%% 22
\begin{excopy}
Suppose $u$ is harmonic in $U$, and 
\(\{u_r: 0 \leq r < 1\}\) is a uniformly integrable subset of 
\(L^1(T)\). (See
Exercise~10, Chap.~6.) Modify the proof of Theorem~11.30 to show that
\(u = P[f]\) for some \(f \in L^1(T)\).

\end{excopy}


%%%%%%%%%%%%%% 23
\begin{excopy}
Put \(\theta_n = 2^{-n}\) and define
\begin{equation*}
u(z) = \sum_{n=1}^\infty n^{-2}\{P(z,e^{i\theta_n}) - P(z,e^{-i\theta_n})\},
\end{equation*}
for \(z \in U\). Show that $u$ is the Poisson integral of a measure on $T$,
 that \(u(x) = 0\) if \(-1 < x < 1\), but
that
\begin{equation*}
u(1 — \epsilon + i\epsilon)
\end{equation*}
is unbounded, as \(\epsilon\) decreases to $0$. 
(Thus $u$ has a radial limit, but no nontangential limit, at $1$.)

\emph{Hint}: lf \(\epsilon = \sin \theta\) is small and 
\(z = 1 — \epsilon + i\epsilon\), then
\begin{equation*}
 P(z,e^{i\theta}) - P(z,e^{-i\theta}) > 1/\epsilon.
\end{equation*}
\end{excopy}


%%%%%%%%%%%%%% 24
\begin{excopy}
Let \(D_n(t)\) be the 
\index{Dirichlet} Dirichlet kernel, as in Sec.~5.11, define the 
\index{Fejer@Fej\'er} Fej\'er 
kernel by
\begin{equation*}
K_N = \frac{1}{N+1} (D_0 + D_1 + \cdots + D_n),
\end{equation*}
put \(L_N(t) = \min(N, \pi^2/Nt^2)\). Prove that
\begin{equation*}
K_{N_1}(t) = \frac{1}{N} \cdot \frac{1 - \cos Nt}{1 - \cos t} \leq L_N(t)
\end{equation*}
and that \(\int_R L_N\,d\sigma \leq 2\).

Use this to prove that the arithmetic means
\begin{equation*}
\sigma_N = \frac{S_0 + S_1 + \cdots + S_N}{N + 1}
\end{equation*}
of the partial sums \(s_n\) of the Fourier series of a functionf
\(f \in L^1(T)\) converge to \(f(e^{i\theta})\) at every Lebesgue
point of $f$ (Show that \(\sup |\sigma_N|\) is dominated by \(Mf\),
 then proceed as in the proof of Theorem~11.23.)
\end{excopy}

%%%%%%%%%%%%%% 25
\begin{excopy}
If \(1 \leq p \leq \infty\) and \(f \in L^1(\R^1)\), prove that
 \((f * h_\lambda)(x)\) is a harmonic function of \(x + i/\lambda\) in the upper
half plane. 
(\(h_\lambda\) is defined in Sec.~9.7; 
it is the Poisson kernel for the half plane.)
\end{excopy}

%%%%%%%%%%%%%%%%%
\end{enumerate}



%%%%%%%%%%%%%%%%%%%%%%%%%%%%%%%%%%%%%%%%%%%%%%%%%%%%%%%%%%%%%%%%%%%%%%%%
%%%%%%%%%%%%%%%%%%%%%%%%%%%%%%%%%%%%%%%%%%%%%%%%%%%%%%%%%%%%%%%%%%%%%%%%
%%%%%%%%%%%%%%%%%%%%%%%%%%%%%%%%%%%%%%%%%%%%%%%%%%%%%%%%%%%%%%%%%%%%%%%%
\chapterTypeout{The Maximum Modulus Principle}


\fi


%%%%%%%%%%%%%%%%%%%%%%%%%%%%%%%%%%%%%%%%%%%%%%%%%%%%%%%%%%%%%%%%%%%%%%%%
%%%%%%%%%%%%%%%%%%%%%%%%%%%%%%%%%%%%%%%%%%%%%%%%%%%%%%%%%%%%%%%%%%%%%%%%
%%%%%%%%%%%%%%%%%%%%%%%%%%%%%%%%%%%%%%%%%%%%%%%%%%%%%%%%%%%%%%%%%%%%%%%%
\appendix
\chapterTypeout{Hausdorff's Maximality Theorem}

\index{Hausdorff}

%%%%%%%%%%%%%%%%%%%%%%%%%%%%%%%%%%%%%%%%%%%%%%%%%%%%%%%%%%%%%%%%%%%%%%%%
%%%%%%%%%%%%%%%%%%%%%%%%%%%%%%%%%%%%%%%%%%%%%%%%%%%%%%%%%%%%%%%%%%%%%%%%
\index{Zorn}
\section{Equivalence of Zorn Lemma}

We recall the definition of a \emph{chain}

Given a partially ordered set $X$, a subset \(A\subset X\) is a chain
if the inherited order is a total order.

Let's state Zorn's lemma.
\begin{llem}
Let $X$ be a partially ordered set. If each chain in $X$ has an upper bound
in $X$, then $X$ has a maximal element.
\end{llem}

This is similar to Hausdorff's maximality theorem.
Let's state it and prove it rigorously.

\begin{llem}
Let $X$ be a partially ordered set.
% The following statements are equivalent
\begin{itemize}
 \item[(\textbf{H})]
 If $X$ be a partially ordered set then there exists a maximal chain.
 \item[(\textbf{Z})]
 If $X$ be a partially ordered set and for every chain \(C\subset X\)
 there exists \(b\in X\) such that \(x\leq b\) for all \(x\in C\).
\end{itemize}
\end{llem}
\begin{thmproof}
Assume (\textbf{H}) is true.
Let $X$ be a partially ordered set such that every chain \(C\subset X\)
has a upper bound \(b\in X\). By (\textbf{H}) there exists a maximal chain
\(C_M\). Let $b$ be an upper bound of \(C_m\). If by negation
$b$ is not a maximal element of $X$ then there exists \(u\in X\)
such that \(b<u\), but then \(C_m \subsetneq C_m \cup \{u\}\)
and both are chains, contradiction to the maximality of \(C_m\).
Thus $b$ is a maximal element in $X$.

Conversely, assume (\textbf{Z}) is true. Let $X$ be a partially ordered set.
Let \frakC\ be the family of chains in $X$.
Order the (chain) elements of \frakC\ by inclusion.
For each chain \(c \in \frakC\),
(chain of chains) its union \(u = \cup c\) is also a chain and is
an upper bound of $c$. Thus the assumption of (\textbf{Z}) hold
for \frakC\ (as $X$)
and hence there is a maximal element \(C_M\in\frakC\) which is a maximal chain.
\end{thmproof}

The (\textbf{H}) statement may be viewed as stronger since
when showing (\textbf{Z}) we stayed with the same $X$.
While when showing the converse, we needed to apply $X$
on \(\frakC\subset P(X)\).


%%%%%%%%%%%%%%%%%%%%%%%%%%%%%%%%%%%%%%%%%%%%%%%%%%%%%%%%%%%%%%%%%%%%%%%%
%%%%%%%%%%%%%%%%%%%%%%%%%%%%%%%%%%%%%%%%%%%%%%%%%%%%%%%%%%%%%%%%%%%%%%%%
%%%%%%%%%%%%%%%%%%%%%%%%%%%%%%%%%%%%%%%%%%%%%%%%%%%%%%%%%%%%%%%%%%%%%%%%
% \bibliographystyle{plain}
\bibliographystyle{alpha}
\bibliography{rudinrca}

%%%%%%%%%%%%%%%%%%%%%%%%%%%%%%%%%%%%%%%%%%%%%%%%%%%%%%%%%%%%%%%%%%%%%%%%
%%%%%%%%%%%%%%%%%%%%%%%%%%%%%%%%%%%%%%%%%%%%%%%%%%%%%%%%%%%%%%%%%%%%%%%%
%%%%%%%%%%%%%%%%%%%%%%%%%%%%%%%%%%%%%%%%%%%%%%%%%%%%%%%%%%%%%%%%%%%%%%%%
\printindex


%%%%%%%%%%%%%%%%%%%%%%%%%%%%%%%%%%%%%%%%%%%%%%%%%%%%%%%%%%%%%%%%%%%%%%%%
%%%%%%%%%%%%%%%%%%%%%%%%%%%%%%%%%%%%%%%%%%%%%%%%%%%%%%%%%%%%%%%%%%%%%%%%
%%%%%%%%%%%%%%%%%%%%%%%%%%%%%%%%%%%%%%%%%%%%%%%%%%%%%%%%%%%%%%%%%%%%%%%%
\closegraphsfile
\end{document}

