% -*- latex -*-
% $Id: rudinrca5.tex,v 1.8 2006/04/30 19:06:58 yotam Exp $


%%%%%%%%%%%%%%%%%%%%%%%%%%%%%%%%%%%%%%%%%%%%%%%%%%%%%%%%%%%%%%%%%%%%%%%%
%%%%%%%%%%%%%%%%%%%%%%%%%%%%%%%%%%%%%%%%%%%%%%%%%%%%%%%%%%%%%%%%%%%%%%%%
%%%%%%%%%%%%%%%%%%%%%%%%%%%%%%%%%%%%%%%%%%%%%%%%%%%%%%%%%%%%%%%%%%%%%%%%
\chapterTypeout{Example of Banach Space Techniques} % 5

%%%%%%%%%%%%%%%%%%%%%%%%%%%%%%%%%%%%%%%%%%%%%%%%%%%%%%%%%%%%%%%%%%%%%%%%
%%%%%%%%%%%%%%%%%%%%%%%%%%%%%%%%%%%%%%%%%%%%%%%%%%%%%%%%%%%%%%%%%%%%%%%%
\section{Notes}

%%%%%%%%%%%%%%%%%%%%%%%%%%%%%%%%%%%%%%%%%%%%%%%%%%%%%%%%%%%%%%%%%%%%%%%%
\subsection{Baire's Category Theorem}

Other applications of Baire's theorem are the following

\begin{llem} \label{lem:count:1cat}
If $A$ is a countable subset of a complete metric space $X$,
then $A$ is of first category.
\end{llem}

\begin{thmproof}
As a set --- every point is a nowhere dense. 
Since  $A$ is a countable union of its elements, it is of first category.
\end{thmproof}

\begin{llem} \label{lem:gdel:2cat}
If $D$ is a dense \(G_\delta\) subset of a complete metric space $X$,
then $D$ is of second category.
\end{llem}

\begin{thmproof}
Let \(D=\cup_{n\in\N}G_n\) where \(G_n\) are open. By assumption
all \(G_n\) are dense. Put \(F_n = X \setminus G_n\), now each \(F_n\) is
closed and nowehere dense, since otherwise \(G_n\) would not be dense.
Thus \(F = \cup_{n\in\N}F_n\) is of first category, 
therefore, since $X$ is of second category, 
so is \(D = X \setminus F\).
\end{thmproof}



%%%%%%%%%%%%%%%%%%%%%%%%%%%%%%%%%%%%%%%%%%%%%%%%%%%%%%%%%%%%%%%%%%%%%%%%
\subsection{Working out Dirichlet's kernel}

\index{Dirichlet's kernel}

Let's verify the equality of \textbf{5.11}(9) of \cite{RudinRCA80}.
First note that
% \begin{equation*}
\(e^{i\theta} = \cos\theta + i\sin\theta\)
% \end{equation*}
and so
\begin{equation*}
e^{i\theta} - e^{-i\theta}
=     \bigl(\cos(\theta) - \cos(-\theta)\bigr)
  +  i\bigl(\sin(\theta) - \sin(-\theta)\bigr)
= 2i\sin\theta.
\end{equation*}
We use the above equality twice, in the following computation
\begin{eqnarray*}
D_n(t)
 &\eqdef& \sum_{k=-n}^n e^{ikt} \\
 &=&      \frac{e^{it/2} - e^{-it/2}}{e^{it/2} - e^{-it/2}}
           \sum_{k=-n}^n e^{ikt} \\
 &=& \left(\sum_{k=-n}^n e^{ikt}(e^{it/2} - e^{-it/2})\right)
     \biggm/
     \bigl(2i\sin(t/2)\bigr)\\
 &=& \bigl(e^{i(n+1/2)t} - e^{-i(n+1/2)t}\bigr) \bigm/
     \bigl(2i\sin(t/2)\bigr) \\
 &=& 2i\sin\bigl((n+1/2)t\bigr) \bigm/
     \bigl(2i\sin(t/2)\bigr) \\
 &=& \sin\bigl((n+1/2)t\bigr) / \sin(t/2). \\
\end{eqnarray*}

%%%%%%%%%%%%%%%%%%%%%%%%%%%%%%%%%%%%%%%%%%%%%%%%%%%%%%%%%%%%%%%%%%%%%%%%
\subsection{Parseval's Identity}
\index{Parseval}

Let's derive an equality in section~5.24. Given 
\begin{equation*}
f(z) = \sum_{n=0}^N b_n(z-z_0)^n
\end{equation*}
we have
\begin{align*}
 \frac{1}{2\pi}
  \int_\pi^\pi \left|f(z_0 + re^{i\theta}\right|^2\,d\theta
&= \frac{1}{2\pi} \int_\pi^\pi 
   \left(\sum_{n=0}^N b_n r^n e^{ni\theta}\right) 
   \cdot
   \overline{\left(\sum_{n=0}^N b_n r^n e^{ni\theta}\right)}
   \,d\theta \\
&= \frac{1}{2\pi} \int_\pi^\pi 
   \left(\sum_{n=0}^N b_n r^n e^{ni\theta}\right) 
   \cdot
   \left(\sum_{n=0}^N \overline{b_n} r^n e^{-ni\theta}\right)
   \,d\theta \\
&=  \frac{1}{2\pi}  \sum_{m,n=0}^N  \int_\pi^\pi 
   b_m \overline{b_n} r^{m+n} e^{(m-n)i\theta}
   \,d\theta \\
&=  \frac{1}{2\pi}  \sum_{m,n=0}^N  
    b_m \overline{b_n} r^{m+n} 
   \int_\pi^\pi e^{(m-n)i\theta}
   \,d\theta \\
&=  \sum_{n=0}^N |b_n|^2 r^{2n} 
\end{align*}


%%%%%%%%%%%%%%%%%%%%%%%%%%%%%%%%%%%%%%%%%%%%%%%%%%%%%%%%%%%%%%%%%%%%%%%%
\subsection{Working out Poisson's kernel}

\index{Poisson's kernel}

Let's verify the equality of \textbf{5.24}(8) of \cite{RudinRCA80}.
\begin{eqnarray*}
1+2\sum_1^\infty \left(ze^{-it}\right)^n
&=& 1 + 2\left(-1 + \sum_0^\infty \left(ze^{-it}\right)^n\right) \\
&=& -1 + 2/\left(1 - ze^{it}\right) \\
&=& \frac{ze^{-it} - 1 + 2}{1 - ze^{-it}} \\
&=& \frac{e^{it} + z}{e^{it} - z} \\
&=& \frac{1 + ze^{-it}}{1 - ze^{-it}} \\
&=& \frac{ (1 + ze^{-it})\overline{(1 - ze^{-it}) } }{
           (1 - ze^{-it})\overline{(1 - ze^{-it}) } } \\
&=& \frac{ 1 + ze^{-it} - \overline{ze^{-it}} - ze^{-it}\overline{ze^{-it}} }{
           |1 - ze^{-it}|^2 } \\
&=& \frac{ 1 + re^{i\theta}e^{-it} - re^{-i\theta}e^{it}
             - re^{i\theta}e^{-it}re^{-i\theta}e^{it} }{
           |1 - ze^{-it}|^2 } \\
&=& \frac{ 1 - r^2 + r\left(e^{i(\theta-t)} - e^{i(t-\theta)}\right)
         }{ |1 - ze^{-it}|^2 } \\
&=& \frac{ 1 - r^2 + 2ir\sin(\theta-t) }{ |1 - ze^{-it}|^2 } \\
\end{eqnarray*}


%%%%%%%%%%%%%%%%%%%%%%%%%%%%%%%%%%%%%%%%%%%%%%%%%%%%%%%%%%%%%%%%%%%%%%%%
%%%%%%%%%%%%%%%%%%%%%%%%%%%%%%%%%%%%%%%%%%%%%%%%%%%%%%%%%%%%%%%%%%%%%%%%
\section{Exercises} % pages 112-115

%%%%%%%%%%%%%%%%%
\begin{enumerate}
%%%%%%%%%%%%%%%%%

%%%%%%%%%%%%%% 1
\begin{excopy}
Let $X$ consist of two points $a$ and $b$,
put \(\mu(\{a\}) = \mu(\{b\}) = \half\),
and let \(L^p(\mu)\) be the resulting \emph{real} \(L^p\)-space.
Identify each real function $f$ on $X$ with the point \((f(a),f(b))\)
in the plane, and skecth the unit balls of \(L^p(\mu)\),
for\(0<p\leq \infty\).
Note that they are convex if and only if \(1\leq p \leq \infty\).
For which $p$ is the unit ball a square? A circle?
If \(\mu(\{a\}) \neq \mu(\{b\})\), how does the situation differ from the
preceding one?
\end{excopy}

The unit balls are \(B_p=\{(x,y)\in\R^2: x^p + y^p \leq 1\}\).
Assume \(1\leq p \leq \infty\) and let \((x_1,y_1),(x_2,y_2)\in B_p\).
By Minkowski's inequality (Theorem~3.5 \cite{RudinRCA87})
\[
\bigl((x_1+x_2)^p/2 + (y_1+y_2)^p/2\bigr)^{1/p}
\leq
   (x_1^p/2 + y_1^p/2)^{1/p}
 + (x_2^p/2 + y_2^p/2)^{1/p}
.\]
This is equivalent to
\(\|(x_1+x_2,y_1+y_2)\|_p \leq \|(x_1,y_1)\|_p + \|(y_2,y_2)\|_p\)
or
\[\left\|\bigl((x_1+x_2)/2,(y_1+y_2)/2\bigr)\right\|_p
 \leq
 \bigl(\|(x_1,y_1)\|_p + \|(y_2,y_2)\|_p\bigr)\,/\,2.
\]
which show convextiy.

If \(0<p<1\), take real \(\alpha\) such that \(\alpha^p = 2\).
Let
\begin{eqnarray*}
(x_1,y_1) &\eqdef& (\alpha, 0) \\
(x_2,y_2) &\eqdef& (0, \alpha)
\end{eqnarray*}
Now clearly
\(\|(x_1,y_1)\|_p =  \|(x_2,y_2)\|_p = 1\)
but
\[
   \|\bigl((x_1+x_2)/2,(y_1+y_2)/2\bigr)\|_p^p
=  (\alpha/2)^p)/2 + (\alpha/2)^p)/2
=  (\alpha/2)^p
=  2^{1-p} > 1.
\]
Hence \(B_p\) is not convex.

When \(p=1\) the unit ball is a square inscribed in the unit circle.
When \(p=2\) the unit ball is the unit circle.
When \(p=\infty\) the unit ball is a square circumscribed around the unit circle.

If \(\mu(\{a\}) \neq \mu(\{b\})\) the squares change to rectangles
and the circle eo ellipse.
% 1-800 340340

%%%%%%%%%%%%%% 2
\begin{excopy}
Prove that the unit ball (open or closed) is convex in every normed linear space.
\end{excopy}

Let $U$ be an open unit ball of a normed linear space
and let \(v_0,v_1\in U\).
For any \(t\in[0,1]\) let \(v_t \eqdef v_0 + t(v_1 - v_0)\).
We need to show that \(v_t\in U\).
We may assume that \(v_0\neq v_1\) and \(0<t<1\), since otherwise
we trivially have \(v_t=v_0\) or \(v_t=v_1\).
By definition of a norm
\[
\|v_t\| = \|(1-t)v_0 + tv_1\|
\leq \|(1-t)v_0\| + \|tv_1\|
=     (1-t)\|v_0\| + t\|v_1\| < (1-t)+t = 1.
\]

If $U$ is a closed unit ball,
we simply change the last strict inequality ($<$) into \(\leq\).


%%%%%%%%%%%%%% 3
\begin{excopy}
If \(1<p<\infty\), prove that the unit ball
of \(L^p(\mu)\) is
\index{strictly convex}
\emph{strictly convex};
this means if
\[ \|f\|_p = \|g\|_p = 1, \qquad f\neq g, \qquad h = \half(f+g), \]
then \(\|h\|_p < 1\).
(Geometrically, the surface of the ball contains no straight lines.)
Show that this fails in every \(L^1(\mu)\), and in every \(C(X)\).
(Ignore trivialities, such as spaces consisting of only one point.)
\end{excopy}

Let $X$ be the functions' domain.
For all \(x\in X\) we have
\[|h(x)| = \half|f(x)+g(x)| \leq \half(|f(x)|+g|(x)|).\]
If there is in equality, then by local lemma~\ref{llem:minkowski:eq},
there are non negative real constants $a$ and $b$ such that
\(af=bg\;\aded\), but since \(\|f\|_p = \|g\|_p\)
we have \(a=b\) and the case is trivial \(f=g\;\aded\).


%%%%%%%%%%%%%%
\begin{excopy}
Let $C$ be the space of all continuous functions on \([0,1]\),
with the supremum norm.
Let $M$ consist of all \(f\in C\) for which
\[ \int_0^\half f(t)\,dt - \int_\half^1 f(t)\,dt = 1. \]
Prove that $M$ is a closed subset of $C$ which contains no element
of minimal norm.
\end{excopy}

If \(f_n\to f\) in the supremum norm, clearly
\[ \lim_{n\to \infty} \int_0^\half f_n(t)\,dt - \int_\half^1 f_n(t)\,dt =
                      \int_0^\half f(t)\,dt - \int_\half^1 f(t)\,dt = 1. \]
Thus $M$ is closed.

We will now show that \(\|f\|_\infty > 1\) for every \(f\in M\).
Let \(f\in C([0,1])\) be such that \(\|f\|_\infty \leq 1\).
and let \(h=f(1/2)\).
Two cases:
\begin{itemize}
\item
If \(|h|<1\) then let \(\delta>0\) be such that \(|f(x)|<(1+|h|)/2\)
for all \(x\in (1/2-\delta,1/2+\delta)\). Now
\begin{eqnarray*}
\left|\int_0^\half f(t)\,dt - \int_\half^1 f(t)\,dt\right|
&=&
\left|\int_0^{1/2-\delta} f(t)\,dt
+ \int_{1/2-\delta}^\half f(t)\,dt
- \int_\half^{\half+\delta} f(t)\,dt
- \int_{\half+\delta}^1 f(t)\,dt\right| \\
&\leq& (1/2-\delta) + (\delta/2 + \delta/2)(1+|h|)/2 + (1/2 - \delta) \\
&=& 1 + 2\bigl((1+|h|)/2 - 1\bigr)\delta \\
&<& 1.
\end{eqnarray*}

\item
If \(|h|=1\), then let \(\delta>0\) be such that \(|f(x)-h| < 1/3\)
when \(x\in (\half-\delta,\half+\delta)\).
Note that if \(|y_i-h|<1/3\) for \(i=1,2\) then \(|y_1-y_2| < 2/3\).
Now
\begin{eqnarray*}
\left|\int_0^\half f(t)\,dt - \int_\half^1 f(t)\,dt\right|
&=&
\left|\int_0^{1/2-\delta} f(t)\,dt
+ \int_{1/2-\delta}^\half f(t)\,dt
- \int_\half^{\half+\delta} f(t)\,dt
- \int_{\half+\delta}^1 f(t)\,dt\right| \\
&\leq& (1/2 - \delta)
       + \int_0^\delta \left(f(t+h-\delta) - f(t+h)\right)dt
       + (1/2 - \delta) \\
&\leq& 1 - 2\delta + (2/3)\delta \\
&<& 1.
\end{eqnarray*}
\end{itemize}
In both case we see that \(f\notin M\).
Next we will show that for any \(\epsilon>0\) there exists \(f\in M\)
such that \(\|f\|_\infty > 1 + \epsilon\).
These results combined, show that $M$ has no minimal normed element.


\[
\left|\int_0^\half f(t)\,dt - \int_\half^1 f(t)\,dt\right|
\leq \int_0^\half |f(t)|\,dt + \int_\half^1 |f(t)|\,dt
= \int_0^1 |f(t)|\,dt \leq \|f\|_\infty
\]
Let \(\epsilon>0\). Define
\[
f(x) = \left\{
       \begin{array}{ll}
       -1 - \epsilon& \qquad 0\leq x \leq 1/2 - \delta \\
        (x-1/2) (1+\epsilon)/\delta
                        & \qquad 1/2 - \delta \leq x \leq 1/2 + \delta \\
       1 + \epsilon  & \qquad 1/2 + \delta \leq x \leq 1
       \end{array}\right.
\]
Clearly \(\|f\|_\infty = 1/2+\epsilon\) and if we pick
\(\delta\) such that
\[2(1+\epsilon)(1/2 - \delta) + 2\delta(1+\epsilon)/2 = 1\]
we can have \(f\in M\).
Simplifying:
\[2(1+\epsilon)(1/2 - \delta) + 2\delta(1+\epsilon)/2
  = (1+\epsilon)(2(1/2 - \delta) + \delta)
  = (1+\epsilon)(1 - \delta)\]
and solving the above,
gives \(\delta = 1 - 1/(1+\epsilon) = \epsilon/(1+\epsilon)\).



%%%%%%%%%%%%%% 5
\begin{excopy}
Let $M$ be the set of all \(f\in L^1([0,1])\), relative to Lebesgue measure,
such that
\[\int_0^1 f(t)\,dt = 1.\]
Show that $M$ is a closed subset of  \(L^1([0,1])\) which contains
infinitely many element of minimal norm.
(Compare this and Exercise~4 with Then~4.10.)
\end{excopy}

Let \(\{f_n\}\) be a sequence in $M$ such that
\(\lim_{n\to\infty}\|f_n - f\|_1 = 0\)
where \(f\in L^1([0,1])\). Now
\begin{eqnarray*}
\|f\|_1 &\leq& \|f_n - f\|_1 + \|f_n\|_1 \\
 \|f_n\|_1 &\leq& \|f_n - f\|_1 + \|f\|_1
\end{eqnarray*}
and so given \(\epsilon > 0\) for sufficiently large $n$
\[ 1 - \epsilon \leq \|f\|_1 \leq 1 + \epsilon\]
thus \(\|f\|_1=1\) and $M$ is closed.



%%%%%%%%%%%%%% 6
\begin{excopy}
Let $f$ be a bounded linear functional on a subspace $M$ of a Hilbert space $H$.
Prove that $f$ has a \emph{unique} norm-preserving extension to a bounded
linear functional on $H$, and that this extension vanishes on \(M^\perp\)
\end{excopy}

The case \(f=0\) is trivial, thus \wlogy by dividing by \(\|f\|\)
we may assume that \(\|f\|=1\).
Clearly for every
\(x=v+w\in H\) where \(v\in M\) and \(w\in M^\perp\)
we can define \(F(x) = f(v)\) which is a well defined extension
and clearly \(\|F\|=1\).

Now we will show that this extension is unique.
Let $F$ be an norm-preserving extension of $f$ on $H$.
By negation, assume that $F$ does not vanish on \(M^\perp\)
we can find \(w\in M^\perp\) such that \(\|w\|=1\)
and \(F(w)=a>0\). Pick and arbitrary \(\epsilon > 0\),
for which pick \(u\in M\) such that \(\|u\|=1\)
and \(h \eqdef f(u) > 1-\epsilon\).
We will contradict the assumtion if we show
that there exist a real $t$ such that
\[\frac{F(u+tw)}{\|u+tw\|} > 1.\]
Equivalently,
\[(h+ta)^2 = (F(u+tw))^2 > \|u+tw\|^2 = t^2 + 1.\]
that is
\[(1-a^2)t^2 - 2hat + (1-h^2) < 0.\]
Since the leaading coefficient \(1-a^2>0\),
by looking at the discriminant, there exist some $T$
satisfying the above inequality(ies), iff
\[4h^2a^2 - 4(1-a^2)(1-h^2) > 0\]
that is if
\begin{equation} \label{eq:ex:5.6:disc}
(h^2a^2)>(1-a^2)(1-h^2).
\end{equation}
Since \(\lim_{\epsilon\to 0} h = 1\),
When \(\epsilon\to 0\), the left side of \eqref{eq:ex:5.6:disc}
converges to \(a^2 > 0\) while zero is the limit of the right side.
Hence such $t$ exists, \(\|F\|>1\) contradicting the assumtion.


%%%%%%%%%%%%%% 7
\begin{excopy}
Construct a bounded linear functional on some subspace of some \(L^1(\mu)\)
which has two (hence inifinitely many) distinct norm-preserving
to \(L^1(\mu)\).
\end{excopy}

Define \(X=\{0,1\}\) and \(\mu(\{0\}) = \mu(\{1\}) = 1/2\)
Let
\[ M \eqdef \{f\in L^1(X,\mu): f(1) = 0\}\]
and the functional on $M$  \(\Lambda f = f(0)/2\).
If \(f\in M\) then \(\|f\|_1 = |f(0)|/2\), and so
\[ \|\Lambda\|
= \sup \{|\Lambda f|: f \in M\;\wedge\;\|f\|=1\}
= \sup \{|f(0)/2|: f \in M\;\wedge\;|f(0)|/2=1\}
= 1. \]

Now we can easily extend \(\Lambda\) to \(L^1(X,\mu)\)
by the following two extensions
\begin{eqnarray*}
\Lambda_1(f) &=& f(0)/2 \\
\Lambda_2(f) &=& f(0)/2 + f(1)/2
\end{eqnarray*}


%%%%%%%%%%%%%% 8
\begin{excopy}
Let $X$ be a normed linear space, and let \(X^*\) be its dual space, as defined
in Sec.~5.21, with the norm
\[\|f\| = \sup \{|f(x)|: \|x\|\leq 1\}.\]
\begin{itemize}
 \itemch{a} Prove that \(X^*\) is a Banach space.
 \itemch{b}
 Prove that the mapping \(f\to f(x)\) is, for each \(x\in X\),
 a bounded linear functional on \(X^*\), of norm \(\|x\|\).
 (This gives a natural imbedding of $X$ on its ``second dual'' \(X^{**}\),
 the dual space of \(X^*\).)
 \itemch{c} Prove that \(\{\|x_n\|\}\) is bounded if \mset{x_n} is a sequence
 in $X$ such that \(\{f(x_n)\}\) is bounded for every \(f\in X^*\).
\end{itemize}
\end{excopy}

\begin{itemize}
 \itemch{a}
 Let \(\Lambda_1,\Lambda_2 \in X^*\) and \(z_1,z_2 \in \C\)
 for any \(x\in X\), by definition, we have
 \[ (z_1\Lambda_1 + z_2\Lambda_2)(x) = z_1\Lambda_1(x) + z_2\Lambda_2)(x)  \]
 and so \(X^*\) is linear vector space over \(\C\).
 Also \(\|z\Lambda(x)\| = |z||\Lambda(x)|\) so to prove \(\|\cdot\|_{X^*}\)
 is a norm, we show the triangle inequality:
 \begin{eqnarray*}
 \|\Lambda_1 + \Lambda_2\| 
 &=& \sup\{|(\Lambda_1 + \Lambda_2)(x)|: x\in X,\;\|x|=1\} \\
 &=& \sup\{|\Lambda_1(x) + \Lambda_2(x)|: x\in X,\;\|x|=1\} \\
 &\leq& \sup\{|\Lambda_1(x)| + |\Lambda_2(x)|: x\in X,\;\|x|=1\} \\
 &\leq&    \sup\{|\Lambda_1(x)|: x\in X,\;\|x|=1\}
         + \sup\{|\Lambda_2(x)|: x\in X,\;\|x|=1\} \\
 &=& \|\Lambda_1\| + \|\Lambda_2\|.
 \end{eqnarray*}
 It is now left to show completeness. 
 Let \mset{\Lambda_n} be a Cauchy sequence in \(X^*\).
 Pick arbitrary \(x\in X\setminus\{0\}\). Now since
 \begin{eqnarray*}
 |\Lambda_m(x) - \Lambda_n(x)|
 = |(\Lambda_m - \Lambda_n)(x)|
 = \|x\|\cdot |(\Lambda_m - \Lambda_n)(x/\|x\|)|
 \leq \|x\|\cdot\|\Lambda_m - \Lambda_n\|
 \end{eqnarray*}
 \mset{\Lambda_n x} is a Cauchy sequence in \(\C\) and converges there,
 and for \(x=0\) as well.
 Thus we can define the limit 
 \[ \Lambda_x \eqdef \lim_{n\to\infty} = \lim_{n\to\infty} \Lambda_n x.\]
 Now for \(x_1,x_2\in X\) and \(z_1,z_2\in \C\)
 \begin{eqnarray*}
 \Lambda (z_1 x_1 + z_2 x_2)
 &=& \lim_{n\to\infty} \Lambda_n (z_1 x_1 + z_2 x_2) \\
 &=& \lim_{n\to\infty} \bigl(z_1\Lambda_n(x_1) + z_2 \Lambda(x_2)\bigr) \\
 &=& \bigl(
             \lim_{n\to\infty} z_1\Lambda_n(x_1)
           + \lim_{n\to\infty} z_2\Lambda_n(x_2) \bigr) \\
 &=& \bigl(  z_1\lim_{n\to\infty} \Lambda_n(x_1)
           + z_2\lim_{n\to\infty} \Lambda_n(x_2) \bigr) \\
 &=& z_1\Lambda(x_1) + z_2\Lambda(x_2).
 \end{eqnarray*}
 To show \(\Lambda\) is bounded, let \(\|x\| \leq 1\) in $X$, now
 \[ |\Lambda(x)| = |\lim_{n\to\infty} \Lambda_n(x)| 
    \leq = \lim_{n\to\infty} \|\Lambda_n\| < \infty\]
 and the limit is independent of $x$.

 Therefore \(X^*\) \emph{is} a banach space.
 
 \itemch{b} Let \(x\in X\). Since
 \[(z_1f_1+z_2f_2)(x) = z_1f_1(x) + z_2f_2(x)
   \qquad f_1,f_2\in X^*\;z_1,z_2\in \C\]
 clearly \(f\to f(x)\) is a linear functional on \(X^*\).
 Define \(x^{**}\in X^{**}\) by \(x^{**}(f) = f(x)\) for all \(f\in X^*\).
 If \(x=0\) then \(\|x^**\|_{X^{**}} = 0\) trivially, for \(x\neq 0\)
 \begin{eqnarray}
 \|x^{**}\|_{x^{**}} 
 &=& \sup\{|f(x)|: \|f\|_{X^*} = 1,\;f\in X^*\} \notag \\
 &=& \sup\bigl\{|f(x)|: \sup\{ |f(w)|: \|w\|=1,\; f\in X^*,\;w\in X\}=1\bigr\} 
     \notag \\
 &\leq& \sup\bigl\{|f(x)|: \sup\{ |f(x/\|x\|)|: f\in X^*\}=1\bigr\} 
        \label{eq:5.8:sup1} \\
 &\leq& \|x\| \label{eq:5.8:sup2}.
 \end{eqnarray}
 We now show the reverese inequality. Define a functional \(f_x\)
 on a 1-dimensional subspace of $X$ generated by $x$, as \(f_x(zx/\|x\|) = z\).
 Clearly \(\|f_x\| = f(x)/\|x\| = 1\)
 and applying Hahn-Banach theorem, we can extend
 \(f_x\) to a functional $f$ on whole $X$, such that 
 \(\|f\|_{X^*} = \|f_x\| = 1\).
 Using it in the supremum of \eqref{eq:5.8:sup1} and \eqref{eq:5.8:sup2} 
 we get an equality and so \(\|x^{**}\|_{X^{**}} = \|x\|\).
 
 \itemch{c}
 By negation, assume \mset{x_n} is unbounded.
 Let $W$ be the subspace of $X$ generated by \mset{x_n}.
 Two cases:

 \paragraph{Case 1.} Assume \(\dim(W) < \infty\) and \seqn{v} its base.
 Thus we have a unique representation 
 \(x = \sum_{i=j}^n a_j(x) v_j\), where \(a_j\in X^*\).
 Since \mset{x_n} is unbounded, there exist some $j$ such that
 \(\{a_j(x_i)\}_{i=1}^\infty\) is not bounded.

 \paragraph{Case 2.} Assume \(\dim(W)\infty\). 
 Let \(W_n\) be the subspace of $W$ generated by \seqn{x}.
 Clearly \(W = \cup_{n=1}^\infty W_n\).
 For convenience, we drop from \seqn{x} any \(x_j\) such that 
 \(x_j \in X_{j-1}\); note that zeros are dropped. 
 This case assumtion, guarantees that the sequence is still infinite.
 The functional $f$ to be constructed, 
 will furnish a contradiction, namely \mset{f(x_n)} unbounded for 
 the original, trivially, as well.

 For each \(w\in W\) there is a unique (finite) representation
 \[ w = \sum_{j=1}^{M(w)} a_j(w) (x_j/\|x_j\|).  \]
 Note that \(a_n(x_n) = \|x_n\|\).
 We will define a sequence of functionals \(f_n: W_n \to \C\).
 \begin{eqnarray*}
 % u_n &=& \overline{a_n(x_n)}/|x_n| \\
 f_n(w) &=& f\left(\sum_{j=1}^{M_w} a_j(w) x_j\right) 
       \eqdef \sum_{j=1}^{M_w} u_j a_j(w) \qquad w \in W_{M_w}
 \end{eqnarray*}
 By construction, 
 \begin{eqnarray*}
  % |u_n| &\in& \{0,1\} \\
  f_n(x_n) &=& \sum_{j=1}^n a_j(x_j) = \|x_n\| \\
  f_n &=& {f_m}_{|W_n}  \qquad \textrm{if}\, n\leq m
 \end{eqnarray*}
 Thus we can define the functional \(f = \cup f_n\), 
 for which \(f(x_n) = \|x_n\|\) for all $n$.
\end{itemize}


%%%%%%%%%%%%%% 9
\begin{excopy}
Let \(c_0\), \(\ell^1\), and \(\ell^\infty\) be Banach spaces consisting
of all complex sequences \(x=\{\xi_i\}\), \(\mbox{i=1,2,3,\ldots,}\) 
defined as follows:
\begin{alignat*}{3}
x&\in\ell^1 & &\quad\textrm{if and only if} 
   \quad && \|x\|_1 = \sum|\xi_i|<\infty.\\
x&\in\ell^\infty && \quad\textrm{if and only if} 
   \quad && \|x\|_\infty = \sup |\xi_i|<\infty.
\end{alignat*}
\(c_0\) is the subspace of \(\ell^\infty\) consisting of all \(x\in\ell^\infty\)
for which \(\xi_i \to 0\) as \(i\to\infty\).

Prove the following statements.
\begin{itemize}
 \itemch{a} If \(y=\{\eta_i\}\in \ell^1\) and \(\Lambda x = \sum \xi_i\eta_i\)
            for every \(x\in c_0\), then \(\Lambda\)
            is a~bounded linear functional on \(c_0\),
            and \(\|\Lambda\| = \|y\|_1\).
            Moreover, every \(\Lambda \in (c_0)^*\) is obtain in this way.
            In brief, \((c_0)^* = \ell^1\).

            (More precisely, these two space are not equal, the preceding
            statement, exhibits as isometric vector space isomorphism
            between them.
 \itemch{b} In the same sense, \((\ell^1)^* = \ell^\infty\).
 \itemch{c} Every \(y\in \ell^1\), induces a bounded linear functional on
            \(\ell^\infty\), as in \ich{a}.
            However, this does \emph{not} give all of \((\ell^\infty)^*\),
            since \((\ell^\infty)^*\) contains nontrivial functionals that
            vanish on all of \(c_0\).
 \itemch{d} \(c_0\) and \(\ell^1\) are separable but \(\ell^\infty\) is not.
\end{itemize}
\end{excopy}

\begin{itemize}
\itemch{a}
Clearly, \(\Lambda\) is linear.
Estimate:
\[ |\Lambda x| 
   = \left|\sum \eta_i\xi_i\right|
   \leq \sum |\eta_i \xi_i|
   \leq \left(\sum |\eta_i|\right) \left(\sup |x_i|\right) 
   = \|y\|_1 \cdot \|x\|\_infty.\]
Thus \(\|\Lambda\| \leq \|y\|_1\). 

For the reverse inequality, we may assume \(y\neq 0\).
let \(\|y\|_1 > \epsilon>0\),
let \(N>0\) be such that \(\sum_{i>N} |\eta_i| < \epsilon\).
% and let $M$ be the number of nonzero components of $y$ upto $N$;
% formally: \(M = |\{i\in\N: 1\leq i \leq N\,\wedge\, \eta_i=0\}|\).
In the next definitions and derivations we freely use \(0/0 = 0\).
Put \(u_i = \overline{\eta_i}/|\eta_i|\).
and define \(x=(\xi_i)_i\) by 
\[
   \xi_i = \left\{\begin{array}{ll}
                   u_i   & \qquad \textrm{if}\;  1\leq i\leq N \\
                   0     & \qquad \textrm{if}\;  i> N
                  \end{array}\right.\]

Now 
\[\Lambda x 
 = \sum \xi_i\eta_i
 = \sum_{i=1}^N \overline{\eta_i}\eta_i \,/\, |\eta_i|
 = \sum_{i=1}^N |\eta_i|
 > \|y\|_1 - \epsilon.\]
Since \(\epsilon\) can be arbitrarily small, \(\|\Lambda\|\geq \|y\|_1\).

Now Pick arbitrary \(f\in c_0^*\). 
Define \mset{e_i} in \(c_0\) as a function \(\N\to \C\)
by \(e_i(n) = delta_{in}\) that is \(e_i(i)=1\) and zero otherwise.
Define \(x=(\xi_i)_{i=1}^\infty\) by \(\xi_i = f(e_i)\)
and \(\Lambda\) as above. 
For each \(j\in \N\), we have
\[\Lambda e_j = \sum_{i=1}^\infty \xi_i e_j(i) = \xi_j.\]
Thus \(\Lambda\) and $f$ agree on the subspace $S$ generated by \mset{e_i}.
It is easy to see that \(c_0 = \overline{S}\) and thus \(\Lambda = f\)
that is every functional on \(c_0\) is given by the above ``producs-sum'' form.

\itemch{b}
 Let \(x=\{\xi_i\}\in \ell^1\) 
 \(y=\{\eta_i\}\in \ell^\infty\) and \(\Lambda x = \sum \eta_i\xi_i\).
 for 
 \[ |\Lambda x| \leq \sum |\eta_i\xi_i| 
    \leq \left(\sup \eta_i\right)\left( \sum |\eta_i\xi_i| \right)
    = \|y\|_\infty \cdot \|x\|_1.\]
 Hence \(\|\Lambda\| \leq \|y\|_\infty\).
 Given \(\epsilon>0\) pick some $J$ such that \(|\eta_j| \geq \|y\|-\epsilon\),
 and let \(e_j\in l_1\) 
 such that \(e_j(n) = 1\) if \(n\neq j\) and \(e_j(j)=1\).
 Now \(\Lambda e_j = \eta_j\) and so \(\|\Lambda\| \geq \|y\|-\epsilon\).
 Since \(\epsilon\) may be arbitrarily small, \(\|\Lambda\| \geq \|y\|_\infty\),
 hence \(\|\Lambda\| = \|y\|_\infty\).

 Now let \(f\in (\ell^1)^*\). Define \(e_j\) as before for all $j$, 
 let \(\eta_j= f(e_j)\) and let \(y=\{\eta_i\}\).
 Since \(\|e_j\|=1\) for all $J$, we have \(|\eta_j| \leq \|f\|\)
 and so \(y\in \ell^\infty\).

\itemch{c}
 Construct a functional \(f\in (\ell^\infty)^*\) as follows.
 It vanishes \(f(x)=0\) for all \(x\in c_0\). 
 Pick \(\mathbf{u}=(u_i)_{i\in\N}\)
 such that \(u_i=1\) for all \(i\in\N\). 
 Clearly \(u\in \ell^\infty\setminus c_0\).
 Now define \(f(u)=1\) and extend $f$ to \(\ell^\infty\) by Hahn-Banach theorem.
 Clearly $f$ is not in the image of \(\ell^1\) embedding in 
 \((\ell^\infty)^{**}\) 
 
\itemch{d}
 The extension field \(\Q(i)\in \C\) of the rational is actually
 \[\Q(i) = \{a+bi:\;a,b\in\Q\}.\]
 Clearly \(|\Q(i)|=\aleph_0\).
 Define the set $E$ of inifinite sequences of \(\Q(i)\) which 
 have only finite nonzero components. Formally:
 \[ E \eqdef \left\{x\in \bigl(Q(i)\bigr)^\N:\; 
                     \exists M,\; \forall n>m\,\Rightarrow x(n)=0\right\}.\]
 Now \(|E|=\aleph_0\) and $E$ is dense in \(c_0\) and \(\ell^1\).
 
 {\small
 \emph{Sketch:} Pick \(v\in c_0\) or \(v \in \ell^1\) and \(\epsilon>0\).
 \(c_0\)-case: \(\exists M, \forall n>m,\, |v_n|<\epsilon/2\). 
 \(\ell^1\)-case: \(\exists M, \sum{j=1}^M |v_j| <\epsilon/2\). 
 Then \(\epsilon/2\)-approximate $v$ trimmed above $M$, by some \(x\in E\).
 Now \(\|x-v\|<\epsilon/2+\epsilon/2=\epsilon\).
 }

 Define a subset
 \[V \eqdef  \{0,1\}^\N \subset \ell^\infty \subset \C^\N.\]
 Note that \(\|x-y\|\in\{0,1\}\) for all \(x,y\in V\).
 Clearly \(|V| = 2^{\aleph_0} > \aleph_0\).
 If by negation \(\ell^\infty\) is separable, let $D$ be a countable dense
 set in  \(\ell^\infty\). 
 Define a mapping \(\nu:V\to D\) by picking 
 for each \(v\in V\) an \(x\in D\) such that \(\|x-v\|<1/2\).
 Existence of \(\nu(v)\) is guaranteed by density of $D$.
 This map must be one-to-one, since if for some \(x\in D\)
 there exist \(v_1,v_2\in V\) such that \(\|x-v_i\|<1/2\) for \(i=1,2\)
 then \(\|v_1-v_2\|<1\) but then \(v_1=v_2\).
 But now a cardinality \(|V|<|D|\) contradiction.
\end{itemize}

%%%%%%%%%%%%%% 10
\begin{excopy}
If \(\sum \alpha_i\xi_i\) converges for every sequence \((\xi_i)\) such that
\(\xi_i \to 0\) as \(i\to\infty\), prove that \(\sum|\alpha_i| < \infty\).
\end{excopy}

This is a consequence of Exercise~9\ich{a} above.

%%%%%%%%%%%%%% 11
\begin{excopy}
For \(0<\alpha\leq 1\),
\index{Lip@\(\Lip\)}
let \(\Lip\alpha\) denote the space of all complex
functions $f$ on \([a,b]\) for which
\[M_f = \sup_{s\neq t} \frac{|f(s)-f(t)|}{|s-t|^\alpha} < \infty.\]
Prove  that \(\Lip\alpha\) is a Banach space, if \(\|f\| = |f(a)| + M_f\);
also if
\[\|f\| = M_f + \sup_x |f(x)|.\]
(The members of \(\Lip\alpha\) are said to satisfy
\index{Lipschitz condition}
a~\emph{Lipschitz condition} of order \(\alpha\).)
\end{excopy}

We first show that \(\|\cdot\|\) is indeed a norm.
\paragraph{Scalar Multiplication}.
Let \(z\in\C\). For all \(f\in \Lip\alpha\)
\[
M_{zf} 
= \sup_{s\neq t} \frac{|(zf)(s)-(zf)(t)|}{|s-t|^\alpha} 
= |z|\sup_{s\neq t} \frac{|f(s)-f(t)|}{|s-t|^\alpha} 
|z| M_f.\]
Hence
\[
\|zf\| = |(zf)(a)| + M_{zf} = |z|\cdot|f(a)| + |z|M_f = |z|\cdot\|f\|.
\]

\paragraph{Sub additivity}. For all \(f,g\in \Lip\alpha\)
\begin{eqnarray*}
\|f+g\| 
&=& |(f+g)(a)| + \sup_{s\neq t} \frac{|(f+g)(s)-(f+g)(t)|}{|s-t|^\alpha} \\
&=& |f(a) + g(a)| + 
  \sup_{s\neq t} \frac{\left|\bigl(f(s)-f(t)\bigr) + 
                             \bigl(g(s)-g(t)\bigr)\right|)}{|s-t|^\alpha} \\
&\leq& |f(a)| + |g(a)| + 
     + \sup_{s\neq t} \frac{|(f(s)-f(t)|}{|s-t|^\alpha} 
     + \sup_{s\neq t} \frac{|(g(s)-g(t)|}{|s-t|^\alpha} \\
&=& \|f\|+\|g\|.
\end{eqnarray*}

Clearly if \(\|f\|=0\) then \(f(a)=0\) and \(f'(t)=0\) for all \(t\in [a,b]\)
and so \(f=0\).

To show that \(\Lip\alpha\) is a Banach space it is left to show completeness.
Let \mset{f_n} be a Cauchy sequence in \(\Lip\alpha\).
Since \(|f_m(a)-f_n(a)| \leq \|f_m - f_n\|\), 
the evaluations \(\{f_n(a)\}\) is a Cauchy sequence in \(\C\).
Similarly for \(t\in(a,b]\)
\[
\frac{|(f_m - f_n)(t) - (f_m - f_n)(a)|}{|s-t|^\alpha} \leq \|f_m - f_n\|,\]
equivalently
\(|(f_m - f_n)(t) - (f_m - f_n)(a)| \leq |s-t|^\alpha \|f_m - f_n\|\)
hence \mset{f_n(t)} is a Cauchy sequence in \(\C\) as well.
Therefore we can define \(f(t) = \lim_{n\to\infty} f_n(t)\).
If by negation \(M_f = \infty\) then for any \(0<M<\infty\) we can find
\(s,t\in[a,b]\) such that \(s\neq t\) and \(|f(s)-f(t)|\geq |s-t|^\alpha M\).
But then we can find some $n$ such that \(f_n\) is sufficiently close 
to $f$ in \(\{s,t\}\) so \(|f_n(s)-f_n(t)|\geq |s-t|^\alpha (M-\epsilon)\).
But then \mset{M_{f_n}} is not bounded, which is a contradiction.


%%%%%%%%%%%%%% 12
\begin{excopy}
Let $K$ be triangle (two-dimensional figure) in the plane,
let $H$ be the set consisting of the vertices of $K$, of the form
\[ f(x,y) = \alpha x + \beta y + \gamma
   \qquad (\alpha, \beta,\, \textrm{and}\, \gamma\; \textrm{real}). \]
Show that to each \((x_0,y_0)\in K\) there corresponds
a unique measure \(\mu\) on $H$ such that
\[ f(x_0,y_0) = \int_H f\,d\mu. \]
(Compare Sec.~5.22.)

Replace $K$ by a square, let $H$ again be the set of its vertices, and let $A$
be as above.
Show that to each point of $K$ there still corresponds a measure on $H$,
with the above property, but that uniqueness is now lost.

Can you extrapolate to a more general theorem?
(Think of other figures, higher dimensional spaces.)
\end{excopy}

Denote the vertices $H$ of the triangle $K$ 
by \(v_i = (x_i,y_i\) for \(i=1,2,3\).
Since any triangle is convex, 
for any \((x_0,y_0)\in K\) there is a convex combination
\[(x_0,y_0) = \sum_{j=1}^3 w_j(x_j,y_j)\]
where the weights \(w_j\in[0,1]\) and \(\sum_{j=1}^3 w_j = 1\).
Put \(\mu(\{v_j\}) = w_j\). 
Now for any \(f(x,y) = \alpha x + \beta y + \gamma\) 
we compute:
\[
f(x_0,y_0)
 = \alpha x_0 + \beta y_0 + \gamma 
 = \sum_{j=1}^3  w_j (\alpha x_j + \beta y_j + \gamma) 
 = \sum_{j=1}^3 f(x_j,y_j) w_j 
 = \int_H^f\,d\mu.
\]

\paragraph{Uniqueness.} Let \(\mu_1\) and \(\mu_2\)
satsisfying the above condition for any $f$ of the above (affine) form.
\Wlogy, we may assume \(\mu_1(\{v_1\}) \neq \mu_2(\{v_1\})\).
We solve the following system of linear equations:
\begin{eqnarray*}
\alpha x_1 + \beta y_1 + \gamma &=& 1 \\
\alpha x_2 + \beta y_2 + \gamma &=& 0 \\
\alpha x_3 + \beta y_3 + \gamma &=& 0
\end{eqnarray*}
Where \(\alpha,\beta,\gamma\in\R\) are the unknowns.
The vertices are not co-linear, and there must exist a solution.
But now,
\[\int_H f\,d\mu_1 = \mu_1(\{v_1\} \neq \mu_2(\{v_1\}) = \int_H f\,d\mu_2\]
is a contradiction.

When $K$ is a squere we do the same, but now there are 
inifnite number of convex combinations of the verices.
Thus the representing measure is not unique in the interior of $K$.

As for higher dimension. In \(\R^n\) for \(n\geq 2\),
any  set $H$ of \(n-1\) points which do not lie in the same hyperplane
generate a compact convex hull $K$.
The condition on $H$ as equivalent of saying that they do not 
simultanously satisfy a single non trivial linear equation in \(\R^n\).
The ``extreme'' vertices of of $K$ are $H$.
Now, for any \(\mathbf{x}\in K\) there exist a unique measure 
\(\mu_{\mathbf{x}}\) on $H$ such that
\[f(\mathbf{x}) 
  = \int_H f\,d\mu_{\mathbf{x}} 
  = \sum_{v\in H} \mu_{\mathbf{x}}(v) \cdot f(v)\,.\]

%%%%%%%%%%%%%% 13
\begin{excopy}
Let \(\{f_n\}\) be a sequence of continuous complex functions on a (nonempty)
complete metric space $X$, such that \(f(x) = \lim f_n(x)\) exists
(as a complex number) for every \(x\in X\).
\begin{itemize}
 \itemch{a} Prove that there is an open set \(V\neq \emptyset\) and a number
            \(M<\infty\) such that \(|f_n(x)| < M\) for all \(x\in V\)
            and \(n=1,2,3,\ldots\).
 \itemch{b} If \(\epsilon>0\), prove that there is an open set \(V\neq\emptyset\)
            and an integer $N$ such that \(|f(x) - f_n(x)| \leq \epsilon\)
            if \(x\in V\) and \(n\geq N\).

\end{itemize}
\emph{Hint for \ich{b}}: For \(N=1,2,3,\ldots\), put
\[ A_N = \{x: |f_m(x)-f_n(x)|\leq \epsilon
\;\textrm{if}\; m\geq N\;\textrm{and}\; n\geq N\}.\]
Since \(X=\cup A_N\), some \(A_N\) has a nonempty interior.
\end{excopy}


\begin{itemize}
\itemch{a}
Let 
\[ B_M \eqdef  \{x\in X: \forall n\in\N,\, |f_n(x)|\leq M\} 
  = \bigcup_{n\in\N} \{x\in X: |f_n(x)|\leq M\}.\]
Clearly \(B_M\) are closed. For any \(x\in X\), 
the sequence \mset{f_n(x)} hence \mset{|f_n(x)|} is bounded.
Therefore, \(X=\cup_{M\in\N} B_M\) and
\index{Baire!category theorem}
by Baire's category theorem~5.6 since $X$ is complete, some \(B_M\) 
must have non empty interior $V$.

\itemch{b}
The sets 
\[ A_N = \{x: |f_m(x)-f_n(x)|\leq \epsilon\;\textrm{if}\; m,n\geq N\}\]
are closed. Similarly, \(X=\cup_{N\in\N} A_N\)
and so again by Baire's theorem, some \(A_N\) has non empty interior $V$.
\end{itemize}

%%%%%%%%%%%%%% 14
\begin{excopy}
Let $C$ be the space of all real continuous functions on \(I=[0,1]\)
with the supremum norm. Let $X$, be the subset of $C$ consisting of those
$f$ for which there exists a \(t\in I\) such that \(|f(s)-f(t)| \leq n|s-t|\)
for all \(s\in I\). Fix $n$ and prove that each open set in $C$ contains
an open set which does not intersect \(X_n\). (Each \(f\in C\) can be uniformly
approximated by a zigzag function $g$ with very large slopes and if
\(\|g-h\|\) is small, \(h\notin X_n\).)
Show that this implies the existence of a dense \(G_\delta\) in $C$ which
consists entirely of nowhere differentiable functions.
\end{excopy}


Fix $n$ and take a base open set in $C$ by picking \(f\in C\) and 
\(\epsilon>0\), and setting
\[V \eqdef \{\phi\in C: \|f-\phi\|_\infty < \epsilon/3\}.\]
In $I$ every continuous functions is uniformly continuous.
Hence there exists \(\delta>0\) such that \(|f(t)-f(s)|<\epsilon\)
whenever \(|s-t|<\delta\). 
Let 
\[s = \lceil 1\bigm/\,\min\bigl(\delta, \epsilon/(3(n+1))\bigr)\rceil\]
and split $I$ by \(x_i = i/(s+1)\) for \(0\leq i \leq s+1\).
Note that \(x_i - x_{i-1} < \delta/(n+1)\).
We will now construct \(g\in V\).
Intuitively, it will have on each subsegment \([x_{i-1},x_i]\) 
a slope of \(\pm(n+1)\), and it will increase or decrease so it follows $f$.
Formally, we define $g$ by induction on \([x_{i-1},x_i]\) for \(0< i \leq s+1\).
On the first subsegment \([x_0,x_1]\) let
\(g(x) = f(0) + (n+1)x\). Assume that $g$ is defined
on \([x_0,x_{i-1}]\), and by induction on \([x_{i-1}, x_i]\) by
\[g(x) = g(x_{i-1}) + \sigma (n+1)(x - x_{i-1})\]
where \(\sigma=1\) if \(g(x_{i-1}) < f(x{i-1})\) and 
\(\sigma= -1\) otherwise. It is easy to see that \(\|g-f\|_\infty < \epsilon/3\)
and that \(g\notin X_n\).
We now need a trivial lemma
\begin{llem}
Let \(f:[x_0,x_1]\to \R\) be defined by \(f(x) = ax+b\). 
For any \(\epsilon>0\)
such that \(\epsilon<|a|\) there exists \(\delta>0\) such that 
if \(h:[x_0,x_1]\to \R\) and \(\|f-h\|_\infty < \delta\)
then 
\begin{equation} \label{eq:ex5.14:lem}
 \max\left( \frac{h(t) - h(x_0)}{t-x_0}, 
              \frac{h(x_1) - h(t)}{x_1 - t} \right) < |a|-\epsilon. 
\end{equation}
for all \(t\in(x_0,x_1)\) we 
\end{llem}
\begin{thmproof}
\iffalse
By defining  \(\tilde{f}:[0,x_1-x_0]\to \R\)
as \(\tilde{f}(x) = |a|x\)
we see that we can assume that \(f(0)=0\) and \(a>0\).
Since we can similarly convert a $h$ 
and get the same ratios of \eqref{eq:ex5.14:lem}.
\fi
\Wlogy, we may assume \(a>0\), the negative case is analogous.
Let \(\delta = \epsilon(x_1-x_0)/4\), and assume \(\|f-h\|_\infty < \delta\).
For any \(t\in (x_0,x_1)\), there are two cases

\paragraph{High Case}: \((x_0+x_1)/2 \leq t < x_1\).
Let us estimate,
\begin{eqnarray*}
\frac{h(t) - h(x_0)}{t-x_0}
&=& \biggl( f(t) + (h(t)-f(t)) - \bigl(f(x_0) + (h(x_0)-f(x_0))\bigr)\biggr)
    \bigm /\,(t - x_0) \\
&\geq& \bigl(f(t) - f(x_0)\bigr) / (t - x_0) 
        - 2\delta / \bigl((x_1 - x_0)/2\bigr) \\
&=& a - 4\delta/(x_1 - x_0) \\
&=& a - \epsilon.
\end{eqnarray*}

\paragraph{Low Case}: \(x_0 < t \leq (x_0+x_1)/2\), similar derivation
as the high-case, but estimating the slope against \((x_0,h(x_0))\).
\end{thmproof}

From this lemma, back to this exercise, we see that we can pick 
some \(\delta>0\) such that for any \(h:I\to\R\) such that
\(\|g-h\|_\infty\), that is a neighborhood of $g$ the zigzag function, 
\(h\notin X_n\). 
Therefore, \[V_n \eqdef \inter{\left(C(I)\setminus X_n\right)}\] is dense.
By the corollary of theorem~5.6 in \cite{RudinRCA87}, 
\(\cap_{n\in\N} V_n\) is a dense \(G_\delta\) set, 
(and in particular non empty) that consists of nowhere differentiable functions.


%%%%%%%%%%%%%% 15
\begin{excopy}Let \(A=(a_{ij})\) be an infinite matrix with complex entries,
wher \(i,j=0,1,2,\ldots\).
$A$ associateswith each
sequence \(\{s_i\}\)
a sequence \(\{\sigma_i\}\), defined by
\[ \sigma_i = \sum_{j=0}^\infty a_{ij} s_j \qquad (i=1,2,3,\ldots), \]
provided that these series converge.

Prove that $A$ transforms every convergent sequence
\(\{s_i\}\)
to a sequence \(\{\sigma_i\}\) which converges to the same limit
if and only if the following conditions are satisfied:
\begin{alignat*}{2}
 \ich{a} & \qquad &\lim_{i\to\infty} a_{ij} &= 0\qquad \textrm{for each}\; j. \\
 \ich{b} & \qquad & \sup_i \sum_{j=0}^\infty |a_{ij}| &< \infty \\
 \ich{c} & \qquad & \lim_{i\to\infty} \sum_{j=0}^\infty a_{ij} &= 1.
\end{alignat*}
The process of passing from
\(\{s_i\}\) to \(\{\sigma_i\}\) is called
\index{summability method}
a~\emph{summability method}. Two examples are
\begin{eqnarray*}
a_{ij} &=&
   \left\{\begin{array}{ll}
          \frac{1}{i+1} & \quad \textrm{if}\; 0\leq j \leq i. \\
          0             & \quad \textrm{if}\; i < j,
          \end{array}\right. \\
\textrm{and}\qquad a_{ij} &=& (1-r_i)r_i^j, \qquad  0<r_i<1,\quad r_i\to 1.
\end{eqnarray*}
Prove that each of these also transforms some divergent sequence \(\{s_i\}\)
(even some unbounded ones) to a convergent sequences \(\{\sigma_i\}\).
\end{excopy}

Following the above notations, we denote the transform as \(\sigma = A(s)\).

\paragraph{Summability implies limits.}
Assume $A$ transforms convergent sequences to convergent sequences.

Assume by negation \ich{a} does not hold for some $j$.
Pick the sequences \mset{s_j}, such that \(s_{i} = \delta_{ij}\).
Clearly \(\lim_{i\to\infty}s_i = 0\), 
but \(\sigma_i = a_{ij}\) which does not converge to $0$, 
and so by contradiction, \ich{a} holds.

We will now show \ich{b}. We first show that 
\begin{equation} \label{eq:ex5.15:finsum}
\sum_{j=0}^\infty |a_{rj}| < \infty.
\end{equation}
for each $r$. By negation, assume \eqref{eq:ex5.15:finsum} does not hold 
for some fixed $r$.
We will construct, in appending steps, 
a~sequence \mset{s_j} such that converges to zero but
\((A(s))_r = \infty\).
We define a sequence of blocks of indices such that 
each block if $A$'s $r$th row is ``not too small''.
Formally, let \(M_0 = 0\) and let \(M_k<\infty\) be the minimal integer
such that \[\sum_{j=M_{k-1}}^{M_k} |a_{rj}| > 1.\]
Clearly \mset{M_k} is an infinitely increasing sequnce.
For \(k\in\N\) let 
\[s_j = e^{i\theta_j}/k \qquad 
 \textrm{where}\;
 M_{k-1} \leq j < M_k
 \;\textrm{and}\;
 \theta_j = -\Arg(a_{rj}).\]
Note that \(\lim_{j\to\infty}s_j = 0\) and \(a_{rj}s_j \geq 0\).
Compute the $r$ component of \(A(s)\):
\begin{eqnarray*}
\sigma_r 
&=& \sum_{j=0}^\infty a_{rj}s_j 
 = \sum_{k=1}^\infty \sum_{j=M_{k-1}}^{M_k} a_{rj}s_j \\
&=& \sum_{k=1}^\infty 
      \left(\sum_{j=M_{k-1}}^{M_k} a_{rj}e^{i\theta_j}\right) \bigm/\,k 
 = \sum_{k=1}^\infty \left(\sum_{j=M_{k-1}}^{M_k} |a_{rj}|\right) \bigm/\,k \\
&\geq& \sum_{k=1}^\infty 1/k \\
&=& \infty.
\end{eqnarray*}
Thus \(\sigma=A(s)\) is not a valid sequence, (limit cannot be defined at all
and thus \eqref{eq:ex5.15:finsum} is true, and we can denote
\[S_r \eqdef \sum{j=0}^\infty |a_{rj}.\]


Now assume by negation \ich{b} does not hold.
We will again construct, in appending steps, 
a~sequence \mset{s_j} that will provide a contradiction.
We will also build increasing sequences, 
\mset{b_k} of column blocks, and \mset{r_k} of rows, such that
\begin{eqnarray}
\sum_{j=0}^{b_k-1} |a_{r_k j}| &\leq& 1 \label{eq:ex5.15:head} \\
\sum_{j=b_k}^{b_{k+1}-1} |a_{r_k j}| &\geq& 2^k \label{eq:ex5.15:mid} \\
\sum_{j=b_{k+1}}^\infty |a_{r_k j}| &\leq& 1 \label{eq:ex5.15:tail}
\end{eqnarray}
for each \(k\in\N\).
Let \(b_0=0\). 
Assume \mset{s_j} is defined for all \(j<b_{k'}\)
By \ich{a}, there exists some \(\rho\) such that 
\(|a_{rj}| < 1/(b_{k'}+1)\)
for all \(r\geq \rho\) and all \(j<b_{k'}\).
To ensure our next row pick, let \(U_k = \max_{r\leq\rho}S_r\).
By our negation hypothesis, there exists \(r_k\) such that 
\[S_{r_k} \geq \max(k^2+2,U_k).\] 
Clearly \(r_k>\rho\), and we can find \(b_{k+1}\) such that 
\[\sum_{j=b_{k+1}}^\infty |a_{rj}| < 1.\]
Now define \(s_j\) for \(b_k \leq j < b_{k+1}\) by
\[s_j = e^{i\theta_j}/k \qquad  \textrm{where}\; \theta_j = -\Arg(a_{r_k j}).\]
By induction we complete the definitions of \mset{s_j} and 
the supporting sequences \mset{b_k} and \mset{r_k}.
Clearly \(\lim_{j\to\infty}s_j = 0\), but
\begin{eqnarray*}
|\lim_{r\to\infty}\sigma_r|
&=& \lim_{r\to\infty} |\sigma_r| 
= \lim_{r\to\infty} \left| \sum_{j=0}^\infty a_{rj}s_j \right| 
= \lim_{k\to\infty} \left| \sum_{j=0}^\infty a_{r_k j}s_j \right| \\
&\geq& \lim_{k\to\infty} 
       \left(
          \left| \sum_{j=b_k}^{b_{k+1}-1} a_{r_k j}s_j \right| 
         - \left| \sum_{j=0}^{b_k-1} a_{r_k j}s_j \right| 
         - \left| \sum_{j=b_{k+1}}^\infty a_{r_k j}s_j \right|
       \right) \\
&=& \lim_{k\to\infty} 
       \left(
          \left| \sum_{j=b_k}^{b_{k+1}-1} |a_{r_k j}| \right| / k
         - \left| \sum_{j=0}^{b_k-1} a_{r_k j}s_j \right| 
         - \left| \sum_{j=b_{k+1}}^\infty a_{r_k j}s_j \right|
       \right) \\
&\geq& \lim_{k\to\infty} 
       \left(
          (k^2+2k) / k
         - \sum_{j=0}^{b_k-1} |a_{r_k j}|
         - \sum_{j=b_{k+1}}^\infty |a_{r_k j}|
       \right) \\
&\geq& \lim_{k\to\infty} k+2 - 1 - 1 \\
&=& \infty.
\end{eqnarray*}
This contradicts the summability, hence \ich{b} is true.

Consider the constant sequences \(s_j=1\) for all \(j\in\N\).
It has \(\lim_{j\to\infty} s_j = 1\) which implies
 \(\lim_{j\to\infty} \sigma_j = 1\) which is actually equivalent
to \ich{c}.

\paragraph{Limits implies summability.}
Conversely, assume conditions \ich{a}, \ich{b}, \ich{c} hold.

\emph{Constant:} If \(s_j=c\) for all \(j\in\N\), 
then by \ich{c} we get \(\lim_{j\to\infty}\sigma_j = c\).
\emph{Vanishing:} Assume \(\lim_{j\to\infty s_j} s_j = 0\).
Pick arbitrary \(\epsilon>0\). 
Let $J$ be such that \(|s_j| < \epsilon/3\) whenever \(j\geq J\).
Put \(h = \max_{1\leq j \leq J} |s_j| + 1\).
By \ich{a} there exists \(\rho_1\) such that \(|a_rj| < \epsilon/(Jh+1)\)
for all \(r\geq\rho_1\) and \(0\leq j \leq J\).
Denote \(T_r = \sum{j=0}^\infty a_{rj}\).
By \ich{c}, pick \(\rho_2\) such that \(|T_r - 1| < \epsilon\)
whenever \(r\geq \rho_2\). 
Note that in this case, 
\[
\left|\left(T_r - \sum_{j=0}^J a_{rj}\right) - 1\right| 
\leq |T_r - 1| + J\epsilon/(Jh+1)
< 2\epsilon.\]
Let \(\rho = \max(\rho_1,\rho_2)\), now for \(r\geq \rho\)
\begin{eqnarray*}
|\sigma_r|
&=& \left|\sum_{j=0}^\infty a_{rj}s_j\right| 
 = \left|\sum_{j=0}^J a_{rj}s_j
        + \sum_{j=J+1}^\infty a_{rj}s_j\right| \\
&\leq&  \left|\sum_{j=0}^J a_{rj}s_j\right| 
      + \left|\sum_{j=J+1}^\infty a_{rj}s_j\right| \\
&\leq&  h\left|\sum_{j=0}^J a_{rj}\right| 
      + \epsilon\left|\sum_{j=J+1}^\infty a_{rj}\right| \\
&\leq&  hJ\epsilon/(Jh+1) + \epsilon(1+2\epsilon) \\
&<& 2\epsilon(\epsilon+1).
\end{eqnarray*}
Hence \(\lim_{r\to\infty} \sigma_r = 0\).

Now for arbitrary converging sequence, let \(\lambda = \lim_{j\to\infty} s_j\).
By the above two case, we have
\begin{eqnarray*}
\lim_{r\to\infty} \sigma_r 
&=& \lim_{r\to\infty} \sum_{j=0^\infty} a_{rj}s_j \\
&=& \lim_{r\to\infty} \sum_{j=0^\infty} a_{rj}(s_j - \lambda + \lambda) \\
&=& \left(\lim_{r\to\infty} \sum_{j=0^\infty} a_{rj}(s_j - \lambda)\right) + 
    \left(\lim_{r\to\infty} \sum_{j=0^\infty} a_{rj} \lambda\right) \\
&=& 0 + \lambda.
\end{eqnarray*}
Thus $A$ satisfies the summability condition.

\paragraph{Specific transformations.}
When 
\begin{eqnarray*}
a_{rj} &=&
   \left\{\begin{array}{ll}
          \frac{1}{r+1} & \quad \textrm{if}\; 0\leq j \leq r. \\
          0             & \quad \textrm{if}\; r < j,
          \end{array}\right.
\end{eqnarray*}
we observe the sequence \mset{s_j} defined by 
% \(s_j = (-1)^j\), that is \((+1,-1,+1,-1,\ldots)\).
% that is \((+1,-1,+1,-1,\ldots)\).
\(s_j = (-1)^j\sqrt{\lfloor j/2\rfloor}\), 
that is \((0,0,+1,-1,+\sqrt{2},-\sqrt{2},\ldots)\).
Clearly does not converge and is unbounded, but since
we observe that 
\[\sum_{j=0}^r s_j 
  = \left\{\begin{array}{ll}
          0 & \qquad r = 1 \bmod 2 \\
          \sqrt{r/2} & \qquad r = 0 \bmod 2 
          \end{array}\right.\]
we see that 
\begin{eqnarray*}
\sigma_r 
&=& \sum_{j=0}^r a_{rj}s_j
 = \left(\sum_{j=0}^r s_j\right)/(r+1) \\
&=& \left\{\begin{array}{ll}
          0 & \qquad r = 1 \bmod 2 \\
          \sqrt{r}/(r+1) & \qquad r = 0 \bmod 2 
          \end{array}\right.
\end{eqnarray*}
Hence \(\lim_{r\to\infty}\sigma_r = 0\).

When \(a_{kj} = (1-r_k)r_k^j\) 
where \(0<r_k<1\) and \(\lim_{k\to\infty} r_k\to 1\),
we pick
\(s_j = (-1)^j\), that is \((+1,-1,+1,-1,\ldots)\).
that is \((+1,-1,+1,-1,\ldots)\).
Clearly does not converge. Compute
\[
\sigma_k
= \sum_{j=0}^r a_{kj}s_j
 =  \sum_{j=0}^r (1-r_k)r_k^j \cdot (-1)^j
 = (1-r_k)\sum_{j=0}^r (-r_k)^j 
 = (1-r_k) \cdot \bigl(1/(1-r_k)\bigr) = 1.\]
In particular,  \(\lim_{k\to\infty}\sigma_k=1\).



%%%%%%%%%%%%%% 16
\begin{excopy}
Suppose $X$ and $Y$ are Banach spaces, and suppose \(\Lambda\)
is a linear mapping of $X$ into $Y$, with the following property:
For every sequence \mset{x_n} in in $X$ for which \(x = \lim x_n\),
and \(y = \lim \Lambda x_n\) exist,
it is true that \(y=\Lambda x\). Prove that \(\Lambda\) is continuous.

This is so called ``closed graph theorem''
\emph{Hint}: Let \(X \oplus Y\) be the set of all ordered pairs
\((x,y)\), \(x\in X\) and \(y\in Y\), with addition and scalar multiplication
defined componentwise.
Prove that \(X\oplus Y\) is a Banach space,
if \(\|x,y)\| = \|x\| + \|y\|\).
The graph $G$ of \(\Lambda\) is a subset of \(X\oplus Y\)
formed by the pairs \((x,\Lambda x)\), \(x\in X\). Note that our hypothesis
says that $G$ is closed; hence $G$ is a banach space.
Note that \((x,\Lambda x) \to x\) is continuous, one-to-one,
and linear and maps $G$ onto $X$.

Observe that there exist \emph{nonlinear} mappings
(of \(\R^1\) onto \(\R^1\), for instance)
whose graph is closed although they are
not continuous: \(f(x)=1/x\) if \(x=0\), \(f(0)=0\).
\end{excopy}

Define projections: 
\begin{alignat*}{2}
p_1 & : G \to X & \qquad p_1(x,\Lambda x) &= x \\
p_2 & : (X,Y) \to X & \qquad p_2(x,y) &= y
\end{alignat*}
These projections are linear and continuous, and \(p_1\) is also one-to-one.
By the open mapping theorem~5.9 (\cite{RudinRCA87}), \(p_1^{-1}\) 
is continuous. But \(\Lambda = p_2 \circ p_1^{-1}\) and thus is continuous.

%%%%%%%%%%%%%% 17
\begin{excopy}
If \(\mu\) is a positive measure, each \(f\in L^\infty(\mu)\) defines
a  multiplication operator \(M_f\)
on \(L^2(\mu)\) into \(L^2(\mu)\) such that \(M_f(g) = fg\).
Prove that \(\M_f\|\leq \|f\|_\infty\).
For which measures \(\mu\) is it true that \(\|M_f\| = \|f\|_\infty\)
for all \(f\in L^\infty(\mu)\)?
For which measures \(f\in L^\infty(\mu)\) does \(M_f\) map
\(L^2(\mu)\) onto \(L^2(\mu)\)?
\end{excopy}

Let \(f\in L^\infty(\mu)\) and \(g \in L^2(\mu)\).
Now
\[ \|M_f(g)\|_2 
   = \left(\int |fg|^2\,d\mu\right)^{1/2}
   \leq \left(\int (\|f\|_\infty |g|)^2\,d\mu\right)^{1/2}
   \leq \|f\|_\infty \left(\int |g|^2\,d\mu\right)^{1/2}.
\]
Hence \(\|M_f\| \leq \|f\|_\infty\).

Now pick some \(f\in L^\infty(X,\mu)\) and an arbitrary \(\epsilon>0\).
By definition, the set
\[ U \eqdef \{x\in X: |f(x)|>\|f\|_\infty - \epsilon\}\]
satisfies \(\mu(U)>0\). Assume that for any such set, 
there exists \(W\subset U\) such that \(\mu(W)<\infty\).
Now we consider \(g=\chi_W\).
Clearly \(\|g\|_2^2 = \mu(W)\). Also
 we see that 
\[
\|M_f(g)\|_2^2
= \int |fg|^2\,d\mu
= \int_W f^2\,d\mu
\geq (\|f\|_\infty - \epsilon)^2 \mu(W).\]
Hence \(\|M_f\| \geq \|f\|_\infty - \epsilon\)
and the equality \(\|M_f\| = \|f\|_\infty\) is established.

Finally, if \(1/f \in L^\infty(\mu)\), then 
\(M_f\) is invertible and \((M_f)^{-1} = M_{1/f}\).
In particular, in this case, \(M_f\) is onto.

%%%%%%%%%%%%%% 18
\begin{excopy}
Suppose \mset{\Lambda_n} is a sequence of bounded linear transformations
from a normed linear space $X$ to a Banach space $Y$,
suppose \(\|\Lambda_n\| \leq M <\infty\) for all $n$, and suppose
there is a dense set \(E\subset X\) such that
\mset{\Lambda_n x} converges for each \(x\in E\).
Prove that \mset{\Lambda_n x} converges for each \(x\in X\).
\end{excopy}

We will show that \mset{\Lambda_n x} is a Cauchy sequence.
Pick some arbitrary \(\epsilon>0\). Pick some \(x'\in E\) such that
\(\|x-x'\|<\epsilon/(3M)\).
Since \mset{\Lambda_n x'} is a Cauchy sequence, there exists
some $N$ such that \(\|\Lambda_m x' - \Lambda_n x'\| < \epsilon/3\)
whenever \(m,n>N\). But also
\begin{eqnarray*}
\|\Lambda_m x - \Lambda_n x\|
&\leq&
 \|\Lambda_m x - \Lambda_m x'\|
 + \|\Lambda_m x' - \Lambda_n x'\|
 + \|\Lambda_n x' - \Lambda_n x\| \\
&\leq& (\|\Lambda_m\| + \|\Lambda_n\|)\cdot\|x-x'\| 
      + \|\Lambda_m x' - \Lambda_n x'\| \\
&\leq& 2M\cdot\epsilon/(3M) + \epsilon/3 = \epsilon.
\end{eqnarray*}
Thus, \mset{\Lambda_n x} is a Cauchy sequence, 
since $Y$ is a Banach space this sequence converges.


%%%%%%%%%%%%%% 19
\begin{excopy}
If \(s_n\) is the $n$th partial sum of the Fourier series of a function
\(f\in C(T)\), prove that \(s_n/\log n \to 0\)
uniformly, as \(n\to \infty\), for each \(f\in C(T)\). That is prove that
\[ \lim_{x\to\infty} \frac{\|s\|_\infty}{\log n} = 0. \]

On the other hand, if \(\lambda_n/\log n \to 0\) prove that there exists an
\(f\in C(T)\) such that the sequence \(\{s_n(f;0)/\lambda_n\}\) is unbounded.
\emph{Hint}: Apply the reasoning of Exercise~18 and that of Sec.~5.11,
with a better estimate of \(\|D_n\|_1\), than used there.
\end{excopy}

We first prove the following Lemmas
(See also Theorems~I-8-1 and II-8-13 in \cite{Zyg:2002}).

\begin{llem} \label{llem:fog:ifoig}
Let $f$,\ and $g$ be integrable functions on each 
subinterval \([a,b']\) such that \(a\leq b' < b\).
Let
\begin{equation*}
F(x) = \int_a^x f(t)\,dt
\qquad
G(x) = \int_a^x g(t)\,dt.
\end{equation*}
Assume that \(g(x)\geq 0\) and that \(\lim_{x\to b} G(x) = \infty\).
If 
\begin{equation*}
\lim_{x\to b} f(x)/g(x) = 0
\end{equation*}
then
\begin{equation*}
\lim_{x\to b} F(x)/G(x) = 0
\end{equation*}
\end{llem}

In $o$-notation: the theorem says:
If \(f(x) = o(g(x))\)
then \(F(x) = o(G(x))\).

\begin{thmproof}
Pick arbitrary \(\epsilon>0\).
Let \(x_0\in[a,b)\) such that 
\(|f(x)/g(x)| < \epsilon/2\) for \(x_0 <x < b\).
Pick \(x_1 \in (x_0,b)\) such that for all \(x\geq x_1\) we have
\begin{equation*}
G(x) > 2\int_a^{X_0} |f(t)|\,dt \bigm/ \epsilon.
\end{equation*}

Thus for \(x\geq x_1\)
\begin{equation*}
|F(x)|
\leq 
   \int_a^{x_0} f(t)\,dt 
 + \int_{x_0}^x f(t)\,dt 
\leq 
   \int_a^{x_0} f(t)\,dt 
 + \epsilon G(x)/2
\leq \epsilon G(x).
\end{equation*}
\end{thmproof}

Clearly similar result and proof holds with 
reversed directions of $a$ and $b$.

The following lemma shows that for continuous functions 
\(\|s_n(f)\|_\infty = o(\log n)\).
\begin{llem} \label{llem:ex:5.19a}
Let \(g\in C(T)\) and \(s_n\) be the $n$-Fourier sum of $g$.
Then 
\begin{equation} \label{eq:ex:5.19}
\lim_{n\to\infty} s_n(f,x) / \log n = 0.
\end{equation}
\end{llem}
\begin{thmproof}
We freely identify $T$ with \(\{x: -\pi \leq x < \pi\}\).
Let \(D_n\) be the Dirichlet's kernel, 
that is \[s_n(f,t) = (D_n * f)(t).\]

Pick arbitrary \(\epsilon>0\).
Since $f$ is uniformly continuous (on $T$), 
there exists \(\delta > 0\)
such that 
\begin{equation} \label{eq:ex5:19:fcont}
|f(x-t) - f(x+t)| < \epsilon
\end{equation}
for all \(t < \delta\) and \(x\in T\).

If \(|t|\leq \pi/2\) then \(2t/\pi \leq \sin t\)
and also
\begin{equation} \label{eq:ex5:19:Dnltt}
|D_n(t)| 
= \left|\frac{\sin\bigl((n+1/2)t\bigr)}{\sin(t/2)}\right|
\leq 1 \bigm/ (\pi t).
\end{equation}

Combining \eqref{eq:ex5:19:fcont} and \eqref{eq:ex5:19:Dnltt}
we get that
\begin{equation*}
\lim_{t\to 0} \bigl(f(x-t) - f(x+t) \bigr) D_n(t)  \bigm/\, (1/t) = 0
\end{equation*}
uniformly for all \(x\in T\).
Using 
\[
\int_{\pi/n}^{\pi} \frac{1}{t}\,dt = \log(\pi) - \log(\pi/n) = \log n.
\]
with local lemma~\ref{llem:fog:ifoig}, we have:
\begin{equation} \label{eq:ex5:19:fDn:olog}
\lim_{n\to \infty} 
 \int_{\pi/n}^{\pi} \bigl(f(x-t) - f(x+t) \bigr) D_n(t)\,dt
   \bigm/\, \log n = 0.
\end{equation}
again, uniformly for all \(x\in T\).

We now estimate \(|s_n(f,x)|\) by integrating two domains.
\begin{eqnarray*}
2\pi |s_n(f,x)|
&=& \left| \int_{-\pi}^{\pi} f(x-t)D_n(t)\,dt \right| \\
&=& 
     \left|\int_0^{\pi/n} \bigl(f(x-t) - f(x+t)\bigr)D_n(t)\,dt\right| 
   + 
     \left|\int_{\pi/n}^{\pi} \bigl(f(x-t) - f(x+t)\bigr)D_n(t)\,dt\right| \\
&\leq&
     (\pi/n) \|f\|_\infty (2n+1)
   + \left|\int_{\pi/n}^{\pi} \bigl(f(x-t) - f(x+t)\bigr)D_n(t)\,dt\right| 
\end{eqnarray*}

Using the fact that \(\lim_{n\to\infty} (2n+1) / (n\log n) = 0\)
and \eqref{eq:ex5:19:fDn:olog} we get the desired \eqref{eq:ex:5.19}.
\end{thmproof}

The following may be viewed as a converse to local lemma~\ref{llem:ex:5.19a}.
It shows that \(\log n\) is the best estimate order for \(\|s_n\|\).
\begin{llem} 
Let \(g\in C(T)\) and \(s_n\) be the $n$-Fourier sum of $g$.
Then 
If \((\lambda_n)_{n\in\N}\) is a sequence of positive numbers
such that \(\lim_{n\to\infty}\lambda_n/\log n = 0\),
then there exists \(f\in C(T)\) such that
\(\{s_n(f;0)/\lambda_n\}_{n\in\N}\) is unbounded.
\end{llem}

\begin{thmproof}
% we may trivially ignore occurrences of \(\lambda_n = 0\).
By Secation 5.11 (\cite{RudinRCA87}) if \(\Lambda_n f = s_n(f;0)\)
then
\begin{equation*}
\| \Lambda_n\| = \|D_n\|_1 
 > \frac{4}{\pi}\sum_{k=1}^\infty \frac{1}{k} = c \log n
\end{equation*}
for some constant \(c>0\).
Hence
\begin{equation*}
\|\Lambda_n\|_\infty /\lambda_n > c \log n / \lambda_n
\end{equation*}
and so 
\begin{equation*}
\lim_{n\to\infty} \|\Lambda_n\|_\infty /\lambda_n 
\geq \lim_{n\to\infty} c \log n / \lambda_n = \infty.
\end{equation*}
and by the uniform bounded principle (theorem~5.8 \cite{RudinRCA87})
there must exists \(f\in C(T)\) such that the sequence
\(\{s_n(f;0)/\lambda_n\}_{n\in\N}\) is unbounded.
\end{thmproof}


\begin{excopy}
{\small [Appears in and refers to second edition].}\newline
Is the lemma of Sec.~4.15 valid on every Banach space?
In every normed linear space?
\end{excopy}

The lemma says:
\begin{quote}
\textsl{
If $V$ is a closed subspace of a Hilbert space $H$,
\(y\in H\),
\(y\notin V\),
and \(V^{*}\) is the space spanned by $V$ and $y$, then 
\(V^{*}\) is closed.
}
\end{quote}
% \newline

The proof is it is in the (old edition) text applies for Banach spaces
as it is. Now we will prove thje following lemma using 
Hahn Banach
\index{Hahn Banach}
theorem~5. \cite{}.

\begin{llem}
If $V$ is a closed subspace of a normed linear space $L$,
\(y\in L\)
and \(V^{*}\) is the space spanned by $V$ and $y$, then 
\(V^{*}\) is closed.
\end{llem}

\begin{thmproof}
We may assume \(y\notin L\) since otherwise the result is trivial.
Now let's define a functional on \(V^{*}\) by
\[ f(v + \lambda y) = \lambda \qquad v\in V,\; \lambda \in\C.\]
By Hahn-Banach theorem we can extend $f$ to all $N$.
Assume \(w\in \overline{V^*}\).
By having a norm, there exists a sequence \((v_n + \lambda_n y)_{n\in\N}\)
such that 
\[\lim_{n\to\infty} v_n + \lambda_n y = w.\]
But then 
\[f(w) = \lim_{n\to\infty} f(v_n + \lambda_n y) 
       = \lim_{n\to\infty} \lambda_n.\] 
Thus \(w = f(y)y + \lim_{n\to\infty} v_n\) and 
% so \(w - f(y)y = \lim_{n\to\infty} v_n\).
since $V$ is closed,
\(w - f(y)y\in V\) and so \(w\in V^*\) and we have shown that \(V^*\) is 
closed.
\end{thmproof}




\end{enumerate}

\nobreak
\begin{enumerate}

\setcounter{enumi}{19}

%%%%%%%%%%%%%% 20
\begin{excopy}
\begin{itemize}

\itemch{a}
Does there exist a sequence of continuous positive functions \(f_n\)
on \(\R^1\) such that \mset{f_n(x)} is unbounded if and only if $x$ is
rational?

\itemch{b}
Replace ``rational'' by irrational in \ich{a} and answer the resulting
question.

\itemch{c}
Replace ``\mset{f_n(x)} is unbounded''
by ``\(f_n(x)\to \infty\) as \(n\to\infty\)''
and answer the resulting analogues of \ich{a} and \ich{b}.
\end{itemize}
\end{excopy}

It is easy to see that for all cases, the existence 
is the same if positive functions are defined on \([0,1]\).
Simply by restricting functions, or uniting them with 
appropriate shifting to maintain continuity.

\begin{itemize}

%%%%%%%%%%
\itemch{a}

Define 
\[G_{m,n} \eqdef \{x\in\R: f_n(x)>m\}.\]

By definition \(\lim_{n\to\infty} f_n(x) = \infty\) iff
\[\forall M\exists N \forall n\geq N\;f_n(x)>M.\] 
Hence the set consists exactly of such points $x$ is
\[L \eqdef \bigcap_M \bigcup_N \bigcap_{n>N} G_{M,n}.\]

By definition \(\limsup_{n\to\infty} f_n(x) = \infty\) iff
\[\forall M\forall N \exists n\geq N\;f_n(x)>M.\]
Hence the set consists exactly of such points $x$ is
\[U \eqdef \bigcap_M \bigcap_N \bigcup_{n>N} G_{M,n}.\]

Thus the subset of \(\R^1\) for which  \mset{f_n(x)} is unbounded
is a \(G_\delta\) set. It cannot be \Q\ since otherwise,
by local lemmas~\ref{lem:count:1cat} and~\ref{lem:gdel:2cat},
\Q\ would be of second category.

%%%%%%%%%%
\itemch{b}

Following \cite{Myerson:1991:FCF}, we will show such sequence.
Interestingly, there it cites \cite{Gelb1996}, Chapter~7 Example~4.

Let \(\{q_j\}_{j\in\N}\) be an enumeration
of all the rationals in \((0,1)\).
Define \(f_n\) as a periodic function of period $1$,
thus we need to define it on \([0,1]\).
Let \(P_n = \{0,1\} \cup \{q_j: j\leq n\}\). 
We firrst define \(f_n\) on \(P_n\).
as follows in \([0,1]\) as follows.
\begin{align*}
f_n(0) = f_n(1) &= 0 \\
\forall j\leq n\quad f_n(q_j) &= j.
\end{align*}

The set \(P_n\) partitions \([0,1]\)
into \(n+1\) sub-segmnets, where we define \(f_n\) to be linear.
It is easy to see that \(f_n\) are continuous, monotonically increasing
and that for all \(x\in\R\)
\begin{equation*}
\lim_{n\to\infty} f_n(x) = 
\left\{\begin{array}{ll}
       j       & x - \lfloor x \rfloor = q_j \\
       \infty  &   x \in \R\setminus \Q
       \end{array}\right.
\end{equation*}



%%%%%%%%%%
\itemch{c}

The analogue for \ich{a} is \emph{false}.
Assume by negation there exists a sequence \(\{f_n\}_{n\in\N}\)
of continuous functions such that 
\begin{equation*}
L \eqdef \{x\in\R: \lim_{n\to\infty} f_n(x) = \infty\} = \Q.
\end{equation*}
Let \(\{q_n\}_{n\in\N}\) be an enumeration of \Q.
We will define a decreasing sequence of closed intervals
\(\{I_n\}_{n\in\N}\) such that 
for all \(n\in\N\)
the following hold:
\begin{align}
m(I_n) &> 0  \label{eq:ex:5.20:c0} \\
I_n &\supset I_{n+1} \label{eq:ex:5.20:c1} \\
q_n &\notin I_n \label{eq:ex:5.20:c2} \\
\forall x \in I_n,\; f_n(x) &\geq n \label{eq:ex:5.20:c3}
\end{align}
Since \(\lim_{n\to\infty} f_n(1/2)=\infty\), 
there exists some \(N_1\) such that 
\(f_n(1/2) > 2\) for all \(n\geq N_1\).
By taking a sufficiently small neighborhood \(V_1\) of \(1/2\)
and picking a closed sub-interval \(I_1 \subset V_1 \setminus \{q_1\}\).
We can ensure that 
\eqref{eq:ex:5.20:c0},
\eqref{eq:ex:5.20:c2} and
\eqref{eq:ex:5.20:c3} hold.

By induction assume that 
\(\{I_j\}_{j=1}^k\) were picked and that
the above 
\eqref{eq:ex:5.20:c0},
\eqref{eq:ex:5.20:c1},
\eqref{eq:ex:5.20:c2} and
\eqref{eq:ex:5.20:c3} hold for \(n< k\).
Since \(m(I_{k-1})>0\) we can
pick some rational \(\alpha \in I_{k-1} \cap \Q\).
Since 
Since \(\lim_{n\to\infty} f_n(\alpha)=\infty\), 
we can find some \(N_k\) such that 
\(f_n(\alpha) > k+1\) for all \(n\geq N_k\).
By taking a sufficiently small neighborhood \(V_k\subset I_{k-1}\) 
of \(\alpha\)
and picking a closed sub-interval \(I_k\subset V_k \setminus \{q_k\}\)
We can ensure that the above 
\eqref{eq:ex:5.20:c0},
\eqref{eq:ex:5.20:c1},
\eqref{eq:ex:5.20:c2} and
\eqref{eq:ex:5.20:c3} hold for \(n=k\) as well.

Since \(\cup_{n\in\N} I_n \neq \emptyset\) 
There exists \(c \in \cup_{n\in\N}\)
and \(c\notin \Q\) by \eqref{eq:ex:5.20:c2}.
Clearly \(\lim_{n\to\infty} f_n(c) = \infty\)
Which gives the contradiction \(c\in L=\Q\).


The analogue for \ich{b} is \emph{true}, since the example shown
for \ich{b} holds for here as well, since
that sequence converges everywhere, either for a real number
or infinity.

\end{itemize}

%%%%%%%%%%%%%% 21
\begin{excopy}
Suppose \(E\subset \R^1\) is measurable, and \(m(E)=0\).
Must there be a translate \(E+x\) of $E$ that does not intersect $E$?
Must there be a homeomorphism $H$ of \(\R^1\) onto \(\R^1\) so that
 \(h(E)\) does not intersect $E$?
\end{excopy}

The answer is no. It is sufficiently to defy the second conjecture.

We use exercise~2 in chapter~2 of \cite{RudinFA79} 
(althogh later in the order of Rudin's text books, but the exercise
does not require further knowledge than we already have).
Let \(I \eqdef [0,1] = E_0\disjunion F_0\) 
be a disjoint union of the unit segment
such that \(m(E_0)=0\) and \(F_0\) is of first category.

Using the translation notation \(E+a = \{x+a: x\in E\}\), define
\[
E = \bigcup_{n\in\Z} E_0 + n \qquad
F = \bigcup_{n\in\Z} F_0 + n \qquad.
\]
Clearly \(\R = E \disjunion F\) where 
such that \(m(E)=0\) and $F$ is of first category.

Assume by negation there is homeomorphism \(T:\R\to\R\) 
such that \(E\cap T(E) = \emptyset\).
But then \(I = F \cup T(F)\) contradiction to the fact that
the unit segment is of second category.


%%%%%%%%%%%%%% 22
\begin{excopy}
Suppose \(f\in C(T)\) and
\index{Lip@\(\Lip\)}
\(f\in \Lip\alpha\) for some \(\alpha > 0\). (See  Exercise~11.)
Prove that the Fourier series of $f$ converges to \(f(x)\),
by completing the following outline:
It is enough to consider the case \(x=0\),
\(f(0)=0\). The difference between the partial sums \(s_n(f;0)\)
and the integrals
\[ \frac{1}{\pi} \int_{-\pi}^\pi f(t)\frac{\sin nt}{t}\,dt \]
tends to $0$ as \(n\to \infty\).
The functions \(f(t)/t\) is in \(L^1(T))\).
\index{Riemann-Lebesgue lemma}
Apply the Riemann-Lebesgue lemma. More careful reasoning shows that the
convergence is actually uniform on $T$.
\end{excopy}

Define \(f_\tau(t) = f(t + \tau)\).
Now 
\begin{align*}
2\pi s_n(f;x)
&= \int_{\pi}^\pi f(t)D_n(x - t)\,dt
 = \int_{\pi}^\pi f(t+x)D_n(x - t + x)\,dt
 = \int_{\pi}^\pi f_x(t)D_n(-t)\,dt \\
& = 2\pi s_n(f_x;0).
\end{align*}
Clearly \(f\in\Lip_\alpha\) iff  \(f_x\in\Lip_\alpha\).
Thus it is sufficient to to show that \(\lim_{n\to\infty} s_n(f;0) = f(0)\).
By looking at \(g(t) = f(t) - f(0)\) we see that 
\(f\in\Lip_\alpha\) iff  \(g\in\Lip_\alpha\)
and \(s_n(f)\) differs from \(s_n(g)\) by the constant \(\hat{f}(0)\).
Thus we may also assume \(f(0)=0\).

Let \(E_n(t) = 2\sin(t)/t\)
and \(H_n(t) = D_n(t) - E_n(t)\).
Applying L'Hospital rules we have
\begin{align*}
\lim_{t\to 0} D_n(t) &= \lim_{t\to 0} \sin((n+1/2)t)/\sin(t/2) = 2n+1\\
\lim_{t\to 0} E_n(t) &= \lim_{t\to 0} 2\sin(nt)/t = 2n\\
\lim_{t\to 0} H_n(t) &= 1
\end{align*}
and also for \(G(t) = 1/\sin t - 1/t\)
\begin{equation*}
\lim_{t\to 0} G(t) % \frac{1}{\sin t} - \frac{1}{t}
= \lim_{t\to 0} \frac{t - \sin t}{t\sin t}
= \lim_{t\to 0} \frac{1 - \cos t}{\sin t + t\cos t} 
= \lim_{t\to 0} \frac{\sin t}{\cos t + \cos t - t\sin t} 
= \frac{0}{1+1+0} = 0.
\end{equation*}
Thus if we define \(G(0)=0\) then $G$ is continuous,
and we have \(U = \sup_{t\in[-\pi,\pi]} |G(t)|+1 < \infty\).
Now  for \(t\in [-\pi/2,\pi/2]\) we can estimate 
\begin{align}
|H_n(2t)|
&\leq \left|\frac{\sin\bigl((2n+1)t\bigr)}{\sin t} 
      - \frac{\sin(2nt)}{t}\right| \notag \\
&\leq \left|\frac{\sin\bigl((2n+1)t\bigr) - \sin(2nt)}{\sin t}\right|
      + \left| \sin(2nt) \left( \frac{1}{\sin t} - \frac{1}{t}\right) \right|
     \notag \\
&\leq \left| \frac{\sin t}{\sin t}\right| 
     + 1\cdot U = U + 2. \label{eq:5.22:sinn1t}
\end{align}
In \eqref{eq:5.22:sinn1t} we used the inequality
\begin{equation*}
\sin\bigl((2n+1)t\bigr) - \sin(2nt) \leq |\sin t|
\end{equation*}
for \(-\pi/2\leq t \leq \pi/2\) derived from
\begin{equation*}
\sin\bigl((2n+1)t\bigr) = \sin(2nt)\cos t + \cos(2nt)\sin t.
\end{equation*}
Note that the bound of \(H_n\) is independent of $n$.

Put \(M=\|f\|_\infty+1\).
Pick arbitrary \(\epsilon>0\), we may assume \(\epsilon < 1\).
Find \(\eta>0\)
such that 
\begin{itemize}
\item \(\eta < \epsilon\).
\item \(|f(t)| < \epsilon/MU\) for all \(t\in[-\eta,+\eta]\).
\end{itemize}

Put 
\(f_d(t) = f(t)/\sin(t/2)\)
and
\(f_e(t) = f(t)/(t/2)\).
Since the three functions in
\(\calF = \{f, f_d, f_e\}\)
are uniformly continuous in \(\T\setminus(-\eta,\eta)\), 
let \(\delta > 0\) be such that \(|g(t_1) - g(t_0)| < \eta\epsilon < \epsilon\)
for \(g\in \calF\)
whenever \(|t_1 - t_0| < \delta\). we may assume \(\delta<\eta\).

Pick \(n_0\in\N\) such that \(2\pi/n_0 < \delta)/MU\) and
take arbitrary \(n\geq n_0\).

Now
\begin{align}
\Delta_n 
&= \left|s_n(f;0) - \frac{1}{\pi}\int_{-\pi}^\pi f(t)\sin nt/t\,dt \right| 
   \notag \\
&= \left|\frac{1}{2\pi} \int_{-\pi}^\pi f(t)H_n(t)\,dt \right| 
 = 
   \frac{1}{2\pi}
   \left|
     \int_{-\eta}^\eta \cdots + \int_{\T\setminus[-\eta,+\eta]} \cdots
   \right| \notag \\
&\leq 
     \frac{1}{2\pi}
     \left(
         2MU\epsilon
       + \left|\int_{-\pi}^{-\eta} f(t)H_n(t),dt\right|
       + \left|\int_\eta^\pi f(t)H_n(t),dt\right|
     \right)  \label{eq:5.22:2int}
\end{align}

We will estimate the last two terms can also be as small as desired.
We will workout the last ($t$-positive) one. % , the other is similar.
For abbreviation, put \(\nu = (n+1/2)\).
The periods of 
\(\sin(\nu t)\) and \(\sin(nt)\) 
are \(\gamma_d=2\pi/\nu = 4\pi/(2n+1)\)
and \(\gamma_e=2\pi/n)\) respectably.
We will separate the integration segments to whole periods as available.
For \(\iota=d,e\) (as symbols), 
find the minimal \(l_\iota\) and maximal \(h_\iota\) such that 
\(\eta\leq l_\iota \gamma_\iota\) and \(h_\iota\gamma_\iota \leq \pi\).
The ``gaps'' size are
\begin{align*}
(l_d\gamma_d - \eta ) + (\pi - h_d\gamma_d) < 2\gamma_d &= 2\pi/(2n+1)\\
(l_e\gamma_e - \eta ) + (\pi - h_e\gamma_e) < 2\gamma_e &= 2\pi/n.
\end{align*}
Now
\begin{eqnarray*}
\left|\int_\eta^\pi f(t)H_n(t),dt\right|
&\leq&
    \left|\int_\eta^{\max(l_d\gamma_d,l_e\gamma_e)} f(t)H_n(t)\,dt\right| \\
& &  + \left|\sum_{k=l_d}^{h_d-1} \int_{k\gamma_d}^{(k+1)\gamma_d} 
           f_d(t)\sin(\nu t)\,dt\right| 
     + \left|\sum_{k=l_e}^{h_e-1} \int_{k\gamma_e}^{(k+1)\gamma_e}
           f_e(t)\sin(nt)\,dt\right| \\
& &  + \left|\int_{\min(l_d\gamma_d,l_e\gamma_e)}^\pi f(t)H_n(t)\,dt\right| \\
&\leq& 2\pi MU/n \\
& &  + \left|\sum_{k=l_d}^{h_d-1} \int_0^{\gamma_d}
           f_d(t+k\gamma_d)\sin\bigl(\nu (t+k\gamma_d)\bigr)\,dt\right| 
\\ & &
     + \left|\sum_{k=l_e}^{h_e-1} \int_0^{\gamma_e}
           f_e(t+k\gamma_e)\sin\bigl(n(t+k\gamma_e)\bigr)\,dt\right| \\
& &  + 2\pi MU/n \\
&\leq& 4\pi MU/n \\
& &  + \left|\sum_{k=l_d}^{h_d-1} \int_0^{\gamma_d/2}
           \bigl(f_d(t+k\gamma_d) - f_d(t+k\gamma_d + \gamma_d/2)\bigr)
           \sin\bigl(\nu (t+k\gamma_d)\bigr)\,dt\right| \\
& &  + \left|\sum_{k=l_e}^{h_e-1} \int_0^{\gamma_e/2}
           \bigl(f_e(t+k\gamma_e) - f_e(t+k\gamma_e + \gamma_e/2) \bigr)
           \sin\bigl(n(t+k\gamma_e)\bigr)\,dt\right| \\
&\leq& \epsilon
       + (h_d - l_d)\gamma_d\epsilon/2
       + (h_e - l_e)\gamma_e\epsilon/2. \\
&\leq& (1+\pi/2+\pi/2)\epsilon.
\end{eqnarray*}

Similar estimation can be done to 
\begin{equation*}
\left|\int_{-\pi}^{-\eta} f(t)H_n(t),dt\right|.
\end{equation*}
Thus the last two terms in \eqref{eq:5.22:2int} can be arbitrarily small,
and so 
\begin{equation*}
\lim_{n\to\infty}
  \left|s_n(f;0) - \frac{1}{\pi}\int_{-\pi}^\pi f(t)\sin nt/t\,dt \right| = 0.
\end{equation*}

Therefore, in order to see that \(\lim_{n\to\infty} s_n(f;0)=0\), 
it is sufficient to show that 
\begin{equation} \label{eq:5.22.suff}
\lim_{t\to 0} \int_{-\pi}^\pi f(t)\sin nt/t\,dt = 0.
\end{equation}

{\small
Note that it is \emph{not} necessarily that \(\lim_{t\to 0}f(t)/t = 0\).
}

Let $K$ be such that \(|f(x)-f(y)|/|x-y|^\alpha\) for all \(x\neq y\).
Then 
\begin{equation*}
|f(t)/t| 
= |f(t)| |t|^{\alpha-1}/|t|^\alpha
\leq K |t|^{\alpha-1}
\end{equation*}
for \(t\neq 0\). Thus
\begin{equation*}
\|f/t\|_1
= \frac{1}{2\pi} \int_{-\pi}^\pi |f(t)/t|\,dt 
\leq \frac{1}{2\pi} \int_{-\pi}^\pi K|t|^{\alpha-1}\,dt 
% = \frac{K}{\alpha\pi}\pi^\alpha.
= K\pi^{\alpha-1}/\alpha < \infty
\end{equation*}
which shows that \(f_e = f(t)/t \in L^1(\T)\).
Now
\begin{equation*}
\int_{-\pi}^\pi f(t)\sin nt/t\,dt 
= (2\pi/2i)\bigl(\hat{f_e}(n) - \hat{f_e}(n)\bigr).
\end{equation*}
By the Riemann-Lebesgue lemma \(\lim_{n\to\infty} \hat{f_e}(n) = 0\).
Hence \eqref{eq:5.22.suff} holds.

All the above bounds and minimal (\(\delta\), \(\eta\)) values
could have been taken for any value other than \(t=0\).
So we could have the limit converge uniformly
also for \(f_{\tau}(t) = f(t+\tau) - f(\tau)\).


\iffalse
Define \(f_\tau(t) = f(t + \tau)\).
Now 
\begin{align*}
2\pi s_n(f;x)
&= \int_{\pi}^\pi f(t)D_n(x - t)\,dt
 = \int_{\pi}^\pi f(t+x)D_n(x - t + x)\,dt
 = \int_{\pi}^\pi f_x(t)D_n(-t)\,dt \\
& = 2\pi s_n(f_x;0).
\end{align*}
Clearly \(f\in\Lip_\alpha\) iff  \(f_x\in\Lip_\alpha\).
Thus it is sufficient to to show that \(\lim_{n\to\infty} s_n(f;0) = f(0)\).
By looking at \(g(t) = f(t) - f(0)\) we see that 
similarly 
Clearly \(f\in\Lip_\alpha\) iff  \(g\in\Lip_\alpha\)
and \(s_n(f)\) differs from \(s_n(g)\) by the constant \(\hat{f}(0)\).
Thus we may also assume \(f(0)=0\).

{\small
Note that it is \emph{not} necessarily that \(\lim_{t\to 0}f(t)/t = 0\).
}

Let $K$ be such that \(|f(x)-f(y)|/|x-y|^\alpha\) for all \(x\neq y\).
Then 
\begin{equation*}
|f(t)/t| 
= |f(t)| |t|^{\alpha-1}/|t|^\alpha
\leq K |t|^{\alpha-1}
\end{equation*}
fot \(t\neq 0\). Thus
\begin{equation*}
\|f/t\|_1
= \frac{1}{2\pi} \int_{-\pi}^\pi |f(t)/t|\,dt 
\leq \frac{1}{2\pi} \int_{-\pi}^\pi K|t|^{\alpha-1}\,dt 
% = \frac{K}{\alpha\pi}\pi^\alpha.
= K\pi^{\alpha-1}/\alpha < \infty
\end{equation*}
which shows that \(f(t)/t \in L^1(\T)\).
\fi


%%%%%%%%%%%%%%%%%
\end{enumerate}
