% -*- latex -*-
% $Id: rudinrca1.tex,v 1.2 2006/09/08 07:29:12 yotam Exp $

%%%%%%%%%%%%%%%%%%%%%%%%%%%%%%%%%%%%%%%%%%%%%%%%%%%%%%%%%%%%%%%%%%%%%%%%
%%%%%%%%%%%%%%%%%%%%%%%%%%%%%%%%%%%%%%%%%%%%%%%%%%%%%%%%%%%%%%%%%%%%%%%%
%%%%%%%%%%%%%%%%%%%%%%%%%%%%%%%%%%%%%%%%%%%%%%%%%%%%%%%%%%%%%%%%%%%%%%%%
\chapterTypeout{Abstract Integration}

%%%%%%%%%%%%%%%%%%%%%%%%%%%%%%%%%%%%%%%%%%%%%%%%%%%%%%%%%%%%%%%%%%%%%%%%
%%%%%%%%%%%%%%%%%%%%%%%%%%%%%%%%%%%%%%%%%%%%%%%%%%%%%%%%%%%%%%%%%%%%%%%%
\section{Notes}

%%%%%%%%%%%%%%%%%%%%%%%%%%%%%%%%%%%%%%%%%%%%%%%%%%%%%%%%%%%%%%%%%%%%%%%%
\subsection{Lebesgue's Monotone Convergence --- Proof Fix}

\index{Lebesgue}
In Theorem~1.26 (page~22), it says:
\begin{quotation}
 \mldots, there exists \(\alpha \in [0,\infty)\) such that
\end{quotation}
It should be:
\begin{quotation}
 \mldots, there exists \(\alpha \in [0,\infty]\) such that
\end{quotation}

%%%%%%%%%%%%%%%%%%%%%%%%%%%%%%%%%%%%%%%%%%%%%%%%%%%%%%%%%%%%%%%%%%%%%%%%
\subsection{Lebesgue's Dominated Convergence --- Variant}

\index{Lebesgue}
Theorem~1.34 on page 27 requires
\begin{equation*}
|f_n(x)| \leq g(x) \qquad \textrm{(}n=1,2,3,\ldots; x\in X\textrm{),}
\end{equation*}
Instead, it could require:
\begin{eqnarray*}
g_n &\to& g \\
\int_X g_n d\mu  &\to& \int_X g d\mu  \\
|f_n(x)| &\leq& g(x).
\end{eqnarray*}
Let's have it explicitly.
%%%%%%%%%%%%%%%%
\begin{llem} \label{lem:Lebesgue:domvar}
Suppose \(\{f_n\}\) is a sequence of complex measurable functions on $X$ such that
\begin{equation*}
 f(x) = \lim_{n\to\infty} f_n(x)
\end{equation*}
exists for every \(x\in X\). If there is a a sequence of functions \(\{g_n\}\)
in \(L^1(\mu)\) and a function \(g\in L^1(\mu)\)
such that
\begin{eqnarray*}
   |f_n(x)| &\leq& g_n(x) \qquad (\forall n\in\N,\;\forall x\in X) \\
  \lim_{n\to\infty} g_n(x) &=& g(x) \qquad (\forall x\in X) \\
  \lim_{n\to\infty} \int_X g_n\,d\mu  &=& \int_X g\,d\mu
\end{eqnarray*}
then
\begin{eqnarray}
  f &\in& L^1(\mu) \notag \\
  \lim_{n\to\infty} \int_X |f_n - f|\,d\mu  & = & 0 \label{eq:leb:domv1} \\
  \lim_{n\to\infty} \int_X f_n\,d\mu  & = & \int_X f\,d\mu \label{eq:leb:domv2}
\end{eqnarray}
\end{llem}
Note: The orginal theorem~1.34 (\cite{RudinRCA80}) follows using \(g_n=g\).

\begin{thmproof}
Since
\begin{equation*}
 |f| = \lim_{n\to\infty} |f_n| \leq \lim_{n\to\infty} |g_n| = |g|
\end{equation*}
and $f$ is measurable, \(f\in L^1(\mu)\).
Since \(|f-f_n|\leq g_n + g\),
Fatou's lemma applies to \(g + g_n - |f_n - f|\) and yields
\begin{eqnarray*}
 2 \int_X g\,d\mu
 &=& \int_X g\,d\mu
     + \int_X \lim_{n\to\infty}g_n
     - \int_X \lim_{n\to\infty}|f_n - f|\,d\mu \\
 &=& \int_X \left(\lim_{n\to\infty}(g + g_n  - |f_n - f|\right)\,d\mu \\
 &=& \int_X \liminf_{n\to\infty} g + g_n  - |f_n - f|\,d\mu \\
 &\leq& \liminf_{n\to\infty}\int_X g + g_n  - |f_n - f|\,d\mu \\
 &=& \int_X g\,d\mu
     + \lim_{n\to\infty} \int_X g_n\,d\mu
     \liminf_{n\to\infty} \left(-\int_X |f_n - f|\,d\mu\right) \\
 &=& 2\int_X g\,d\mu - \limsup_{n\to\infty} \int_X |f_n - f|\,d\mu .
\end{eqnarray*}
Since \(2\int_X g\,d\mu\) is finite
\begin{equation*}
 \liminf_{n\to\infty} \int_X |f_n - f|\,d\mu \leq 0.
\end{equation*}
which shows (\ref{eq:leb:domv1}).
The last assertion (\ref{eq:leb:domv2}) follows immediately from
\begin{equation*}
 \left| \int_X f_n\,d\mu - \int_X f\,d\mu\right|  \leq \int_X |f_n - f|\,d\mu.
\end{equation*}
\end{thmproof}





%%%%%%%%%%%%%%%%%%%%%%%%%%%%%%%%%%%%%%%%%%%%%%%%%%%%%%%%%%%%%%%%%%%%%%%%
%%%%%%%%%%%%%%%%%%%%%%%%%%%%%%%%%%%%%%%%%%%%%%%%%%%%%%%%%%%%%%%%%%%%%%%%
\section{Exercises} % pages 32-33

%%%%%%%%%%%%%%%%%
\begin{enumerate}
%%%%%%%%%%%%%%%%%

%%%%%%%%%%%%%%
\begin{excopy}
Does there exist an infinite \salgebra\ %
which has only countably many members?
\end{excopy}

No.

Say be negation a \salgebra\ \M\ in $X$ %
has countably many members \(\{A_i\}_{i=1}^\infty\).
We will build an infinite countable family out of \M,
mutually disjoint, that will be a base for \M.

For each \(x\in X\), let
\begin{equation} \label{eq:cntcap}
G_x = \bigcap_{x\in A_i} A_i.
\end{equation}
Clearly, \(x\in G_x\in \M\). The latter membership relation
by the face that the intersection in (\ref{eq:cntcap}) is
of at most countable number of sets.
We observe that if \(G_x \cap G_y \neq \emptyset\)
then \(G_x = G_y\).

By negation, if \(G_x \neq G_y\), then
\(G_x\setminus G_y \neq \emptyset\)
or
\(G_y\setminus G_x \neq \emptyset\).
\(G_x\cap G_y \subsetneq G_y\)
Put \(G = G_x\cap G_y \in \M\)
and so
\(G\subsetneq G_x\)
or
\(G \subsetneq G_y\).
\Wlogy, \(G \subsetneq G_x\).
If \(x\in G\) then $G$ participates in the intersection of (\ref{eq:cntcap})
and thus \(G_x\subset G\) leading to the \(G_x \subsetneq G_x\) contradiction.
Otherwise, \(x\neq G\) and so \(x \in G_y^c \in \M\)
and so \(G_y^c\) participates in (\ref{eq:cntcap}) and so
\(x\in G_x \subset G_y^c = G_x^c\) which is also a contradiction.

Thus the family \(\calB = \{G_x\}_{x\in X}\) is a subset of \M\
of disjoint (ignoring repetitions) subsets of $X$.
For each \(A\in \M\) we define
\begin{equation*}
A' = \bigcup_{x\in A} G_x.
\end{equation*}
Clearly, \(A\subset A'\). To show the reverse inclusion,
let \(w\in A'\).
Then for some \(x\in A\) we have \(w \in G_X\).
But this means that for every set \(S\in M\),
if \(x\in S\) then \(w\in S\). In particular, \(x\in A\) and so
\(A'\subset A\), hence \(A=A'\).

We showed that every set in \M\ is a disjoint union
of some subset of \calB.
Thus, \calB\ cannot be finite and by being subset of \M\
is countable. Therefore, the union of every subset of
\calB\ is in \M, and by being disjoint
\(|\M| = 2^{|\calB|} > \aleph_0\).


%%%%%%%%%%%%%%
\begin{excopy}
Prove and analogue of Theorem~1.8 for $n$ functions.
\end{excopy}

\begin{lthm}
Let \(\{u_i\}_{i=1}^n\) be real measurable functions on a measurable
space $X$, let
\(\Phi\)\ be a continuous mapped of \(\R^n\) into a topological space $Y$,
and define
\begin{equation*}
h(x) = \Phi(u_1(x),u_2(x),\ldots,u_n(x))
\end{equation*}
for \(x\in X\). Then \(h: X\to Y\) is measurable.
\end{lthm}
\begin{thmproof}
Put \(f(x) = (u_1(x),u_2(x),\ldots,u_n(x))\).

Then $f$ maps $X$ into \(\R^n\). Since
\(h = \Phi\circ f\), Theorem~1.7 shows that it is enough to prove
the measurability of $f$.

If $B$ is any open box in \(\R^n\), with sides parallel to the axes,
then $B$ is a cartesian product of $n$ segments \(\seqn{I}\), and
\begin{equation*}
f^{-1}(B) = \bigcap_{j=1}^n u_j^{-1}(I_j),
\end{equation*}
which is measurable, by our assertions on \(u_j\).
Every open set $V$ in \(\R^n\) is a countable union of such boxes \(B_i\),
and since
\begin{equation*}
f^{-1}(V) = f^{-1}\left(\bigcup_{i=1}^\infty B_i\right)
          = \bigcup_{i=1}^\infty f^{-1}(B_i),
\end{equation*}
\(f^{-1}\) is measurable.
\end{thmproof}

%%%%%%%%%%%%%%
\begin{excopy}
Prove that if $f$ is a real function on a measurable space $X$
such that \(\{x:f(x)\geq r\}\) is measurable for every rational $r$,
then $f$ is measurable.
\end{excopy}

Let \(\alpha\in \R\). Choose some decreasing sequence \(\{q_i\}_{i\in\N}\)
such that \(q_i\in\Q\) and \(q_i\to\alpha\). Now
\begin{equation*}
f^{-1}((\alpha,\infty]) = \bigcup_{i} f^{-1}((q_i,\infty]).
\end{equation*}
By Theorem~1.12(c), $f$ is measurable.

%%%%%%%%%%%%%%
\begin{excopy}
Let \(\{a_n\}\) and \(\{b_n\}\) be sequences in \([-\infty,\infty]\),
and prove the following assertions:

\begin{itemize}

\item[(a)] \qquad
   \(\displaystyle
      \limsup_{n\to \infty} (-a_n) =
     -\liminf_{n\to \infty} a_n
   \).

\item[(b)] \qquad
   \(\displaystyle
      \limsup_{n\to \infty} (a_n + b_n) \leq
      \limsup_{n\to \infty} a_n +
      \limsup_{n\to \infty} b_n\)

 provided none of the sums is of the form \(\infty - \infty\).

\item[(c)] If \(a_n\leq b_n\) for all $n$, then
 \[\liminf_{n\to\infty} a_n \leq \liminf_{n\to\infty} b_n\,.\]
\end{itemize}
Show by an example that strict inequality can hold in (b).
\end{excopy}

Let us have formalized definitions:
\begin{eqnarray*}
 \limsup_{n\to \infty} a_n  &=& \inf_{n\in\N}\, \sup_{i\geq n} a_i \\
 \liminf_{n\to \infty} a_n  &=& \sup_{n\in\N}\, \inf_{i\geq n} a_i
\end{eqnarray*}

\begin{itemize}
 \item[(a)]
  \begin{eqnarray*}
    \limsup_{n\to \infty} (-a_n)
     &=& \inf_{n\in\N}\, \sup_{i\geq n} -a_i \\
     &=& \inf_{n\in\N}\, -\inf_{i\geq n} a_i \\
     &=& -\sup_{n\in\N}\, \inf_{i\geq n} a_i \\
     &=& -\liminf_{n\to \infty} a_n
  \end{eqnarray*}

 \item[(b)]

 Let
 \begin{eqnarray}
  A_k &=& \sup_{n\geq k} a_n       \label{eq:Ak:limsup} \\
  B_k &=& \sup_{n\geq k} b_n       \label{eq:Bk:limsup} \\
  S_k &=& \sup_{n\geq k} a_n+b_n   \notag \\
 \end{eqnarray}
 If by negation, \(S_k > A_k + b_k\) then put
 \(\epsilon = S_k - (A_k+b_k)\).
 By definition, there exists \(m\geq k\) such that
 \begin{equation*}
 a_m+b_m > S_k - \epsilon \geq A_k + b_k
 \end{equation*}
 and so \(a_m > A_k\) or \(b_m > B_k\), but each is a contradiction.
 Hence,
 \begin{equation*}
  \sup_{n\geq k} a_n+b_n \leq \sup_{n\geq k} a_n + \sup_{n\geq k} b_n
 \end{equation*}
 for any \(k\in\N\).
 Clearly  \(\{A_k\}_{k\in\N}\) and \(\{A_k\}_{k\in\N}\)
 are decreasing and their infimum is their existing limit. Thus

 \begin{eqnarray*}
   \limsup_{n\to \infty} (a_n + b_n)
   &=& \inf_{n\in\N}\, \sup_{i\geq n} a_i+b_i \\
   &\leq& \inf_{k\in\N}\,   \left(\sup_{n\geq k} a_i
                          +       \sup_{n\geq k} b_i\right) \\
   &=& \inf_{k\in\N} A_k + B_k \\
   &=& \lim_{k\in\N} A_k + B_k \\
   &=& \lim_{k\in\N} A_k + \lim_{k\in\N} B_k \\
   &=& \inf_{k\in\N} A_k + \inf_{k\in\N} B_k \\
   &=& \limsup_{n\in\N} a_n + \limsup_{n\in\N} a_n
 \end{eqnarray*}

  Let \(a_n = (-1)^n n\) and \(b_n = (-1)^{n+1} n\) and
  so
  \(a_n+b_n = ((-1)^n + (-1)^{n+1}) n = 0\)
  But clearly \(\limsup a_n = \limsup b_n = \infty\).

 \item[(c)]
  Define the monotonic increasing sequences:
  \begin{eqnarray*}
  A_k &=& \inf_{n\geq k} a_n  \\
  B_k &=& \inf_{n\geq k} b_n  \\
  \end{eqnarray*}
  and so
  \begin{equation*}
  \liminf_{n\to\infty} a_n = \sup A_k = \lim A_k
      \leq \lim B_k = \sup B_k = \liminf_{n\to\infty} b_n\,.
  \end{equation*}

\end{itemize}

%%%%%%%%%%%%%%
\begin{excopy}
Prove that the set of points at which a sequence of measurable
real functions converges is a measurable set.
\end{excopy}

Let \(\{f_n\}_{n\in\N}\) be a sequence of measurable real functions on $X$.
Let \(\overline{f} = \limsup f_n\)
and \(\underline{f} = \liminf f_n\).
For any \(r\in \R\),
\begin{equation*}
E_r = \overline{f}^{-1}([r,\infty])
 = \bigcap_{n\in\N} \bigcup_{k\geq n} f_k^{-1}([r,\infty]).
\end{equation*}
Thus \(\overline{f}\) is measurable and similarly, so is \(\underline{f}\).
Now \(d = \overline{f} - \underline{f}\) is also measurable.
Surely
\begin{equation*}
E_0 = X \setminus d^{-1}\left([-\infty,0)\cup(0,\infty]\right) = d^{-1}(0)
\end{equation*}
is measurable. But \(E_0\) is exactly the set where
\(\overline{f}\) and \(\underline{f}\) agree
which is the set of point where \(\{f_n\}_{n\in\N}\) converge.


%%%%%%%%%%%%%%
\begin{excopy}
Let $X$ be an uncountable set, let \M\ be the collection of
all sets \(E\subset X\) such that either $E$ or \(E^c\)
is at most countable,
and define \(\mu(E)=0\) in the first case,
\(\mu(E)=1\) in the second.
Prove that \(\mu\) is a \salgebra\ in $X$ and that \(\mu\) is a measure on \M.
\end{excopy}

Clearly \(\emptyset, X\in \M\) and
\(\mu(\emptyset) = 0\)
and \(\mu(X) = 1\).
By definition, \M\ is closed under complement operation.
Now let \(\{A_n\}_{n\in\N}\) be a countable family in \M.
If some \(A_i\) is uncountable, then so is \(U = \cup A_i\in \M\)
and the \(\mu(U) = 1\).
Otherwise, since \(\aleph_0 \times \aleph_0 = \aleph_0\),
then \(|U| = |\cup A_i| = \aleph_0\) and so \(U\in M\) with \(\mu(U) = 0\).
Thus \M\ is a \salgebra.


%%%%%%%%%%%%%%
\begin{excopy}
Suppose \(f_n:X\to[0,\infty]\) is measurable for \(n=1,2,3,\ldots\),
\(f_1 \geq f_2 \geq f_3 \geq \cdots \geq 0\),
\(f_n(x)\to f(x)\) as \(n\to\infty\), for every \(x\in X\),
and \(f_1\in L^1(\mu)\).
Prove that then
\begin{equation*}
 \lim_{n\to\infty}\int_X f_n d\mu = \int_X fd\mu
\end{equation*}
and show that this conclusion does \emph{not} follow if the condition
``\(f_1\in L^1(\mu)\)'' is omitted.
\end{excopy}

\index{Lebesgue}
This is an application of Lebesgue's Dominated Convergence Theorem,
with \(g(x) = f_1(x)\) serving as the dominating function.

Say the condition \(f_1\in L^1(\mu)\)'' is omitted.
Let, \(X = [0,1]\) with the natural measure $m$, and for \(n>=1\), let
\begin{equation*}
f_n(x) = \left\{\begin{array}{l@{\qquad}l}
                \infty &  0\leq x<1/n \\
                0      &  1/n \leq x \leq 1
                \end{array}\right.
\end{equation*}
Now we easily see that \(f_n\to 0\) and so
\begin{equation*}
\int_{[0,1]} \lim_{n\to\infty} f_n(x)dm
= 0 < \infty
= \lim_{n\to\infty} \int_{[0,1]} f_n(x)dm.
\end{equation*}

%%%%%%%%%%%%%%
\begin{excopy}
Put \(f_n = \chhi_E\) if $N$ is odd, \(f_n = 1 - \chhi_E\) if $n$ is even.
What is the relevance of this example
\index{Fatou's lemma}
to Fatou's lemma.
\end{excopy}

In this case, the inequality of Fatou's Lemma becomes strict.
Say the measure is $m$ on $X$.
\begin{equation*}
\int_X \liminf_{n\to\infty} f_n(x)dm
= 0 < \min(\mu(E),\mu(X\setminus E))
= \liminf_{n\to\infty} \int_X f_n(x)dm.
\end{equation*}

%%%%%%%%%%%%%%
\begin{excopy}
Suppose \(\mu\) is a positive measure on $X$, \(f:X\to[0,\infty]\)
is measurable, \(\int_X fd\mu = c\), where \(0<c<\infty\),
and \(\alpha\) is a constant. Prove that
\begin{equation*}
\lim_{n\to\infty} \int_X n\log \left[ 1 + (f/n)^\alpha\right]d\mu
 =\left\{\begin{array}{l@{\qquad}l}
         \infty & \textrm{if }\; 0<\alpha<1,\\
         c      & \textrm{if }\; \alpha = 1,\\
         0      & \textrm{if }\; 1 < \alpha < \infty.
         \end{array}
         \right.
\end{equation*}


\emph{Hint}: If \(\alpha\geq 1\), then the integrands are dominated by
\(\alpha f\).
If \(\alpha<1\),
\index{Fatou's lemma}
Fatou's lemma  can be applied.
\end{excopy}

First we need to show some inequalities that
will establish the hint.
\begin{llem} \label{llem:apbp:geq}
If \(a \geq b \geq 0\) and \(a,\alpha\geq 1\), then
\(a^\alpha - b^\alpha \geq a - b\).
\end{llem}

\begin{thmproof}
Since \(a\geq b\) and \(\alpha-1\geq 0\) we have
\(a^{\alpha-1} \geq b^{\alpha-1}\) and so
\begin{equation*}
a^{\alpha-1} - 1 \geq b^{\alpha-1} - 1
\end{equation*}
The left hand side must be non negative.
We look at the sign of the right side of the last inequality.
In either case, we can deduce:
\begin{equation*}
a (a^{\alpha-1} - 1) \geq b(b^{\alpha-1} - 1).
\end{equation*}
The above is equivalent to
\begin{equation*}
a^\alpha - b^\alpha \geq a - b.
\end{equation*}
\end{thmproof}


\begin{llem} \label{llem:nlogx:leq}
Let \(n\in\N\)
and let \(x,\alpha\in\R\) such that \(x\geq 0\) and \(\alpha\geq 1\)
then
\begin{equation}
n\log\left(1 + \left(\frac{x}{n}\right)^\alpha\right) \leq \alpha x.
\end{equation}
\end{llem}

\begin{thmproof}
We assume \(n\in\N\) and the requirements of the lemma hold for
\(x,\alpha\in\R\).
Let
\begin{displaymath}
g(t) = e^t - t.
\end{displaymath}
Clearly, \(g(0)=1\) and for \(t\geq 0\)
we have \(g'(t) = e^t - 1 \geq 0\) and thus
\begin{equation} \label{eq:etmtg1}
e^t - t \geq 1 \qquad \textrm{for}\, t\geq 0.
\end{equation}
\begin{displaymath}
f(x) = e^{\alpha x/n} - x/n.
\end{displaymath}
If \(\alpha=1\) then from (\ref{eq:etmtg1}, we have \(f(x)\geq 1\)
for \(x\geq 0\).

Using Lemma~\ref{llem:apbp:geq}
with \(a=e^{x/n}\) and \(b=x/n\),
we see that
\begin{equation*}
e^{\alpha x/n} - (x/n)^\alpha \geq e^{x/n} - x/n \geq 1.
\end{equation*}

Hence
\begin{equation*}
1 + (x/n)^\alpha \leq e^{\alpha x/n}.
\end{equation*}
Equivalently,
\begin{equation*}
\log\left(1 + (x/n)^\alpha\right) \leq \alpha x/n,
\end{equation*}
that is,
\begin{equation*}
n\log\left(1 + (x/n)^\alpha\right) \leq \alpha x.
\end{equation*}
\end{thmproof}

% Hey, I found a hand-written (Hebrew) workout I made about 20 years ago!
% Here is a sort of edited copy.

Put
\begin{equation*}
f_n = n\log\left[ 1 + (f/n)^\alpha\right]
    = n^{1-\alpha}
      \log\left[\left(1 + f^\alpha/n^\alpha\right)^{n^\alpha}\right]
\end{equation*}
Note that if \(f(x)=0\) then \(f_n(x)=0\) and thus
\begin{equation*}
X' = f^{-1}\{(0,\infty]\}\subset X
\end{equation*}
and for every $n$ we have \(\int_X f_n = \int_{X'} f_n\)
and we may assume that \(f,f_n > 0\).

Now
\begin{eqnarray*}
\lim_{n\to\infty} f_n
 &=& \left(\lim_{n\to\infty} n^{1-\alpha}\right) \cdot
     \log\left(\lim_{n\to\infty}
         \left(1 + f^\alpha/n^\alpha\right)^{n^\alpha}\right) \\
 &=& f^\alpha  \lim_{n\to\infty} n^{1-\alpha} \\
 &=&  \left\{\begin{array}{l@{\qquad}l}
         \infty & \textrm{if }\; 0<\alpha<1,\\
         f      & \textrm{if }\; \alpha = 1,\\
         0      & \textrm{if }\; 1 < \alpha < \infty.
         \end{array}
      \right.
\end{eqnarray*}

Assume \(\alpha<1\).
\index{Fatou's lemma}
From Fatou's lemma,
\begin{equation*}
\lim_n \int f_n \geq \int \lim_n f_n = \infty.
\end{equation*}

Assume \(\alpha\geq 1\).
From local lemma~\ref{llem:nlogx:leq} we have \(f_n\leq \alpha f\).
\index{Lebesgue's!Dominated Convergence Theorem}
\index{Dominated Convergence Theorem}
Using Lebesgue's Dominated Convergence Theorem,
\begin{equation*}
\lim_n\int_X f_n\, d\mu = \int \lim_n f_n\,d\mu
 = \left\{\begin{array}{l@{\qquad}l}
         \int_X f = c     & \textrm{if }\; \alpha = 1,\\
         \int_X 0 = 0     & \textrm{if }\; 1 < \alpha
         \end{array}
  \right.
\end{equation*}




%% This is a newer trial, not remembering  the good old
\iffalse
First we need to show some inequalities that
will establish the hint.
\begin{llem} \label{llem:apbp:geq}
If \(a \geq b \geq 0\) and \(a,\alpha\geq 1\), then
\(a^\alpha - b^\alpha \geq a - b\).
\end{llem}

\begin{thmproof}
Since \(a\geq b\) and \(\alpha-1\geq 0\) we have
\(a^{\alpha-1} \geq b^{\alpha-1}\) and so
\begin{equation*}
a^{\alpha-1} - 1 \geq b^{\alpha-1} - 1
\end{equation*}
The left hand side must be non negative.
We look at the sign of the right side of the last inequality.
In either case, we can deduce:
\begin{equation*}
a (a^{\alpha-1} - 1) \geq b(b^{\alpha-1} - 1).
\end{equation*}
The above is equivalent to
\begin{equation*}
a^\alpha - b^\alpha \geq a - b.
\end{equation*}
\end{thmproof}


\begin{llem} \label{llem:nlogx:leq}
Let \(n\in\N\)
and let \(x,\alpha\in\R\) such that \(x\geq 0\) and \(\alpha\geq 1\)
then
\begin{equation}
n\log\left(1 + \left(\frac{x}{n}\right)^\alpha\right) \leq \alpha x.
\end{equation}
\end{llem}

\begin{thmproof}
We assume \(n\in\N\) and the requirements of the lemma hold for
\(x,\alpha\in\R\).
Let
\begin{displaymath}
g(t) = e^t - t.
\end{displaymath}
Clearly, \(g(0)=1\) and for \(t\geq 0\)
we have \(g'(t) = e^t - 1 \geq 0\) and thus
\begin{equation} \label{eq:etmtg1}
e^t - t \geq 1 \qquad \textrm{for}\, t\geq 0.
\end{equation}
\begin{displaymath}
f(x) = e^{\alpha x/n} - x/n.
\end{displaymath}
If \(\alpha=1\) then from (\ref{eq:etmtg1}, we have \(f(x)\geq 1\)
for \(x\geq 0\).

Using Lemma~\ref{llem:apbp:geq}
with \(a=e^{x/n}\) and \(b=x/n\),
we see that
\begin{equation*}
e^{\alpha x/n} - (x/n)^\alpha \geq e^{x/n} - x/n \geq 1.
\end{equation*}

Hence
\begin{equation*}
1 + (x/n)^\alpha \leq e^{\alpha x/n}.
\end{equation*}
Equivalently,
\begin{equation*}
\log\left(1 + (x/n)^\alpha\right) \leq \alpha x/n,
\end{equation*}
that is,
\begin{equation*}
n\log\left(1 + (x/n)^\alpha\right) \leq \alpha x.
\end{equation*}
\end{thmproof}


Back to the exercise.
We have three cases:
\begin{itemize}

 \item \(\alpha=1\)

 \begin{eqnarray*}
  \lim_{n\to\infty} n\log (1 + f/n)
   &=& \lim_{n\to\infty} \log \left((1 + f/n)^n\right) \\
   &=& \log \left(\lim_{n\to\infty} (1 + f/n)^n\right) \\
   &=& \log e^f = f
 \end{eqnarray*}
 Thus, using Lebesgue's dominated convergence theorem
 we have
 \begin{equation*}
 \lim_{n\to\infty} \int_X n\log \left(1 + f/n\right)d\mu
  = \int_X f\,d\mu = c.
 \end{equation*}

 \item \(\alpha<1\)

 \begin{eqnarray*}
      \left(1 + (x/n)^\alpha\right)^n
 &=& (1 + n^{1-\alpha}x^\alpha/n)^n \\
 &=& \sum_{k=0}^n \binom{n}{k}\left(n^{1-\alpha}x^\alpha/n\right)^k \\
 &\geq& \sum_{k=1} \cdots \\
 &\geq& n(n^{1-\alpha}x^\alpha/n) \\
 &=& n^{1-\alpha}x^\alpha.
 \end{eqnarray*}

 Thus
 \begin{equation*}
 \lim_{n\to\infty} n\log(1 + (x/n)^\alpha)
 = \lim_{n\to\infty} \log\left((1 + (x/n)^\alpha)^n\right) = \infty.
 \end{equation*}

 Again, using Lebesgue's dominated convergence theorem
 we have:
 \begin{equation*}
 \lim_{n\to\infty} \int_X n\log \left(1 + (f/n)^\alpha\right)d\mu \\
 = \int_X
      \left(\lim_{n\to\infty} n\log \left(1+(f/n)^\alpha\right)\right)d\mu \\
 = \infty.
 \end{equation*}

 \item \(\alpha>1\)

\end{itemize}
\fi % and of clumsy new trial


%%%%%%%%%%%%%%
\begin{excopy} % 10
Suppose \(\mu(X)<\infty\), \(\{f_n\}\) is a sequence of bounded complex
measurable functions on $X$, and \(f_n\to f\) uniformly on $X$. Prove that
\begin{equation*}
\lim_{n\to\infty} \int_X f_n d\mu = \int_X fd\mu,
\end{equation*}
and show that the hypothesis ``\(\mu(X)<\infty\)'' cannot be omitted.
\end{excopy}

Taking \(\epsilon = 1\), there is \(M_1>0\) such that
for any \(n\geq M_1\), we have \(|f_n(x) - f(x)| < M_1\) for all \(x\in X\).
Thus \(|f| < |f_n| + M_1\). In particular, $f$ is bounded.
Also the constant function \(g(x) = \|f\| + M_1\) dominates \(\{f_n\}\).
Applying Lebesgue's dominated convergence theorem gives the desired result

%%%%%%%%%%%%%%
\begin{excopy} % 11
Show that
\begin{equation*}
 A = \bigcap_{n=1}^\infty \bigcup_{k=n}^\infty E_k
\end{equation*}
in Theorem~1.41, and hence provided th theorem without any reference
to integration.
\end{excopy}

The set $A$ is defined as the set of all \(x\in X\)
that belong to infinitely many \(E_k\).
If \(x\in X\) then \(x \in U_k = \cup_{k=n}^\infty E_k\) for any \(n>0\).
Conversely, if \(x\notin A\) then there must be some \(N>0\)
such that \(x\notin E_k\) for all \(k\geq N\).
Clearly, \(x\notin U_N\) and the set equality is shown.

Now, let's quote the Theorem:
\begin{quotation}
\setcounter{quotethm}{40} % to get 41
  \begin{quotethm}
   Let \(\{E_k\}\) be a sequence of measurable sets in $X$, such that
   \begin{equation} \label{eq:thm41}
        \sum_{k=1}^\infty \mu(E_k) < \infty.
   \end{equation}
   Then almost all \(x\in X\) lie in at most finitely many of the sets \(E_k\).
  \end{quotethm}
  In view of this exercise, we have to show that \(\mu(A) = 0\).
  But the fact that the series in (\ref{eq:thm41}) conversges
  means that for any \(\epsilon>0\) there is an $N$ such that
  \(\sum_{k\geq N} \mu(E_k) < \epsilon\) and so
  \begin{equation*}
    \mu(A)
    = \mu \left(\bigcap_{n=1}^\infty \bigcup_{k=n}^\infty E_k\right)
    \leq \mu \left(\bigcup_{k=N}^\infty E_k\right) < \epsilon.
  \end{equation*}
  Thus \(\mu(A) = 0\).
\end{quotation}



%%%%%%%%%%%%%%
\begin{excopy}
Suppose \(f\in L^1(\mu)\). Prove that to each \(\epsilon\)
there exists a \(\delta > 0\) such that
\(\int_E |f|d\mu < \epsilon\) whenever \(\mu(E) < \delta\).
\end{excopy}

Let $X$ be the space on which \(\mu\) is defined.
\begin{equation*}
E_n = \{x\in X: n - 1 \leq |f(x)| < n
 \qquad \textrm{for } n\geq 1.
\end{equation*}
Clearly
\begin{equation*}
X = \Disjunion_{n=1}^\infty E_n\,.
\end{equation*}
Now,
\begin{equation} \label{eq:intf:sum:Ek}
\int_X |f|d\mu = \sum_{n=1}^\infty \int_{E_n} |f|d\mu < \infty
\end{equation}

Let \(\epsilon > 0\). By (\ref{eq:intf:sum:Ek}), there is
\(N>0\) such that
\begin{equation}
\sum_{n=N}^\infty \int_{E_n} |f|d\mu < \epsilon/2.
\end{equation}
Put \(H = \cup_{i=1}^{N-1} E_i\)
and \(T = \cup_{i=N}^\infty E_i\).
Note that \(X = H \disjunion T\).

Take \(\delta = \epsilon/2N\).
Now if $E$ is \(\mu\) measurable, and \(\mu(E)<\delta\) then
\begin{eqnarray*}
\int_E |f|d\mu
&=& \int_{E\cap H} |f|d\mu + \int_{E\cap T} |f|d\mu \\
&\leq& N\mu(E\cap H) + \int_{T} |f|d\mu \\
&\leq& N\mu(E) + \int_T |f|d\mu \\
&=& \epsilon/2  + \epsilon/2 = \epsilon.
\end{eqnarray*}


%%%%%%%%%%%%%%%
\end{enumerate}
%%%%%%%%%%%%%%%
