% -*- latex -*-

%%%%%%%%%%%%%%%%%%%%%%%%%%%%%%%%%%%%%%%%%%%%%%%%%%%%%%%%%%%%%%%%%%%%%%%%
%%%%%%%%%%%%%%%%%%%%%%%%%%%%%%%%%%%%%%%%%%%%%%%%%%%%%%%%%%%%%%%%%%%%%%%%
%%%%%%%%%%%%%%%%%%%%%%%%%%%%%%%%%%%%%%%%%%%%%%%%%%%%%%%%%%%%%%%%%%%%%%%%
\chapterTypeout{Positive Borel Measures}

%%%%%%%%%%%%%%%%%%%%%%%%%%%%%%%%%%%%%%%%%%%%%%%%%%%%%%%%%%%%%%%%%%%%%%%%
%%%%%%%%%%%%%%%%%%%%%%%%%%%%%%%%%%%%%%%%%%%%%%%%%%%%%%%%%%%%%%%%%%%%%%%%
\section{Notes}

While working on Exercise~\ref{ex:2:10}, I worked out the following
lemma, that eventually were not used. Here they are, so they
will not be lost.

\begin{llem} \label{llem:interval:subsum}
Assume the disjoint union
\begin{equation*}
 \Disjunion_{i\in\N} I_i \subset [0,1]
\end{equation*}
where \(I_i\) are intervals of the form
\((a_i,b_i)\),
\((a_i,b_i]\),
\([a_i,b_i)\) or
\([a_i,b_i]\). If
\begin{equation*}
 \sum_{i\in\N} m(I_i) =  \sum_{i\in\N} (b_i - a_i) < 1
\end{equation*}
then there exists \(u\in[0,1]\) such that for any open interval $J$,
where \(u\in J\subset [0,1]\), the following inequality
\begin{equation*}
 \sum_{i\in\N} m(I_i \cap J) < m(J)
\end{equation*}
holds.
\end{llem}

\begin{thmproof}
Let \(U = \disjunion_{i\in\N}\).
We will construct a decreasing sequence of clsoed intervals \(\{K_i\}_{i\in\N}\)
such that for all \(i\in\N\)
\begin{itemize}
 \item \(K_{i+i} \subset K_i\)
 \item \(m(K_i) = 2^{-i}\)
 \item \(\sum_{i\in\N} m(I_i\cap K_i) < m(K_i) = 2^{-i}\).
\end{itemize}
Put \(K_0 = [0,1]\). Now by induction, assume \(K_i = [\alpha,\beta]\)
is defined and for which satisfies the above requirements.
Put \(\gamma = (\alpha+\beta)/2\), consider the two subintervals
\(L=[\alpha,\gamma]\) and
\(R=[\gamma,\beta]\). The first two requirements hold for \(K_{i+1}\)
and at least one of them satisfies the third, and we pick it as \(K_{i+1}\).
Now let \(u = \cap_{i\in\N} K_i\) (identifying a singleton with the element),
such $u$ exists as an intersection of non empty compact sets (and also is
unique).

Now assume \(u\in J = (a,b)\) an open interval.
Then there is some (sufficiently small) \(K_j \subset J\).
Using \(m(J) = m(K_j) + m(J\setminus K_j)\) we have
\begin{eqnarray*}
 \sum_{i\in\N} m(I_i \cap J)
 &=& \sum_{i\in\N} m\left(I_i \cap (K_j \disjunion (J\setminus K_j)\right) \\
 &=& \sum_{i\in\N} {   m(I_i \cap K_j)
                     + m\left(I_i \cap (J\setminus K_j)\right)} \\
 &=& \sum_{i\in\N} m(I_i \cap K_j) +
     \sum_{i\in\N} m\left(I_i \cap (J\setminus K_j)\right) \\
 &<&     m(K_j) + \sum_{i\in\N} m\left(I_i \cap (J\setminus K_j)\right) \\
 &\leq&  m(K_j) + m(J\setminus K_j) \\
 &=& m(J).
\end{eqnarray*}
\end{thmproof}


\begin{llem} \label{llem:sumintervals:}
Assume the unit interval is a disjoint union
\begin{equation*}
 [0,1] = \Disjunion_{i\in\N} I_i
\end{equation*}
where \(I_i\) are intervals of the form
\((a_i,b_i)\),
\((a_i,b_i]\),
\([a_i,b_i)\) or
\([a_i,b_i]\). Then
\begin{equation*}
 \sum_{i\in\N} m(I_i) =  \sum_{i\in\N} (b_i - a_i)= 1.
\end{equation*}
\end{llem}

\begin{thmproof}
For any finite sub-sum \(\sum_{i=1}^N m(I_i) \leq 1\), hence
\(\sum_{i\in\N} m(I_i)\leq 1\).
By negation, we assume \(\sum_{i\in\N} m(I_i)< 1\).
Now the assumptions of the previous lemma~\ref{llem:interval:subsum} hold
giving \(u\in[0,1]\) such that for any open interval \(J\subset[0,1]\) we have
\(\sum_{i\in\N} m(I_i \cap J) < m(J)\).
By assumptions there must be (a unique) $j$ such that \(u\in I_j\).
Now there are two cases.
\begin{itemize}
 \item[(\emph{i})]
   (Internal) $u$ is inetrnal point of \(I_j\), then
   we can pick an open interval $J$ such that \(u\in J \subset I_j\).
   Note, that here we use the induced topology, so $u$ may
   also be $0$ or $1$.
   For the sake of unifying with the proof continuation, we denote
   a dummy \(j' = j + 1\).
 \item[(\emph{ii})]
   (Boundary) \(u\in\partial I_j\) that is (\(I_j\) is not open and)
   \(u=a_j\in I_j\) or \(u=b_j\in I_j\).  In that case there must be
   \(j'\), the index of the neighbor interval, such that
   \(u\in\partial I_{j'}\).  \(L=I_j\disjunion I_{j'}\) is an
   interval.  Now we can pick an open interval $J$ such that
   \(u\in J\subset L=I_j\disjunion I_{j'}\) (where $L$ is an interval).
\end{itemize}
In both cases \(J \subset I_j \disjunion I_{j'}\).
Now by lemma~\ref{llem:interval:subsum} \(\sum_{i\in\N} m(I_i\cap J) < m(J)\),
but
\begin{equation*}
 \sum_{i\in\N} m(I_i\cap J)
 \geq   m(I_j \cap J) +  m(I_{j'} \cap J) \\
 =     m\left( (I_j \disjunion I_{j'}) \cap J\right)  = m(J)
\end{equation*}
which is a contradiction.
\end{thmproof}


Let us generalize 
\index{Lusin}
Lusin theorem~2.24.
\begin{llem}
Let \(f:\R\to\C\) be a measurable function.
For each \(\epsilon>0\) there exists a continuous function \(g:\R\to\C\)
such that 
\begin{align*}
m\bigl(\{x\in\R: f(x)\neq g(x)\}\bigr) &< \epsilon\\
\forall x\in\R,\quad |g(x)| &\leq |f(x)|\,.
\end{align*}
\end{llem}
Note that the lemma does \emph{not} imply \emph{uniform} continuity of $g$.
\\
\begin{thmproof}
Pick an \(\epsilon>0\), \wlogy\ \(\epsilon<1/2\).
For each \(n\in\Z\) consider the restriction
\(f_n:[n,n+2]\to\C\) of $f$ to \([n,n+2]\) 
(actually, \(f_n = f_{\restriction[n,n+2]}\)).
By Lusin theorem~2.24 we can find \(g_n:[n,n+2]\to\C\)
such that 
\begin{align*}
m\bigl(\{x\in[n,n+2]: f(x)\neq g_n(x)\}\bigr) &< \epsilon_n = 2^{|n|+2}\epsilon\\
\forall x\in[n.n+2],\quad |g_n(x)| &\leq |f(x)|\,.
\end{align*}
We will connect \(g_n\) to define $g$.
For every $n$ pick \(t_n\in(n, n+1)\) such that 
\begin{equation*}
g_{n-1}(t_n) = g_n(t_n) = f_n(t_n) = f(t_n).
\end{equation*}
The existence of \(t_n\) is ensured by \(\epsilon_n < 1/2\)
and thus \(g_n\) and \(g_{n+1}\) could differ from $f$ in \([n,n+1]\)
ona set of measure  at most less than $1$.
For each \(x\in\R\) there is a unique \(n\in\Z\) such that 
\(t_n \leq t_{n+1}\),  and we define \(g(x) = g_n(x)\).
Clearly $g$ is continuous and differs from $f$ on a set whose
measure is at most
\begin{equation*}
\sum_{n\in\Z}\epsilon_n = \epsilon \sum_{n\in\Z}  2^{|n|+2} < \epsilon\,.
\end{equation*}
\end{thmproof}

%%%%%%%%%%%%%%%%%%%%%%%%%%%%%%%%%%%%%%%%%%%%%%%%%%%%%%%%%%%%%%%%%%%%%%%%
%%%%%%%%%%%%%%%%%%%%%%%%%%%%%%%%%%%%%%%%%%%%%%%%%%%%%%%%%%%%%%%%%%%%%%%%
\section{Exercises Support}

%%%%%%%%%%%%%%%%%%%%%%%%%%%%%%%%%%%%%%%%%%%%%%%%%%%%%%%%%%%%%%%%%%%%%%%%
\subsection{Topology}

We need to establish some more set-theoretic topological results.
Using the
\index{Stone-Cech compactification@Stone-\Cech\ compactification}
Stone-\Cech\ compactification (\cite{Dug1966}, \textsf{XI 8.3}).

\paragraph{Definition} (\cite{Dug1966}, \textsf{VII 7.1}).
A Hausdorff space $X$ is
\index{completely regular}
\emph{completely regular}
\index{Tychonoff}
(or Tychonoff)
if for each point \(p\in X\) and a closed \(A\subset X\) such that \(p\notin A\)
there is a continuous \(f:X\rightarrow I=[0,1]\) such that \(f(p) = 1\)
and \(\forall x\in A, f(x)=0\).

\paragraph{Definition} (\cite{Dug1966}, \textsf{VII~7.1}).
Let $X$ be a completely regular topological space.
Let \(P=C(X,I)\) be the set of continuous functions \(f:X\rightarrow I\)
where \(I=[0,1]\) the unit interval. The product space \(I^P\)
is compact by the Tychonoff theorem  (\cite{Dug1966}, \textsf{XI~1.4}).
We define the map
\begin{eqnarray} \label{eq:stonecech:rho}
 \rho: X & \rightarrow & I^P \\
 \rho(x) &=& (\lambda(x))_{\lambda \in C(X,I)} \notag
\end{eqnarray}
The Stone-\Cech\ compactification is
\begin{equation*}
 \beta(X) = \overline{\rho(X)},
\end{equation*}
where the closure is on \(I^P\).

\begin{llem} \label{llem:stonecech:omega1}
The Stone-\Cech\ compactification of \([0,\omega_1)\) is homeomorphic
to \([0,\omega_1]\).
\end{llem}
\begin{thmproof}
For each \(f\in C([0,\omega_1],I\), its restriction
\begin{equation*}
 f_{|[0,\omega_1)} \in C([0,\omega_1),I).
\end{equation*}
From lemma~\ref{llem:Vickery} every \(g \in C([0,\omega_1),I)\),
can be extended to \(\overline{f}\in C([0,\omega_1],I)\)
by defining \(\overline{f}(\omega_1)\) as the tail value.
Thus,
\begin{equation*}
 C([0,\omega_1),I) \cong C([0,\omega_1],I)
\end{equation*}
By using the notations of (\ref{eq:stonecech:rho}),
\begin{equation*}
 \rho\left(C([0,\omega_1),I)\right) \subset
 \rho\left(C([0,\omega_1],I)\right)
\end{equation*}
Since \([0,\omega_1]\) is compact, it image is compact
and so equals to its closure. We now have:
\begin{equation*}
 \rho\left(C([0,\omega_1),I)\right) \subset
 \overline{\rho\left(C([0,\omega_1),I)\right)} \subset
 \overline{\rho\left(C([0,\omega_1],I)\right)} =
 \rho\left(C([0,\omega_1],I)\right)
\end{equation*}
\end{thmproof}


\begin{llem}
Let \(K_1,K_2\subset [0,\omega_1]\) be uncountable compact.
Then \(K_1\cap K_2\) is uncountable compact.
\end{llem}
\begin{thmproof}
Clearly \(K = K_1\cap K_2\) is compact.
By negation assume $K$ it is countable.
For \(i=1,2\), let \(H_i = K_i\cap [0,\omega_1)\)
(note that \(\omega_1\in K_i\)),
% both \(H_i\) are uncountable and compact in the inherited topology of
and \(H=K\setminus \{\omega_1\}\).
For $H$, there exists by lemma~\ref{llem:countable:ub}
an upper bound \(b\in[0,\omega_1)\).
Since \(H_i\cap[0,b)\) is countable, we define
in the space \([0,\omega_1)\)
new compact (in the inherited topology) subsets
\begin{equation*}
 L+i = H_i \cap [b,\omega_1) = K_i \cap [b,\omega_1)
 \qquad \textrm{for}\; i=1,2.
\end{equation*}

\end{thmproof}


\iffalse
%%% NOT TRUE!!  only if Y closed.
\begin{llem}
Let $X$ be a topological space, \(Y\subset X\) a subspace
with the topology inherited from $X$.
If \(K\subset X\) compact then \(K\cap Y\) is compact in $Y$
\end{llem}
\begin{thmproof}
Let \(\{V_i\}_{i\in I}\) be an open cover in $Y$ of \(K\cap Y\).
By the definition of the inherited topology, there are open sets
 \(\{V_i\}_{i\in I}\) in $X$ such that \(U_i = V_i \cap Y\) (for \(i\in I\).
For each \(i\in I\) we define \(W_i = V_i
\end{thmproof}
\fi


%%%%%%%%%%%%%%%%%%%%%%%%%%%%%%%%%%%%%%%%%%%%%%%%%%%%%%%%%%%%%%%%%%%%%%%%
%%%%%%%%%%%%%%%%%%%%%%%%%%%%%%%%%%%%%%%%%%%%%%%%%%%%%%%%%%%%%%%%%%%%%%%%
\section{Exercises} % pages 58-61

%%%%%%%%%%%%%%%%%
\begin{enumerate}
%%%%%%%%%%%%%%%%%

%%%%%%%%%%%%%%
\begin{excopy}
Let \(\{f_n\}\) be a sequence of real nonnegative functions on \(\R^1\),
and consider the following four statements:
\begin{itemize}
 \itemch{a}
   If \(f_1\) and \(f_2\) are upper semicontinuous,
   then \(f_1 + f_2\) is upper semicontinuous.
 \itemch{b}
   If \(f_1\) and \(f_2\) are lower semicontinuous,
   then \(f_1 + f_2\) is lower semicontinuous.
 \itemch{c}
   If each \(f_n\) is upper semicontinuous, then \(\sum_1^\infty f_n\)
   is upper semicontinuous.
 \itemch{d}
   If each \(f_n\) is lower semicontinuous, then \(\sum_1^\infty f_n\)
   is lower semicontinuous.
\end{itemize}
Show that three of these atr true and one is false.
What happens if the word ``nonnegative'' is omitted?
Is the truth of the statements affected if \(\R^1\) is replaced
by a general topological space?
\end{excopy}

We will show that only \ich{c} is false.

\begin{itemize}
 \itemch{a}
  True, for
  \begin{equation*}
  \{x\in\R: f_1(x)+f_2(x) < a\}
  = \bigcup_{\alpha\in\R} \left(\{x\in\R: f_1(x)< \alpha\} \cap
                             \{x\in\R: f_2(x) < a-\alpha\} \right).
  \end{equation*}
  and since the intersection of two open sets is open, the above
  set is open as an infinite union of open sets.

 \itemch{b}
  True, for
  \begin{equation*}
  \{x\in\R: f_1(x)+f_2(x) > a\}
  = \bigcup_{\alpha\in\R} \left(\{x\in\R: f_1(x)> \alpha\} \cap
                             \{x\in\R: f_2(x) > a-\alpha\} \right).
  \end{equation*}
  and since the intersection of two open sets is open, the above
  set is open as an infinite union of open sets.
 \itemch{c}
 False. Let's construct the following sequence of upper semicontinuous
 functions. For \(n\geq 1\), let
 \begin{equation*}
  f_n(x) = \left\{\begin{array}{lc}
        1 & \qquad\textrm{if }\  1/(n+1) \leq x \leq 1/n \\
        0 & \textrm{Otherwise}
                  \end{array}\right..
 \end{equation*}
 Putting \(F = \sum f_n\).
 Now clearly \(\{x\in R: x < 1/2\} = \{0\}\) a singleton which is clearly
 \emph{not} open.

 \itemch{d}
 True.
 Using \ich{b} and induction, we see that  \(F_n = \sum_{k=1}^n f_k\)
 is lower semicontinuous. Now for any \(x\in\R\) such that
 \(\sum_{k=1}^\infty f_k(x) > a\) there exists $n$ such that \(F_n(x) > a\),
 and so
 \begin{equation*}
  \{x\in\R: \sum_{k=1}^\infty f_k(x)\ > a\} =
  \bigcup_n \{x\in\R: F_n(x)\ > a\}.
 \end{equation*}
 is open.
\end{itemize}


%%%%%%%%%%%%%%
\begin{excopy}
Let $f$ be an arbitrary complex function on \(\R^1\), end define
\begin{eqnarray*}
 \varphi(x,\delta) & = & \sup\{|f(s) - f(t)|: s,t\in (x-\delta, x+\delta)\}\\
 \varphi(x)        & = & \inf\{\varphi(x,\delta): \delta > 0\}.
\end{eqnarray*}
Prove that \(\varphi\) is upper semicontinuous, that $f$ is continuous
at a point $x$ if and only if \(\varphi(x)=0\), and hence that the set of
points of continuity of an arbitrary complex function is a \(G_\delta\).

Formulate and prove an analogous statement for general topological
space in place of \(\R^1\).
\end{excopy}

Assume \(\alpha \in \R\) and \(G=\varphi^{-1}(-\infty,\alpha)\).
Let \(b\in G\setminus\inter{G} = \) a boundary point.
Now for any \(\delta>0\), there exists
\(w\in (b-\delta/2,b+\delta/2)\) such that \(w\notin G\).
Hence there are \(x,y\in(w-\delta/2,w+\delta/2)\) such that
\(|f(x)-f(y)| \geq \alpha\). But \(x,y\in (b-\delta,b+\delta)\) as well,
and since \(\delta\) is arbitrary, \(\varphi(b)\geq \alpha\) and
so \(b\notin G\) hence $G$ is open showing that \(\varphi\) is
upper semicontinuous.

If $f$ is continuous at $w$, then for any
% (replacing classic roles of \(\epsilon\) and \(\delta\))
\(\epsilon>0\) there exists \(\delta>0\) such that
\(|f(w+h)-f(w)| < \epsilon/2\) whenever \(|h|<\delta\).
In such case, clearly for any \(x,y\in (w-\delta,w+\delta)\)
we have
\begin{equation*}
|f(x) - f(y)| \leq
|f(x) - f(w)| + |f(w) - f(y)| < \epsilon/2 + \epsilon/2 = \epsilon.
\end{equation*}
That \(\varphi(w) < \epsilon\) and since \(epsilon>0\) was arbitrary,
\(varphi(w)=0\).
Conversely, if \(varphi(w)=0\) then for any \(\epsilon>0\)
there exists \(\delta>0\) such that for any \(x,y\in (w-\delta,w+\delta)\)
we have \(|f(x)-f(y)| < \epsilon\). In particular,
 \(|f(x)-f(w)| < \epsilon\) for any \(x \in (w-\delta,w+\delta)\),
hence $f$ is continuous at $w$.

The set of points where an arbitrary function $f$ is continuous, is
 \(\cap_n \varphi^{-1}(-\infty, 1/n)\) and by what was just shown
this set is an intersection of countably many open sets, that is
a \(G_\delta\) type of set.

\paragraph{Generalization} Let \(X,T\) be a topological space
(points are in $X$ and $T$ is the family of opens sets in $X$).
Define
\begin{eqnarray*}
 \varphi(x,V) & = & \sup\{|f(s) - f(t)|: s,t\in V\}
                             \qquad \textrm{where}\; x\in V\in T\\
 \varphi(x)        & = & \inf\{\varphi(x,V): V\in T\}.
\end{eqnarray*}
The proof goes similarly, just replacing \(\delta\) and \(delta/2\)
with sub-neighborhoods.


%%%%%%%%%%%%%%
\begin{excopy}
Let $X$ be a metric space, with metric \(\rho\).
For any non empty \(E\subset X\), define
\begin{equation*}
 \rho_E(x) = \inf\{\rho(x,y): y\in E\}.
\end{equation*}
Show that \(\rho_E\) is a uniformly continuous function on $X$.
If $A$ and $B$ are disjoint nonempty closed subsets of $X$, examine
the relevance of the function
\begin{equation*}
  f(x) = \frac{\rho_A(x)}{\rho_A(x) +  \rho_B(x)}
\end{equation*}
\index{Urysohn's lemma}
to Urysohn's lemma.
\end{excopy}

Let \(x\in X\) and \(\emptyset \neq E \subset X\).
For any \(\epsilon>0\), there exists \(u\in E\) such that
\(\rho(x,u) < \rho_E(x) + \epsilon\).
Now
\begin{equation*}
\rho_E(y) \leq \rho(y,u)
          \leq \rho(x,y) + \rho(x,u)
             < \rho(x,y) + \rho_E(x) + \epsilon.
\end{equation*}
Hence \(\rho_E(y) - \rho_E(x) \leq \rho(x,y)\) and by symmetry,
\begin{equation*}
|\rho_E(y) - \rho_E(x)| \leq \rho(x,y).
\end{equation*}
From which continuity of \(\rho_E\) follows.

If a non empty \(A\subset X\) is closed and \(x\in X\setminus A\)
then \(\rho_A(x) > 0\), since otherwise \(x\in \overline{A}\).
Hence by \(A\cap B = \emptyset\), the denominator of $f$ satisfies
\(\rho_A(x)+\rho_B(x) > 0\) for any \(x\in X\) and so $f$ is
well defined and continuous.

Now this gives a simple algabraic construction of a function
as desired in Urysohn's Lemma. We can easily see
that \(f(x)=0\) for \(x\in A\) and
that \(f(x)=1\) for \(x\in B\).

%%%%%%%%%%%%%% 4
\begin{excopy}
Examine the proof of the
\index{Riesz theorem}
Riesz theorem and prove the following two statements:
\begin{itemize}
 \itemch{a}
   If \(E_1 \subset V_1\) and \(E_2 \subset V_2\), where \(V_1\) and \(V_2\)
   are disjoint open sets, then \(\mu(E_1\cup E_2) = \mu(E_1) + \mu(E_2)\),
   even if \(E_1\) and \(E_2\) are not in \frakM.
 \itemch{b}
   If \(E\in \frakM_F\) then
   \(E = N\cup K_1\cup K_2 \cup \cdots\), where \(\{K_i\}\)
   is a disjoint countable collection of compact sets and \(\mu(N) = 0\).
\end{itemize}
\end{excopy}


\begin{itemize}
 \itemch{a}
   By \textsc{step~i} of the proof of Riesz Theorem (\cite{RudinRCA80} page~44),
   we know that
   \begin{equation*}
   \mu(E_1\cup E_2) \leq \mu(E_1) + \mu(E_2).
   \end{equation*}
   For the opposite inequality, let \(\epsilon>0\) and pick
   some open set $V$ such that \(E_1\cup E_2\subset V\) and
   \(\mu(V) \leq \mu(E_1\cup E_2) + \epsilon\).
   We now have:
   \begin{eqnarray*}
   \mu(E_1) + \mu(E_2)
    & \leq & \mu(V\cap V_1) + \mu(V\cap V_2) \\
    & = & \mu\left(V\cap (V_1 \cup V_2)\right) \\
    & \leq & \mu(V) \leq \mu(E_1\cup E_2) + \epsilon.
   \end{eqnarray*}
   Since \(\epsilon\) was arbitrary, we have
   \begin{equation*}
   \mu(E_1) + \mu(E_2) \leq \mu(E_).
   \end{equation*}

 \itemch{b}
   By \textsc{step~viii} of the proof of Riesz Theorem
   (\cite{RudinRCA80} page~47), we know that \frakM\ contains all
   Borel sets with finite \(\mu\) measure.
   Now define by induction, \(K_1\) a compact set such that
   \(K_1\subset E\) and \(\mu(K_1) > \mu(E)/2\).

   If \(\{K_i\}_{1\leq i<n}\) are defined, we know that
   \(D_{n-1} = E\setminus \cup_{i<n} K_i \in \frakM_F\) as such
   we can define \(K_n\)
   as a compact set, such that \(K_n \subset D_{n-1}\)
   and \(\mu(K_n) > \mu(D_{n-1}E)/2\).
   We can easily see that
   \begin{equation*}
    \sum_{i=1}^n \mu(K_i) \geq \sum_{i=1}^n 1/2^n = 1 - 1/2^{n+1}.
   \end{equation*}
   Put \(N = E \setminus \cup_{i=1}^\infty K_i\)
   and since \(\mu(E) = \sum_{i=1}^\infty \mu(K_i)\),
   we conclude that \(\mu(N) = 0\).
\end{itemize}

\end{enumerate}

\index{Lebesgue}
In Exercises 5 to 8, $m$ stands for Lebesgue's measure on \(R^1\).
\nobreak
\begin{enumerate}

\setcounter{enumi}{4}

%%%%%%%%%%%%%% 5
\begin{excopy}
Let $E$ be
\index{Cantor}
Cantor's familiar ``middle thirds'' set.
Show that \(m(E) = 0\), even though $E$ and \(\R^1\) have the same cardinality.
\end{excopy}

After removing the thirds on the $n$ step, the set \(C_n\) ``remain'' with
\(m(C_N) = (2/3)^n\). By being in the \(\sigma\)-algebra,
the Cantor set $C$ has a measure
\begin{equation*}
m(C) =
\lim_{n\rightarrow\infty}m(C_n) = \lim_{n\rightarrow\infty}(2/3)^n = 0.
\end{equation*}

For compuing the cardinality, let's consider the ternary representation
of each \(\alpha\in [0,1]\),
\begin{equation*}
 \alpha = \sum_{i=1}^\infty t_i(\alpha) 3^{-i}
    \qquad \textrm{where}\; t_i(\alpha) \in \{0,1,2\}
\end{equation*}
Numbers, except for $0$,  with finite representation
(with \(t_i(\alpha)=0\) for all \(i > N\) for some $N$)
such as
\begin{equation*}
\alpha = \sum_{i=1}^N t_i(\alpha) 3^{-i}
\end{equation*}
where \(0\neq t_N(\alpha) \in \{1,2\}\) ---
also have an infinite representation as in:
\begin{equation*}
\sum_{i=1}^N t_i(\alpha) 3^{-i} =
\sum_{i=1}^{N-1} t_i(\alpha) 3^{-i}
+ (t_N(\alpha)-1) 3^{-N}
+ \sum_{i=N+1}^\infty 2\cdot 3^{-i}
\end{equation*}

In such cases,
we resolve the ambiguity, by choosing:
\begin{itemize}
 \item the infinite representation if \(t_N = 1\).
 \item the finite representation if \(t_N = 2\).
\end{itemize}

Now the Cantor set $C$, is exactly the numbers in \([0,1]\)
whose ternary representation (with the above choice made)
does \emph{not} contain any factor (digit) $1$, that is
\(1 \neq t_i\in \{0,2\}\) for all $i$.

Using \(b_i=t_i/2\in \{0,1\}\), we build
 the following map \(f:C\rightarrow [0,1]\)
\begin{equation*}
f(\alpha)
  = f\left(\sum_{i=1}^\infty t_i(\alpha) 3^{-i} \right)
  = \sum_{i=1}^\infty (t_i(\alpha)/2) 2^{-i}.
\end{equation*}
Maps the Cantor set \emph{onto} the unit interval, using \emph{binary}
representation. This shows directly that the cardinality of $C$ is
the same as that of \([0,1]\) which is known to be \(|\R^1|\).

%%%%%%%%%%%%%% 6
\begin{excopy}
Construct \label{ex:disc:K}
a totally disconnected compact set \(K\subset \R^1\) such that
\(m(K) > 0\).
($K$ is to have no connected subset consisting of more than one point.)

If $v$ is lower semicontinuous  and \(v\leq \chhi_K\), show that actually
\(v \leq 0\). Hence \(\chhi_K\) cannot be approximated by lower semicontinuous
function, in the sense of
\index{Vitaly}
\index{Carath\'eodory}
Vitaly-Carath\'eodory Theorem.
\end{excopy}

Construct the following similar to Cantor set.
Start from the unit interval,
in each step we ``break'' any segment of the previous step,
but (contrary to Cantor's) ensuring that the sum of open segment
taken out is ``much'' less than $1$.

More formally, we start (after step-0), with \(D_0 = [0,1]\).
After step $n$, \(D_n\) is a union of \(2^n\) closed intervals.
In step $n$, from each interval \([a,b]\) of \(D_{n-1}\)
we substract from its center an open sub-interval with length of \(2^{-(2n+1)}\)
Hence in this step we substract a total of
\begin{equation*}
 2^n \cdot 2^{-(2n+1)} = 2^{-n-1}.
\end{equation*}
The total length of open intervals removed by all steps upto step $n$ is
\begin{equation*}
 \sum_{k=1}^n 2^{-k-1} = (1/2) \sum_{k=1}^n 2^{-k} < 1/2
\end{equation*}
Thus \(K = \cap_n D_n\) is compact, \(m(D) \geq 1/2\)
and it is totally disconnected, since any interval eventually gets cut.

Now if $v$ satisfies the assumptions of the exercise, and by negation
\(v(w) > 0\) for some \(w\in \R\) then for \(a = v(w)/2\) then
\(V = v^{-1}(a,\infty)\) is open and \(w\in V\) thus there is an open
interval $I$ for which \(f(x)>a\) for all \(x\in I\),
but since $K$ is totally disconnected, there must be some \(r\in I\setminus K\)
giving a the contradiction \(\chhi(r) = 0\).

%%%%%%%%%%%%%% 7
\begin{excopy}
Given \(\epsilon > 0\), construct an open set \(E\subset [0,1]\) which is dense
in \([0,1]\), such that \(m(E)=\epsilon\).
(To say that $A$ is dense in $B$ means that the closure of $A$ contains $B$.)
\end{excopy}

Of course we should assume \(\epsilon \leq 1\).
Put \(\Q\cap[0,1]\) into q sequence \((q_i)_{i=1}^\infty\).
Now define the following sequence open sets \((G_k)_{k=1}^\infty\).
Given \(k\leq 1\), let \(H_k =  \cup_{j<k} \overline{G_j}\)
(closed, where \(H_1 = \emptyset\)),
let $i$ be the minimal such that \(q_i\notin H_k\).
Pick an open interval \(I_k\) of \(q_i\) with length \(l_k\)
such that \(l_k \leq 2^{-k}\epsilon\)
(Note: by ``open'' we mean here --- open \emph{in} \([0,1]\).
Thus \([0,a)\) and \((b,1]\) are considered open).
If \(l_k < 2^{-k} \epsilon\), also pick some finite number
of open intervals \(\{J_{k,m}\}\) such that
\begin{equation*}
G_k = I_k \cup\,\bigcup_{m=1}^{N_k} J_{k,m}
\end{equation*}
satisfies
\begin{equation*}
m(G_k) = 2^{-k} \qquad \textrm{and} \qquad
  G_k \cap \left(\cup_{j<k} G_j\right) = \emptyset.
\end{equation*}
Let \(G = \cup G_k\), surely \(m(G) = \epsilon\) and \(\Q\cap [0,1] \subset G\),
and so $G$ is dense in \([0,1]\).


%%%%%%%%%%%%%% 8
\begin{excopy}
Construct a Borel set \(E\subset \R^1\) such that
\begin{equation*}
 0 < m(E\cap I) < m(I)
\end{equation*}
for every nonempty segment $I$. Is it possible to have \(m(E) < \infty\)
for such a set?
\end{excopy}


Given \(0<a<1\) we will first build a set \(F\subset [0,1]\) such that
\(m(F) = a\) and
\begin{equation*}
0 < m(F\cap I) < m(I)
\end{equation*}
We will build $F$, and its complement \(G = [0,1] \setminus F\)
as an countable infinite union of Borel sets,
\(F = \cup_n F_n\)
and
\(G = \cup_n G_n\)
with measures \(m(F_i) = 2^{-i}a\) and \(m(G_i) = 2^{-i}(1-a)\).
At each stage,
\begin{equation*}
 R_n = [0,1] \setminus
       \left( \bigcup_{i=1}^n F_i \cup \bigcup_{i=1}^n G_i\right)
\end{equation*}
will be a countable family of intervals ---
close, open are half closed and open. That is, of
any of the forms:
\([a,b]\),
\((a,b)\),
\([a,b)\) or
\((a,b]\).

\paragraph{Building \(F_1\).}
For any interval \(I\subset[0,1]\).
Similar to Exercise~\ref{ex:disc:K} above, we construct a totally disconnected
subset \(F_1\) of [0,1]
whose complement is a countable family of open intervals.
We carefully choose the sizes of the open intervals, so
that \(m(F_1) = a/2\).

\paragraph{Building \(G_1\).}
The set \(F_1^c = [0,1]\setminus G_1\) consists of countably many \(\cal{N}\)
intervals (where \(\cal{N}<\infty\) or \({\cal{N}} = \aleph_0\)).
Within each interval \(I_i\) of \(F_1^c\) we build a totally disconnected
subset \(T_i\), such that the sum of measures of these sets equals \((1-a)/2\).
This is done either by choosing a measure of
\((1-a)/\cal{N}\) if  \(\cal{N}<\infty\),
or a measure of \(2^{i+1}(1-a)\) otherwise.
We let \(G_1= \cup_i T_i\).


\paragraph{Building \(F_n\).}
Assume \(\{F_i\}_{i=1}^{n-1}\)
and \(\{G_i\}_{i=1}^{n-1}\) are built,
The set
\begin{equation*}
 R_{n-1} = [0,1] \setminus
  \left( \bigcup_{i=1}^{n-1} F_i \cup \bigcup_{i=1}^{n-1} G_i\right)
\end{equation*}
consists of countably many intervals.
Within each of these intervals we build a totally disconnected set \(U_{n,i}\)
such that when letting \(F_n = \cup_i U_{n,i}\) be a disjoint union, we have
\begin{equation*}
m(F_n) = m\left(\cup_i U_{n,i}\right) = \sum_i m(U_{n,i}) = 2^{-(n+1)} a
\end{equation*}

\paragraph{Building \(G_n\).}
Assume \(\{F_i\}_{i=1}^{n}\)
and \(\{G_i\}_{i=1}^{n-1}\) are built,
The set
\begin{equation*}
 S_{n-1} = [0,1] \setminus
  \left( \bigcup_{i=1}^n F_i \cup \bigcup_{i=1}^{n-1} G_i\right)
\end{equation*}
consists of countably many intervals.
Within each of these intervals we build a totally disconnected set \(V_{n,i}\)
such that when letting \(G_n = \cup_i V_{n,i}\) be a disjoint union, we have
\begin{equation*}
m(G_n) = m\left(\cup_i V_{n,i}\right) = \sum_i m(U_{n,i}) = 2^{-(n+1)} (1-a).
\end{equation*}

Now $F$ and $G$ are well defined, and \(F\disjunion G = [0,1]\).

Note that the way \(\{F_i\}\) and \(\{G_i\}\) were build,
for each \(x,y\in F\), where \(x<y\), there is some $n$, such that
\(x,y\in \cup_{i\leq n} F_i\) and \(m((x,y) \cap G_i) > 0\).

Similarly,
for each \(x,y\in G\), where \(x<y\), there is some $n$, such that
\(x,y\in \cup_{i\leq n} G_i\) and \(m((x,y) \cap  F_{i+1}) > 0\).

The last observations show
the inequalities \(0 < m(F\cap\, [x,y]\,) < m([x,y])\).
To generalize it, we repeat the process of building $F_i$ for
any interval \(\{[n,n+1]\}_{n\in\Z}\), with \(a = 2^{-|n|}\).
Put \(E = \cup F_i\). Now \(m(E) = \sum_{n\in\Z} = 3 < \infty\)
and the desired inequalities \(0 < m(E\cap\, I\,) < m(I)\)
for and interval $I$ in \R\ is established.




%%%%%%%%%%%%%%
\begin{excopy}
Construct a sequence of continuous function \(f_n\) on \([0,1]\) such that
\(0\leq f_n \leq 1\), such that
\begin{equation*}
 \lim_{n\rightarrow \infty} \int_0^1 f_n(x)dx = 0,
\end{equation*}
but such that the sequence \(\{f_n(x)\}\) converges for no \(x\in[0,1]\).
\end{excopy}

Put \(N_0 =0\) and \(N_k = \sum_{i=1}^{k+1} i = (k+1)(k+2)/2\).
We will define the \(f_n\) functions in $k$-batches.
For each \(k>0\), we  define \(\{f_n:  N_{k=1} < n \leq N_k\}\):
\begin{equation*}
f_n(x) = f_{N_{k-1}+i}(x) = \left\{
 \begin{array}{l@{\quad}c}
   1 - k|x - (i-1)/k| & \textrm{if}\; |x - (i-1)/k|<1/k \\
   0                  & \textrm{Otherwise}
 \end{array}\right.
\end{equation*}
for \(1\leq i \leq N_k\).
We see that \(\int_0^1 f(n) = 1/k\) and so \(\int_0^1 f(n)\to 0\)
but for every $k$-batch, for any \(x\in[0,1]\)
there is some $n$ such that \(N_{k-1}< n \leq N_k\)
and \(f_n(x) \geq 1/2\).


%%%%%%%%%%%%%%
\begin{excopy} % 10
If \label{ex:2:10}
\(\{f_n\}\) is a sequence of continuous functions on \([0,1]\) such that
\(0\leq f_n \leq 1\), and such that
\(f_n(x)\to 0\) as \(n \to \infty\),
for every \(x\in[0,1]\), then
\begin{equation*}
 \lim_{n\rightarrow \infty} \int_0^1 f_n(x)dx = 0,
\end{equation*}
Try to prove this without using any measure theory or any theorem
about Lebesgue integration. (This is to impress you with the power of
the Lebesgue integral. A nice proof was given
\index{Eberlein}
by W.F.~Eberlein in \emph{communications on Pure and Applied Mathematics},
vol.~X, pp.~357-360, 1957.)
\end{excopy}

We will allow ourselves to use some trivial measure concepts
such as the sum of lengths of finite union of intervals.
That is if \(\{I_i\}_{i=1}^n\),
where \(I_i\) are intervals of the form
\((a_i,b_i)\),
\((a_i,b_i]\),
\([a_i,b_i)\) or
\([a_i,b_i]\), and \(I_i \cap I_j = \emptyset\) whenever \(i\neq j\) then
we use
\begin{equation*}
m\left(\Disjunion_{i=1}^n I_i\right) =
\sum_{i=1}^n m(I_i) = \sum_{i=1}^n b_i - a_i.
\end{equation*}

\iffalse
Back to the exercise. Let us define some sequences of subsets of \([0,1]\)
(with nicknames) based on \(\{f_n\}_{i\in\N}\).
\begin{equation*}
\begin{array}{lcl@{\qquad}r}
 U_{n,k} &=&
    \{x\in [0,1]: |f_n(x)| \geq 1/k\}    & \textrm{(Upper)} \\ \\
 T_{n,k} &=&
   \bigcup\limits_{i=n}^\infty U_{i,k}   & \textrm{(Tail)} \\ \\
 R_k     &=&
    \bigcap\limits_{i=1}^\infty T_{i,k}  & \textrm{(Resistance)} \\
\end{array}
\end{equation*}

When looked closely, we see that \(R_k\) consists of all \(x\in[0,1]\)
such that \(|f_n(x)| \geq 1/k\)  for infinitely many $n$.
Thus the assumption that \(f_n(x)\to 0\) for all \(x\in[0,1]\)
means that
\begin{equation*}
\cap_{k\in\N} R_k = \emptyset.
\end{equation*}
\fi

Back to the exercise. Let
\begin{equation*}
 g_n(x) = \sup_{k\geq n} |f_k(x)|
\end{equation*}
Clearly \(\{g_n\}_{n\in\N}\) is a decreasing sequence,
converges pointwise \(g_n(x)\xrightarrow{n\to\infty} 0\) and
\begin{equation*}
 \left|\int_0^1 f_n(x)dx\right|
 \leq  \int_0^1 |f_n(x)|dx \leq  \int_0^1 g_n(x)dx.
\end{equation*}
Hence it is sufficient to show that
\begin{equation} \label{eq:integ:gto0}
\lim_{n\to\infty} \int_0^1 g_n(x)dx = 0.
\end{equation}

Let \(\epsilon>0\) be arbitrary, and define
\begin{equation*}
 L_n = L_{n,\epsilon} = \{x\in[0,1]: |f_n(x) < \epsilon\}.
\end{equation*}
By definition and assumptions, \(\{L_n\}_{n\in\N}\) is an increasing
sequence of open sets with union
\begin{equation} \label{eq:ULn}
\bigcup_{n\in\N} L_n = [0,1].
\end{equation}
Each \(L_n\) is a union of countably many open intervals,
\begin{equation*}
 L_n = \bigcup_{k\in \N} I_{n,k} = \bigcup_{k\in \N} (a_{n,k}, b_{n,k})
 \qquad \textrm{for each}\; n\in\N.
\end{equation*}

The (still limited to intervals) total measure
\begin{equation}
 M_n = M(L_n) =\sum_{k\in \N} m(I_{n,k})
\end{equation}
is clearly an increasing sequence (by simply looking at sub-finite sums).
We will now show that it
converges to the measure of the unit \(m([0,1])\), that is:
\begin{equation} \label{eq:Lsubint:to1}
 \lim_{n\to\infty} \sum_{k\in \N} m(I_{n,k}) = 1.
\end{equation}
By negation, assume
\begin{equation*}
 \lim_{n\to\infty} M_n = \beta < 1.
\end{equation*}

For any sub-intervals \(I,[a,b]\subset [0,1]\)
trivially
\begin{equation*}
m(I\cap[a,b]) = m(I\cap[a,(a+b)/2]) + m(I\cap[(a+b)/2, b]).
\end{equation*}
Similarly
\begin{equation} \label{eq:MLn}
 M(L_n \cap [a,b]) =  M(L_n \cap [a,(a+b)/2]) + M(L_n\cap[(a+b)/2, b])
\end{equation}
Starting with \(a_0=0\), \(b_0=1\), and recursively bisect \([a_i,b_i]\),
we get a decreasing sequence of closed sub-segments with sizes
\(m([a_i,b_i]) = 2^{-i}\),
such that
\begin{equation*}
 M(L_n \cap [a_i,b_i]) \leq 2^{-i}\beta < 2^{-i}
\end{equation*}
for all \(n\in\N\).
This can be done, by choosing the smaller of the two choices
in each bisection step, and using (\ref{eq:MLn}).
Let $b$ be the intersection point \(\{b\} = \cap_{i\in\N} [a_i,b_i]\).
But by (\ref{eq:ULn}), there exists $n$, such that \(b\in L_n\).
Since \(L_n\) is open, there is some $i$ such that \(b\in[a_i,b_i]\subset L_n\).
Now
\begin{equation*}
M(L_n \cap [a_i,b_i]) = M([a_i,b_i]) = 2^{-i}
\end{equation*}
gives a contradiction and thus (\ref{eq:Lsubint:to1}) is true.

Let \(N<\infty\) be such that
\begin{equation*}
M_N = \sum{k\in \N} m(I_{N,k}) > 1 - \epsilon
\end{equation*}
and \(K<\infty\) be such that
\begin{equation*}
M_N = \sum{k=1}^K m(I_{N,k}) > 1 - 2\epsilon.
\end{equation*}
Set
\begin{eqnarray*}
  D & = & \cup{k=1}^K I_{N,k} \\
  U & = & [0,1] \setminus D
\end{eqnarray*}
Both $D$ and $U$ are unions of finite number of intervals.
Since \(m(D) > 1 - 2\epsilon\), the complement has \(m(U) < 2\epsilon\).
Now we compute:
\begin{equation*}
\int_0^1 g_N(x)dx = \int_D g_N(x)dx + \int_U g_N(x)dx
 \leq (1-2\epsilon)\epsilon + 2\epsilon\cdot 1 < 3\epsilon.
\end{equation*}
Since \(\epsilon>0\) was arbitrary and
\(\{g_n\}_{n\in\N}\) is a decreasing sequence,
the convergence of (\ref{eq:integ:gto0}) is shown.


%%%%%%%%%%%%%% 11
\begin{excopy}
Let \(\mu\) be a regular Borel measure on a compact Hausdorff space $X$:
assume \(\mu(X) = 1\). Prove that there is a compact set \(K\subset X\)
\index{carrier}
\index{support!measure}
(the \emph{carrier} or \emph{support} of \(\mu\))
such that \(\mu(K) = 1\) but \(\mu(H)<1\)
for every proper compact subset $H$ of $K$.
\emph{Hint}: Let $K$ be the intersection of all compact \(K_\alpha\) with
\(\mu(K_\alpha)=1\);
show that every open set $V$ which contains $K$ also contains some \(K_\alpha\).
Regularity of \(\mu\) is needed; compare Exercise~18.
Show that \(K^c\) is the largset open set in $X$ whose measure is $0$.
\end{excopy}

If both \(K_1\) and \(K_2\) are compact then
\(K_1 \cap K_2\) is compact. If
\(\mu(K_1) = \mu(K_2) = 1\), then
\(mu(X\setminus K_1) = \mu(X\setminus K_2) = 0\) and so
\begin{equation*}
\mu(K_1 \cap K_2) \geq \mu(X) - (\mu(X\setminus K_1) + \mu(X\setminus K_2) =
        1 - 0 = 1.
\end{equation*}
By induction \(\mu(\cap_{i=1}^n K_i = 1\) for finite intersection.

Let $K$ be the intersection as in the hint.
Let $V$ be an open set such that \(\cap_\alpha K_\alpha \subset V \subset X\).
Now \(V^c \subset \cup_\alpha K_\alpha^c\), this is an open covering
of the compact \(V^c\). Hence there is a sub-finite covering,
that is we have \(\{\alpha_i\}_{i=1}^m\) such that
\(V^c \subset \cup_{i=1}^m K_{\alpha_i}^c\), equivalently
\(L = \cap_{i=1}^m K_{\alpha_i} \subset V\). By our initial remarks,
$L$ is compact and \(\mu(L)=1\) thus \(L=K_\alpha\) for some \(\alpha\).

By regulartiy,
\begin{equation*}
\mu(K) = \inf \{\mu(V): V\;\textrm{is open, and }\; K\subset V\}
       \geq \mu(K_\alpha) = 1.
\end{equation*}
Thus \(\mu(K) = 1\) and by construction, $K$ is minimal compact with
such measure.


%%%%%%%%%%%%%% 12
\begin{excopy}
Show \label{ex:2:12}
that every compact set of \(\R^1\) is the support of a Borel measure.
\end{excopy}

Let \(K\subset\R^1\) be a compact set, with the inherited topology.
For any function \(f\in C_c(K) = C(K)\) we will define an extension
\(\tilde{f} \in C_c(\R)\). By $K$ being bounded, we can pick
\(b_0 < \min(K)\) and \(b_1 > \max(K)\).
Denote the compact \(\tilde{K} = K \cup \{b_0,b_1\}\)
and the complement open set \(G=\R\setminus \tilde{K}\).
For any \(x\in G\) there is a maximal open interval \((a_x,b_x) \subset $G$\)
whose endpoints \(a_x,b_x\in \tilde{K}\). Define:
\begin{equation*}
 \tilde{f}(x) = \left\{\begin{array}{l@{\qquad}l}
                  f(x) & x \in K \\
                  0    & x \leq b_0 \quad \textrm{or} \quad b_1 \leq x \\
                  \left(\tilde{f}(b_x) - \tilde{f}(a_x)\right)
                  \frac{x - a_x}{b_x - a_x}
                  + \tilde{f}(a_x)
                    & x \in (a_x,b_x)\subset G \quad\textrm{and}\quad
                      a_x,b_x \in \tilde{K}
                  \end{array}\right.
\end{equation*}
Note the ``pseudo'' recursive definition is fine, since
only \(\tilde{f}(b_x)\) and  \(\tilde{f}(a_x)\) are used
and for them \(\tilde{f}\) is well defined on previous cases.
The mapping \(f\to\tilde{f}\) is clearly linear mapping of \(C(K) \to C_c(\R)\).
We now define a positive functional on \(C(K)\)
\begin{equation*}
 \Lambda(f) = \int_{\R} \tilde{f}\,dm =  \int_{b_0}^{b_1} \tilde{f}\,dm
\end{equation*}
where $m$ is the regular Lebesgue's measure on \(\R\).

\index{Riesz}
From Riesz Theorem~2.14, there is Borel measure \(\mu\) on $K$
such that for all \(f\in C(K)\)
\begin{equation*}
 \int_{\R^1} f\,d\mu = \Lambda f.
\end{equation*}

We will now show that $K$ is the support of \(\mu\).
By negation, let \(H\subsetneq K\) be a compact such that
 \(\mu(H) = \mu(K)=1\), hence \(\mu(K\setminus H) = 0\).
Pick \(x\in K\setminus H\). By being a compact Hausdorff space,
there exists an open set $V$ such that \(x\notin V\) and \(H\subset V\).
\index{Urysohn's lemma}
By Urysohn's Lemma~2.12, we can find a continuous function \(g\in C(K)\)
such that \(\{x\} \prec g \prec V\).
Now
\begin{equation} \label{eq:Ksupp:mu}
 \Lambda g = \int_K g\,d\mu = \int_H g\,d\mu + \int_{K\setminus H} g\,d\mu
           = \int_H 0\,d\mu + \int_{\emptyset} g\,d\mu = 0+0 = 0.
\end{equation}
But \(\tilde{g} \geq 0\) is continuous, with \(\tilde{g}(x) = g(x) > 0\),
hence
\(\Lambda g = \int_{\R} \tilde{g}dm > 0\) a contradiction
to (\ref{eq:Ksupp:mu}).


%%%%%%%%%%%%%% 13
\begin{excopy}
Is it true that every compact subset of \(\R^1\) is a support of a continuous
functions? If not, can you describe the class of all compact sets in
\(\R^1\) whoch are supports of continuous functions?
Is your description valid in other topological spaces?
\end{excopy}

The first claim is false. Take a any singleton, say \(\supp f = \{a\}\)
then \(f(a) \neq 0\), by continuity, using a positive \(\epsilon < |f(a)|\)
we can find a \(\delta>0\) such that
\begin{equation*}
 \supp f \supset (a-\delta, a + \delta)
         \subset \{x\in\R: |f(x)| > |f(a)| - \epsilon > 0\}.
\end{equation*}


Here is the classification:
\begin{llem}
A compact \(K\in\R\) is a support of some \(f\in C(\R)\)
iff \(K = \overline{\inter{K}}\) (equals the closure of its interior).
\end{llem}
\begin{thmproof}
Let $K$ be  a compact $K$ set.

Assume \(K = \supp f\), for some \(f\in C(\R)\).
For any closed set $K$, (in particular compact),
\(K \subset \overline{\inter{K}}\).
If by negation \(K \subsetneq \overline{\inter{K}}\), then we can find
\(x \in K \setminus \overline{\inter{K}}\).
If \(f(x) \neq 0\) then there is a neighborhood $V$ of $x$
such that \(x\in V \subset \supp f\) which gives the contradiction
\(x\in \inter{K}\). Thus \(f(x) = 0\) but then
\(K = \supp f \subset \overline{\inter{K}}\) gives a contradiction.

Conversely, assume \(K = \overline{\inter{K}}\).
Since \(G = \inter{K} = \cup_{i\in\N} I_i\), is a countable union
of open  intervals \(I_i = (a_i, b_i)\), we denote their
length  \(l_i = b_i - a_i\) and
center
\(c_i = (a_i + b_i)/2\) to define
\begin{equation*}
f(x) = \left\{\begin{array}{l@{\qquad}l}
               l_i/2 - |x - c_i| &  a_i \leq x \leq b_i \\
               0 & x \notin K
              \end{array}\right.
\end{equation*}
Now clearly $f$ is continuous and \(\supp f = K\).
\end{thmproof}

From the proof, we can see that the classification
can extend to any space $X$ where any open set $G$ is a union
of open sets \(\{V_\alpha\}_{\alpha\in I}\) with disjoint closures
such that for each \(V_\alpha\) there is a function \(f\in C(X)\)
such that \(K \prec f \prec V\) for any compact \(K\subset V\).
This is true for \(\R^n\).

%%%%%%%%%%%%%% 14
\begin{excopy}
If $f$ is a Lebesgue measurable complex function on \(\R^1\),
prove that there is a Borel function $g$ on \(\R^1\) such that
\(f=g\)~a.e.[$m$].
\end{excopy}

With the understanding (similar to the definition
\index{Borel}
of \emph{Borel function} --- see:
\cite{RudinRCA80} page~13)
that
\index{Lebesgue!measurable function}
\(f:X\to \C\) is a \emph{Lebesgue measurable function} iff
\(f^{-1}(V)\) is Lebesgue measurable for any open set \(V\subset X\).

We will build a subset \(M \subset \R\) with \(m(M) = 0\)
and define
\begin{equation} \label{eq:g:Borel}
g(x) = \left\{\begin{array}{l@{\qquad}l}
             f(x) & x\notin M \\
             0    & x\in M
             \end{array}\right..
\end{equation}

By Theorem~2.20 for any Lebesgue measurable set $L$, we can find
Borel (\(F_{\sigma}\) set \(B\subset L\) such that \(m(L\setminus B) = 0\).

Enumerate all open rectangles with rational boundaries
\({\cal{R}} = \{R_i\}_{i\in\N}\)
where
\begin{equation*}
 R_i = \{z\in\C : r_0\leq \Re(z) \leq r_1 \;\wedge\;
                  q_0\leq \Im(z) \leq q_1 \;\wedge\;r_0,r_1,q_0,q_1\in\Q\}.
\end{equation*}

Now \(L_i = f^{-1}(R_i)\) is a Lebesgue measurable function.
Let \(B_i\subset L_i\) be a Borel measurable set
(as on the preceding remark) such that and \(m(L_i\setminus B_i) = 0\).
We define \(M = \cup_{i\in\N} (L_i\setminus B_i)\).
By \(\sigma\)-additivity, \(m(M)=0\) as requried.
Now, it is left to show
that $g$ defined in (\ref{eq:g:Borel}) is a Borel measurable function.

Any (origin-less) open set \(G\subset \C\setminus\{0\}\)
can be represented as a countable union of intervals from \(\cal{R}\),
say
\begin{equation*}
G = \cup_{i\in\N} R_{k_i}
\end{equation*}
Now computing the inverse images
\begin{eqnarray*}
 g^{-1}(G)
 &=&  f^{-1}(G) \setminus M \\
 &=& f^{-1}\left(\cup_{i\in\N} R_{k_i}\right) \;\setminus M
     = \bigcup_{i\in\N} f^{-1}( R_{k_i}) \;\setminus M
     = \bigcup_{i\in\N} L_{k_i} \;\setminus M \\
 &=& \left(\bigcup_{i\in\N} B_{k_i}
     \disjunion \left(L_{k_i}\setminus B_{k_i}\right)\right) \;\setminus M \\
 &=& \left(\bigcup_{i\in\N} B_{k_i}
     \,\cup\, \bigcup_{i\in\N}  \left(L_{k_i}\setminus B_{k_i}\right)
     \right) \;\setminus M \\
 &\subset&  \left(\bigcup_{i\in\N} B_{k_i} \cup M'\right) \;\setminus M \\
 &=& \bigcup_{i\in\N} B_{k_i}.
\end{eqnarray*}
Where
\begin{equation*}
 M' = \bigcup_{i\in\N}  \left(L_{k_i}\setminus B_{k_i}\right) \subset M.
\end{equation*}
So \(g^{-1}(G)\) is a Borel set. In particular \(g^{-1}(\C\setminus\{0\})\)
is Borel set, and by \calB\ being \(\sigma\)-algebra,
the complement \(g^{-1}(\{0\})\) is a Borel set as well.


%%%%%%%%%%%%%% 15
\begin{excopy}
It is easy to gueass the limits of
\begin{equation*}
 \int_0^\pi \left(1 - \frac{x}{n}\right)^n e^{x/2}\, dx
 \quad\textrm{and}\quad
 \int_0^\pi \left(1 + \frac{x}{n}\right)^n e^{-2x}\, dx.
\end{equation*}
as \(n\to\infty\). Prove that your guesses are correct.
\end{excopy}

\begin{itemize}

\item
We compute the limit
\begin{equation*}
\lim_{n\to\infty} \left(1 - \frac{x}{n}\right)^n e^{x/2}
=  e^{x/2} \lim_{n\to\infty} \left(1 - \frac{x}{n}\right)^n \\
=  e^{x/2} e^{-x} = e^{-x/2}.
\end{equation*}

The integrand \(\left(1 - \frac{x}{n}\right)^n e^{x/2}\)
is bounded on \([0,\pi]\) by  \(1\cdot e^{\pi/2}\).
Using \(\int e^{-x/2}\,dx = -2e^{-x/2} + c\) and
Lebesgue Dominated Theorem
\begin{equation*}
 \lim_{n\to\infty} \int_0^\pi \left(1 - \frac{x}{n}\right)^n e^{x/2}\, dx =
 \int_0^\pi e^{-x/2}\,dx = -2e^{-\pi/2} - (-2e^{-0/2}) = -2e^{-\pi/2} - 2.
\end{equation*}

\item
We compute the limit
\begin{equation*}
\lim_{n\to\infty} \left(1 + \frac{x}{n}\right)^n e^{-2x}
=  e^{-2x} \lim_{n\to\infty} \left(1 + \frac{x}{n}\right)^n \\
=  e^{-2x} e^x = e^{-x}.
\end{equation*}

The integrand \(\left(1 + \frac{x}{n}\right)^n e^{-2x}\)
is bounded on \([0,\pi]\) by  $2$.
Using \(\int e^{-x}\,dx = -e^{-x} + c\) and
Lebesgue Dominated Theorem
\begin{equation*}
 \lim_{n\to\infty} \int_0^\pi \left(1 + \frac{x}{n}\right)^n e^{-2x}\, dx =
 \int_0^\pi e^{-x}\,dx = -e^{\pi} - (-e^0) = -e^{-\pi/2} + 1.
\end{equation*}

\end{itemize}


%%%%%%%%%%%%%%
\begin{excopy}
If $m$ is a Lebesgue measurable on \(\R^k\), prove that \(m(-E) = m(E)\),
where
\begin{equation*}
 -E  = \{-x: x\in E\},
\end{equation*}
and hence that
\begin{equation*}
 \int_{\R^k} f(x)dx = \int _{\R^k} f(-x)dx =
\end{equation*}
for all \(f\in L^1(\R^k)\).
\end{excopy}

For any $k$-cell $W$, we have
\begin{equation*}
m(W) = \vol(W) = \prod_{i=1}^k (\beta_i - \alpha_i) =
\prod_{i=1}^k ((-\alpha_i) - (-\beta_i)) = \vol(-W) = m(W).
\end{equation*}
Since $E$ is a limit of countable union of $k$-cells
\(m(E) = m(-E)\) follows.
The equality of integration is now simply matter of mapping \(\R^k \to \R^k\)
and a change of variable.

%%%%%%%%%%%%%% 17
\begin{excopy}
Let $X$ be the plane, with the following topology: A set is open
if and only if its intersection with every vertical line is an open subset
of that line,
with respect to the usual topology of \(\R^1\).
Show that this $X$ s a locally compact Hausdorff space.
If \(f\in C_c(X)\), let \seqxn\ be those values of $x$
for which \(f(x,y)\neq 0\) for at least one $y$
(there are only finitely many such $x$!), and define
\begin{equation*}
\Lambda f = \sum_{j=1}^n \int_{-\infty}^\infty f(x_j,y) dy.
\end{equation*}
Let \(\mu\) be the measure associated with this \(\Lambda\) by Theorem~2.14.
If $E$ is the $x$-axis, show that \(\mu(E) = \infty\) although
\(\mu(K) = 0\) for every compact \(K\subset E\).
\end{excopy}

For any two distinct point \(P_i = (x_i,y_i)\in \R^2 = X\) where \(i=1,2\)
we present open sets \(V_1\) and \(V_2\) that separate the points.
If \(x_1\neq x_2\), then we take  \(V_i = \{x_1\}\times\R\),
otherwise \(y_1\neq y_2\) and we take
\begin{equation*}
V_i = \{(x_i,y)\in\R^2: |y-y_i| < |y_1-y_2|/2\}.
\end{equation*}

For Any point \(P=(x,y)\in\R^2=X\) and any neighborhood $V$ of $P$,
we can find \(\epsilon>0\) such that
\begin{equation*}
 G = \{(x,v)\in\R^2: |v-y| < \epsilon\} \subset
\overline{G} = \{(x,v)\in\R^2: |v-y| \leq \epsilon\}  \subset V
\end{equation*}
and clearly \(\overline{G}\) is compact,

Now that we have shown that $X$ is locally compact Hausdorff space,
we proceed. Let \(f\in C_c(X)\).
Let \(W\subset \R\)  be of all
\(x\in W\) such that there exist $y$ such that \(f(x,y)\neq 0\).
If by negation $W$ is inifinite, then
the family of open sets
\begin{equation*}
 \{\,\{x\}\times\R\, : x\in W\}
\end{equation*}
is a covering of \(\supp f\) which has no finite sub-covering.
This contradicts the assumption that \(f\in C_c(\R)\).

By looking at vertical finite segments
\(S = \{(x,y)\in\R^2: x=x_0 \wedge y_0\leq y\leq y_1\}\),
clearly \(\Lambda \chhi_S = y_1 - y_0\) hence \(\mu(S) = y_1-y_0\).
Let $E$ is the $x$-axis. The last assertion is also true for any
non empty sub interval \([a,b]\subset E\).
Any open super set \(V \supset [a,b]\) contains a continuum
of vertical segments and hence \(\mu(V) = \infty\).
By Theorem~2.14(c) \(\mu([a,b] = \infty\).

The inifinite subsets of $E$
are covered by inifinitely many vertical open segments
with no sub-finite sub-covering.
Hence, the compact subsets of $E$ are exactly the finite subsets.
Clearly the measure of a singleton \(\mu(\{(x,y)\}) = 0\),
and thus \(\mu(K)=0\) for any compact \(K\subset E\).



%%%%%%%%%%%%%% 18
\begin{excopy}
This exercise requires more set-theoretic skill than the preceding
ones.  Let $X$ be a well-ordered uncountable set which has a last
element \(\omega_1\), such that every predecessor of \(\omega_1\) has
at most countably many predecessors.
(``Construction'': Take any well-ordered set which has elements with uncountably
many predecessors, and let \(\omega_1\) be the first of these;
\(\omega_1\) is called the first uncountable ordinal.)
For \(\alpha\in X\), let \(P_\alpha\) [\(S_\alpha\)] be the set of all
predecessors (successors) of \(\alpha\),
and call a subset of $X$ open if it is
a \(P_\alpha\) or an \( S_\beta\)
or an  \(P_\alpha \cap S_\beta\)
or a union of such sets.
Prove that $X$ is then a compact Hausdorff space.
(\emph{Hint}: No well-ordered set contains an infinite decreasing sequence.)

Prove that the complement of the point \(\omega_1\) is an open set
\index{sigma-compact@\(\sigma\)-compact}
which is not \(\sigma\)-compact.

Prove that to every \(f\in C(X)\)  there corresponds an \(\alpha \neq \omega_1\)
such that $f$ is constant on \(S_\alpha\).

Prove that the intersection of every countable collection \(\{K_\alpha\}\)
of uncountable compact subsets of $X$ is uncountable.
(\emph{Hint}: Consider limits of increasing countable sequences in $X$ which
intersects each \(K_n\) in infinitely many points.)

Let \frakM\ be the collection of all \(E\subset X\) such that either
\(E\cup \{\omega_1\}\) or
\(E^c\cup \{\omega_1\}\) contains an uncountable compact set; in the first case,
define \(\lambda(E)=1\); in the second case, define \(\lambda(E) = 0\).
Prove that \frakM\ is a \(\sigma\)-algebra which contains all Borel sets
in $X$, that \(\lambda\) is a measure on \frakM\ which is \emph{not}
regular (every neighborhood of \(\omega_1\) has measure $1$), and that
\begin{equation*}
 f(\omega_1) = \int_X f\, d\lambda
\end{equation*}
for every \(f\in C(X)\). Describe the regular \(\mu\) which Theorem~2.14
associates with this linear functional.
\end{excopy}


\paragraph{Hausdorff.} Let \(\alpha<\beta\) be any two points in $X$.
Clearly \(\alpha+1\leq \beta\) and then  \(P_{\alpha+1}\) and \(S_{\alpha}\)
are open sets that separate the two points.

\paragraph{Compactness.} Let \(\{G_i\}_{i\in I}\) be a family of open
sets such that \(X=\cup_{i\in I}G_i\). By negation, assume that
there is no finite sub-covering. We build
\begin{itemize}
\item A sequence of sets
     \(\{G_{i_j}: j\in\{0\}\cup \N\;\textrm{and}\; i_j \in I\}\).
\item A strictly decreasing sequence of points \(\{\alpha_j\}_{i\in\N}\)
\end{itemize}
by induction.
For notational convenience we use ``\(G_j=G_{i_j}\)''
and we denote \(U_n = \cup_{j=0}^n G_j\).
Pick \(G_0\) such that \(\omega_1\in G_0\).

By looking at base neighborhoods, we see that
each non empty open set (except for \(\{1\}\)).
contains a
non empty open set of the form \(S_\alpha\) or \(S_\alpha \cap P_\beta\)

In step $n$, let \(\alpha_n\) be the least point such that
\(S_{\alpha_j} \subset U_{n-1}\).
or \(S_{\alpha_j}\cap P_\beta \subset U_{n-1}\) for some
\(\beta \geq \alpha+1\).
Note that \(\alpha_j \notin \cup_{j<n} G_j\). Now we pick \(G_n\) such that
\(\alpha_n \in G_n\).
By construction, there is a base neighborhood of \(\alpha_1\)
that is contained in \(U_n\). Thus \(\{\alpha_j\}_{i\in\N}\) is strictly
decreasing. Hence has no minimal element in contradiction to $X$ being
well ordered.


\paragraph{Non \(\sigma\)-compactness of \(X\setminus\{\omega_1\}\).}
Let \(Y= X\setminus\{\omega_1\}\). Now
\(\{P_y: y\in Y\}\) is an open covering of $Y$.
By exercise's, assumption \(|P_y| \leq \aleph_0\) for all \(y\in Y\).
If by negation $X$ is \(\sigma\)-compact, then be looking at the sub-compacts
\(X=\cup_{i\in\N} K_n\), the finite sub-covering of each and finally
joining the sub-coverings to a countable sub-covering
\begin{equation*}
 Y = \bigcup_{i\in N} P_{y_i}.
\end{equation*}
But then \(|Y| \leq |\N|\cdot\aleph_0 = \aleph_0\)
contradiction to the fact \(|Y|=|X|>\aleph_0\).

\paragraph{Constant Tail.}
Indeed some ``set-theoretic skill'' is needed.
Here we use some terms and results from \cite{Dug1966}.
\paragraph{Definitions} (\cite{Dug1966} \textsf{II 3.1}).
Let $W$ be a well ordered set
\index{ideal!set}
\index{initial interval}
The set of ideals \(I(W)\),
an initial interval \(W(a)\) for each \(a\in W\))
and the set of initial interval \(J(W)\)
are defined as follows:
\begin{eqnarray*}
I(W) &=& \{S\subset W: \forall x\in S\wedge y < x \Rightarrow y\in S \\
W(a) &=& \{x\in W: x < a \wedge x\neq a\}  \\
J(W) &=& \{W(a): a\in W\}
\end{eqnarray*}

%%%%%%%%%%%%%%%%%%%%%%%%%%%%%%%%
\begin{llem} \label{llem:set:ideals}
\textnormal{(\cite{Dug1966} \textsf{II 3.2(a)})}
Let $W$ be a well ordered set, then \(J(W) = I(W) \setminus \{W\}\).
\end{llem}
\begin{thmproof}
\(J(W) \subset I(W) - \{W\}\): Each initial interval is obviously an ideal
but not the whole $W$.
Conversely, let \(S\in I(W) \setminus \{W\}\) Then
\(U = W\setminus S\neq \emptyset\), so $U$ has a first element $a$.

We now show that \(W(a)=S\).
If \(x\in W(a)\), and if by negation \(x notin S\) then \(x\in U\)
contradiction to \(a=\min(U)\) since \(x<a\).
Otherwise \(x\notin W(a)\), but then \(a\leq x\) and so \(x\notin S\)
for if \(x\in S\) by being ideal we would have \(a\in S\).
\end{thmproof}

%%%%%%%%%%%%%%%%%%%%%%%%%%%%%%%%
\begin{llem} \label{llem:countable:ub}
\textnormal{(\cite{Dug1966} \textsf{II 9.1})}
Each countable subset of the ordinals \([0,\omega_1)\) has an upper bound in
 \([0,\omega_1)\).
\end{llem}
\begin{thmproof}
Let \(L = [0,\omega_1) \) the set of all ordinals less than \(\omega_1\).
Let \(A\subset L\) and let $S$ be the ideal
\begin{equation} \label{eq:UWa}
  S =
  \bigcup_{\alpha\in A} W(\alpha) = \bigcup_{\alpha\in A} \{x\in L: x< \alpha\}
  \subset L.
\end{equation}
Since the cardinality of the ordinal \(\omega_1\) is not equal
to any smaller ordinal, \(|W(\alpha)| \leq \aleph_0\)
for each \(\alpha < \omega_1\).
So the above \(\ref{eq:UWa}\) is a countable union of countable sets.
% Since \(\aleph_0\cdot \aleph_0 = \aleph_0\) using
\begin{equation*}
 |S| = \aleph_0 \cdot  \aleph_0 = \aleph_0 < \aleph_1 = |L|.
\end{equation*}
Consequently $S$ cannot be isomorphic to $L$. By lemma~\ref{llem:set:ideals},
\(S=[0,\beta)\) for some \(\beta < \omega_1\) and \(\beta\) is
the least upper bound of $A$.
\end{thmproof}

The following lemma solves the ``constant tail'' statement.
Note that in \cite{Dug1966} the underlying  space does \emph{not}
contain (``the last'') \(\omega_1\).

\index{Vickery}
\begin{llem} \textnormal{(\cite{Dug1966} \textsf{III 8.4 Ex.7}
                         \textrm{Vickery})}
 \label{llem:Vickery}
With the above interval topology defined above on \([0,\omega_1]\) and
induced on \([0,\omega_1)\),
If \(f\in C([0,\omega_1])\)
then $f$ is continuous on a tail \([\beta,\omega_1]\).
\end{llem}
\begin{thmproof}
We first assert that
\begin{equation} \label{eq:f:omega:Cauchy}
 \forall n\in\N,
 \exists \alpha_n < \omega_1\,
 \forall \xi \in (\alpha_n,\omega_1)\;
         |f(\xi) - f(\alpha_n)| < 1/n.
\end{equation}
Intuitively --- $f$ behaves like a Cauchy sequence.
By negation, assume
\begin{equation*}
 \exists n_0\in\N,
 \forall \alpha < \omega_1\,
 \exists \xi \in (\alpha,\omega_1)\;
         |f(\xi) - f(\alpha)| \geq 1/n_0.
\end{equation*}
Now we build by induction an increasing sequence \(\{\xi_i\}_{i\in \N}\)
such that \(|f(\xi) - f(\alpha)| \geq 1/n_0\).
In the  \((i+1)\)-th step of the induction we look for the first \(\xi_{i+1}\)
within \((\xi_i,\omega_1)\) that satisfies the hypothesis.
By lemma~\ref{llem:countable:ub}, \(\xi_{i+1}\) have a least upper bound
\(\gamma < \omega_1\). But then $f$ would not be continuous at \(\gamma\).
This proves (\ref{eq:f:omega:Cauchy}).


Now let \(\beta\) be an upper bound of \(\{\alpha_i\}_{i\in \N}\).
Again by lemma~\ref{llem:countable:ub} \(\beta<\omega_1\).
Now if \(\zeta\in(\beta,\omega_1)\), then
\begin{equation*}
|f(\zeta) - f(\beta)| \leq
|f(\zeta) - f(\alpha_n)| + |f(f(\beta) - f(\alpha_n)| = 2/n
\end{equation*}
for every $n$, thus \(f(\zeta) = f(\beta)\), for every \(\zeta\) such that
\(\beta < \zeta < \omega_1\).
As for last point, If by negation \(f(\omega_1)\neq f(\beta)\) then we could
find a base neighborhood interval \(I=S_{\delta}\) of \(\omega_1\)
such that for every \(x\in I\), we have
\begin{equation*}
|f(x)-f(\omega_1)| < |f(\omega_1) - f(\beta)| / 2 > 0
\end{equation*}
But then there must be an \(x\in I\setminus\{\omega_1\}\)
for which we know that \(f(x)=f(\beta)\).
Thus $f$ is constant on \(S_\beta\).
\end{thmproof}

An immediate consequence is:
\begin{llem}
With the above interval topology defined above on \([0,\omega_1]\),
If \(f\in C([0,\omega_1])\)
then $f$ is continuous on a tail \([\beta,\omega_1]\).
\end{llem}
\begin{thmproof}
With such \(f\in C([0,\omega_1])\),
put \(\tilde{f} = f_{|[0,\omega_1)}\in C([0,\omega_1))\).
Applying lemma~\ref{llem:Vickery}, we see that \(\tilde{f}\)
is constant on a tail. By continuity \(f(\omega_1)\) has this tail value,
and so $f$ is constant on a tail.
\end{thmproof}


\paragraph{Intersection of Compacts.}
Let \(\{K_n\}_{n\in\N}\) a countable family of uncountable compact sets
in \([0,\omega_1]\). We note that \(\omega_1\in K_n\) for all \(n\in\N\)
since otherwise each of their upper bound would imply countable cardinality.
Let \(K = \bigcap_{n\in\N} K_n\) It is clealy compact.
Let \(K'=K\setminus \{\omega_1\}\).
Assume by negation $K$ that is countable, and so is \(K'\).
Let $b$ be a an upper bound for \(K'\) whose existence
is implied by lemma~\ref{llem:countable:ub}.

We will build an strictly increasing sequence \(\{a_n\}_{n\in\N}\)
in \([0,\omega_1)\). We look at blocks \(B_k\) of indices of increasing size.
Let
\begin{eqnarray*}
b_0 &=& 0 \\
b_k &=& \sum_{i=1}^k i = k(k+1)/2 \qquad \textrm{for}\; k\in\N \\
B_k &=& \{m\in\N: b_{k-1} < m \leq b_k\}
\end{eqnarray*}
Note that \(\N = \disjunion_{k\in\N} B_k\). Hence
for each \(n\in\N\) there are unique \(k(n)\in\N\) such that \(n\in B_{k(n)}\).
We denote the position of $n$ within \(B_k(n)\)
by \(j(n) = n - b_{k(n)-1}\).

We now define the desired sequence by induction.
\begin{eqnarray}
a_1 &=& \min(K_1) \cap [b+1,\omega_1)  \label{eq:a1K1} \\
a_n &=& \min\left(K_{j(n)} \cap [a_{n-1},\omega_1)\right). \notag
\end{eqnarray}

The sequence \(\{a_n\}_{n\in\N}\) intersects each of the
compact sets \(\{K_n\}_{n\in\N}\) in infinitely many points.
Again by lemma~\ref{llem:countable:ub},
The limit \(u = \lim_{n\in\N} a_n < \omega_1\).
This $u$, must be the same limit of all sub-sequences, in particular
intersection of \(\{a_n\}_{n\in\N}\) with \(K_i\). Thus \(u\in K_i\),
hence \(u\in K'\), but by construction (\ref{eq:a1K1})
we have a contradiction \(u>b\).

\paragraph{The \(\lambda\) measure}
Let \frakM\ be defined as in the exercise. Clearly by definition
it is closed under completion, and it contains $X$ and \(\emptyset\).
To show it is a \(\sigma\)-algebra we need to show it is closed
under countable union. Let \(A_n\in\frakM\) for \(n\in\N\) and
let \(A=\cup_{n\in\N}A_n\). If there exists \(n\in\N\) such that
\(A_n\cup\{\omega_1\}\) contains an uncountable compact subset,
then so does $A$.  Otherwise, for every \(n\in\N\) the sets
\(A_n^c\cup\{\omega_1\}\) each contains an uncountable compact subset.
By previous result so is their intersection
\begin{equation*}
 \bigcap_{n\in\N} A_n^c\cup\{\omega_1\} =
 \left(\bigcap_{n\in\N} A_n^c\right)\cup\{\omega_1\} =
 A^c\cup\{\omega_1\}.
\end{equation*}
Hence, in both cases \(A\in\frakM\).

A closed interval \[a,b\] is uncountable iff \(a<b=\omega_1\). So clearly
\frakM\ contains all closed intervals. By completion closure, \frakM\ contains
all open intervals, hence all Borel sets.

\paragraph{The \(\mu\) measure}
Let \(\mu\) be the measure on \([0,\omega_1]\) provided by
\index{Riesz}
Riesz representation theorem. In the constructive proof
(\cite{RudinRCA80} Theorem 2.14, formula (3)) we have the definition
\begin{equation*}
 \mu(E) = \sup\{\mu(K): K\subset E,\, K\, \textrm{compact}\}.
\end{equation*}
Since the singleton \(\{\omega_1\}\) is compact, and for all \(f\in C(X)\)
we actually have \(f(\omega_1)\) as the value of the functional,
we have the following unique measure
\begin{equation*}
 \lambda(E) = \left\{\begin{array}{l@{\qquad}l}
                      1 & \omega_1 \in E \\
                      0 & \omega_1 \notin E
                     \end{array}\right.
\end{equation*}
for all Borael sets $E$.

%%%%%%%%%%%%%% 19
\begin{excopy}
If \label{ex:2:19}
\(\mu\) is an arbitrary possible measure and \(f\in L^1(\mu)\),
prove that  \(\{x: f(x)\neq 0\}\) has \(\sigma\)-finite measure.
\end{excopy}

Say the measure \(\mu\) is over $X$. Let
\begin{eqnarray*}
A_0 &=& \emptyset  \\
A_n &=& \{x\in X: |f(x)| > 1/n\} \setminus A_{n-1}
\end{eqnarray*}
Now
\begin{equation*}
\int |f|\,d\mu
 =    \cup_{n\in\N} \int_{A_n} |f|\,d\mu
 \geq \cup_{n\in\N} \mu(A_n)/n.
\end{equation*}
If \(mu\) were not \(\sigma\)-finite measure, then there would exist
\(n\in\N\) such that \(\mu(A_n)=\infty\).
But then \(\int |f|\,d\mu = \infty\) contrsicting the
assumption \(f\in L^1(\mu)\).


%%%%%%%%%%%%%%
\begin{excopy}
A positive measure \(\mu\) on a set $X$ is called
\index{sigma-finite@\(\sigma\)-finite}
\emph{\(\sigma\)-finite} if $X$ is a countable union of sets \(X_i\)
with \(\mu(X_i) < \infty\). Prove that \(\mu\) is \(\sigma\)-finite
if and only if there exists  \(f\in L^1(\mu)\)
such that \(f(x)>0\) for every \(x\in X\).
\end{excopy}

One direction was shown in the above exercise~\ref{ex:2:19}.
Conversely, say \(\mu\) is a \(\sigma\)-finite measure over $X$.
Hence we have \(X = \disjunion_{n\in\N} X_i\) and \(\mu(X_n)<\infty\).
Define
\begin{equation*}
f(x) = 2^{-n}\,/\,\max(\mu(X_n),1) \qquad \textrm{if}\; x \in X_i.
% \left\{ \begin{array}{l@{\qquad}l}
\end{equation*}
Now clearly
\begin{equation*}
\int_X |f|\,d\mu
 = \sum_{n\in\N} \int_{X_n} |f|\,d\mu
 \leq \sum_{n\in\N} 2^{-n} = 1.
\end{equation*}


%%%%%%%%%%%%%%%
\end{enumerate}
%%%%%%%%%%%%%%%
