%%%%%%%%%%%%%%%%%%%%%%%%%%%%%%%%%%%%%%%%%%%%%%%%%%%%%%%%%%%%%%%%%%%%%%%%
%%%%%%%%%%%%%%%%%%%%%%%%%%%%%%%%%%%%%%%%%%%%%%%%%%%%%%%%%%%%%%%%%%%%%%%%
%%%%%%%%%%%%%%%%%%%%%%%%%%%%%%%%%%%%%%%%%%%%%%%%%%%%%%%%%%%%%%%%%%%%%%%%
\chapterTypeout{Bounded Operators on a Hillbert Space}

%%%%%%%%%%%%%%%%%%%%%%%%%%%%%%%%%%%%%%%%%%%%%%%%%%%%%%%%%%%%%%%%%%%%%%%%
%%%%%%%%%%%%%%%%%%%%%%%%%%%%%%%%%%%%%%%%%%%%%%%%%%%%%%%%%%%%%%%%%%%%%%%%
\section{Notes}

%%%%%%%%%%%%%%%%%%%%%%%%%%%%%%%%
\subsection{Proof of Theorem 12.4}

% \newcommand{\inprod}[2]{\ensuremath{\langle #1, #2 \rangle}}

The proof of Theorem~12.4 says that \(E^\perp\) is closed by the
Schwartz inequality of Theorem~12.2. Here is the justification.

Let \(x_n \xrightarrow{n\to \infty} x\) and \(x_n \in E^\perp\).

Pick \(\epsilon > 0\). Let \(N<\infty\) such that
\(\forall n>N \quad \|x-x_n\| < \epsilon/\|y\|\).
Now
\begin{equation*}
|\lrangle{x, y}| = |\lrangle{x-x_n, y} + \lrangle{x_n, y}| =
  |\lrangle{x-x_n, y}| 
 \leq \|x - x_n\|\cdot \|y\| \leq (\epsilon/\|y\|)\|y\| = \epsilon.
\end{equation*}
Since \(\epsilon\) is arbitrary \(\lrangle{x,y} = 0\) and \(E^\perp\) is closed.

%%%%%%%%%%%%%%%%%%%%%%%%%%%%%%%%
\subsection{Proof of Theorem 12.5}

The equality
\begin{equation*}
(\Lambda x)s - (\Lambda s)x \in \scrN(\Lambda)
\end{equation*}
is true for all \(s\in H\) (not just \(s\in \scrN(\Lambda)^\perp\)).
Simply because:
\begin{equation*}
\Lambda\left((\Lambda x)s - (\Lambda s)x\right)
= (\Lambda x)\Lambda(s) - (\Lambda s)(\Lambda(x).
\end{equation*}

Then using \((\Lambda x)\lrangle{z,z} - (\Lambda z)\lrangle{x, z} = 0\),
and putting \(y = \lrangle{z, z}^{-1}\overline{\left(\Lambda z\right)}z\)
we compute:
\begin{equation*}
\lrangle{x, y}
  = \lrangle{x, \lrangle{z, z}^{-1}\overline{\left(\Lambda z\right)}z}
  = \overline{\lrangle{z, z}^{-1}}(\Lambda z)\lrangle{x, z}
  = \overline{\lrangle{z, z}^{-1}}(\Lambda x)\lrangle{z, z}
  = \Lambda x
\end{equation*}
\unfinished

%%%%%%%%%%%%%%%%%%%%%%%%%%%%%%%%%%%%%%%%%%%%%%%%%%%%%%%%%%%%%%%%%%%%%%%%
%%%%%%%%%%%%%%%%%%%%%%%%%%%%%%%%%%%%%%%%%%%%%%%%%%%%%%%%%%%%%%%%%%%%%%%%
\section{Exercises} % pages 341-346

Throughout these exercises, the letter $H$ denotes a Hillbert space.

%%%%%%%%%%%%%%%%%
\begin{enumerate}
%%%%%%%%%%%%%%%%%

%%%%%%% 1
\begin{excopy}
The completion of an inner product space is a Hilbert space.
Make this statement more precise, and prove it.
(See the proof of Theorem 12.40 for an application.)
\end{excopy}

Let $G$ be an inner product space.
Let $S$ be the set of all Cauchy sequences in $G$.
Define an equivalence relation on $S$.
Denote \(s_k = (s_{k,j})_{j\in\N}\) for \(k=1,2\)
where  \((s_1 \simeq s_2\) (in $S$)
if \((t_j)_{j\in\N}\) defined by \(t_j = s_{1,j}-s_{2,j}\)
is a Cauchy sequence (clearly) converging to \(0\in G\).
Let \(H = S/\simeq\). We identify \(T: G\to H\) by
\(T(g)=(x_j)_{j\in\N}\)  where \(\forall j\in\N\, x_j=g\).
Define 
\begin{equation*}
\lrangle{s_1,s_2} = \lim_{j\to\infty} \lrangle{s_{1,j},s_{2,j}}.
\end{equation*}
By Theorem~12.2 that does not use completeness, it is clear
that the deinition is independent of representatives.

%%%%%%% 2
\begin{excopy}
Suppose $N$ is a positive integer,
\(\alpha \in \C\), \(\alpha^N = 1\), and \(\alpha^2 \neq 1\).
Prove that every
Hilbert space inner product satisfies the identities
\begin{equation*}
 (x, y) = \frac{1}{N}\sum_{n=1}^N \|x + \alpha^n y\|^2 \alpha^n
\end{equation*}
and
\begin{equation*}
 (x, y) = \frac{1}{2\pi}\int_{-\pi}^\pi \|x + e^{i\theta} y\|^2 e^{i\theta}\,d\theta.
\end{equation*}

Generalize this: Which functions $f$ and
measures \(\mu\) on a set \(\Omega\) give rise to the
identity
\begin{equation*}
 (x, y) = \int_\Omega \|x + f(p)y\|^2 \,d\mu(p) \,\textnormal{?}
\end{equation*}

\end{excopy}

Let \(A = \{\alpha^j: 0\leq j < N\) and
\(\alpha\cdot A = \{\alpha x: x\in A\}\) and \(S=\sum_{x\in A} x\).
Clearly \(\alpha \cdot A = A\).
Thus \(\alpha S = S\) and so \(S=0\) since \(\alpha \neq 1\).

Let \(S_2 = \sum_{0\leq j<N} \alpha^{2j}\).
If $N$ is odd then \(\{x^2: x\in A\} = A\) and so \(S_2 = 0\).
If $N$ is even put \(\alpha S_2 = \sum_{0\leq j<N} \alpha\cdot\alpha^{2j}\).
Thus \((1+\alpha)S_2 = 2A = 0\). Since \(\alpha \neq -1\) we have \(S_2=0\).

Now
\begin{equation*}
 \|x + \alpha^n y\|^2
   = \|x\|^2 + \lrangle{x, \alpha^n y} + \lrangle{\alpha^n y, x} + \|y\|^2
   = \|x\|^2 + \|y\|^2 + \alpha^{-n}\lrangle{x, y} + \alpha^n \lrangle{y, x} 
\end{equation*}

Now
\begin{align*}
\sum_{n=1}^N \alpha^n \cdot \|x + \alpha^n y\|^2 
 &= N\left(S(\|x\|^2 + \|y\|^2)
    + \alpha^{-n+n}\lrangle{x, y} + \alpha^{2n} \lrangle{y, x}\right) \\
 &= N\left( 1\cdot \lrangle{x, y} + S_2\lrangle{y, x}\right)
    = N\cdot \lrangle{x, y}.
\end{align*}
This proves the first desired equality.

By periodic argument \(\int_{-\pi}^\pi e^{i\theta} = 0\)
and also \(\int_{-\pi}^\pi e^{2i\theta} = 0\).

\begin{align*}
\int_{-\pi}^\pi \|x + e^{i\theta} y\|^2 e^{i\theta}\,d\theta
 &= \int_{-\pi}^\pi e^{i\theta}\left(\|x\|^2 + e^0\|y\|^2 +
     e^{-i\theta}\lrangle{x, y} + e^{i\theta}\lrangle{y, x}\right) \,d\theta \\
 &= \left(\|x\|^2 + e^0\|y\|^2 \right)\int_{-\pi}^\pi e^{i\theta}\,d\theta
    + \int_{-\pi}^\pi e^0\lrangle{x, y}\,d\theta
    + \int_{-\pi}^\pi e^{2i\theta}\lrangle{x, y}\,d\theta \\
 &= 0 + 2\pi \lrangle{x, y} + 0 =  2\pi \lrangle{x, y}.
\end{align*}
This proves the second desired equality.

Now if
\begin{itemize}
\item \(|f(p)| = 1\) (and then \(f(p)\overline{f(p)}=1\).
\item \(\int_\Omega 1\,d\mu(p) = 0\).
\item \(\int_\Omega \overline{f(p)}\,d\mu(p) = 1\).
\item \(\int_\Omega f(p)\,d\mu(p) = 0\).
\end{itemize}
then
\begin{align*}
\int_\Omega \|x + f(p)y\|^2 \,d\mu(p)
 &= \int_\Omega \left(\|x\|^2 + f(p)\overline{f(p)}\|y\|^2
   + \overline{f(p)}\lrangle{x, y}
   + f(p)\lrangle{y, x}\right)\,d\mu(p) \\
 &= \left(\|x\|^2 + \|y\|^2\right)\int_\Omega 1\,d\mu(p)
    + \lrangle{x, y}\int_\Omega \overline{f(p)}\,d\mu(p)
    + \lrangle{y, x}\int_\Omega f(p)\,d\mu(p) \\
 &= 0 + \lrangle{x, y} + 0 = \mu(\Omega)\lrangle{x, y}.
\end{align*}
and the last identity holds.

%%%%%%% 3
\begin{excopy}
\begin{itemize}
\itemch{a}
Assume \(X_n\) and \(y_n\) are in the closed unique ball of $H$,
and \(\lrangle{x_n, y_n} \to 1\) as \(n\to\infty\).
Prove that \(\|x_n -y_n\| \to 0\).
\itemch{b}
Assume \(x_n\in H\), \(x_n\to x\) weakly, and \(\|x_n\| \to \|x\|\).
Prove that then \(\|x_n - x\|\to 0\).
\end{itemize}
\end{excopy}

\begin{llem} \label{llem:mindist-z-real}
If \(z\in\C\) and \(r\in\C\) and \(r\geq 0\)
then \(\left|\,|z| - r\right| \leq |z-r|\). 
\end{llem}
\begin{proof}
Let \(z = a+ib\) where \(a,b\in\R\). We have following derivations:
\begin{align*}
a &\leq |a| \leq \sqrt{a^2+b^2} \\
2ar &\leq 2r\sqrt{a^2+b^2} \\
2ar &\leq 2r\sqrt{a^2+b^2} \\
- 2r\sqrt{a^2+b^2} &\leq - 2ar \\
a^2+b^2+r^2 - 2r\sqrt{a^2+b^2} &\leq (a-r)^2+b^2 = a^2+b^2+r^2 - 2ar \\
\left|\sqrt{a^2+b^2} - r\right| &\leq \sqrt{(a-r)^2+b^2} \\
\left|\,|a + ib| - r\right| &\leq |a+ib-r| \\
\left|\,|z| - r\right| &\leq |z-r|
\end{align*}
\end{proof}

\begin{itemize}
\itemch{a}
By negation assume \(d > 0\) and for any \(N<\infty\) we have \(n\geq N\)
such that \(\|x_n - y_n\| \geq d\),
Since \(\|x_n - y_n\|\leq 2\) we can ensure \(d \leq 2\).
Abbreviate \(x=x_n\) and \(y=y_n\). We may assume \(\|y\|\leq \|x\|\).
Decomosing \(y = y_x + y_\perp\) such that \(y_x = ax\) where \(a \in \C\)
and \(\lrangle{y_\perp,x} = 0\). Clearly \(|a|\leq 1\).
By triangle inequality
\begin{equation*}
d \leq \|x - y\| = \|x - y_x - y_\perp\| \leq \|x - y_x\| + \|y_\perp\|.
\end{equation*}
Hence
\begin{equation*}
\|y_\perp\| \geq d/2 \;\vee\: \|x - y_x\| \geq d/2.
\end{equation*}
Two cases:
\begin{enumerate}
\itemch{1} \(\|y_\perp\| \geq d/2\).\\
  Since \(\|y_x\|^2 + \|y_\perp\|^2 = \|y\|^2 \leq \|x\|^2 \leq 1\)
  we have \
  \begin{equation*}
   \|y_x\| \leq \sqrt{1 - \|y_\perp\|^2} \leq \sqrt{1 - d^2/4}.
  \end{equation*}
  That leads to the contradiction:
  \begin{equation*}
  1 = \lim_{n\to\infty}\lrangle{x_n,y_n} =
   \lim_{n\to\infty}\left|\lrangle{x_n,y_n}\right| \leq \|y_{nx}\|
   \leq \sqrt{1 - d^2/4} < 1.
  \end{equation*}
\itemch{2} \(\|x - y_x\| \geq d/2\).\\
  Let \(a = b +ic\) where \(b,c\in\R\) and \(-1\leq b,c,\leq 1\). Now let
  \begin{equation*}
  h = \|x - y_x\| = |1-a|\cdot\|x\| \geq d/2.
  \end{equation*}
  Squaring:
  \begin{equation*}
  h^2 = ((1-b)^2 + c^2)\cdot\|x\|^2 \geq d^2/4.
  \end{equation*}
  Again two cases: 
  \begin{itemize}
  \itemch{i} \((1-b)\|x\| \geq \sqrt{2}d/4\).\\
    Then \(b \leq 1 - \sqrt{2}d/\left(4\|x\|\right) \leq 1 - \sqrt{2}d/4\).
    Now
    \begin{align*}
    |1 - \lrangle{y,x}|
    &= |1 - \lrangle{y_x,x}|
    = |1 - (b + ic)\|x\|^2
    = \left((1 - b)^2 + c^2\right)\cdot\|x\|^2 \\
    &\geq (1 - b)^2\|x\|^2
    \geq d\|x\|^2/8
    \end{align*}
  \itemch{ii} \(|c|\cdot\|x\| \geq \sqrt{2}d/4\).\\
    Then \(|c| \geq \sqrt{2}d/\left(4\|x\|\right) \geq \sqrt{2}d/4\).
    Now similarly,
    \begin{align*}
    |1 - \lrangle{y,x}|
    &= |1 - \lrangle{y_x,x}|
    = |1 - (b + ic)\|x\|^2
    = \left((1 - b)^2 + c^2\right)\cdot\|x\|^2 \\
    &\geq c^2\|x\|^2
    \geq d\|x\|^2/8
    \end{align*}
  \end{itemize}
    For (infinitely many) \(x_n\) such that
    \(\|x_n\| > \sqrt{2}\cdot2/3\)  % \sqrt{8/9}
    we have
    \begin{equation*}
    |1 - \lrangle{y_n,x_n}| \geq d \|x_n\|^2/8 > d/9 > 0.
    \end{equation*}
    Contadiction to \(\lim_{n\to\infty} \lrangle{x_n,y_n} = 0\).
\end{enumerate}
\itemch{b}
If \(x=0\) then \(\lim_{n\to\infty}\|x_n - x\| = \lim_{n\to\infty}\|x_n\| = \|x\|=0\).
Otherwise we want to use \ich{a} of the exercise.
By dividing \(x_n\) and $x$ by \(\|x\|\) we may assume
that \(\|x\|=1\).
Define for \(n\in\N\)
\begin{equation*}
 z_n =
 \left\{\begin{array}{ll}
 0 \quad & \textnormal{if}\;\|x_n\| > 1 \;\wedge \|x_n - x\| \geq 1 \\ 
 x_n \quad & \textnormal{if}\;\|x_n\|\leq 1 \\
 2x_n/\|x_n\|-x_n   \quad & \textnormal{otherwise}
 \end{array}
 \right.
\end{equation*}
Clearly \(\|z_n\|\leq 1\), \(z_n=0\) for mostly finite number of \(n\in \N\).
Also \hbox{\(\lim_{n\to\infty}\|x_n-z_n\| = 0\)}, hence by weak convergence
\begin{equation*}
\lim_{n\to\infty}\lrangle{z_n,x} = \lim_{n\to\infty}\lrangle{x_n,x}
 = \lrangle{x,x} = 1.
\end{equation*}
By previous \ich{a} \(\lim_{n\to\infty} \|x_n - x\| = 0\).
\end{itemize}

%%%%%%% 4
\begin{excopy}
Let \(H^*\) be the dual space of $H$; define \(\psi w: H^* \to H\) by
\begin{equation*}
  y^*(x) = \lrangle{x,\psi y^*} \qquad (x\in H, y^*\in H^*).
\end{equation*}
(See Theorem 12.5) Prove that \(H^*\) is a Hilbert space,
relative to the innder product:
\begin{equation*}
 [x^*,y^*] = \lrangle{\psi y*,\psi x^*}.
\end{equation*}
If \(\phi: H^{**}\to H\) satisfies \(z^{**}(y^*) = [y^*, \phi z^{**}]\)
for all \(y^*\in H^*\) and \(z^{**} \in H^{**}\),
prove that \(\psi\phi\) is an isomorphism of \(H^{**}\) onto $H$
whose existence implies that $H$ is reflexive.
\end{excopy}
By Theorem~12.5 \(\psi\) is conjugate-linear isometry
\begin{enumerate}
\item
  \(
   [y^*,x^*]
   = \lrangle{\psi x*,\psi y^*}
   = \overline{\lrangle{\psi y*,\psi x^*}}
   = \overline{[x^*, y^*]}
  \)

\item 
  \(
  [x^* + y^*, z]
  = \lrangle{z, \psi(x^* + y^+)}
  = \lrangle{z, \psi x^* + \psi y^+}
  = \lrangle{z, \psi x^*} + \lrangle{z, \psi y^*}
  + [x^*, z] + [y^*, z]
  \)

\item
   \(
  [\alpha x^*, y^*]
  = \lrangle{\psi y^*, \psi (\alpha x^*)}
  = \lrangle{\psi y^*, \overline{\alpha}\psi x^*}
  = \alpha\lrangle{\psi y^*, \psi x^*} 
  = \alpha[x^*, y^*]
  \)

\item
  \([x^*, x^*] = \lrangle{\psi x^*, \psi x^*)} \geq  0\).

\item
  If \([x^*, x^*]=0\) then
  \(\lrangle{\psi x^*, \psi x^*)} = 0 \in \C\)
  and then \(\psi x^* = 0 \in H\).
  As we can see in Theorem~12.5 proof, if \(x^* \neq 0 \in H^*\)
  then the \(y\in H\) selected for satisfying \(x^*(x) = \lrangle{x, y}\)
  resulted as \(x^* = y\neq 0\in H\).

\end{enumerate}

%%%%%%% 5
\begin{excopy}
Suppose \(\{u_n\}\) is a sequence of unit vectors in $H$
(that is \(\|u_m\| = 1\)), and assume
that
\begin{equation*}
\Gamma = \sum_{i\neq j} |\lrangle{u_i,u_j}|^2 < \infty.
\end{equation*}
If \(\{\alpha_n\}\) is any sequence of scalars, prove that
\begin{equation*}
 (1 - \Gamma)\sum_{i=m}^n |\alpha_i|^2
 \leq \left\| \sum_{i=m}^n \alpha_i u_i \right\|^2
 \leq (1 + \Gamma) \sum_{i=m}^n |\alpha_i|^2 ,
\end{equation*}
and deduce that the following three properties of \(\{\alpha_i\}\)
are equivalent to each other:
\begin{enumerate}
\item \(\sum_{i=1}^\infty |\alpha_i|^2 < \infty\).
\item \(\sum_{i=1}^\infty \alpha_i u_i\) converges, in the norm of $H$.
\item \(\sum_{i=1}^\infty \alpha_i \lrangle{u_i,y}\) converges,
  for even \(y\in H\).
\end{enumerate}
This generalizes Theorem 12.6.
\end{excopy}
Denote
\begin{align*}
S &= \sum_{j,k=m \wedge j\neq k}^n \lrangle{\alpha_j u_j, \alpha_k u_k}
  = \sum_{m\leq j<k\leq n} 2\Re\left(\lrangle{\alpha_j u_j, \alpha_k u_k}\right)
  \in \R \\
A &= \sum\nolimits_{j=m}^n |\alpha_j|^2 \\
M &= \left\| \sum\nolimits_{j=m}^n \alpha_i u_j \right\|^2
\end{align*}
Now, let \(a=\max_{j=m}^n\{|\alpha_j|\}\)
\begin{align*}
S
  &\leq \sum_{j,k=m \wedge j\neq k}^n |\alpha_j \alpha_k|\cdot
    \left|\lrangle{u_j, u_k}\right|
  \leq a^2 \sum_{j,k=m \wedge j\neq k}^n \left|\lrangle{u_j, u_k}\right| \\
  &\leq  \left(\sum_{j=m}^n |\alpha_j|^2\right)\cdot
        \sum_{j,k=m \wedge j\neq k}^n \left|\lrangle{u_j, u_k}\right|
  \leq  \left(\sum_{j=m}^n |\alpha_j|^2\right)\cdot
        \sum_{j\neq k} \left|\lrangle{u_j, u_k}\right| \\
  &= \Gamma A.
\end{align*}
Then
\begin{align*}
M
 &= \lrangle{\sum_{j=m}^n \alpha_j u_j, \sum_{j=m}^n \alpha_j u_j}
 = \left(\sum_{j=m}^n \lrangle{\alpha_j u_j, \alpha_j u_j}\right)
   + \sum_{j,k=m \wedge j\neq k}^n \lrangle{\alpha_j u_j, \alpha_k u_k} \\
 &= \sum_{j=m}^n |\alpha_j|^2\|u_j\|^2
    + \sum_{m\leq j<k\leq n}
      2\Re\left(\lrangle{\alpha_j u_j, \alpha_k u_k}\right) \\
 &= S + A
\end{align*}
Thus we get the desired double inequality
\begin{equation*}
(1 - \Gamma)A \leq M \leq (1 + \Gamma)A.
\end{equation*}

Assuming \ich{a}.
Pick \(\epsilon > 0\). By the inequality we have, there's \(\mu < 0\) such that
\begin{equation*}
\left\| \sum\nolimits_{j=\mu}^\infty \alpha_j u_j\right\| \leq
 \left(\left(\sum\nolimits_{j=\mu}^\infty |\alpha_j|\right) /
   (1 + \Gamma)\right)^{1/2}.
\end{equation*}
Thus \ich{b} holds.

Assuming \ich{b}. By Theorem~12.5
\(\exists \Lambda\in H^*\) such that
\(\forall x\in H\; \Lambda x = \lrangle{x, y}\).
Since \(\Lambda\) is continuous
\begin{equation*}
\sum\nolimits_{j=1}^\infty \alpha_j \lrangle{u_j,y} =
\sum\nolimits_{j=1}^\infty \lrangle{\alpha_j u_j,y} =
\sum\nolimits_{j=1}^\infty \Lambda(\alpha_j u_j) =
\Lambda\left(\sum\nolimits_{j=1}^\infty \alpha_j u_j\right).
\end{equation*}
Thus \ich{c} holds.

Assuming \ich{c}.
Define \(\Lambda_n \in H^*\) by
\begin{equation*}
\Lambda_n x = \sum\nolimits_{j=1}^n \lrangle{x, \alpha_j u_j}
\qquad (x\in H, n\in \Z^+)
\end{equation*}
By \ich{c} \(\{\Lambda_n x\}\) converges for all \(x\in H\).
By Banach-Steinhaus Theorem~2.5 \(\{\|\Lambda_m\|\}\) is bounded.
But
\begin{equation*}
\|\Lambda_n\|
 = \left\|\sum\nolimits_{j=1}^n \alpha_j u_j\right\|
 = \left(\sum\nolimits_{j=1}^n \|\alpha_j u_j\|^2\right)^{1/2}
 = \left(\sum\nolimits_{j=1}^n |\alpha_j|^2\right)^{1/2}
\end{equation*}
Thus \ich{b} holds.

%%%%%%% 6
\begin{excopy}
Suppose $E$ is a resolution of the identity, as in Section~12.17, and prove that
\begin{equation*}
\left|E_{x,y}(\omega)\right|^2 \leq E_{x,x}(\omega) E_{y,y}(\omega)
\end{equation*}
for all \(x\in H\), \(y\in H\), and \(\omega \in \frakM\).
\end{excopy}

This is similar to Theorem~11.31\ich{b}.

Let \(T=E(\omega)\in\scrB(H)\) which by definition is self-adjoint projection.
So \(E_{x,y}(\omega) = \lrangle{Tx,y}\).
We need to show
\begin{equation} \label{eq:12.6:goal}
\left|\lrangle{Tx,y}\right|^2
 \leq
 \lrangle{Tx,x}\lrangle{Ty,y}
\end{equation}

Since $T$ is a projection (\(T=T^2\)), for any \(x\in H\)
\begin{equation*}
\lrangle{Tx,x}
= \lrangle{Tx,x - Tx + Tx} 
= \lrangle{Tx,x - Tx} + \|Tx\|^2
= \lrangle{x,Tx - T^2x} + \|Tx\|^2
= \|Tx\|^2 \geq 0.
\end{equation*}
Thus \eqref{eq:12.6:goal} becomes
\begin{equation} \label{eq:12.6:goal2}
\left|\lrangle{Tx,y}\right|^2 \leq \|Tx\|^2\|Ty\|^2.
\end{equation}

The inequality trivially holds if \(\lrangle{Tx,y} = 0\).
So we may assume \(r=\lrangle{Tx,y} \neq 0\).
Let \(u = r/|r|\).
Now for all \(t\in \R\)
\begin{equation*}
\lrangle{T(x+tuy),x+tuy}
= \|Tx\|^2 + t\overline{u}r + tu\overline{r} + t^2\|Ty\|^2
= \|Tx\|^2 + 2|r|t + t^2\|Ty\|^2
\geq 0.
\end{equation*}
Hence the discriminant \(4|r|^2 -4 \|Tx\|^2\|Ty\|^2 \leq 0\)
implying \eqref{eq:12.6:goal2}.

%%%%%%% 7
\begin{excopy}
Suppose \( U \in \scrB(H)\) is unitary,
and \(\epsilon > 0\). Prove that scalars \(\alpha_0,\ldots,\alpha_n\)
can be chosen so that
\begin{equation*}
\left\|U^{-1} - \alpha_0 I - \alpha_1 U - \alpha_n U^n\right\| < \epsilon,
\end{equation*}
if \(\sigma(U)\) is a proper subset of the unit circle,
but that this norm is never less than $1$ if
\(\sigma(U)\) covers the whole circle.
\end{excopy}
\unfinished

%%%%%%% MM
\begin{excopy}
\end{excopy}
\unfinished

%%%%%%%%%%%%%%%
\end{enumerate}
%%%%%%%%%%%%%%%


