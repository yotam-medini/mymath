%%%%%%%%%%%%%%%%%%%%%%%%%%%%%%%%%%%%%%%%%%%%%%%%%%%%%%%%%%%%%%%%%%%%%%%%
%%%%%%%%%%%%%%%%%%%%%%%%%%%%%%%%%%%%%%%%%%%%%%%%%%%%%%%%%%%%%%%%%%%%%%%%
%%%%%%%%%%%%%%%%%%%%%%%%%%%%%%%%%%%%%%%%%%%%%%%%%%%%%%%%%%%%%%%%%%%%%%%%
\chapterTypeout{Bounded Operators on a Hillbert Space}

%%%%%%%%%%%%%%%%%%%%%%%%%%%%%%%%%%%%%%%%%%%%%%%%%%%%%%%%%%%%%%%%%%%%%%%%
%%%%%%%%%%%%%%%%%%%%%%%%%%%%%%%%%%%%%%%%%%%%%%%%%%%%%%%%%%%%%%%%%%%%%%%%
\section{Exercises} % pages 341-346

Throughout these exercises, the letter $H$ denotes a Hillbert space.

%%%%%%%%%%%%%%%%%
\begin{enumerate}
%%%%%%%%%%%%%%%%%

%%%%%%% 1
\begin{excopy}
The completion of an inner product space is a Hilbert space.
Make this statement more precise, and prove it.
(See the proof of Theorem 12.40 for an application.)
\end{excopy}

%%%%%%% 2
\begin{excopy}
Suppose $N$ is a positive integer,
\(\alpha \in \C\), \(\alpha^N = 1\), and \(\alpha^2 \neq 1\).
Prove that every
Hilbert space inner product satisfies the identities
\begin{equation*}
 (x, y) = \frac{1}{N}\sum_{n=1}^N \|x + \alpha^n y\|^2 \alpha^n
\end{equation*}
and
\begin{equation*}
 (x, y) = \frac{1}{a\pi}\int_{-\pi}^\pi \|x + e^{i\theta} y\|^2 e^{i\theta}\,d\theta.
\end{equation*}

Generalize this: Which functions $f$ and
measures \(\mu\) on a set \(\Omega\) give rise to the
identity
\begin{equation*}
 (x, y) = int_\Omega \|x + f(p)y\|^2 \,d\mu(p) \,\textnormal{?}
\end{equation*}

\end{excopy}


\unfinished

%%%%%%%%%%%%%%%
\end{enumerate}
%%%%%%%%%%%%%%%


