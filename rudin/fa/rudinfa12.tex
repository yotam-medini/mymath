%%%%%%%%%%%%%%%%%%%%%%%%%%%%%%%%%%%%%%%%%%%%%%%%%%%%%%%%%%%%%%%%%%%%%%%%
%%%%%%%%%%%%%%%%%%%%%%%%%%%%%%%%%%%%%%%%%%%%%%%%%%%%%%%%%%%%%%%%%%%%%%%%
%%%%%%%%%%%%%%%%%%%%%%%%%%%%%%%%%%%%%%%%%%%%%%%%%%%%%%%%%%%%%%%%%%%%%%%%
\chapterTypeout{Bounded Operators on a Hillbert Space}

%%%%%%%%%%%%%%%%%%%%%%%%%%%%%%%%%%%%%%%%%%%%%%%%%%%%%%%%%%%%%%%%%%%%%%%%
%%%%%%%%%%%%%%%%%%%%%%%%%%%%%%%%%%%%%%%%%%%%%%%%%%%%%%%%%%%%%%%%%%%%%%%%
\section{Notes}

\subsection{Proof of Theorem 12.4}

% \newcommand{\inprod}[2]{\ensuremath{\langle #1, #2 \rangle}}

The proof of Theorem~12.4 says that \(E^\perp\) is closed by the
Schwartz inequality of Theorem~12.2. Here is the justification.

Let \(x_n \xrightarrow{n\to \infty} x\) and \(x_n \in E^\perp\).

Pick \(\epsilon > 0\). Let \(N<\infty\) such that
\(\forall n>N \quad \|x-x_n\| < \epsilon/\|y\|\).
Now
\begin{equation*}
|\lrangle{x, y}| = |\lrangle{x-x_n, y} + \lrangle{x_n, y}| =
  |\lrangle{x-x_n, y}| 
 \leq \|x - x_n\|\cdot \|y\| \leq (\epsilon/\|y\|)\|y\| = \epsilon.
\end{equation*}
Since \(\epsilon\) is arbitrary \(\lrangle{x,y} = 0\) and \(E^\perp\) is closed.

%%%%%%%%%%%%%%%%%%%%%%%%%%%%%%%%%%%%%%%%%%%%%%%%%%%%%%%%%%%%%%%%%%%%%%%%
%%%%%%%%%%%%%%%%%%%%%%%%%%%%%%%%%%%%%%%%%%%%%%%%%%%%%%%%%%%%%%%%%%%%%%%%
\section{Exercises} % pages 341-346

Throughout these exercises, the letter $H$ denotes a Hillbert space.

%%%%%%%%%%%%%%%%%
\begin{enumerate}
%%%%%%%%%%%%%%%%%

%%%%%%% 1
\begin{excopy}
The completion of an inner product space is a Hilbert space.
Make this statement more precise, and prove it.
(See the proof of Theorem 12.40 for an application.)
\end{excopy}

Let $G$ be an inner product space.
Let $S$ be the set of all Cauchy sequences in $G$.
Define an equivalence relation on $S$.
Denote \(s_k = (s_{k,j})_{j\in\N}\) for \(k=1,2\)
where  \((s_1 \simeq s_2\) (in $S$)
if \((t_j)_{j\in\N}\) defined by \(t_j = s_{1,j}-s_{2,j}\)
is a Cauchy sequence (clearly) converging to \(0\in G\).
Let \(H = S/\simeq\). We identify \(T: G\to H\) by
\(T(g)=(x_j)_{j\in\N}\)  where \(\forall j\in\N\, x_j=g\).
Define 
\begin{equation*}
\lrangle{s_1,s_2} = \lim_{j\to\infty} \lrangle{s_{1,j},s_{2,j}}.
\end{equation*}
By Theorem~12.2 that does not use completeness, it is clear
that the deinition is independent of representatives.

%%%%%%% 2
\begin{excopy}
Suppose $N$ is a positive integer,
\(\alpha \in \C\), \(\alpha^N = 1\), and \(\alpha^2 \neq 1\).
Prove that every
Hilbert space inner product satisfies the identities
\begin{equation*}
 (x, y) = \frac{1}{N}\sum_{n=1}^N \|x + \alpha^n y\|^2 \alpha^n
\end{equation*}
and
\begin{equation*}
 (x, y) = \frac{1}{2\pi}\int_{-\pi}^\pi \|x + e^{i\theta} y\|^2 e^{i\theta}\,d\theta.
\end{equation*}

Generalize this: Which functions $f$ and
measures \(\mu\) on a set \(\Omega\) give rise to the
identity
\begin{equation*}
 (x, y) = \int_\Omega \|x + f(p)y\|^2 \,d\mu(p) \,\textnormal{?}
\end{equation*}

\end{excopy}

Let \(A = \{\alpha^j: 0\leq j < N\) and
\(\alpha\cdot A = \{\alpha x: x\in A\}\) and \(S=\sum_{x\in A} x\).
Clearly \(\alpha \cdot A = A\).
Thus \(\alpha S = S\) and so \(S=0\) since \(\alpha \neq 1\).

Let \(S_2 = \sum_{0\leq j<N} \alpha^{2j}\).
If $N$ is odd then \(\{x^2: x\in A\} = A\) and so \(S_2 = 0\).
If $N$ is even put \(\alpha S_2 = \sum_{0\leq j<N} \alpha\cdot\alpha^{2j}\).
Thus \((1+\alpha)S_2 = 2A = 0\). Since \(\alpha \neq -1\) we have \(S_2=0\).

Now
\begin{equation*}
 \|x + \alpha^n y\|^2
   = \|x\|^2 + \lrangle{x, \alpha^n y} + \lrangle{\alpha^n y, x} + \|y\|^2
   = \|x\|^2 + \|y\|^2 + \alpha^{-n}\lrangle{x, y} + \alpha^n \lrangle{y, x} 
\end{equation*}

Now
\begin{align*}
\sum_{n=1}^N \alpha^n \cdot \|x + \alpha^n y\|^2 
 &= N\left(S(\|x\|^2 + \|y\|^2)
    + \alpha^{-n+n}\lrangle{x, y} + \alpha^{2n} \lrangle{y, x}\right) \\
 &= N\left( 1\cdot \lrangle{x, y} + S_2\lrangle{y, x}\right)
    = N\cdot \lrangle{x, y}.
\end{align*}
This proves the first desired equality.

By periodic argument \(\int_{-\pi}^\pi e^{i\theta} = 0\)
and also \(\int_{-\pi}^\pi e^{2i\theta} = 0\).

\begin{align*}
\int_{-\pi}^\pi \|x + e^{i\theta} y\|^2 e^{i\theta}\,d\theta
 &= \int_{-\pi}^\pi e^{i\theta}\left(\|x\|^2 + e^0\|y\|^2 +
     e^{-i\theta}\lrangle{x, y} + e^{i\theta}\lrangle{y, x}\right) \,d\theta \\
 &= \left(\|x\|^2 + e^0\|y\|^2 \right)\int_{-\pi}^\pi e^{i\theta}\,d\theta
    + \int_{-\pi}^\pi e^0\lrangle{x, y}\,d\theta
    + \int_{-\pi}^\pi e^{2i\theta}\lrangle{x, y}\,d\theta \\
 &= 0 + 2\pi \lrangle{x, y} + 0 =  2\pi \lrangle{x, y}.
\end{align*}
This proves the second desired equality.

Now if
\begin{itemize}
\item \(|f(p)| = 1\) (and then \(f(p)\overline{f(p)}=1\).
\item \(\int_\Omega 1\,d\mu(p) = 0\).
\item \(\int_\Omega \overline{f(p)}\,d\mu(p) = 1\).
\item \(\int_\Omega f(p)\,d\mu(p) = 0\).
\end{itemize}
then
\begin{align*}
\int_\Omega \|x + f(p)y\|^2 \,d\mu(p)
 &= \int_\Omega \left(\|x\|^2 + f(p)\overline{f(p)}\|y\|^2
   + \overline{f(p)}\lrangle{x, y}
   + f(p)\lrangle{y, x}\right)\,d\mu(p) \\
 &= \left(\|x\|^2 + \|y\|^2\right)\int_\Omega 1\,d\mu(p)
    + \lrangle{x, y}\int_\Omega \overline{f(p)}\,d\mu(p)
    + \lrangle{y, x}\int_\Omega f(p)\,d\mu(p) \\
 &= 0 + \lrangle{x, y} + 0 = \mu(\Omega)\lrangle{x, y}.
\end{align*}
and the last identity holds.

%%%%%%% 3
\begin{excopy}
\begin{itemize}
\itemch{a}
Assume \(X_n\) and \(y_n\) are in the closed unique ball of $H$,
and \(\lrangle{x_n, y_n} \to 1\) as \(n\to\infty\).
Prove that \(\|x_N -y_n\| \to 0\).
\itemch{b}
Assume \(x_n\in H\), \(x_n\to x\) weakly, and \(\|x_n\| \to \|x\|\).
Prove rgar then \(\|x_n - x\|\to 0\).
\end{itemize}
\end{excopy}

\unfinished

%%%%%%%%%%%%%%%
\end{enumerate}
%%%%%%%%%%%%%%%


