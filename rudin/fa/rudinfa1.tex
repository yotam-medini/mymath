%%%%%%%%%%%%%%%%%%%%%%%%%%%%%%%%%%%%%%%%%%%%%%%%%%%%%%%%%%%%%%%%%%%%%%%%
%%%%%%%%%%%%%%%%%%%%%%%%%%%%%%%%%%%%%%%%%%%%%%%%%%%%%%%%%%%%%%%%%%%%%%%%
%%%%%%%%%%%%%%%%%%%%%%%%%%%%%%%%%%%%%%%%%%%%%%%%%%%%%%%%%%%%%%%%%%%%%%%%
\chapterTypeout{General Theory}

%%%%%%%%%%%%%%%%%%%%%%%%%%%%%%%%%%%%%%%%%%%%%%%%%%%%%%%%%%%%%%%%%%%%%%%%
%%%%%%%%%%%%%%%%%%%%%%%%%%%%%%%%%%%%%%%%%%%%%%%%%%%%%%%%%%%%%%%%%%%%%%%%
\section{Notes}

Lemma to be used for Exercise~13 in page~\pageref{ex:1:13}.

\begin{llem}
Let $X$ be some topological space having a countable dense subset.
Let \(C(X)\) be the set of complex continuous functions over $X$.
\end{llem}
\textbf{Proof.}
Let $D$ be a countable dense subset of $X$.
\proofend

In Section~1.47 the inequality
\begin{equation} \label{eq:abp:concave}
(a + b)^p \leq a^p + b^p
\qquad (a,b \geq 0, \quad 0 < p < 1)
\end{equation}
is used. 
\iffalse
It is implied by the convexity of \(t \to t^p\),
see \cite{RudinRCA80} the proof of Theorem~3.5.
\fi

Look at \(f(x) = a^p + x^p - (a + x)^p\) with \(x \geq 0\).
Clearly \(a+x \geq x\) and so \(x^{p-1} \leq (a+x)^{p-1}\).
Thus
\begin{equation*}
f'(x) = px^{p-1} - p(a+x)^{p-1} = p\left(x^{p-1} - (a+x)^{p-1}\right) \geq 0.
\end{equation*}
Since \(f(0) = 0\), \eqref{eq:abp:concave} holds.


%%%%%%%%%%%%%%%%%%%%%%%%%%%%%%%%%%%%%%%%%%%%%%%%%%%%%%%%%%%%%%%%%%%%%%%%
%%%%%%%%%%%%%%%%%%%%%%%%%%%%%%%%%%%%%%%%%%%%%%%%%%%%%%%%%%%%%%%%%%%%%%%%
\section{Exercises} % pages 36-40

%%%%%%%%%%%%%%%%%
\begin{enumerate}
%%%%%%%%%%%%%%%%%

%%%%%%%%%%%%%%
\begin{excopy}
Suppose $X$ is a vector space. All sets mentioned below are understood
to be subsets of $X$.
Prove the following statements from the axioms as given in Section~1.4
(Some of these are tacitly used in the text.)
\begin{itemize}
 \itemch{a}
   If \(x\in X\) and \(y\in X\) there is a unique \(z\in X\)
   such that \(x+z=y\).
 \itemch{b} \(0x = 0 = \alpha 0\) if \(x\in X\) and \(\alpha\) a scalar.
 \itemch{c} \(2A \subset A + A\); it may happen that \(2A \neq A + A\).
 \itemch{d}
   $A$ is convex if and only if \((s+t)A = sA + tA\) for all positive
   scalars $s$ and $t$.
 \itemch{e} Every union (and intersection) of balanced sets is balanced.
 \itemch{f} Every intersection of convex sets is convex.
 \itemch{g}
   If \(\Gamma\) is a collection is a collection 
   of convex sets that is totally ordered
   by set inclusion, then the union of all members of \(\Gamma\) is convex.
 \itemch{h} If $A$ and $B$ are convex, so is \(A+B\).
 \itemch{i} If $A$ and $B$ are balanced, so is \(A+B\).
 \itemch{j}
   Show that parts \ich{f}, \ich{g}, and \ich{h} hold
   with subspaces in place of convex sets.
\end{itemize}
\end{excopy}

\begin{itemize}
 \itemch{a}
   Put \(z = (-x) + y\). Now 
   \(x+z = x + ((-x) + y) = (x + (-x)) + y = 0 + y = y\).
   For uniqueness, say \(x+z' = y\). Thus
   \(z = (-x) + y = (-x) + (x+z') = ((-x) + x) + z' = 0 + z' = z'\).
 \itemch{b}
   First note that the defintion of the zero vector
   can be generalized. If  \(x,z\in X\) and \(x+z = x\) then
   \(y+z = y\) for all \(y\in X\).
   This is because
  \[z = z+(x+(-x)) = (z+x)+(-x) = x+(-x) = 0.\]

   From the ditributive law \[x = 1x = (0 + 1)x = 0x + 1x = 0x + x.\]
   Now \[0x = 0x + (x + (-x)) = (0x + x) + (-x) = x + (-x) = 0.\]

   Take some \(x\in X\). From other ditributive law
   \[\alpha x = \alpha (x+0) = \alpha x + \alpha 0.\]
   Hence \(\alpha 0 = 0\).
 \itemch{c}
   Let \(a\in 2A\) then there exists \(x\in A\) such that \(a=2x\).
   Now \(2x = x + x \in A + A\).
   Take \(X=\R\) and \(A = \{0,1\}\).
   Then \(2A = \{0,2\} \subsetneq \{0,1,2\} = A+A\).
 \itemch{d}
   If \(s,t > 0\), then \((s+t)A \subset sA + tA\) trivially.
   If in addition, $A$ is convex, then   
   for any \(x,y\in A\) we have \(s/(s+t)x + t/(s+t)y \in A\).
   Equivalently \(sx + ty \in (s+t)A\). Thus
   \(sA + tA \subset (s+t)A\). Thus \((s+t)A = sA + tA\)

   Conversely, assume \((s+t)A = sA + tA\) for any \(s,t > 0\).
   In particular, \(sA + tA \subset (s+t)A\) for any \(s,t > 0\).
   This means that if \(x,y\in A\) then 
   \(sx+ty\in A\) for any \(s,t > 0\), which is convexity.

 \itemch{e}
   Let \(A_i\) be a family of balanced sets, that is for any 
   \(\alpha\) in the field \(\Phi\) such that \(|\alpha| = 1\)
   we have \(\alpha A_i = A_i\).

   Indeed, take an arbitrary such \(\alpha\).

   \textbf{Union:}
   Put \(U = \cup_{i\in I} A_i\) and say \(x \in U\).
   Then, there exists \(j \in I\) such that \(x\in A_i\).
   Since \(A_i\) is balanced, \(x \in \alpha A_i \subset \alpha U\).
   Thus \(U \subset \alpha U\).
   Similarly, we can show that \(U \subset \alpha^{-1} U\)
   which is equivalent to the converse inclusion
   \(\alpha U \subset U\) and so  \(U = \alpha U\).

   \textbf{Intersection:}
   Put \(W = \cap_{i\in I} A_i\) and say \(x \in W\).
   Then \(x \in A_i\) for all \(i \in I\) and in turn
   so \(x \in \alpha A_i\) for all \(i \in I\).
   Hence, \(x \in \cap_{i\in I} \alpha A_i = W\).
   Thus \(W \subset \alpha W\). With similar ending argumentation
   as in the `Union' case we get that \(W = \alpha W\).
 \itemch{f}
  Say \(\{A_i\}_{i\in I}\) is a family of convex sets.
  Let \(W = \cap_{i\in I} A_i\), and let \(x,y \in W\).
  Take an arbitrary \(i \in I\) and now
  for any \(s,t\geq 0\) such that \(s+t=1\) 
  we have \(sx+ty \in A_i\). Since $i$ is arbitrary, 
  \(sx+ty \in W\) and so $W$ is convex.
 \itemch{g}
  Let $U$ be the union of \(\Gamma\). Then any two point \(x,y\in U\)
  belong to some convex set \(A \in \Gamma\) and so any convex combination
  of $x$ and $y$ is in \(A\subset U\).
 \itemch{h}
  Let \(x_0,x_1 \in A+B\)
  For \(i=0,1\) there exist \(a_i\in A\) and \(b_i \in B\) 
  such that \(x_i = a_i + b_i\).
  Now for any \(s,t\geq 0\) such that \(s+t=1\) 
  we have 
  \begin{eqnarray*}
    sa_0 + ta_1 &\in& A \\
    sb_0 + tb_1 &\in& B
  \end{eqnarray*}
  and so 
 \begin{equation*}
   sx_0 + tx_1 = s(a_0+b_0) + t(a_1+b_1)  \in A+B.
 \end{equation*}

 \itemch{i}
  Let \(a\in A\), \(b\in B\) and a scalar \(\alpha\) such that \(|\alpha|=1\).
  Since $A$ and $B$ are balanced, we have 
  \(\alpha a \in A\) and \(\alpha b \in B\)
  and so \(\alpha(a+b)\in A + B\).
  Thus \(A+B \subset \alpha(A+B)\). Using \(\alpha^{-1}\) we get the 
  converse inclusion and so 
   \(A+B = \alpha(A+B)\) and \(A+B\) is balanced.
 \itemch{j}
  \begin{itemize}
   \item[{[f]}] 
        Say \(\{A_i\}_{i\in I}\) is a family of subspaces of $X$.
        Put \(W = \cap_{i\in I} A_i\).
        For any \(x,y\in W\)
        all linear combinations of $x$ and $y$ are in 
        all \(\{A_i\}\) and so must also be in $W$. Hence $W$ contains
        any linear combinations of its vectors, and so $W$ is a subspace.
   \item[{[{\small g}]}] 
        Let \(\Gamma\) be a collection of subspaces of $X$
        that are totally ordered by inclusion, and let $U$ be the union.
        For any \(x,y\in U\) thetre is \(A\in \Gamma\) such that
        \(x,y\in A\). Clearly any linear combination of $x$ and $y$
        is in \(A\subset U\) and so $U$ is a subspace.
   \item[\ich{h}] 
        Say $A$ and $B$ are subspaces of $X$.
        Any element \(w_i\in A+B\) can be represented as \(w_i=x_i+y_i\)
        where \(x_i\in A\) and \(y_i\in B\).
        Let \(v = s w_0 + t w_1\) be a linear combination of \(w_i\in W\).
        Now 
        \begin{equation*}
         v = s w_0 + t w_1 = (sx_0 + tx_1) + (sy_0 + ty_1) \in A + B
        \end{equation*}
        Thus \(A+B\) is a subspace of $X$.
  
  \end{itemize}
\end{itemize}

%%%%%%%%%%%%%%
\begin{excopy}
The
\index{convex hull}
\emph{convex hull} of a set $A$ in a vector space $X$ is the set of all
\index{convex combination}
\emph{convex combinations} of members f $A$, that is, the set of all sums
\begin{equation*}
  t_1 x_1 + \cdots + t_n x_n
\end{equation*}
in which
\(x_i\in A\),
\(t_i\geq 0\),
\(\sum t_i = 1\);
$n$ is arbitrary. Prove that the convex hull of $A$ is convex
and that it is the intersection of all convex sets that contain $A$.
\end{excopy}

Let $H$ be the set of all such combinations, namely \(\sum_{i=1}^n t_ix_i\),
where \(t_i\in \Phi\) and \(x_i\in A\).

First, we show that $A$ is convex.
Let \(v_1,v_2\in H\) and \(0\leq r,s \leq 1\) with \(r+s=1\).
By definition both \(v_j\) are (for \(j=1,2\) (convex) combination as in
\begin{equation*}
v_j = \sum_{i=1}^{n_j} t_{ij} x_{ij}.
\end{equation*}
By adjoining \(\{x_{i0}\}\) and \(\{x_{i1}\}\), into 
a single set \(\{x_i\}\)
and assigning zeros to
some coefficients of the newly created set of generators, we can have
\begin{equation*}
v_j = \sum_{i=1}^{N} t_{ij} x_i.  \qquad \textnormal{for }\, j=1,2
\end{equation*}
where \(\sum_i t_{ij} = 1\) for \(j=1,2\).
Now 
\begin{equation*}
rv_1 + sv_2 = 
r\sum_{i=1}^N t_{i0} x_i +
s\sum_{i=1}^N t_{i1} x_i =
\sum_{i=1}^N (rt_{i0} + st_{i1}) x_i
\end{equation*}
and
\begin{eqnarray*}
\sum_{i=1}^N r t_{i0} + s t_{i1} 
 &=&  \left(\sum_{i=1}^N r t_{i0}\right) +
      \left(\sum_{i=1}^N s t_{i1}\right) \\
 &=&  r\left(\sum_{i=1}^N t_{i0}\right) +
      s\left(\sum_{i=1}^N t_{i1}\right) \\
 &=&  r\cdot 1 + s \cdot 1 = 1.
\end{eqnarray*}
Thus \(rv_1 + sv_2 \in H\) and so $H$ is convex.

Now we show that $H$ is the minimal convex set that contains $A$.
Let \(H_n\) be the set of all convex combination of size $n$
that is \(\sum_{i=1}^n t_i x_i\) with positive \(t_i\) 
and \(\sum_{i=1}^n t_i = 1\).
By induction will show that \(H_n \subset \hull(A)\) for all $n$.
Clearly \(H_1 = A \subset \hull(A)\).
Now assume \(H_k \subset \hull(A)\).
Let \(v = \sum_{i=1}^{k+1} t_i x_i\) 
be some convex combination, that is \(t_i\geq 0\) 
and \(\sum_{i=1}^{k+1} t_i = 1\).
If \(t_k{+1} = 1\) then trivailly \(v\in \hull(A)\).
Otherwise,  \(1 - t_{k+1} = \sum_{i=1}^k t_i \neq 0 \) 
and we have
\begin{eqnarray} \label{eq:convex:k1}
v &=& \sum_{i=1}^{k+1} t_i x_i \\
 &=& \left(\sum_{i=1}^k t_i x_i \right) + t_{k+1}x_{k+1} \\
 &=& (1 -t_{k+1}) 
      \left(\sum_{i=1}^k \left(t_i/\sum_{j=1}^k t_j\right)x_i \right) 
      + t_{k+1}x_{k+1} 
\end{eqnarray}
Since 
\begin{equation*}
 \sum_{i=1}^k \left(t_i/\sum_{j=1}^k t_j\right) = 1
\end{equation*}
We have by induction
\begin{equation*}
 \left(\sum_{i=1}^k \left(t_i/\sum_{i=j}^k t_j\right)x_i \right)
  \in H_k \subset \hull(A).
\end{equation*}

Hence, the last expression in (\ref{eq:convex:k1}) is a conve combination
in \(\hull(A)\) and so \(v\in \hull(A)\).
Thus \(H_{k+1}\subset \hull(A)\).




%%%%%%%%%%%%%%
\begin{excopy}
Let $X$ be a topological vector space. All sets mentioned below
are understood to be subsets of $X$. Prove the following statements:
\begin{itemize}
 \itemch{a}The convex hull of every open set is open.
 \itemch{b}
   If $X$ is locally convex the convex hull of every bounded set is bounded.
   (This is false without convexity; see Section~1.47.)
 \itemch{c} If $A$ and $B$ are bounded, so is \(A+B\)
 \itemch{d} If $A$ and $B$ are compact, so is \(A+B\)
 \itemch{e} If $A$ is compact and $B$ is closed, then \(A+B\) is closed.
 \itemch{f}
   The sum of two closed sets may fail to be closed.
   [The inclusion in \ich{b} of Theorem~1.13 may therefore be strict.]
\end{itemize}
\end{excopy}

\begin{itemize}
 \itemch{a}
   Let $A$ be an open set in $X$ and let $H$ be the convex hull of $A$.
   Take an arbitrary \(h\in H\). 
   By previous exercise, we have the representation
   \(h = \sum_{i=1}^n t_i x_i\) 
   where \(\sum_{i=1}^n t_i = 1\) and \(t_i\geq 0\) and with \(x_i\in A\).
   By $A$ being open, there exist
   open neighborhoods  \(V_i\) of $0$, such that \(x_i + V_i \subset A\).
   Taking finite intersection, we have \(V = \cap_{i=1}^n V_i\) an open set.

   We claim that \(h + V \subset H\).
   Let \(h` \in h+V\) so \(d = h`-h\in V\).
   Now for every \(i=1,\ldots,n\), we have
   \begin{equation*}
   x_i + d \in x_i+V \subset x_i+V_i \subset A.
   \end{equation*}

   Now 
   \begin{equation*}
   h` = h + d = \left(\sum_{i=1}^n t_i x_i\right) + 1\cdot d \\
      = \sum_{i=1}^n t_i (x_i + d) \\
   \end{equation*}

   and so \(h'\) is a convex combination of vectors in $A$.
   Thus \(h\in \inter{(\hull(A))}\) and so \(\hull(A)\) is open.

 \itemch{b}
   Let $B$ be bounded in $X$ and \(H = \hull(B)\).
   Let $V'$ be an arbitrary neighborhood of the origin.
   By being locally convex, we can take a convex \(V\subset V'\) 
   which is also a $0$-neighborhood.
   Since $B$ is bounded, there exists \(0 < m < \infty\) such that
   \(B\subset mV\).

   Now since \(mV\) is convex and $H$ is the minimal convex set
   containing $B$, we must have \(H \subset mV\) and so $H$ is bounded.

 \itemch{c}
   Let $A$ and $B$ be bounded. Let $V$ be an arbitrary neighborhood of $0$.
   By definition, there exist \(m_A, m_B < \infty\)
   such that \(A \subset m_A V\) and \(B \subset m_B V\).
   let \(x=a+b\in A+B\) with \(a\in A\) and \(b\in B\).
   Now
   \begin{eqnarray*}
   a &\in& A \subset m_A V \subset (m_A + m_B)V \\
   b &\in& B \subset m_B V \subset (m_A + m_B)V \\
   \end{eqnarray*}
   and so \(x \in  (m_A + m_B)V\), hence \(A+B\) is bounded.

 \itemch{d}
  We use general point set topolgy.
  \index{Tychonoff}
  By Tychonoff Theorem, the set \(A\times B\) is compact in \(X\times X\).
  Also the image of a compact set under a continuous mapping is compact.
  Combining these with the fact that
  the addition mapping \(+: X\times X \rightarrow X\) 
  is continuous, we conclude that
  the image \(A+B\) is compact.
   
 \itemch{e}
   Let $K$ be compact and $C$ be closed.
   By negation say \(x\in \overline{K+C}\setminus(K+C)\).
   Now, since \(x\notin K+C\), we must have \((x-K)\cap C = \emptyset\).
   But since \(x-K\) is compact,
   by Theorem~1.10 it can be separated from $C$ by open $V$ --- 
   a neighborhood of $0$.
   More precisely, 
   \begin{equation*}
   (x-K+V) \cap (C+V) = \emptyset
   \end{equation*}
   Hence,
   \begin{equation*}
   (x+V) \cap (C+K+V) = \emptyset
   \end{equation*}
   and in particular, \(x \notin C+K+V \supset \overline{C+K}\),
   which is a contradiction.
   
 \itemch{f}
   In \(\R^2\), let 
   \begin{eqnarray*}
   A & = & \{(x,y)\in\R^2: y = e^x\} \\
   B & = & \{(x,y)\in\R^2: y = 0\} \\
   \end{eqnarray*}
   Now clearly \((0,0) \notin A+B\), but for any \(\epsilon > 0\)
   There is some \(M > 0\) such that \(e^{-M} < \epsilon\).
   So \(p_m = (-M,e^{-M}) + (M,0) \in A+B\), but 
   \(\|p_M - (0,0)\| = \|p_M\| < \epsilon\). Thus
   \(p_M \in \overline{A+B} \setminus (A+B)\), which shows that \(A+B\) is 
   not closed.

\end{itemize}


%%%%%%%%%%%%%%
\begin{excopy}
Let \(B=\{(z_1,z_2)\in\C: |z_1|\leq|z_2|\}\).
Show that $B$ is balanced but that its interior is not.
[Compare with \ich{e} of Theorem~1.13.]
\end{excopy}

Say \(\alpha\in \C\) such that \(|\alpha|\leq 1\).
Assume \((z_1,z_2)\in B\), then \(|z_1|\leq|z_2|\) and 
so 
\begin{equation*}
 |\alpha z_1| = |\alpha|\cdot|z_1| \leq
 |\alpha|\cdot|z_2| = |\alpha z_2|.
\end{equation*}
Thus \(\alpha B \subset B\) and $B$ is balanced.
But \(0\notin \inter{B}\) and in particular,
\begin{equation*}
0\cdot(0,0) = (0,0) \in B \setminus \inter{B}.
\end{equation*}
Thus \inter{B} is not balanced.

%%%%%%%%%%%%%%
\begin{excopy}
Consider the definition of 
\index{bounded set}
``bounded set'' given in Section~1.6.
Would the content of this definition be altered if it were required merely
that to every neighborhood $V$ of $0$
corresponds \emph{some} \(t>0\) such that \(E \subset tV\)?
\end{excopy}

The answer is: Yes.

Call the suggested definition in the exercise: \emph{weak-bounded}.
Clearly a bounded set is weak bounded.

Conversely, say $E$ is a weak-bounded set.
Let $V$ be an arbitrary neighborhood of $0$.
Since the multiplication between scalar and vector 
\(\cdot:\Phi\times X\rightarrow X\) is a continuous mapping,
there is an \(\epsilon > 0\) and a neighborhood $V'$ 
such that \(\alpha v \in V\) % where \(\alpha\in \Phi\) and \(v\in V'\)
whenever \(|\alpha| < \epsilon\) and \(v\in V'\).
By minizing and intersection, we can pick \(\epsilon\) and $V'$
such that \(\epsilon \leq 1\) and \(V' \subset V\).
Hence, if \(|\alpha|< \epsilon\) then  \(\alpha V' \subset V\).
Now by being weak-bounded, there is \(t>0\) such that \(E \subset tV'\).
Putting \(M = t/\epsilon\), for every $s$ such that
\(|s| > M\) we have \(|t/s| < \epsilon\) and so 
\((t/s)V' \subset V\), equivalently 
\(tV' \subset sV\). Thus
\begin{equation*}
E \subset tV' \subset sV.
\end{equation*}
We conclude that $E$ is bounded by the original definition.
This justifies the positive answer.

%%%%%%%%%%%%%%
\begin{excopy}
Prove that a set $E$ in a topological vector space is bounded if and only if
every countable subset of $E$ is bounded.
\end{excopy}

If $E$ is bounded then clearly any subset of $E$ is bounded, 
in  particular countable subsets.

Conversely, assume every countable subset of $E$ is bounded
and by negation $E$ is not bounded.
Thus there exists some neighborhood $V$ of $0$, such that 
\(E\not\subset nV\) for every \(n > 0\).
Pick \(v_n \in E \setminus nV\). 
Now the countable set \(\{v_n\}_{n\in \N}\) is not bounded.

%%%%%%%%%%%%%%
\begin{excopy}
Let $X$ be a vector space of all complex functions
on the unit interval \([0,1]\), topologized by the family of seminorms
\begin{equation*}
 \rho_x(f) = |f(x)|\qquad (0\leq x \leq 1).
\end{equation*}
This topology is called the 
\index{pointwise convergence!topology}
\emph{topology of pointwise convergence}.
Justify this terminology.

Show that there is a sequence \(\{f_n\}\) in $X$ such that
(\emph{a}) \(\{f_n\}\) converges to $0$ as \(n\rightarrow \infty\),
but (\emph{b}) if \(\{\gamma_n\}\) is any sequence of scalars such that
\(\gamma_n \rightarrow \infty\)
then \(\{\gamma_n f_n\}\) does not converge to $0$.
(Use the fact that the collection of all complex sequences converging to $0$
has the same cardinality as \([0,1]\).)

This shows that metrizability cannot be omitted in (\emph{b}) of Theorem~1.28.
\end{excopy}


For any sequence \(\{f_n(x)\}\) of functions that converge to \(f(x)\)
pointwise, we have \(\rho_x(f_n - f) = |f_n(x) - f(x) | \rightarrow 0\).
This explain the terminology.

We now compute the cardinality of the set $S$ of complex sequences converging
to zero. Clearly 
\begin{equation*}
|S| \geq |\C| = \mathfrak{c} = 2^{\aleph_0}.
\end{equation*}
The set $T$ of all complex sequences has the cardinality:
\begin{equation*}
|T| = \mathfrak{c}^{\aleph_0} = \left(2^{\aleph_0}\right)^{\aleph_0}
 = 2^{\aleph_0\aleph_0} = 2^{\aleph_0}.
\end{equation*}
Since \(S\subset T\) we have \(|S|\leq |T|\) and( so \(|S| = \mathfrak{c}\).

Choose some 1-1 onto mapping \(\sigma:[0,1]\rightarrow S\)
and define the sequence of functions
\(f_n(x) = \sigma(x)_n\).
By construction, for any complex sequence converging to zero
coincide with \(\{f_n(x)\}_{n=1}^\infty\) for some \(x\in[0,1]\).

Clearly \(f_n(x) \rightarrow 0\) for all \(x\in [0,1]\).
Now let, \(\gamma_n \rightarrow \infty\)
The sequence 
\begin{equation*}
 \alpha_n = \left\{\begin{array}{cl}
            1/\gamma_n & \qquad \gamma_n \neq 0 \\
            1          & \qquad \gamma_n = 0 \\
            \end{array}\right.
\end{equation*}
Clearly \(\alpha_n \rightarrow 0\). By construction, there exists some
\(\alpha\in [0,1]\) such that \(f_n(\alpha) = \alpha_n\) for all $n$.
But \(\gamma_n f_n(\alpha) \rightarrow 1 \neq 0\). In particular
\(\gamma_n f_n \not\rightarrow 0\).


%%%%%%%%%%%%%%
\begin{excopy}
 \begin{itemize}
  \itemch{a}
   Suppose \scrP\ is a separating family of seminorms 
   on a vector space $X$. Let \scrQ\ be the smallest family of 
   seminorms on $X$ that contains  \scrP\ and is closed under max.
   [This means: If \(p_1\in \scrQ\), \(p_2\in \scrQ\) and
   \(p = \max(p_1,p_2)\), then \(p\in \scrQ\).]
   If the construction of Theorem~1.37 is applied to \scrP\ and to \scrQ,
   show that the two resulting topologies coincide.
   The main difference is that \scrQ\ leads directly to a base,
   rather than to a subbase.
   [See Remark \ich{a} of section~1.38]
  \itemch{b}
   Suppose \scrQ\ is as in part \ich{a} and \(\Lambda\) is a linear functional
   on $X$. Show that \(\Lambda\) is continuous if and only if there exists
   a \(p\in \scrQ\) such that \(|\Lambda x| \leq Mp(x)\)
   for all \(x\in X\) and some constant \(M< \infty\).
 \end{itemize}
\end{excopy}

\begin{itemize}
 \itemch{a}
  Let us denote the topologies by \(\tau_{\scrP}\) and \(\tau_{\scrQ}\)
  respectably.
  Clearly  \(\tau_{\scrP} \subset \tau_{\scrQ}\).
  To show the opposite inclusion, let $G$ be a \(\tau_{\scrQ}\)
  neighborhood of $0$. By being a subbase, for \(k=1,\ldots,n\)
  there exist
  \begin{equation*}
    V_k = V(p_k,\alpha_k) = \{x\in X: q_k(x) < \alpha_k\}
     \qquad \textrm{where }\, q_k\in \scrQ, \alpha > 0
  \end{equation*}
  such that \(\cap_{k=1}^n V_k \subset G\).
  But each \(q_k\in \scrQ\) can be represented as
  \begin{equation*}
  q_k = \max_{1\leq j\leq N_k} {}_{k}p_j
  \end{equation*}
  and so 
  \begin{equation*}
  V_k = \{x\in X: q_k(x) < \alpha_k\} 
      = \{x\in X:  \max_{1\leq j\leq N_k} {}_{k}p_j(x) < \alpha_k\}
      = \bigcap_{j=1}^{N_k} \{x\in X: {}_{k}p_j(x) < \alpha_k\}
  \end{equation*}
  Thus 
  \begin{equation*}\
    \cap_{k=1}^n V_k 
     = \cap_{k=1}^n\cap_{j=1}^{N_k} \{x\in X: {}_{k}p_j(x) < \alpha_k\}.
  \end{equation*}
  We showed that $G$ constains a \(\tau_{\scrP}\) neighborhood of $0$,
  and so \(\tau_{\scrQ} \subset \tau_{\scrP}\) and the topologies
  coincide.

 \itemch{b}
  Assume there exists \(p\in \scrQ\) and some \(M<\infty\) 
  such that \(|\Lambda x| \leq Mp(x)\) for  all \(x\in X\).
  Look at a base $0$ neighborhood \(V = \{x\in X: p(x)<1\}\).
  Let \(x\in V\), 
  clearly \(|\Lambda x| \leq Mp(x) \leq M\).
  Thus \(\Lambda\) is bounded on some neighborhood of $0$
  and by Theorem~11.8 \(\Lambda\) is continuous.

  Conversely, assume \(\Lambda\) is continuous. 
  By Theorem~1.18, there exist some
  neighborhood $W$ of 0 such that \(\Lambda\) is bounded on $W$ by $b$.
  By the topology construction, there exist a base neighborhood \(V\subset W\)
  such that
  \begin{equation*}
   V =  \cap_{j=1}^{n} \{x\in X: p_j(x) < \alpha_j\}
  \end{equation*}
  for some seminorms \(p_j\in \scrQ\) and \(\alpha_j > 0\).
  By induction in the hypothesis on \scrQ, 
  \(p = \max_{1\leq j\leq n} p_j \in \scrQ\).
  Put \(\alpha = \min_{1\leq j\leq n} \alpha_j\).
  Now 
  \begin{equation*}
   U = \{x\in X: p(x) < \alpha\} \subset V
  \end{equation*}
  is a neighborhood of $0$ and for any \(x\in U\), we have \(|\Lambda x| < M\).

  Now let \(x \in X\), there are two cases.
  \begin{itemize}
   \item[(i)]
     \(p(x)=0\).\newline
     Let \(c = \Lambda x\).
     If by negation \(c\neq 0\), then \(\Lambda(\alpha x/c) = \alpha\).
     Hence \(\alpha x/c \notin U\) but \(p(\alpha x/c) = (\alpha/c)p(x) = 0\)
     giving the \(\alpha x/c \in U\) contradiction.
   \item[(ii)]
    \(p(x)\neq 0\).\newline
     Now \(p(\alpha x/2p(x)) = \alpha/2\).
     Thus \(\alpha x/2p(x) \in U\).
     \begin{equation*}
     |\Lambda x| = | (2p(x)/\alpha) \Lambda (\alpha  x/2p(x))| 
      \leq (2p(x)b/\alpha).
     \end{equation*}
   \end{itemize}
   Putting \(M = 2b/\alpha\) and we have for all \(x\in X\)
   \(|\Lambda x| \leq Mp(x)\) as desired.
  
\end{itemize}

%%%%%%%%%%%%%% 9
\begin{excopy}
Suppose \label{ex:1:9}
 \begin{itemize}
  \itemch{a} $X$ and $Y$ are topological vector spaces,
  \itemch{b} \(\Lambda: X\rightarrow Y\) is linear,
  \itemch{c} $N$ is a closed subspace of $X$,
  \itemch{d} \(\pi:X \rightarrow X/N\) is the quotient map, and
  \itemch{e} \(\Lambda x = 0\) for every \(x\in N\).
 \end{itemize}
Prove that there is a unique \(f:X/N \rightarrow Y\) which satsifies
\(\Lambda = f \circ \pi\), that is,
\(\Lambda x = f(\pi(x))\) for all \(x\in X\).
Prove that this $f$ is linear and that \(\Lambda\) is continuous
if and only if $f$ is continuous. Also, \(\Lambda\) is open if and open if 
$f$ is open.
\end{excopy}

Let \(x+N\in X/N\) define \(f(x+N) = \Lambda x\).
To see that this is well defined, let \(x_1,x_2\in X\)
such that \(x_1 + N = x_2 + N \subset \Ker \Lambda\).
Now clearly \(x_2 - x_1 \in N\) and so
\begin{equation*}
f(x_1 + N) = \Lambda (x_1) = \Lambda(x_1) + \Lambda(x_2 - x_1) = 
  \Lambda(x_2) = f(x_2 + N).
\end{equation*}
The linearity  of $f$ is trivailly derived by \(\Lambda\).

Theorem~1.41 shows that \(\pi\) is a continuous and open mapping.
If $f$ is continuous, sp is then \(\Lambda = f \circ \pi\).
If $f$ is open, so is \(\Lambda = f \circ \pi\).

Conversely,

\begin{itemize}

\item
Assume \(\Lambda\) is continuous and let $W$ be any open set in $Y$.
We now show that 
\begin{equation} \label{eq:fw:Lambda}
f^{-1}(W) = \pi(\Lambda^{-1}(W)).
\end{equation}
We have 
\begin{eqnarray*}
x+N \in f^{-1}(W) 
  & \Leftrightarrow & \exists w\in W,\, f(x)=w \\
  & \Leftrightarrow & \Lambda x \in W \\
  & \Leftrightarrow & x \in \Lambda^{-1}(W) \\
  & \Leftrightarrow & x+N \in  \pi(\Lambda^{-1}(W))
\end{eqnarray*}

By definition \(U = \Lambda^{-1}(W)\) is open in $X$.
Hence by (\ref{eq:fw:Lambda}) and \(\pi\) being open,
\(f^{-1}(W)\) is open and $f$ is continuous.

\item 
Assume \(\Lambda\) is open and let $V$ be open in \(X/N\).
By definition of the quotient topological, \(U = \pi^{-1}(V)\) 
is open in $X$. Clearly \(f(V) = \Lambda(U)\) and thus $f$ is open.
\end{itemize}


%%%%%%%%%%%%%% 10
\begin{excopy}
Suppose $X$ and $Y$ are topological vector spaces, \(\dim Y = \infty\),
\(\Lambda: X\rightarrow Y\) is linear, and \(\Lambda(X)=Y\).
\begin{itemize}
 \itemch{a} Prove that \(\Lambda\) is an open mapping.
 \itemch{b}
   Assume, in addition, that the null space of \(\Lambda\) is closed,
   and prove that \(\Lambda\) is then continuous.
\end{itemize}
\end{excopy}

\begin{itemize}
 \itemch{a} 
  Let $V$ be a neighborhood of $0$ in $X$.
  Since translation is homeomorphism, it is sufficient to show
  that \(\Lambda V\) contains a neighborhood of $0$ in $Y$.
  Let \seqn{y} be a basis for $Y$
  and pick \seqxn\ such that \(\Lambda x_i = y_i\) for \(1\leq i \leq n\).
  Since scalar muliplication is continuous there are \seqn{\epsilon}
  such that \((-\epsilon_i,+\epsilon_i)\cdot x_i \subset V\).
  Define the ``sub-box''
  \begin{equation*}
   B = \left\{\sum_{i=1}^n a_i x_i: |a_i| < \epsilon_i\right\} \subset V.
  \end{equation*}
  Clearly 
  \begin{equation*}
   \Lambda B = \left\{\sum_{i=1}^n a_i y_i: |a_i| < \epsilon_i\right\}
  \end{equation*}
  is an open neighborhood of $0$ in $Y$.
  
 \itemch{b}
  We have a vector space isomorphism \(X/N \cong Y\).
  Since $Y$ is finite dimensional, this must also be topological homeomorphism.
  Using the previous exercise results and notations, 
  the isomorphism map is $f$, being continuous, so is \(\Lambda\).
  
  % Note: Here we did not use the fact that \(\Lambda\) is open.
\end{itemize}

%%%%%%%%%%%%%% 11
\begin{excopy}
If $N$ is a subspace of a vector space $X$, the 
\index{codimension}
\emph{codimension} id $N$ in $X$ is, by definition, 
the dimension of the quotient space \(X/N\).
Suppose \(0<p<1\) and prove that every subspace of a finite codimension
is dense in \(L^p\). (see Section~1.47).
\end{excopy}

Let $N$ be a subspace of finite codimension in \(L^p\) where \(0<p<1\).
Assume by negation that $N$ is not dense in \(L^p\).
Let \(\pi: L^p \rightarrow L^p/\overline{N}\) be the projection map.
Thus \(\overline{N} \subsetneq L^p\) and \(Y = L^p/\overline{N}\)
is a vector space of finite dimension. By previous exercise
and Theorem~1.21, $Y$ is isomorphic to \(\C^n\) and \(n\geq 1\). 
Let $G$ be some open convex set in \(\C^n\) not containing the origin.
For example (identifying $Y$ with  \(\C^n\) )
\begin{equation*}
G = \{(z_i)\in \C^n: 1<\Re(z_i),\Im(z_i)<2\}.
\end{equation*}
Now \(H = \pi^{-1}(G)\) is convex and open in \(L^{-1}\). 
But obviously \(\emptyset \neq H \subsetneq L^p\).
Cntradiction to the result in pages~35--36 \cite{RudinFA79} showing that
\(L^p\)
has no non trivial convex open subsets
when \(0<p<1\).


%%%%%%%%%%%%%% 12
\begin{excopy}
Suppose
 \(d_1(x,y) = |x-y|\), 
 \(d_2(x,y) = |\phi(x) - \phi(y)|\),
where \(\phi(x) = x/(1+|x|)\).
Prove that \(d_1\) and \(d_2\) are metrics in \R\ that induce
the same topology, although 
\(d_1\) is complete and \(d_2\) is not.
\end{excopy}

The identity map \((\C,\tau_{d_1}) \rightarrow (\C,\tau_{d_2})\) 
is clearly homeomorphism, thus the topologies are identical.
Now the sequence \(1,2,3\ldots\) is 
\index{Cauchy!sequence} 
Cauchy in the \(d_2\) metric space, but not in the \(d_1\) metric space.

%%%%%%%%%%%%%% 13
\begin{excopy} 
Let 
\label{ex:1:13}
$C$ be the vector space of all complex continuous functions on \([0,1]\).
Define
\begin{equation*}
 d(f,g) = \int_0^1 \frac{|f(x) - g(x)|}{1 + |f(x) - g(x)|} dx.
\end{equation*}
Let \((C,\sigma)\) be $C$ with the topology induced by the metric.
Let \((C,\tau)\) be the topological vector space defined by the seminorms
\begin{equation*}
 p_x(f) = |f(x)| \qquad (0\leq x \leq 1).
\end{equation*}
in accordance with Theorem~1.37.

\begin{itemize}
 \itemch{a}
  Prove that every \(\tau\)-bounded set in $C$ is also \(\sigma\)-bounded
  and that the identity map 
  \(\id:(C,\sigma) \rightarrow (C,\tau)\) therefore carries bounded ses
  into bounded sets.
 \itemch{b}
  Prove that
  \(\id:(C,\sigma) \rightarrow (C,\tau)\) is nevertheless not continuous,
  although it is sequentially continuous (by Lebesgue's convergence theorem).
  Hence \((C,\tau)\) is not metrizable. 
  (See Appendix A6, or Theorem~1.32.) Show also directly that \((C,\tau)\)
  has no countable local base.
 \itemch{c}
  Prove that every continuous linear functional on \((C,\tau)\) is of the form
  \begin{equation*}
    f = \sum_{i=1}^n c_i f(x_i)
  \end{equation*}
  for some choices of \seqxn\ in \([0,1]\) and some \(c_i\in \C\).
 \itemch{d}
  Prove that \((C,\sigma)\) contains no convex open sets other than 
  \(\emptyset\) and $C$.
 \itemch{e}
  Prove that \(\id:(C,\sigma) \rightarrow (C,\tau)\) is not continuous.
\end{itemize}
\end{excopy}


\begin{itemize}
 \itemch{a}
   \textbf{Note}: If \(\sigma\) would be replaced by a \(\sigma'\) 
   topology induced
   by \(L^1\) (using a simpler integrand) the identity mapping does \emph{not}
   always map \(\tau\)-bounded  sets to \(\sigma'\)-bounded sets.
   For let \(a_n = 1-1/(n+1)\) and \(d_n = (a_{n+1} - a_n)/2\).
   Now define \(f_n\in C[0,1]\) as following:
   \begin{equation*}
     f_n(x) = \left\{\begin{array}{ll}
                    0              & \quad x < a_n - d_n \\
                    (n/d_n)|x - a_n| & \quad a_n - d_n \leq x \leq a_n + d_n \\
                    0  & \quad x > a_n + d_n \\
                    \end{array}\right.
   \end{equation*}
   Clearly \(f_n\)'s are continuous, and for any \(x\in [0,1]\),
   the set \(F = \{f_n(x)\}\) is bounded. Thus $F$ is \(\tau\)-bounded,
   but since \(\int_0^1 |f_n|dm = 2n\) the set $F$ is not \(\sigma'\)-bounded.



   Back to the actual exercise.
   Let $E$ be a bounded set in the \(\tau\) topology.
   Assume by negation, $E$ is \emph{not} \(\sigma\)-bounded.
   Hence there exist some $0$-neighborhood 
   \begin{equation*}
   V_\epsilon = \{f\in C[0,1]: \sigma(0,f) < \epsilon\}
   \end{equation*}
   such that for any \(n > 0\), there exists \(f_n\in E \setminus nV_\epsilon\).
   Therefore, \(f_n / n \notin V\), that is
   \begin{equation} \label{eq:sigma:fn:eps}
    \sigma(f_n,0) = \int_0^1 |f_n/n|/(1 + |f_n|/n) dm 
      % = \int_0^1 |f_n|/(n + |f_n|) dm 
      \geq \epsilon.
   \end{equation}
   
   Since $E$ is \(\tau\) bounded, the set \(\{f_n(x)\}\) is bounded.
   Thus \(f_n/n \rightarrow 0\) pointwise and so 
   \begin{equation*}
   0 \leq \sigma(f_n,0) = 
   |f_n(x)/n|/\left(1 + |f_n(x)/n|\right)  \leq |f_n(x)/n| \rightarrow 0
   \end{equation*}

   Now since \(f_n \leq 1\), by Lebesgue's \index{Lebesgue}
   bounded convergence theorem, \(\sigma(f_n,0) \rightarrow 0\)
   which contradicts (\ref{eq:sigma:fn:eps}).

 \itemch{b}
   Let \(V = \{f\in C: d(f,0)<1/2\}\).
   This $0$-neighborhood does not contain any base \(\tau\)-neighborhood
   which are of the form 
   \begin{equation*}
     U = \bigcap_{i=1}^n p_{x_i}^{-1}\left((-\epsilon_i,\epsilon_i)\right)
   \end{equation*}
   (We can always find \(x \in [0,1]\setminus\{\seqxn\}\)).

   Sequential continuity is trivial 
   by Lebesgue's \index{Lebesgue} dominated theorem.

   If by negation \((C,\tau)\) is metrizable, then 
   Theorem~1.32 ({\small(d)\(\Rightarrow\)(a)}) \cite{RudinFA79}
   leads to a contradiction that the identity \emph{is} continuous.
   

   \textbf{Non existence of countable local base for \((C,\tau)\):}\newline
   For any \(\tau\)-open set $U$, the set \(U_x = \{|f(x)|: f\in U\}\)
   is not bounded except for \emph{finite} number of \(x_i\ni[0,1]\).
   If there would have been a countable local base, we could have
   found \(x\in[0,1]\) such that \(V(p_x,1)\) does not contain any
   member of the basis.
 
 \itemch{c}
   Let \(\Lambda\) such a functional. The set \(\Lambda^{-1}(-1,1)\)
   contains some base open set of the form 
   \begin{equation*}
    V = \{f\in C: \forall 1\leq i\leq n, |f(x_i)| < \epsilon_i\}
      \qquad \textrm{where}\; \epsilon_i > 0.
   \end{equation*}
   It is easy to see that if $f$ vansihes on \seqxn\ then \(\Lambda f = 0\).
   By looking at some \(e_i\in C\) such that \(e_i(x_j) = \delta_{i,j}\)
   we see that \(\Lambda f\) is determined, and in fact a linear combination
   of the values of \(\{f(x_i)\}_{i=1}^n\).

 \itemch{d}
   Let \(V\neq\emptyset\) be open and convex.
   \Wlogy\ \(0\in V\). We will show that \(V=C\).
   By the assumptions, there exists \(B(0,r)\subset \) with \(r>0\).
   We construction \(\{f_i\}_{i=1}^n\) such that \(f=\sum_{i=1}^n f_i\)
   and \(nf_i\in B(0,r)\). By convexity \(f\in V\) and we will be done.


   \textbf{Construction by Split of Unity}:\newline
   Take some integer $n$ such that \(n > 3/2\epsilon\).
   We will define $n$ continuous functions 
   \(e_i: [0,1] \rightarrow [0,1]\) such that 
   \(\sum_{i=1}^n e_i = 1\) and 
   their support  \(\supp(e_i) = [2i/(2n+1),(2i+3)/2n+1]\) as follows.
   Let \(d = 1/(2n+1)\) and \(I_i = [a_i,b_i] = [2(i-1)d, (2i+1)d]\)
   for \(1\leq i \leq n\).
   Clearly \(|I_i| = 3d = 3/(2n+1\) and \(|I_i \cap I_{i+1}| = d\).
   Now
   \begin{equation*}
     e_i(x) = \left\{\begin{array}{ll}
                     0          & x  \leq a_i \\
                     (x-a_i)/d  & a_i   \leq x \leq a_i + d \\
                     1          & a_i+d \leq x \leq b_i - d \\
                     (x-b_i)/d  & b_i+d \leq x \leq b_i \\
                     0          & b_i \leq x \\
                     \end{array}\right. \qquad (1\leq i \leq n).
   \end{equation*}
   Now clearly the \(e_i\) functions satisfy the requirements.
   Now we put \(f_i = n f e_i\) and 
   thus \(d(0,f_i) < 3d = 3/(2n+1) < \epsilon\) as desired.

   

 \itemch{e}
   Let \(U=\{f\in C: |f(0)| < 1\}\). It is clearly 
   an open neighborhood of $0$.
   But $U$ cannot be open in \((C,\sigma)\), since for any \(\epsilon>0\)
   the base \(\sigma\)-neighborhood
   \(V_\epsilon = \{f\in C: d(f,0)<\epsilon\}\)
   has \(g\in C\setminus U\) defined as:
   \begin{equation*}
   g(x) = \left\{\begin{array}{ll}
                2(\epsilon-x)/\epsilon & x \leq \epsilon \\
                0                      & x \geq \epsilon 
                \end{array}\right. .
   \end{equation*}
\end{itemize}

%%%%%%%%%%%%%% 14
\begin{excopy}
Put \(K=[0,1]\) and defined \(\scrD_K\) as in Section~1.46.
Show that the following three families of seminorms (where \(n=0,1,2,\ldots\))
define the same topology on \(\scrD_K\), 
if \(D = d/dx\):
\begin{itemize}
 \itemch{a} \(\|D^n f\|_\infty = \sup\{|D^n f(x)|: -\infty < x < \infty\}\).
 \itemch{b} \(\|D^n f\|_1  = \int_0^1 |D^n f(x)|dx\).
 \itemch{c} \(\|D^n f\|_2  = \left\{\int_0^1 |D^n f(x)|^2dx\right\}^{1/2}\).
\end{itemize}
\end{excopy}

Since the measure of \([0,1]\) is $1$, we easily see that
for any measurable \(g(x)\) on \([0,1]\) we have
\(\|g\|_p \leq \|g\|_\infty \) for \(1\leq p \leq \infty\). 
Setting \(g=D^n f\) we get
 \(\|D^n f\|_2 \leq \|D^n f\|_\infty\).

\index{Jensen!inequality}
By Jensen's inequality, 
or more directly from 
\index{Schwartz!inequality}
Schwartz inequality (See \cite{RudinRCA80}, Theorems~3.3, 3.5),
we also see that \(\|D^n f\|_1 \leq \|D^n f\|_2\).

Thus, \( \|\cdot\|_1 \leq \|\cdot\|_2 \leq \|\cdot\|_\infty\)
and by compaing at balls of the above seminorm
we see that 
the topology of 
\(\|\cdot\|_\infty\) (c) is finer than the topology of \(\|\cdot\|_2\) (b)
and similarly
the topology of 
\(\|\cdot\|_2\) (b) is finer than the topology of \(\|\cdot\|_1\) (a).
Therefore, it suffices to show that the topology of (a) is finer than or equal
to that of (c).

Consider a basis open set of the (c)-topology 
\begin{equation*}
 V = \{f\in \scrD_K: \|D^n f\|_\infty < \epsilon\}.
\end{equation*}
It is easy to see that there exist \(\epsilon_1,\epsilon_2>0\) such that
\begin{equation} \label{eq:1:UinV}
 U = \{f\in \scrD_K: 
          \|D^n f\|_1 < \epsilon_1 
          \,\wedge\,
          \|D^{n+1} f\|_1 < \epsilon_2 
      \} \subset V.
\end{equation}

This is by applying the equality
\begin{equation*}
D^n f(x_1) = D^n f(x_0) + \int_{x_0}^{x_1} D^{n+1} f(x)dx
\end{equation*}
we can actually take \(\epsilon_1 = \epsilon_2 = \epsilon/2\) as follows.

Let \(f\in U\).
Pick some \(x_0\in[0,1]\) 
such that \(|D^n f(x_0)| \leq \int_0^1 D^{n+1} f(x)dx\)
that obviously must exist. Now
\begin{equation*}
\sup_{x\in[0,1]} |D^n f(x)| \leq |D^n f(x_0)| + \int_0^1 |D^{n+1} f(x)|dx
 \leq \epsilon_1 + \epsilon_2 = \epsilon.
\end{equation*}
Thus, \(f\in V\) and (\ref{eq:1:UinV}) must hold.

%%%%%%%%%%%%%%
\begin{excopy}
Prove that the spaces \(C(\Omega)\) (Section~1.44) do not have the
\index{Heine-Borel}
Heine-Borel property.
\end{excopy}

Since the cardinality of \(\Omega\) is greater than \(\aleph_0\)
and it is a union of countable number of compact sets \(K_i\), we can
take some compact set $K_n$ with infinitely many points,
say \(\{x_i\}\subset K\). Choose continuous 
functions \(\{f_i\}: \Omega\rightarrow [0,1]\) 
such that
\(f_k(x_i) = 0\) if \(i < k\) and \(f_k(x_k) = 1\).
The set \(F = \{f_i:i\in \N\}\) is bounded, it has no limit
since the with seminorm \(p_n(f) = \sup\{|f(x)|: x\in K_n\}\)
the base neighborhood \(V = V(p_n,1/2)\) contains none fo the \(f_i\) functions.
Thus $F$ is trivially closed. Now the infinite family \(\{f_i+V\}\) of
opens sets covers $F$, but if \(i\neq j\) then \(f_j\notin f_i+V\)
and there is no subcovering, in particulat no finite subcovering.
Hence $F$ is not compact.



%%%%%%%%%%%%%%
\begin{excopy}
Prove that the topology of \(C(\Omega)\) does not depend on the particular
choice of \(\{K_n\}\), as long as this sequence satisfies the conditions
specified in Section~1.44. Do the same for \(C^\infty(\Omega)\) (Section~1.46).
\end{excopy}

Let \(\{K'_i\}\) be an alternative set of compact sets with similar
requirements, that gives the corresponding \(\{{p'}_i\}\) seminorms.
Now since \(K_i = \inter{K_{i+1}}\) we actually have an open covering
\(\Omega = \cup_i \int K_i\). Hence for each compact \(K'_i\)
there exist some index \(j\geq1\) such that \(K'_i \subset K_j\).
Thus for any \(f\in C(\Omega)\), we have
\({p'}_i(f) \leq p_j(f)\). this shows that the (``origina'') topology
is finer or equal to that indeuced by \(\{K'_i\}\). By symmetrical argument
we show the opposite inclusion of topologies, hence the topologies
are independent of the choice of the compact increasing covering.

Now for the \(C^\infty(\Omega)\). We generlize the set of seminorms
from the original 
\begin{equation} \label{eq:pn:orig}
 p_N(f) = \max\{|D^\alpha f(x)|: x\in K_N, |\alpha| \leq N\}
\end{equation}
as apears in section 1.46 of \cite{RudinFA79}, into
\begin{equation*}
 p_{M,N}(f) = \max\{|D^\alpha f(x)|: x\in K_N, |\alpha| \leq M\}
\end{equation*}
Now the set of seminorm is richer, but still induces the same topology
since for any \(f\in C^\infty\) we have
\begin{equation*}
 p_{m,n}(f) \leq p_{\max(m,n),\max(m,n)} (f) =  p_{\max(m,n)}(f).
\end{equation*}
The last, right expression should be referred as the ``original'' seminorm
as in (\ref{eq:pn:orig}).

Now we can vary the compact sets as was done 
in the beginning of this exercise with \(C(\Omega)\)
\emph{without} effecting the order of \(\alpha\) derivations.
This will show that the choice of compact sets has 
has no effect on the topology of \(C^\infty(\Omega)\) as well.

%%%%%%%%%%%%%%
\begin{excopy}
In the setting of Section~1,46, prove that \(f\rightarrow D^\alpha f\)
is a continuous mapping of 
\(C^\infty(\Omega)\) into \(C^\infty(\Omega)\)  and also of
\(\scrD_K\) into \(\scrD_K\),
for every multi-index \(\alpha\).
\end{excopy}

Almost by definition. Let \(a=|\alpha\) and look at some base open set
\begin{equation*}
 V = \{f\in C^\infty(\Omega): p_n(f)< 1/n\}
\end{equation*}
(Note that \(V_n\) are getting finer as $n$ increases).
We need to show that the inverse image set \(W = (D^\alpha)^{-1}(V)\) is open.

But actually
\begin{eqnarray*}
 W 
 &=&  (D^\alpha)^{-1}(V) \\
 &=&  \{f\in C^\infty(\Omega): D^\alpha f \in V\} \\
 &=&  \{f\in C^\infty(\Omega): p_{n+a}(D^\alpha f)< 1/(n+a)\}
\end{eqnarray*}
which is another base open set.

Thus \(\alpha\)-derivation is continuous on \(C^\infty(\Omega)\)
and the same goes for \(\scrD_K\) for it is merely
a restriction. we just need to observer that 
\((D^\alpha)^{-1}(\scrD_K) \subset \scrD_K)\).



%%%%%%%%%%%%%%
\begin{excopy}
The seminorms \label{ex:1:18}
\begin{equation*}
 p_n(f) = \sup\{|f(x)|: -n \leq x \leq n\}
\end{equation*}
induce the metric
\begin{equation*}
  d(f,g) = \sum_{n=1}^\infty \frac{2^{-n} p_n(f-g)}{1 +  p_n(f-g)}
\end{equation*}
in the space \(C(R)\); compare Section~1.46 and remark~(c) of
Section~1.38.
Define
\begin{equation*}
 f(x) = \max(0,1-|x|), \qquad g(x) = 100f(x-2), \qquad 2h = f+g,
\end{equation*}
and compute that
\begin{equation*}
 d(f,0)=\frac{1}{2}, \qquad d(g,0) = \frac{50}{100}, 
                     \qquad d(h,0) = \frac{1}{6}+\frac{50}{102}.
\end{equation*}
The balls with radius \(\frac{1}{2}\) are therefore not convexity, although $d$
is compatible  with the usual locally convex topology of \(C(R)\).
\end{excopy}

% Here the compacts are \(K_n = [-n,n]\).
The balls (\(0<r<1\)) are \emph{not} convex.
Let $N$ be such that 
\begin{equation} \label{eq:1:r:2N}
2^{-N} \leq r \leq 2^{-(N-1)}.
\end{equation}
We construct monotonic functions $f$, $g$ such that
\begin{equation*}
p_n(f) = p_n(g) = f(n), \qquad d(g,0) \leq d(f,0) = r
\end{equation*}
but for \(h=(f+g)/2\), we have \(d(h,0)> r\).

Define
\begin{equation*}
f(x) = \left\{\begin{array}{lc}
              0               & x \leq N - 1 \\
              \alpha(x-(N-1)) & N -1 \leq x \leq N  \\
              \alpha          & x \geq N 
              \end{array}\right..
\end{equation*}
Where \(\alpha = r/(2^{-N+1} - r)\) 
and so 
\begin{equation*}
d(f,0) = \sum_{n=N}^\infty 2^{-n}\alpha/(1+\alpha) = 2^{-N+1}\alpha(1+\alpha)
       = r.
\end{equation*}

The function $g$ will be defined in two stages. First
\begin{equation*}
g(x) = \left\{\begin{array}{lc}
              0               & x \leq N - 1 \\
              \beta(x-(N-1)) & N -1 \leq x \leq N  \\
              \end{array}\right.
\end{equation*}
Where \(\beta\) satisfies the equation
\begin{equation*}
 2^{-N}\beta/(1+\beta) = r - 2^{-N}
\end{equation*}
Which must exist by (\ref{eq:1:r:2N}).

We can already have the estimation
\begin{equation*}
d(g,0) \leq 2^{-N}\beta/(1+\beta) + 2^{-N} \leq r.
\end{equation*}

The definition of $g$ will be completed  for \(x\geq N\) once  $h$
is defined. 
We require \(h=(f+g)/2\) that is \(g(x)=2h(x)-\alpha\) for \(x\geq N\).

For \(x\leq N\) we should define 
\begin{equation*}
h(x) = (f(x)+g(x))/2.
\end{equation*}

Now \(r - 2^{-N} < r/2\) and so \(\beta < (\alpha+\beta)/2 < \alpha\)
(since \(x \rightarrow x/(1+x)\) is increasing).
Hence, \(\beta < h(N) = (\alpha+\beta)/2\), and so 
\begin{equation*}
\epsilon = 2^{-N}h(N)/(1+h(N)) - (r - 2^{-N}) > 0.
\end{equation*}
Let \(\gamma\) satisfy 
\begin{equation*}
 \sum_{n=N+1}^\infty 2^{-n} \gamma/(1+\gamma) = 2^{-N} \gamma/(1+\gamma) 
 > 2^{-N} - \epsilon/2.
\end{equation*}
Now we complete the define of $h$
\begin{equation*}
 h(x) = \left\{\begin{array}{lc}
               (\gamma - \beta)(x-N) + \beta &  N \leq x \leq N+1 \\
               \gamma                        & x \geq N + 1
               \end{array}\right..
\end{equation*}

This guarantee that \(d(h,0) \geq r\) and the ball \(B(0,r)\) is not convex.

\textbf{Computations:}

\begin{itemize}

 \item 
 We observe that \(p_n(f) = 1\) for all $n$.
 \begin{equation*}
 d(f,0) = \sum_{n=1}^\infty 2^{-n} 1/(1 + 1) = \sum_{n=0}^\infty 2^{-n} = 1/2
 \end{equation*}

 \item
 We observe that \(p_1(g) = 0\) and \(p_n(g) = 100\) for \(n\geq 1\).
 \begin{equation*}
 d(g,0) = \sum_{n=2}^\infty 2^{-n} 100/(100 + 1) = 
      (1/2)\cdot (100/101) = 50/101.
 \end{equation*}


 \item
 We observe that \(p_1(h) = 1/2\) and \(p_n(h) = 50\) for \(n\geq 2\).
 \begin{eqnarray*}
 d(h,0) 
  &=& (1/2)\cdot\frac{1/2}{1+1/2} + \sum_{n=2}^\infty 2^{-n} \cdot 50/(50+1) \\
  &=& (1/2)\cdot(1/3) + (1/2)\cdot(50/51) = 1/6 + 50/102 = 67/102.
 \end{eqnarray*}

\end{itemize}


%%%%%%%%%%%%%% 19
\begin{excopy}
Suppose $M$ is a dense subspace of a topological vector space $X$,
$Y$ is an $F$-space, and \(\Lambda:M\rightarrow Y\) is a continuous
(relative to the topology that $M$ inherits from $X$) and linear.
Prove that \(\Lambda\) has a continuous linear extension 
\(\tilde{\Lambda}: X\rightarrow Y\).

\qquad \emph{Suggestion:} Let \(V_n\) be balanced neighborhoods of $0$ in $X$
such that \(V_n + V_n \subset V_{n-1}\) and such that 
\(d(0,\Lambda x) < 2^{-n}\) if \(x\in M\cap V_n\).
If \(x\in X\) and \(x_n \in (x+V_n)\cap M\), show that
\(\{\Lambda x_n\}\) is a Cauchy \index{Cauchy!sequence} sequence in $Y$, 
and define 
\(\tilde{\Lambda} x\) to be its limit.
Show that \(\tilde\Lambda\) is well defined, that 
\(\tilde{\Lambda} x = \Lambda x\) if \(x\in M\), and that \(\tilde\Lambda\)
is linear and continuous.
\end{excopy}

Using the suggestion. 
For any \(n > 0\) we can find a and open balanced  neighborhood \(V_n\)
such that (distance of point to set) \(d(0,\Lambda V_n) < 2^{-n}\)
and (for \(n>1\)) we also have \(V_n + V_n \subset V_{n-1}\).

Let \(x\in X\setminus M\). By density of $M$
we can find a sequence \(\{x_n\}\) such that
\(x_n \in (x+V_n)\cap M\). 
% We will show that \(\{\Lambda x_n\}\) is a Cauchy sequence.
Assume \(m<n\), then 
\begin{equation*}
x_m - x_n \in (x + V_m) - (x + V_n) = V_m - V_n = V_m + V_n
 \subset V_m + V_m \subset V_{m-1}.
\end{equation*}
Hence \(d(\Lambda x_m, \Lambda x_n) = d(0, \Lambda (x_m - x_n)) < 2^{m-1}\)
and \(\{\Lambda x_n\}\) is a Cauchy sequence.
We define \(\tilde{\Lambda}x\) to be the limit. 
If \(\{w_n\}\) is another Cauchy sequence, such that \(w_n\in x+V_n\)
then it is easy to see that the comobined sequence:
\begin{equation*}
 \Lambda x_1, \Lambda w_1, \Lambda x_2, \Lambda w_2, \ldots 
 \Lambda x_n, \Lambda w_n, \ldots
\end{equation*}
is a Cauchy sequence as well. Since its limit is the same as the limit
of any of its subsequences, the extension \(\tilde{\Lambda}\) is well defined.
Continuity is immediate from the density of $M$.
Since the addition in $X$ and scalar muliplication are continuous,
if \(x_m\rightarrow x\) and \(y_n \rightarrow y\) 
where \(x_m,y_n\in M\) then
it is easy to see that 
\begin{equation*}
\lim_{m,n\rightarrow \infty} \Lambda(ax_m + by_n) 
= \lim_{m,n\rightarrow \infty} a\Lambda x_m + b \Lambda y_n
= a \lim_{m\rightarrow \infty} \Lambda x_m + 
  b \lim_{n\rightarrow \infty} \Lambda y_n.
\end{equation*}
and linearity of the extension is established.
 

%%%%%%%%%%%%%% 20
\begin{excopy}
For each real number $t$ and each integer $n$, define \(e_n(t) = e^{int}\),
and define 
\begin{equation*}
  f_n = e_{-n} + ne_n \qquad (n=1,2,3,\ldots).
\end{equation*}
Regard these functions as members of \(L^2(-\pi,\pi)\).
Let \(X_1\) be the smallest closed subspace of \(L^2\) that contains 
\(e_0,e_1,e_2,\ldots\), and let \(X_2\) be the smallest closed subspace
of \(L^2\) that contains \(f_1,f_2,f_3,\ldots\). Show that \(X_1+X_2\)
is dense in \(L^2\) but not closed.
For instance the vector
\begin{equation*}
 x = \sum_{n=1}^\infty n^{-1}e_{-n}
\end{equation*}
is in \(L^2\) but not in \(X_1+X_2\). (Compare with Theorem~1.42.)
\end{excopy}

For \(n>0\), we have \(e^{i(-n)t} = f_n - n e_n\) and so
the subspace \(X_1+X_2\) is dense in \(L^2\) since it contains 
all of \(\{e^{int}\}\).


The fact that \(x\in L^2\) is application of the 
\index{Riesz-Fischer}
\emph{Riesz-Fischer} theorem (see \cite{RudinRCA80} 4.17, 4.26).
The analysis there also shows that \(\{e_n\}_{n\in\Z}\) are
linearly independent.


Let us now show that \(X_1\cap X_2 = \{0\}\).
Let \(g(t)\in X_1\cap X_2\) and \(g(t)\neq 0\).
We have a unique representation \(g=\sum_{n\in\Z} a_n e_n\),
where as we know 
\(a_n = \langle g,e_n\rangle = \int_{-\pi}^{+\pi} g(t)e^{int}dt\).
By negation, \(x\neq 0\) and so there must be at least some \(n\in\Z\)
such that \(a_n\neq 0\).
There are three cases:

\textbf{Case (i)}:  There is some \(j<0\) for which \(a_j\neq 0\).
For any finite linear combination of \(X_1\)'s basis \(\{e_k\}_{k\geq 0}\),
that is \(u = \sum_{k=0}^m u_i e_i\) we have

\begin{eqnarray*}
\|g - u\|
&=& \left\|\sum_{n\in Z} a_i e_n - \sum_{k=0}^m u_k e_k\right\| \\
&=& \left\|\left(\sum_{n<0} a_i e_n + \sum_{n\geq 0} a_i e_n\right)
          - \sum_{k=0}^m u_k e_k\right\| \\
&\geq & \left\| \sum_{n<0} a_n e_n\right\| \\
&\geq & | a_j | > 0
\end{eqnarray*}

Thus \(d(g,X_1) \geq |a_j| > 0\) contradicts the assumption of \(g\in X_1\).

\textbf{Case (ii)}:  
For all \(n<0\), we have \(a_n=0\), but \(a_0\neq 0\).
For any finite linear combination of \(X_2\)'s basis \(\{f_k\}_{k\geq 1}\),
that is \(w = \sum_{k=1}^m w_k f_k\) we have
we have:
\begin{equation*}
\|g - w\| \geq |a_0| > 0.
\end{equation*}
Thus \(d(g,X_2) \geq |a_0| > 0\) which contradicts 
the assumption of \(g\in X_2\).

\textbf{Case (iii)}:  
For all \(n\leq0\), we have \(a_n=0\). Let \(j \geq 1\) be the minimal
such that \(a_j \neq 0\).
Similarly, For any finite linear combination of \(\{f_k\}_{k\geq 1}\),
that is \(w = \sum_{k=1}^m w_k f_k\) we have

\begin{eqnarray}
\|g - w\| 
&=& \left\|\sum_{n\in Z} a_i e_n - \sum_{k=1}^m w_k f_k\right\| \notag \\
&=& \left\|\sum_{n\geq 0} a_i e_n - 
           \sum_{k=1}^m (w_k e_{-k} + kw_k e_k)\right\|  \notag \\
&\geq& \left\| a_j e_j - (w_j e_{-j} + jw_j e_j)\right\| \label{eq:justj} \\
&=&  \left( (a_j - jw_j)^2 + w_j^2\right)^{1/2}  \label{eq:g-w:last}
\end{eqnarray}

We get the inequality (\ref{eq:justj}) by restricting our attention
to the projection over \(\langle e_j \rangle\).
We observe the last expression with the sqaure root.
As a function of \(w_j\), it is a quadratic polynomial.
\begin{equation*}
p(w_j) = (j^2+1)w_j^2 + 2ja_j w_j + a_j^2
\end{equation*}
whose determinant is 
\begin{equation*}
\Delta(p) = 4j^2a_j^2 - 4(j^2+1)a_j^2 = -4a_j^2 < 0.
\end{equation*}
Thus $p$ has no real solutions, and 
since \(p(0) = a_j^2 > 0\), the last term in (\ref{eq:g-w:last})
is positive and has a lower bound independent of \(w_j\).
Again showing \(d(g,X_2) > 0\) and contradicting the assumption
of \(g\in X_2\).

Now that we establish that \(L^2\) is an (inner) direct sum
of \(X_1 + X_2\) we proceed with a lemma.

\begin{llem}
Let $X$ be a  topological vector space
and \(X_1\), \(X_2\) closed subspaces such that
\(X_1 + X_2 = X\) is an inner direct sum.
Then for \(i=1,2\), the mappings \(P_i: X \rightarrow X_i\)
defined by \(P_i(x) = x_i\) where \(x = x_1 + x_2\) 
\textnormal{(unique, by \(X_1+X_2\) being a direct sum)}
are continuous.
\end{llem}

\textbf{Proof.}
The mapping \(P_1\) gives
the algebraic isomorphism \(\pi_1: X/X_2 \cong  X_1\).
By the definition of the quotient topology of \(X/X_2\)
\(\pi_1\) is also a topological isomorphism.
By Exercise~\ref{ex:1:9}, the mapping \(P_1\) is continuous.
Analogically, \(P_2\) is continuous too.
\proofend

Now we put \(x_m = \sum_{n=1}^m n{-1}e_{-n}\)
Clearly \(x_m \rightarrow x\). If by negation \(X_2\) is closed,
then \(P_1\) --- defined as in the above lemma ---
is continuous. But \(n^{-1}e_{-n} = n^{-1}f_n - e_n\), thus
\begin{equation*}
x_m = \sum_{n=1}^m n{-1}e_{-n}
  = \sum_{n=1}^m n^{-1}f_n - e_n
\end{equation*}
and
\begin{equation*}
P_1(x_m)
  = P_1\left(\sum_{n=1}^m n^{-1}f_n \right) +
    P_1\left(\sum_{n=1}^m e_n \right) = \sum_{n=1}^m e_n.
\end{equation*}
Surely the last expression does not converge as \(m\rightarrow \infty\)
contradiction to \(P_1\) being continuous.


%%%%%%%%%%%%%%
\begin{excopy}
Let $V$ be a neighborhood of $0$ in a topological vector space $X$.
Prove that there is a real continuous functions $F$ on $X$ such that
\(f(0) = 0\) and \(f(x)=1\) outside $V$.
(Thus $X$ is a 
\index{completely regular}
\emph{completely regular} topological space).
\emph{Suggestion:} Let \(V_n\) be balanced neighborhoods of $0$ 
such that \(V_1+V_1 \subset V\) and
such that \(V_{n+1}+V_{n+1} \subset V_n\).
Construct $f$ as in the proof of Therefore~1.24.
Show that $f$ is continuous at $0$ and that
\begin{equation*}
 |f(x) - f(y)| \leq f(x-y).
\end{equation*}
\end{excopy}

We follow the suggestion. 
The existence of such neighborhoods \(\{V_i\}_{n=1}^\infty\)
is provided by \cite{RudinFA79} Sections~1.6 and~1.10.

We define 
\begin{equation*}
D = \left\{r\in \Q: 
      r = \sum_{n=1}^m c_n(r)2^{-n},\; \textrm{where}\;c_i(r)=0,1,\;
      1\leq m < \infty \right\}.
\end{equation*}


We put \(V_0 = V\) and define absorbing ``neighborhood'' function
\(A: D \rightarrow P(X)\):
\begin{equation*}
A(r) = \left\{\begin{array}{l@{\qquad}c}
  X & r = 1 \\
  \sum_{i=1}^m c_i V_i & r = \sum_{n=1}^m c_n(r)2^{-n} < 1
  \end{array}\right..
\end{equation*}

We now define the desired \(f: X \rightarrow [0,1]\) as:
\begin{equation*}
f(x) = \inf\{r\in D: x\in A(r)\}.
\end{equation*}

Clearly \(f(X) \subset [0,1]\) and
\(f(x) = 1\) if \(x\notin V\) and \(f(0) = 0\).


Let \(\epsilon > 0\), take $n$ such that \(2^{-n} < \epsilon\).
Now $f$ is continuous at $0$ because \(f(V) \subset (-\epsilon, +\epsilon)\).

Now we show the inequality. If \(f(x) < f(y)\) then it si trivial,
so we can now assume \(f(x)\geq f(y)\).
Since \(f(x) - f(y) < 1 - 0 = 1\),
we can exclude the trivial case of  \(f(x-y) \geq 1\) and
further assume that \(f(x-y) < 1\).

Now for any \(r,s\in D\) such that 
\(f(x-y)\leq r\) and
\(f(y)\leq s\) we have
\(x-y \in  A(r)\) and
\(y \in  A(s)\) and so \(x=(x-y)+y \in A(r) + A(s) \subset A(r+s)\)
which shows that \(f(x) \leq r + s\). Since $D$ is dense in  \([0,1]\)
we have \(f(x) \leq f(x-y) + f(y)\).

With the inequality established, we can show that $f$ is continuous
at any \(x_0\in X\) since if \(f(V) \subset (-\epsilon, \epsilon)\)
for a balanced neighborhood $V$ of $0$ 
then for \(x \in x_0 + V\) we have
\begin{equation*}
|f(x)- f(x_0)| \leq f(x - x_0) < \epsilon.
\end{equation*}


%%%%%%%%%%%%%% 22
\begin{excopy}
If $f$ is a complex  function defined on the compact interval 
\(I=[0,1]\subset \R\), define
\begin{equation*}
 \omega_\delta(f) = \sup \{|f(x) - f(y)|: |x-y|\leq \delta, x\in I, y\in I\}
\end{equation*}
If \(0<\alpha \leq 1\), the corresponding 
\index{Lipschitz space}
\emph{Lipschitz space} \(\Lip \alpha\) consists of all $f$ for which
\begin{equation*}
 \|f\| = |f(0)| + \sup\{\delta^{-\alpha} \omega_\delta(f): \delta > 0\}
\end{equation*}
is finite. Define
\begin{equation*}
 \lip \alpha = \{f\in \Lip \alpha: 
           \lim_{\delta\rightarrow 0} \{\delta^{-\alpha} \omega_\delta(f) = 0\}
\end{equation*}
Prove that \(\Lip \alpha\) is a 
\index{Banach}
Banach space and that \(\lip \alpha\) is a closed subspace of \(\Lip \alpha\).
\end{excopy}

The triangle inequality is trivial
from 
\begin{equation*}
|(f+g)(x) - (f+g)(y)| = |f(x) + g(y) - f(x) - f(y)| 
 \leq |f(x) - f(y)| + |g(x) - g(y)|
\end{equation*}
followed by
\begin{eqnarray*}
 \omega_\delta(f+g)
 &=& \sup \{|(f+g)(x) - (f+g)(y)|: |x-y|\leq \delta,\, x,y\in I\} \\
 &\leq& \sup \{|f(x)-f(y)| + |g(x)-g(y)|: |x-y|\leq \delta,\, x,y\in I\}\\
 &\leq&   \sup \{|f(x)-f(y)|: |x-y|\leq \delta,\, x,y\in I\}
        + \sup \{|g(x)-g(y)|: |x-y|\leq \delta,\, x,y\in I\}\\
 &=& \omega_\delta(f) + \omega_\delta(g).
\end{eqnarray*}
Thus
\begin{eqnarray*}
\|f+g\|
&=&    |(f+g)(0)| + \sup\{\delta^{-\alpha} \omega_\delta(f+g): \delta > 0\} \\
&\leq&    |f(0)| + \sup\{\delta^{-\alpha} \omega_\delta(f): \delta > 0\} +
          |g(0)| + \sup\{\delta^{-\alpha} \omega_\delta(g): \delta > 0\} \\
&\leq& \|f\| + \|g\|.
\end{eqnarray*}

Now for any \(f\in \C^{[0,1]}\) there exists \(x\in[0,1]\) such that
for \(0<\alpha \leq 1\),
\begin{equation*}
\|f\|_\infty = |f(x)| \leq 
 |f(0)| + |f(x) - f(0)| \leq |f(0)| + \omega_x(f) \leq 
 |f(0)| + x^{-\alpha} \omega_x(f) \leq \|f\|_\alpha
\end{equation*}
Therefore, the topology induced by this norm is stronger than 
the supremum norm and completeness is easily established.

For the last claim, let \(f_n\in \lip\alpha\)
such that 
\begin{equation*}
f_n \xrightarrow{\|\cdot\|_\alpha} f \in \Lip \alpha.
\end{equation*}
we have
\begin{eqnarray*}
\lim_{\delta\rightarrow 0} \delta^{-\alpha} \omega_\delta(f_n - f)
 &\leq& \sup_{\delta > 0} \delta^{-\alpha} \omega_\delta(f_n - f) \\
 &\leq&  \left| \sup_{\delta > 0} \delta^{-\alpha} \omega_\delta(f_n) -
                \sup_{\delta > 0} \delta^{-\alpha} \omega_\delta(f) \right|
       \xrightarrow{n\rightarrow\infty} 0.
\end{eqnarray*}
Hence \(f\in \lip\alpha\).

%%%%%%%%%%%%%%
\begin{excopy}
Let $X$ be a vector space of all continuous functions on the open segment
\((0,1)\). For 
\(f\in X\) and \(r>0\), let \(V(f,r)\) consist of all \(g\in X\) such that
\(|f(x)-g(x)|<r\) for all \(x\in(0,1)\).
Let \(\tau\) be the topology on $X$ that these sets \(V(f,r)\) generate.
Show that addition is \(\tau\)-continuous but scalar multiplication is not.
\end{excopy}

We have 
\begin{equation*}
V(f,r) + V(g,s) \subset V(f+g,r+s)
\end{equation*}
From which implies the continuity of the addition.

To show that scalar multiplication is not continuous, let's look
at the function \(f(x) = 1/x\) and look for a decreasing sequence
\(a_n \rightarrow 1\) such that \(a_n f \notin V(f,\epsilon)\) for any 
\(\epsilon > 0\).
Let \(a_n = (n+1)/n\). Now \(a_nf(x)-f(x) = 1/(nx)\)
Clearly for any \(n<\infty\) we can find \(x_n\in(0,1)\)
such that 
\(x_n < 1/(n\epsilon)\). Now 
\(1/(nx_n) > \epsilon\) and thus \(a_nf(x_n)-f(x_n) > \epsilon\)
and so \(a_nf \notin V(f,\epsilon)\).


%%%%%%%%%%%%%% 24
\begin{excopy}
Show that the set $W$ that occurs in the proof of Theorem~1.14 need
not be convex, and that $A$ need not be balanced unless $U$ is convex.
\end{excopy}


The balls of the space in Exercise~\ref{ex:1:18} are balanced
but not convex. so are their interiors which form 
a local base for the topology, $W$ could be such a ball.

Consider the space \(\C^2\).
For \(n=1,2\) define
\begin{equation*}
 R_n = \left\{(z,0) \in \C^2: |z| = n\right\}.
\end{equation*}
Let $V$ be a \((0,0)\)-neighborhood such that
\begin{equation*}
 R_1 \cap V = \emptyset \qquad \textnormal{and} \qquad R_2 \subset V.
\end{equation*}
Now we ``symmetrize'' $V$ by \(U = \cup_{|\alpha|=1} V\).
We can see that also for $U$
\begin{equation*}
 R_1 \cap U = \emptyset \qquad \textnormal{and} \qquad R_2 \subset U.
\end{equation*}
Clearly \(A=\cap_{|\alpha|=1} U = U\), but $U$ is not balanced,
Since \(R_1 \subset (1/2)U\).

%%%%%%%%%%%%%%%
\end{enumerate}
%%%%%%%%%%%%%%%
