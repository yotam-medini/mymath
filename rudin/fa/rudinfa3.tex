%%%%%%%%%%%%%%%%%%%%%%%%%%%%%%%%%%%%%%%%%%%%%%%%%%%%%%%%%%%%%%%%%%%%%%%%
%%%%%%%%%%%%%%%%%%%%%%%%%%%%%%%%%%%%%%%%%%%%%%%%%%%%%%%%%%%%%%%%%%%%%%%%
%%%%%%%%%%%%%%%%%%%%%%%%%%%%%%%%%%%%%%%%%%%%%%%%%%%%%%%%%%%%%%%%%%%%%%%%
\chapterTypeout{Convexity}

%%%%%%%%%%%%%%%%%%%%%%%%%%%%%%%%%%%%%%%%%%%%%%%%%%%%%%%%%%%%%%%%%%%%%%%%
%%%%%%%%%%%%%%%%%%%%%%%%%%%%%%%%%%%%%%%%%%%%%%%%%%%%%%%%%%%%%%%%%%%%%%%%
\section{Notes}

\subsection{Missing \texorpdfstring{$t$}{t}}

In Theorem~3.2 Assumption \ich{b}, the condition misses some $t$,
it should be
\begin{equation*}
\ldots \qquad\textnormal{and}\qquad p(tx) = tp(x)
\end{equation*}


\subsection{Lowering Case \texorpdfstring{$a_0$}{a0}}

The proof of Theorem~3.4\ich{a} starts with:
\begin{quote}
\ich{a} Fix \(A_0\in A\),
\end{quote}
It should be:
\begin{quote}
\ich{a} Fix \(a_0\in A\),
\end{quote}


\subsection{\texorpdfstring{$a$}{a} (not alpha)}

The proof of Theorem~3.4\ich{a} has a paragraph that starts with
\begin{quote}
If now \(\alpha\in A\) and
\end{quote}
It should be:
\begin{quote}
If now \(a\in A\) and
\end{quote}

\subsection{\texorpdfstring{$y$ (not $\gamma$)}{y (not gamma)}}

In page~87, exercise 10, instead of:
\begin{quote}
\ldots\; of all real functions \(\gamma\) on $S$
\end{quote}
It should be:
\begin{quote}
\ldots\; of all real functions $y$ on $S$
\end{quote}

\subsection{Holomorphic Functions}

In the proof of Theorem~3.31 equality~(5) can be deduced as follows:
\begin{align*}
\frac{\Lambda f)(z) - (\Lambda f)(0)}{z}
&= \frac{1}{2\pi iz}\int_\Gamma 
   \left(\frac{(\Lambda f)(\zeta)}{\zeta - z} - \frac{(\Lambda f)(\zeta)}{\zeta}
   \right)d\zeta
 = \frac{1}{2\pi i}\int_\Gamma 
   \frac{\zeta(\Lambda f)(\zeta) - (\zeta - z)(\Lambda f)(\zeta)}{
         \zeta(\zeta - z)z} \,d\zeta \\
&= \frac{1}{2\pi i}\
   \int_\Gamma \frac{(\Lambda f)(\zeta)}{\zeta(\zeta - z)} \,d\zeta .
\end{align*}


%%%%%%%%%%%%%%%%%%%%%%%%%%%%%%%%%%%%%%%%%%%%%%%%%%%%%%%%%%%%%%%%%%%%%%%%
%%%%%%%%%%%%%%%%%%%%%%%%%%%%%%%%%%%%%%%%%%%%%%%%%%%%%%%%%%%%%%%%%%%%%%%%
\section{Exercises} % pages 85-91

%%%%%%%%%%%%%%%%%
\begin{enumerate}
%%%%%%%%%%%%%%%%%

%%%%%%% 1
\begin{excopy}
Call a set \(H\subset \R^n\) a
\index{hyperplane}
\emph{hyperplane} if there exist real numbers \seqn{a}, $c$
(with \(a_i\neq 0\) for at least one~$i$) such that~$H$ consists of all points
\(x=(x_1,\ldots,x_n)\) that satisfy \(\sum a_ix_i = c\).

Suppose $E$ is a convex set in \(\R^n\), with nonempty interior,
and $y$ is a boundary point of $E$. Prove that there is a hyperplane plane~$H$
such that \(y\in H\) and~$E$ lies entirely on one side of~$H$.
(state the conclusion more precisely.)
\emph{Suggestion:} Suppose $0$ is an interior point of~$E$,
let $M$ be the one dimensional subspace that contains~$y$, and apply
Theorem~3.2.
\end{excopy}

We are looking for $H$ defined as above, such that
\(y\in H\) and \(\sum a_ix_i \geq c\) for all \(x\in E\).

\Wlogy\ \(0\in \Int(E)\), otherwise we simply shift $X$ by some \(-v\)
where \(v\in\Int(E)\).
Let \(V=B(0;r)\subset \Int(E)\).
We will show that the half open segment \([0,y) = \{ty: 0\leq t<1\} \subset E\).
By negation, there exists
some \(0 < a < 1\) such that \(ay \notin E\).
Let \(\{y_n\}_{n\in\N}\) be a sequence in $E$ such that
\(\lim_{n\to\infty} y_n = y\) and \(\|y_n - y\| < 1-a\).
The last inequality ensures \(0\neq y_n \neq ay\).
For each \(y_n\) look at the line \(\ell_n\) determined by \(y_n\) and \(ay\).
Let \(x_n\in \ell_n\) be (the closest to the origin) such that
\(\|x_n\| = \min(\{\|x\|: x\in\ell_n\})\).
If \(x_n = ay + t(y_n - ay)\) where \(t\in\R\) then we know that
\(X_n\) is perpendicular to \(\ell_n\), hence
\begin{equation*}
0 = \langle x_n, y_n-ay\rangle
  = \langle ay, y_n-ay\rangle + t \|y_n -ay\|_2^2.
\end{equation*}
It is easy to see that \(\lim_{n\to\infty} \|x_n\| = 0\)
and so \(x_n \in V\) for some $n$. Now \(ay\) is a convex combination
of \(\{x_n,y_n\}\) which is a contradiction.

Since $E$ contains the neighborhood $V$ of $0$, it is an absorbing set and
by Theorem~1.35 the
\index{Minkowski functional}
Minkowski functional \(\mu_E(x) = \inf\{t\geq 0: x\in tE\}\)
is a seminorm.

Since \([0,y)\subset E\), \(y\in tE\) whenever \(t>1\)
and so \(\mu_E(y)\leq 1\).

If by negation \(\mu_E(y) < 1\), we can find \(a \in(0,1)\) such that
\(y\in aE\). Hence \(y/a \in aE\).
We look at the homeomorphic mapping
\(h(x) = (1-a)x+a(y/a)\)
restricted to $V$.
It maps each \(v\in V\) to a convex combination of $v$ and \(y/a\),
hence \(h(V) \subset E\) and is open since $h$ is homeomorphic.
But this gives the contradiction \(h(0) = y \in h(V) \Int(E)\).
We have shown
\begin{equation*}
\mu_E(y) = 1.
\end{equation*}

Let $M$ be the subspace generated by $y$ and define
\(f(ay) = a\) for each \(a\in\R\).
By Theorem~3.2 $f$ can be extended to a functional \(\Lambda:\R^n\to\R\)
such that \(-\mu_E(-x) \leq \Lambda x \leq \mu_E(x)\) for all \(x\in\R^n\).
By looking at the standard base of \(\R^n\)
there exist \(\{a_j\in\R: 1\leq j \leq n\}\) such that
\(\Lambda((x_{j=1}^n)) = \sum a_ix_i\). Let \(c=1\)
and now \(\Lambda(x) \leq 1\) for all \(c\in E\).


The condition of $E$ having nonempty interior is actually not necessary.
\Wlogy, we may assume \(0\in E\).
Now look at the vector sub-space \(V\subset\R^n\) spanned by $E$.
Unless the trivial case where \(|E|<2\) then because of convexity
$E$ has nonempty interior relative to $V$.
Now we can apply the initial result.

%%%%%%% 2
\begin{excopy}
Suppose \(L^2=L^2([-1,1])\). with respect to Lebesgue measure.
For each scalar \(\alpha\) let \(E_\alpha\) be the set of all continuous
functions $f$ on \([-1,1]\) such that \(f(0)=\alpha\).
Show that each \(E_\alpha\) is convex and that each is dense in \(L^2\).
thus \(E_\alpha\) and \(E_\beta\) are disjoint convex sets
(if \(\alpha \neq \beta\)) which cannot be separated by any continuous linear
functional \(\Lambda\) on \(L^2\). \emph{Hint:} What is \(\Lambda(E_\alpha)\)?
\end{excopy}

Clearly, every linear combination of
members from \(E_\alpha\) are in \(E_\alpha\).
In particular convex combinations. Hence \(E_\alpha\) is convex.

Let \(f\in E_\alpha\) and \(\epsilon>0\).
We can find some continuous \(g\in C([0,1])\)
such that \(\|f-g\|_2<\epsilon/2\).
Let \(M = \max|g|\) and for \(\delta>0\) define \(h_\delta\in E_\alpha\) by
\begin{equation*}
h_\delta(t) = \left\{\begin{array}{ll}
 g(t) \quad & |t-\alpha|\geq\delta \\
 (t-\alpha)g(\max(0,t-\delta))/\delta \quad &
     \max(0,\alpha-\delta) \leq t \leq \alpha \\
 (t-\alpha)g(\min(1,t+\delta))/\delta \quad &
     \alpha \leq t < \min(1,\alpha+\delta)
     \end{array}\right.
\end{equation*}
Since \(\|g-h\|_2 \leq \sqrt{2\delta} M\)
it is easy to find \(\delta>0\) such that \(\|g-h_\delta\|_2 < \epsilon/2\).
Hence \(E_\alpha\) is dense in \(L^2\).

The image \(\Lambda(E_\alpha)\) must be convex.
Assume \(\alpha\neq\beta\) and by negation that \(\Lambda\) is a functional
such that \(\Lambda(E_\alpha)\cap \Lambda(E_\beta) = \emptyset\).
By \index{Reisz} Reisz representation theorem
(\cite{RudinRCA87} Theorem~2.14) there exists a Borel measure \(\mu\)
such that \(\Lambda(f) = \int_{[-1,1]}f\,d\mu\) for all \(f\in L^2([-1,1])\).
We know that \(\Lambda\neq 0\) and thus \(\mu\neq 0\). By looking
at \(\supp(\mu)\setminus[-\epsilon,\epsilon]\)
is is clear that we can find \(f\in E_\lambda\)
such that \(\Lambda(f)\) is positively and negatively large as we need. Hence
\(\Lambda(E_\alpha)\) and \(\Lambda(E_\beta)\) are unbounded.
Hence \(\Lambda(E_\alpha)=\Lambda(E_\beta)=\R\)
and in particular these images are \emph{not} disjoint.


%%%%%%% 3
\begin{excopy}
Suppose $X$ is a real vector space (without topology).
Call a point \(x_0\in A\subset X\) an \emph{internal} point of~$A$ if
\(A-x_0\) is an absorbing set.
\begin{itemize}
\itemch{a} Suppose $A$ and $B$ are disjoint convex sets in $X$, and $A$
has an internal point.
Prove that there is a nonconstant linear functional \(\Lambda\)
on $X$ such that \(\Lambda(A)\cap \Lambda(B)\) contains at most one point.
(The proof is similar to that of Theorem~3.4)
\itemch{b} Show (with \(X=\R^2\), for example) that it may not be possible
to have \(\Lambda(A)\) and \(\Lambda(B)\) disjoint, under the hypothesis
of \ich{a}.
\end{itemize}
\end{excopy}

\paragraph{\ich{a}}
Let \(a\in A\) be an internal point of $A$ and pick arbitrary \(b\in B\).
Consider the set \(C = A - B + (b-a)\).
Since \(0\in -B+b\) we have \(A-a\subset C\) and so $C$ is absorbing.
It is wasy to see that $C$ is convex, since $A$ and $B$ are.
Look at the
\index{Minkowski} Minkowski functional \(\mu_C\) (See Section~1.33).
By Theorem~1.35\ich{c} \(p=\mu_C\) is a seminorm.
Since \(A\cap B=\emptyset\),
we see that \(x\notin C\), and so \(p(x_0)\geq 1\).

Define \(f(tx_0) = t\) on the subspace \(M=\R x_0\).
By Theorem~3.2, $f$ extends to a linear functional \(\Lambda\) on $X$
such that \(|\Lambda| \leq p\). In particular, \(\Lambda \leq 1\) on $C$.
If now \(a\in A\) and \(b\in B\), we have
\begin{equation*}
\Lambda(a) - \Lambda(b) + 1
= \Lambda(a - b + x_0) \leq p(a-b + x_0) \leq 1.
\end{equation*}
{\small[Note: here it is different from the proof of Theorem~3.4, since
 $C$ (and $A$ as well) is not necessary open.]}'
Thus \(\Lambda(a) \leq \Lambda(b)\).
Therefore \(\sup_{a\in A} \Lambda(a) \leq \inf_{b\in B} \Lambda(b)\)
and \(\Lambda(A)\cap\Lambda(B)\) have one value in common at most.


\paragraph{\ich{b}}
Here is the required example:
\begin{align*}
A &= \{(x,y)\in\R^2: x>0 \;\vee\; x = 0 \wedge y > 0\} \\
B &= \{(x,y)\in\R^2: x<0 \;\vee\; x = 0 \wedge y < 0\}.
\end{align*}


%%%%%%% 4
\begin{excopy}
Let \(\ellinf\) be the space of all real bounded functions $x$ on
positive integers. Let \(\tau\) be the translation operator defined on
\ellinf\ by the equation
\begin{equation*}
(\tau x)(n) = x(n+1) \qquad (n=1,2,3,\ldots).
\end{equation*}
Prove that there exists a linear functional \(\Lambda\)
on \ellinf\ (called a
\index{Banach limit}
\emph{Banach limit}) such that
\begin{itemize}
\itemch{a} \(\Lambda\tau x = \Lambda x\), and
\itemch{b}
 \(\liminf_{n\to\infty} x(n) \leq \Lambda x \leq \limsup_{n\to\infty} x(n)\)
 for every \(x\in\ellinf\).

\end{itemize}
\emph{Suggestion:} Define
\begin{align*}
\Lambda_n(x) &= \frac{x(1) + \cdots + x(n)}{n} \\
M &= \{x\in\ellinf: \lim_{n\to\infty} \Lambda_n x = \Lambda x\;
    \textnormal{exists}\} \\
p(x) &= \limsup_{n\to\infty} \Lambda_n x
\end{align*}
and apply Theorem~3.2.
\end{excopy}

Following the suggestion, except for correcting the definition
\begin{equation*}
p(x) = \limsup_{n\to\infty} |\Lambda_n x|
\end{equation*}
It is clear that \(\Lambda_n\in (\ellinf)^*\)
and that $M$ is a (proper) subspace.

Let \(f(x) = \lim_{n\to\infty} \Lambda_n(x)\) for all \(x\in M\).
Clearly $f$ is linear and \(-p(x)\leq f(x) \leq p(x)\).
Now Theorem~3.2 gives an extension \(\Lambda\) of $f$
such that \(-p(x)\leq \Lambda(x) \leq p(x)\) for all \(x\in \ellinf\).

It is clear that for all \(x\in \ellinf\)
\begin{equation*}
\liminf_{n\to\infty} x(n)
= \liminf_{n\to\infty} x(n+1)
\qquad
 \limsup_{n\to\infty} x(n)
= \limsup_{n\to\infty} x(n+1)
\end{equation*}
and for all \(x\in M\)
\begin{equation*}
\liminf_{n\to\infty} x(n)
 \leq \lim_{n\to\infty}\Lambda(x)
 = \lim_{n\to\infty}\Lambda(\tau x)
 \leq \limsup_{n\to\infty} x(n)
\end{equation*}

Pick \(x\in \ellinf\) and put \(y = x - \tau x\). Now
\begin{equation*}
\Lambda_n(y) = \left(\Lambda(x_1) - \Lambda(x_{n+1})\right)/n
\end{equation*}
hence \(\lim_{n\to\infty} \Lambda_n(y) = 0\)
and so \(y\in M\) and \(\Lambda(y) = 0\).
Thus \ich{a} holds for all \(x\in\ellinf\).


%%%%%%% 5
\begin{excopy}
For \(0<p<\infty\), let \ellp\ be the space of all functions $x$,
(real or complex, as the case may be) on the positive integers, such that
\begin{equation*}
\sum_{n=1}^\infty |x(n)|^p < \infty.
\end{equation*}
For \(1\leq p<\infty\), define \(\|x\|_p = \left\{\sum|x(n)|^p\right\}^{1/p}\),
and define \(\|x\|_\infty = \sup_n|x(n)|\).

\begin{itemize}
%%%%%%%%
\itemch{a}
Assume \(1\leq p<\infty\). Prove that \(\|x\|_p\) and \(\|x\|_\infty\)
make \ellp\ and \ellinf\ into Banach spaces.
If \(p^{-1} + q^{-1} = 1\), prove that \((\ellp)^* = \ell^q\), in the following
sense: There is a one-to-one correspondence \(\Lambda \leftrightarrow y\)
between \((\ellp)^*\) and \(\ell^q\), given by
\begin{equation*}
\Lambda x = \sum x(n)y(n) \qquad (x\in \ellp).
\end{equation*}

%%%%%%%%
\itemch{b}
Assume \(1< p<\infty\) and prove that \(\ellp\) contains sequences that
converge weakly but not strongly.
%%%%%%%%
\itemch{c}
On the other hand, prove that every weakly convergent sequence in \ellone\ %
converges strongly, in spite of th fact that the weak topology of \ellone\ %
is difference from its strongly topology (which is induced by the norm).
%%%%%%%%
\itemch{d}
If \(0<p<1\), prove that \ellp\ metrized by
\begin{equation*}
d(x,y) = \sum |x(n) - y(n)|^p,
\end{equation*}
is a locally bounded $F$-space which is not locally convex
but that \((\ellp)^*\) nevertheless separates points in \ellp,
(Thus there are many convex open sets in \ellp\ but not enough to form a base
for its topology.)
Show that \((\ellp)^* = \ellinf\) in the sense as in \ich{a}.
Show also that the set of all $x$ with \(\sum|x(n)|<1\)
is weakly bounded but not originally bounded.
%%%%%%%%
\itemch{e}
For \(0<p\leq 1\), let \(\tau_p\) be the weak \upstar-topology
induced on \ellinf\ by \ellp; see \ich{a} and \ich{d}.
If \(0<p<r\leq 1\), show that \(\tau_p\) and \(\tau_q\) are \emph{different}
topologies (is one weaker that the other?)
but that they induce the same topology on each norm-bounded subset of \ellinf.
\emph{Hint:} The norm-closed unit ball of \ellinf\ is weak \upstar-compact.
\end{itemize}
\end{excopy}

\begin{itemize}
%%%%%%%%
\itemch{a}
By particular case of Theorems~3.8 and~6.16 of \cite{RudinRCA87}.

%%%%%%%%
\itemch{b}
Take \(e_n\in \ell^p\), that is (e\(_n(m) = \delta_{m,n}\)).
Now \(\|e_n\|_p = 1\), but for all \(x\in\ell^q\)
\begin{equation*}
\Lambda_n(e_n) = x(n) \xrightarrow {n\to\infty}  0.
\end{equation*}

%%%%%%%%
\itemch{c}
Assume
\begin{equation} \label{eq:ex3.5:weakly}
\lim_{j\to\infty}x_j = 0 \quad \textnormal{weakly}.
\end{equation}
If by negation
\(x_j \stackrel{\|\cdot\|_1}{\nrightarrow} 0\)
it implies
\begin{equation*}
\exists \alpha>0,\,\forall m,\,\exists j>m,\; \|x_j\|>\alpha.
\end{equation*}
Hence, by changing \((x_j)_{j\in\N}\) into a subsequence and multiplying
by a constant factor, we may assume
\begin{equation} \label{eq:ex3.5:geq1}
\forall j\in\N,\; \|x_j\|_1 \geq 1.
\end{equation}

Define an increasing sequences of natural numbers
\(s_j\) (subsequence indices)
and
\(b_j\) (blocks)
by induction on $j$.
Let \(s_1=1\) and let \(b_1\) be such that \(\sum_{j<b_1} |x_1(j)| > 1/6\).
Say \(b_n\) and \(s_n\) were defined for \(n\leq k\).
By \eqref{eq:ex3.5:weakly}, intuitively the
``finite initial \(b_{k}\)-blocks'' of
\((x_j)_{j\in\N}\) converge to zero. Formally we can say that
there exists \(m\in\N\) such that
\begin{equation*}
\forall s>m,\; \sum_{j=1}^{k-1} |x_s(j)| < 1/6.
\end{equation*}
With \eqref{eq:ex3.5:geq1} it implies
\begin{equation*}
\forall s>m,\; \sum_{j=k}^\infty |x_s(j)| \geq 5/6.
\end{equation*}

Define \(s_{k+1} = \max(m,s_k)+1\)
and let \(b_{k+1}\) be such that
\begin{equation*}
\sum_{j = b_{k+1}}^\infty |x_{s_{k+1}}(j)| < 1/6
\end{equation*}
which also implies
\begin{equation*}
% \sum_{b_k \leq j < b_{k+1}} |x_{s_{k+1}}(j)|
\sum_{j=b_k}^{b_{k+1}-1} |x_{s_{k+1}}(j)|
\geq \|x_{s_{k+1}}\|_1
     - \sum_{j < b_k} |x_{s_{k+1}}(j)|
     - \sum_{j \geq b_{k+1}} |x_{s_{k+1}}(j)|
\geq 1 - 2/6 = 2/3
\end{equation*}

Now that the sequences
\(s_j\), \(b_j\) are defined, let us
define \(u\in\ellinf\) by blocks.

For each \(j\in \N\)
there is a unique $k$ such that \(b_k \leq j < b_{k+1}\).
Let \(u(j)\) be such that
\begin{equation*}
u(j)\cdot x_{s_k}(j) = |x_{s_k}(j)| \quad\textnormal{and}\quad |u(j)|=1.
\end{equation*}
Considering \(u\in(\ellone)^*\) we now have
\begin{align*}
u(x_{s_k})
&= \left|\sum_{j=1}^\infty u(j) \cdot x_{s_k}(j)\right|
 \geq \left|\sum_{j=b_k}^{b_{k+1}-1} u(j) \cdot x_{s_k}(j)\right|
     - \sum_{j=1}^{b_k-1} |x_{s_k}(j)|
     - \sum_{j=b_{k+1}}^\infty |x_{s_k}(j)| \\
&\geq 2/3 - 1/6 - 1/6 = 1/3
\end{align*}
This is a contradiction to the
weakly convergence \eqref{eq:ex3.5:weakly} assumption.


%%%%%%%%
\itemch{d}
For \(0<p><1\) the space \ellp\ is locally bounded $F$-space
by similar arguments of section~1.47.
The balls \(B_r\) are not convex. To show this take
\begin{equation*}
r' = r^{1/p}, \qquad v_i = r'e_j \quad (j=1,2), \qquad v_{12} = (v_1+v_2)/2.
\end{equation*}
Now \(\|v_i\|_p = r\) and \(\|v_{12}\|_p = 2(r'/2)^p = (2/2^p)r > r\)x.

The equality \((\ellp)^* = \ellinf\) follows again from the
\emph{proofs} of Theorems \cite{RudinRCA87}~3.8 and~6.16, since in these
proofs wherever \(p=1\) the same could be done with \(0 < p \leq 1\).

There is an error in some editions of the ook here.
The unbounded but weakly bound set shoule be
\begin{equation*}
E := \{x\in\ellp: \sum|x(n)|<1\}.
\end{equation*}
Looking on \ellone, it is easy to see that
\begin{equation*}
\forall \Lambda\in (\ellp)^* \cong \ellinf,\,\exists \gamma,\,|\Lambda x|<\gamma,
\end{equation*}
hence $E$ is weakly bounded.
Suppose by negation that $E$ is bounded. Then for
\begin{equation*}
V := \{x\in\ellp: \sum|x(n)|^p < 1\}
\end{equation*}
\(\exists t<\infty,\,E\subset tV\).
In particular
\begin{align}
            & \forall n\in\N,\, \frac{1}{n}\sum_{j=1}^n e_j \in E \subset tV
 \notag \\
\Rightarrow & \forall n\in\N,\, \frac{1}{nt}\sum_{j=1}^n e_j \in E \subset V
 \notag \\
\Rightarrow & \forall n\in\N,\, n\cdot\left(\frac{1}{nt}\right)^p < 1
 \notag \\
\Rightarrow & \forall n\in\N,\, t^p > n^{1-p}. \label{eq:ex3.5d:contra}
\end{align}
But \(\lim_{n\to\infty} n^{1-p} = \infty\)
which contradicts \eqref{eq:ex3.5d:contra}.

%%%%%%%%
\itemch{e}
We show \(\ellp \subset \ell^r\). Let \(v\in \ellp\). Hence
\begin{align*}
& \exists n_0\in\N,\,\forall n>n_0,\,|v(n)|<1 \\
\Rightarrow & \forall n>n_0,\, |v(n)|^r < |v(n)|^p \\
\Rightarrow & v\in \ell^r.
\end{align*}
Therefore \(\ell^r\) induces a stronger topology, so \(\tau_p \subseteq \tau_r\).
By Theorem~3.10 if \(\tau_p = \tau_r\) then \(\ellp = \ell^r\),
but take \(x(n)=n^{-p}\) then \(x\notin \ellp\)
but
\begin{equation*}
\sum_{n\in\N} |x()|^r \leq \int_0^\infty (1/t)^{r/p}\,dt  < \infty
\end{equation*}
hence \(x\in\ell^r\setminus\ellp\), so \(\tau_p \subsetneq \tau_r\).

Let \(E\subset \ellinf\) be a norm bounded set.
So \(\exists M\geq 1,\,\forall x\in E,\,\|x\|_\infty \leq M\).
Let
\begin{equation*}
V_p := \{y\in\ellp: \sum_{n\in\N} |y(n)|<1/M\}.
\end{equation*}
Clearly \(V_p \subset \{v\in\ellinf: \|v\|_\infty \leq 1\}\).
Now  \(\forall\, y\in V_p,\, \forall x\in E\) satisfy:
\begin{align*}
& |\Lambda_x y| = \left|\sum_{n\in\N} x(n)\cdot y(n)\right|
                = \|x\|_\infty \left|\sum_{n\in\N} y(n)\right|
                \leq M \sum_{n\in\N} |y(n)|^p \\
\Rightarrow &
  E \subset K_p :=
  \{\Lambda \in (\ellp)^*\cong \ellinf: \forall y\in V_p,\, |\Lambda y|\leq 1\}.
\end{align*}
The set \(K_p\) is \(\tau_p\)-compact by
\index{Banach-Alaoglu}
\index{Alaoglu}
Banach-Alaoglu Theorem~3.15.
Hence also \(\tau_r\)-compact since it is a weaker topology.
On \(K_p\) both topologies are \(T_2\)-compact, so they coincide, particularly
on~$E$.
\end{itemize}


%%%%%%% 6
\begin{excopy}
Put \(f_n(t)=e^{ixt}\) (\(-\pi \leq t \leq \pi\)); Let
\(L^p = L^p(-\pi,\pi)\), with respect to Lebesgue measure.
If \(1\leq p < \infty\), prove that \(f_n\to 0\) weakly in \(L^p\),
but not strongly.
\end{excopy}

Clearly \(\|f_n\|_p = 1\). But for every trigonometric polynomial $p$
\begin{equation*}
\forall n \geq \deg(p)\,\; \int_{-\pi}^\pi p f_n = 0,
\end{equation*}
so \(\lim_{n\to\infty}\int_{-\pi}^{\pi} pf_n = 0\).
By Theorem~4.25 in \cite{RudinRCA87} the trigonometric polynomials
are dense in \(C([-\pi,\pi])\) and by Theorem~3.14 \cite{RudinRCA87}
\(C([-\pi,\pi])\) is dense in \(L^1([-\pi,\pi])\).

Hence for any \(g\in L^q([-\pi,\pi]) \subset L^1([-\pi,\pi])\)
and any \(\epsilon>0\) there exists a trigonometric polynomial $p$
such that \(\int_{-\pi}^\pi |g-p|^p<\epsilon\) and so
\begin{equation*}
\int_{-\pi}^\pi gf_n<\epsilon
\leq \int_{-\pi}^\pi +\left|\int_{-\pi}^\pi  pf_n\right|
\stackrel{n\to\infty}{\longrightarrow} \alpha < \epsilon.
\end{equation*}
Since \(\epsilon\) was chosen arbitrarily, so
\(\lim_{n\to\infty}\int_{-\pi}^\pi  gf_n = 0\) and
\(f_n \stackrel{n\to\infty}{\longrightarrow} 0\) weakly.


%%%%%%% 7
\begin{excopy}
\(L^\infty([0,1])\) has its norm topology
\(\|f\|_\infty\) is the essential supremum of \(|f|\) and
its weak \upstar-topology as the dual of \(L^1\).
Show that $C$, the space of all continuous functions on \([0,1]\),
is dense in \(L^\infty\) in one of these topologies but not in the other.
(Compare with the corollaries to Theorem~3.12.)
Show the same with ``closed'' in place of ``dense''.
\end{excopy}

Since \(C \subsetneq L^\infty\),
in any topology $C$ cannot be both closed and dense.
In the \(\|\cdot\|_\infty\)-topology $C$ is closed by
Theorem~7.12 of \cite{RudinPMA85}.
Now we will show that $C$ it is dense in the weak topology induced by \(L^1\).
But first, we will prove the the following assertion.
\begin{llem}
Given a measurable space \((X,\mu,\frakM)\).
If \(\mu(X)<\infty\) and \(f\in L^1(\mu)\) then
\begin{equation*}
\forall \epsilon>0,\, \exists \delta>0,\, \forall A\in\frakM, \;
  \mu(A) < \delta \;\Rightarrow\; \int_A|f|\,d\mu < \epsilon.
\end{equation*}
\end{llem}
\begin{thmproof}
For \(n\in\N\), let
\begin{align*}
X_n &= \{x\in X: n-1 \leq |f|<n\} \\
S_n &= \cup_{j\leq n} X_m \\
T_m &= \cup_{j>n} X_m = X \setminus S_n.
\end{align*}
We have
\begin{equation*}
\sum_{n\in\N} n\mu(X_n) \leq \int_X(|f|+1)\,d\mu \leq \|f\|_1+\mu(X) < \infty.
\end{equation*}
Let \(\epsilon>0\) \(\exists n\, \sum_{m>n} m\mu(X_m) < \epsilon/2\).
Put \(\delta = \epsilon/(2n)\).
Now if \(\mu(A)<\delta\), then
\begin{equation*}
\int_A|f|\,d\mu
\leq \int_{T_n} |f|\,d\mu +  \int_{A\cap S_n} |f|\,d\mu
\leq \sum_{m>n} n\mu(X_n) + \delta n < \epsilon.
\end{equation*}
\end{thmproof}

To establish the denisty, we need to show
\begin{equation*}
\forall f\in L^\infty,\,\epsilon>0,\,h\in L^1,\,\exists g\in C,
\; \int h(f-g)<\epsilon.
\end{equation*}
Pick such arbitrary $f$, \(\epsilon\), $h$.
NOw apply the above lemma with $h$ instead of $f$,
and \(\epsilon/(2\|f\|_\infty\) instead of \(\epsilon\) and get the appropriate
\(\delta>0\). Now by
\index{Lusin}
Lusin Theorem~2.23 in \cite{RudinRCA87}
\begin{equation*}
\exists g\in C(I),\, m\bigl(\{x\in I: g(x)\neq f(x)\}\bigr) < \delta
\qquad \textnormal{and} \qquad \|g\|_\infty \leq \|f\|_\infty.
\end{equation*}
and so
\begin{equation*}
\left|\int h(f-g)\right|
\leq \int |h|\cdot|f-g| \leq 2\|f\|\cdot \int_{\{x: g(x)\neq f(x)\}} |h| < \epsilon.
\end{equation*}

%%%%%%% 8
\begin{excopy}
Let $C$ be the Banach space of all complex continuous functions on \([0,1]\),
with the supremum norm.
Let $B$ be the closed unit ball of $C$.
Show that there exist continuous linear functionals \(\Lambda\) on $C$
for which \(\Lambda(B)\) is an \emph{open} subset of the
complex plane; in particular, \(|\Lambda|\) attains no maximum on $B$.
\end{excopy}

Put \(\Q\cap I\) in a sequences \(\{q_n\}_{n\in\N}\). Put
\begin{equation*}
\Lambda f := \sum_{n-1}^\infty (-1)^n f(q_n).
\end{equation*}
\(\|\Lambda\| \leq 1\) trivially.
For all $n$ \(\exists f_n \in B,\,\forall 1\leq j\leq n,\, f_n(q_j) = (-1)^j\),
so \(\lim_{n\to\infty} \Lambda f_n = 1\), and \(\|\Lambda\|=1\).
But if \(\forall n\in\N,\, h(q_n) = u(-1)^n\) and \(|u|=1\)
then \(h\notin C\), so
\begin{equation*}
\Lambda(B) = \{z\in\C: |z|<1\}.
\end{equation*}

%%%%%%% 9
\begin{excopy}
Let  \(E \subset L^2(-\pi,\pi)\) be the set of all functions
\begin{equation*}
f_{m,n}(t) = e^{imt} + me^{int},
\end{equation*}
where \(m,n\) are integers amd \(0\leq m \leq n\). Let \(E_1\) be the set
of all \(g\in L^2\) such that some sequence in $E$ converges weakly to $g$.
(\(E_1\) is called the
\index{weak sequential closure}
\emph{weak sequential closure} of $E$.)
\begin{itemize}
\itemch{a} Find all \(g\in E_1\).
\itemch{b} Find all $g$ in the weak closure \(\overline{E}_w\) of $E$.
\itemch{c}
Show that \(0\in \overline{E}_w\) bit $0$ is not in \(E_1\),
although $0$ lies in the weak sequential closure of \(E_1\).
\end{itemize}
This example shows that a weak sequential closure need not be
weak sequentially closed. The passage from a set to its weak sequential
closure is therefore not a closure operation, in the sense in which that term
is usually used in topology. (see also Exercise~28.)
\end{excopy}

Abbreviate \(L^2 = L^2(-\pi,\pi)\).
\begin{itemize}
\itemch{a}
Since
\(\lim_{n\to\infty} e^{int} = 0\) weakly, we
clearly have
\begin{equation*}
\lim_{n\to\infty} e^{imt} + me^{int} = e^{int} \qquad\textnormal{weakly}^*.
\end{equation*}
Hence \(\forall m\in\Z^+,\,e^{imt} \in E_1\).

We need to show that these are all the non-trivial weakly sequential limits.
Assume
\begin{equation*}
g_j = f_{m_j,n_j} = e^{im_jt} + m_j e^{in_jt} \qquad (j\in\N)
\end{equation*}
is a sequence with a weak sequential limit \(g\in L^2\).
If \(\{m_j\}_{j\in\N}\) is bounded we can find a subsequence
of \(\{g_j\}_{j\in\N}\) with constant \(\overline{m}=m_j\) and it converges
to \(e^{i\overline{m}t}\).
Otherwise \(\{m_j\}_{j\in\N}\) is unbounded and by moving to a subsequence,
we may assume \(m_j \leq n_j < m_{j+1}\) for all \(j\in\N\).
Note that this also implues \(n_j < n_{j+1}\) for all \(j\in\N\).
Consider now the
function $h$ defined by
\begin{equation*}
h(t) = \sum_{k\in\N} e^{-in_k t} / m_k.
\end{equation*}
Clearly \(h\in L^2\) since
\begin{equation*}
\|h\|_2 = \sum_{k\in\N} m_k^{-2} \leq \sum_{k\in\N} k^{-2}
\leq 1 + \sum_{k=2}^\infty \int_{k-1}^{k} x^{-2}\,dx
= 1 + \int_1^\infty x^{-2}\,dx = 2 < \infty.
\end{equation*}
Now
\begin{align*}
\int_{-\pi}^\pi h(t)\cdot g_j(t)\,dt
&= \int_{-\pi}^\pi \left(\sum_{k\in\N} e^{-in_k t} / m_k\right)
                \cdot \left(e^{im_jt} + m_j e^{in_jt}\right)\,dt
  \\
&= 2\pi \sum_{k\in\N} \delta(n_k,m_k)/m_k + 1 = \infty
\end{align*}
in contradiction, since \(h \cdot g_j\) cannot converge in \(L^2\).

\itemch{b}
Since \(e^{imt} \in E_1\) by what we saw in \ich{a} we have
\(0 \in \overline{E}_w\setminus E_1\).
Put \(E_2 := E_1 \cup \{0\}\).
We will show that actually \(\overline{E}_w = E_2\).
Pick
\(f\in L^2) \setminus E_2\). Take the representation
\begin{equation*}
f(t) = \sum_{k\in\Z} a_je^{ikt}
\qquad \textnormal{where}\quad
\sum_{k\in\Z} |a_k|^2 < \infty
\end{equation*}
We need to show that there is a weak neighborhood $V$ of $f$
such that \(V\cap E_2 = \emptyset\).
Since \(L^2\) is a Hilbert place
we may identify all \(g\in L^2\) with \(\Lambda_g\in (L^2)^*\)
via
\begin{equation*}
\Lambda_g(f) = \langle f, g\rangle
 = \frac{1}{2\pi} \int_{-\pi}^\pi f(t)\overline{g(t)}\,dt.
\end{equation*}
For each such $f$ we will find some \(g\in L^2\) and a neighborhood
\begin{equation*}
V = \Lambda_g^{-1}\bigl(B(\Lambda_g(f);\delta)\bigr)
  = \{h\in L^2: |\Lambda_g(h-f)| < \delta\}.
\end{equation*}
Consider cases by order.
Each case implicitly  assumes previous cases do not hold.
\paragraph{Case 1.} There exists some \(k\in\Z\) such that \(a_k\notin\{0,1\}1\).
We take \(g(t)=e^{-ikt}\) and
\(V = \Lambda_g^{-1}\left(B(a_k;\min(|a_k|,|1-a_k|)/2)\right)\).

\paragraph{Case 2.} There exists some \(k<0\) such that \(a_k=1\).
We take \(g(t)=e^{-ikt}\) and \(V = \Lambda_g^{-1}(B(1;1/2))\).

\paragraph{Case 2.} There is a finite set \(K\subset \Z^+\)
such that \(a_k = 1\) iff \(k\in K\) and \(a_k = 0\) otherwise.
That is \(f(t) = \sum_{k\in K} e^{ikt}\).
Note that the case \(K=\{1,n\}\) with \(1<n\) is impossible
since \(f\notin E\).
We define
\begin{align*}
g_k(t) &= e^{-ikt} \\
V_k &= \Lambda_{g_k}^{-1}(B(1;1/2)) \\
V &= \cap_{k\in K} V_k
\end{align*}
Now $V$ is neighborhood of $f$ and for each \(h\in E_2\)
there exists some \(k\in K\) such that \(h\notin V_k\subset V\).

\itemch{c}
Shown in previous item.
\end{itemize}

%%%%%%% 10
\begin{excopy}
Representation \ellone\ as the space off all real functions $x$ on
\(S = \{(m,n): m\leq 1,\; n \leq 1\}\), such that
\begin{equation*}
\|x\|_1 = \sum |x(m,n)| < \infty.
\end{equation*}
Let \(c_0\) be the space of all real functions $y$ on $S$ such that
\(y(m,n) \to 0\) as \(m+n\to\infty\),
with the norm \(\|y\|_\infty = \sup |y(m,n)|\).

Let $M$ be the subspace of \ellone\ consisting of all \(x\in\ellone\)
that satisfy the equations
\begin{equation*}
mx(m, 1) = \sum_{n=2}^\infty x(m,n) \qquad (m=1,2,3,\ldots).
\end{equation*}
\begin{itemize}
\itemch{a}
Prove that \(\ellone = (c_0)^*\). (see also Exercise~24, Chapter~4.)
\itemch{b}
Prove that $M$ is a norm-closed subspace of \ellone.
\itemch{c}
Prove that $M$ is a weak \upstar-dense in \ellone\ [relative to the
weak \upstar-topology given by \ich{a}].
\itemch{d}
Let $B$ be the norm-closed unit ball of \ellone. In spite of \ich{c},
prove that the weak \upstar-closure of \(M\cap B\) contains no ball.
\emph{Suggestion:} If \(\delta>0\) and \(m > 2/\delta\), then
\begin{equation*}
|x(m,1)| \leq \frac{\|x\|}{m} < \frac{\delta}{2}
\end{equation*}
if \(x\in M \cap B\) although  \(x(m,1) = \delta\) for some \(x\in \delta B\).
Thus \(\delta B\) is not in the weak \upstar-closure of \(M\cap B\).
Extend this to balls with other centers.
\itemch{e}
Put \(x_0(m,1) = m^{-2}\), \(x_0(m,n)=0\) when \(n\geq 2\).
Prove that no \emph{sequence} in $M$ is weak \upstar-convergent to \(x_0\),
in spite of \ich{c}.
\emph{Hint:} Weak \hbox{\upstar-convergence} of
\(\{x_j\}\) to \(x_0\) implies
that \mbox{\(x_j(m,n)\to x_0(m,n)\)} for all \(m,n\), as \(j\to\infty\),
and that \(\{\|x_j\|_1\}\) is bounded.
\end{itemize}
\end{excopy}
\begin{itemize}

\itemch{a}
Identifying $S$ with \N, this was show in \cite{RudinRCA87} Chapter~4,
Exercise~9\ich{a}.

\itemch{b}
% Let \(M_1\) be the weak closure of $M$ in \ellone.
Assume by negation \(x \in \overline{M}\setminus M\).
Then we can find some \(m\in\N\) such that
\begin{equation*}
mx(m, 1) \neq \sum_{n=1}^\infty x(m,n).
\end{equation*}
Let
\begin{equation*}
\delta = \left| mx(m, 1) - \sum_{n=1}^\infty x(m,n) \right| > 0.
\end{equation*}
Now the open ball \(B(x;\delta)\) is a norm neighborhood of $X$
disjoint from $M$, a contradiction to the assumption.
Hence \(\overline{M} = M\).

\itemch{c}
Pick arbitrary \(\xi\in\ellone\) and a base weak\upstar- neighborhood of
it $V$, by picking arbitrary
where \(\seq{v}{k}\in c_0\) and \(\epsilon > 0\) and
defined by
\begin{equation*}
V = \left\{ x\in\ellone: \forall j\in\N_k\;
      \sum_{m,n\in\N}
        \left| v_j(m,n) \cdot \bigl(x(m,n) - \xi(m,n)\bigr) \right|
      < \epsilon
    \right\}.
\end{equation*}
To show that $M$ is weak\upstar\ dense, we need to find some \(x\in M\cap V\).
Define the ``row errors''
\begin{equation*}
e_m = mx(m,1) - \sum_{n=2}^\infty x(m,n) \qquad (m \in \N).
\end{equation*}
We have the following finite values
\begin{align*}
\sup_{m\in\N} |e_m|/m &\leq \sum_{m\in\N} |e_m|/m
   \leq \sum_{m\in\N} \left(|x(m,1)| + (1/m)\sum_{n=2}^\infty |x(m,n)|\right) \\
  &\leq \sum_{m\in\N} \left(|x(m,1)| + \sum_{n=2}^\infty |x(m,n)|\right)
   = 2\|x\|_1 < \infty \\
\|\xi\|_\infty &= \max_{m,n\in\N} |\xi(m,n)| \leq \|\xi\|_1 < \infty.
\end{align*}
Since \(\seq{v}{k}\in c_0\), for all \(j\in\N_k\).
\begin{itemize}
\item
There exists some \(\mu<0\) such that
\begin{equation*}
\left(\sup_{m\in\N} |e_m|/m\right)\sum_{m=\mu+1}^\infty |v_j(m,1)| < \epsilon/2.
\end{equation*}
\item
For each \(m\in\N\)
(just \(m\in\N_\mu\) is needed)
we can find some \(\nu_m\in\N\)
such that
\begin{equation*}
|v_j(m,\nu_m)\cdot e_m| < 2^{-m-1}\epsilon.
\end{equation*}
\end{itemize}
Now we define \(x\in\ellone\) as follows
\begin{equation*}
x(m,n) =
\left\{
 \begin{array}{ll}
 \xi(m,n) \quad &
    \textnormal{if}\; m \leq \mu \;\textnormal{or}\; n \neq \nu_m \\
 \xi(m,n) + e_m \quad &
    \textnormal{if}\;  m \leq \mu \;\textnormal{and}\;n = \nu_m \\
 \xi(m,1) - e_m/m \quad &
    \textnormal{if}\;  m > \mu
 \end{array}
\right.
\end{equation*}
Now
\begin{equation*}
\|x\|_1
 \leq \|\xi\|_1 + \sum_{m=1}^\mu |e_m| + \sum_{m=\mu+1}^\infty |\xi(m,1)|
 \leq 2\|\xi\|_1 + \sum_{m=1}^\mu |e_m|
 < \infty
\end{equation*}
and so \(x\in\ellone\) and directly by definition \(x\in M\).
Finally for each \(j\in\N_k\) we have
\begin{align*}
d_j
&=
\sum_{m,n\in\N} \left| v_j(m,n) \cdot \bigl(x(m,n) - \xi(m,n)\bigr) \right| \\
&=   \sum_{m=1}^\mu | v_j(m,\nu_m)\cdot e_m |
  + \sum_{m=\mu+1}^\infty  |v_j(m,1) \cdot e_m/m | \\
&\leq \sum_{m=1}^\mu  2^{-m-1}\epsilon
    + \left(\sup_{m\in\N} |e_m|/m\right)\sum_{m=\mu+1}^\infty  |v_j(m,1)|
 \leq \epsilon/2 + \epsilon/2 = \epsilon.
\end{align*}
Hence \(x\in V\) and thus $M$ is a weak \upstar-dense in \ellone.

\itemch{d}
Put \(G = M \cap B\) and $H$ is the weak\upstar closure of $G$.
The suggestion shows that $H$ does not contain any non trivial balls
 centered at~$0$.

By negation assume \(C := \overline{B(x;r)} \subset H\) for some \(r>0\)
and clearly \(x\in B = \overline{B(0;1)}\).
Since \ellone\ is locally convex and $C$ is clearly convex,
Corollary~\ich{a} of Theorem~3.12, that $C$ is also weak\upstar\ closed.
Since \(c_0\) separates points in \ellone, hence any singleton \ellone\
is weak\upstar\ closed. Thus \(G\cap C\) contains at lease two distinct points.
Being a vector space, $M$ is also convex, hence $C$ contains some
internal point \(y\in M\), and some \(s>0\) such that
\(B(y;s) \subset B(x,r)\).
Translation is homeomorphism in both topologies, \(M + y = M\) and
so \(B(0;s) \subset M \cap B\), which contradicts the initial result
shown by the suggestion.

\itemch{e}
We begin by proving the claim in the hint by steps.

\begin{llem}
Let \(\{x_j\}\) to \(x_0\) weakly in \ellone\ then
  \(\{\|x_j\|_1\}\) is bounded.
\end{llem}
\begin{proof}
{\nullfont dum}
\newline
\textbf{Claim-I.}
For any \(m,n>0\) the set \(\{|x_j(m,n)|: j\in\N\}\)  is bounded.
This is trivial by looking at \(y\in c_0\) defined by \(y(j,k)=1\)
if \((j,k)=(m,n)\) and \(y(j,k)=0\) otherwise.

\textbf{Claim-II.}
For any finite set $F$ of pairs
\(m,n>0\) the set
\begin{equation*}
\{|x_j(m,n)|: j\in\N\;\land\; (m,n)\in F\}
\end{equation*}
is bounded.
\newline
The bound is a simple maximum taken on the finite sets of
bounds each of a fixed pair \((m,n)\in F\).

Now by negation assume that the set  \(\{\|x_j\|_1:j\in\N\}\) is not bounded.
We will construct \(y\in c_0\) such that the set
\(\{|y(x_j)|:\,j\in\N\}\) is not bounded, that is by changing to
sub-sequence \((s(j))_{j\in\N}\) we can have
\begin{equation*}
\lim_{j\to\infty}|y(x_{s(j)})|
 = \lim_{j\to\infty} \left| \sum_{(m,n)\in\N^2} y(m,n)\cdot x_{s(j)}(m,n) \right|
 = \infty
\end{equation*}
which contradicts the assumption of this lemma.

We start by constructing mutually disjoints finite
sets \(D_n \subset \N^2\) by induction.
We put
\begin{alignat*}{2}
E_0 &= 0    &  d_0 &= 0 \\
E_j &= E_{j-1} \cup D_j   \qquad &   d_j &= \max\{m, n: (m,n)\in E_j\}
   \qquad (\forall j\in\N)
\end{alignat*}
Let \(D_0 = \emptyset\).
Assume \(D_j\) are defined for all \(j < k\).
Let $b$ be a bound (provided by Claim~II) of
\begin{equation*}
\{|x_j(m,n)|: \, j\in\N \land (m,n) \in E_{k-1}\}.
\end{equation*}
pick an index \(s(k)\) such that \(x_{s(k)}\) satisfy the following
\begin{align*}
\sum_{(m,n) \in \N^2 \setminus E_{k-1}} |x(m,n)| > k(b + k + 1)
\end{align*}
Since \(x_j\in \ellone\) we can find some \(d_k\) such that
\begin{align*}
\sum_{(m,n > d_k) \in \N^2 \setminus E_{k-1}} |x(m,n)| < 1
\end{align*}
We define \(\lrcorner\)-like set
\begin{equation*}
D_k = \{(m,n)\in \N^2: m,n \leq d_k\} \setminus D_{k-1}
\end{equation*}
Clearly \(\N^2 = \disjunion_{j\in\N} D_j\), so we can define
\(y(m,n) = e^{i\theta(m,n)}/s(k)\) iff \((m,n)\in D_k\)
setting the argument \(\theta(m,n)\)
so that \(e^{i\theta(m,n)} x_{s(k)}(m,n) = |x_{s(k)}(m,n)|\).
Using the abbreviation
\begin{equation*}
c(m,n) =  y(m,n)\cdot x_{s(k)}(m,n)
\end{equation*}
 we have
\begin{align*}
|y(x_{s_k})|
&= \left|
   \left(\sum_{(m,n)\in E_{k-1}} c(m,n)\right)
   + \left(\sum_{(m,n)\in D_k} c(m,n)\right)
   + \left(\sum_{(m,n)\in \N^2 \setminus E_k} c(m,n)\right)
   \right| \\
&\geq
   \left|\sum_{(m,n)\in D_k} c(m,n)\right|
  - \left|\sum_{(m,n)\in E_{k-1}} c(m,n)\right|
  - \left|\sum_{(m,n)\in \N^2 \setminus E_k} c(m,n)\right| \\
&\geq
   \left(\sum_{(m,n)\in D_k} x_{s_k}(m,n)\right)\bigm/ k
  - \sum_{(m,n)\in E_{k-1}} |c(m,n)|
  - \sum_{(m,n)\in \N^2 \setminus E_k} |c(m,n)| \\
&\geq (b+k+1) - b - 1 = k.
\end{align*}
\end{proof}


With the lemma established, assume by negation that the sequence \(\{x_j\}\)
where \(x_j\in M\),
converges \upstar-weakly to \(x_0\).
Let $M$ be the bound of \(\{\|x_j\|_1:j\in\N\}\) provided by the lemma.
Let $K$ be such that \(\sum_{j=1}^K 1/j > M + 1\).
%Let \(v\in c_0\) defined by \(v(m,n)=1\) iff \(1\leq m \leq K\) and \(n=1\)
% and  \(v(m,n)=0\) otherwise.
By convergence, for each \(\epsilon>0\) there must exists some $k$ such that
if \(y = x_j \in M\) then
\begin{equation*}
|y(m,1) - x_0(m,1)| = |y(m,1) - m^{-2}| < \epsilon
  \qquad (1 \leq m \leq K)
\end{equation*}
But then,
\begin{align*}
\|y\|_1
&= \sum_{m=1}^\infty \left(|y(m,1)| + \sum_{j=2}^\infty|y(m,j)|\right)
 \geq \sum_{m=1}^\infty
    \left(|y(m,1)| + \left|\sum_{j=2}^\infty|y(m,j)\right|\right)   \\
&= \sum_{m=1}^\infty |y(m,1)| +m|y(m,1)|
 = \sum_{m=1}^\infty (m+1)|y(m,1)| \\
&\geq \sum_{m=1}^K (m+1)(m^{-2} - \epsilon)
 \geq \left(\sum_{m=1}^K m^{-1}\right) (K+1)\epsilon.
\end{align*}
But \(\epsilon\) can be arbitrary small, thus
we get the contradiction
\(\|y\|_1 = \|x_j\|_1 > M\).
\end{itemize}

%%%%%%% 11
\begin{excopy}
Let $X$ be an infinite-dimensional
\index{Frechet@Fr\'echet}
Fr\'echet space. Prove that \(X^*\), with its weak\upstar-topology is
of the first category in itself.
\end{excopy}

Let $d$ be the invariant metric on $X$.
Let
\begin{align*}
V &= \{x\in X: d(x,0) < 1\} \\
B_n &= \left\{\Lambda\in X^*: \sup_{x\in V} |\Lambda(x)| \leq n\right\}.
\end{align*}
\index{Banach-Alaoglu}
Clearly \(B_n = nB_1\).
By Banach-Alaoglu Theorem~3.15 \(B_n\) are weak\upstar-compact sets.
and thus are closed.
Assume by negation that \(B_n\)  contains non empty interior for some $n$.
Then by the remark
at the end of section~3.12 (using tranlation to the origin)
applied to \(X^*\) which is obviously infinite-dimensional,
\(B_n\) is not weak\upstar\ bounded.
But this contradicts \(B_n\)'s compactness.
Now
\(X^* = \cup_{n\in\N} B_n\) hence a countable union of nowhere desnse sets.
Hence \(X^*\) is of first category in its weak\upstar-topology.

%%%%%%% 12
\begin{excopy}
Show that the norm-closed unit ball of \(c_0\) is not weakly compact.
recall that \((c_0)^* = \ellone\) (Exercise~10).
\end{excopy}

Let \(u_n\in c_0\) be defined by \(u_n(j) = \delta_{n,j}\).
Clearly  \(u_n\) are in the unit ball. This sequence
has no weak-accumulation point
(and in particular, no accumulation in the unit ball). Thus the unit ball
is not weakly compact.

%%%%%%% 13
\begin{excopy}
Put \(f_N(t) = N^{-1} \sum_{n=1}^{N^2} e^{int}\). Prove that \(f_n\to 0\) weakly
in \(L^2(-\pi,\pi)\).

By Theorem~3.13, some sequence of convex combinations of the \(f_N\)
converges to $0$ in the \(L^2\) norm.
Find such a sequence.
Show that \(g_N = N^{-1}(f_1 + \cdots + f_N)\) will not do.
\end{excopy}

\iffalse
We first solve the following
(similar Exercise~7 of Chapter~3 in \cite{RudinPMA85})

\begin{llem}
If \(\sum_{n=1}^\infty a_n^2 < \infty\) where \(a_n \geq 0\)
then the series \sum_{n=1}^\infty a_n/n\) converges.
\end{llem}
\begin{proof}
\Wlogy\ we assume \(a_n \geq 0\).
Using the root ratio test (Theorem~3.33 in \cite{RudinPMA85}.
\begin{align*}
\limsup_{n\to\infty} \sqrt[n]{a_n/n}
=       \left(\limsup_{n\to\infty} \sqrt[n]{a_n}\right)
  \cdot \left(\lim_{n\to\infty} \sqrt[n]{1/n}\right) \\
&= \limsup_{n\to\infty} \sqrt[n]{a_n}
\end{align*}
\end{proof}
\fi

We assume
 \(\langle f,g \rangle = \frac{1}{2\pi}\int_{-\pi}^\pi f(t)\overline{g(t)}\,dt\).

To show weakly convergence to~0, we pick arbitrary \(h\in L^2(-\pi,\pi)\)
such that \(h(t) = \sum_{n\in\Z} a_n e^{int}\)
with norm  and \(\|h\|_2 = \sum_{n\in\Z} |a_n|^2 < \infty\).
Note that by looking at the cases
\(n|a_n| \leq 1\) or \(n|a_n| \geq 1\) we always have
\(|a_n|/n \leq \max(1/n^2, |a_n|^2)\).


\iffalse
% Take an arbitrary \(h \in  \(L^2 = L^2(-\pi,\pi) \equiv (L^2)^*\).
% We have the representation \(h(t) = \sum_{n=-\infty}^\infty a_n e^{int}\)
% with norm \(\|h\|_2 = \sum{n=-\infty}^\infty |a_n|^2 < \infty\).
Using the root ratio test (Theorem~3.33 in \cite{RudinPMA85}
we know that
\begin{equation*}
\limsup_{n\to\infty} \sqrt[n]{|a_n|^2} < 1.
\end{equation*}
\fi

Pick arbitrary \(\epsilon>0\).
There exists \(t\) such that
both \(\sum_{n>t}  1/n^2 < \epsilon\)
and \(\sum_{n>t}  |a_n|^2 < \epsilon\).
Pick \(T \geq t\) such that
\begin{equation*}
T^{-1} \sum_{n=1}^{t^2} |a_n| < \epsilon.
\end{equation*}
Now for \(N \geq T\)
\begin{align*}
\left|\langle h, f_N \rangle\right|
&= \frac{1}{2\pi} \left|\int_{-\pi}^\pi h(t)\overline{f_N(t)}\,dt\right|
 = N^{-1} \left| \sum_{n=1}^{N^2} a_n \right| \\
&\leq N^{-1} \sum_{n=1}^{N^2} |a_n|
  \leq N^{-1} \sum_{n=1}^{T^2} |a_n| + N^{-1} \sum_{n=t^2+1}^\infty |a_n|  \\
 &\leq T^{-1} \sum_{n=1}^{t^2} |a_n| + \sum_{n=t^2+1}^\infty \max(1/n^2, |a_n|^2)
  \leq \epsilon + \sum_{n=t^2+1}^\infty 1/n^2 + \sum_{n=t^2+1}^\infty |a_n|^2 \\
 &\leq 3\epsilon
\end{align*}
Hence \(\lim_{N\to\infty} \langle  h, f_N \rangle = 0\).
(This is similar to Exercise~7 of Chapter~3 in \cite{RudinPMA85}.)

Compute \(L^2\) norms.
\begin{align}
 e_n(t) &= e^{int} \label{eq:en:eint} \\
\|e_n\|_2 &= 1
  % \qquad \textnormal{conventional division by}\; 2\pi
  \notag \\
\|f_n\|_2^2 &=  n^{-2} \sum_{j=1}^{n^2} \|e_n\|_2^2 = 1 \notag \\
g_N &=  N^{-1}\sum_{n=1}^N f_n
    = N^{-1}\sum_{n=1}^N \left(\sum_{k=n}^N 1/k\right)
                       \left(\sum_{k=(n-1)^2+1}^{n^2} e_k\right) \notag \\
\|g_N\|_2^2 &=  N^{-2}\sum_{n=1}^N \left(\sum_{k=n}^N 1/k\right)^2
                                \left(n^2 - (n-1)^2\right)
            =  N^{-2}\sum_{n=1}^N (2n-1)\left(\sum_{k=n}^N 1/k\right)^2
  \label{eq:ex3.13:gN2}
\end{align}

In order to analyze the behavior of \(\|g_N\|_2\) we diverge
to show some equalities related to harmonic numbers.

Given the finite sequence
\((a_j)_{j=l}^{h}\) and
\((b_j)_{j=l}^{h}\),
we have the  \emph{summation by parts} formula
\begin{equation} \label{eq:sumbyparts}
\sum_{j=l}^{h-1} (a_{j+1} - a_j)b_j
 = a_{h} b_{h} - a_l b_l - \sum_{j=l}^{h-1} a_{j+1}(b_{j+1} - b_j).
\end{equation}
which can be easily proved by induction on \(h-l\).

Define the harmonic numbers:
\begin{equation} \label{eq:harmonic:numbers}
H_n = \sum_{k=1}^n 1/k
\end{equation}
Clearly \(\sum_{k=l}^h = H_h - H_{l-1}\).

We will use \eqref{eq:sumbyparts} with \(l=1\), \(h=N+1\), that is:
\begin{equation}   \label{eq:sumbyparts:1N}
\sum_{n=1}^N n \sum_{k=n}^N 1/k
= a_{N+1}b_{N+1} - a_1 b_1 - \sum_{n=1}^N a_{n+1}(b_{n+1} - b_n)
\end{equation}
with some special cases for
\((a_n)_{n=1}^{N+1}\) and \((b_n)_{n=1}^{N+1}\).

\textbf{Case 1.}
\begin{alignat*}{2}
a_n &= (n^2-n)/2  \qquad  &           b_n &= \sum_{k=n}^N 1/k \\
a_{n+1} - a_n &= n  &  \qquad b_{n+1} - b_n &= -1/n \\
a_1 &= 0           &               b_{N+1} &= 0.
\end{alignat*}
Now \eqref{eq:sumbyparts:1N} becomes
\begin{align}
\sum_{n=1}^N n \sum_{k=n}^N 1/k
&= a_{N+1}\cdot 0 - 0 \cdot b_1 - \sum_{n=1}^N \left((n^2+n)/2\right)(-1/n)
= \frac{1}{2} \sum_{n=1}^N (n+1)
\notag \\
&= (N(N+1)/2+N)/2
% = (n(n+1)+2n)/4
 = N(N+3)/4 \label{eq:harmonic:case1}
\end{align}


\textbf{Case 2.}
\begin{alignat*}{2}
a_n &= (n-1)^2      &           b_n &= \sum_{k=n}^N 1/k \\
a_{n+1} - a_n &= 2n-1 \qquad &  \qquad b_{n+1} - b_n &= -1/n \\
a_1 &= 0              &               b_{N+1} &= 0.
\end{alignat*}
Now \eqref{eq:sumbyparts:1N} becomes
\begin{equation*}
\sum_{n=1}^N (2n-1) \sum_{k=n}^N 1/k
= a_{N+1}\cdot 0 - 0 \cdot b_1 - \sum_{n=1}^N n^2(-1/n)
= \frac{1}{2} \sum_{n=1}^N n
= N(N+1)/2
\end{equation*}


\textbf{Case 3.}
\begin{alignat*}{2}
a_n &= (n-1)^2  \qquad &           b_n &= \left(\sum_{k=n}^N 1/k\right)^2 \\
a_{n+1} - a_n &= 2n-1 \qquad & \qquad
  b_{n+1} - b_n &=
    \left(2\left(\sum_{k=n}^N 1/k\right) - 1/n\right)\cdot(-1/n) \\
a_1 &= 0           &               b_{N+1} &= 0.
\end{alignat*}
Now using  \eqref{eq:sumbyparts:1N} and \eqref{eq:harmonic:case1} we get
\begin{align}
\sum_{n=1}^N (2n-1) \left(\sum_{k=n}^N 1/k\right)^2
&= a_{N+1}\cdot 0 - 0 \cdot b_1
   - \sum_{n=1}^N n^2
     \cdot
     \left(2\left(\sum_{k=n}^N 1/k\right) - 1/n\right)\cdot(-1/n) \notag \\
&= \sum_{n=1}^N n \left(2\left(\sum_{k=n}^N 1/k\right) - 1/n\right)
 = \left(2\sum_{n=1}^N n \sum_{k=n}^N 1/k\right)
    - \sum_{n=1}^N n(1/n) \notag \\
&= 2N(N+3)/4 - N
 = N(N+1)/2 \label{eq:harmonic:for:gN}
\end{align}

We now can show that \(g_N\) cannot converge to zero.
By \eqref{eq:ex3.13:gN2} and \eqref{eq:harmonic:for:gN} we have
\begin{equation*}
\|g_N\|_2^2 = N(N+1)/(2N^2) \geq 1/2.
\end{equation*}
Actually \(\lim_{N\to\infty} \|g_N\|_2 = (1/2)^{1/2} > 0\).
Thus clearly \(g_n\) will not do.

\paragraph{Converging to $0$}.
Finally we show that the following, using \eqref{eq:harmonic:numbers},
sequence of convex combinations of \(\{f_j: j\in\N\}\) converges to~$0$.
\begin{equation*}
g_N = (1/H_N)\sum_{n=1}^N f_n /n
\end{equation*}
Hence
\begin{equation*}
H_N \cdot g_N = \sum_{n=1}^N f_n/n.
\end{equation*}
With \eqref{eq:en:eint} we note that
\begin{equation*}
H_N \cdot g_N = \sum_{n=1}^N \left(\sum_{k=n}^N 1/k\right)\cdot
   \left(\sum_{k=(n-1)^2+1}^{n^2} e_k/k\right)
\end{equation*}
and
\begin{equation*}
\sum_{k=m}^n 1/k^2 \leq \int_m^{n+1} x^{-2}\,dx
% = 1/(n+1) - 1/m
= \frac{1}{n+1} - \frac{1}{m}.
\end{equation*}
Now using summation by parts
with \(a_n=(n-1)^2\) and \(b_n = \sum_{k=n}^N 1/k^2\), \(b_{N+1} = 0\)
we get
\begin{align*}
\|H_N \cdot g_N\|_2^2
&= \sum_{n=1}^N \left(n^2 - (n-1)^2\right) \sum_{k=n}^N 1/k^2 \\
% \leq \sum_{n=1}^N (2n - 1)(1/(n+1) - 1/m) \\
&= a_{N+1} \cdot 0 - 0 \cdot b_0
  - \sum_{n=1}^N n^2
    \left(
      \left(\sum_{k=n+1}^N \frac{1}{k^2}\right)^2
      -
      \left(\sum_{k=n}^N \frac{1}{k^2}\right)^2
    \right) \\
&= \sum_{n=1}^N n^2
   \left(
     \left(2\sum_{k=n}^N \frac{1}{k^2}\right) - \frac{1}{n^2}
   \right)
   \cdot \left(\frac{1}{n^2}\right)
 = \sum_{n=1}^N \left(2\sum_{k=n}^N \frac{1}{k^2}\right) - \frac{1}{n^2} \\
&\leq 2 \sum_{n=1}^N \sum_{k=n}^N \frac{1}{k^2}
 \leq 2 \sum_{n=1}^N \bigl(1/n - 1/(N+1)\bigr)
 \leq 2 \sum_{n=1}^N 1/n  = 2H_N.
\end{align*}
Hence
\(\|g_N\|_2^2 \leq 2/H_N \)
and
\(\lim_{N\to\infty} \|g_N\|_2 = 0\).


%%%%%%% 14
\begin{excopy}
\begin{itemize}
%%
\itemch{a}
Suppose \(\Omega\) is a locally compact Hausdorff space.
For each compact \(K\subset \Omega\) define a seminorm \(p_K\) on
\(C(\Omega)\), the space of all complex continuous functions on \(\Omega\), by
\begin{equation*}
p_K(f) = \sup \{|f(x)|: x\in K\}.
\end{equation*}
Give \(C(\Omega)\) the topology induced by this collection of seminorms.
Prove that to every \(\Lambda \in C(\Omega)^*\) correspond a compact
 \(K\subset \Omega\) and a complex Borel  measurable \(\mu\) on $K$ such that
\begin{equation*}
\Lambda f = \int_K f\,d\mu \qquad [f \in C(\Omega).]
\end{equation*}
%%
\itemch{b}
Suppose \(\Omega\) is an open set in \(\C\).
Find a countable collection \(\Gamma\) of measures with compact support in
\(\Omega\) such that \(H(\Omega)\) (the space of all holomorphic functions
in \(\Omega\)) consists of exactly those \(f\in C(\Omega)\) which satisfy
\(\int f\,d\mu = 0\) for every \(\mu\in\Gamma\).
\end{itemize}
\end{excopy}

\begin{itemize}
%%
\itemch{a}
Pick \(\Lambda \in C(\Omega)^*\) and let 
\begin{equation*}
V = \{f \in C(\Omega): |\Lambda f| < 1\}
\end{equation*}
be a neighborhood of the origin. 
It must contain some base neighborhood. In other words
there exists finite number of compact sets \(\{K_j: 1 \leq j \leq n\}\)
such that \(K_j \subset \Omega\) and \(\delta_j > 0\) for \(j \in \N_n\)
and
\begin{equation*}
U := \bigcap_{j=1}^n \{f \in C(\Omega): p_{K_j}(f) < \delta_j\} \subset V.
\end{equation*}
If we put \(K = \cup _{j=1}^n K_j\) and \(\delta = \min\{\delta_j: j\in \N_n\}\)
then 
\begin{equation*}
U' =  \{f \in C(\Omega): p_K(f) < \delta\} \subset U \subset V.
\end{equation*}
Now if \(f \in C(\Omega)\) if \(f_{\restriction \Omega \setminus K} \equiv 0\)
then \(af \in U'\) for all \(a\in \C\) and so 
\(|a\Lambda f| < \delta\)  for all \(a\in \C\), hence \(\Lambda f = 0\).
Thus we can define a functional \(\Lambda_K : C(K) \to \C\)
by \(\Lambda_K f = \Lambda \tilde{f}\)
where \(\tilde{f}\) is some continuous extension of \(f\in C(K)\)
to \(C(\Omega)\) (for example, using Urysohn's lemma).
Now the topology of \(C(K)\) induced by that of \(C(\Omega)\)
is the same as that of the supremum-norm \(\|\cdot\|_\infty\) in $K$.
Applying Riesz representation theorem 
(Theorem~6.19 \cite{RudinRCA87}) gives the desired result.

%%
\itemch{b}
For each closed triangle \(T\subset \Omega\) let \(\gamma_S\)
be its parametrized piecewise differntable boundary
\(\gamma_S : [0,1] \to \boundary{S}\).
Define the regular complex integration for each \(f\in C(\Omega)\)
\begin{equation*}
\int_\Omega  f\,d\mu_S = \frac{1}{2\pi i}\int_0^4 f(\gamma_S(t))\gamma_S'(t)\,dt
\end{equation*}
Now let \(\Gamma\) be the collection of all such \(\mu_S\).
By Morera's Theorem \index{Morera's Theorem} (\cite{RudinRCA87} 10.17)
this \(\Gamma\) with the condition above characterize \(H(\Omega)\).
\end{itemize}

%%%%%%% 15
\begin{excopy}
Let $X$ be a topological vector space on which \(X^*\) separates points.
Prove that the weak \upstar-topology of \(X^*\) is metrizable if and only if
$X$ has a finite or countable Hamel base.
(See Exercise~1, Chapter~2 for the definition.)
\end{excopy}

Assume \(\{u_n: n\in\N\}\) is a Hamel base for $X$. Then
\begin{equation*}
d(\phi,\psi) = \sum_{n=1}^\infty 2^{-n}\cdot |\phi(u_n) - \psi(u_n)|
\end{equation*}
is a metric for \(X^*\). The fact that \(X^*\) separates points
gives \(d(\phi,\psi)=0\) iff \(\phi = \psi\).

(Inspired by \cite{Munkres2000} 2.\S{}21 \textsc{Example}~2.)\newline
Conversely assume $X$ is metrizable with a metric $d$
and \(\{x_\alpha: \alpha \in J\}\)
is a Hamel basis for $X$. By negation assume $J$ is uncountable.
Each \(x\in X\) has a unique representation
\begin{equation*}
x = \sum_{\alpha \in F_x} a_\alpha x_\alpha \qquad
   (a_\alpha \in \C \;\wedge\; |F_x| < \infty)
\end{equation*}
Each \(\Lambda \in X^*\) is determined by the values
 \(\{\Lambda x_\alpha: \alpha \in J\}\).
Now define the set \(A \subset X^*\) such that 
it is ``almost always''~$1$ on the basis, formally
\begin{equation*}
A = \bigl\{ \Lambda\in X^*: 
      \{\alpha \in J: \Lambda x_\alpha \neq 1\} < \infty 
    \bigr\}.
\end{equation*}
We will show that \(0 \in \overline{A}\).
Pick a base neighborhood of \(0\in X^*\) by selecting some finte 
set \(F \subset J\) and \(\delta>0\) and define
\begin{equation*}
V = \{\Lambda \in X^*: \forall \alpha\in F\; |\Lambda x_\alpha| < \delta\}.
\end{equation*}
Define \(T\in X^*\) by 
\begin{equation*}
\T x_\alpha = 
\left\{
 \begin{array}{ll}
 0 \quad& \alpha \in F \\
 1 \quad& \alpha \notin F
 \end{array}
\right.
\end{equation*}
Clearly \(T \in V\), hence \(0 \in \overline{A}\).

By the assumed existence of $d$, we define the open balls
\begin{equation*}
B_n = \{\Lambda \in X^*: d(0,\Lambda) < 1/n\}.
\end{equation*}
Since \(0 \in \overline{A} \subset X^*\), we can find
\(a_n \in B_n \cap A\) thus 
\begin{equation} \label{eq:metriz:lim:a_n:0}
\lim_{n\to\infty} a_n = 0\,.
\end{equation}
Define ``non-$1$'' sets
\begin{equation*}
F_n = \{\alpha \in J: a_n(x_\alpha) \neq 1\}.
\end{equation*}
By definition of $A$, we have \(|F_n| < \infty\)
and so \(|\cup_{n\in\N} F_n|< \aleph_0\).
But $J$ is uncountable (the Hamel basis) and hence there exists
\(\gamma \in J \setminus \cup_{n\in\N} F_n\).
Thus \(a_n(x_\gamma) = 1\) for all \(n\in\N\), 
which contradicts \eqref{eq:metriz:lim:a_n:0}.


%%%%%%% 16
\begin{excopy}
Prove that the close unit ball of \(L^1\) (relative to the Lebesgue measure
on the unit interval) has no extreme points but that every point on the
"surface" of the unit ball in \(L^p\) (\(1<p<\infty\)) is an extreme
point of the ball.
\end{excopy}

\paragraph{No exterme.}
Given \(f\in L^2([0,1])\) such that \(\|f\|_1 = 1\),
% Define \(a:[0,1] \to \{z\in\C: |z|=1\}\) by \(a(t) = f(t)/|f(t)|\) if
% \(f(t)\neq 0\) and \(a(t)=1\) otherwise.
by regularity of the Lebesgue measure, there exists \(b\in[0,1]\)
such that
\begin{equation*}
\int_0^b |f(t)|\,dt = \int_b^1 |f(t)|\,dt = 1/2.
\end{equation*}
Now define \(f_1,f_2\in L^1([0.1])\) by
\begin{equation}  \label{eq:f1f2:conv}
\bigl(f_1(t),f_2(t)\bigr) =
\left\{
\begin{array}{ll}
\bigl(0,2f(t)\bigr) \quad & 0 \leq t \leq b \\
\bigl(2f(t),0\bigr) \quad & b \leq t \leq 1
\end{array}
\right.
\end{equation}
Clearly \(\|f_1\|_1 = \|f_2\|_1 = 1\)
and \(f = (f_1 + f_2)/2\). Hence the unit ball of \(L^1\)
has no extreme points.

\paragraph{All exterme.}
Suppose \(f\in L^p([0,1])\) where \(0<p<1\) and \(\int_0^1 |f|^p = 1\).
By regularity of the Lebesgue measure, there exists \(a \in (0,1)\) such that 
\begin{equation*}
\int_0^a |f|^p = \int_a^1 |f|^p = 1/2.
\end{equation*}
Define
\begin{equation*}
f_1(t) = \left\{
  \begin{array}{ll}
  2f(t) \quad & 0 \leq t < a \\
  0 \quad & a \leq t \leq 1
  \end{array}
\right.
\qquad
f_2(t) = \left\{
  \begin{array}{ll}
  0 \quad & 0 \leq t < a \\
  2f(t) \quad & a \leq t \leq 1
  \end{array}
\right.
\end{equation*}
It is clear that we have a convex combination \(f = (f_1 + f_2)/2\) and
\begin{equation*}
\int_0^1 |f_j|^p\,dm 
= \int_{\{0,a\}}^{\{a,1\}} |2f|^p\,dm
= 2^p \int_{\{0,a\}}^{\{a,1\}} |f|^p\,dm 
= 2^{p-1} < 1.
\end{equation*}
for \(j=1,2\). Hence \(f_1\),\(f_2\) are in the unit ball of \(L^p\)
and $f$ is not an extreme point there.


%%%%%%% 17
\begin{excopy}
Determine the extreme points of the closed unit ball of $C$, the space of all
continuous functions on the unit interval, with the supremum norm.
(The answer depends on the choice of the scalar field.)
\end{excopy}

When the scalar field is \R\ then the extreme points are
\(f=1\) and \(f= -1\).
When the scalar field is \C\ then the set of extreme points is
\begin{equation*}
\{f\in C(I): \forall t\in I,\, |f(t)|=1\}.
\end{equation*}


%%%%%%% 18
\begin{excopy}Let $K$ be the smallest convex set in \(\R^3\) that contains
the points
\((1,0,1)\)
\((1,0,-1)\), and \((\cos \theta, \sin \theta, 0)\)
for \(0\leq \theta \leq 2\pi\).
Show that $K$ is compact but that the set of all extreme points of $K$ is not
compact. Does such an example exist in \(\R^2\)?
\end{excopy}

Look at the set
\begin{equation*}
G = \{((1,0,1), (1,0,-1)\} \cup
    \{(\cos \theta, \sin \theta, 0): t \in [0,2\pi]\}.
\end{equation*}
and the mapping
\begin{align*}
\varphi: G \times G \times [0,1] &\to E \\
\varphi(u,v,t) &= tu + (1-t)v\,.
\end{align*}
Now $K$ is the image of the continuous mapping with the compact domain
 \(G^2\times [0,1]\) and thus $K$ is compact.
But the set of extreme points is \(E = G \setminus \{(1, 0, 0)\}\)
which is \emph{not} closed and thus not compact.

Such situation cannot happen in \(\R^2\). By negation let \(K\subset \R^2\)
be a compact convex set and \(E(K)\) be not compact.
Thus we have a sequence \((x_n)_{n\in\N}\) in $E$ such that
\(x = \lim_{n\to\infty} x_n \in K\setminus E\).
Now since $x$ is not an extreme point \(x \in (a,b)\) where \([a,b]\subset K\).
\Wlogy, we may assume (by changing to a sub-sequenc) that 
\((x_n)_{n\in\N}\) lie on one side of \([b,c]\).
Pick a sufficently small neighborhood $V$ of $x$ such that \(a,b\notin V\).
Now thee must be some \(j>1\) such that \(x_j \in V\).
Now \(x_j\) is internal to the triangle \(\triangle(x_1,a,b)\),
a~contradiction to \(x_j \in E\).


%%%%%%% 19
\begin{excopy}
Suppose $K$ is a compact convex set in \(\R^n\). Prove that every \(x\in K\)
is a convex combination of at most \(n+1\) extreme points of $K$.
\emph{Suggestion:} Use induction on $n$. Draw a line from some extreme point
of $K$ through $x$ to where it leaves $K$. Use Exercise~1.
\end{excopy}

If there are $m$ extreme points in $K$ and
\(m < n+1\), then the hyperplane
generated by them is isomorphic with isometry is to \(\R^k\)
with \(k \leq m-1\) and we can reduce the problem to \(\R^k\).
Hence we may assume there are at least \(n+1\) extreme points.

\iffalse
If \(b \in \partial K\) is a non-extreme boundary point of $K$, define its
\emph{face} \(F_b\) by
\begin{equation*}
F_b = \{tv + (1-t)w:
   v,w\in K \wedge 0\leq t \leq 1 \wedge\;
   \exists \tau \in(0,1),\, b = \tau v + (1-\tau)u\}.
\end{equation*}

We want to show that \(F_b\) is comact, convex and lies with a hyperplane
of dimension less that that of $K$.
\fi

Following the suggestion. If \(n=1\) then $K$ must be a simple segment
and the result easily follows. Now assume the claim holds for all \(n < m\).
Now assume \(n=m\). Pick some extreme point $v$ and let ray $r$ coming from $v$
towards $x$, that is \(r=\{v + t(x-v): t\in\R^+\}\).
This ray intersects $K$ as a line segment whose end points are
the extreme point $v$ and another boundary point $b$ of $K$.
We pass a hyperplane $H$ in $b$ such that $K$ lies on of of its sides.
Now \(K' = H \cap K\) is clearly compact and convex and can be viewed
(by isometry) as a compact convex set in \(R^{m-1}\).

We claim that the extreme points of \(K'\) are extreme points of $K$.
Otherwise by negation, we have an extreme point \(y \in K'\) which lies in
an interior of line segment
\begin{equation*}
y \in (y_0,y_1) \subsetneq [y_0,y_1] \subset K.
\end{equation*}
But \([y_0,y_1]\) lies within one side of $H$ and since it intersects $H$
in $b$, both \(y_0,y_1 \in H\) contradiction to $y$ being extreme in \(K'\).

Now by induction $b$ can be represented as a convex combination
of less than \(m+1\) extreme points. Consequently, since \(x \in [b,v]\),
there exists a convex combination of
\(m+1\) extreme points (or less) for $x$.

%%%%%%% 20
\begin{excopy}
Let \(\{u_1, u_2, u_3,\ldots\}\) be a sequence of pairwise orthogonal unit
vectors in a Hilbert space.
Let $K$ consist of the vectors $0$ and \(n^{-1}u_n\) (\(n\geq 1\)).
Show that
\ich{a} $K$ is compact;
\ich{b} \(\co(K)\) is bounded.
\ich{c}~\(\co(K)\) is not closed.
Find all extreme points of \(\overline{\co}(K)\).
\end{excopy}

\begin{itemize}
\itemch{a}
Let \(G = \cup_{j\in J}V_j\) be a open cover of $K$.
Pick some \(j_0\in J\) such that \(0\in V_{j_0}\).
Now for some \(N<\infty\) we have \(u_n/n \in V_j\) for all \(n > N\).
Take \(V_{j_0}\) and completeness a finite subfamily of G by picking
\(\{j_n\in J: 1\leq n \leq N\}\) such that \(u_n/n \in V_{j_n}\)
whenever \(1\leq n \leq N\).
\itemch{b}
For each \(v\in \co(K)\) clearly \(\|x\|\leq 1\) hence \(\co(K)\) is bounded.
\itemch{c}
The points \(v_n = \sum_{j=1}^n 2^{-j}u_j \in \co(K)\)
but the limit
   \(\sum_{j=1}^\infty 2^{-j}u_j \in \overline{\co(K)} \setminus \co(K)\).
Hence \(\co(K)\) is not closed.
\end{itemize}
The extreme points of \(\overline{\co(K)}\) are $0$, \(\{u_n:n\in\N\}\)
and all ``infinite convex combinations'' points of the form
\(\sum_{j=1}^\infty a_j u_j\) such that
\begin{gather*}
\forall j\in\N,\;0 \leq a_j < 1 \\
\sum_{j=1}^\infty a_j = 1\\
a_j > 0 \qquad \textnormal{for infinitely many }\; j\in\N.
\end{gather*}


%%%%%%% 21
\begin{excopy}
If \(0<p<1\), every \(f\in L^p\) (except \(f=0\)) is the arithmetic mean
of two functions whose distance from~$0$ is less than that of $f$.
(See Section~1.47.)
Use this to construct an explicit example of a countable compact set $K$
in \(L^p\) (with $0$ as its only limit point) which has no extreme point.
\end{excopy}

Assume \(0<p<1\).
Pick \(f \in L^p([0,1]) \setminus \{0\}\).
Let \(b\in[0,1]\) such that 
\begin{equation*}
\int_0^b |f(t)|^p\,dt = \int_b^1 |f(t)|^p\,dt.
\end{equation*}
Define \(f_1\), \(f_2\) as in~\eqref{eq:f1f2:conv}.
Clearly \(f = (f_1 + f_2)/2\), and
\begin{equation*}
d(0,f_1) 
= \int_0^b |f(t)|^p\,dt 
= 2^p \int_0^b |f(t)|\,dt 
= 2^p (d(0,f)/2)
= 2^{p-1} d(0,f) < d(0,f).
\end{equation*}
And similarly, \(d(0,f_2) < f(0,f)\).

To construct the example, let \(u == 1 \in L^p([0,1]\)
and let \(S_0 = \{f\}\).
By induction, Given a set \(S_k\) of functions in \(L^p([0,1])\)
define \(S_{k+1}\) as the set of functions \(f_1\), \(f_2\) 
defined in~\eqref{eq:f1f2:conv} where \(f\in S_k\)
(thus \(|S_k| = n^k\)).
Clearly if \(f\in S_k\) then \(d(0,f) = 2^{n(p-1)}\)
and actualy there exists some \(0 < j \leq 2^k\) such that 
\(f(t)=1\) if \(t \in (2^{-k}(j-1), 2^{-k}j)\) and 
\(f(t)=0\) if \(t \in [0,1] \setminus [2^{-k}(j-1), 2^{-k}j]\).

Let \(U = \cup_{k=\N} S_k\). Now $U$ has no extreme point, 
since each \(f\in S_k\) is th average of two functions
from \(S_{k+1}\). If \(f_k\in S_k\) then \(\lim_{k\to\infty} d(0,f_k) = 0\), 
hence \(0 \in \overline{U}\) thus if 
\(U^- = \{-f: f\in U\}\), then
\begin{equation*}
K = U \cup \{0\} \cup U^-
\end{equation*}
is the desired compact set with no extreme point.

Notes:
\begin{itemize}
\item
 Eac open neighborhood of $0$ leaves out just a finite 
 number of elements of $K$, hence $K$ is compact.
\item
 \(U \cup \{0\}\) is compact but $0$ is extreme point there.
\end{itemize}


%%%%%%% 22
\begin{excopy}
If \(0<p<1\), show that \ellp\ contains a compact set $K$ whose convex hull is
unbounded. This happens in spite of the fact that \((\ellp)^*\)
separates points in \ellp; see Exercise~5.
\emph{Suggestion:} Define \(x_n \in \ellp\) by
\begin{equation*}
x_n(n) = n^{p-1}, \qquad x_n(m) = 0 \qquad \textnormal{if}\; m\neq n.
\end{equation*}
Let $K$ consist of \(0, x_1, x_2, x_3,\dots\) If
\begin{equation*}
y_N = N^{-1}(x_1 + \cdots + x_N),
\end{equation*}
show that \(\{y_n\}\) is unbounded in \ellp.
\end{excopy}

Clearly \(\lim_{n\to\infty} d(0,x_n) = n^{p(p-1)} = 0\).
Hence single limit point ($0$) and $K$ is compact.
Now
\begin{align*}
d(0,y_N) 
&= \sum_{k=1}^N \left(N^{-1} k^{p-1}\right)^p
 = N^{-p} \sum_{k=1}^N  k^{p(p-1)}
 \geq N^{-p} \int_2^N x^{p(p-1)}\,dx \\
&= N^{-p} \frac{1}{p(p-1)+1}\left(x^{p(p-1)+1}\right)\bigm|_2^N
\end{align*}
But
\begin{align*}
\lim_{N\to\infty} N^{-p}N^{p(p-1)+1}
= \lim_{N\to\infty} N^{p(p-1)+1-p}
= \lim_{N\to\infty} N^{(p-1)^2} = \infty.
\end{align*}
Hence \(\lim_{N\to\infty} d(0,y_N) = \infty\).


%%%%%%% 23
\begin{excopy}
Suppose \(\mu\) is a Borel probability measure on a compact Hausdorff
space $Q$. $X$ is a Fr\'echet space, and \(f:Q\to X\) is continuous.
A \emph{partition} of $Q$ is, by definition, a finite collection of disjoint
Borel subsets of $Q$ whose union is $Q$. Prove that to every neighborhood
$V$ of $0$ in $X$ there corresponds a partition \(\{E_i\}\) such that the
difference
\begin{equation*}
z = \int_Q f\,d\mu - \sum_i \mu(E_i)f(s_i)
\end{equation*}
lies in $V$ for every choice of \(s_i \in E_i\).
(this exhibits the integral as a strong limit of ``Riemann sums.'')
\emph{Suggestion:} Take $V$ convex and balanced.
If \(\Lambda \in X^*\) and if \(|\Lambda x| \leq 1\) for every \(x \in V\),
then \(|\Lambda x| \leq 1\), provided that the sets \(E_i\) are chosen so
that \(f(s) - f(t) \in V\) wherever $s$ and $t$ lie in the same \(E_i\).
\end{excopy}

Let $V_0$ be an arbitrary neighborhood of $0$.
Pick a balanced neighborhood $V$ of \(0\in X\) 
such that \(\overline{V} \subset V_0\).
For each \(q\in Q\) find a neighborhood \(U_q \ni q\) such that 
\(f(U_q) \subset f(q) + V/2 = \{x\in X: 2(x - f(q)) \in V\}\).
By compactness of $Q$, there is 
a~finite set \(F = \{q_1,\ldots,q_n\} \subset Q\)
such that \(Q = \cup_{q\in F} U_q\).

Define \(E_1 = U_{q_1}\) and \(E_j = U_{q_j} \setminus \cup_{k<j} E_k\)
for \(1 < j \leq n\).
If \(s_1,s_2\in E_j\) then
\(f(s_k) - f(q) \in V/2\) for \(k=1,2\) and so 
\(f(s_1) - f(s_2) \in V\).

If \(\Lambda \in X^*\) then
\begin{align*}
\Lambda z
 &= \Lambda\left( \int_Q f\,d\mu - \sum_i \mu(E_i)f(s_i) \right)
  = \int_Q \Lambda f\,d\mu - \sum_i \mu(E_i) \Lambda f(s_i)  \\
 &= \sum_i \int_{E_i} \bigl(\Lambda f -  \Lambda f(s_i)\bigr)\,d\mu
  = \sum_i \int_{E_i} \Lambda \bigl(f - f(s_i)\bigr)\,d\mu.
\end{align*}

Assume \(|\Lambda x| \leq 1\) for every \(x \in V\). Now
\begin{equation}  \label{eq:lambdaz:leq1}
|\Lambda z| 
 \leq \sum_i \int_{E_i} \Lambda \bigl(f - f(s_i)\bigr)\,d\mu
 \leq \sum_i \int_{E_i} 1,d\mu
 = \sum_i \mu(E_i) = 1.
\end{equation}

If by negation \(z\notin V_0\) then by Theorem~3.7
 there exists \(\Lambda\in X^*\)
such that \(|\Lambda x| \leq 1\) for all \(x \in V\)
and \(\Lambda z > 1\). This contradicts \eqref{eq:lambdaz:leq1}.


%%%%%%% 24
\begin{excopy}
In addition to the hypothesis of Theorem~3.27, assume that $T$ us a continuous
linear mapping of $X$ into a topological vector space $Y$ on which \(Y^*\)
separates points and prove that
\begin{equation*}
T\int_Q f\,d\mu = \int_Q (Tf)\,d\mu.
\end{equation*}
\emph{Hint:} \(\Lambda T \in X^*\) for every \(\Lambda \in Y^*\).
\end{excopy}

We use the hint and the result of previous exercise, namely approximation 
of the intergrals via Riemann sums.

Pick arbitrary neighborhood \(U \subset Y\) of \(o\in Y\).
Let \(V = T^{-1}(U)\) which is clearly a neighborhood of \(0\in X\).
For each \(P = \{E_j: j \in N_n\}\) 
and a selection $S$ of \(s_j \in E_j\) for each \(j\in \N_n\) 
we note that
\begin{equation*}
r(P,S) = T\left(\sum_{j=1}^n \mu(E_i)f(s_j)\right) = \sum_{j=1}^n \mu(E_i)(Tf)(s_j).
\end{equation*}

Let \(P = \{E_j: j \in N_n\}\) be a Borel disjoint partition of $X$ such that if
\begin{align*}
z &= \int_Q f\,d\mu - \sum_{j=1}^n \mu(E_i)f(s_j) \in X \\
w &= \int_Q Tf\,d\mu - \sum_{j=1}^n \mu(E_i)(Tf)(s_j) \in Y
\end{align*}
then \(z \in V\) and \(w \in U\) for every choice of $S$.
A mutual partition can be taken by a ``cross intersection'' of two partitions
such that each condition is satisfied by one partition.

Now, such $P$ gives
\begin{align*}
\xi
&= T\int_Q f\,d\mu - \int_Q (Tf)\,d\mu
 = T\left(z + \sum_{j=1}^n \mu(E_i)f(s_j)\right) 
   - w - \sum_{j=1}^n \mu(E_i)(Tf)(s_j) \\
&= Tz - w + r(P,S) - r(P,S) = Tz - w \in U + U.
\end{align*}
Since $U$ was an arbitrary neighborhood of \(0 \in Y\), we must have \(\xi = 0\).


%%%%%%% 25
\begin{excopy}
Let $E$ be the set of all extreme points of a compact set $K$ in a topological
vector space $X$ on which \(X^*\) separates points.
Prove that to every \(y \in K\) corresponds a regular Borel probability measure
\(\mu\) on \(Q = \overline{E}\) such that
\begin{equation*}
y = \int_Q x\,d\mu(x).
\end{equation*}
\end{excopy}

We start with some generalization of 
the 
Banach-Alaoglu
\index{Banach-Alaoglu} Theorem~3.15.
The theorem
 shows that
if $V$ is a neighborhood of \(0\in X\) then the polar
\begin{equation}  \label{eq:polar:sup:nbd}
P^*(V) = \{\Lambda \in X^*: \forall x\in V,\;|\Lambda x| \leq 1 \}
\end{equation}
is weak\upstar-compact.
Following the proof one can see that it holds also when
$V$ is any subset that \emph{contains} a neighborhood of \(0\in X\)
(but $V$ is not necessarily open).


The set $E$ may not be compact, but it is locally compact
since \(E \subset \overline{E} \subset K\).
Since  \(Q \subset K\) it is also compact.
We look at the space \(C(Q) = C_c(Q)\) and its dual space \(C(Q)^*\).

The set \(\conv(E)\) consists of finite convex combinations of points of $E$.
Thus for each \(v \in \conv(E)\) we have
\begin{align*}
v &= \sum_{k=1}^n a_k w_k \qquad (0 < a_k \leq 1, \; w_k \in Q) \\
1 &= \sum_{k=1}^n  a_k
\end{align*}
and we assoicate \(\Lambda_v \in C(K)^*\) by
\begin{equation*}
\Lambda_v (f) = \sum_{k=1}^n a_k f(w_k) \qquad (f \in C(K)).
\end{equation*}

By Riesz representation Theorem (\cite{RudinRCA80})
there exists a (not necessarily unique) measure \(\mu_v\) such that 
\(\Lambda_v(f) = \int_Q f(x)\,d\mu_v(x)\),
but we can simply define \(\mu_v\) by \(\mu(\{w_k\}) = a_k\)
and \(\mu_v(A) = 0\) for any Borel subset \(A\subset Q\)
such that \(\forall k,\; w_k\notin A\).
Now for \(f(x)=x\) we have
\begin{equation} \label{eq:conv:measure:x:v}
\int_Q x\,d\mu_v(x) = \sum_{k=1}^n a_k w_k = v.
\end{equation}

Let \(F\subset C(X)^*\) be all the functionals \(\Lambda_v\)
where for each \(v\in \conv(Q)\) we take all choices of \(\Lambda_v\).
If \(U = \{f\in C(K): \|f\|_\infty \leq 1\}\), then  clearly
\(F \subset \in P^*(U)\) (as defined in \eqref{eq:polar:sup:nbd}).

Now \(\overline{F} \subset P^*(U)\), hence it is weak\upstar-compact.
If we look at the identity function \(\Id(x)=x\)
as a functional on \(C(X)^*\) then the image of \(\Id(\overline{F})\)
is compact. Since we saw that \(\conv{Q} \subset \Id(\overline{F})\)
then  \(\overline{\conv{Q}} \subset \Id(\overline{F})\) as well
(weak\upstar-closure).
By Krein-Milman
\index{Krein-Milman}
Theorem~3.23 \(K = \overline{\conv(Q)}\).
Hence, for every \(y \in K\) there exists some 
\(\Lambda_y \in \overline{F}\) such that \(Id(\Lambda) = y\).
Hence by the Riesz  representation Theorem (\cite{RudinRCA80})
\eqref{eq:conv:measure:x:v} also for all \(y \in K\)
for some Borel measure \(\mu_y\).
Being in the closure of probability measures, such measures
is a probability measure as well.


%%%%%%% 26
\begin{excopy}
Suppose \(\Omega\) is a region in \C, $X$ is a Fr\'echet space, and
\(f:\Omega\to X\) is holomorphic.
\begin{itemize}
\itemch{a}
State and prove a theorem concerning the power series representation of $f$,
that is, concerning the formula \(f(z) = \sum(z-a)^n c_n\), where \(c_n\in X\).
\itemch{b}
Generalize Morera's theorem to $X$-valued holomorphic functions.
\itemch{c}
For a sequence of complex holomorphic functions in \(\Omega\),
uniform convergence on a compact subsets of \(\Omega\) implies that the limit
is holomorphic.
Does this generalize to $X$-valued holomorphic functions?
\end{itemize}
\end{excopy}

\begin{itemize}
\itemch{a}
With the assumptions above,
If \(f:\Omega \to X\) is holomorphic, then for each open circle $U$
with center $a$, such that \(\overline{U} \subset \Omega\)
there exists a sequence \((c_n)_{n\in\Z^+}\) in $X$ such that 
\begin{equation*}
f(z) = \sum(z-a)^n c_n
\end{equation*}
holds for all \(z\in U\).

Let \(\Gamma\) be a parmetrized boundary of $U$.
By Theorem~3.31\ich{b}
\begin{equation*}
f(z) = \frac{1}{2\pi i}\int_\Gamma (\zeta -z)f(\zeta)\,d\zeta.
\end{equation*}
Using ideas as in Theorems~10.7 and 10.16 of \cite{RudinRCA80}
we note that
\begin{equation*}
\frac{1}{\zeta - z} = \sum_{n=0}^\infty \frac{(z-a)^n}{(\zeta - a)^{n+1}}
\end{equation*}
and define
\begin{equation*}
c_n = \dtwopii \int_\Gamma (\zeta - z)^{-n-1}f(\zeta)\,d\zeta
\end{equation*}
By Theorem~3.31\ich{b}
\begin{align}
f(z) 
&= \frac{1}{2\pi i}\int_\Gamma (\zeta -z)^{-1} f(\zeta)\,d\zeta
 = \frac{1}{2\pi i}\int_\Gamma 
   \left(\sum_{n=0}^\infty \frac{(z-a)^n}{(\zeta - a)^{n+1}}\right) 
   f(\zeta)\,d\zeta \notag \\
&= \frac{1}{2\pi i}\int_\Gamma 
   \left(\sum_{n=0}^\infty (z-a)^n(\zeta - a)^{-n-1}\right) f(\zeta)\,d\zeta
   \notag\\
&= \label{eq:holo:tvs:sigmaintegral}
   \frac{1}{2\pi i}
   \sum_{n=0}^\infty
     \left(\int_\Gamma (\zeta - a)^{-n-1} f(\zeta)\,d\zeta \right)
     (z-a)^n \notag \\
&= \sum(z-a)^n c_n
\end{align}
The equality \eqref{eq:holo:tvs:sigmaintegral} is justified
by the uniform convergence of the infinite sum
and the fact that \(\Lambda f\) is bounded on \(\Gamma^*\)
for each \(\Lambda \in X^*\).

\itemch{b}
With the assumptions above,
if for any closed path \(\gamma\) such that \(\gamma^* \subset \Omega\)
the equality
\begin{equation*}
\int_\gamma f(z) = \int_0^1 f(\gamma(t))\gamma'(t)\,dt = 0
\end{equation*}
holds, then $f$ is holomorphic.

\itemch{c}
Yes it does generalize.
Say \(f_n:\Omega \to X\) is such a sequence of holomorphic functions.
Let \(f(z) = \lim_{n\to\infty} f_n(z)\).
Now for any triangle \(T \subset \Omega\) clearly 
\begin{equation*}
\int_{\boundary{T}} f(z)\,dz
= \int_{\boundary{T}} \left(\lim_{n\to\infty} f_n(z)\right) dz 
= \lim_{n\to\infty} \int_{\boundary{T}} f_n(z) dz 
= 0
\end{equation*}
Now by \ich{b} $f$ is holomorphic.
\end{itemize}


%%%%%%% 27
\begin{excopy}
Suppose \(\{\alpha_i\}\) is a bounded set of distinct complex numbers,
\(f(x) = \sum_0^\infty c_n z^n\) is an entire function with every \(c_n\neq 0\)
and
\begin{equation*}
g_i(z) = f(\alpha_iz).
\end{equation*}
Prove that the vector space generated by the functions \(g_i\) is dense in
the Fr\'echet space \(H(\C)\) defined in Section~1.45.

\emph{Suggestion:} Assume \(\mu\) is a measure with compact support such that
\(\int g_i\,d\mu = 0\) for all $i$. Put
\begin{equation*}
\phi(w) = \int f(wz)\,d\mu(z) \qquad (w\in\C).
\end{equation*}
Prove that \(\phi(w) = 0\) for all $w$. Deduce that \(\int z^n\,d\mu(z) = 0\)
for \(n=1,2,3,\ldots\). Use Exercise~14.

Describe the closed subspace \(H(\C)\) generated by the functions \(g_i\),
if some of the \(c_n\) are~$0$.
\end{excopy}

\textbf{Note.} The assumption should require explicitly that the 
set  \(\{\alpha_i\}\) is infinite. Otherwise there are trivial examples
that make the claim false.

Following the suggestion.
\begin{equation*}
\phi(w)
= \int \left(\sum_0^\infty c_n (wz)^n\right)d\mu(z)
= \sum_0^\infty c_n  \left(\int z^n\,d\mu(z)\right) w^n.
\end{equation*}
Hence \(\phi\) is holomorphic.
Since \(\phi(\alpha_j) = 0\) and \(\{\alpha_j:j\in J\}\)
has a limit point in \C,
thus \(\phi\) is identically zero by Theorem~10.14~\cite{RudinRCA80}.
But this shows that 
\begin{equation*}
c_n  \int z^n\,d\mu(z) = 0
\end{equation*}
for all \(n\in \N^+\),
and also
\begin{equation} \label{eq:integral:mu:zn:eq0:HC}
\forall n\in\N+\; \int z^n\,d\mu(z) = 0
\end{equation}
 since \(c_n \neq 0\) for all \(n\in \N^+\).

The functions \(z\to z^n\) span a vector space that is dense in
\(C(K)\) for any compact \(K \subset \C\).
By Exercise~14 \(\int f\,d\mu = 0\) for all \(f \in H(\C)\).

If by negation the vector space $V$ generated by \(\{g_j\}\)
were not dense in \(H(\C\) then by Hahn-Banach Theorem~3.5
and Riesz representation Theorem~3.16 \cite{RudinRCA80}
we could find some Borel measure \(\mu\) with compact support
such that  \(\int f\,d\mu = 0\) for all \(f\in \overline{V}\)
but still \(\int f\,d\mu \neq 0\)
for some \(f \in H(\C) \setminus \overline{V}\).
But this contradicts \eqref{eq:integral:mu:zn:eq0:HC}.

If some \(c_n = 0\) then the closed subspace generated by the
the functions \(\{g_j\}\) has co-dimension 1.


%%%%%%% 28
\begin{excopy}
Suppose $X$ is a Fr\'echet space (or, more generally, a metrizable locally
convex space). Prove the following statements:
\begin{itemize}
\itemch{a}
\(X^*\) is the union of countably many weak\upstar-compact sets \(E_n\).
\itemch{b}
If $X$ is separable, each \(E_n\) is metrizable. The weak\upstar-topology of
\(X^*\) is therefore separable, and some countable subsets of \(X^*\) separates
points on~$X$.
(Compare with Exercise~15.)
\itemch{c}
If $K$ is a weakly compact subset of $X$ and if \(x_0\in K\) is a  weak limit
point of some countable set \(E\subset K\), then there is a sequence
\(\{x_n\}\) in $E$ which converges weakly to \(x_0\).
\emph{Hint:} Let $Y$ be the smallest closed subspace of $X$ that contains $E$.
Apply \ich{b} to $Y$ to conclude that the weak topology of \(K \cap Y\) is
metrizable.

\emph{Remark:} The point of \ich{c} is the existence of convergent
\emph{subsequences} rather than \emph{subnets}. Note that there exist compact
Hausdorff spaces in which no sequence of distinct points converges.
For an example, see Exercise~18, Chapter~11.
\end{itemize}
\end{excopy}

\begin{itemize}
\itemch{a}
By definition, there is a metric $d$ on $X$, such that
\begin{equation*}
U = \{x\in X: d(x,0) \leq 1\}
\end{equation*}
is convex.
Let 
\begin{equation*}
K_n = \{\Lambda \in X^*: \forall x\in U,\, |\Lambda x| \leq n\}\
\qquad (n\in\N).
\end{equation*}
Clearly \(K_n = \{n\Lambda \in X^*: \Lambda \in K_1\}\) and
by
\index{Banach-Alaoglu}
\index{Alaoglu}
Banach-Alaoglu Theorem~3.15 the sets \(K_n\) are weak\upstar-compact
and clearly \(X^* \cup_{n\in\N} K_n\).

\itemch{b}
By Theorem~3.16 each \(E_n\) is metrizable.
For each \(m\in\N\) there is a finite covering of \(E_n\)
with balls of radius \(1/m\). Thus \(E_n\) separable
and consequently so is \(X^* = \cup_{n\in\N} E_n\).

Let \(A \subset X^*\) be a countable weak\upstar\ dense set.
By negation let \(x_1,x_2\in X\) and \(x_1\neq x_2\)
and \(\Lambda x_1 = \Lambda x_2\) for all \(\Lambda \in A\).
By Hahn-Banach Theorem~3.4 
(with \(A=\{x_1\}\) and \(B=\{x_2\}\))
there exists \(\Lambda \in X^*\)
such that \(\Lambda x_1 \neq \Lambda x_2\).
% Let \(z_j = \Lambda x_j\) for \(j=1,2\).
Let
% \(\epsilon = |z_1 - z_2|/2\) and
\(V_j = \{T\in X^*: |T x_j - \Lambda x_j| < \epsilon/2\}\)
for \(j=1,2\).
Clearly \(V_1 \cap A = \emptyset\) or \(V_2 \cap A = \emptyset\).
Therefore $A$ separates points  in $X$.

\itemch{c}
Following the hint.
The subspace $Y$ is originally closed, and by Corollary~\ich{a}
of Theorem~3.12 it is also weak\upstar\ closed.
Thus \(K \cap Y\) is weak\upstar\ compact.
The set
\begin{equation*}
A = \{(q+ir)v \in Y: v\in E \;\wedge\; q,r \in \Q\}
\end{equation*}
is countble and dense in $Y$, thus $Y$ is separable.
and so is \(K \cap Y\).
By \ich{b} there exists a countble set \(S \subset Y^*\) that separates
points in $Y$

By remark \ich{c} of Section~3.8
(where $A$ should be extended to sufficiently rich real valued functions)
we see that  \(K \cap Y\) is metrizable.
With such metric $d$, we give \(x_0\) a countable local base
 \(\{B_d(x_0;1/n): n\in\N\}\).
By choosing \(x_n \in B_d(x_0;1/n) \cap E\) we have 
\begin{equation*}
{\lim}_{n\to\infty}^{\textnormal{w}^*} x_n = x_0
\end{equation*}
as desired.
\end{itemize}


%%%%%%% 29
\begin{excopy}
Let \(C(K)\) be the Banach space of all continuous complex functions on the
compact Hausdorff space $K$, with the supremum norm.
For \(p\in K\), define \(\Lambda_p\in C(K)^*\) by \(\Lambda_p f = f(p)\).
Show that \(p \to \Lambda_p\) is a homeomorphism of $K$ into \(C(K)^*\),
equipped with its weak\upstar-topology. Part \ich{c} of Exercise~28
can therefore not be extended to weak\upstar-compact sets.
\end{excopy}

Call the mapping  \(\varphi\), and
by Urysohn's lemma (\cite{RudinRCA87} 2.12)
it is injective.
We will form a base neighborhood of \(\Lambda_p\).
Pick a finite \(F = \{f_j\in C(K): 1 \leq j \leq n\}\) set 
and \(\epsilon > 0\). Now
\begin{equation*}
V = \{T \in C(X)^*: \forall j\in\N_n,\; |T(f_j) - \Lambda_p(f_j)| < \epsilon\}.
\end{equation*}
By continuity of \(f_j\in F\) there exists some neighborhood \(U \in K\)
of $p$ such that \(|f_j(x) - f_j(p)| < \epsilon\)
for all \(x\in U\) and \(j\in\N_n\).
Hence \(\Lambda_x \in V\) for all \(x\in U\).
Thus the mapping is continuous (and so \(\varphi(K)\) is compact).
The induced topology \(\tau_1\), of \(C(K)^*\) on \(\varphi(K)\)
cannot be richer than that of induced by \(\varphi\), that is
\begin{equation*}
\tau_2 = \{\varphi(G): G \;\textnormal{open in}\; K\}.
\end{equation*}
By continuity of \(\varphi\) we have \(\tau_1 \subset \tau_2\).
Since \(\tau_1\) is clearly Hausdorff, by the remark~\ich{a}
of Section~3.8 therefore \(\varphi\) is a homeomorphism.


%%%%%%% 30
\begin{excopy}
Suppose $p$ is an extreme point of some convex set $K$, and that
\(p = t_1x_1 + \cdots t_n x_n\), where \(\sum t_i = 1\), \(t_i > 0\)
and \(x_i \in K\) for all $i$.
Prove that \(x_i = p\) for all $i$.
\end{excopy}

By induction. The case \(n=1\) is trivial and the case \(n=2\)
is immediate from the definition of extreme point.
Assume the claim holds whenever \(n < k\),
and now assume \(n=k\).
put \(a = 1 - t_n\) and \(y = (p - t_nx_n)/a\). Clearly
\(y = \sum_{i=1}^{n-1} (t_i/a) x_i\) and \(y \in \co(K)\) since
\(\sum_{i=1}^{n-1} (t_i/a) = 1\).
Now \(p = ay + t_nx_n\). By the trivial case of \(n=2\)\
we have \(p = y = x_n\).
By symmetry or induction, we also have  \(p = x_i\) for \(1 \leq i < n\).


%%%%%%% 31
\begin{excopy}
Suppose that \(A_1,\ldots,A_n\) are convex sets in a vector space $X$.
Prove that every \(x \in \co(A_1 \cup \cdots \cup A_n)\) can be represented
in the form
\begin{equation*}
x = t_1 a_1 + \cdots + t_n a_n,
\end{equation*}
with \(a_i \in A_i\) and \(t_i \geq 0\) for all $i$, \(\sum t_i = 1\).
\end{excopy}

Let \(x = \sum_{j=1}^m = s_j u_j\) where \(u_j \in \cup_{k=1}^n A_n\) and
\(\sum_{j=1}^m s_j = 1\) and \(s_j \in [0,1]\) for all \(j\in\N_m\).
Clearly all such $x$ combinations form \(\co\left(\cup_{k=1}^n A_k\right)\).
Now we reorder \((s_j u_j)_{j=1}^m\) such that 
we have increasing sequence of indices satisfying
\begin{gather*}
p_0 = 0
 \qquad
 p_{k-1} \leq p_k \;\textnormal{for}\; (k \in \N_n)
 \qquad
 p_n = m
 \\
 \forall k\in\N_n\; p_{k-1} < j \leq p_k \;\Rightarrow\; u_j \in A_k.
\end{gather*}
Putting
\begin{equation*}
t_k = \sum_{j=p_{k-1}+1}^{p_k} s_j \qquad (k \in \N_n)
\end{equation*}
gives
\begin{equation*}
x
 = \sum_{j=1}^m = s_j u_j
 = \sum_{k=1}^n \sum_{j=p_{k-1}+1}^{p_k} s_j u_j
 = \sum_{k=1}^n t_k \sum_{j=p_{k-1}+1}^{p_k} (s_j/t_k) u_j
\end{equation*}
which shows the desired representation, since
\(a_k = \sum_{j=p_{k-1}+1}^{p_k} (s_j/t_k) u_j \in A_k\) for all \(k\in\N_n\)
by the convexity of \(A_k\).

%%%%%%% 32
\begin{excopy}
Let $X$ be an infinite-dimensional Banach space and let
\(S = \{x \in X: \|x\| = 1\}\) be the unit sphere of $X$. We want to cover $S$
with finitely many closed balls, none of which contains the origin of $X$.
Can this be done in \ich{a} every $X$, \ich{b} some $X$,  \ich{c} no $X$ ?
\end{excopy}

It can never be done.

First let's establish simple algebraic lemma about finite co-dimensions
within infinite-dimensional vector space.

\begin{llem} \label{llem:infiker}
Let $X$ be a infinite-dimensional vector space, and \(S_j \subset X\)
subspaces such that \(X/S_j\) are finite-dimensional vector spaces
for \(j\in\N_n\). If \(S = \cap_{j=1}^n S_j\) then \(X/S\) is finite-dimensional.
\end{llem}
\begin{proof}
let \(\pi_j : X \to X/S_j\) be the projection mappings (\(j\in\N_n\).
Define the map:
\begin{align*}
p: X &\to \prod_{j=1}^n (X/S_j) \\
p(x) &= (\pi_j(x))_{j=1}^n.
\end{align*}
Clearly \(S = \Ker p\)
 and
\begin{equation*}
\dim(\Image p) = \prod_{j=1}^n(\dim(X/S_j)) < \infty.
\end{equation*}
Since \(\dim X = \dim(\Ker p) + \dim(\Image p)\)
we must have \(\,\dim S = \infty\).
\end{proof}

By negation, let \(\{B_j: j\in\N_n\}\) be such covering set, 
where \(B_j = \overline{B(c_j;r_j)}\).
Clearly \(r_j < \|c_j\|\) for all \(j\in\N_n\).

By Corollary to Theorem~3.3 there exists \(\Lambda_j \in X^*\) 
such that 
\begin{align*}
\Lambda_j c_j &= \|c_j\| \\
\forall x\in X,\; |\Lambda_j x| &\leq \|x\|
\end{align*}
for all \(j\in\N_n\).  Put
\begin{equation*}
N = \cap_{j=1}^n \Ker \Lambda_j.
\end{equation*}
By the above local lemma~\ref{llem:infiker} \(\dim N = \infty\),
we actually need just \(\dim N > 0\).
Thus there exists some \(u \in N\) such that \(\|u\| = 1\).
We will show the contradiction \(u \notin \cup_{j=1}^n B_j \).

For each \(j\in\N_n\) we have
\begin{align*}
\|c_j - u\|
 &\geq |\Lambda_j(c_j - u)|
 = |\Lambda_j(c_j) - \Lambda(u)|
 = |\Lambda_j(c_j)| = \|c_j\| > r_j.
\end{align*}
Hence \(u \notin B_j\).

See also \cite{Fabian2001} Lemma~6.15.




%%%%%%% 33
\begin{excopy}
Let \(C(I)\) be the Banach space of all continuous complex functions on
the closed unit interval $I$, with the supremum norm.
Let \(M = C(I)^*\), the space of all complex Borel measures on $I$.
Give $M$ the weak\upstar-topology induced by \(C(I)\).

For each \(t \in I\), let \(e_t \in M\) be the ``evaluation functional''
defined by \(e_t f = f(t)\), and define \(\Lambda \in M\) by
\(\Lambda f = \int_o^1 f(s)\,ds\).
\begin{itemize}
\itemch{a}
Show that \(t \to e_t\) is a continuous map from $I$ into $M$ 
and that \(K = \{e_t: t\in I\}\) is a compact set in $M$.
\itemch{b} Show that \(\Lambda \in \overline{\co}(K)\).
\itemch{c} Find all \(\mu \in \overline{\co}(K)\).
\itemch{d} Let $X$ be the subspace of $M$ consisting of all finite linear 
combinations 
\begin{equation*}
c_0 \Lambda + c_1 e_{t_1} + \cdots + c_n e_{t_n}
\end{equation*}
with complex coefficients \(c_j\). Note that \(\co(K) \subset K\) and that 
\(X \cap \overline{\co}(K)\) is the closed convex hull of $K$ 
within $X$. Prove that \(\Lambda\) is an extreme point 
of \(X \cap \overline{\co}(K)\), even though \(\Lambda\) is not in $K$.
\end{itemize}
\end{excopy}

\begin{itemize}
\itemch{a}
A special case of Exercise~29.

\itemch{b}
Answered by \ich{c}

\itemch{c}
Let 
\begin{equation*}
F = \{\mu: \mu \geq 0 \;\wedge\; \|\mu\| = 1\}.
\end{equation*}
We calim that  \(\overline{\co}(K) = F\),
that is the set of all positive measures \(\mu\)
such that \(\|\mu\| = 1\) where
\begin{equation*}
\|\mu\|  = \sup 
  \bigl\{\int_I |f|\,d\mu: f\in C(I) \;\wedge\; \|f\|_\infty \leq 1\bigr\}.
\end{equation*}
Since \(\mu \geq 0\) we simply have
 \(F = \{\mu\in C(I)^*: \mu\geq 0 \wedge \mu(I)=1\}\).

By Banach-Alaoglu
\index{Alaoglu}
Theorem~3.15, $F$ is weak\upstar\ compact and it is easy to see
that $F$ is convex. We just need to show that \(\ext(F) = K\).

Assume for some 
\begin{equation*}
e_t = a_1\mu_1 + a_2 \mu_2 \qquad
  (a_1,a_2 \in [0,1],\; \mu_j \geq 0,\; \|\mu\| = 1).
\end{equation*}
Clearly \(a_1 \mu_1(\{t\}) + a_2 \mu_2(\{t\})\)
and so \(\mu_j(\{t\}) = 1\) for \(j=1,2\) and since \(\|\mu_j\| = 1\)
we have \(\mu_j(E) = 0\) for any measurable \(E \subset I \setminus \{t\}\).
Hence \(\mu_1 = \mu_2 = e_t\) which is exterme point of $K$.

Now assume by negation \(\nu \in \ext(F) \setminus K\).
Thus we have a partition \(I = A \disjunion B\) and \(0 < \nu(A), \nu(B) < 1\).
Put
\begin{equation*}
a = \nu(A) \qquad b = \nu(B)
\end{equation*}
and define
\begin{equation*}
\nu_s(E) 
=   \frac{(a+b)\nu(E \cap A)}{a}(1 - t)
  + \frac{(a+b)\nu(E \cap B)}{b}t  \qquad ( 0 \leq s \leq 1).
\end{equation*}
for each Borel measurable \(E\subset I\). Clearly
\begin{gather*}
\forall\in[0,1]\; \nu_s \in A. \\
\nu_0 \neq \nu \neq \nu_1.\\
s = b/(a+b) \;\Rightarrow\; \nu_s = \nu.
\end{gather*}
Hence \(\nu \notin \ext(F)\).
Our claim is finally proven by applying Krein-Milman
\index{Krein-Milman}
Theorem~3.23.

\itemch{d}
Enumerate the numbers of \(I\cap \Q\) as \((q_j)_{j\in\N}\).
Define
\begin{equation*}
\Lambda_n = \frac{1}{n}\sum_{j=1}^n e_t.
\end{equation*}
Clearly \(\Lambda_n \in \co(K)\) and 
\(\lim_{n\to\infty} = \Lambda\) in \(C(I)^*\).
Hence \(\Lambda \in X \cap \overline{\co}(K)\).
Now if
\begin{equation*}
\mu = c_0\Lambda + \sum_{j=1}^n c_j e_{t_j} \qquad
  (c_j \neq 0\;\wedge\; t_j < t_{j+1})
\end{equation*}
then
\begin{equation*}
\|\mu\| = \sum_{j=0}^n |c_j|
\end{equation*}
this is justified by picking arbitrary \(\epsilon>0\) 
picking intervals \(J_j\) such that \(t_j\in J_j\) for \(j\in\N_n\)
and 
\begin{equation*}
\sum_{j=1}^n m(J_j) < \epsilon/\max\{|c_j|:j\in\N_n\}
\end{equation*}
and forming a function 
\(f\in C(I)\) such that 
\begin{gather*}
\|f\|_\infty \leq 1 \\
\forall j\in\N_n,\; f(t_j) = \overline{c_j}/|c_j| \\
c_0 \neq 0 \;\Rightarrow\; f(x) = \overline{c_0}/|c_0|
\end{gather*}

Note that if \(T = c_0\Lambda + \sum_{j=1}^n c_je_{t_j}\)
and \(t_j < t_{j+1}\) then \(\T\| = \sum_{j=0}^n |c_j|\).

Now assume
\begin{align*}
\mu &= b_0\Lambda + \sum_{j=1}^n b_j e_{t_j} \\
\nu &= c_0\Lambda + \sum_{j=1}^n c_j e_{u_j} 
\end{align*}
and both
 \(\|\mu\| = \|\nu\| = 1\).
if \(\Lambda = \mu (1-t) + \nu t\)
then by looking at all Borel measurable sets
 \(E \subset I \setminus 
     \left(
     \left(\cup_{j=m} \{t_j\}\right)
     \cup
     \left(\cup_{j=n} \{u_j\}\right)
     \right)\)
it is easy to see that \(\Lambda(E) =  b_0\Lambda(E) + c_0 \Lambda(E)\)
Thus \(b_0 + c_0 = 1\) and \(b_j = c_j = 0\) for all \(j>0\).
Hence it is a trivial convex combination for representating \(\Lambda\)
which is exterme as desired.
\end{itemize}




%%%%%%%%%%%%%%%
\end{enumerate}
%%%%%%%%%%%%%%%
