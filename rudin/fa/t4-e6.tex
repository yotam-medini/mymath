\documentclass{book}

\usepackage{amsmath}
\usepackage{amssymb}
% \usepackage{eucal}
\usepackage{mathrsfs}

% \usepackage{fullpage}

\usepackage{geometry}
\geometry{a4paper, left=2cm, right=2cm, top=2cm, bottom=2cm, includeheadfoot}

\setlength{\parindent}{0pt}
\setlength{\parskip}{6pt}


% are we in pdftex ????
\ifx\pdfoutput\undefined % We're not running pdftex
\else
\RequirePackage[colorlinks,hyperindex,plainpages=false]{hyperref}
\def\pdfBorderAttrs{/Border [0 0 0] } % No border arround Links
\fi

% \usepackage{fancyheadings}
\usepackage{fancyhdr}
\usepackage{pifont}

\pagestyle{fancy}
% \addtolength{\headwidth}{\marginparsep}
% \addtolength{\headwidth}{\marginparwidth}
%  \addtolength{\textheight}{2pt}

\newcommand{\ineqjton}{\overset{1\leq i,j \leq n}{i \neq j}}
\newcommand{\srightmark}{\rightmark}
\newcommand{\sfbfpg}{\sffamily\bfseries{\thepage}}
  \newcommand{\symenvelop}{%
     {\nullfont\ }\relax\lower.2ex\hbox{\large\Pisymbol{pzd}{41}}}
% \renewcommand{\chaptermark}[1]{\markboth{\thechapter.\ #1}}

\iffalse
% \lhead[\fancyplain{}{{\sfbfpg}}]{\fancyplain{}\bfseries\srightmark}
\lhead[\fancyplain{}{{\sfbfpg}}]{\fancyplain{}\sl\srightmark}
% \rhead[\fancyplain{}\bfseries\leftmark]{\fancyplain{}{{\sfbfpg}}}
\rhead[\fancyplain{}\sl\leftmark]{\fancyplain{}{{\sfbfpg}}}
\lfoot{\today}
\cfoot{Yotam Medini \copyright}
  \newcommand{\symenvelop}{%
     {\nullfont a}\relax\lower.2ex\hbox{\large\Pisymbol{pzd}{41}}}
\rfoot{\symenvelop\ \texttt{yotam.medini@gmail.com}}

\renewcommand{\headrulewidth}{0.4pt}
\renewcommand{\footrulewidth}{0.4pt}
\fi

\setlength{\headheight}{16pt}
\fancyplain{plain}{%
 \fancyhf{}
 \fancyhead[LE,RO]{\fancyplain{}{{\sfbfpg}}}
 \fancyhead[RE,LO]{\sl\leftmark}
 \fancyfoot[L]{\today}
 \fancyfoot[C]{Yotam Medini \copyright}
 \fancyfoot[R]{\symenvelop\ \texttt{yotam.medini@gmail.com}}
 \renewcommand{\headrulewidth}{0.4pt}
 \renewcommand{\footrulewidth}{0.4pt}
}

% \usepackage{amstex}
% \usepackage{amsmath}
% \usepackage{amssymb}
\usepackage{amsthm}
\usepackage{bm}
\usepackage{makeidx}
\makeindex % enable

% 'Inspired' by:
%% This is file `uwamaths.sty',
%%%     author   = "Greg Gamble",
%%%     email     = "gregg@csee.uq.edu.au (Internet)",

\makeatletter
\def\DOTSB{\relax}
\def\dotcup{\DOTSB\mathop{\overset{\textstyle.}\cup}}
 \def\@avr#1{\vrule height #1ex width 0pt}
 \def\@dotbigcupD{\smash\bigcup\@avr{2.1}}
 \def\@dotbigcupT{\smash\bigcup\@avr{1.5}}
 \def\dotbigcupD{\DOTSB\mathop{\overset{\textstyle.}\@dotbigcupD%
                               \vphantom{\bigcup}}}

\def\dotbigcupT{\DOTSB\smash{\mathop{\overset{\textstyle.}\@dotbigcupT%
                              \vphantom{\bigcup}}}%
                       \vphantom{\bigcup}\@avr{2.0}}
\def\dotbigcup{\mathop{\mathchoice{\dotbigcupD}{\dotbigcupT}
                                  {\dotbigcupT}{\dotbigcupT}}}
\let\disjunion\dotcup
\let\Disjunion\dotbigcup
\makeatother


\newcommand{\half}{\ensuremath{\frac{1}{2}}}



\newcommand{\C}{\ensuremath{\mathbb{C}}} % The Complex set
\newcommand{\aded}{\ensuremath{\textrm{a.e.}}} % almost everyehere
\newcommand{\chhi}{\raise2pt\hbox{\ensuremath\chi}}           %raise the chi
\newcommand{\calA}{\ensuremath{\mathcal{A}}}
\newcommand{\calB}{\ensuremath{\mathcal{B}}}
\newcommand{\calE}{\ensuremath{\mathcal{E}}}
\newcommand{\calF}{\ensuremath{\mathcal{F}}}
\newcommand{\calG}{\ensuremath{\mathcal{G}}}
\newcommand{\calM}{\ensuremath{\mathcal{M}}}
\newcommand{\calR}{\ensuremath{\mathcal{R}}}
\newcommand{\eqdef}{\ensuremath{\stackrel{\mbox{\upshape\tiny def}}{=}}}
\newcommand{\frakB}{\ensuremath{\mathfrak{B}}}
\newcommand{\frakC}{\ensuremath{\mathfrak{C}}}
\newcommand{\frakF}{\ensuremath{\mathfrak{F}}}
\newcommand{\frakG}{\ensuremath{\mathfrak{G}}}
\newcommand{\frakI}{\ensuremath{\mathfrak{I}}}
\newcommand{\frakM}{\ensuremath{\mathfrak{M}}}
\newcommand{\scrA}{\ensuremath{\mathscr{A}}}
\newcommand{\scrB}{\ensuremath{\mathscr{B}}}
\newcommand{\scrD}{\ensuremath{\mathscr{D}}}
\newcommand{\scrF}{\ensuremath{\mathscr{F}}}
\newcommand{\scrN}{\ensuremath{\mathscr{N}}}
\newcommand{\scrP}{\ensuremath{\mathscr{P}}}
\newcommand{\scrQ}{\ensuremath{\mathscr{Q}}}
\newcommand{\scrR}{\ensuremath{\mathscr{R}}}
\newcommand{\scrT}{\ensuremath{\mathscr{T}}}
\newcommand{\Lp}[1]{\ensuremath{\mathbf{L}^{#1}}} % Lp space
\newcommand{\N}{\ensuremath{\mathbb{N}}} % The Natural Set
\newcommand{\bbP}{\ensuremath{\mathbb{P}}} % Some partially ordered set
\newcommand{\Q}{\ensuremath{\mathbb{Q}}} % The Rational set
\newcommand{\R}{\ensuremath{\mathbb{R}}} % The Real Set
\newcommand{\T}{\ensuremath{\mathbb{T}}} % The Thorus [-pi,\pi)
\newcommand{\Z}{\ensuremath{\mathbb{Z}}} % The Integer Set
\newcommand{\intR}{\int_{-\infty}^{\infty}} % Integral over the reals
\newcommand{\posthat}[1]{#1{\,\hat{}\,}}

% sequences
\newcommand{\seq}[2]{\ensuremath{#1_1,\ldots,#1_{#2}}}
\newcommand{\seqn}[1]{\seq{#1}{n}}
\newcommand{\seqan}{\seq{a}{n}}
\newcommand{\seqxn}{\seq{x}{n}}
\newcommand{\seqalphn}{\seq{\alpha}{n}}

\newcommand{\mset}[1]{\ensuremath{\{#1\}}}


%%%%%%%%%%%%
%% math op's
\newcommand{\Alt}{\mathop{\rm Alt}\nolimits}
\newcommand{\Ang}{\mathop{\rm Ang}\nolimits}
\newcommand{\Arg}{\mathop{\rm Arg}\nolimits}
\newcommand{\co}{\mathop{\rm co}\nolimits}
\newcommand{\conv}{\mathop{\rm conv}\nolimits}
\newcommand{\diam}{\mathop{\rm diam}\nolimits}
\newcommand{\dom}{\mathop{\rm dom}\nolimits}
% \newcommand{\dim}{\mathop{\rm dim}\nolimits}
% \newcommand{\esssup}{\mathop{\rm ess\ sup}\nolimits}
\DeclareMathOperator*{\esssup}{ess\,sup}
\newcommand{\ext}{\mathop{\rm ext}\nolimits}
\newcommand{\Id}{\mathop{\rm Id}\nolimits}
\newcommand{\Image}{\mathop{\rm Im}\nolimits}
\newcommand{\Ind}{\mathop{\rm Ind}\nolimits}
\newcommand{\Lip}{\mathop{\rm Lip}\nolimits}
\newcommand{\lip}{\mathop{\rm lip}\nolimits}
\newcommand{\percB}{
  \mathbin{\ooalign{$\hidewidth\%\hidewidth$\cr$\phantom{+}$}}}
\newcommand{\bres}[2]{\ensuremath{#1 \percB #2}}

\newcommand{\Ker}{\mathop{\rm Ker}\nolimits}
\newcommand{\rank}{\mathop{\rm rank}\nolimits}
\newcommand{\rng}{\mathop{\rm rng}\nolimits}
\newcommand{\Res}{\mathop{\rm Res}\nolimits}
\newcommand{\supp}{\mathop{\rm supp}\nolimits}
\newcommand{\vol}{\mathop{\rm vol}\nolimits}
\newcommand{\vspan}{\mathop{\rm span}\nolimits}

% I wish this was more standardized
\renewcommand{\Re}{\mathop{\bf Re}\nolimits}
\renewcommand{\Im}{\mathop{\bf Im}\nolimits}

\newcommand{\inter}[1]{\ensuremath{#1^{\circ}}}  % interior
\newcommand{\closure}[1]{\ensuremath{\overline{#1}}} % closure
\newcommand{\boundary}[1]{\ensuremath{\partial #1}} % closure


\newcommand{\ich}[1]{(\textit{#1})}
\newcommand{\itemch}[1]{\item[\ich{#1}]}
\newcommand{\itemdim}{\item[\(\diamond\)]}

% Special names
\newcommand{\Cech}{\u{C}ech}

\author{Yotam Medini}


%%%%%%%%%%%
%% Theorems
%%
\makeatletter
\@ifclassloaded{book}{
 \newtheorem{thm}{Theorem}[chapter]
 \newtheorem{cor}[thm]{Corollary}
 \newtheorem{lem}[thm]{Lemma}
 \newtheorem{llem}[thm]{Local Lemma}
 \newtheorem{lthm}[thm]{Local Theorem}
 % \newtheorem{quotecor}{Corollary}
 % \newtheorem{quotelem}{Lemma}[section]
 \newtheorem{quotethm}{Theorem}[chapter]
}{}
\makeatother
\newtheorem{Def}{Definition}

\newtheorem{manualtheoreminner}{Theorem}
\newenvironment{manualtheorem}[1]{%
  \renewcommand\themanualtheoreminner{#1}%
  \manualtheoreminner
}{\endmanualtheoreminner}

\newtheorem{manuallemmainner}{Lemma}
\newenvironment{manuallemma}[1]{%
  \renewcommand\themanuallemmainner{#1}%
  \manuallemmainner
}{\endmanuallemmainner}

\newcommand{\loclemma}{Lemma}


% \newcommand{\proofend}{\(\bullet\)}
% \newcommand{\proofend}{\hfill\(\blacksquare\)}
\newcommand{\proofend}{\hfill\(\Box\)}
\newenvironment{thmproof}
{\textbf{Proof.}}
{\proofend}

\newcommand{\chapterTypeout}[1]{\typeout{#1} \chapter{#1}}
\newcommand{\sectionTypeout}[1]{\typeout{#1} \section{#1}}

% abbreviations, ensuremath
\newcommand{\fx}{\ensuremath{f(x)}}
\newcommand{\gx}{\ensuremath{g(x)}}
\newcommand{\lrangle}[1]{\ensuremath{\left\langle #1 \right\rangle}}
\newcommand{\lrbangle}[1]{\ensuremath{\left\langle #1 \right\rangle}}
\newcommand{\M}{\ensuremath{\mathfrak{M}}}
\newcommand{\mldots}{\ensuremath{\ldots}}
\newcommand{\salgebra}{\(\sigma\)-algebra}
\newcommand{\swedge}{\;\wedge\;}
\newcommand{\wlogy}{without loss of generality}
\newcommand{\Wlogy}{Without loss of generality}
\newcommand{\twopii}{\ensuremath{2\pi i}}
\newcommand{\dtwopii}{\ensuremath{\frac{1}{\twopii}}}

% https://tex.stackexchange.com/
% questions/22252/how-to-typeset-function-restrictions
\newcommand\restr[2]{\ensuremath{% we make the whole thing an ordinary symbol
  \left.\kern-\nulldelimiterspace % automatically resize the bar with \right
  #1 % the function
  \vphantom{\big|} % pretend it's a little taller at normal size
  \right|_{#2} % this is the delimiter
  }}

\newenvironment{excopyOLD}
{\item\begin{minipage}[t]{.8\textwidth}\footnotesize}
{\smallskip\hrule\end{minipage}}

\newenvironment{excopy}
{\item % \relax
 \begin{list}{}{
 \setlength{\topsep}{0pt}
 \setlength{\partopsep}{0pt}
 \setlength{\itemsep}{0pt}
 \setlength{\parsep}{0pt}
 \setlength{\leftmargin}{0pt}
 \setlength{\rightmargin}{20pt}
 \setlength{\listparindent}{0pt}
 \setlength{\itemindent}{0pt}
 % \setlength{\labelsep}{0pt}
 \setlength{\labelwidth}{0pt}
 \footnotesize
 }
 \item
}
{\par
 % {\nullfont 0}
 \hrulefill
 \end{list}
}

\makeatletter
\@ifclassloaded{book}{
 \newcommand{\isbook}{Yes}
}{
 \newcommand{\isbook}{No}
}
\@ifclassloaded{article}{
 \newcommand{\isarticle}{Yes}
}{
 \newcommand{\isarticle}{No}
}

\makeatother

\newcommand{\upstar}{\raise.5ex\hbox{\(*\)}}
\newcommand{\unfinished}{\par\textbf{Unfinished !!!!!!!!!!!!!}\par}

\title{Weak-To-Weak}
\author{Yotam Medini}

\begin{document}

\maketitle


Here is a basic topological result.

See \cite{Megginson1998} Corollary~2.4.5
\begin{llem} \label{llem:cont:bymaps}
Let \((W,\tau)\) be a topological space and $X$ be a set topologized
by a a family \calF\ of mappings \(f: X \to Y_f\).
A mapping \(g:W\to X\) is continuous iff \(f\circ g\) is continuous
for all \(f\in\calF\).
\end{llem}
\begin{proof}
If $g$ is continuous then \(f\circ g\) are continuous simply
by composition of continuous mappings.

Conversely, we assume \(f\circ g\) is continuous for all \(f\in\calF\).
It will be sufficient to show that \(g^{-1}(B)\) is open in $W$
for any set $B$ in a basis for the topology of $X$.

Let $G$ be a family of inverse images of open sets \(f^{-1}(V)\)
where \(f\in\calF\) and $V$ is open in \(Y_f\).
Following  the discussion of Section~3.8
let the $S$ be the family that consists of finite intersections
of sets from $G$. It is a subbasis for 
the topology of $X$ which consists of unions of sets from $S$.

Pick an arbitrary set
'\(B = \cap_{j\in J} f_j^{-1}(V_j)\) from the subbasis $S$ where \(V_j\) is open 
in \(Y_{f_j}\) and $J$ is a finite set of \calF-indices.
% (The finiteness $J$ is actually not needed for our arguments).
Put \(U_j = f_j^{-1}(V_j) \subset X\).
By the assumption each \((f_j\circ g)^{-1}(V_j) = g^{-1}(U_j)\) 
is open in $W$.
It is easy to see that
\begin{equation*}
g^{-1}(B) = g^{-1}\left(\cap_{j\in J} U_j\right)
  = \cap_{j\in J} g^{-1}(U_j).
\end{equation*}
Hence \(g^{-1}(B) \in \tau\) and so $g$ is continuous.
\end{proof}


The relation between continuity and boundness
Will now be shown form normed spaces.
\begin{llem} \label{llem:op:cont:bounded}
Let \(T:X \to Y\) be a linear mapping between normed spaces.
Then $T$ is continuous iff \(T(E)\) is bounded in $Y$ whenever $E$
is bounded in $X$.
\end{llem}
\begin{proof}
Assume $T$ is continuous.
Let \(E\subset X\) be bounded subset.
Pick an arbitrary neighborhood \(0\in V \subset Y\)
By assumption there exists a neighborhood \(0 \in U \subset X\)
such that \(T(U) \subset V\) and there exists some \(n < 0\) such that
\(E \subset nU\). Hence 
\(T(E) \subset nT(U) \subset nV\). Hence the desired boundness condition 
holds for $T$. (Note that this direction did not use any norm).

Conversely, assume that $T$ maps bounded sets to bounded sets.
Then \(T(\{x\in X: \|x\| \leq 1\})\) is bounded in $Y$.
Thus there exists some \(M < \infty\) such that
\(\|Tx\| < M\) whenever \(\|x\| \leq 1\) where \(x \in X\).
Pick a neighborhood \(0\in V \subset Y\).
We can find some \(r > 0\) such that \(\{y\in Y: \|y\| \leq r\} \subset V\).
Put the open set \(W = \{x \in X: \|x\| < r/M\}\) and clearly
\(T(W) \subset U\). Thus $T$ is continuous at the origin.
By linearity $T$ is continuous.
\end{proof}


Here is a characterization of linear operator's boundness
(\cite{Megginson1998} Proposition~2.5.10).
\begin{llem}
Assume \(T:X \to Y\) is a linear operator between normed spaces.
Then $T$ is bounded  iff \(y^* \circ T \in X^*\) for all \(y^*\in Y^*\).
\end{llem}
\begin{proof}
If a functional is weakly-continuous then obviously it is
continuous in the norm topology. Theorem~3.10  gives the converse
that a any continuous functional is also weakly-continuous.
By Local-Lemma~\ref{llem:cont:bymaps} $T$ is weak-to-weak continuous.

The mapping $T$ is normed-continuous iff it \(T(E)\) is norm-bounded
in $Y$ whenever $E$ is normed bounded in $X$.
By Theorem~3.18 this is equivalent 
to the condition that \(T(E)\) is weakly-bounded
whenever $E$ weakly-bounded.
\end{proof}

The relation between operator continuity and its weak-continuity
is shown in the next result
\begin{llem} \label{llem:weak:to:weak}
Let \(T:X \to Y\) be a linear map between normed spaces.
Then $T$ is continuous iff $T$ is weak-to-weak continuous.
\end{llem}
\begin{proof}
By Local Leamm~\ref{llem:op:cont:bounded}
$T$ is continuous iff it map bounded sets to bounded sets.
By Theorem~3.18 it is equivalent to
require that $T$ maps weakly bounded sets to weakly bounded sets.
\end{proof}

Here a are simple lemmas regarding bounded and weak boundness of a set.
\begin{llem} \label{llem:bound:weakbound}
If $E$ is bounded then it is weakly bounded.
\end{llem}
\begin{proof}
Trivial since every weak neighborhood is a neighborhood
in the original topology.
\end{proof}


\begin{llem} \label{llem:bound:funcbound}
If  $E$ is a bounded subset of  a topological vector space $X$,
then \(\Lambda (E)\) is bounded in \C\ for every \(\Lambda \in X^*\).
\end{llem}
\begin{proof}
By  negation assume \(\Lambda(E)\) is unbounded.
Thus we can find a sequence \(x_n\in E\) such that \(|\Lambda x_n| \geq n\).
Define the open $0$ neighborhood \(V = \{x \in X: |\Lambda x\| < 1\}\).
Clearly \(x_n \notin nV\) for all $n$. Hence \(E \not\subset nV\) for all $n$
contradicting the assumption.
\end{proof}

We have a ``weak equivalnce''.
\begin{llem} \label{llem:weakbound:funcbound}
A subset  $E$ is weakly bounded in a topological vector space $X$,
iff \(\Lambda (E)\) is bounded in \C\ for every \(\Lambda \in X^*\).
\end{llem}
\begin{proof}
This was shown in the discussion of Section~3.11.
\end{proof}

With norm we have an equivalnce.
\begin{llem} \label{llem:bound:norm:funcbound}
Let  \(E \subset X\) where $X$ is a normed vector space.
Then $E$ is bounded iff \(\Lambda (E)\) is bounded in \C\ for
every \(\Lambda \in X^*\).
\end{llem}
\begin{proof}
If $E$ is bounded ,then it is weakly bounded and by
Local Lemma~\ref{llem:op:cont:bounded}
\(\Lambda (E)\) is bounded in \C\ for
every \(\Lambda \in X^*\).
Conversely, assume \(\Lambda (E)\) is bounded in \C\ for
every \(\Lambda \in X^*\). 
By Local Lemma~\ref{llem:weakbound:funcbound} again,
$E$ is weakly bounbded. By Theorem~3.18 it is also originally bounded.
\end{proof}

% Up to here - can be put in Chapter 3
%%%%%%%%%%%%%%%%%%%%%%%%%%%%%%%%
% To be put in Chapter 4

The following lemma shows that in boundness of operator
can be checked by looking at weak topologies, when the spaces are normed
(See \cite{Megginson1998} Proposition~2.5.11).
\begin{llem} \label{llem:op:normbd:weakbd}
Let \(T:X\to Y\) be a linear operator between normed spaces.
Then $T$ is bounded iff it is weak-to-weak bounded.
\end{llem}
\begin{proof}
Let \(B_X = \{x\in X: \|x\|\leq 1\}\).
By definitions, \(\Lambda\) is bounded if \(T(B_X)\) is bounded in $Y$. 
By Local Lemma~\ref{llem:bound:norm:funcbound} 
This is equivalent to  \((\Lambda T)(B_X)\) be bounded in \C\ for
all \(\Lambda \in Y^*\).
Since all these \(\Lambda T \in X^*\) for all \(\Lambda \in Y^*\)
the last condition means, that $T$ is bounded
iff  \((\Lambda T)(B_X)\) are continuous mappings \(X\to\C\)
for all \(\Lambda \in Y*\) considering the weak topology of $X$.

In turn, by Local Lemma~\ref{llem:cont:bymaps} this is equivalent 
to $T$ being continuous mapping 
from $X$ with its weak to $Y$ with its weak topology.
\end{proof}

\begin{center}
Now Back To the 2nd part to be inserted in the Exercises!!
\end{center}

Conversely, assume $S$ is 
continuous linear mapping of \((Y^*, \tau)\) into \((X^*, \sigma)\).
Let \(\psi_X\) and \(\psi_Y\) the isomorphisms of $X$ and $Y$
to their second duals.
Pick \(x\in X\). By composition \(\psi_X \circ S\) is weakly\upstar\ continuous
functional on~\(Y^*\). Hence \(\psi_X \circ S \in \psi_Y(Y)\).
Define
\begin{equation*}
Tx = \psi_Y^{-1}(\psi_X \circ S) \in Y.
\end{equation*}
The linearity of $T$ is clear. By definition it is weak-to-weak continuous.
By local lemma~\ref{???} $T$ is also normed-continuous.
\end{document}
