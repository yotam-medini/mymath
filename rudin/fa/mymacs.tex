\usepackage{amsmath}
\usepackage{amssymb}
% \usepackage{eucal}
\usepackage{mathrsfs}

% \usepackage{fullpage}

\usepackage{geometry}
\geometry{left=1in, right=1in, top=1in, bottom=1in}

\setlength{\parindent}{0pt}
\setlength{\parskip}{6pt}


% are we in pdftex ????
\ifx\pdfoutput\undefined % We're not running pdftex
\else
\RequirePackage[colorlinks,hyperindex,plainpages=false]{hyperref}
\def\pdfBorderAttrs{/Border [0 0 0] } % No border arround Links
\fi

% \usepackage{fancyheadings}
\usepackage{fancyhdr}
\usepackage{pifont}

\pagestyle{fancy}
% \addtolength{\headwidth}{\marginparsep}
% \addtolength{\headwidth}{\marginparwidth}
%  \addtolength{\textheight}{2pt}

\newcommand{\ineqjton}{\overset{1\leq i,j \leq n}{i \neq j}}
\newcommand{\srightmark}{\rightmark}
\newcommand{\sfbfpg}{\sffamily\bfseries{\thepage}}
  \newcommand{\symenvelop}{%
     {\nullfont\ }\relax\lower.2ex\hbox{\large\Pisymbol{pzd}{41}}}
% \renewcommand{\chaptermark}[1]{\markboth{\thechapter.\ #1}}

\iffalse
% \lhead[\fancyplain{}{{\sfbfpg}}]{\fancyplain{}\bfseries\srightmark}
\lhead[\fancyplain{}{{\sfbfpg}}]{\fancyplain{}\sl\srightmark}
% \rhead[\fancyplain{}\bfseries\leftmark]{\fancyplain{}{{\sfbfpg}}}
\rhead[\fancyplain{}\sl\leftmark]{\fancyplain{}{{\sfbfpg}}}
\lfoot{\today}
\cfoot{Yotam Medini \copyright}
  \newcommand{\symenvelop}{%
     {\nullfont a}\relax\lower.2ex\hbox{\large\Pisymbol{pzd}{41}}}
\rfoot{\symenvelop\ \texttt{yotam.medini@gmail.com}}

\renewcommand{\headrulewidth}{0.4pt}
\renewcommand{\footrulewidth}{0.4pt}
\fi

\fancyplain{plain}{%
 \fancyhf{}
 \fancyhead[LE,RO]{\fancyplain{}{{\sfbfpg}}}
 \fancyhead[RE,LO]{\sl\leftmark}
 \fancyfoot[L]{\today}
 \fancyfoot[C]{Yotam Medini \copyright}
 \fancyfoot[R]{\symenvelop\ \texttt{yotam.medini@gmail.com}}
 \renewcommand{\headrulewidth}{0.4pt}
 \renewcommand{\footrulewidth}{0.4pt}
}

% \usepackage{amstex}
% \usepackage{amsmath}
% \usepackage{amssymb}
\usepackage{amsthm}
\usepackage{bm}
\usepackage{makeidx}
\makeindex % enable

% 'Inspired' by:
%% This is file `uwamaths.sty',
%%%     author   = "Greg Gamble",
%%%     email     = "gregg@csee.uq.edu.au (Internet)",

\makeatletter
\def\DOTSB{\relax}
\def\dotcup{\DOTSB\mathop{\overset{\textstyle.}\cup}}
 \def\@avr#1{\vrule height #1ex width 0pt}
 \def\@dotbigcupD{\smash\bigcup\@avr{2.1}}
 \def\@dotbigcupT{\smash\bigcup\@avr{1.5}}
 \def\dotbigcupD{\DOTSB\mathop{\overset{\textstyle.}\@dotbigcupD%
                               \vphantom{\bigcup}}}

\def\dotbigcupT{\DOTSB\smash{\mathop{\overset{\textstyle.}\@dotbigcupT%
                              \vphantom{\bigcup}}}%
                       \vphantom{\bigcup}\@avr{2.0}}
\def\dotbigcup{\mathop{\mathchoice{\dotbigcupD}{\dotbigcupT}
                                  {\dotbigcupT}{\dotbigcupT}}}
\let\disjunion\dotcup
\let\Disjunion\dotbigcup
\makeatother


\newcommand{\half}{\ensuremath{\frac{1}{2}}}



\newcommand{\C}{\ensuremath{\mathbb{C}}} % The Complex set
\newcommand{\aded}{\ensuremath{\textrm{a.e.}}} % almost everyehere
\newcommand{\chhi}{\raise2pt\hbox{\ensuremath\chi}}           %raise the chi
\newcommand{\calB}{\ensuremath{\mathcal{B}}}
\newcommand{\calE}{\ensuremath{\mathcal{E}}}
\newcommand{\calF}{\ensuremath{\mathcal{F}}}
\newcommand{\calG}{\ensuremath{\mathcal{G}}}
\newcommand{\calM}{\ensuremath{\mathcal{M}}}
\newcommand{\calR}{\ensuremath{\mathcal{R}}}
\newcommand{\eqdef}{\ensuremath{\stackrel{\mbox{\upshape\tiny def}}{=}}}
\newcommand{\frakB}{\ensuremath{\mathfrak{B}}}
\newcommand{\frakC}{\ensuremath{\mathfrak{C}}}
\newcommand{\frakF}{\ensuremath{\mathfrak{F}}}
\newcommand{\frakG}{\ensuremath{\mathfrak{G}}}
\newcommand{\frakI}{\ensuremath{\mathfrak{I}}}
\newcommand{\frakM}{\ensuremath{\mathfrak{M}}}
\newcommand{\scrB}{\ensuremath{\mathscr{B}}}
\newcommand{\scrD}{\ensuremath{\mathscr{D}}}
\newcommand{\scrN}{\ensuremath{\mathscr{N}}}
\newcommand{\scrP}{\ensuremath{\mathscr{P}}}
\newcommand{\scrQ}{\ensuremath{\mathscr{Q}}}
\newcommand{\scrR}{\ensuremath{\mathscr{R}}}
\newcommand{\Lp}[1]{\ensuremath{\mathbf{L}^{#1}}} % Lp space
\newcommand{\N}{\ensuremath{\mathbb{N}}} % The Natural Set
\newcommand{\Q}{\ensuremath{\mathbb{Q}}} % The Rational set
\newcommand{\R}{\ensuremath{\mathbb{R}}} % The Real Set
\newcommand{\T}{\ensuremath{\mathbb{T}}} % The Thorus [-pi,\pi)
\newcommand{\Z}{\ensuremath{\mathbb{Z}}} % The Integer Set
\newcommand{\intR}{\int_{-\infty}^{\infty}} % Integral over the reals
\newcommand{\posthat}[1]{#1{\,\hat{}\,}}

% sequences
\newcommand{\seq}[2]{\ensuremath{#1_1,\ldots,#1_{#2}}}
\newcommand{\seqn}[1]{\seq{#1}{n}}
\newcommand{\seqan}{\seq{a}{n}}
\newcommand{\seqxn}{\seq{x}{n}}
\newcommand{\seqalphn}{\seq{\alpha}{n}}

\newcommand{\mset}[1]{\ensuremath{\{#1\}}}


%%%%%%%%%%%%
%% math op's
\newcommand{\Ang}{\mathop{\rm Ang}\nolimits}
\newcommand{\Arg}{\mathop{\rm Arg}\nolimits}
\newcommand{\co}{\mathop{\rm co}\nolimits}
\newcommand{\conv}{\mathop{\rm conv}\nolimits}
\newcommand{\diam}{\mathop{\rm diam}\nolimits}
% \newcommand{\dim}{\mathop{\rm dim}\nolimits}
% \newcommand{\esssup}{\mathop{\rm ess\ sup}\nolimits}
\DeclareMathOperator*{\esssup}{ess\,sup}
\newcommand{\ext}{\mathop{\rm ext}\nolimits}
\newcommand{\Id}{\mathop{\rm Id}\nolimits}
\newcommand{\Image}{\mathop{\rm Im}\nolimits}
\newcommand{\Ind}{\mathop{\rm Ind}\nolimits}
\newcommand{\Lip}{\mathop{\rm Lip}\nolimits}
\newcommand{\lip}{\mathop{\rm lip}\nolimits}
\newcommand{\Ker}{\mathop{\rm Ker}\nolimits}
\newcommand{\rank}{\mathop{\rm rank}\nolimits}
\newcommand{\Res}{\mathop{\rm Res}\nolimits}
\newcommand{\supp}{\mathop{\rm supp}\nolimits}
\newcommand{\vol}{\mathop{\rm vol}\nolimits}
\newcommand{\vspan}{\mathop{\rm span}\nolimits}

% I wish this was more standardized
\renewcommand{\Re}{\mathop{\bf Re}\nolimits}
\renewcommand{\Im}{\mathop{\bf Im}\nolimits}

\newcommand{\inter}[1]{\ensuremath{#1^{\circ}}}  % interior
\newcommand{\closure}[1]{\ensuremath{\overline{#1}}} % closure
\newcommand{\boundary}[1]{\ensuremath{\partial #1}} % closure


\newcommand{\ich}[1]{(\textit{#1})}
\newcommand{\itemch}[1]{\item[\ich{#1}]}
\newcommand{\itemdim}{\item[\(\diamond\)]}

% Special names
\newcommand{\Cech}{\u{C}ech}

\author{Yotam Medini}


%%%%%%%%%%%
%% Theorems
%%
\makeatletter
\@ifclassloaded{book}{
 \newtheorem{thm}{Theorem}[chapter]
 \newtheorem{cor}[thm]{Corollary}
 \newtheorem{lem}[thm]{Lemma}
 \newtheorem{llem}[thm]{Local Lemma}
 \newtheorem{lthm}[thm]{Local Theorem}
 % \newtheorem{quotecor}{Corollary}
 % \newtheorem{quotelem}{Lemma}[section]
 \newtheorem{quotethm}{Theorem}[chapter]
}{}
\makeatother
\newtheorem{Def}{Definition}


\newcommand{\loclemma}{Lemma}


% \newcommand{\proofend}{\(\bullet\)}
% \newcommand{\proofend}{\hfill\(\blacksquare\)}
\newcommand{\proofend}{\hfill\(\Box\)}
\newenvironment{thmproof}
{\textbf{Proof.}}
{\proofend}

\newcommand{\chapterTypeout}[1]{\typeout{#1} \chapter{#1}}
\newcommand{\sectionTypeout}[1]{\typeout{#1} \section{#1}}

% abbreviations, ensuremath
\newcommand{\fx}{\ensuremath{f(x)}}
\newcommand{\gx}{\ensuremath{g(x)}}
\newcommand{\lrangle}[1]{\ensuremath{\langle #1 \rangle}}
\newcommand{\lrbangle}[1]{\ensuremath{\left\langle #1 \right\rangle}}
\newcommand{\M}{\ensuremath{\mathfrak{M}}}
\newcommand{\mldots}{\ensuremath{\ldots}}
\newcommand{\salgebra}{\(\sigma\)-algebra}
\newcommand{\swedge}{\;\wedge\;}
\newcommand{\wlogy}{without loss of generality}
\newcommand{\Wlogy}{Without loss of generality}
\newcommand{\twopii}{\ensuremath{2\pi i}}
\newcommand{\dtwopii}{\ensuremath{\frac{1}{\twopii}}}

% https://tex.stackexchange.com/
% questions/22252/how-to-typeset-function-restrictions
\newcommand\restr[2]{\ensuremath{% we make the whole thing an ordinary symbol
  \left.\kern-\nulldelimiterspace % automatically resize the bar with \right
  #1 % the function
  \vphantom{\big|} % pretend it's a little taller at normal size
  \right|_{#2} % this is the delimiter
  }}

\newenvironment{excopyOLD}
{\item\begin{minipage}[t]{.8\textwidth}\footnotesize}
{\smallskip\hrule\end{minipage}}

\newenvironment{excopy}
{\item % \relax
 \begin{list}{}{
 \setlength{\topsep}{0pt}
 \setlength{\partopsep}{0pt}
 \setlength{\itemsep}{0pt}
 \setlength{\parsep}{0pt}
 \setlength{\leftmargin}{0pt}
 \setlength{\rightmargin}{20pt}
 \setlength{\listparindent}{0pt}
 \setlength{\itemindent}{0pt}
 % \setlength{\labelsep}{0pt}
 \setlength{\labelwidth}{0pt}
 \footnotesize
 }
 \item
}
{\par
 % {\nullfont 0}
 \hrulefill
 \end{list}
}
