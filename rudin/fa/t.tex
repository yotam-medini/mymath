\documentclass[11pt,twoside,a4paper]{book}

\usepackage{amsmath}
\usepackage{amssymb}
% \usepackage[mathcal]{euscript}
\usepackage{mathrsfs}

\usepackage{geometry}
\geometry{left=1in, right=1in, top=1in, bottom=1in}
\setlength{\parindent}{0pt}
\setlength{\parskip}{6pt}

% \usepackage{fancyheadings}
\usepackage{fancyhdr}
\usepackage{pifont}

% \pagestyle{fancy}
% \addtolength{\headwidth}{\marginparsep}
% \addtolength{\headwidth}{\marginparwidth}
%  \addtolength{\textheight}{2pt}
\newcommand{\srightmark}{\rightmark}
\newcommand{\sfbfpg}{\sffamily\bfseries{\thepage}}
  \newcommand{\symenvelop}{%
     {\nullfont a}\relax\lower.2ex\hbox{\large\Pisymbol{pzd}{41}}}
% \renewcommand{\chaptermark}[1]{\markboth{\thechapter.\ #1}}

\fancyplain{plain}{%
 \fancyhf{}
 \fancyhead[LE,RO]{\fancyplain{}{{\sfbfpg}}}
 \fancyhead[RE,LO]{\sl\leftmark}
 \fancyfoot[L]{\today}
 \fancyfoot[C]{Yotam Medini \copyright}
 \fancyfoot[R]{\symenvelop\ \texttt{yotam\_medini@fastmail.fm}}
 \renewcommand{\headrulewidth}{0.4pt}
 \renewcommand{\footrulewidth}{0.4pt}
}

% \usepackage{amstex}
\usepackage{amsmath}
\usepackage{amssymb}
\usepackage{amsthm}
\usepackage{bm}
\usepackage{makeidx}
\makeindex % enable

% 'Inspired' by:
%% This is file `uwamaths.sty',
%%%     author   = "Greg Gamble",
%%%     email     = "gregg@csee.uq.edu.au (Internet)",

\makeatletter
\def\DOTSB{\relax}
\def\dotcup{\DOTSB\mathop{\overset{\textstyle.}\cup}}
 \def\@avr#1{\vrule height #1ex width 0pt}
 \def\@dotbigcupD{\smash\bigcup\@avr{2.1}}
 \def\@dotbigcupT{\smash\bigcup\@avr{1.5}}
 \def\dotbigcupD{\DOTSB\mathop{\overset{\textstyle.}\@dotbigcupD%
                               \vphantom{\bigcup}}}

\def\dotbigcupT{\DOTSB\smash{\mathop{\overset{\textstyle.}\@dotbigcupT%
                              \vphantom{\bigcup}}}%
                       \vphantom{\bigcup}\@avr{2.0}}
\def\dotbigcup{\mathop{\mathchoice{\dotbigcupD}{\dotbigcupT}
                                  {\dotbigcupT}{\dotbigcupT}}}
\let\disjunion\dotcup
\let\Disjunion\dotbigcup
\makeatother


\newcommand{\C}{\mathbb{C}} % The Complex set
\newcommand{\calB}{\ensuremath{\mathcal{B}}}
\newcommand{\calG}{\ensuremath{\mathcal{G}}}
\newcommand{\scrD}{\ensuremath{\mathscr{D}}}
\newcommand{\scrP}{\ensuremath{\mathscr{P}}}
\newcommand{\scrQ}{\ensuremath{\mathscr{Q}}}
\newcommand{\Lp}[1]{\ensuremath{\mathbf{L}^{#1}}} % Lp space
\newcommand{\N}{\mathbb{N}} % The Natural Set
\newcommand{\Q}{\ensuremath{\mathbb{Q}}} % The Rational set
\newcommand{\R}{\ensuremath{\mathbb{R}}} % The Real Set
\newcommand{\Z}{\ensuremath{\mathbb{Z}}} % The Integer Set
\newcommand{\intR}{\int_{-\infty}^{\infty}} % Integral over the reals

% sequences
\newcommand{\seq}[2]{\ensuremath{#1_1,\ldots,#1_{#2}}}
\newcommand{\seqn}[1]{\seq{#1}{n}}
\newcommand{\seqan}{\seq{a}{n}}
\newcommand{\seqxn}{\seq{x}{n}}
\newcommand{\seqalphn}{\seq{\alpha}{n}}

\newcommand{\ich}[1]{(\textit{#1})}
\newcommand{\itemch}[1]{\item[\ich{#1}]}


\title{Notes and Solutions to Exercises \\
          from \\
       Real and Complex Analysis / Walter Rudin}
\author{Yotam Medini}


%%%%%%%%%%%
%% Theorems
%%
\newtheorem{thm}{Theorem}[chapter]
\newtheorem{cor}[thm]{Corollary}
\newtheorem{Def}{Definition}
\newtheorem{lem}[thm]{Lemma}
\newtheorem{llem}[thm]{Local Lemma}
\newtheorem{lthm}[thm]{Local Theorem}

% \newtheorem{quotecor}{Corollary}
% \newtheorem{quotelem}{Lemma}[section]
\newtheorem{quotethm}{Theorem}[chapter]


% \newcommand{\proofend}{\(\bullet\)}
\newcommand{\proofend}{\hfill\(\blacksquare\)}
\newenvironment{thmproof}
{\textbf{Proof.}}
{\proofend}

\newcommand{\chapterTypeout}[1]{\typeout{#1} \chapter{#1}}
\newcommand{\sectionTypeout}[1]{\typeout{#1} \section{#1}}

% abbreviations, ensuremath
\newcommand{\eqdef}{\ensuremath{\stackrel{\mbox{\upshape\tiny def}}{=}}}
\newcommand{\fx}{\ensuremath{f(x)}}
\newcommand{\gx}{\ensuremath{g(x)}}
\newcommand{\lrangle}[1]{\ensuremath{\langle #1 \rangle}}
\newcommand{\calD}{\ensuremath{\mathcal{D}}}
\newcommand{\frakD}{\ensuremath{\mathfrak{D}}}
\newcommand{\M}{\ensuremath{\mathfrak{M}}}
\newcommand{\mldots}{\ensuremath{\ldots}}
\newcommand{\salgebra}{\(\sigma\)-algebra}
\newcommand{\wlogy}{without loss of generality}
\newcommand{\Wlogy}{Without loss of generality}

%%%%%%%%%%%%
%% math op's
%%
\newcommand{\hull}{\mathop{\rm hull}\nolimits}
\newcommand{\id}{\mathop{\rm id}\nolimits}
\newcommand{\Int}{\mathop{\rm Int}\nolimits}
\newcommand{\inter}[1]{\ensuremath{#1^{\circ}}}
% \newcommand{\inter}[1]{\ensuremath{\textrm{int{#1}^\circ}}
\def\Lip{\mathop{\rm Lip}\nolimits}
\def\lip{\mathop{\rm lip}\nolimits}
\def\Ker{\mathop{\rm Ker}\nolimits}
\def\ker{\mathop{\rm ker}\nolimits}
\def\Re{\mathop{\rm Re}\nolimits}
\def\Im{\mathop{\rm Im}\nolimits}
\def\supp{\mathop{\rm supp}\nolimits}


\newenvironment{excopy}
{\item\begin{minipage}[t]{.8\textwidth}\footnotesize}
{\smallskip\hrule\end{minipage}}


%%%%%%%%%%%%%%%%%%%%%%%%%%%%%%%%%%%%%%%%%%%%%%%%%%%%%%%%%%%%%%%%%%%%%%%%
%%%%%%%%%%%%%%%%%%%%%%%%%%%%%%%%%%%%%%%%%%%%%%%%%%%%%%%%%%%%%%%%%%%%%%%%
%%%%%%%%%%%%%%%%%%%%%%%%%%%%%%%%%%%%%%%%%%%%%%%%%%%%%%%%%%%%%%%%%%%%%%%%
\begin{document}

  Assume again \(1\leq p < q < \infty\). 
  Put \(q' = q/q-1\) the exponent conjugate.
  We want to find scalars \(\alpha\) and \(\beta\) such that if
  \begin{equation*}
    g_n(x) = \left\{\begin{array}{ll}
                     n & \qquad x \in [0,n^\alpha] \\
                     0 & \qquad x \in (n^\alpha, 1]
                    \end{array}\right.
  \end{equation*}
  and 
  \begin{equation*}
    \psi(x) = \left\{\begin{array}{ll}
                      x^\beta & \qquad x\in (0,1] \\
                      0       & \qquad x = 0
                    \end{array}\right.
  \end{equation*}
  then
  \begin{eqnarray}
   \forall f\in L^q([0,1])\qquad 
   \lim_{n\to \infty} \int_{[0,1]} fg_n\,dm &=& 0
                         \label{eq:ex2.4b:i} \\
   \psi &\in&  L^p([0,1]) \setminus L^p([0,1])
                         \label{eq:ex2.4b:psi} \\
   \lim_{n\to \infty} \int_{[0,1]} \psi g_n\,dm &>& 0
                         \label{eq:ex2.4b:iii}
  \end{eqnarray}

  To ensure \eqref{eq:ex2.4b:i}, we use H\"older inequality
  \begin{equation*}
  \left|\int_{[0,1]} fg_n\,dm \right| \leq \|f\|_q \|g_n\|_{q'}
  \end{equation*}
  and require that
  \(\lim_{n\to\infty} \|g_n\|_{q'}^{q'} = 0.\)
  Computing
  \begin{equation*}
   \|g_n\|_{q'}^{q'} = \int_0^{n^\alpha} n^{q'}\,dm 
   = n^\alpha \cdot n^{q'} = n^{\alpha + q/(q-1)}.
  \end{equation*}
  Thus 
  \begin{equation} \label{eq:ex2.4b:alpha}
  \alpha < -q/(q-1)
  \end{equation}
  must hold.

  To ensure \eqref{eq:ex2.4b:psi}, we require that 
  \(p\beta + 1 > 0\)
  \(q\beta + 1 < 0\), or equivalently
  \begin{equation} \label{eq:ex2.4b:beta1}
   -1 \leq -1/p < \beta < -1/q < 0.
  \end{equation}

  We compute
  \begin{equation*}
   \int_{[0,1]} \psi g_n\,dm 
   = \int_0^{n^\alpha} n x^\beta = n^{\alpha(\beta+1) + 1} / (\beta+1)
  \end{equation*}
   and since \(\beta+1 > 0\)
   to ensure \eqref{eq:ex2.4b:iii},  we require that 
   \begin{equation}  \label{eq:ex2.4b:beta2}
     \alpha(\beta+1) + 1 \geq 0.
   \end{equation}

   Combining the requirements of 
   \eqref{eq:ex2.4b:alpha}, 
   \eqref{eq:ex2.4b:beta1}, and 
   \eqref{eq:ex2.4b:beta2}
   \begin{eqnarray} 
     -1/(\beta + 1) \leq &\alpha& < -q/(q-1) \label{eq:ex2.4b:s1} \\
           -1/p     <    &\beta& < -1/q .    \label{eq:ex2.4b:s2}
   \end{eqnarray}
   We will define \(\beta = -1/p + \epsilon\) 
   for sufficiently small \(\epsilon > 0\) 
   so \eqref{eq:ex2.4b:s1} holds which is trivial,
   but will also allow for \eqref{eq:ex2.4b:s1} to hold for some \(\alpha\),
   as we now show.
   We first note that the mapping
   \(t \to -t(t-1) = -1/(t-1) -1\) is increasing.
   We deal with two similar cases.
   \paragraph{Case 1.} Assume \(p>1\). 
   Since
   \begin{equation*}
     -1/((-1/p) + 1) = -p/(p-1) < -q/(q-1),
   \end{equation*}
   For sufficiently small \(\epsilon>0\) 
   \begin{equation*}
     -p/(p-1) <  -1\big/\bigl(((1/p)+\epsilon)\, + 1\bigr) < -q/(q-1)
   \end{equation*}
   and thus we can find \(\alpha\) so \eqref{eq:ex2.4b:s1}s.
   \paragraph{Case 2.} Assume \(p=1\). The convergence
   \begin{equation*}
   \lim_{0<\epsilon\to 0} -1\big/\bigl((-1/p)+\epsilon + 1\bigr) = -\infty
   \end{equation*}
   Shows that for sufficiently small \(\epsilon > 0\)
   \begin{equation*}
      -1\big/\bigl((-1/p)+\epsilon + 1\bigr) < -q/(q-1)
   \end{equation*}
   and again we have \(\beta\) and \(\alpha\) so \eqref{eq:ex2.4b:s1} holds.



\end{document}
