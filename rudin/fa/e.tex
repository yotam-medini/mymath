\itemch{e}
We begin by proving the claim in the hint by steps.

\begin{llem}
Let \(\{x_j\}\) to \(x_0\) weakly in \ellone\ then
  \(\{\|x_j\|_1\}\) is bounded.
\end{llem}
\begin{proof}
{\nullfont dum}
\newline
\textbf{Claim-I.}
For any \(m,n>0\) the set \(\{|x_j(m,n)|: j\in\N\}\)  is bounded.
This is trivial by looking at \(y\in c_0\) defined by \(y(j,k)=1\)
if \((j,k)=(m,n)\) and \(y(j,k)=0\) otherwise.

\textbf{Claim-II.}
For any finite set $F$ of pairs 
\(m,n>0\) the set 
\begin{equation*}
\{|x_j(m,n)|: j\in\N\;\land\; (m,n)\in F\}
\end{equation*}
is bounded.
\newline
The bound is a simple maximum taken on the finite sets of 
bounds each of a fixed pair \((m,n)\in F\).

Now by negation assume that the set  \(\{\|x_j\|_1:j\in\N\}\) is not bounded.
We will construct \(y\in c_0\) such that the set
\(\{|y(x_j)|:\,j\in\N\}\) is not bounded, that is by changing to 
sub-sequence \((s(j))_{j\in\N}\) we can have
\begin{equation*}
\lim_{j\to\infty}|y(x_{s(j)})| 
 = \lim_{j\to\infty} \left| \sum_{(m,n)\in\N^2} y(m,n)\cdot x_{s(j)}(m,n) \right| 
 = \infty
\end{equation*}
which contradicts the assumption of this lemma.

We start by constructing mutually disjoints finite
sets \(D_n \subset \N^2\) by induction.
We put
\begin{alignat*}{2}
E_0 &= 0    &  d_0 &= 0 \\
E_j &= E_{j-1} \cup D_j   \qquad &   d_j &= \max\{m, n: (m,n)\in E_j\}
   \qquad (\forall j\in\N)
\end{alignat*}
Let \(D_0 = \emptyset\).
Assume \(D_j\) are defined for all \(j < k\).
Let $b$ be a bound (provided by Claim~II) of
\begin{equation*}
\{|x_j(m,n)|: \, j\in\N \land (m,n) \in E_{k-1}\}.
\end{equation*}
pick an index \(s(k)\) such that \(x_{s(k)}\) satisfy the following
\begin{align*} 
\sum_{(m,n) \in \N^2 \setminus E_{k-1}} |x(m,n)| > k(b + k + 1)
\end{align*}
Since \(x_j\in \ellone\) we can find some \(d_k\) such that 
\begin{align*} 
\sum_{(m,n > d_k) \in \N^2 \setminus E_{k-1}} |x(m,n)| < 1
\end{align*}
We define \(\lrcorner\)-like set
\begin{equation*}
D_k = \{(m,n)\in \N^2: m,n \leq d_k\} \setminus D_{k-1}
\end{equation*}
Clearly \(\N^2 = \disjunion_{j\in\N} D_j\), so we can define
\(y(m,n) = e^{i\theta(m,n)}/s(k)\) iff \((m,n)\in D_k\)
setting the argument \(\theta(m,n)\)
so that \(e^{i\theta(m,n)} x_{s(k)}(m,n) = |x_{s(k)}(m,n)|\).
Using the abbreviation 
\begin{equation*}
c(m,n) =  y(m,n)\cdot x_{s(k)}(m,n)
\end{equation*}
 we have 
%% \begin{align*}
%% |y(x_{s_k})|
%% &= \left| 
%%    \left(\sum_{(m,n)\in E_{k-1}} c(m,n)\right)
%%    + \left(\sum_{(m,n)\in D_k} c(m,n)\right)
%%    + \left(\sum_{(m,n)\in \N^2 \setminus E_k} c(m,n)\right)
%%    \right| \\
%% &\geq 
%%    \left|\sum_{(m,n)\in D_k} c(m,n)\right|
%%   - \left|\sum_{(m,n)\in E_{k-1}} c(m,n)\right|
%%   - \left|\sum_{(m,n)\in \N^2 \setminus E_k} c(m,n)\right| \\
%% &\geq
%%    \left(\sum_{(m,n)\in D_k} x_{s_k}(m,n)\right)\bigm/ k
%%   - \sum_{(m,n)\in E_{k-1}} |c(m,n)|
%%   - \sum_{(m,n)\in \N^2 \setminus E_k} |c(m,n)| \\
%% &\geq (b+k+1) - b - 1 = k.
%% \end{align*}
\end{proof}


%% With the lemma established, assume by negation that the sequence \(\{x_j\}\) 
%% where \(x_j\in M\), 
%% converges \upstar-weakly to \(x_0\).
%% Let $M$ be the bound of \(\{\|x_j\|_1:j\in\N\}\) provided by the lemma.
%% Let $K$ be such that \(\sum_{j=1}^K 1/j > M + 1\).
%% Let \(v\in c_0\) defined by \(v(m,n)=1\) iff \(1\leq m \leq K\) and \(n=1\)
%% and  \(v(m,n)=0\) otherwise.
%% By convergence, dor each \(\epsilon>0\) there must exists some $k$ such that 
%% \begin{equation*}
%% |(x_k - x_0)(v)| = \left|\sum_{m=1}^K x_k(m, 1) - m^{-2}\right| < \epsilon.
%% \end{equation*}
%% and then, putting \(y=x_k\) we have
%% \begin{align*}
%% \|y\|_1
%% &= \sum_{m=1}^\infty \left(|y(m,1)| + \sum_{j=2}^\infty|y(m,j)|\right) \\
%% &\geq \sum_{m=1}^\infty 
%%     \left(|y(m,1)| + \left|\sum_{j=2}^\infty|y(m,j)\right|\right)   \\
%% &= \sum_{m=1}^\infty |y(m,1)| +m|y(m,1)|
%%  = \sum_{m=1}^\infty (m+1)|y(m,1)|
%% \end{align*}
