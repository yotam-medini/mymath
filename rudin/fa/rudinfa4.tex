%%%%%%%%%%%%%%%%%%%%%%%%%%%%%%%%%%%%%%%%%%%%%%%%%%%%%%%%%%%%%%%%%%%%%%%%
%%%%%%%%%%%%%%%%%%%%%%%%%%%%%%%%%%%%%%%%%%%%%%%%%%%%%%%%%%%%%%%%%%%%%%%%
%%%%%%%%%%%%%%%%%%%%%%%%%%%%%%%%%%%%%%%%%%%%%%%%%%%%%%%%%%%%%%%%%%%%%%%%
\chapterTypeout{Duality in Banach Spaces}

\newcommand{\BXY}{\ensuremath{\scrB(X,Y)}}
\newcommand{\Xss}{\ensuremath{X^{**}}}

%%%%%%%%%%%%%%%%%%%%%%%%%%%%%%%%%%%%%%%%%%%%%%%%%%%%%%%%%%%%%%%%%%%%%%%%
%%%%%%%%%%%%%%%%%%%%%%%%%%%%%%%%%%%%%%%%%%%%%%%%%%%%%%%%%%%%%%%%%%%%%%%%
\section{Notes}

\subsection{More boundness Characterizations}

Some more results concering bounded functionals and bounded operators.

See \cite{Megginson1998} Corollary~2.4.5
\begin{llem} \label{llem:cont:bymaps}
Let \((W,\tau)\) be a topological space and $X$ be a set topologized
by a a family \calF\ of mappings \(f: X \to Y_f\).
A mapping \(g:W\to X\) is continuous iff \(f\circ g\) is continuous
for all \(f\in\calF\).
\end{llem}
\begin{proof}
If $g$ is continuous then \(f\circ g\) are continuous simply
by composition of continuous mappings.

Conversely, we assume \(f\circ g\) is continuous for all \(f\in\calF\).
It will be sufficient to show that \(g^{-1}(B)\) is open in $W$
for any set $B$ in a basis for the topology of $X$.

Let $G$ be a family of inverse images of open sets \(f^{-1}(V)\)
where \(f\in\calF\) and $V$ is open in \(Y_f\).
Following  the discussion of Section~3.8
let the $S$ be the family that consists of finite intersections
of sets from $G$. It is a subbasis for 
the topology of $X$ which consists of unions of sets from $S$.

Pick an arbitrary set
'\(B = \cap_{j\in J} f_j^{-1}(V_j)\) from the subbasis $S$ where \(V_j\) is open 
in \(Y_{f_j}\) and $J$ is a finite set of \calF-indices.
% (The finiteness $J$ is actually not needed for our arguments).
Put \(U_j = f_j^{-1}(V_j) \subset X\).
By the assumption each \((f_j\circ g)^{-1}(V_j) = g^{-1}(U_j)\) 
is open in $W$.
It is easy to see that
\begin{equation*}
g^{-1}(B) = g^{-1}\left(\cap_{j\in J} U_j\right)
  = \cap_{j\in J} g^{-1}(U_j).
\end{equation*}
Hence \(g^{-1}(B) \in \tau\) and so $g$ is continuous.
\end{proof}


The relation between continuity and boundness
will now be shown for normed spaces.
\begin{llem} \label{llem:op:cont:bounded}
Let \(T:X \to Y\) be a linear mapping between normed spaces.
Then $T$ is continuous iff \(T(E)\) is bounded in $Y$ whenever $E$
is bounded in $X$.
\end{llem}
\begin{proof}
Assume $T$ is continuous.
Let \(E\subset X\) be bounded subset.
Pick an arbitrary neighborhood \(0\in V \subset Y\)
By assumption there exists a neighborhood \(0 \in U \subset X\)
such that \(T(U) \subset V\) and there exists some \(n < 0\) such that
\(E \subset nU\). Hence 
\(T(E) \subset nT(U) \subset nV\). Hence the desired boundness condition 
holds for $T$. (Note that this direction did not use any norm).

Conversely, assume that $T$ maps bounded sets to bounded sets.
Then \(T(\{x\in X: \|x\| \leq 1\})\) is bounded in $Y$.
Thus there exists some \(M < \infty\) such that
\(\|Tx\| < M\) whenever \(\|x\| \leq 1\) where \(x \in X\).
Pick a neighborhood \(0\in V \subset Y\).
We can find some \(r > 0\) such that \(\{y\in Y: \|y\| \leq r\} \subset V\).
Put the open set \(W = \{x \in X: \|x\| < r/M\}\) and clearly
\(T(W) \subset U\). Thus $T$ is continuous at the origin.
By linearity $T$ is continuous.
\end{proof}


Here is a characterization of linear operator's boundness
(\cite{Megginson1998} Proposition~2.5.10).
\begin{llem}
Assume \(T:X \to Y\) is a linear operator between normed spaces.
Then $T$ is bounded  iff \(y^* \circ T \in X^*\) for all \(y^*\in Y^*\).
\end{llem}
\begin{proof}
If a functional is weakly-continuous then obviously it is
continuous in the norm topology. Theorem~3.10  gives the converse
that a any continuous functional is also weakly-continuous.
By Local-Lemma~\ref{llem:cont:bymaps} $T$ is weak-to-weak continuous.

The mapping $T$ is normed-continuous iff it \(T(E)\) is norm-bounded
in $Y$ whenever $E$ is normed bounded in $X$.
By Theorem~3.18 this is equivalent 
to the condition that \(T(E)\) is weakly-bounded
whenever $E$ weakly-bounded.
\end{proof}

The relation between operator continuity and its weak-continuity
is shown in the next result
\begin{llem} \label{llem:weak:to:weak}
Let \(T:X \to Y\) be a linear map between normed spaces.
Then $T$ is continuous iff $T$ is weak-to-weak continuous.
\end{llem}
\begin{proof}
By Local Lemma~\ref{llem:op:cont:bounded}
$T$ is continuous iff it map bounded sets to bounded sets.
By Theorem~3.18 it is equivalent to
require that $T$ maps weakly bounded sets to weakly bounded sets.
\end{proof}

Here a are simple lemmas regarding bounded and weak boundness of a set.
\begin{llem} \label{llem:bound:weakbound}
If $E$ is bounded then it is weakly bounded.
\end{llem}
\begin{proof}
Trivial since every weak neighborhood is a neighborhood
in the original topology.
\end{proof}


\begin{llem} \label{llem:bound:funcbound}
If  $E$ is a bounded subset of  a topological vector space $X$,
then \(\Lambda (E)\) is bounded in \C\ for every \(\Lambda \in X^*\).
\end{llem}
\begin{proof}
By  negation assume \(\Lambda(E)\) is unbounded.
Thus we can find a sequence \(x_n\in E\) such that \(|\Lambda x_n| \geq n\).
Define the open $0$ neighborhood \(V = \{x \in X: |\Lambda x\| < 1\}\).
Clearly \(x_n \notin nV\) for all $n$. Hence \(E \not\subset nV\) for all $n$
contradicting the assumption.
\end{proof}

We have a ``weak equivalnce''.
\begin{llem} \label{llem:weakbound:funcbound}
A subset  $E$ is weakly bounded in a topological vector space $X$,
iff \(\Lambda (E)\) is bounded in \C\ for every \(\Lambda \in X^*\).
\end{llem}
\begin{proof}
This was shown in the discussion of Section~3.11.
\end{proof}

With norm we have an equivalnce.
\begin{llem} \label{llem:bound:norm:funcbound}
Let  \(E \subset X\) where $X$ is a normed vector space.
Then $E$ is bounded iff \(\Lambda (E)\) is bounded in \C\ for
every \(\Lambda \in X^*\).
\end{llem}
\begin{proof}
If $E$ is bounded ,then it is weakly bounded and by
Local Lemma~\ref{llem:op:cont:bounded}
\(\Lambda (E)\) is bounded in \C\ for
every \(\Lambda \in X^*\).
Conversely, assume \(\Lambda (E)\) is bounded in \C\ for
every \(\Lambda \in X^*\). 
By Local Lemma~\ref{llem:weakbound:funcbound} again,
$E$ is weakly bounbded. By Theorem~3.18 it is also originally bounded.
\end{proof}

The following lemma shows that in boundness of operator
can be checked by looking at weak topologies, when the spaces are normed
(See \cite{Megginson1998} Proposition~2.5.11).
\begin{llem} \label{llem:op:normbd:weakbd}
Let \(T:X\to Y\) be a linear operator between normed spaces.
Then $T$ is bounded iff it is weak-to-weak bounded.
\end{llem}
\begin{proof}
Let \(B_X = \{x\in X: \|x\|\leq 1\}\).
By definitions, \(\Lambda\) is bounded if \(T(B_X)\) is bounded in $Y$. 
By Local Lemma~\ref{llem:bound:norm:funcbound} 
This is equivalent to  \((\Lambda T)(B_X)\) be bounded in \C\ for
all \(\Lambda \in Y^*\).
Since all these \(\Lambda T \in X^*\) for all \(\Lambda \in Y^*\)
the last condition means, that $T$ is bounded
iff  \((\Lambda T)(B_X)\) are continuous mappings \(X\to\C\)
for all \(\Lambda \in Y*\) considering the weak topology of $X$.

In turn, by Local Lemma~\ref{llem:cont:bymaps} this is equivalent 
to $T$ being continuous mapping 
from $X$ with its weak to $Y$ with its weak topology.
\end{proof}

%%%%%%%%%%%%%%%%%%%%%%%%%%%%%%%%%%%%%%%%%%%%%%%%%%%%%%%%%%%%%%%%%%%%%%%%
\subsection{Compact operators}

In \textbf{4.16} the text mentions equivalent conditions
for being comapct operators without proof.

\begin{llem} \label{lem:compact:op}
Let \(T\in \BXY\). The following conditions are equivalent.
\begin{itemize}
\itemch{a} $T$ is a compact operator.
\itemch{b} If \(U = \{x\in X: \|x\|<1\}\) is the open unit ball,
        then \(T(U)\) is totally bounded.
\itemch{c} Every bounded \(\{x_n\}\) sequence in $X$ has
         a subsequence \(\{x_{n_i}\}\) such that \(\{T(x_{n_i})\}\)
        converges to a point \(y\in Y\).
\end{itemize}
\end{llem}
\begin{proof}

[\ich{a} \(\to\) \ich{b}].
Assume $T$ is compact, and \(\epsilon>0\).
Let \(G_u = \{y\in Y: \|y - T(u)\|< \epsilon\}\) be a family
of open balls for each \(u\in U\).
Since \(\cup_{u\in U} G_u  \subset T(U)\) by \ich{a} 
there is a finite sib-family that covers T(U).
Thus \(T(U)\) is totally bounded.

[\ich{b} \(\to\) \ich{c}].
Let \(\{x_n\}\) be bounded by \(M > 0\) in $X$.
By moving to  \(\{x_n/M\}\) we may assume \(M\leq1\), \(x_n\in U\).
Clearly \(\Gamma=\{T(x_n)\}\subset T(U)\)
which is bounded by $T$ being bounded.
By \ich{b} for each \(\epsilon>0\)
By looking at finite balls covering of radius \(\epsilon\)
there exists some ball with infinite points from \(\Gamma\).
Thus  \(\Gamma\) is a Cauchy sequence and by completeness of $Y$
 \(\Gamma\) has a converging subsequence.

[\ich{c} \(\to\) \ich{a}].
We start by definition.
Given a metric space \((X,d)\) with an open covering
\(\{G_\alpha: \alpha\in I\}\).
If there exist \(\delta>0\) 
--- \emph{Lebesgue's number} \index{Lebesgue number} --
such that 
if \(A\subset X\) and its diameter \(\sup \{d(x,y): x,y\in A\} < \delta\)
then  \(A \subset G_\alpha\) for some \(\alpha\in I\).

Put \(V = \overline{T(U)}\).
Let \(\Gamma=\{G_\alpha: \alpha\in I\}\) be a covering of $V$.
Assume by negation, \(\Gamma\) has no Lebesgue number.
Thus for any \(n\in\Z^1\) we can find \(C_n=B(y_n,1/n)\)
such that \(\forall \alpha\in I, C_n \not\subset G_\alpha\).

By \ich{c} there exists \(a\in V\) that is
a limit of some subsequence of \((y_n)\).
Pick \(G_\alpha \ni a\) and \(B(a,\epsilon)\subset G_\alpha\).
Now there are infinite numbers $n$ such 
that \(y_n \in B(a,\epsilon/2)\), for each (such) \(n > 2/\epsilon\)
\begin{equation*}
B(y_n,1/n) \subset B(y_n,\epsilon/2) \subset B(a,\epsilon) \subset G_\alpha
\end{equation*}
a contradiction. Thus \(\Gamma\) has a Lebesgue number \(\delta>0\).

Assume by negation \(\Gamma\) does not have finite sub-cover
 of \(\overline{T(U)}\). We will define a sequence \((y_n)\) in $V$.
Pick arbitrary \(y_1\in V\) and \(G_{\alpha_1} \supset
 B(y_1, \delta/2)\).
Say \((x_n)\) and \(G_{\alpha_n}\) where chosen for \(n<k\).
Since 
\begin{equation*}
V \not\supset H_{k-1} = \cup_{n<k} G_{\alpha_n}
\end{equation*}
we pick \(y_k \in V \setminus H_{k-1}\)
and again \(G_{\alpha_k} \supset B(y_k, \delta/2)\).
Note that \(d(y_i,y_j)\geq \delta/2\) when \(i\neq j\).
Thus \((y_n)\) has no converging subsequence, contradiction to \ich{c}.
\end{proof}

%%%%%%%%%%%%%%%%%%%%%%%%%%%%%%%%%%%%%%%%%%%%%%%%%%%%%%%%%%%%%%%%%%%%%%%%
\subsection{Lemmas for the Exercises}

\begin{llem} \label{lem:compact:separable}
A compact set $K$ in a metric space is separable.
\end{llem}
\begin{proof}
Let $d$ be the metric.
For any \(n\in\N\), consider the open covering
\begin{align*}
G_{n,x} &= \{y\in K: d(y,x) < 1/n\}\\
\calG_n &= \cup_{x\in K} G_{n,x}
\end{align*}
By compactness, for any \(n\in\N\) there is a finite 
sub-covering of \(\calF_n = \calG_n\).
Let \(C_n\) be the finite set of centers of the \(\calF_n\) balls.
Now \(F = \cup_{n\in\N} C_n\) is countable, and clearly dense in $K$.
\end{proof}

%%%%%%%%%%%%%%%%%%%%%%%%%%%%%%%%%%%%%%%%%%%%%%%%%%%%%%%%%%%%%%%%%%%%%%%%
%%%%%%%%%%%%%%%%%%%%%%%%%%%%%%%%%%%%%%%%%%%%%%%%%%%%%%%%%%%%%%%%%%%%%%%%
\section{Exercises} % pages 111-115

Throughout this set of exercises, $X$ and $Y$ denote Banach spaces, unless
the contrary is explicitly stated.

%%%%%%%%%%%%%%%%%
\begin{enumerate}
%%%%%%%%%%%%%%%%%

%%%%%%% 1
\begin{excopy}
Let \(\phi\) be the embeddings of $X$ onto \Xss\ in Section~4.5.
Let \(\tau\) be the weak topology of $X$, and let \(\sigma\)
be the weak\upstar-topology of \Xss\ --- the one induced by \(X^*\).
\begin{itemize}

\itemch{a}
Prove that \(\phi\) is a homeomorphism of \((X,\tau)\) onto a dense subspace
of \((\Xss, \sigma)\).

\itemch{b}
If $B$ is the closed unit ball of $X$, prove that \(\phi(B)\)
is \(\sigma\)-dense in the closed unit ball
of \Xss. (Use the Hahn-Banach separation theorem.)

\itemch{c}
Use \ich{a}, \ich{b}, and the Banach-Alaoglu theorem to prove that $X$
 is reflexive if and only if $B$ is weakly compact.

\itemch{d}
Deduce from \ich{c} that every norm-closed subspace of reflexive space $X$
 is reflexive.

\itemch{e}
If $X$ is reflexive and $Y$ is a closed subspace of $X$, prove that \(X/Y\) 
is reflexive.

\itemch{f}
Prove that $X$ is reflexive if and only if \(X^*\) is reflexive.\\
\emph{Suggestion:} One half follows from \ich{c}; for the other half,
 apply \ich{d} to the subspace \(\phi(X)\) of \Xss.

\end{itemize}
\end{excopy}

See \cite{Megginson1998} Theorems (or alike)~1.11.17,
(Goldstine)\index{Goldstine}~2.6.26, 2.6.28.

\begin{itemize}
\itemch{a}
The homeomorphism is implied by definitions of weak topologies.
We need to show that the image \(\phi(X)\) is weak\upstar\ dense.
If \(\dim X < \infty\),
 the result is trivial since the topologies are equivalent.
Thus we may assume \(\dim X = \infty\).
Pick some base weak\upstar\ neighborhood of~0,
by fixing some \(\epsilon>0\) and \(x_j^*\in X^*\) for \(j\in\N_n\) and
setting
\begin{equation*}
V = \{w^{**}\in X^{**}: \forall j\in\N_n,\; |w^{**}(x_j^*)| < \epsilon\}.
\end{equation*}
Let \(x^{**}\in X^{**}\), we will show that
\begin{equation*}
(x^{**} + V) \cap \phi(X) \neq \emptyset.
\end{equation*}
\iffalse
We may assume that \(\{x_j^*:j\in\N_n\}\) are linearly independent,
since otherwise, we can take a linearly independent subset,
and the rest of the evaluations will be determined.
\fi
We may assume that \(\{x_j^*:j\in\N_m\}\) are linearly independent (\(m\leq n\)).
We would like to show that actually there exists \(x\in X\) such that
\begin{equation} \label{eq:xjsx:eq:xssxj}
\forall j\in\N_m,\; x_j^*(x) = x^{**}(x_j^*).
\end{equation}
This will imply, that \eqref{eq:xjsx:eq:xssxj} holds \(\forall j\in N_n\)
as well.

Put
\begin{align*}
N &= \bigcap_{k\in\N_n} \Ker(x_k^*) \\
N_j &= \bigcap_{k\in\N_n\setminus\{j\}} \Ker(x_k^*) \qquad (j\in\N_n).
\end{align*}
Note that if \(T_j:X \to \C^{n-1}\) defined by
\begin{equation*}
T_j(x) = 
 \bigl(
   x_1^*(x),x_2^*(x),\ldots,x_{j-1}^*(x),x_{j+1}^*(x),\ldots,x_n^*(x)
 \bigr)
\end{equation*}
then \(N_j = \Ker(T_j)\), and thus \(\dim N_j = \infty\).

Since \(\{x_j^*:j\in\N_m\}\) are linearly independent, by Lemma~3.9
for each \(j\in\N_n\) there exists \(x_j \in N \setminus N_j\)
such that \(X_j^*(x_k) = \delta_{j,k}\) for all \(j,k\in\N_n\).

Now we can form a system of linear equations
\begin{equation*}
\sum_{k=1}^n \alpha_k X_j^*(x_k) 
  = x_j^*\left(\sum_{k=1}^n \alpha_k (x_k)\right) = x^{**}(x_j^*)
\end{equation*}
Solving this, gives the desired
\(x = \sum_{k=1}^n \alpha_k (x_k)\).

\itemch{b}
The (closed) unit ball $B$ is convex, and so  \(\phi(B)\) 
is convex as well.
Clearly \(U := \overline{\phi(B)}^{w^*}\) is convex as well
and also balanced.
We look at \(X^*\) as a family of functionals on~\(X^{**}\).
By definition of the latter, the former is a separating family.
The topological vector space
\((X^{**},\sigma)\) is 
by Theorem~3.10,
locally convex and its dual is \(X^*\).

Assume by negation \(x_0^{**} \in X^{**}\setminus U\) 
and \(\|x_0^{**}\| \leq 1\).
By Theorems~3.7 there exists \(\Lambda \in X^*\)
such that 
  \(|x^{**}(\Lambda)| \leq 1\) for all \(x^{**} \in U\) and
  \(|x_0^{**}(\Lambda)| > 1\).
Thus we can find a \(\sigma\)-neighborhood \(V \ni x_0\)
such that \(V \cap U = \emptyset\) which gives a~contradiction.


\itemch{c}
We note that both \(\sigma\) and \(\tau\) are 
topologies induced by \(X^*\). Hence they are equivalent via \(\phi\).

Assume $X$ is reflexive. 
The ball $B$ is closed by the norm topology,
and by Corollary~\ich{a} of Theorem~3.12 it is \(\sigma\) closed.
Hence \(\phi(B)\) is \(\tau\)-closed and 
by Banach-Alaoglu Theorem~3.15 \(\phi(B)\) is \(\tau\)-compact
and so $B$ is weakly compact.

Conversely, assume $B$ is weakly compact.
The map \(\phi: (X,\sigma \to (X^{**},\tau)\) is homeomorphism,
hence  \(\phi(B)\) is \(\tau\)-compact in \(\phi(X)\).
But this implies that \(\phi(B)\) is compact also in \(X^{**}\).
In Hausdorff space, any compact set is closed, and thus
\(\phi(B)\) is \(\tau\)-closed in \(X^{**}\) and by \ich{b}
we see that \(\phi(B)\) is the closed unit ball of \(X^{**}\).
By linearity \(\phi(X) = X^{**}\) and so $X$ is relative.

\itemch{d}
Let \(Y \subset X\) be a normed-closed subspace of $X$.
It will be sufficient to show that \(B\cap Y\) is weakly compact.

We first show that the 
weak topology \(\sigma_Y\) restricted by \(\sigma\)
is the same as the
weak topology of $Y$ (induced by \(Y^*\)).
The inverse images of open sets in \C\ by the functionals \(\Lambda \in X^*\)
are a  sub-base of \(\sigma_Y\). This sub-base is actually the
same as 
The inverse images of open sets in \C\ by the 
the functionals \(\{\Lambda_Y: \Lambda \in X^*\} \subset Y^*\).
But the last inclusion is actually an equality,
since by the Hahn-Banach Theorem~3.6 each functional \(f\in Y^*\)
can be extended to some \(\Lambda \in X^*\).
Thus the two topologies have a same sub-base and so they are equal.

Since $Y$ is norm-closed and convex, by Corollary~\ich{a} of Theorem~3.12
It is also weakly closed in $X$. Hence \(B \cap Y\) is weakly compact.

\itemch{e}
It would be better to clearly ignore the trivial \(X = Y\) case.
Note the if \(P: X \to X/Y\) is the projection,
then \(P(B)\) is not necessarily the unit ball of \(X/Y\).

The projection \(P:X\to X/Y\) gives a natural injection
\begin{align*}
 T: (X/Y)^* &\to X^* \\
 T(\Lambda^*) &= \Lambda \\
 \Lambda(x) &= \Lambda^*(x + Y) \qquad (x\in X)
\end{align*}
The last setting is clearly well defined.

Assume now that $X$ is reflexive. We will show that the natural
\(\tilde{\phi}:(X/Y) \to (X/Y)^{**}\) 
 injection 
is surjective.
Let \(\tilde{x}^{**} \in (X/Y)^{**}\).
It is a functional on \((X/Y)^*\) 
that can be viewed as a functional on \(T((X/Y)^*) \subset X^*\).
It can be extended by Hahn-Banach Theorem~3.6
to some \(x^{**} \in X^{**}\). Since $X$ is reflexive
\(\phi(x) = x^{**}\) for some (unique) \(x\in X\).
Now clearly
\begin{equation*}
\forall x^* \in X^*,\; (x+Y,x^*) = (x^*,\tilde{x}^{**}).
\end{equation*}
Thus \(\tilde{\phi}(x) = \tilde{x}^{**}\) and \(\tilde{\phi}\)
is surjective.

\itemch{f}
Assume $X$ is reflexive. Since the dual space depends only on the original
and since \(X \cong X^{**}\) via the natural embedding,
similar embedding gives \(X^* \cong (X^{**})^* = X^{***}\).

Conversely, assume \(X^*\) is reflexive. By what was just shown, 
\(X^*\) is reflexive and by \ich{d}, its norm-closed subspace \(\phi(X)\)
is reflexive. By isomorphism, $X$ is reflexive.
\end{itemize}


%%%%%%% 2
\begin{excopy}
Which of the spaces \(c_0\), \ellone, \ellp, \ellinf are reflexive?
Prove that every finite-dimensional
normed space is reflexive. Prove that $C$, the supremum-normed space of all
complex
continuous functions, on the unit interval, is not reflexive.
\end{excopy}

We know that (upto natural isomorphism) 
\(c_0^* = \ellone\),
\((\ellone)^* = \ellinf\)
and \(c_0 \not\cong \ellinf\) hence 
by previous Exercise~1\ich{f} these 3 spaces are \emph{not} reflexive.
When \(1<p<\infty\) we have \((\ellp)^* = \ell^q\)
where \(q = p/(p-1)\) and similarly \((\ell^q)^* = \ellp\)
hence \ellp\ are reflexive.

As a vector space, every normed space $V$ of 
dimension \(n < \infty\) is isomorphic
to \(\C^n\). Every functional on $V$ is determined by its values 
on a basis of $V$. Thus \(\dim V^* = \dim V = n\).
Hence \(\dim V = \dim V^{**}\).
Now in a finite-dimensional vector space, a proper sub-space
must be of smaller degree, hence the natural embedding of $V$
into \(V^{**}\) must be surjective, hence every finite-dimensional normed
space is reflexive.

By Riesz representation \cite{RudinRCA80} Theorem~6.19
\(C(I)^*\) are the complex Lebesgue measure on $I$
which can be identified with \(L^1(I)\), hence \(C(I)^{**} \cong L^\infty(I)\)
and so \(C(I)\) is not reflexive.


%%%%%%% 3
\begin{excopy}
Prove that a subset $E$ of \BXY\ is equicontinuous if and only if there exists
\(M < \infty\)
such that \(\|\Lambda\| < M\) for every \(\Lambda \in E\).
\end{excopy}

Assume $E$ is equicontinuous.
Pick \(\epsilon = 1\), then there exist some \(\delta > 0\)
such that \(\|\Lambda x \| < 1\) whenever \(\Lambda \in E\) 
and \(\|x\| < \delta\).
Now take \(M = 1/\delta\) and now \(\|\Lambda u\| < M\)
for all \(\Lambda \in E\) and \(u \in \{x\in X: \|x\| \leq 1\}\).

Conversely, assume such an $M$ exists.
By linearity, it is sufficient to show that 
$E$ is equicontinuous at the origin \(0 \in X\).
for any neighborhood $V'$ of \(0\in Y\)
we can find some \(\epsilon > 0\) such that 
\begin{equation*}
V := \{y \in Y: \|y\|< \epsilon\} \subset V'.
\end{equation*}
Now let 
\(W := \{x \in X: \|x\|< \epsilon/M\}\).
Since 
\begin{equation*}
\forall \Lambda \in E,\, \forall x\in X\; \|\Lambda x \| < M\cdot \|x\|
\end{equation*}
we have
\begin{equation*}
\forall \Lambda \in E,\, \forall x\in  W\; \Lambda x \in V.
\end{equation*}
Thus $E$ is equicontinuous.

%%%%%%% 4
\begin{excopy}
Recall that \(X^* = \scrB(X,\C)\) if \C\ is the scalar field.
Hence \(\Lambda^* \in \scrB(\C,X^*)\) for every
\(\Lambda \in X^*\). Identify the range of \(\Lambda^*\).
\end{excopy}

The range is \(\{\lambda \Lambda: \lambda\in\C\}\).


%%%%%%% 5
\begin{excopy}
Prove that \(T \in \BXY\) is an isometry of $X$ onto $Y$ if and only if
\(T^*\) is an isometry of \(Y^*\) onto \(X^*\).
\end{excopy}

Let 
\begin{alignat*}{2}
B_X  &= \{x \in X: \|x\| \leq 1\} \qquad &
  B_Y  &= \{y \in Y: \|y\| \leq 1\} \\
B_X^* &= \{\Lambda \in X^*: \|\Lambda\| \leq 1\}  \qquad &
  B_Y^* &= \{\Psi \in Y^*: \|\Psi\| \leq 1\} 
\end{alignat*}


Assume $T$ is an isometry, then \(T(B_X)\subset B_Y\).
Moreover, \(T(B_X) = B_Y\) since $T$ is onto.
If~\(\Psi\in B_Y^* \) then
\begin{equation*}
\|T^*(\Psi)\|
 = \sup_{x\in B_X} |(T^*(\Psi))(x)| 
 = \sup_{x\in B_X} |\Psi(Tx)|
 = \sup_{y\in B_Y} |\Psi(y)| 
 = \|\Psi\|.
\end{equation*}
Hence \(T^*\) is an isometry. 
To show it is surjective, pick arbitrary \(\Lambda_X \in X^*\).
We simply define \(\Lambda_Y \in Y^*\) by 
\(\Lambda_Y(y) = \Lambda_X(T^{-1}(y))\) which is well defined
since $T$ is onto and being isometry is also injective.


Conversely, assume \(T^*\) is an onto isometry.
Assume by negation \(\|x\| \neq \|Tx\|\) for some \(x\in X\)
and put \(y = Tx\).
There are two cases.

\textbf{Case 1.}
Assume by negation \(\|x\| < \|y\|\).
By Hahn-Banach Theorem~3.6, we can find a functional \(\Lambda \in Y^*\)
such that \(\Lambda y = \|y\|\) and \(\|\Lambda\| = 1\).
Now \(T^*(\Lambda) \in X^*\) and \((T^*\Lambda)(x) = |y|\)
which gives the \(\|T^*(\Lambda)\| > 1\) contradiction.
Note that for this case we did not use the fact that \(T^*\) is onto.


\textbf{Case 2.}
Assume by negation \(\|x\| > \|y\|\).
We can find some \(\Lambda_X \in X^*\) such that 
(again wth Hahn-Banach Theorem~3.6)
\(\Lambda_X x = \|x\|\)
and
\(\|\Lambda_X\| = 1\).
Since \(T^*\) is onto, we can also find (unique by isometry)
\(\Lambda_Y \in Y^*\) such that \(T^*(\Lambda_Y) = \Lambda_X\).
Now
\begin{equation*}
  \Lambda_Y y
= \Lambda_Y (T x) 
= (T^*(\Lambda_Y)) (x) 
= \Lambda_X x
= \|x\|.
\end{equation*}
If \(y \neq 0\) then the above equality gives the \(\|x\| = 0\) contradiction.
Otherwise  \(\|\Lambda_Y\| \geq \|x\|/\|y\| > 1\) is a contradiction
as well, and so $T$ is an isometry.

The surjectivity of $T$ follows from Theorem~4.15.



%%%%%%% 6
\begin{excopy}
Let \(\sigma\) and \(\tau\) be the weak\upstar-topologies of \(X^*\) and \(Y^*\)
respectively, and prove that $S$ a
continuous linear mapping of \((Y^*, \tau)\) into \((X^*, \sigma)\)
 if and only if \(S = T^*\) for some \(T \in \BXY\).
\end{excopy}

See \cite{Megginson1998} Theorem~3.1.11.

Assume \(T \in \BXY\), then for any \(x \in X\) % \(\Lambda \in Y^*\)
\begin{equation*}
(S \Lambda)(x) = (T^* \Lambda)(x) = \Lambda(T x) \qquad (\Lambda \in Y^*)
\end{equation*}
is a continuous function simply by composition of continuous functions
in the norm topologies of \(X^*\) and \(Y^*\).
The linearity of $S$ is clear so it is sufficient to show 
the weak\upstar-to-weak\upstar\ continuity in the origin.
Let \(V'\subset X^*\) be a \(\sigma\)-neighborhood of $0$.
We can find a base sub-neighborhood \(V \subset V'\)
\begin{equation*}
V = \{x^*\in X^*: \forall j\in\N_m,\; |x*(x_j)| < \epsilon\}
\end{equation*}
for some \(\epsilon > 0\) and \(x_j \in X\) for \(1\leq j \leq m\).
Now
\begin{align*}
(T^*)^{-1}(V)
 &= \{y^*\in Y^*: T^*(y^*) \in V\}
  = \cap_{j=1}^m \{y^*\in Y^*: |\langle x_j, T^*(y^*)\rangle | < \epsilon\} \\
 &= \cap_{j=1}^m \{y^*\in Y^*: |\langle T(x_j), y^*\rangle | < \epsilon\}
\end{align*}
which is \(\tau\)-open in \(Y^*\). Thus $S$ is also
continuous  \((Y^*, \tau) \,\to\, (X^*, \sigma)\) mapping.

Conversely, assume $S$ is
continuous linear mapping of \((Y^*, \tau)\) into \((X^*, \sigma)\).
Let \(\psi_X\) and \(\psi_Y\) the isomorphisms of $X$ and $Y$
to their second duals.
Pick \(x\in X\). By composition \(\psi_X \circ S\) is weakly\upstar\ continuous
functional on~\(Y^*\). Hence \(\psi_X \circ S \in \psi_Y(Y)\).
Define
\begin{equation*}
Tx = \psi_Y^{-1}(\psi_X \circ S) \in Y.
\end{equation*}
The linearity of $T$ is clear. By definition it is weak-to-weak continuous.
By local lemma~\ref{llem:op:normbd:weakbd} $T$ is also normed-continuous.


%%%%%%% 7
\begin{excopy}
Let \(L^1\) be the usual space of integrable functions on the closed unit
 interval $J$, relative
to Lebesgue measure. Suppose \(T \in \scrB(L^1, Y)\),
 so that \(T^* \in \scrB(Y^*, L^\infty)\). Suppose
\(\scrR(T^*)\) contains every continuous function on $J$.
 What can you deduce about $T$?
\end{excopy}

Since \(C(J)\) is dense in \(L^\infty(J)\) we have
\(\scrR(T^*)^\perp = \{0\}\).
By Theorem~4.12 \(\scrN(T) = \{0\}\), hence $T$ is injective.


%%%%%%% 8
\begin{excopy}
Prove that \((ST)^* = T^*S^*\). Supply the hypotheses under which this makes
 sense.
\end{excopy}

Let $X$, $Y$, $Z$ be Banach spaces and let 
\begin{equation*}
T: X \to Y \qquad S: Y\to Z
\end{equation*}
linear bounded mapping.
Let \(\Lambda_Z \in Z^*\) and \(x\in X\).
\begin{align*}
\bigl((ST)^*(\Lambda_Z)\bigr)(x)
 &= \Lambda_Z\bigl((ST)(x)\bigr)
  = \Lambda_Z\bigl(S(Tx)\bigr)
  = (S^*\Lambda_Z)(Tx)
  = \bigl(T^*(S^*\Lambda_Z)\bigr)(x) \\
  &= \bigl((T^*S^*)\Lambda_Z\bigr)(x).
\end{align*}
Hence \((ST)^* = T^*S^*\).


%%%%%%% 9
\begin{excopy}
Suppose \(S\in \scrB(X)\), \(T\in \scrB(X)\).
\begin{itemize}
\itemch{a}
Show, by an example, that \(ST = I\) does not imply \(TS = I\).
\itemch{b}
However, assume $T$ is compact, show that
\begin{equation*}
S(I - T) = I \quad \textnormal{if and only if} \quad (I - T)S = I,
\end{equation*}
and show that either of these equalities implies that 
\(I - (I - T)^{-1}\) ts compact.
\end{itemize}

\end{excopy}

\begin{itemize}
\itemch{a}
Consider \(X = \ellp\) with $S$ and $T$ the left and right shift operators
respectively. That is 
\begin{align*}
(S(x))(n) &= x(n+1) \\
(R(x))(n) &= x(n-1) \quad \textnormal{if} \quad n>1 
\qquad \textnormal{and} \quad (R(x))(0) = 0.
\end{align*}
\itemch{b}
Assume \(S(I - T) = I\).
Then \(\scrN(I -T) = \{0\}\) 
% and \(\scr(S) = X\),
hence \(\dim\scrN(I -T) = 0\) and
by Theorem~4.25
\(\dim X / \scrR(I - T) = 0\) and so 
\(\scrR(I - T) = X\).
By negation assume \(y = (I - T)Sx \neq x\) for some \(x\in X\).
Then \(Sy = S(I - T)Sx = Sx\) and so \(0 \neq (x-y) \in \scrN(S)\).
Now we can find some \(w\in X\) such that \(x-y = (I-T)w\)
which gives the following \(0 = S(I - T)w = w \neq 0\) contradiction.

Conversely, assume \((I - T)S = I\).
Then \(\scrR(I - T) = X\) 
hence \(\dim X / \scrR(I - T) = 0\)
and by Theorem~4.25 \(\dim \scrN(I -T) = 0\) 
and so \(\scrN(I -T) = \{0\}\).
Assume by negation \(y = S(I - T)x \neq x\) for some \(x\in X\).
Now \((I - T)y = (I - T)S(I - T)x = (I - T)x\) and so 
\(0 \neq x - y \in \scrN(I - T)\) which is a contradiction.

If any if the equalities \(S(I - T) = I = (I - T)S\) holds
then \(S = (I - T)^{-1}\). 
Now \(I - S) = (I - T)S - S = -TS\) which is compact by Theorem~4.18\ich{f}.
\end{itemize}

%%%%%%% 10
\begin{excopy}
Assume \(T \in \scrB(X)\) is compact, and assume either that \(\dim X = \infty\)
or that the scalar
field is \C. Prove that \(\sigma(T)\) is not empty. However, \(\sigma(T)\)
 may be empty if \(\dim X < \infty\)
and the scalar field is \R.
\end{excopy}

For \(X = \R^2\), define (rotation) by \(T((x,y)) = (-y, x)\) which has
no eigenvalue. Now 
\begin{equation*}
(T - \lambda I)((x,y)) = (-\lambda x - y, x -\lambda y).
\end{equation*}
and
\begin{equation*}
(T - \lambda I)^{-1}((v,w)) 
  = \frac{1}{\lambda^2 + 1}(\lambda v + w, -v -\lambda w).
\end{equation*}
Hence \(T - \lambda I\) has inverse for any \(\lambda \in \C\).

If \(X = \C^n\) then $T$ can be represented 
as an~\(n\times n\) matrix \(A = A_T\)
whose characteristic polynomial \(p_A(t) = \det(tI - A)\).
This polynomial must have a root in \C\ which is an eigenvalue
of $A$ and actually of $T$. For such root \(\lambda\),
that is \(p_A(t) = 0\). Since now \(T -\lambda I\) 
has nonzero eigenvector of the \(\lambda\) eigenvalue,
hence \(T - \lambda I\) is not injective, not invertible
and so \(\lambda \in \sigma(T)\).

Now assume \(\dim X = \infty\) and the scalar field is \C.
\emph{Note:} A compact operator may not have a finite rank,
see \cite{Megginson1998} Section~3.4.5.
Theorem~4.18\ich{e} shows that \(0 \in \sigma(T)\),
thus \(\sigma(T)\neq\emptyset\).


%%%%%%% 11
\begin{excopy}
Suppose \(\dim X < \infty\) and show that the equality \(\beta^* = \beta\)
 of Theorem~4.25 reduces to the
statement that the row rank of a square matrix is equal to its column rank.
\end{excopy}

Say \(\dim X = n < \infty\).
By the last paragraphs of Section~4.9 titled \textbf{Adjoints},
an operator \(T\in B(X)\) can be views as \(n\times n\)-matrix.
Then the adjoint operator \(T^*\) is actually represented exactly
by the transposed matrix \(T^t\).

By basic results of linear algebra the rank of $T$ and \(T^t\) 
are equal and actually are the raw-rang and column rank of $T$.
Clearly by looking at standard basis \((e_k)_{k=1}^n\) of \(\C^n\),
they are the dimension of the images of $T$ and \(T^*\) operators.


%%%%%%% 12
\begin{excopy}
Suppose \(T \in \BXY\) and \(\scrR(T)\) is closed in $Y$. Prove that
\begin{align*}
\dim \scrN(T) &= \dim X^*/\scrR(T^*) \\
\dim \scrN(T^*) &= \dim Y/\scrR(T)
\end{align*}
This generalizes thed assertions \(\alpha = \beta^*\) and \(\alpha^* = \beta\)
 of Theorem~4.25.
\end{excopy}

\emph{Note:} In this case the dimensions are not necessarily finite.

From Theorem~4.25, 
we use similar proof ideas and the similar notations
\begin{alignat*}{2}
\alpha &= \dim \scrN(T)   \qquad &   \beta &= \dim Y / \scrR(T) \\
\alpha^* &= \dim \scrN(T^*) \qquad & \beta^* &= \dim X^* / \scrR(T^*).
\end{alignat*}
Thus we need to show \(\alpha = \beta^*\) and \(\alpha^* = \beta\).
As shown there if \(M_0\) is a closed subspace of locally convex space $Z$ and
\begin{equation*}
\Sigma = \bigl\{\Lambda \in Z^*: \Lambda(M_0) = \{0\}\bigr\}
\end{equation*}
then 
\begin{equation}  \label{eq:zm0:leq:sigma}
 \dim Z/M_0 \leq \dim \Sigma.
\end{equation}
We can strengthen this to  an equality.
The case of \(\dim Z/M_0 = \infty\)
becomes trivial with the note regarding \(\infty\)
preceding the proof of Theorem~4.25.
Let \(\dim Z/M_0 = k < \infty\) and 
\(\{z_j+M_0:j \in\N_k\}\) be a basis of \(Z/M_0\).
Define 
\begin{equation*}
M_j = \vspan(M_0 \cup \{z_h: 1\leq h \leq k \wedge h \neq j\}
\qquad \textnormal{for} j\in\N_k
\end{equation*}
and let \(\Lambda_j \in Z^*\) be functionals such that 
\(\Lambda_j(z_j)=1\) but \(\Lambda_j(M_j) = \{0\}\).
Pick arbitrary \(\Lambda \in \Sigma\).
Clearly
\begin{equation*}
\Lambda' = \sum_{j=1}^k (\Lambda z_j)\Lambda_j
\end{equation*}
agrees with \(\Lambda\) on the basis \(\{z_j: j\in\N_k\}\),
thus \(\Lambda = \Lambda'\).
Hence, any \(\Lambda \in \Sigma\) 
is a linear combination of \((\Lambda_j)_{j=1}^k\).
This gives the desired equality in \eqref{eq:zm0:leq:sigma}.


Apply with \(Z = Y\), \(M_0 = \scrR(T)\) which is given to be closed.
Also, \(\Sigma = \scrR(T)^\perp = \scrN(T^*)\) by Theorem~4.12.
Now we get 
\begin{equation} 
   \beta = \dim Y / \scrR(T) = \dim \Sigma = \dim\scrN(T^*) = \alpha^*.
\end{equation}

Next, apply with \(Z=X^*\) with its weak\upstar\ topology,
\(M_0=\scrR(T^*)\) which is closed by Theorem~4.14.
Now \(\Sigma\) consists of all weak\upstar-continuous functionals \(\Lambda^*\)
on \(X^*\) such that \(\Lambda^*(M_0) = \{0\}\).
Hence \(\Sigma = {}^\perp\scrR(T^*) = \scrN(T)\) by Theorem~4.12. 
Now we get
\begin{equation*}
   \beta^* = \dim X^* / \scrR(T^*) = \dim \Sigma = \dim \scrN(T) = \alpha.
\end{equation*}


%%%%%%% 13
\begin{excopy}
\begin{itemize}
\itemch{a}
Suppose \(T \in \BXY\), \(T_n \in \BXY\) for \(n =1,2,3,\ldots\), 
each \(T_n\) has finite-dimensional
 range, and \(\lim \|T - T_n\| = 0\). Prove that $T$ is compact.
\itemch{b}
Assume $Y$ is a Hilbert space, and prove the converse of \ich{a}: Every compact
\(T \in \BXY\) can be approximated in the operator norm by operators with 
finite-dimensional ranges.
\emph{Hint:} In a Hilbert space there are linear projections of norm $1$
onto any closed subspace. (See Theorems~5.16, 12.4.)
\end{itemize}
\end{excopy}

\begin{itemize}
\itemch{a}
The limit $T$ is in the norm-closure of the set of compact operators.
By Theorem~4.18\ich{c} it is a closed set, thus $T$ is compact.
\itemch{b}
Let \(T \in \BXY\) be a compact operator, so by definition
\(K = \overline{T(B_X(0;1)}\) is compact.
By Local Lemma~\ref{lem:compact:separable}
$K$ is separable. Statring from a countable dense subset of $K$,
using ortho-normalization process we generate a countable ortho-normal basis
\(\{v_j:j\in\N\}\) for \(\overline{T(X)}\).
Define the finite subsets \(V_1 = \{v_1\}\) and \(V_n = V_{n-1} \cup \{v_n\}\)
for \(n > 1\). 
We now define \(P_n:Y \to \vspan(V_n)\) to be the projection defined by
\begin{equation*}
P_n(y) = \sum_{j=1}^n \langle v_j, y\rangle \cdot v_j.
\end{equation*}
Put \(T_n = P_n \circ T\), and clearly \(P_n\) has a finite rank.
For each \(x\in X\) we have \(\lim_{n\to\infty}(T - T_n)x = 0\).
This convergence is in $K$, and by compactness it is a uniform 
on \(B_X(0;1)\), thus \(\lim_{n\to\infty}\|T-T_n\| = 0\).
\end{itemize}


%%%%%%% 14
\begin{excopy}
Define s shift operator S and a multiplication operator M on \(\ell^2\) by
\begin{align*}
(Sx)(n) &= 
\left\{
 \begin{array}{ll}
 0 & \qquad \textnormal{if}\; n = 0,\\
 x(n-1) & \qquad \textnormal{if}\; n \geq 1
 \end{array} 
\right.\\
(Mx)(n) &= (n+1)^{-1}x(n) \qquad \textnormal{if}\; n \geq 0.
\end{align*}
Put \(T = MS\). Show that $T$ is a compact operator which has no eigenvalue
 and whose
spectrum consists of exactly one point.
 Compute \(\|T^n\|\) for \(n=1,2,3,\ldots\), and compute
\(\lim_{n\to\infty} \|T^n\|^{1/n}\).
\end{excopy}

Let \(P_n:\ell^2\to\ell^2\) be projections defined by
\((P_n(v))(j) = v(j)\) if \(j\leq n\) and 
\((P_n(v))(j) = 0\) otherwise.
Clearly \(P_nT\) are compact operators.
Now
\begin{align*}
\|T-P_nT\| 
&= \sup\left\{\|(T-P_nT)v\|_2: \|v\|_2 \leq 1\right\}
 = \sup\left\{\sum_{j>n} |v(j-1)|^2/(j+1)^2|: \|v\|_2 \leq 1\right\} \\
&\leq (n+1)^{-2} \sup\left\{\sum_{j>n} |v(j-1)|^2: \|v\|_2 \leq 1\right\}
 \leq (n+1)^{-2}.
\end{align*}
Hence $T$ is compact.

Assume by negation \(\lambda \in \sigma(T)\setminus\{0\}\).
By Theorem~4.25\ich{b}
\(\lambda\in\C\) is an eigenvalue for $T$, 
with eigenvector \(v\in \ell^2\setminus\{0\}\).
Let \(k\in\Z^+\) be minimal such that \(v(k)\neq 0\).
Since \((T(v))(n) = 0/(n+1) = 0\) we get the contradiction \(\lambda = 0\).

Obviously \(e_0 = (1,0,0,\ldots) \notin (T-0\cdot I)(\ell^2)\).
Hence $0$ is the only point in the spectrum of $T$.

Clearly
\begin{equation} \label{eq:Tn:j}
\bigl(T^n(x)\bigr)(j) = \left\{
\begin{array}{ll}
0 \qquad &\textnormal{if}\; j < n \\
j!\cdot x(j - n)/(n + j + 1)! \qquad &\textnormal{if}\; j \geq n
\end{array}\right.
\end{equation}

In order to compute \(\|T^n\|_2\) we want to show that 
the actual \(\sup \{\|T^n x\|_2: \|x\|_2\leq 1\}\)
is achieved in \(e_0 = (1,0,0,\ldots)\).
By \eqref{eq:Tn:j} \(\|T^n(e_0)\|_2 = 1/n!\).
For any \(x \in \ell^2\) such that \(\|x\|_2\leq 1\) we have
\begin{equation*}
\|T^nx\|_2^2 
= \sum_{j=0}^\infty (|x(j)|/(j+n)!)^2 
\leq \sum_{j=0}^\infty (|x(j)|/n!)^2 
= (n!)^{-2}\sum_{j=0}^\infty |x(j)|^2 
= (\|x\|_2/n!)^2.
\end{equation*}
Hence \(\|T^n\|_2 = 1/n!\).

We will now show that 
\begin{equation} \label{eq:fact:roor:infty}
\lim_{n\to\infty} (n!)^{1/n} = \infty.
\end{equation}
Pick arbitrary \(M>0\), find some \(n<\infty\) such that \(\log n > 2M\).
Now
\begin{equation*}
\log ((2n)!)^{1/2n}
= (1/2n)\sum_{k=1}^{2n} \log k
\geq (1/2n)\sum_{k=n}^{2n} \log k
\geq (n/2n) \log n
= (\log n)/2 > M.
\end{equation*}
Since \(\log\) is strictly monotonic increasing, \eqref{eq:fact:roor:infty}
is true.
Hence
\begin{equation*}
\lim_{n\to\infty} \|T^n\|^{1/n}
= \lim_{n\to\infty} (n!)^{-1/n} = 0
= \lim_{n\to\infty} \exp(\log|1/(n!)|^{1/n} = 0
\end{equation*}


%%%%%%% 15
\begin{excopy}
Suppose \(\mu\) is a finite (or \(\sigma\)-finite) positive measure
on a measure space \(\Omega\), \(\mu\times\mu\) is the
corresponding product measure on \(\Omega\times\Omega\),
 and \(K \in L^2(\mu\times\mu)\). Define
\begin{equation*}
(Tf)(s) = \int_\Omega K(s,t)f(t)\,d\mu(t) \qquad [f \in L^2(\mu)].
\end{equation*}
\begin{itemize}
\itemch{a}
Prove that \(T \in \scrB(L^2(\mu))\) and that
\begin{equation*}
\|T\|^2 \leq \int_\Omega\int_\Omega |K(s,t)|^2\,d\mu(s)\,d\mu(t).
\end{equation*}

\itemch{b}
Suppose \(a_i\), \(b_i\) are members of \(L^2(\mu)\), for \(1 \leq i \leq n\),
 , put \(K_1(s,t) = \sum a_i(s)b_i(t)\), and
define \(T_1\) in terms of \(K_1\) as $T$ was defined in terms of $K$.
 Prove that \(\dim \scrR(T_1) \leq n\).

\itemch{c}
Deduce that $T$ is a compact operator on \(L^2(\mu)\).
\emph{Hint:} Use Exercise~13.

\itemch{d}
Suppose \(\lambda \in \C\), \(\lambda \neq 0\). Prove: Either the equation
\begin{equation*}
Tf - \lambda f = g
\end{equation*}
has a unique solution \(f \in L^2(\mu)\) for every \(g \in L^2(\mu)\)
 or there are inlinitely many
solutions for some $g$ and none for others.
 (This is known as the 
\index{Fredholm alternatives}
Fredholm alternatives.)

\itemch{e}
Describe the adjoint of $T$.
\end{itemize}
\end{excopy}

See \cite{Conway1990}~Proposition~4.7.
\begin{itemize}
\itemch{a}
Let \(f\in L^2(\mu)\),
then by Fubini \index{Fubini} Theorem~8.8 of \cite{RudinRCA80}
and Cauchy-Schwarz \index{Cauchy-Schwarz} inequality
(Theorem~3.5 in \cite{RudinRCA80}) 
\begin{align*}
\|Tf\|_2^2 
&= \int_\Omega\left|\int_\Omega K(s,t)f(t)\,d\mu(t)\right|^2\,d\mu(s) \\
&\leq \int_\Omega
   \left(
   \left(\int_\Omega |K(s,t)|^2\,d\mu(t)\right)
   \cdot
   \left(\int_\Omega |f(t)|^2\,d\mu(t)\right)
   \right)
   \,d\mu(s) \\
&= \|f\|_2^2 \cdot
   \int_\Omega\left(\int_\Omega |K(s,t)|^2\,d\mu(t)\right)\,d\mu(s) \\
&= \|f\|_2 \cdot \int_{\Omega\times\Omega} |K(s,t)|^2\,d(\mu\times\mu)(s,t)
 = \|K\|_2^2 \cdot \|f\|_2^2.
\end{align*}
\itemch{b}
Let \(f\in L^2(\mu)\), then
\begin{align*}
(T_1f)(s) 
&= \int_\Omega K_1(s,t)f(t)\,d\mu(t)
 = \int_\Omega \left(\sum_{i=1}^n a_i(s)b_i(t)\right)f(t)\,d\mu(t) \\
&= \left(\int_\Omega f(t)\left(\sum_{i=1}^n b_i(t)\right)\,d\mu(t)\right)
    \cdot a_i(s)
\end{align*}
Thus the range of \(T_1\) is spanned by \(\{a_i:i\in\N_n\}\). 
\itemch{c}
By Theorem~4.22 of \cite{RudinRCA80}, let \((e_j)_{j\in J}\) be 
an orthonormal basis for \(L^2(\mu)\).
We will show that \(\scrB = \{e_j(s)\cdot e_k(t): j,k\in J\}\)
is an orthonormal set % basis 
for \(L^2(\Omega\times\Omega,\mu\times\mu)\).

\textbf{Normality}:
\begin{align*}
\lrangle{e_j\cdot e_k, e_j\cdot e_k}
&= 
 \int_{\Omega\times\Omega} 
 \bigl(e_j(s)\cdot e_k(t)\bigr)
 \cdot
 \overline{\bigl(e_j(s)\cdot e_k(t)\bigr)}
 \,d(\mu\times\mu)(s,t) \\
&= \int_\Omega 
   e_k(t)\overline{e_k(t)}
   \left(\int_\Omega e_j(s)\overline{e_j(s)}\,d\mu(s)\right)
   \,d\mu(t) \\
&= \int_\Omega e_k(t)\overline{e_k(t)}\cdot 1\,d\mu(t)
 = 1
\end{align*}

\textbf{Orthogonality}:
\begin{align*}
\lrangle{e_j\cdot e_k, e_{j'}\cdot e_{k'}}
&= 
 \int_{\Omega\times\Omega} 
 \bigl(e_j(s)\cdot e_k(t)\bigr)
 \cdot
 \overline{\bigl(e_{j'}(s)\cdot e_{k'}(t)\bigr)}
 \,d(\mu\times\mu)(s,t) \\
&= \int_\Omega 
   \left(e_k(t)\cdot\overline{e_{k'}(t)}\right)
   \cdot \lrangle{e_j,e_{j'}}
   \,d\mu(t)
 = \lrangle{e_j,e_{j'}}\cdot \lrangle{e_k,e_{k'}}
\end{align*}
If \(j\neq j'\) or \(k\neq k'\) then the last product is clearly $0$.

\iffalse
\textbf{Basis.}
Pick arbitrary \(g \in L^2(\mu\times\mu)\) there are at most countable
functions \(f\in \scrB\) such that \(\lrangle{g,f} \neq 0\).
Denote this countable subset \(\scrD \subset \scrB\) and
put it in a sequence \((f_n)_{n\in\N}\)
where each is of the form \(f_j(s,t) = e_{j(n)}(s)\cdot e_{k(n)}(t)\).
Similarly, let \(P_N: L^2(\mu\times\mu) \to L^2(\mu\times\mu)\)
be the projection to the finite-dimensional subspace spanned by
\(\{f_n: n\leq N\}\). Now
\begin{align*}
\|g - P_N(g)\|_2^2
&= \int_{\Omega^2} |g(s,t) - \sum_{n=1}^N g(s,t)\cdot f_n(s,t)|^2\,d(\mu^2)(s,t)
\end{align*}
\fi

For any \(g \in L^2(\mu\times\mu)\) there are at most countable
functions \(h\in \scrB\) such that \(\lrangle{g,h} \neq 0\).
Thus we can use the presentation 
\(g(s,t) = \sum_{h\in\scrB} \lrangle{g,h}\cdot h(s,t)\).
In particular let \(\scrD \subset \scrB\) the countable
subset of functions $h$ such that  \(\lrangle{k,h}\neq 0\).
Put \scrD\ in a sequence \((h_n)_{n\in\N}\),
putting 
\begin{equation*}
h_n(s,t) = e_{j(n)}(s)\cdot e_{k(n)}(t)
\end{equation*}
and let 
\(P_N: L^2(\mu\times\mu) \to L^2(\mu\times\mu)\)
be the projection to the finite-dimensional subspace spanned by
\(\{h_n: n\leq N\}\).
Put 
\begin{equation*}
% K_n = TP_n + P_nT - P_n T P_n.
K_n = P_nT.
\end{equation*}
Clearly \(\dim K_n(L^2(\mu\times\mu)) < \infty\).
We will now show that 
\begin{equation} \label{eq:limKn:T}
\lim_{n\to\infty} K_n = T
\end{equation}
in the operator norm.
Pick \(f\in L^2(\mu)\) such that \(\|f\|_2 \leq 1\)
and put \(\alpha_j = \lrangle{f,e_j}\).
Thus \(f=\sum_{j\in J} \alpha_j\cdot e_j\)
<and \(\|f\|_2^2 = \sum_{j\in J} |\alpha_j|^2 \leq 1\).
Now by the orthogonality of \scrB
\begin{align}
\|T-K_n\|^2 
&\leq 
 \|Tf - K_n f\|_2^2
 = \sum_{j\in J} |\lrangle{Tf - K_nf, e_j}|^2  \notag \\
&= \sum_{j\in J} 
   \left|
     \lrbangle{(T - K_n)
       \left(\sum_{k\in J}\alpha_k\cdot e_k\right), e_j}
   \right|^2 \notag \\
&\leq \label{eq:ex4.15:csineq}
 \sum_{j\in J} 
   \left(\sum_{j\in J} |\alpha_j|^2\right)
   \cdot
   \left(\sum_{j\in J} \left|\lrbangle{(T - K_n)e_k, e_j}\right|^2 \right) \\
\iffalse
&\leq \sum_{j\in J} \sum_{k\in J} 
 \left|
  \lrbangle{Te_k,ej} 
  - \lrbangle{TP_ne_k,e_j} - \lrbangle{P_nTe_k,e_j} + \lrbangle{P_nTP_ne_k,e_j}
 \right|^2 \notag
\fi
&\leq \sum_{j\in J} \sum_{k\in J} 
 \left|  \lrbangle{Te_k,e_j} - \lrbangle{P_nTe_k,e_j} \right|^2 \notag
= \sum_{p=n+1}^\infty
 \left|  \lrbangle{Te_{k(p)},e_{j(p)}} \right|^2 \notag
\end{align}
where the Cauchy-Schwarz inequality (Theorem~3.5 \cite{RudinRCA80})
gives the inequality of~\eqref{eq:ex4.15:csineq}.
Clearly the last expression is smaller than any \(\epsilon>0\)
for sufficiently large~$n$.
Thus \eqref{eq:limKn:T} is shown.
By Exercise~13\ich{a} $T$ is a compact operator.

\itemch{d}
This claim actually holds for every compact operator on a Banach space.
If \(\lambda \notin \sigma(T)\) then \(T -\lambda I\) is invertible
and thus the equation has a unique solutions for every \(g\in L^2(\mu)\).

Assume now \(\lambda \in \sigma(T)\). By Theorem~4.25
\begin{equation*}
\dim \scrN(T-\lambda I) = \dim\left(L^2(\mu)/\scrR(T-\lambda I)\right) > 0.
\end{equation*}
If \(g \in \scrR(T-\lambda I)\) then 
there exist some solution $f$ to the equation and
so are \(\{f+h: h \in \scrN(T-\lambda I)\}\).
Otethrwise  \(g \notin \scrR(T-\lambda I)\) 
and there is no solution to the equation.

\itemch{e}
The Hilbert-adjoint of $T$ is defined (like transpose conjugate matrix) by
\begin{equation*}
(T^*f)(s) = \int_\Omega \overline{K(t,s)}f(t)\,d\mu(t) \qquad [f \in L^2(\mu)].
\end{equation*}

\end{itemize}


%%%%%%% 16
\begin{excopy}
Define 
\begin{equation*}
K(s,t) = 
\left\{
\begin{array}{ll}
(1-s)t \qquad & \textnormal{if}\; 0 \leq t \leq s \\
(1-t)s \qquad & \textnormal{if}\; s \leq t \leq 1
\end{array}
\right.
\end{equation*}
and define \(T \in \scrB(L^2(0,1))\) by
\begin{equation*}
(Tf)(s) = \int_0^1 K(s,t)f(t)\,dt \qquad (0 \leq s \leq 1)
\end{equation*}
\begin{itemize}

\itemch{a}
Show that the eigenvalues of $T$ are \((n\pi)^{-2}\), \(n=1,2,3,\ldots\),
 that the corresponding
eigenfunctions are \(\sin n\pi x\), and that each eigenspace is one-dimensional.
\emph{Hint:} If
\(\lambda \neq 0\), the equation \(Tf = \lambda f\) implies that $f$ is
 infinitely differentiable, that \(\lambda f'' + f = 0\),
and that \(f(0) = f(1) = 0\).
 The case \(\lambda = 0\) can be treated separately.

\itemch{b}
Show that the above eigenfunctions form an orthogonal basis for \(L^2(0, 1)\).

\itemch{c}
Suppose \(g(t) = \sum c_n \sin n \pi t\).
 Discuss the equation \(Tf - \lambda f = g\).

\itemch{d}
Show that $T$ is also compact operator on $C$, the space of all continuous
functions
on \([0,1]\). 
 \emph{Hint:}~If \(\{f_i\}\) is uniformly bounded,
then \(\{Tf_i\}\) is equicontinuous.

\end{itemize}
\end{excopy}

\begin{itemize}
\itemch{a}
Compute
\begin{align*}
(Tf)(s)
&= \int_0^1 K(s,t)f(t)\,dt
 = \int_0^s K(s,t)f(t)\,dt + \int_s^1 K(s,t)f(t)\,dt \\
&= \int_0^s (1-s)t\cdot f(t)\,dt + \int_s^1 (1-t)s\cdot f(t)\,dt \\
&= (1-s)\int_0^s t\cdot f(t)\,dt + s\int_s^1 (1-t)\cdot f(t)\,dt \\
&= \int_0^s t\cdot f(t)\,dt + s\int_s^1 f(t)\,dt -s \int_0^1  t\cdot f(t)\,dt\\
\end{align*}
Clearly: \(Tf)(0) = (Tf)(1) = \).
Differentiate:
\begin{align*}
(Tf)'(s)
&= -\int_0^s t\cdot f(t)\,dt + (1-s)s\cdot f(s) + 
   \int_s^1 (1-t)\cdot f(t)\,dt - s(1-s)\cdot f(s) \\
&=  - \int_0^s t\cdot f(t)\,dt + \int_s^1 (1-t)\cdot f(t)\,dt
\end{align*}
Hence
\begin{equation*}
% (Tf)''(s) = -f(s).
(Tf)''(s) = -s\cdot f(s) - (1 - s)\cdot f(s) = -f(s)
\end{equation*}
% \frac{d}{dt}(t\sin n\pi t) = \sin n\pi t + n\pi t \cos n\pi t
% \frac{d}{dt}(t\sin n\pi t - (\cos n\pi t / n\pi) =  n\pi t \cos n\pi
% t
Thus if \(T f = 0\) then \((T f)'' = 0 = -f\) so \(\Ker(T) = \{0\}\).

Now if \(T f = \lambda f\) then \(\lambda f'' + f = 0\).

Note that
\begin{equation*}
\int t\cos(n\pi t)\,dt 
 = \frac{1}{n\pi} t\sin(n\pi t) + \frac{1}{(n\pi)^2} \cos(n\pi t)
\end{equation*}

\iffalse
If \(\kappa(t) = \cos(n\pi t)\) then
\begin{eqnarray*}
(T \kappa)(s) 
&=& \int_0^s t\cdot \cos(n \pi t)\,dt + s\int_s^1 \cos(n \pi t)\,dt 
   -s \int_0^1  t\cdot \cos(n \pi t)\,dt \\
&=& \left.\left(
  \frac{1}{n\pi} t\sin(n\pi t) + \frac{1}{(n\pi)^2} \cos(n\pi t)
  \right)\right\rvert_0^s \\
& & + \frac{s}{n\pi}\left.\left(
       \sin(n \pi t)
      \right)\right\rvert_s^1 \\
& & + s\left.\left(
  \frac{1}{n\pi} t\sin(n\pi t) + \frac{1}{(n\pi)^2} \cos(n\pi t)
  \right)\right\rvert_0^1 \\
&=& \frac{1}{(n\pi)^2}(\cos(n \pi s) - 1) 
    + \frac{s}{n\pi} \sin(n \pi s)
    + s\left( \frac{1}{n\pi} s\sin(n\pi)
    + \frac{1}{(n\pi)^2}(\cos(n\pi s) - 1)\right)
\end{eqnarray*}
\fi 
\unfinished

\itemch{b}
Let \(m>n>0\). Since \(\sin(a)\sin(b) = \left(\cos(a-b)
- \cos(a+b)\right)/2\)
we have
\begin{eqnarray*}
2\int_0^1 \sin(m\pi t) \sin(n\pi t)\,dt 
&=& \int_0^1 \cos((m-n)\pi t)\,dt - \int_0^1 \cos((m+n)\pi t)\,dt \\
&=& \frac{1}{m-n}\left (\sin((m-n)\pi 1) -  \sin((m-n)\pi 0)\right) - \\
& & \frac{1}{m+n}\left (\sin((m+n)\pi 1) -  \sin((m+n)\pi 0)\right)
\end{eqnarray*}

\itemch{c} Apply (\emph{d}) of previous exercise.

\itemch{d}
Assuming the supremum norm on $C$
and \(\{f_i\}\) are uniformly bounded by $M$.
% Pick arbitrary \(\epsilon>0\),
Let \(s_1 \leq s_2 \in [0,1]\).
\begin{eqnarray*}
\left|(Tf)(s_1) - (Tf)(s_2)\right|
 &\leq& \int_0^{s_1}\left|K(s_1,t) - K(s_2,t)\right|\cdot M\,dt \\
 &    & + \int_{s_1}^{s_2} \left|K(s_1,t) - K(s_2,t)\right|\cdot M\,dt \\
 &    & + \int_{s_2}^1 \left|K(s_1,t) - K(s_2,t)\right|\cdot M\,dt \\
 &\leq& \int_0^{s_1}(s_2 - s_1) M\,dt
         + \int_{s_1}^{s_2}  M\,dt 
         + \int_{s_2}^1 (s_2 - s_1) M\,dt \\
 &\leq& (s_2 - s_1) M.
\end{eqnarray*}
Thus \(\{Tf_i\}\) are equicontinuous.
Enumerate \(\Q\cap[0,1]\) by \((q_n)_{n\in\Z^+}\).
Starting with  \(\frakF_0=\{Tf_i\}\),
by induction pick repatedly subsequence \(\frakF_n\)
of \(\frakF_{n-1}\) such that  \(\frakF_{n-1}\) converges
pointwise on \(q_n\).
By diagonaly picking the $n$-th function of \(\frakF_n\)
we get a subsequence of \(\frakF_0\) that converges
on \(\Q\cap[0,1]\). Since they are equicontinuous, the convergence
is on the whole \([0,1]\).
Thus $T$ is compact By local lemma \ref{lem:compact:op}.
\end{itemize}

%%%%%%% 17
\begin{excopy}
If \(L^2 = L^2(0,\infty)\) relative to Lebesgue measure, and if
\begin{equation*}
(Tf)(s) = \frac{1}{s} \int_0^s f(t)\,dt \qquad (0 < s < \infty),
\end{equation*}
prove that \(T \in \scrB(L^2)\) and that $T$ is not compact.
 (The fact that \(\|T\| \leq 2\) is a special case
of \index{Hardy} Hardy`s inequality. See p.~72 of~[23].)
\end{excopy}

For \(L^p\), Hardy inequality is \(\|Tf\|_p \leq (p/(p-1))\|f\|_p\).
Applying for \(p=2\) gives \(\|T\| \leq 2\) thus $T$ is bounded.

For \(a > 0\), define
\begin{equation*}
f_a(x) = \left\{
  \begin{array}{ll}
  a^{-1/2} \quad & x \leq a \\
  0        & x > a
  \end{array}
  \right.
\end{equation*}
Now
\begin{equation*}
\|f_a\|+2^2 = \int_0^\infty (f_a(x))^2\,dx 
  = a\cdot \left(a^{-1/2}\right)^2 = 1.
\end{equation*}
If \(a < b\) then
\begin{equation*}
\|(T f_a) - (T f_b)\|_2^2
  = \int_0^\infty \left|(T f_a)(x) - (T f_b)(x)\right|^2\,dx
  \geq a \left(a^{-1/2} - b^{-1/2}\right)^2 
\end{equation*}

We set \(\alpha = (1 - 2^{-1/2})^{-2} > 1\) and if \(b \geq \alpha a\)
then \(\left(a^{-1/2} - b^{-1/2}\right)^2 = 1/2\).
If \(g_n = f_{\alpha^{n-1}}\)
 then \(\|(T g_m) - T g_n\|_2 \geq 2^{-1/2}\) whenever \(m \neq n\)
and \((T g_n)\) has no converging subsequence. Thus $T$ is not compact.

%%%%%%% 18
\begin{excopy}
Prove the following statements.

\begin{itemize}

\itemch{a}
If \(\{x_n\}\) is a weakly convergent sequence in $X$ then \(\{\|x_n\|\}\)
 is bounded.

\itemch{b}
If \(T \in \BXY\) and \(x_n \to x\) weakly, then \(Tx_n \to Tx\) weakly.

\itemch{c}
If \(T \in \BXY\), if \(x_n \to x\) weakly, and if $T$ is compact,
 then \(\|Tx_n - Tx\| \to 0\).

\itemch{d}
Conversely, if $X$ ts reflexive, if \(T \in \BXY\),
 and if \(\|Tx_n - Tx\| \to 0\) whenever
\(x_n \to x\) weakly, then $T$ is compact.
\emph{Hint:}~Use \ich{c} of Exercise~1, and part \ich{c} of Exercise~28
 in Chapter~3.

\itemch{e}
If $X$ is reflexive and \(T \in \BXY\), then $T$ is compact.
 Hence \(\scrR(T) \neq \ellone\).
\emph{Hint:}~Use \ich{c} of Exercise~5 of Chapter~3.

\itemch{f}
If $Y$ is reflexive and \(T\in \scrB(c_0, Y)\), then $T$ is compact.

\end{itemize}

\end{excopy}

\begin{itemize}

\itemch{a}
Viewing \(x_n \in X^{**}\).
For all \(\Lambda \in X^*\) we have
\begin{equation*}
x_n(\Lambda) = \Lambda(x_n) \stackrel{n}{\rightarrow} 0.
\end{equation*}
Thus 
\begin{equation*}
c_\Lambda = \sup_n \left|x_n(\Lambda)\right| < \infty.
\end{equation*}
By the uniform bound principle (Theorem~2.6)
\((x_n\) are equicontinuous in \(X^{**}\), thus \((\|x_n\|)\) are bounded.

\itemch{b}

\itemch{c}

\itemch{d}

\itemch{e}

\itemch{f}

\end{itemize}


\unfinished

%%%%%%% 19
\begin{excopy}
Suppose $Y$ is a closed subspace of $X$, and \(x_0^* \in X^*\), Put
\begin{align*}
\mu &= \sup\{|\langle x, x_0^*\rangle|: x\in Y, \|x\| \leq 1\}, \\
\delta &= \inf\{\| x -  x_0^*\|: x^*\in Y^\perp\}
\end{align*}
In other words, \(\mu\) is the norm of the restriction of \(x_0^*\) to $X$
 and \(\delta\) is the distance from
\(x_0^*\) to the annihilator of $Y$. Prove that \(\mu = \delta\).
 Prove also that \(\delta = \|x^* - x_0^*\|\) for at
least one \(x^* \in Y^\perp \).
\end{excopy}


\unfinished

%%%%%%% 20
\begin{excopy}
Extend Sections~4.6 to 4.9 to locally convex spaces.
 (The word ``isometric'' must of
course be deleted from statement of Theorem~4.9.)
\end{excopy}


\unfinished

%%%%%%% 21
\begin{excopy}
Let $B$ and \(B^*\) be the closed unit balls in $X$ and \(X^*\), respectively.
 The following is a
converse to the Banach-Alaoglu theorem:
\textsl{If $E$ is a convex set in \(X^*\) such that \(E \cap rB^*)\)
is weak\upstar-compact
 for every \(r > 0\),
 then $E$ is
 weak\upstar-closed}.
 (Corollary: A subspace of \(X^*\)
is weak\upstar-closed if and only if its intersection with \(B^*\)
 is weak\upstar-compact.)

Complete the following outline of the proof.

\begin{itemize}

\itemch{i}
$E$ is norm-closed.

\itemch{ii}
Associated to each \(F \subset X\) its polar
\begin{equation*}
P(F) = \{x^*: |\langle x, x^* \rangle| \leq 1 \;\textnormal{for all}\; x\in F\}.
\end{equation*}
The intersection of all sets \(P(F)\), as $F$ ranges over the collection of all
 finite subsets
of \(r^{-1}B\), is exactly \(rB^*\).

\itemch{iii}
The theorem is a consequence of the following proposition:
\textsl{
 If, in addition to the
stated hypotheses, \(E \cap B^* = \emptyset\), there exists \(x \in X\)
 such that \(\Re \lrangle{x, x^*} \geq 1\) for
every \(x^* \in E\).
}

\itemch{iv}
Proof of the proposition: Put \(F_0 = \{0\}\).
 Assume finite sets \(F_0,\ldots,F_{k-1}\) have been
chosen so that \(iF_i \subset B\) and so that
\begin{equation} \label{eq:BA-conv:1}
P(F_0) \cap \cdots \cap P(F_{k-1}) \cap E \cap kB^* = \emptyset.
\end{equation}
Note that \eqref{eq:BA-conv:1} is true for \(k = 1\). Put
\begin{equation*}
Q = P(F_0) \cap \cdots \cap P(F_{k-1}) \cap E \cap (k+1)B^* 
\end{equation*}
If \(P(F) \cap Q \neq \emptyset\) for every finite set
 \(F \subset k^{-1}B\), the weak\upstar-compactness of $Q$,
together with \ich{ii}, implies that \((kB^*) \cap Q \neq \emptyset\),
 which contradicts \eqref{eq:BA-conv:1}. Hence there
is a finite set \(F_k \subset k^{-1}B\) such that \eqref{eq:BA-conv:1} holds
 with \(k+1\) in place of $k$. The construction
can thus proceed. It yields
\begin{equation} \label{eq:BA-conv:2}
E \cap \bigcap_{k=1}^\infty P(F_k) = \emptyset.
\end{equation}
Arrange the members \(\cup F_k\) in a sequence \(\{x_n\}\).
 Then \(x_n\| \to 0\). Define \(T: X^* \to c_0\)
by
\begin{equation*}
Tx^* = \{\lrangle{x_n,x^*}\}.
\end{equation*}
Then \(T(E)\) is a convex subset of \(c_0\). By \eqref{eq:BA-conv:2}
for every \(x^* \in E\). Hence there is a scalar sequence \(\{\alpha_n\}\),
 with \(\sum |\alpha_n| < \infty\), such that
\begin{equation*}
\Re \sum_{n=1}^\infty \alpha_n \lrangle{x_n,x^*} \geq 1
\end{equation*}
for every \(x^* \in E\). To complete the proof, put \(x = \sum \alpha_n x_n\).

\end{itemize}
\end{excopy}



\unfinished

%%%%%%% 22
\begin{excopy}
Suppose \(T \in \scrB(X)\), $T$ is compact, \(\lambda \neq 0\),
 and \(S = T - \lambda I\).
\begin{itemize}

\itemch{a}
If \(\scrN(S^n) = \scrN(S^{n+1})\) for some  nonnegative integer $n$,
 prove that \(\scrN(S^n) = \scrN(S^{n+k})\)
for \(k=1,2,3,\ldots\).

\itemch{b}
Prove that \ich{a} must happen for some $n$. 
\emph{Hint:}: Consider the proof of Theorem~4.24.)

\itemch{c}
Let $n$ be the smallest nonnegative integer for which \ich{a} holds.
 Prove that
\begin{equation*}
X = \scrN(S^n) \oplus \scrR(S^n),
\end{equation*}
and that the restriction of $S$ to \(\scrR(S^n)\) is a one-to-one mapping
 of \(\scrR(S^n)\) onto \(\scrR(S^n)\).

\end{itemize}

\end{excopy}

\begin{itemize}

\itemch{a}

\itemch{b}

\itemch{c}

\end{itemize}



\unfinished

%%%%%%% 23
\begin{excopy}
Suppose \(\{x_n\}\) is sequence in a Banach space $X$, and
\begin{equation*}
 \sum_{n=1}^\infty \|x_n\| = M < \infty.
\end{equation*}
Prove that the series \(\sum x_n\) converges to some \(x \in X\).
 Explicitly, prove that
\begin{equation*}
\lim_{n\to\infty} \|x - (x_1 + \cdots + x_n)\| = 0.
\end{equation*}
Prove also that \(\|x\| \leq M\).
 (These facts were used in the proof of Lemma~4.13.)
\end{excopy}


\unfinished

%%%%%%% 24
\begin{excopy}
Let $c$ be the space of all complex sequences
\begin{equation*}
x = \{x_1,x_2,x_3,\ldots\}
\end{equation*}
for which \(x_\infty = \lim x_n\) exists (in \C).
 Put \(\|x\| = \sup |x_n|\). Let \(c_0\) be the subspace of $c$ that
consists of all $x$ with \(x_\infty = 0\).
\begin{itemize}

\itemch{a}
Describe explicitly two isomeric isomorphisms $u$ and $v$,
 such that $u$ maps \(c^*\) unto \(\ellone\)
and $v$ maps \(c_0^*\) onto \ellone.

\itemch{b}
Define \(S: c_0 \to c\) by \(Sf = f\) Describe the operator
\(vS^*u^{-1}\) that maps \ellone\ to \ellone.

\itemch{c}
Define \(T: c \to c_0\) by setting
\begin{equation*}
 y_1 = x_\infty,\quad y_{n+1} = x_n - x_\infty \qquad \textnormal{if}\; n \geq 1.
\end{equation*}
Prove that $T$ is one-to—one and that \(Tc = c_0\).
 Find \(\|T\|\) and \(\|T^{-1}\|\). Describe the
operator \(uT^*v\) that maps \ellone\ to \ellone.
\end{itemize}
\end{excopy}

\begin{itemize}

\itemch{a}

\itemch{b}

\itemch{c}

\end{itemize}


\unfinished

%%%%%%% 25
\begin{excopy}
If \(T \in \BXY\) and \(\scrR(T^*) = \scrN(T)^\perp\),
prove that \(\scrR(T)\) is closed.
\end{excopy}

\unfinished

%%%%%%% 26
\begin{excopy}
Assume \(T \in \BXY\) and \(T(X)=Y\). Show that there exists \(\delta>0\)
such that \(S(X)=Y\) for all \(S\in\BXY\) with \(\|S-T\|<\delta\).
\end{excopy}

%%%%%%% 27
\begin{excopy}
Suppose \(T \in \scrB(X)\). Prove that \(\lambda \in \sigma(T)\) if and only if
there is a sequence \(\{x_n\}\) in~$X$, \(\|x_n\|=1\), for which 
\begin{equation*}
\lim_{n\to\infty} \|T x_n - \lambda x_n\| = 0.
\end{equation*}
[Thus every \(\lambda \in \sigma(T)\) which is not an eigenvalue of $T$
is an ``approximate'' eigenvalue.]
\end{excopy}

\unfinished

%%%%%%%%%%%%%%%
\end{enumerate}
%%%%%%%%%%%%%%%



