\documentclass{article}

% \usepackage{underscore}
\makeatletter
\catcode`\_\active
\DeclareRobustCommand{\myunderscore}{\ifmmode\sb\else
  \leavevmode\nobreak\hskip\z@skip
  \_\-\nobreak\hskip\z@skip \fi}
\let_\myunderscore
\makeatother


\usepackage{amssymb}
\usepackage{geometry}
\geometry{left=1in, right=1in, top=1in, bottom=1in}
\usepackage{amsmath}

\newtheorem{thm}{Theorem}[section]
\newtheorem{conj}{Conjecture}[section]

\setlength{\parindent}{0pt}
\setlength{\parskip}{8pt plus 1pt minus 1pt}

\title{Conic behavior of Norm}
\author{Yotam Medini \\ \texttt{yotam.medini@gmail.com}}

\begin{document}

\maketitle

Given a finite dimensional Banach space $X$ (not necessarily Hilbert
space), for any proper subspace \(Y \subsetneq X\),
it is easy to find a unit vector
\(u \in X\setminus Y\) such that \(\|u\| = d(u,Y) = 1\).  It is
natural to ask whether there could be some other natural restrictions
that could lead to some kind of uniqueness of such $u$.  One can think
of ``choosing'' such $u$ by requiring some symmetrical ``conic-like''
behavior of its distance with vectors in $Y$.
In some sense this could generalize the notion of orthogonality.

Is the following true? Is there a counterexample?
\begin{conj}
Let $X$ be an \((n+1)\)-dimensional Banach space and
\(Y\subset X\) an $n$-dimensional subspace.
There exists \(x\in X\) such that
\begin{gather}
\|x\| = 1  \label{eq:normx1} \\
\forall y\in Y,\; \|x+y\| = \|x-y\| \geq 1. \label{eq:xpy:xmy:geq1}
\end{gather}
\textnormal{(Uniqueness:)}
If \(x'\) satisfies \eqref{eq:normx1} and \eqref{eq:xpy:xmy:geq1}
then \(x' = \pm x\).
\end{conj}

Note: The equality in \eqref{eq:xpy:xmy:geq1} is the main claim of the
conjecture.  The inequality also gives \(d(x,Y)=1\).

\end{document}
