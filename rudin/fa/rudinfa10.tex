%%%%%%%%%%%%%%%%%%%%%%%%%%%%%%%%%%%%%%%%%%%%%%%%%%%%%%%%%%%%%%%%%%%%%%%%
%%%%%%%%%%%%%%%%%%%%%%%%%%%%%%%%%%%%%%%%%%%%%%%%%%%%%%%%%%%%%%%%%%%%%%%%
%%%%%%%%%%%%%%%%%%%%%%%%%%%%%%%%%%%%%%%%%%%%%%%%%%%%%%%%%%%%%%%%%%%%%%%%
\chapterTypeout{Banach Algebras}

%%%%%%%%%%%%%%%%%%%%%%%%%%%%%%%%%%%%%%%%%%%%%%%%%%%%%%%%%%%%%%%%%%%%%%%%
%%%%%%%%%%%%%%%%%%%%%%%%%%%%%%%%%%%%%%%%%%%%%%%%%%%%%%%%%%%%%%%%%%%%%%%%
\section{Exercises} % pages 271-274

Throughout this set of exercises,
$A$ denotes a Banach algebra and \(x, y, \ldots\) denote
elements of $A$, unless the contrary is stated.

%%%%%%%%%%%%%%%%%
\begin{enumerate}
%%%%%%%%%%%%%%%%%

%%%%%%% 1
\begin{excopy}
Use the identity \((xy)^n = x(yx)^{n-1}y\) to prove that \(xy\) and \(yx\)
always have the same spectral radius.
\end{excopy}

\begin{align*}
 \rho(xy) &= \lim_{n\to\infty} \|(xy)^n\|^{1/n} = \lim_{n\to\infty}\|x(yx)^{n-1}y\|^{1/n} \\
  &\leq \lim_{n\to\infty} \left(\|x\|\cdot \|(yx)^{n-1}\|\cdot \|y\|\right)^{1/n} \\
  &= \left(\lim_{n\to\infty} \|x\|^{1/n}\right)
     \cdot
     \left(\lim_{n\to\infty} \|(yx)^{n-1}\|^{1/n}\right)
     \cdot
     \left(\lim_{n\to\infty} \|y\|^{1/n}\right) \\
  &= 1 \cdot \left(\lim_{n\to\infty} \|(yx)^{n-1}\|^{1/n}\right) \cdot 1
   = \lim_{n\to\infty} \|(yx)^{n-1}\|^{1/n} \\
  &= \lim_{n\to\infty} \left(\|(yx)^{n-1}\|^{1/(n-1)}\right)^{n/n-1} \\
  &= \lim_{n\to\infty} \left(\rho(yx)\right)^{n/(n-1)} = \rho(yx)
\end{align*}

%%%%%%% 2
\begin{excopy}
\begin{itemize}
  \itemch{a}
    If $x$ and \(xy\) are invertible in $A$, prove that $y$ is invertible.
  \itemch{b}
    If \(xy\) and \(yx\) are invertible in $A$,
    prove that $x$ and $y$ are invertible. (The
    commutative case of this was tacitly used in the proofs of Theorems~10.13
    and 10.28.)
  \itemch{c}
    Show that it is possible to have \(xy = e\) but \(yx \neq e\).
    For example, consider the
    right and left shifts \(S_R\), and \(S_L\),
    defined on some Banach space of functions $f$
    on the nonnegative integers by
    \begin{align*}
     (S_R f)(n) &= f(n - 1) \qquad \textnormal{if}\quad n \geq 1,\\
     (S_R f)(0) &= 0, \\
     (S_L f)(0) &= f(n + 1) \qquad \textnormal{for all}\quad n \geq 0.
    \end{align*}
  \itemch{d}
    If \(xy = e\) and \(yx = z \neq e\),
    show that $z$ is a nontrivial idempotent. (This
    means that \(z^2 = z\), \(z \neq 0\), \(z \neq e\).)
\end{itemize}
\end{excopy}

\begin{itemize}
  \itemch{a}
    Let \(z = (xy)^{-1} x\). Now using associativity
    \begin{equation*}
      zy = ((xy)^{-1} x)y = (xy)^{-1}(xy) = e
    \end{equation*}
    and 
    \begin{align*}
      yz &= y((xy)^{-1} x) = (x^{-1} x)(y((xy)^{-1} x) = x^{-1}(xy)(xy)^{-1} x \\
         &= x^{-1}((xy)(xy)^{-1})x = x^{-1}ex = e.
    \end{align*}
    Hence \(z = y^{-1}\).
  \itemch{b}
    We have two sides ``inverses'':
    \(((yx)^{-1}y)\cdot x = e\)
    and
    \(x\cdot(y(xy)^{-1}) = e\).
    Put \(a = (yx)^{-1}y\) and \(b = y(xy)^{-1}\)
    so \(ax = xb = e\). Now
    \(a = a(xb) = (ax)b = b\)
    Thus \(x^{-1} = a = b\).
    By symmetry $y$ is invertible as well.
  \itemch{c}
    Indeed \(S_L\cdot S_R = e\) but \(S_R\cdot S_L \neq e\).
  \itemch{d}
    Showing idempotent: \(z^2 = (yx)(yx) = y(xy)x = yex = yx = z\).
    If by negation \(z=0\) then \(e = xy = x(yx)y = 0\) contradiction.
\end{itemize}

%%%%%%% 3
\begin{excopy}
Prove that every finite-dimensional $A$ is isomorphic to an algebra of matrices.
\emph{Hint}:
The proof of Theorem~10.2 shows that every $A$ is isomorphic to
a sub-algebra of \(\scrB(A)\).
Conclude that \(xy = e\) implies \(yx = e\) if \(\dim A < \infty\).
\end{excopy}

Let \seq{a}{n} be a base for $A$.
Using the unique linear combination of the base,
let \(\beta: A \to \C^n\) defined such that
\begin{equation*}
 \forall x\in A, \qquad
 x = \sum\nolimits_{j=1}^n \left(\beta(x)\right)_j a_j.
\end{equation*}
Clearly \(\beta\) is a linear operator.
Any operator in \(\scrB(A)\) is determined by its action on the base.
\iffalse
For any \(j,k\in\N_n\) let \(b_{j,k}\) be defined such that
\begin{equation*}
a_j\cdot a_k = \sum\nolimits_{k=1}^n b_{j,k} a_k.
\end{equation*}
\fi
Let \(\rho,x \in A\), and \(\rho\cdot x = y\).
Let see how \(\beta(x)\) is \(\rho\)-mapped to \(\beta(y)\).
\begin{equation*}
\rho\cdot x =
  \left(\sum\nolimits_{j=1}^n \left(\beta(\rho)\right)_j a_j\right)
  \cdot
  \left(\sum\nolimits_{j=1}^n \left(\beta(x)\right)_j a_j\right)
  =
  \left(\sum\nolimits_{j=1}^n \left(\beta(y)\right)_j a_j\right)
\end{equation*}
Let \(r_{j,k} = \left(\beta(\rho(a_j)\right)_k\) where \(j,k\in\N_n\).
Let \(R_\rho = (r_{j,k})_{j,k=1}^n\) be a \(n\times n\) matrix.
we need to show the following equality in \(\C^n\)
\begin{equation*}
R_\rho\cdot \beta(x) = \beta(y).
\end{equation*}
By linearity, it is sufficient to show the equality for \(x=a_j\).
For each \(k\in\N_n\)
\begin{equation*}
\left(R_\rho\cdot \beta(a_j) \right)_k = r_{j,k}.
\end{equation*}
Thus we have an isomorphism \(A \leftrightarrow \C^{n^2}\).
From the theory of matrices, we know that a matrix $M$ is
invertible iff \(\det(M)\neq 0\).

Now say the isomorphism maps \(x\to M_1\) and \(y\to M_2\).
If \(M_1\times M_2 = I_n\) as \(n\times n\) matrices,
then \(\det(M_1) \neq 0 \neq \det(M_2)\).
Thus both matrices are invertible, the inverse is unique
and so  \(M_2\times M_1 = I_n\) as well.
So if \(xy=e\) then \(yx=e\).

%%%%%%% 4
\begin{excopy}
\begin{itemize}
\itemch{a}
Prove that \(e - yx\) is invertible in $A$
if \(e - xy\) is invertible. \emph{Suggestion:} Put
\(z =(e-xy)^{-1}\), write $z$ as a geometric series
(assume \(\|x\| < 1\), \(\|y\| < 1\)), and
use the identity stated in Exercise~1
to obtain a finite formula for \((e - yx)^{-1}\)
in terms of $x$, $y$, $z$. Then show that this formula works without any
restrictions on \(\|x\|\) or \(\|y\|\).
\itemch{b}
If \(\lambda \in \C\) \(\lambda \neq 0\),
  and \(\lambda \in \sigma(xy)\), prove that \(\lambda \in \sigma(yx)\).
  Thus \(\sigma(xy) \cup \{0\} = \sigma(yx) \cup \{0\}\).
  Show, however, that \(\sigma(xy)\) is not always equal to \(\sigma(yx)\).
\end{itemize}
\end{excopy}

\begin{itemize}
\itemch{a}
As geometric series
\begin{equation*}
z = \sum\nolimits_{k=0}^\infty (xy)^k
\end{equation*}
Now
\begin{equation*}
yzx
= \sum\nolimits_{k=0}^\infty y(xy)^kx
= \sum\nolimits_{k=1}^\infty (yx)^k
\end{equation*}
so as a conjecture
\begin{equation} \label{eq:conj:inv:emyx}
e + yzx = \sum\nolimits_{k=0}^\infty (yx)^k = (e - yx)^{-1}.
\end{equation}
Checking from Left
\begin{align*}
(e + yzx)\cdot(e - yx)
 &= e - yx + yzx - yzxyx
  = e + y(z - zxy - e)x = e + y(z(e - xy) - e)x \\
  = e + y\cdot 0\cdot x = e
\end{align*}
and checking from right
\begin{align*}
(e - yx)\cdot(e + yzx)
 &= e + yzx - yx - yxyzx = e + y(z - xyz - e)x \\
 &= e + y((e - xy)z - e)x
    = e + e\cdot 0\cdot x = e
\end{align*}
So the \eqref{eq:conj:inv:emyx} conjecture holds.
\itemch{b}
\end{itemize}

\unfinished

%%%%%%% 5
\begin{excopy}
Let \(A_0\) and \(A_1\),
be the algebras of all complex 2-by-2 matrices of the form

\begin{equation*}
\left(
  \begin{array}{ll}
   \alpha & 0 \\
   0 & \beta
  \end{array}
\right),
\qquad
\left(
  \begin{array}{ll}
   \alpha & \beta \\
   0 & \alpha
  \end{array}
\right).
\end{equation*}
Prove that every two-dimensional complex algebra $A$ with unit $e$ is isomorphic
to one of these
  and that \(A_0\) is not isomorphic to \(A_1\).
    \emph{Hint:} Show that $A$ has a
basis \(\{e, a\}\) in which \(a^2 = \lambda e\) for some \(\lambda \in \C\).
  Distinguish between the cases
  \(\lambda=0\), \(\lambda \neq 0\).
  Show that there exists a three-dimensional noncommutative
Banach algebra.
\end{excopy}

\unfinished

%%%%%%% nn
\begin{excopy}
\end{excopy}

\unfinished

%%%%%%%%%%%%%%%
\end{enumerate}
%%%%%%%%%%%%%%%
