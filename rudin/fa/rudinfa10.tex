%%%%%%%%%%%%%%%%%%%%%%%%%%%%%%%%%%%%%%%%%%%%%%%%%%%%%%%%%%%%%%%%%%%%%%%%
%%%%%%%%%%%%%%%%%%%%%%%%%%%%%%%%%%%%%%%%%%%%%%%%%%%%%%%%%%%%%%%%%%%%%%%%
%%%%%%%%%%%%%%%%%%%%%%%%%%%%%%%%%%%%%%%%%%%%%%%%%%%%%%%%%%%%%%%%%%%%%%%%
\chapterTypeout{Banach Algebras}

%%%%%%%%%%%%%%%%%%%%%%%%%%%%%%%%%%%%%%%%%%%%%%%%%%%%%%%%%%%%%%%%%%%%%%%%
%%%%%%%%%%%%%%%%%%%%%%%%%%%%%%%%%%%%%%%%%%%%%%%%%%%%%%%%%%%%%%%%%%%%%%%%
\section{Exercises} % pages 271-274

Throughout this set of exercises,
$A$ denotes a Banach algebra and \(x, y, \ldots\) denote
elements of $A$, unless the contrary is stated.

%%%%%%%%%%%%%%%%%
\begin{enumerate}
%%%%%%%%%%%%%%%%%

%%%%%%% 1
\begin{excopy}
Use the identity \((xy)^n = x(yx)^{n-1}y\) to prove that \(xy\) and \(yx\)
always have the same spectral radius.
\end{excopy}

\begin{align*}
 \rho(xy) &= \lim_{n\to\infty} \|(xy)^n\|^{1/n} = \lim_{n\to\infty}\|x(yx)^{n-1}y\|^{1/n} \\
  &\leq \lim_{n\to\infty} \left(\|x\|\cdot \|(yx)^{n-1}\|\cdot \|y\|\right)^{1/n} \\
  &= \left(\lim_{n\to\infty} \|x\|^{1/n}\right)
     \cdot
     \left(\lim_{n\to\infty} \|(yx)^{n-1}\|^{1/n}\right)
     \cdot
     \left(\lim_{n\to\infty} \|y\|^{1/n}\right) \\
  &= 1 \cdot \left(\lim_{n\to\infty} \|(yx)^{n-1}\|^{1/n}\right) \cdot 1
   = \lim_{n\to\infty} \|(yx)^{n-1}\|^{1/n} \\
  &= \lim_{n\to\infty} \left(\|(yx)^{n-1}\|^{1/(n-1)}\right)^{n/n-1} \\
  &= \lim_{n\to\infty} \left(\rho(yx)\right)^{n/(n-1)} = \rho(yx)
\end{align*}

%%%%%%% 2
\begin{excopy}
\begin{itemize}
  \itemch{a}
    If $x$ and \(xy\) are invertible in $A$, prove that $y$ is invertible.
  \itemch{b}
    If \(xy\) and \(yx\) are invertible in $A$,
    prove that $x$ and $y$ are invertible. (The
    commutative case of this was tacitly used in the proofs of Theorems~10.13
    and 10.28.)
  \itemch{c}
    Show that it is possible to have \(xy = e\) but \(yx \neq e\).
    For example, consider the
    right and left shifts \(S_R\), and \(S_L\),
    defined on some Banach space of functions $f$
    on the nonnegative integers by
    \begin{align*}
     (S_R f)(n) &= f(n - 1) \qquad \textnormal{if}\quad n \geq 1,\\
     (S_R f)(0) &= 0, \\
     (S_L f)(0) &= f(n + 1) \qquad \textnormal{for all}\quad n \geq 0.
    \end{align*}
  \itemch{d}
    If \(xy = e\) and \(yx = z \neq e\),
    show that $z$ is a nontrivial idempotent. (This
    means that \(z^2 = z\), \(z \neq 0\), \(z \neq 0\).)
\end{itemize}
\end{excopy}

\unfinished

%%%%%%%%%%%%%%%
\end{enumerate}
%%%%%%%%%%%%%%%
