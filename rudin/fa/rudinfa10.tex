%%%%%%%%%%%%%%%%%%%%%%%%%%%%%%%%%%%%%%%%%%%%%%%%%%%%%%%%%%%%%%%%%%%%%%%%
%%%%%%%%%%%%%%%%%%%%%%%%%%%%%%%%%%%%%%%%%%%%%%%%%%%%%%%%%%%%%%%%%%%%%%%%
%%%%%%%%%%%%%%%%%%%%%%%%%%%%%%%%%%%%%%%%%%%%%%%%%%%%%%%%%%%%%%%%%%%%%%%%
\chapterTypeout{Banach Algebras}

%%%%%%%%%%%%%%%%%%%%%%%%%%%%%%%%%%%%%%%%%%%%%%%%%%%%%%%%%%%%%%%%%%%%%%%%
%%%%%%%%%%%%%%%%%%%%%%%%%%%%%%%%%%%%%%%%%%%%%%%%%%%%%%%%%%%%%%%%%%%%%%%%
\section{Exercises} % pages 271-274

%%%%%%%%%%%%%%%%%
\begin{enumerate}
%%%%%%%%%%%%%%%%%

%%%%%%% 1
\begin{excopy}
Use the identity \((xy)^n = x(yx)^{n-1}y\) to prove that \(xy\) and \(yx\)
always have the same spectral radius.
\end{excopy}

\begin{align*}
 \rho(xy) &= \lim_{n\to\infty} \|(xy)^n\|^{1/n} = \lim_{n\to\infty}\|x(yx)^{n-1}y\|^{1/n} \\
  &\leq \lim_{n\to\infty} \left(\|x\|\cdot \|(yx)^{n-1}\|\cdot \|y\|\right)^{1/n} \\
  &= \left(\lim_{n\to\infty} \|x\|^{1/n}\right)
     \cdot
     \left(\lim_{n\to\infty} \|(yx)^{n-1}\|^{1/n}\right)
     \cdot
     \left(\lim_{n\to\infty} \|y\|^{1/n}\right) \\
  &= 1 \cdot \left(\lim_{n\to\infty} \|(yx)^{n-1}\|^{1/n}\right) \cdot 1
   = \lim_{n\to\infty} \|(yx)^{n-1}\|^{1/n} \\
  &= \lim_{n\to\infty} \left(\|(yx)^{n-1}\|^{1/(n-1)}\right)^{n/n-1} \\
  &= \lim_{n\to\infty} \left(\rho(yx)\right)^{n/(n-1)} = \rho(yx)
\end{align*}

%%%%%%%%%%%%%%%
\end{enumerate}
%%%%%%%%%%%%%%%
