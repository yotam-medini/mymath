%%%%%%%%%%%%%%%%%%%%%%%%%%%%%%%%%%%%%%%%%%%%%%%%%%%%%%%%%%%%%%%%%%%%%%%%
%%%%%%%%%%%%%%%%%%%%%%%%%%%%%%%%%%%%%%%%%%%%%%%%%%%%%%%%%%%%%%%%%%%%%%%%
%%%%%%%%%%%%%%%%%%%%%%%%%%%%%%%%%%%%%%%%%%%%%%%%%%%%%%%%%%%%%%%%%%%%%%%%
\chapterTypeout{Banach Algebras}

%%%%%%%%%%%%%%%%%%%%%%%%%%%%%%%%%%%%%%%%%%%%%%%%%%%%%%%%%%%%%%%%%%%%%%%%
%%%%%%%%%%%%%%%%%%%%%%%%%%%%%%%%%%%%%%%%%%%%%%%%%%%%%%%%%%%%%%%%%%%%%%%%
\section{Exercises} % pages 271-274

Throughout this set of exercises,
$A$ denotes a Banach algebra and \(x, y, \ldots\) denote
elements of $A$, unless the contrary is stated.

%%%%%%%%%%%%%%%%%
\begin{enumerate}
%%%%%%%%%%%%%%%%%

%%%%%%% 1
\begin{excopy}
Use the identity \((xy)^n = x(yx)^{n-1}y\) to prove that \(xy\) and \(yx\)
always have the same spectral radius.
\end{excopy}

\begin{align*}
 \rho(xy) &= \lim_{n\to\infty} \|(xy)^n\|^{1/n} = \lim_{n\to\infty}\|x(yx)^{n-1}y\|^{1/n} \\
  &\leq \lim_{n\to\infty} \left(\|x\|\cdot \|(yx)^{n-1}\|\cdot \|y\|\right)^{1/n} \\
  &= \left(\lim_{n\to\infty} \|x\|^{1/n}\right)
     \cdot
     \left(\lim_{n\to\infty} \|(yx)^{n-1}\|^{1/n}\right)
     \cdot
     \left(\lim_{n\to\infty} \|y\|^{1/n}\right) \\
  &= 1 \cdot \left(\lim_{n\to\infty} \|(yx)^{n-1}\|^{1/n}\right) \cdot 1
   = \lim_{n\to\infty} \|(yx)^{n-1}\|^{1/n} \\
  &= \lim_{n\to\infty} \left(\|(yx)^{n-1}\|^{1/(n-1)}\right)^{n/n-1} \\
  &= \lim_{n\to\infty} \left(\rho(yx)\right)^{n/(n-1)} = \rho(yx)
\end{align*}

%%%%%%% 2
\begin{excopy}
\begin{itemize}
  \itemch{a}
    If $x$ and \(xy\) are invertible in $A$, prove that $y$ is invertible.
  \itemch{b}
    If \(xy\) and \(yx\) are invertible in $A$,
    prove that $x$ and $y$ are invertible. (The
    commutative case of this was tacitly used in the proofs of Theorems~10.13
    and 10.28.)
  \itemch{c}
    Show that it is possible to have \(xy = e\) but \(yx \neq e\).
    For example, consider the
    right and left shifts \(S_R\), and \(S_L\),
    defined on some Banach space of functions $f$
    on the nonnegative integers by
    \begin{align*}
     (S_R f)(n) &= f(n - 1) \qquad \textnormal{if}\quad n \geq 1,\\
     (S_R f)(0) &= 0, \\
     (S_L f)(0) &= f(n + 1) \qquad \textnormal{for all}\quad n \geq 0.
    \end{align*}
  \itemch{d}
    If \(xy = e\) and \(yx = z \neq e\),
    show that $z$ is a nontrivial idempotent. (This
    means that \(z^2 = z\), \(z \neq 0\), \(z \neq e\).)
\end{itemize}
\end{excopy}

\begin{itemize}
  \itemch{a}
    Let \(z = (xy)^{-1} x\). Now using associativity
    \begin{equation*}
      zy = ((xy)^{-1} x)y = (xy)^{-1}(xy) = e
    \end{equation*}
    and 
    \begin{align*}
      yz &= y((xy)^{-1} x) = (x^{-1} x)(y((xy)^{-1} x) = x^{-1}(xy)(xy)^{-1} x \\
         &= x^{-1}((xy)(xy)^{-1})x = x^{-1}ex = e.
    \end{align*}
    Hence \(z = y^{-1}\).
  \itemch{b}
    We have two sides ``inverses'':
    \(((yx)^{-1}y)\cdot x = e\)
    and
    \(x\cdot(y(xy)^{-1}) = e\).
    Put \(a = (yx)^{-1}y\) and \(b = y(xy)^{-1}\)
    so \(ax = xb = e\). Now
    \(a = a(xb) = (ax)b = b\)
    Thus \(x^{-1} = a = b\).
    By symmetry $y$ is invertible as well.
  \itemch{c}
    Indeed \(S_L\cdot S_R = e\) but \(S_R\cdot S_L \neq e\).
  \itemch{d}
    Showing idempotent: \(z^2 = (yx)(yx) = y(xy)x = yex = yx = z\).
    If by negation \(z=0\) then \(e = xy = x(yx)y = 0\) contradiction.
\end{itemize}

%%%%%%% 3
\begin{excopy}
Prove that every finite-dimensional $A$ is isomorphic to an algebra of matrices.
\emph{Hint}:
The proof of Theorem~10.2 shows that every $A$ is isomorphic to
a sub-algebra of \(\scrB(A)\).
Conclude that \(xy = e\) implies \(yx = e\) if \(\dim A < \infty\).
\end{excopy}

Let \seq{a}{n} be a base for $A$.
Using the unique linear combination of the base,
let \(\beta: A \to \C^n\) defined such that
\begin{equation*}
 \forall x\in A, \qquad
 x = \sum\nolimits_{j=1}^n \left(\beta(x)\right)_j a_j.
\end{equation*}
Clearly \(\beta\) is a linear operator.
Any operator in \(\scrB(A)\) is determined by its action on the base.
\iffalse
For any \(j,k\in\N_n\) let \(b_{j,k}\) be defined such that
\begin{equation*}
a_j\cdot a_k = \sum\nolimits_{k=1}^n b_{j,k} a_k.
\end{equation*}
\fi
Let \(\rho,x \in A\), and \(\rho\cdot x = y\).
Let see how \(\beta(x)\) is \(\rho\)-mapped to \(\beta(y)\).
\begin{equation*}
\rho\cdot x =
  \left(\sum\nolimits_{j=1}^n \left(\beta(\rho)\right)_j a_j\right)
  \cdot
  \left(\sum\nolimits_{j=1}^n \left(\beta(x)\right)_j a_j\right)
  =
  \left(\sum\nolimits_{j=1}^n \left(\beta(y)\right)_j a_j\right)
\end{equation*}
Let \(r_{j,k} = \left(\beta(\rho(a_j)\right)_k\) where \(j,k\in\N_n\).
Let \(R_\rho = (r_{j,k})_{j,k=1}^n\) be a \(n\times n\) matrix.
we need to show the following equality in \(\C^n\)
\begin{equation*}
R_\rho\cdot \beta(x) = \beta(y).
\end{equation*}
By linearity, it is sufficient to show the equality for \(x=a_j\).
For each \(k\in\N_n\)
\begin{equation*}
\left(R_\rho\cdot \beta(a_j) \right)_k = r_{j,k}.
\end{equation*}
Thus we have an isomorphism \(A \leftrightarrow \C^{n^2}\).
From the theory of matrices, we know that a matrix $M$ is
invertible iff \(\det(M)\neq 0\).

Now say the isomorphism maps \(x\to M_1\) and \(y\to M_2\).
If \(M_1\times M_2 = I_n\) as \(n\times n\) matrices,
then \(\det(M_1) \neq 0 \neq \det(M_2)\).
Thus both matrices are invertible, the inverse is unique
and so  \(M_2\times M_1 = I_n\) as well.
So if \(xy=e\) then \(yx=e\).

%%%%%%% 4
\begin{excopy}
\begin{itemize}
\itemch{a}
Prove that \(e - yx\) is invertible in $A$
if \(e - xy\) is invertible. \emph{Suggestion:} Put
\(z =(e-xy)^{-1}\), write $z$ as a geometric series
(assume \(\|x\| < 1\), \(\|y\| < 1\)), and
use the identity stated in Exercise~1
to obtain a finite formula for \((e - yx)^{-1}\)
in terms of $x$, $y$, $z$. Then show that this formula works without any
restrictions on \(\|x\|\) or \(\|y\|\).
\itemch{b}
If \(\lambda \in \C\) \(\lambda \neq 0\),
  and \(\lambda \in \sigma(xy)\), prove that \(\lambda \in \sigma(yx)\).
  Thus \(\sigma(xy) \cup \{0\} = \sigma(yx) \cup \{0\}\).
  Show, however, that \(\sigma(xy)\) is not always equal to \(\sigma(yx)\).
\end{itemize}
\end{excopy}

\begin{itemize}
\itemch{a}
As geometric series
\begin{equation*}
z = \sum\nolimits_{k=0}^\infty (xy)^k
\end{equation*}
Now
\begin{equation*}
yzx
= \sum\nolimits_{k=0}^\infty y(xy)^kx
= \sum\nolimits_{k=1}^\infty (yx)^k
\end{equation*}
so as a conjecture
\begin{equation} \label{eq:conj:inv:emyx}
e + yzx = \sum\nolimits_{k=0}^\infty (yx)^k = (e - yx)^{-1}.
\end{equation}
Checking from left
\begin{align*}
(e + yzx)\cdot(e - yx)
 &= e - yx + yzx - yzxyx
  = e + y(z - zxy - e)x \\
 &= e + y(z(e - xy) - e)x 
  = e + y\cdot 0\cdot x = e
\end{align*}
and checking from right
\begin{align*}
(e - yx)\cdot(e + yzx)
 &= e + yzx - yx - yxyzx = e + y(z - xyz - e)x \\
 &= e + y((e - xy)z - e)x
    = e + e\cdot 0\cdot x = e
\end{align*}
So the \eqref{eq:conj:inv:emyx} conjecture holds.
\itemch{b}
Let \(\mu = \lambda^{-1}\).
Note that \(\mu (xy) = (\mu x)y = x(\mu y)\)
and \(\mu (yx) = (\mu y)x = y(\mu x)\).
By assumption \(xy -\lambda e\) is not invertible
and so \(\mu xy - e\) is not invertible.
By \ich{a} \(\mu yx - e\) is not invertible
and so \(yx - \lambda e\) is not invertible. Thus \(\lambda \in \sigma(yx)\).

Using exercise~2.\ich{c},
We have \(S_L\cdot S_R = e\) and \(S_R\cdot S_L \neq e\)
and for any \(x\in\ell^p\) we have \(((S_R\cdot S_L)(x))(0) = 0\)
which shows that \(S_R\cdot S_L\) is not invertible.
Thus \(0 \in \sigma(S_R\cdot S_L) \setminus  \sigma(S_L\cdot S_R) \),
and \(\sigma(S_R\cdot S_L) \neq  \sigma(S_L\cdot S_R)\).
\end{itemize}

%%%%%%% 5
\begin{excopy}
Let \(A_0\) and \(A_1\),
be the algebras of all complex 2-by-2 matrices of the form

\begin{equation*}
\left(
  \begin{array}{ll}
   \alpha & 0 \\
   0 & \beta
  \end{array}
\right),
\qquad
\left(
  \begin{array}{ll}
   \alpha & \beta \\
   0 & \alpha
  \end{array}
\right).
\end{equation*}
Prove that every two-dimensional complex algebra $A$ with unit $e$ is isomorphic
to one of these
  and that \(A_0\) is not isomorphic to \(A_1\).
    \emph{Hint:} Show that $A$ has a
basis \(\{e, a\}\) in which \(a^2 = \lambda e\) for some \(\lambda \in \C\).
  Distinguish between the cases
  \(\lambda=0\), \(\lambda \neq 0\).
  Show that there exists a three-dimensional noncommutative
Banach algebra.
\end{excopy}

We have \(a\in A_0\) such that \(\{I, a\}\) is a base and
\(a^2 = \lambda e\) with \(\lambda \neq 0\).
for example: \(a = \left(
\begin{smallmatrix}
1 & 0 \\
0 & -1
\end{smallmatrix}
\right)\).
But by negation if we have such an \(a\in A_1\) we would have
\begin{equation*}
a^2 =
\left(
\begin{array}{ll}
\alpha & \beta \\
0 & \alpha \\
\end{array}\right)^2 =
\left(
\begin{array}{ll}
\alpha^2 & 2\alpha\beta \\
0 & \alpha^2 \\
\end{array}\right) = \lambda I
\end{equation*}
But then \(\alpha=0\) or \(\beta=0\) and in both cases \(\{I, a\}\)
would not span $A$, contradicting the base assumption.
This \(A_0\) and \(A_1\) are not isomorphic.

Let \(\{e, b\}\) be an arbitrary base for 2-dimensional $A$.
Look at a linear combination \(b^2 = \alpha e + \beta b\)
where \(\alpha,\beta \in \C\).
Consider
\((b + ze)^2 = (\alpha + z^2) e + (\beta + 2z)b\),
so we pick \(z = -\beta/2\), \(a = b + ze\) and \(\lambda = \alpha + z^2\)
so we have (as the hint) a base \(\{e, a\}\) and \(a^2 = \lambda e\).
If \(\lambda \neq 0\) we have isomorphism with \(A_0\)
\begin{equation*}
  \alpha e + \beta a \to
  \left(
  \begin{array}{ll}
  \alpha + \beta & 0 \\
  0 & \alpha - \beta
  \end{array}
  \right).
\end{equation*}
If \(\lambda = 0\) we have isomorphism with \(A_1\)
\begin{equation*}
  \alpha e + \beta a \to
  \left(
  \begin{array}{ll}
  \alpha & \beta \\
  0 & \alpha
  \end{array}
  \right).
\end{equation*}

Look at the $3$-dimensional algebra of the \(2\times 2\)
complex matrices of the form
\(\left(
\begin{smallmatrix}
\alpha & \beta \\
0 & \gamma
\end{smallmatrix}
\right)\).
Now
\begin{equation*}
\left(
  \begin{array}{ll}
   \alpha_1 & \beta_1 \\
   0 & \gamma_1
  \end{array}
\right)
\cdot
\left(
  \begin{array}{ll}
   \alpha_2 & \beta_2 \\
   0 & \gamma_2
  \end{array}
\right)
=
\left(
  \begin{array}{ll}
   \alpha_1\alpha_2 & \alpha_1\beta_2 + \beta_1\gamma_2 \\
   0 & \gamma_1\gamma_2
  \end{array}
\right).
\end{equation*}
and
\begin{equation*}
\left(
  \begin{array}{ll}
   \alpha_2 & \beta_2 \\
   0 & \gamma_2
  \end{array}
\right)
\cdot
\left(
  \begin{array}{ll}
   \alpha_1 & \beta_1 \\
   0 & \gamma_1
  \end{array}
\right)
=
\left(
  \begin{array}{ll}
   \alpha_2\alpha_1 & \alpha_2\beta_1 + \beta_2\gamma_1 \\
   0 & \gamma_2\gamma_1
  \end{array}
\right).
\end{equation*}
So if we pick \(\alpha_1 = \alpha_2 = \beta_1 = \gamma_2 = 0\)
and \(\beta_2 = \gamma_1 = 1\) we see that this algebra is noncommutative.

%%%%%%% 6
\begin{excopy}
Let $A$ be the algebra of all complex functions $f$ on
\(\{1, 2, 3, \ldots\}\) which are $0$
except at finitely many points, with pointwise addition and multiplication and
norm
\begin{equation*}
 \|f\| = \sum\nolimits_{k=1}^\infty k^{-2}|f(k)|.
\end{equation*}
Show that multiplication is left-continuous
  (hence also right-continuous, since $A$
is commutative) but not jointly continuous. (Adjunction of a unit element, as
suggested in Section~10.1, would make no difference.)
  Show, in fact, that there is
a~sequence \(\{f_n\}\) in $A$ so that \(\|f_n\| \to 0\)
  but \(\|f_n^2\| \to \infty\), as \(n \to \infty\).
\end{excopy}

The continuity of single side is trivial
since in the limit of
\(\lim_{n\to\infty} f_n\cdot g\)
or
\(\lim_{n\to\infty} g\cdot f_n\)
the g function is constant
and the formal infinite sum is actually a finite sum,
since $g$ is $0$ except or finite set of points.
The finite sum of limits is clearly the limit of the sums.

Let \(h(n) = (1/\left(\log(n))^{1/4}\right)\)
Clearly
Define \(f_n : \N \to \C\) by
\begin{equation*}
f_n(k) = \left\{
\begin{array}{ll}
h(n)\cdot k^{2/3} & \quad \textnormal{if}\; n > 1 \land 0 < k \leq n \\
0 & \quad \textnormal{if}\; n = 1 \lor k > n 
\end{array}\right.
\end{equation*}
\begin{equation*}
h(n)\cdot\|f_n\|
  = \sum\nolimits_{k=2}^n k^{-2+2/3}\
  \leq \sum\nolimits_{k=2}^\infty k^{-4/3}
  \leq \int\nolimits_0^\infty x^{-4/3}\,dx
  = -1/(-4/3 + 1) = 3.
\end{equation*}
Thus \(\lim_{n\to\infty} \|f_n\| = 0\).
But
\begin{align*}
\lim_{n\to\infty} \|f_n^2\|
 &= \lim_{n\to\infty} \sum\nolimits_{k=1}^n (h(n))^2\cdot k^{-2+4/3}
   = \lim_{n\to\infty} h(n)^2\cdot \sum\nolimits_{k=1}^n k^{-2/3} \\
 & \geq \lim_{n\to\infty} \left(\log(n)\right)^{-1/2}
     \cdot  \sum\nolimits_{k=1}^\infty k^{-1}
   \geq \lim_{n\to\infty} \left(\log(n)\right)^{-1/2} \log(n) \\
 &= \lim_{n\to\infty} \left(\log(n)\right)^{1/2} = \infty.
\end{align*}

%%%%%%% 7
\begin{excopy}
Let \(C^2 = C^2([0, 1))\) be the space of all complex functions
  on \([0, 1]\) whose
second derivative is continuous. Choose \(a > 0\), \(b > 0\), and define
\begin{equation*}
\|f\| = \|f\|_\infty + a\|f'\|_\infty +  b\|f''\|_\infty +
\end{equation*}
Show that this makes \(C^2\) into a Banach space,
  for every choice of $a$, $b$, but that
the Banach algebra axioms hold
 if and only if \(2b \leq a^2\).
 (For necessity, consider $x$ and \(x^2\).)
\end{excopy}

Assume the defined \(\|\|\) is an algebra norm.
Let \(f(x)=x\). So \((f^2)'(x) = 2x\) and
\hbox{\((f^2)''(x) = 2\)}.
We need \(\|f\cdot f\| \leq \|f\|\cdot\|f\|\) to hold.
\begin{align*}
1 + a\cdot 2 + b\cdot 2 &\leq (1 + a\cdot 1 + b\cdot 0)^2 \\
\Rightarrow 2a + 2b + 1 &\leq (a + 1)^2 \\
\Rightarrow 2b &\leq a^2.
\end{align*}

Conversely, assume \(2b \leq a^2\).
Pick arbitrary \(f,g\in C^2\).
\begin{equation*}
\newcommand{\IN}[1]{\|{#1}\|_\infty}
\arraycolsep=1.4pt
\def\arraystretch{1.6}
\begin{array}{lcl}
\|fg\|
 &=& \IN{fg} + a\IN{f'g + fg'} +  b\IN{f''g + 2f'g' + fg''} \\
 &\leq&  \IN{f}\cdot\IN{g} +
         a\left(\IN{f'}\cdot\IN{g} + \IN{f}\cdot\IN{g'}\right) + \\
 &&      b\left(\IN{f''}\cdot\IN{g} + \IN{f}\cdot\IN{g''}\right) +
         2b\IN{f'}\cdot\IN{g'} \\
 &\leq& \IN{f}\cdot\IN{g} +
         a\left(\IN{f'}\cdot\IN{g} + \IN{f}\cdot\IN{g'}\right) + \\
 &&      b\left(\IN{f''}\cdot\IN{g} + \IN{f}\cdot\IN{g''}\right) +
         a^2\IN{f'}\cdot\IN{g'} \\
 &\leq& \IN{f}\cdot\IN{g} +
        a\left(\IN{f}\cdot\IN{g'} + \IN{f'}\cdot\IN{g}\right) + \\
 &&     b\left(\IN{f}\cdot\IN{g''} + \IN{f''}\cdot\IN{g}\right) +
        a^2\IN{f''}\IN{g''} +
        ab\IN{f'}\IN{g''} \\
 &=& (\IN{f} + a\IN{f'} +  b\IN{f''})
     \cdot
     (\IN{g} + a\IN{g'} +  b\IN{g''}) \\
 &=& \IN{f}\cdot\IN{g}.
\end{array}
\end{equation*}
So the inequality of norm required for algebra holds.

%%%%%%% 8
\begin{excopy}
What happens if the process of adjoining a unit (described in Section~10.1) is
applied to an algebra $A$ which already has a unit? Clearly,
  the result cannot be
an algebra $A$, with two units, Explain.
\end{excopy}

Following the Section~10.1 notation.
If the original algebra $A$ has a unit \(e_A\)
then it is mapped to \((e_A,0) \in A_1\)
which is \emph{not} a unit in \(A_1\).
For example:
\begin{equation*}
  (e_A,0)\cdot(0, 1) = (e_A,0) \neq (0, 1).
\end{equation*}

%%%%%%% 9
\begin{excopy}
Suppose that \(\Omega\) is open in \(\C\)
and that \(f: \Omega \to A\)
and \(\varphi: \Omega \to \C\) are holomorphic.
Prove that \(\varphi f : \Omega \to A\) is holomorphic.
[This was used in the proof of Theorem~10.13,
  with \(\phi(\lambda) = \lambda^n\).]
\end{excopy}

[See \cite{RudinPMA85} Theorem~5.3]
If \(\varphi f \) is holomorphic, then
\((\varphi f)' = \varphi'f + \varphi f'\).
Using the continuity of multiplication with scalar
\begin{align*}
 \lim_{w\to z} \frac{\varphi(w)f(w) - \varphi(z)f(z)}{w - z}
 &= \lim_{w\to z} \frac{
   \varphi(w)(f(w) - f(z)) + (\varphi(w) - \varphi(z))f(z)}{w - z} \\
 &= \left(\lim_{w\to z} \varphi(w)\frac{f(w) - f(z)}{w - z}\right) +
    \left(\lim_{w\to z} \frac{\varphi(w) - \varphi(z)}{w - z}\right)f(z) \\
 &= \varphi(z)f'(z) + \varphi'(z)+f(z).
\end{align*}
and the limits exist.

%%%%%%% 10
\begin{excopy}
Another proof that \(\sigma(x) \neq \emptyset\)
  can be based on Liouville’s theorem~3.32 and the
fact that \((\lambda e - x)^{-1} \to 0\) as \(\lambda \to \infty\).
  Complete the details.
\end{excopy}

If \(x = \lambda e\) for some \(\lambda \in C\) then
clearly this \(\lambda \in \sigma(x)\).

So we can assume that \(x \neq \lambda e\) for all  \(\lambda \in C\).
If by negation  \(\sigma(x) = \emptyset\) then
and \(f(\lambda) = ((\lambda e - x)^{-1}\) is entire holomorphic
function (using proof similar to that of Theorem~5.3 \cite{RudinPMA85}.
Clearly \(\lim_{\lambda\to\infty}f(\lambda) = 0\).
Hence there exists \(r>0\) such that \(|f(z)| < 1\) whenever \(|z|>r\)
and \(|f|\) achieves a maximum in the \(B(0,r)\) ball
and so $f$ is bounded. By Liouville’s theorem~3.32 $f$ is constant
which is a contrdiction (\(|f(0)| > 0\) and it vanishes at infinity).

%%%%%%% 11
\begin{excopy}
Call \(x \in A\) a \emph{topological divisor of zero}
if there is a sequence \(\{y_n\}\) in $A$, with
\(\|y_n\| = 1\), such that
\begin{equation*}
\lim_{n\to\infty} xy_n, = 0 = \lim_{n\to\infty} y_nx.
\end{equation*}
\begin{itemize}
\itemch{a}
Prove that every boundary point $x$
  of the set of all invertible elements of $A$ is
a topological divisor of zeroq.
  \emph{Hint:} Take \(y_n = x_n^{-1} /\|x_n^{-1}\|\), where \(x_n \to x\).
\itemch{b}
In which Banach algebras is $0$ the only topological divisor of $0$?
\end{itemize}
\end{excopy}

\begin{itemize}
\itemch{a}
  \newcommand{\limn}{\lim_{n\to\infty}}
  % using
  % mathonline.wikidot.com/boundary-points-of-inv-a-are-topological-divisors-of-zero-in
  If \(x=0\in A\) then it is trivially topological divisor of $0$.
  So we may assume \(x\neq 0\).
  Let \(G(A)\) be the group of invertible elements of $A$.
  By Theorem~10.11 \(G(A)\) is open, hence \(x\in \boundary{G(A)}\)
  is not invertible.
  Let \((x_n)_{n\in\N}\) be a sequence in \(G(A)\)
  such that \(\limn x_n = x\).

  Assume by negation \(\|x_n^{-1}\|\neq M < \infty\) is bounded.
  \begin{equation*}
  \|a_m^{-1} - a_n^{-1}\|
  = \|a_m^{-1}(a_n - a_m)a_n^{-1}\|
  \leq 2M\|a_n - a_m\|.
  \end{equation*}
  Since \((x_n)_{n\in\N}\) is a Cauchy sequence, so is \((x_n^{-1}\)
  that converges to \(y = \limn a_n^{-1}\) since $A$ is
  topologically complete. Since the multiplication is continuous,
  \begin{equation*}
  xy = \left(\limn x_n\right)\cdot\left(\lim nx_n\right)
  = \limn x_n\cdot x_n^{-1} = 1.
  \end{equation*}
  A contradiction to $X$ not being invertible.
  Thus  \((\|x_n^{-1})\|)\) not bounded.

  By moving to a subsequence  \((x_n)_{n\in\N}\),
  we may now assume that \(\|x_n^{-1}\| \geq n\).
  Let \(z_n = x_n^{-1}/\|x_n^{-1}\|\) so \(\|z_n\|=1\) for all \(n\in\N\).
  Now, from right
  \begin{align*}
  \limn \|xz_n\|
    &= \limn \|(x - x_n)z_n + x_n z_n\| \\
    &\leq \limn \|(x - x_n\| + \limn \|x_n \cdot x_n^{-1}/\|x_n{-1}\|
    = 0 + 0 = 0.
  \end{align*}
  And from left
  \begin{align*}
  \limn \|z_n x\|
    &= \limn \|z_n(x - x_n) + z_n x_n\| \\
    &\leq \limn \|(x - x_n\| + \limn \|x_n^{-1} \cdot x_n/\|x_n{-1}\|
    = 0 + 0 = 0.
  \end{align*}
  Thus $x$ is a topological divisor of $0$.
\itemch{b}
  From Theorem~10.11 the invertible elements \(G(A)\) form an open set.
  From \ich{a} such algebras must have all its nonzero elements be invertible.
  Fron Theorem~10.14 these are isomorphic to \(\C\).
\end{itemize}

%%%%%%% 12
\begin{excopy}
Find the spectrum of the operator \(T \in \scrB(\ell^2)\) given by
\begin{equation*}
T(x_1,x_2,x_3,x_4,\ldots) = (-x_2,x_1,-x_4,x_3,\ldots).
\end{equation*}
\end{excopy}
It is sufficient to look on \((x_1,x_2)\)
and restricy $T$ to \(\C^2\). Thus $T$ can be represented as
\(\left(
\begin{smallmatrix}
0 & -1 \\
1 & 0
\end{smallmatrix}
\right)\). It characteristic polynomial is \(p_T(x) = x^2 + 1\)
whose roots are \(i\) and \(-i\).
Indeed \(T \pm iI\) are non invertible with two non empty kernels.
One kernel
consisting of all \((-x_2,x_1,-x_4,x_3,\ldots)\)
such that \(x_{2n} = ix_{2n-1}\) for all \(n\in\N\).
A Similar kernel, such that \(x_{2n} = -ix_{2n-1}\) for all \(n\in\N\).
So \(\sigma(T) = \{i, -i\}\).

%%%%%%% 13
\begin{excopy}
Suppose \(K = \{\lambda \in \C: 1 \leq |\lambda| \leq 2\}\).
  Put \(f(\lambda) = \lambda\). Let $A$ be the smallest closed
subalgebra of \(C(K)\) that contains $1$ and $f$.
Let $B$ be the smallest closed sub-algebra
  of \(C(K)\) that contains $f$ and \(1/f\).
  Describe the spectra \(\sigma_A(f)\) and \(\sigma_B(f)\) .
\end{excopy}
\unfinished

%%%%%%% 14
\begin{excopy}
\begin{itemize}
\itemch{a}
Fubini’s theorem was applied to vector-valued integrals in the proof of
Theorem~10.29. Justify this.
\itemch{b}
Construct a proof of Theorem~10.29 that uses no contour integrals, as
follows: Prove the theorem first for polynomials $g$, then for rational
  functions
  $g$ with no poles in \(\Omega\), and obtain the general case from Runge’s
  theorem.
\end{itemize}
\end{excopy}
\unfinished

%%%%%%% nn
\begin{excopy}
\end{excopy}
\unfinished

%%%%%%%%%%%%%%%
\end{enumerate}
%%%%%%%%%%%%%%%
