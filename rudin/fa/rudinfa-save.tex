\usepackage{array}
\usepackage{amsmath}
\usepackage{amssymb}
% \usepackage[mathcal]{euscript}
\usepackage{mathrsfs}

% \usepackage{fullpage}
\usepackage{geometry}
\geometry{left=1in, right=1in, top=1in, bottom=1in}
\setlength{\parindent}{0pt}
\setlength{\parskip}{6pt}


% are we in pdftex ????
\ifx\pdfoutput\undefined % We're not running pdftex
\else
\RequirePackage[colorlinks,hyperindex,plainpages=false]{hyperref}
\def\pdfBorderAttrs{/Border [0 0 0] } % No border arround Links
\fi

% \usepackage{fancyheadings}
\usepackage{fancyhdr}
\usepackage{pifont}

\pagestyle{fancy}
% \addtolength{\headwidth}{\marginparsep}
% \addtolength{\headwidth}{\marginparwidth}
%  \addtolength{\textheight}{2pt}
\newcommand{\srightmark}{\rightmark}
\newcommand{\sfbfpg}{\sffamily\bfseries{\thepage}}
  \newcommand{\symenvelop}{%
     {\nullfont a}\relax\lower.2ex\hbox{\large\Pisymbol{pzd}{41}}}
% \renewcommand{\chaptermark}[1]{\markboth{\thechapter.\ #1}}

\fancyplain{plain}{%
 \fancyhf{}
 \fancyhead[LE,RO]{\fancyplain{}{{\sfbfpg}}}
 \fancyhead[RE,LO]{\sl\leftmark}
 \fancyfoot[L]{\today}
 \fancyfoot[C]{Yotam Medini \copyright}
 \fancyfoot[R]{\symenvelop\ \texttt{yotam.medini@gmail.com}}
 \renewcommand{\headrulewidth}{0.4pt}
 \renewcommand{\footrulewidth}{0.4pt}
}

% \usepackage{amstex}
\usepackage{amsmath}
\usepackage{amssymb}
\usepackage{amsthm}
\usepackage{bm}
\usepackage{makeidx}
\makeindex % enable

% 'Inspired' by:
%% This is file `uwamaths.sty',
%%%     author   = "Greg Gamble",
%%%     email     = "gregg@csee.uq.edu.au (Internet)",

\makeatletter
\def\DOTSB{\relax}
\def\dotcup{\DOTSB\mathop{\overset{\textstyle.}\cup}}
 \def\@avr#1{\vrule height #1ex width 0pt}
 \def\@dotbigcupD{\smash\bigcup\@avr{2.1}}
 \def\@dotbigcupT{\smash\bigcup\@avr{1.5}}
 \def\dotbigcupD{\DOTSB\mathop{\overset{\textstyle.}\@dotbigcupD%
                               \vphantom{\bigcup}}}

\def\dotbigcupT{\DOTSB\smash{\mathop{\overset{\textstyle.}\@dotbigcupT%
                              \vphantom{\bigcup}}}%
                       \vphantom{\bigcup}\@avr{2.0}}
\def\dotbigcup{\mathop{\mathchoice{\dotbigcupD}{\dotbigcupT}
                                  {\dotbigcupT}{\dotbigcupT}}}
\let\disjunion\dotcup
\let\Disjunion\dotbigcup
\makeatother


\newcommand{\C}{\ensuremath{\mathbb{C}}} % The Complex set
\newcommand{\calB}{\ensuremath{\mathcal{B}}}
\newcommand{\calG}{\ensuremath{\mathcal{G}}}
\newcommand{\scrD}{\ensuremath{\mathscr{D}}}
\newcommand{\scrP}{\ensuremath{\mathscr{P}}}
\newcommand{\scrQ}{\ensuremath{\mathscr{Q}}}
\newcommand{\Lp}[1]{\ensuremath{\mathbf{L}^{#1}}} % Lp space
\newcommand{\N}{\mathbb{N}} % The Natural Set
\newcommand{\Q}{\ensuremath{\mathbb{Q}}} % The Rational set
\newcommand{\R}{\ensuremath{\mathbb{R}}} % The Real Set
\newcommand{\Z}{\ensuremath{\mathbb{Z}}} % The Integer Set
\newcommand{\intR}{\int_{-\infty}^{\infty}} % Integral over the reals
\newcommand{\posthat}[1]{#1{\,\hat{}\,}}

% sequences
\newcommand{\seq}[2]{\ensuremath{#1_1,\ldots,#1_{#2}}}
\newcommand{\seqn}[1]{\seq{#1}{n}}
\newcommand{\seqan}{\seq{a}{n}}
\newcommand{\seqxn}{\seq{x}{n}}
\newcommand{\seqalphn}{\seq{\alpha}{n}}

\newcommand{\ich}[1]{(\textit{#1})}
\newcommand{\itemch}[1]{\item[\ich{#1}]}


\title{Notes and Solutions to Exercises \\
          from \\
       Functional Analysis / Walter Rudin}
\author{Yotam Medini}


%%%%%%%%%%%
%% Theorems
%%
\newtheorem{thm}{Theorem}[chapter]
\newtheorem{cor}[thm]{Corollary}
\newtheorem{Def}{Definition}
\newtheorem{lem}[thm]{Lemma}
\newtheorem{llem}[thm]{Local Lemma}
\newtheorem{lthm}[thm]{Local Theorem}

% \newtheorem{quotecor}{Corollary}
% \newtheorem{quotelem}{Lemma}[section]
\newtheorem{quotethm}{Theorem}[chapter]


% \newcommand{\proofend}{\(\bullet\)}
\newcommand{\proofend}{\hfill\(\blacksquare\)}
\newenvironment{thmproof}
{\textbf{Proof.}}
{\proofend}

\newcommand{\chapterTypeout}[1]{\typeout{#1} \chapter{#1}}
\newcommand{\sectionTypeout}[1]{\typeout{#1} \section{#1}}

% abbreviations, ensuremath
\newcommand{\eqdef}{\ensuremath{\stackrel{\mbox{\upshape\tiny def}}{=}}}
\newcommand{\fx}{\ensuremath{f(x)}}
\newcommand{\gx}{\ensuremath{g(x)}}
\newcommand{\lrangle}[1]{\ensuremath{\langle #1 \rangle}}
\newcommand{\calD}{\ensuremath{\mathcal{D}}}
\newcommand{\frakD}{\ensuremath{\mathfrak{D}}}
\newcommand{\frakM}{\ensuremath{\mathfrak{M}}}
\newcommand{\M}{\ensuremath{\mathfrak{M}}}
\newcommand{\mldots}{\ensuremath{\ldots}}
\newcommand{\salgebra}{\(\sigma\)-algebra}
\newcommand{\wlogy}{without loss of generality}
\newcommand{\Wlogy}{Without loss of generality}
\newcommand{\unfinished}{\par\textbf{Unfinished !!!!!!!!!!!!!}\par}

%%%%%%%%%%%%
%% math op's
%%
\newcommand{\hull}{\mathop{\rm hull}\nolimits}
\newcommand{\id}{\mathop{\rm id}\nolimits}
\newcommand{\Int}{\mathop{\rm Int}\nolimits}
\newcommand{\inter}[1]{\ensuremath{#1^{\circ}}}
% \newcommand{\inter}[1]{\ensuremath{\textrm{int{#1}^\circ}}
\def\Lip{\mathop{\rm Lip}\nolimits}
\def\lip{\mathop{\rm lip}\nolimits}
\def\Ker{\mathop{\rm Ker}\nolimits}
\def\ker{\mathop{\rm ker}\nolimits}
\def\Re{\mathop{\rm Re}\nolimits}
\def\Im{\mathop{\rm Im}\nolimits}
\def\supp{\mathop{\rm supp}\nolimits}


\newenvironment{excopyOLD}
{\item\begin{minipage}[t]{.8\textwidth}\footnotesize}
{\smallskip\hrule\end{minipage}}

\newenvironment{excopy}
{\item % \relax
 \begin{list}{}{
 \setlength{\topsep}{0pt}
 \setlength{\partopsep}{0pt}
 \setlength{\itemsep}{0pt}
 \setlength{\parsep}{0pt}
 \setlength{\leftmargin}{0pt}
 \setlength{\rightmargin}{20pt}
 \setlength{\listparindent}{0pt}
 \setlength{\itemindent}{0pt}
 % \setlength{\labelsep}{0pt}
 \setlength{\labelwidth}{0pt}
 \footnotesize
 }
 \item
}
{\par
 {\nullfont a}\hrulefill
 \end{list}
}


%%%%%%%%%%%%%%%%%%%%%%%%%%%%%%%%%%%%%%%%%%%%%%%%%%%%%%%%%%%%%%%%%%%%%%%%
%%%%%%%%%%%%%%%%%%%%%%%%%%%%%%%%%%%%%%%%%%%%%%%%%%%%%%%%%%%%%%%%%%%%%%%%
%%%%%%%%%%%%%%%%%%%%%%%%%%%%%%%%%%%%%%%%%%%%%%%%%%%%%%%%%%%%%%%%%%%%%%%%
\begin{document}
\maketitle
\newpage
\tableofcontents
\newpage


\maketitle

%%%%%%%%%%%%%%%%%%%%%%%%%%%%%%%%%%%%%%%%%%%%%%%%%%%%%%%%%%%%%%%%%%%%%%%%
%%%%%%%%%%%%%%%%%%%%%%%%%%%%%%%%%%%%%%%%%%%%%%%%%%%%%%%%%%%%%%%%%%%%%%%%
%%%%%%%%%%%%%%%%%%%%%%%%%%%%%%%%%%%%%%%%%%%%%%%%%%%%%%%%%%%%%%%%%%%%%%%%
\setcounter{chapter}{-1}
\chapterTypeout{About this Document}

Here I try to solve problems
from the book \cite{RudinFA79}
\begin{center}
\textbf{Functional Analysis}\\
by
\textbf{Walter Rudin}
\end{center}

The publisher  McGraw-Hill has the following:
\begin{center}
\texttt{http://catalogs.mhhe.com/mhhe/viewProductDetails.do?isbn=0070542368}
\end{center}
web-page for this book.


\begin{center}
Whenever there is reference to theorem or lemma, it implicitly
targets this text, unless otherwise specified.
Furthermore, we have in this document lemmas or theorems
which are referred to as \emph{local} lemma
or \emph{local} theorem.
\end{center}

%%%%%%%%%%%%%%%%%%%%%%%%%%%%%%%%%%%%%%%%%%%%%%%%%%%%%%%%%%%%%%%%%%%%%%%%
%%%%%%%%%%%%%%%%%%%%%%%%%%%%%%%%%%%%%%%%%%%%%%%%%%%%%%%%%%%%%%%%%%%%%%%%
%%%%%%%%%%%%%%%%%%%%%%%%%%%%%%%%%%%%%%%%%%%%%%%%%%%%%%%%%%%%%%%%%%%%%%%%
\section*{Notation}

% Common notations

For each natural \(n\in\N\) we define
\begin{equation*}
\N_n \eqdef \{m\in\N: 1\leq m \leq n\} \qquad
\Z_n \eqdef \{m\in\Z: 0\leq m < n\}.
\end{equation*}

We define the positive subsets
\begin{alignat*}{2}
\Q^+ &\eqdef \{q\in\Q: q>0\}
 &\qquad
 \Q^\oplus &\eqdef \{q\in\Q: q\geq 0\} \\
\R^+ &\eqdef \{r\in\R: r>0\}
 &\qquad
 \R^\oplus &\eqdef \{r\in\R: r\geq 0\}.
\end{alignat*}
The non-negative and negation integers
\begin{align*}
\Z^+ = \{n\in\Z: n\geq 0\} = \{0\} \cup \N
\qquad
\Z^- =  \{n\in\Z: n < 0\} = \Z \setminus \Z^+.
\end{align*}


We use the following single side limit notations

\begin{alignat*}{2}
\lim_{t\to a^+} f(x) &= \lim_{\stackrel{h\to a}{h>a}} f(x)
 &\qquad
  \lim_{t\to a^-} f(x) &= \lim_{\stackrel{h\to a}{h<a}} f(x) \\
\varlimsup_{t\to a^+} f(x) &= \varlimsup_{\stackrel{h\to a}{h>a}} f(x)
 &\qquad
  \varlimsup_{t\to a^-} f(x) &= \varlimsup_{\stackrel{h\to a}{h<a}} f(x) \\
\varliminf_{t\to a^+} f(x) &= \varliminf_{\stackrel{h\to a}{h>a}} f(x)
 &\qquad
  \varliminf_{t\to a^-} f(x) &= \varliminf_{\stackrel{h\to a}{h<a}} f(x)
\end{alignat*}

For topolgical notions we use
\begin{center}
\begin{tabular}{ll}
\(\inter{A}\) & Interior of $A$ \\
\(\closure{A}\) & Closure of $A$ \\
\(\boundary{A}\) & Boundary of $A$ \\
\end{tabular}
\end{center}



%%%%%%%%%%%%%%%%%%%%%%%%%%%%%%%%%%%%%%%%%%%%%%%%%%%%%%%%%%%%%%%%%%%%%%%%
%%%%%%%%%%%%%%%%%%%%%%%%%%%%%%%%%%%%%%%%%%%%%%%%%%%%%%%%%%%%%%%%%%%%%%%%
%%%%%%%%%%%%%%%%%%%%%%%%%%%%%%%%%%%%%%%%%%%%%%%%%%%%%%%%%%%%%%%%%%%%%%%%
\iftrue
 %%%%%%%%%%%%%%%%%%%%%%%%%%%%%%%%%%%%%%%%%%%%%%%%%%%%%%%%%%%%%%%%%%%%%%%%
%%%%%%%%%%%%%%%%%%%%%%%%%%%%%%%%%%%%%%%%%%%%%%%%%%%%%%%%%%%%%%%%%%%%%%%%
%%%%%%%%%%%%%%%%%%%%%%%%%%%%%%%%%%%%%%%%%%%%%%%%%%%%%%%%%%%%%%%%%%%%%%%%
\chapterTypeout{General Theory}

%%%%%%%%%%%%%%%%%%%%%%%%%%%%%%%%%%%%%%%%%%%%%%%%%%%%%%%%%%%%%%%%%%%%%%%%
%%%%%%%%%%%%%%%%%%%%%%%%%%%%%%%%%%%%%%%%%%%%%%%%%%%%%%%%%%%%%%%%%%%%%%%%
\section{Notes}

Lemma to be used for Exercise~13 in page~\pageref{ex:1:13}.

\begin{llem}
Let $X$ be some topological space having a countable dense subset.
Let \(C(X)\) be the set of complex continuous functions over $X$.
\end{llem}
\textbf{Proof.}
Let $D$ be a countable dense subset of $X$.
\proofend

In Section~1.47 the inequality
\begin{equation} \label{eq:abp:concave}
(a + b)^p \leq a^p + b^p
\qquad (a,b \geq 0, \quad 0 < p < 1)
\end{equation}
is used. 
\iffalse
It is implied by the convexity of \(t \to t^p\),
see \cite{RudinRCA80} the proof of Theorem~3.5.
\fi

Look at \(f(x) = a^p + x^p - (a + x)^p\) with \(x \geq 0\).
Clearly \(a+x \geq x\) and so \(x^{p-1} \leq (a+x)^{p-1}\).
Thus
\begin{equation*}
f'(x) = px^{p-1} - p(a+x)^{p-1} = p\left(x^{p-1} - (a+x)^{p-1}\right) \geq 0.
\end{equation*}
Since \(f(0) = 0\), \eqref{eq:abp:concave} holds.


%%%%%%%%%%%%%%%%%%%%%%%%%%%%%%%%%%%%%%%%%%%%%%%%%%%%%%%%%%%%%%%%%%%%%%%%
%%%%%%%%%%%%%%%%%%%%%%%%%%%%%%%%%%%%%%%%%%%%%%%%%%%%%%%%%%%%%%%%%%%%%%%%
\section{Exercises} % pages 36-40

%%%%%%%%%%%%%%%%%
\begin{enumerate}
%%%%%%%%%%%%%%%%%

%%%%%%%%%%%%%%
\begin{excopy}
Suppose $X$ is a vector space. All sets mentioned below are understood
to be subsets of $X$.
Prove the following statements from the axioms as given in Section~1.4
(Some of these are tacitly used in the text.)
\begin{itemize}
 \itemch{a}
   If \(x\in X\) and \(y\in X\) there is a unique \(z\in X\)
   such that \(x+z=y\).
 \itemch{b} \(0x = 0 = \alpha 0\) if \(x\in X\) and \(\alpha\) a scalar.
 \itemch{c} \(2A \subset A + A\); it may happen that \(2A \neq A + A\).
 \itemch{d}
   $A$ is convex if and only if \((s+t)A = sA + tA\) for all positive
   scalars $s$ and $t$.
 \itemch{e} Every union (and intersection) of balanced sets is balanced.
 \itemch{f} Every intersection of convex sets is convex.
 \itemch{g}
   If \(\Gamma\) is a collection is a collection 
   of convex sets that is totally ordered
   by set inclusion, then the union of all members of \(\Gamma\) is convex.
 \itemch{h} If $A$ and $B$ are convex, so is \(A+B\).
 \itemch{i} If $A$ and $B$ are balanced, so is \(A+B\).
 \itemch{j}
   Show that parts \ich{f}, \ich{g}, and \ich{h} hold
   with subspaces in place of convex sets.
\end{itemize}
\end{excopy}

\begin{itemize}
 \itemch{a}
   Put \(z = (-x) + y\). Now 
   \(x+z = x + ((-x) + y) = (x + (-x)) + y = 0 + y = y\).
   For uniqueness, say \(x+z' = y\). Thus
   \(z = (-x) + y = (-x) + (x+z') = ((-x) + x) + z' = 0 + z' = z'\).
 \itemch{b}
   First note that the defintion of the zero vector
   can be generalized. If  \(x,z\in X\) and \(x+z = x\) then
   \(y+z = y\) for all \(y\in X\).
   This is because
  \[z = z+(x+(-x)) = (z+x)+(-x) = x+(-x) = 0.\]

   From the ditributive law \[x = 1x = (0 + 1)x = 0x + 1x = 0x + x.\]
   Now \[0x = 0x + (x + (-x)) = (0x + x) + (-x) = x + (-x) = 0.\]

   Take some \(x\in X\). From other ditributive law
   \[\alpha x = \alpha (x+0) = \alpha x + \alpha 0.\]
   Hence \(\alpha 0 = 0\).
 \itemch{c}
   Let \(a\in 2A\) then there exists \(x\in A\) such that \(a=2x\).
   Now \(2x = x + x \in A + A\).
   Take \(X=\R\) and \(A = \{0,1\}\).
   Then \(2A = \{0,2\} \subsetneq \{0,1,2\} = A+A\).
 \itemch{d}
   If \(s,t > 0\), then \((s+t)A \subset sA + tA\) trivially.
   If in addition, $A$ is convex, then   
   for any \(x,y\in A\) we have \(s/(s+t)x + t/(s+t)y \in A\).
   Equivalently \(sx + ty \in (s+t)A\). Thus
   \(sA + tA \subset (s+t)A\). Thus \((s+t)A = sA + tA\)

   Conversely, assume \((s+t)A = sA + tA\) for any \(s,t > 0\).
   In particular, \(sA + tA \subset (s+t)A\) for any \(s,t > 0\).
   This means that if \(x,y\in A\) then 
   \(sx+ty\in A\) for any \(s,t > 0\), which is convexity.

 \itemch{e}
   Let \(A_i\) be a family of balanced sets, that is for any 
   \(\alpha\) in the field \(\Phi\) such that \(|\alpha| = 1\)
   we have \(\alpha A_i = A_i\).

   Indeed, take an arbitrary such \(\alpha\).

   \textbf{Union:}
   Put \(U = \cup_{i\in I} A_i\) and say \(x \in U\).
   Then, there exists \(j \in I\) such that \(x\in A_i\).
   Since \(A_i\) is balanced, \(x \in \alpha A_i \subset \alpha U\).
   Thus \(U \subset \alpha U\).
   Similarly, we can show that \(U \subset \alpha^{-1} U\)
   which is equivalent to the converse inclusion
   \(\alpha U \subset U\) and so  \(U = \alpha U\).

   \textbf{Intersection:}
   Put \(W = \cap_{i\in I} A_i\) and say \(x \in W\).
   Then \(x \in A_i\) for all \(i \in I\) and in turn
   so \(x \in \alpha A_i\) for all \(i \in I\).
   Hence, \(x \in \cap_{i\in I} \alpha A_i = W\).
   Thus \(W \subset \alpha W\). With similar ending argumentation
   as in the `Union' case we get that \(W = \alpha W\).
 \itemch{f}
  Say \(\{A_i\}_{i\in I}\) is a family of convex sets.
  Let \(W = \cap_{i\in I} A_i\), and let \(x,y \in W\).
  Take an arbitrary \(i \in I\) and now
  for any \(s,t\geq 0\) such that \(s+t=1\) 
  we have \(sx+ty \in A_i\). Since $i$ is arbitrary, 
  \(sx+ty \in W\) and so $W$ is convex.
 \itemch{g}
  Let $U$ be the union of \(\Gamma\). Then any two point \(x,y\in U\)
  belong to some convex set \(A \in \Gamma\) and so any convex combination
  of $x$ and $y$ is in \(A\subset U\).
 \itemch{h}
  Let \(x_0,x_1 \in A+B\)
  For \(i=0,1\) there exist \(a_i\in A\) and \(b_i \in B\) 
  such that \(x_i = a_i + b_i\).
  Now for any \(s,t\geq 0\) such that \(s+t=1\) 
  we have 
  \begin{eqnarray*}
    sa_0 + ta_1 &\in& A \\
    sb_0 + tb_1 &\in& B
  \end{eqnarray*}
  and so 
 \begin{equation*}
   sx_0 + tx_1 = s(a_0+b_0) + t(a_1+b_1)  \in A+B.
 \end{equation*}

 \itemch{i}
  Let \(a\in A\), \(b\in B\) and a scalar \(\alpha\) such that \(|\alpha|=1\).
  Since $A$ and $B$ are balanced, we have 
  \(\alpha a \in A\) and \(\alpha b \in B\)
  and so \(\alpha(a+b)\in A + B\).
  Thus \(A+B \subset \alpha(A+B)\). Using \(\alpha^{-1}\) we get the 
  converse inclusion and so 
   \(A+B = \alpha(A+B)\) and \(A+B\) is balanced.
 \itemch{j}
  \begin{itemize}
   \item[{[f]}] 
        Say \(\{A_i\}_{i\in I}\) is a family of subspaces of $X$.
        Put \(W = \cap_{i\in I} A_i\).
        For any \(x,y\in W\)
        all linear combinations of $x$ and $y$ are in 
        all \(\{A_i\}\) and so must also be in $W$. Hence $W$ contains
        any linear combinations of its vectors, and so $W$ is a subspace.
   \item[{[{\small g}]}] 
        Let \(\Gamma\) be a collection of subspaces of $X$
        that are totally ordered by inclusion, and let $U$ be the union.
        For any \(x,y\in U\) thetre is \(A\in \Gamma\) such that
        \(x,y\in A\). Clearly any linear combination of $x$ and $y$
        is in \(A\subset U\) and so $U$ is a subspace.
   \item[\ich{h}] 
        Say $A$ and $B$ are subspaces of $X$.
        Any element \(w_i\in A+B\) can be represented as \(w_i=x_i+y_i\)
        where \(x_i\in A\) and \(y_i\in B\).
        Let \(v = s w_0 + t w_1\) be a linear combination of \(w_i\in W\).
        Now 
        \begin{equation*}
         v = s w_0 + t w_1 = (sx_0 + tx_1) + (sy_0 + ty_1) \in A + B
        \end{equation*}
        Thus \(A+B\) is a subspace of $X$.
  
  \end{itemize}
\end{itemize}

%%%%%%%%%%%%%%
\begin{excopy}
The
\index{convex hull}
\emph{convex hull} of a set $A$ in a vector space $X$ is the set of all
\index{convex combination}
\emph{convex combinations} of members f $A$, that is, the set of all sums
\begin{equation*}
  t_1 x_1 + \cdots + t_n x_n
\end{equation*}
in which
\(x_i\in A\),
\(t_i\geq 0\),
\(\sum t_i = 1\);
$n$ is arbitrary. Prove that the convex hull of $A$ is convex
and that it is the intersection of all convex sets that contain $A$.
\end{excopy}

Let $H$ be the set of all such combinations, namely \(\sum_{i=1}^n t_ix_i\),
where \(t_i\in \Phi\) and \(x_i\in A\).

First, we show that $A$ is convex.
Let \(v_1,v_2\in H\) and \(0\leq r,s \leq 1\) with \(r+s=1\).
By definition both \(v_j\) are (for \(j=1,2\) (convex) combination as in
\begin{equation*}
v_j = \sum_{i=1}^{n_j} t_{ij} x_{ij}.
\end{equation*}
By adjoining \(\{x_{i0}\}\) and \(\{x_{i1}\}\), into 
a single set \(\{x_i\}\)
and assigning zeros to
some coefficients of the newly created set of generators, we can have
\begin{equation*}
v_j = \sum_{i=1}^{N} t_{ij} x_i.  \qquad \textnormal{for }\, j=1,2
\end{equation*}
where \(\sum_i t_{ij} = 1\) for \(j=1,2\).
Now 
\begin{equation*}
rv_1 + sv_2 = 
r\sum_{i=1}^N t_{i0} x_i +
s\sum_{i=1}^N t_{i1} x_i =
\sum_{i=1}^N (rt_{i0} + st_{i1}) x_i
\end{equation*}
and
\begin{eqnarray*}
\sum_{i=1}^N r t_{i0} + s t_{i1} 
 &=&  \left(\sum_{i=1}^N r t_{i0}\right) +
      \left(\sum_{i=1}^N s t_{i1}\right) \\
 &=&  r\left(\sum_{i=1}^N t_{i0}\right) +
      s\left(\sum_{i=1}^N t_{i1}\right) \\
 &=&  r\cdot 1 + s \cdot 1 = 1.
\end{eqnarray*}
Thus \(rv_1 + sv_2 \in H\) and so $H$ is convex.

Now we show that $H$ is the minimal convex set that contains $A$.
Let \(H_n\) be the set of all convex combination of size $n$
that is \(\sum_{i=1}^n t_i x_i\) with positive \(t_i\) 
and \(\sum_{i=1}^n t_i = 1\).
By induction will show that \(H_n \subset \hull(A)\) for all $n$.
Clearly \(H_1 = A \subset \hull(A)\).
Now assume \(H_k \subset \hull(A)\).
Let \(v = \sum_{i=1}^{k+1} t_i x_i\) 
be some convex combination, that is \(t_i\geq 0\) 
and \(\sum_{i=1}^{k+1} t_i = 1\).
If \(t_k{+1} = 1\) then trivailly \(v\in \hull(A)\).
Otherwise,  \(1 - t_{k+1} = \sum_{i=1}^k t_i \neq 0 \) 
and we have
\begin{eqnarray} \label{eq:convex:k1}
v &=& \sum_{i=1}^{k+1} t_i x_i \\
 &=& \left(\sum_{i=1}^k t_i x_i \right) + t_{k+1}x_{k+1} \\
 &=& (1 -t_{k+1}) 
      \left(\sum_{i=1}^k \left(t_i/\sum_{j=1}^k t_j\right)x_i \right) 
      + t_{k+1}x_{k+1} 
\end{eqnarray}
Since 
\begin{equation*}
 \sum_{i=1}^k \left(t_i/\sum_{j=1}^k t_j\right) = 1
\end{equation*}
We have by induction
\begin{equation*}
 \left(\sum_{i=1}^k \left(t_i/\sum_{i=j}^k t_j\right)x_i \right)
  \in H_k \subset \hull(A).
\end{equation*}

Hence, the last expression in (\ref{eq:convex:k1}) is a conve combination
in \(\hull(A)\) and so \(v\in \hull(A)\).
Thus \(H_{k+1}\subset \hull(A)\).




%%%%%%%%%%%%%%
\begin{excopy}
Let $X$ be a topological vector space. All sets mentioned below
are understood to be subsets of $X$. Prove the following statements:
\begin{itemize}
 \itemch{a}The convex hull of every open set is open.
 \itemch{b}
   If $X$ is locally convex the convex hull of every bounded set is bounded.
   (This is false without convexity; see Section~1.47.)
 \itemch{c} If $A$ and $B$ are bounded, so is \(A+B\)
 \itemch{d} If $A$ and $B$ are compact, so is \(A+B\)
 \itemch{e} If $A$ is compact and $B$ is closed, then \(A+B\) is closed.
 \itemch{f}
   The sum of two closed sets may fail to be closed.
   [The inclusion in \ich{b} of Theorem~1.13 may therefore be strict.]
\end{itemize}
\end{excopy}

\begin{itemize}
 \itemch{a}
   Let $A$ be an open set in $X$ and let $H$ be the convex hull of $A$.
   Take an arbitrary \(h\in H\). 
   By previous exercise, we have the representation
   \(h = \sum_{i=1}^n t_i x_i\) 
   where \(\sum_{i=1}^n t_i = 1\) and \(t_i\geq 0\) and with \(x_i\in A\).
   By $A$ being open, there exist
   open neighborhoods  \(V_i\) of $0$, such that \(x_i + V_i \subset A\).
   Taking finite intersection, we have \(V = \cap_{i=1}^n V_i\) an open set.

   We claim that \(h + V \subset H\).
   Let \(h` \in h+V\) so \(d = h`-h\in V\).
   Now for every \(i=1,\ldots,n\), we have
   \begin{equation*}
   x_i + d \in x_i+V \subset x_i+V_i \subset A.
   \end{equation*}

   Now 
   \begin{equation*}
   h` = h + d = \left(\sum_{i=1}^n t_i x_i\right) + 1\cdot d \\
      = \sum_{i=1}^n t_i (x_i + d) \\
   \end{equation*}

   and so \(h'\) is a convex combination of vectors in $A$.
   Thus \(h\in \inter{(\hull(A))}\) and so \(\hull(A)\) is open.

 \itemch{b}
   Let $B$ be bounded in $X$ and \(H = \hull(B)\).
   Let $V'$ be an arbitrary neighborhood of the origin.
   By being locally convex, we can take a convex \(V\subset V'\) 
   which is also a $0$-neighborhood.
   Since $B$ is bounded, there exists \(0 < m < \infty\) such that
   \(B\subset mV\).

   Now since \(mV\) is convex and $H$ is the minimal convex set
   containing $B$, we must have \(H \subset mV\) and so $H$ is bounded.

 \itemch{c}
   Let $A$ and $B$ be bounded. Let $V$ be an arbitrary neighborhood of $0$.
   By definition, there exist \(m_A, m_B < \infty\)
   such that \(A \subset m_A V\) and \(B \subset m_B V\).
   let \(x=a+b\in A+B\) with \(a\in A\) and \(b\in B\).
   Now
   \begin{eqnarray*}
   a &\in& A \subset m_A V \subset (m_A + m_B)V \\
   b &\in& B \subset m_B V \subset (m_A + m_B)V \\
   \end{eqnarray*}
   and so \(x \in  (m_A + m_B)V\), hence \(A+B\) is bounded.

 \itemch{d}
  We use general point set topolgy.
  \index{Tychonoff}
  By Tychonoff Theorem, the set \(A\times B\) is compact in \(X\times X\).
  Also the image of a compact set under a continuous mapping is compact.
  Combining these with the fact that
  the addition mapping \(+: X\times X \rightarrow X\) 
  is continuous, we conclude that
  the image \(A+B\) is compact.
   
 \itemch{e}
   Let $K$ be compact and $C$ be closed.
   By negation say \(x\in \overline{K+C}\setminus(K+C)\).
   Now, since \(x\notin K+C\), we must have \((x-K)\cap C = \emptyset\).
   But since \(x-K\) is compact,
   by Theorem~1.10 it can be separated from $C$ by open $V$ --- 
   a neighborhood of $0$.
   More precisely, 
   \begin{equation*}
   (x-K+V) \cap (C+V) = \emptyset
   \end{equation*}
   Hence,
   \begin{equation*}
   (x+V) \cap (C+K+V) = \emptyset
   \end{equation*}
   and in particular, \(x \notin C+K+V \supset \overline{C+K}\),
   which is a contradiction.
   
 \itemch{f}
   In \(\R^2\), let 
   \begin{eqnarray*}
   A & = & \{(x,y)\in\R^2: y = e^x\} \\
   B & = & \{(x,y)\in\R^2: y = 0\} \\
   \end{eqnarray*}
   Now clearly \((0,0) \notin A+B\), but for any \(\epsilon > 0\)
   There is some \(M > 0\) such that \(e^{-M} < \epsilon\).
   So \(p_m = (-M,e^{-M}) + (M,0) \in A+B\), but 
   \(\|p_M - (0,0)\| = \|p_M\| < \epsilon\). Thus
   \(p_M \in \overline{A+B} \setminus (A+B)\), which shows that \(A+B\) is 
   not closed.

\end{itemize}


%%%%%%%%%%%%%%
\begin{excopy}
Let \(B=\{(z_1,z_2)\in\C: |z_1|\leq|z_2|\}\).
Show that $B$ is balanced but that its interior is not.
[Compare with \ich{e} of Theorem~1.13.]
\end{excopy}

Say \(\alpha\in \C\) such that \(|\alpha|\leq 1\).
Assume \((z_1,z_2)\in B\), then \(|z_1|\leq|z_2|\) and 
so 
\begin{equation*}
 |\alpha z_1| = |\alpha|\cdot|z_1| \leq
 |\alpha|\cdot|z_2| = |\alpha z_2|.
\end{equation*}
Thus \(\alpha B \subset B\) and $B$ is balanced.
But \(0\notin \inter{B}\) and in particular,
\begin{equation*}
0\cdot(0,0) = (0,0) \in B \setminus \inter{B}.
\end{equation*}
Thus \inter{B} is not balanced.

%%%%%%%%%%%%%%
\begin{excopy}
Consider the definition of 
\index{bounded set}
``bounded set'' given in Section~1.6.
Would the content of this definition be altered if it were required merely
that to every neighborhood $V$ of $0$
corresponds \emph{some} \(t>0\) such that \(E \subset tV\)?
\end{excopy}

The answer is: Yes.

Call the suggested definition in the exercise: \emph{weak-bounded}.
Clearly a bounded set is weak bounded.

Conversely, say $E$ is a weak-bounded set.
Let $V$ be an arbitrary neighborhood of $0$.
Since the multiplication between scalar and vector 
\(\cdot:\Phi\times X\rightarrow X\) is a continuous mapping,
there is an \(\epsilon > 0\) and a neighborhood $V'$ 
such that \(\alpha v \in V\) % where \(\alpha\in \Phi\) and \(v\in V'\)
whenever \(|\alpha| < \epsilon\) and \(v\in V'\).
By minizing and intersection, we can pick \(\epsilon\) and $V'$
such that \(\epsilon \leq 1\) and \(V' \subset V\).
Hence, if \(|\alpha|< \epsilon\) then  \(\alpha V' \subset V\).
Now by being weak-bounded, there is \(t>0\) such that \(E \subset tV'\).
Putting \(M = t/\epsilon\), for every $s$ such that
\(|s| > M\) we have \(|t/s| < \epsilon\) and so 
\((t/s)V' \subset V\), equivalently 
\(tV' \subset sV\). Thus
\begin{equation*}
E \subset tV' \subset sV.
\end{equation*}
We conclude that $E$ is bounded by the original definition.
This justifies the positive answer.

%%%%%%%%%%%%%%
\begin{excopy}
Prove that a set $E$ in a topological vector space is bounded if and only if
every countable subset of $E$ is bounded.
\end{excopy}

If $E$ is bounded then clearly any subset of $E$ is bounded, 
in  particular countable subsets.

Conversely, assume every countable subset of $E$ is bounded
and by negation $E$ is not bounded.
Thus there exists some neighborhood $V$ of $0$, such that 
\(E\not\subset nV\) for every \(n > 0\).
Pick \(v_n \in E \setminus nV\). 
Now the countable set \(\{v_n\}_{n\in \N}\) is not bounded.

%%%%%%%%%%%%%%
\begin{excopy}
Let $X$ be a vector space of all complex functions
on the unit interval \([0,1]\), topologized by the family of seminorms
\begin{equation*}
 \rho_x(f) = |f(x)|\qquad (0\leq x \leq 1).
\end{equation*}
This topology is called the 
\index{pointwise convergence!topology}
\emph{topology of pointwise convergence}.
Justify this terminology.

Show that there is a sequence \(\{f_n\}\) in $X$ such that
(\emph{a}) \(\{f_n\}\) converges to $0$ as \(n\rightarrow \infty\),
but (\emph{b}) if \(\{\gamma_n\}\) is any sequence of scalars such that
\(\gamma_n \rightarrow \infty\)
then \(\{\gamma_n f_n\}\) does not converge to $0$.
(Use the fact that the collection of all complex sequences converging to $0$
has the same cardinality as \([0,1]\).)

This shows that metrizability cannot be omitted in (\emph{b}) of Theorem~1.28.
\end{excopy}


For any sequence \(\{f_n(x)\}\) of functions that converge to \(f(x)\)
pointwise, we have \(\rho_x(f_n - f) = |f_n(x) - f(x) | \rightarrow 0\).
This explain the terminology.

We now compute the cardinality of the set $S$ of complex sequences converging
to zero. Clearly 
\begin{equation*}
|S| \geq |\C| = \mathfrak{c} = 2^{\aleph_0}.
\end{equation*}
The set $T$ of all complex sequences has the cardinality:
\begin{equation*}
|T| = \mathfrak{c}^{\aleph_0} = \left(2^{\aleph_0}\right)^{\aleph_0}
 = 2^{\aleph_0\aleph_0} = 2^{\aleph_0}.
\end{equation*}
Since \(S\subset T\) we have \(|S|\leq |T|\) and( so \(|S| = \mathfrak{c}\).

Choose some 1-1 onto mapping \(\sigma:[0,1]\rightarrow S\)
and define the sequence of functions
\(f_n(x) = \sigma(x)_n\).
By construction, for any complex sequence converging to zero
coincide with \(\{f_n(x)\}_{n=1}^\infty\) for some \(x\in[0,1]\).

Clearly \(f_n(x) \rightarrow 0\) for all \(x\in [0,1]\).
Now let, \(\gamma_n \rightarrow \infty\)
The sequence 
\begin{equation*}
 \alpha_n = \left\{\begin{array}{cl}
            1/\gamma_n & \qquad \gamma_n \neq 0 \\
            1          & \qquad \gamma_n = 0 \\
            \end{array}\right.
\end{equation*}
Clearly \(\alpha_n \rightarrow 0\). By construction, there exists some
\(\alpha\in [0,1]\) such that \(f_n(\alpha) = \alpha_n\) for all $n$.
But \(\gamma_n f_n(\alpha) \rightarrow 1 \neq 0\). In particular
\(\gamma_n f_n \not\rightarrow 0\).


%%%%%%%%%%%%%%
\begin{excopy}
 \begin{itemize}
  \itemch{a}
   Suppose \scrP\ is a separating family of seminorms 
   on a vector space $X$. Let \scrQ\ be the smallest family of 
   seminorms on $X$ that contains  \scrP\ and is closed under max.
   [This means: If \(p_1\in \scrQ\), \(p_2\in \scrQ\) and
   \(p = \max(p_1,p_2)\), then \(p\in \scrQ\).]
   If the construction of Theorem~1.37 is applied to \scrP\ and to \scrQ,
   show that the two resulting topologies coincide.
   The main difference is that \scrQ\ leads directly to a base,
   rather than to a subbase.
   [See Remark \ich{a} of section~1.38]
  \itemch{b}
   Suppose \scrQ\ is as in part \ich{a} and \(\Lambda\) is a linear functional
   on $X$. Show that \(\Lambda\) is continuous if and only if there exists
   a \(p\in \scrQ\) such that \(|\Lambda x| \leq Mp(x)\)
   for all \(x\in X\) and some constant \(M< \infty\).
 \end{itemize}
\end{excopy}

\begin{itemize}
 \itemch{a}
  Let us denote the topologies by \(\tau_{\scrP}\) and \(\tau_{\scrQ}\)
  respectably.
  Clearly  \(\tau_{\scrP} \subset \tau_{\scrQ}\).
  To show the opposite inclusion, let $G$ be a \(\tau_{\scrQ}\)
  neighborhood of $0$. By being a subbase, for \(k=1,\ldots,n\)
  there exist
  \begin{equation*}
    V_k = V(p_k,\alpha_k) = \{x\in X: q_k(x) < \alpha_k\}
     \qquad \textrm{where }\, q_k\in \scrQ, \alpha > 0
  \end{equation*}
  such that \(\cap_{k=1}^n V_k \subset G\).
  But each \(q_k\in \scrQ\) can be represented as
  \begin{equation*}
  q_k = \max_{1\leq j\leq N_k} {}_{k}p_j
  \end{equation*}
  and so 
  \begin{equation*}
  V_k = \{x\in X: q_k(x) < \alpha_k\} 
      = \{x\in X:  \max_{1\leq j\leq N_k} {}_{k}p_j(x) < \alpha_k\}
      = \bigcap_{j=1}^{N_k} \{x\in X: {}_{k}p_j(x) < \alpha_k\}
  \end{equation*}
  Thus 
  \begin{equation*}\
    \cap_{k=1}^n V_k 
     = \cap_{k=1}^n\cap_{j=1}^{N_k} \{x\in X: {}_{k}p_j(x) < \alpha_k\}.
  \end{equation*}
  We showed that $G$ constains a \(\tau_{\scrP}\) neighborhood of $0$,
  and so \(\tau_{\scrQ} \subset \tau_{\scrP}\) and the topologies
  coincide.

 \itemch{b}
  Assume there exists \(p\in \scrQ\) and some \(M<\infty\) 
  such that \(|\Lambda x| \leq Mp(x)\) for  all \(x\in X\).
  Look at a base $0$ neighborhood \(V = \{x\in X: p(x)<1\}\).
  Let \(x\in V\), 
  clearly \(|\Lambda x| \leq Mp(x) \leq M\).
  Thus \(\Lambda\) is bounded on some neighborhood of $0$
  and by Theorem~11.8 \(\Lambda\) is continuous.

  Conversely, assume \(\Lambda\) is continuous. 
  By Theorem~1.18, there exist some
  neighborhood $W$ of 0 such that \(\Lambda\) is bounded on $W$ by $b$.
  By the topology construction, there exist a base neighborhood \(V\subset W\)
  such that
  \begin{equation*}
   V =  \cap_{j=1}^{n} \{x\in X: p_j(x) < \alpha_j\}
  \end{equation*}
  for some seminorms \(p_j\in \scrQ\) and \(\alpha_j > 0\).
  By induction in the hypothesis on \scrQ, 
  \(p = \max_{1\leq j\leq n} p_j \in \scrQ\).
  Put \(\alpha = \min_{1\leq j\leq n} \alpha_j\).
  Now 
  \begin{equation*}
   U = \{x\in X: p(x) < \alpha\} \subset V
  \end{equation*}
  is a neighborhood of $0$ and for any \(x\in U\), we have \(|\Lambda x| < M\).

  Now let \(x \in X\), there are two cases.
  \begin{itemize}
   \item[(i)]
     \(p(x)=0\).\newline
     Let \(c = \Lambda x\).
     If by negation \(c\neq 0\), then \(\Lambda(\alpha x/c) = \alpha\).
     Hence \(\alpha x/c \notin U\) but \(p(\alpha x/c) = (\alpha/c)p(x) = 0\)
     giving the \(\alpha x/c \in U\) contradiction.
   \item[(ii)]
    \(p(x)\neq 0\).\newline
     Now \(p(\alpha x/2p(x)) = \alpha/2\).
     Thus \(\alpha x/2p(x) \in U\).
     \begin{equation*}
     |\Lambda x| = | (2p(x)/\alpha) \Lambda (\alpha  x/2p(x))| 
      \leq (2p(x)b/\alpha).
     \end{equation*}
   \end{itemize}
   Putting \(M = 2b/\alpha\) and we have for all \(x\in X\)
   \(|\Lambda x| \leq Mp(x)\) as desired.
  
\end{itemize}

%%%%%%%%%%%%%% 9
\begin{excopy}
Suppose \label{ex:1:9}
 \begin{itemize}
  \itemch{a} $X$ and $Y$ are topological vector spaces,
  \itemch{b} \(\Lambda: X\rightarrow Y\) is linear,
  \itemch{c} $N$ is a closed subspace of $X$,
  \itemch{d} \(\pi:X \rightarrow X/N\) is the quotient map, and
  \itemch{e} \(\Lambda x = 0\) for every \(x\in N\).
 \end{itemize}
Prove that there is a unique \(f:X/N \rightarrow Y\) which satsifies
\(\Lambda = f \circ \pi\), that is,
\(\Lambda x = f(\pi(x))\) for all \(x\in X\).
Prove that this $f$ is linear and that \(\Lambda\) is continuous
if and only if $f$ is continuous. Also, \(\Lambda\) is open if and open if 
$f$ is open.
\end{excopy}

Let \(x+N\in X/N\) define \(f(x+N) = \Lambda x\).
To see that this is well defined, let \(x_1,x_2\in X\)
such that \(x_1 + N = x_2 + N \subset \Ker \Lambda\).
Now clearly \(x_2 - x_1 \in N\) and so
\begin{equation*}
f(x_1 + N) = \Lambda (x_1) = \Lambda(x_1) + \Lambda(x_2 - x_1) = 
  \Lambda(x_2) = f(x_2 + N).
\end{equation*}
The linearity  of $f$ is trivailly derived by \(\Lambda\).

Theorem~1.41 shows that \(\pi\) is a continuous and open mapping.
If $f$ is continuous, sp is then \(\Lambda = f \circ \pi\).
If $f$ is open, so is \(\Lambda = f \circ \pi\).

Conversely,

\begin{itemize}

\item
Assume \(\Lambda\) is continuous and let $W$ be any open set in $Y$.
We now show that 
\begin{equation} \label{eq:fw:Lambda}
f^{-1}(W) = \pi(\Lambda^{-1}(W)).
\end{equation}
We have 
\begin{eqnarray*}
x+N \in f^{-1}(W) 
  & \Leftrightarrow & \exists w\in W,\, f(x)=w \\
  & \Leftrightarrow & \Lambda x \in W \\
  & \Leftrightarrow & x \in \Lambda^{-1}(W) \\
  & \Leftrightarrow & x+N \in  \pi(\Lambda^{-1}(W))
\end{eqnarray*}

By definition \(U = \Lambda^{-1}(W)\) is open in $X$.
Hence by (\ref{eq:fw:Lambda}) and \(\pi\) being open,
\(f^{-1}(W)\) is open and $f$ is continuous.

\item 
Assume \(\Lambda\) is open and let $V$ be open in \(X/N\).
By definition of the quotient topological, \(U = \pi^{-1}(V)\) 
is open in $X$. Clearly \(f(V) = \Lambda(U)\) and thus $f$ is open.
\end{itemize}


%%%%%%%%%%%%%% 10
\begin{excopy}
Suppose $X$ and $Y$ are topological vector spaces, \(\dim Y = \infty\),
\(\Lambda: X\rightarrow Y\) is linear, and \(\Lambda(X)=Y\).
\begin{itemize}
 \itemch{a} Prove that \(\Lambda\) is an open mapping.
 \itemch{b}
   Assume, in addition, that the null space of \(\Lambda\) is closed,
   and prove that \(\Lambda\) is then continuous.
\end{itemize}
\end{excopy}

\begin{itemize}
 \itemch{a} 
  Let $V$ be a neighborhood of $0$ in $X$.
  Since translation is homeomorphism, it is sufficient to show
  that \(\Lambda V\) contains a neighborhood of $0$ in $Y$.
  Let \seqn{y} be a basis for $Y$
  and pick \seqxn\ such that \(\Lambda x_i = y_i\) for \(1\leq i \leq n\).
  Since scalar muliplication is continuous there are \seqn{\epsilon}
  such that \((-\epsilon_i,+\epsilon_i)\cdot x_i \subset V\).
  Define the ``sub-box''
  \begin{equation*}
   B = \left\{\sum_{i=1}^n a_i x_i: |a_i| < \epsilon_i\right\} \subset V.
  \end{equation*}
  Clearly 
  \begin{equation*}
   \Lambda B = \left\{\sum_{i=1}^n a_i y_i: |a_i| < \epsilon_i\right\}
  \end{equation*}
  is an open neighborhood of $0$ in $Y$.
  
 \itemch{b}
  We have a vector space isomorphism \(X/N \cong Y\).
  Since $Y$ is finite dimensional, this must also be topological homeomorphism.
  Using the previous exercise results and notations, 
  the isomorphism map is $f$, being continuous, so is \(\Lambda\).
  
  % Note: Here we did not use the fact that \(\Lambda\) is open.
\end{itemize}

%%%%%%%%%%%%%% 11
\begin{excopy}
If $N$ is a subspace of a vector space $X$, the 
\index{codimension}
\emph{codimension} id $N$ in $X$ is, by definition, 
the dimension of the quotient space \(X/N\).
Suppose \(0<p<1\) and prove that every subspace of a finite codimension
is dense in \(L^p\). (see Section~1.47).
\end{excopy}

Let $N$ be a subspace of finite codimension in \(L^p\) where \(0<p<1\).
Assume by negation that $N$ is not dense in \(L^p\).
Let \(\pi: L^p \rightarrow L^p/\overline{N}\) be the projection map.
Thus \(\overline{N} \subsetneq L^p\) and \(Y = L^p/\overline{N}\)
is a vector space of finite dimension. By previous exercise
and Theorem~1.21, $Y$ is isomorphic to \(\C^n\) and \(n\geq 1\). 
Let $G$ be some open convex set in \(\C^n\) not containing the origin.
For example (identifying $Y$ with  \(\C^n\) )
\begin{equation*}
G = \{(z_i)\in \C^n: 1<\Re(z_i),\Im(z_i)<2\}.
\end{equation*}
Now \(H = \pi^{-1}(G)\) is convex and open in \(L^{-1}\). 
But obviously \(\emptyset \neq H \subsetneq L^p\).
Cntradiction to the result in pages~35--36 \cite{RudinFA79} showing that
\(L^p\)
has no non trivial convex open subsets
when \(0<p<1\).


%%%%%%%%%%%%%% 12
\begin{excopy}
Suppose
 \(d_1(x,y) = |x-y|\), 
 \(d_2(x,y) = |\phi(x) - \phi(y)|\),
where \(\phi(x) = x/(1+|x|)\).
Prove that \(d_1\) and \(d_2\) are metrics in \R\ that induce
the same topology, although 
\(d_1\) is complete and \(d_2\) is not.
\end{excopy}

The identity map \((\C,\tau_{d_1}) \rightarrow (\C,\tau_{d_2})\) 
is clearly homeomorphism, thus the topologies are identical.
Now the sequence \(1,2,3\ldots\) is 
\index{Cauchy!sequence} 
Cauchy in the \(d_2\) metric space, but not in the \(d_1\) metric space.

%%%%%%%%%%%%%% 13
\begin{excopy} 
Let 
\label{ex:1:13}
$C$ be the vector space of all complex continuous functions on \([0,1]\).
Define
\begin{equation*}
 d(f,g) = \int_0^1 \frac{|f(x) - g(x)|}{1 + |f(x) - g(x)|} dx.
\end{equation*}
Let \((C,\sigma)\) be $C$ with the topology induced by the metric.
Let \((C,\tau)\) be the topological vector space defined by the seminorms
\begin{equation*}
 p_x(f) = |f(x)| \qquad (0\leq x \leq 1).
\end{equation*}
in accordance with Theorem~1.37.

\begin{itemize}
 \itemch{a}
  Prove that every \(\tau\)-bounded set in $C$ is also \(\sigma\)-bounded
  and that the identity map 
  \(\id:(C,\sigma) \rightarrow (C,\tau)\) therefore carries bounded ses
  into bounded sets.
 \itemch{b}
  Prove that
  \(\id:(C,\sigma) \rightarrow (C,\tau)\) is nevertheless not continuous,
  although it is sequentially continuous (by Lebesgue's convergence theorem).
  Hence \((C,\tau)\) is not metrizable. 
  (See Appendix A6, or Theorem~1.32.) Show also directly that \((C,\tau)\)
  has no countable local base.
 \itemch{c}
  Prove that every continuous linear functional on \((C,\tau)\) is of the form
  \begin{equation*}
    f = \sum_{i=1}^n c_i f(x_i)
  \end{equation*}
  for some choices of \seqxn\ in \([0,1]\) and some \(c_i\in \C\).
 \itemch{d}
  Prove that \((C,\sigma)\) contains no convex open sets other than 
  \(\emptyset\) and $C$.
 \itemch{e}
  Prove that \(\id:(C,\sigma) \rightarrow (C,\tau)\) is not continuous.
\end{itemize}
\end{excopy}


\begin{itemize}
 \itemch{a}
   \textbf{Note}: If \(\sigma\) would be replaced by a \(\sigma'\) 
   topology induced
   by \(L^1\) (using a simpler integrand) the identity mapping does \emph{not}
   always map \(\tau\)-bounded  sets to \(\sigma'\)-bounded sets.
   For let \(a_n = 1-1/(n+1)\) and \(d_n = (a_{n+1} - a_n)/2\).
   Now define \(f_n\in C[0,1]\) as following:
   \begin{equation*}
     f_n(x) = \left\{\begin{array}{ll}
                    0              & \quad x < a_n - d_n \\
                    (n/d_n)|x - a_n| & \quad a_n - d_n \leq x \leq a_n + d_n \\
                    0  & \quad x > a_n + d_n \\
                    \end{array}\right.
   \end{equation*}
   Clearly \(f_n\)'s are continuous, and for any \(x\in [0,1]\),
   the set \(F = \{f_n(x)\}\) is bounded. Thus $F$ is \(\tau\)-bounded,
   but since \(\int_0^1 |f_n|dm = 2n\) the set $F$ is not \(\sigma'\)-bounded.



   Back to the actual exercise.
   Let $E$ be a bounded set in the \(\tau\) topology.
   Assume by negation, $E$ is \emph{not} \(\sigma\)-bounded.
   Hence there exist some $0$-neighborhood 
   \begin{equation*}
   V_\epsilon = \{f\in C[0,1]: \sigma(0,f) < \epsilon\}
   \end{equation*}
   such that for any \(n > 0\), there exists \(f_n\in E \setminus nV_\epsilon\).
   Therefore, \(f_n / n \notin V\), that is
   \begin{equation} \label{eq:sigma:fn:eps}
    \sigma(f_n,0) = \int_0^1 |f_n/n|/(1 + |f_n|/n) dm 
      % = \int_0^1 |f_n|/(n + |f_n|) dm 
      \geq \epsilon.
   \end{equation}
   
   Since $E$ is \(\tau\) bounded, the set \(\{f_n(x)\}\) is bounded.
   Thus \(f_n/n \rightarrow 0\) pointwise and so 
   \begin{equation*}
   0 \leq \sigma(f_n,0) = 
   |f_n(x)/n|/\left(1 + |f_n(x)/n|\right)  \leq |f_n(x)/n| \rightarrow 0
   \end{equation*}

   Now since \(f_n \leq 1\), by Lebesgue's \index{Lebesgue}
   bounded convergence theorem, \(\sigma(f_n,0) \rightarrow 0\)
   which contradicts (\ref{eq:sigma:fn:eps}).

 \itemch{b}
   Let \(V = \{f\in C: d(f,0)<1/2\}\).
   This $0$-neighborhood does not contain any base \(\tau\)-neighborhood
   which are of the form 
   \begin{equation*}
     U = \bigcap_{i=1}^n p_{x_i}^{-1}\left((-\epsilon_i,\epsilon_i)\right)
   \end{equation*}
   (We can always find \(x \in [0,1]\setminus\{\seqxn\}\)).

   Sequential continuity is trivial 
   by Lebesgue's \index{Lebesgue} dominated theorem.

   If by negation \((C,\tau)\) is metrizable, then 
   Theorem~1.32 ({\small(d)\(\Rightarrow\)(a)}) \cite{RudinFA79}
   leads to a contradiction that the identity \emph{is} continuous.
   

   \textbf{Non existence of countable local base for \((C,\tau)\):}\newline
   For any \(\tau\)-open set $U$, the set \(U_x = \{|f(x)|: f\in U\}\)
   is not bounded except for \emph{finite} number of \(x_i\ni[0,1]\).
   If there would have been a countable local base, we could have
   found \(x\in[0,1]\) such that \(V(p_x,1)\) does not contain any
   member of the basis.
 
 \itemch{c}
   Let \(\Lambda\) such a functional. The set \(\Lambda^{-1}(-1,1)\)
   contains some base open set of the form 
   \begin{equation*}
    V = \{f\in C: \forall 1\leq i\leq n, |f(x_i)| < \epsilon_i\}
      \qquad \textrm{where}\; \epsilon_i > 0.
   \end{equation*}
   It is easy to see that if $f$ vansihes on \seqxn\ then \(\Lambda f = 0\).
   By looking at some \(e_i\in C\) such that \(e_i(x_j) = \delta_{i,j}\)
   we see that \(\Lambda f\) is determined, and in fact a linear combination
   of the values of \(\{f(x_i)\}_{i=1}^n\).

 \itemch{d}
   Let \(V\neq\emptyset\) be open and convex.
   \Wlogy\ \(0\in V\). We will show that \(V=C\).
   By the assumptions, there exists \(B(0,r)\subset \) with \(r>0\).
   We construction \(\{f_i\}_{i=1}^n\) such that \(f=\sum_{i=1}^n f_i\)
   and \(nf_i\in B(0,r)\). By convexity \(f\in V\) and we will be done.


   \textbf{Construction by Split of Unity}:\newline
   Take some integer $n$ such that \(n > 3/2\epsilon\).
   We will define $n$ continuous functions 
   \(e_i: [0,1] \rightarrow [0,1]\) such that 
   \(\sum_{i=1}^n e_i = 1\) and 
   their support  \(\supp(e_i) = [2i/(2n+1),(2i+3)/2n+1]\) as follows.
   Let \(d = 1/(2n+1)\) and \(I_i = [a_i,b_i] = [2(i-1)d, (2i+1)d]\)
   for \(1\leq i \leq n\).
   Clearly \(|I_i| = 3d = 3/(2n+1\) and \(|I_i \cap I_{i+1}| = d\).
   Now
   \begin{equation*}
     e_i(x) = \left\{\begin{array}{ll}
                     0          & x  \leq a_i \\
                     (x-a_i)/d  & a_i   \leq x \leq a_i + d \\
                     1          & a_i+d \leq x \leq b_i - d \\
                     (x-b_i)/d  & b_i+d \leq x \leq b_i \\
                     0          & b_i \leq x \\
                     \end{array}\right. \qquad (1\leq i \leq n).
   \end{equation*}
   Now clearly the \(e_i\) functions satisfy the requirements.
   Now we put \(f_i = n f e_i\) and 
   thus \(d(0,f_i) < 3d = 3/(2n+1) < \epsilon\) as desired.

   

 \itemch{e}
   Let \(U=\{f\in C: |f(0)| < 1\}\). It is clearly 
   an open neighborhood of $0$.
   But $U$ cannot be open in \((C,\sigma)\), since for any \(\epsilon>0\)
   the base \(\sigma\)-neighborhood
   \(V_\epsilon = \{f\in C: d(f,0)<\epsilon\}\)
   has \(g\in C\setminus U\) defined as:
   \begin{equation*}
   g(x) = \left\{\begin{array}{ll}
                2(\epsilon-x)/\epsilon & x \leq \epsilon \\
                0                      & x \geq \epsilon 
                \end{array}\right. .
   \end{equation*}
\end{itemize}

%%%%%%%%%%%%%% 14
\begin{excopy}
Put \(K=[0,1]\) and defined \(\scrD_K\) as in Section~1.46.
Show that the following three families of seminorms (where \(n=0,1,2,\ldots\))
define the same topology on \(\scrD_K\), 
if \(D = d/dx\):
\begin{itemize}
 \itemch{a} \(\|D^n f\|_\infty = \sup\{|D^n f(x)|: -\infty < x < \infty\}\).
 \itemch{b} \(\|D^n f\|_1  = \int_0^1 |D^n f(x)|dx\).
 \itemch{c} \(\|D^n f\|_2  = \left\{\int_0^1 |D^n f(x)|^2dx\right\}^{1/2}\).
\end{itemize}
\end{excopy}

Since the measure of \([0,1]\) is $1$, we easily see that
for any measurable \(g(x)\) on \([0,1]\) we have
\(\|g\|_p \leq \|g\|_\infty \) for \(1\leq p \leq \infty\). 
Setting \(g=D^n f\) we get
 \(\|D^n f\|_2 \leq \|D^n f\|_\infty\).

\index{Jensen!inequality}
By Jensen's inequality, 
or more directly from 
\index{Schwartz!inequality}
Schwartz inequality (See \cite{RudinRCA80}, Theorems~3.3, 3.5),
we also see that \(\|D^n f\|_1 \leq \|D^n f\|_2\).

Thus, \( \|\cdot\|_1 \leq \|\cdot\|_2 \leq \|\cdot\|_\infty\)
and by compaing at balls of the above seminorm
we see that 
the topology of 
\(\|\cdot\|_\infty\) (c) is finer than the topology of \(\|\cdot\|_2\) (b)
and similarly
the topology of 
\(\|\cdot\|_2\) (b) is finer than the topology of \(\|\cdot\|_1\) (a).
Therefore, it suffices to show that the topology of (a) is finer than or equal
to that of (c).

Consider a basis open set of the (c)-topology 
\begin{equation*}
 V = \{f\in \scrD_K: \|D^n f\|_\infty < \epsilon\}.
\end{equation*}
It is easy to see that there exist \(\epsilon_1,\epsilon_2>0\) such that
\begin{equation} \label{eq:1:UinV}
 U = \{f\in \scrD_K: 
          \|D^n f\|_1 < \epsilon_1 
          \,\wedge\,
          \|D^{n+1} f\|_1 < \epsilon_2 
      \} \subset V.
\end{equation}

This is by applying the equality
\begin{equation*}
D^n f(x_1) = D^n f(x_0) + \int_{x_0}^{x_1} D^{n+1} f(x)dx
\end{equation*}
we can actually take \(\epsilon_1 = \epsilon_2 = \epsilon/2\) as follows.

Let \(f\in U\).
Pick some \(x_0\in[0,1]\) 
such that \(|D^n f(x_0)| \leq \int_0^1 D^{n+1} f(x)dx\)
that obviously must exist. Now
\begin{equation*}
\sup_{x\in[0,1]} |D^n f(x)| \leq |D^n f(x_0)| + \int_0^1 |D^{n+1} f(x)|dx
 \leq \epsilon_1 + \epsilon_2 = \epsilon.
\end{equation*}
Thus, \(f\in V\) and (\ref{eq:1:UinV}) must hold.

%%%%%%%%%%%%%%
\begin{excopy}
Prove that the spaces \(C(\Omega)\) (Section~1.44) do not have the
\index{Heine-Borel}
Heine-Borel property.
\end{excopy}

Since the cardinality of \(\Omega\) is greater than \(\aleph_0\)
and it is a union of countable number of compact sets \(K_i\), we can
take some compact set $K_n$ with infinitely many points,
say \(\{x_i\}\subset K\). Choose continuous 
functions \(\{f_i\}: \Omega\rightarrow [0,1]\) 
such that
\(f_k(x_i) = 0\) if \(i < k\) and \(f_k(x_k) = 1\).
The set \(F = \{f_i:i\in \N\}\) is bounded, it has no limit
since the with seminorm \(p_n(f) = \sup\{|f(x)|: x\in K_n\}\)
the base neighborhood \(V = V(p_n,1/2)\) contains none fo the \(f_i\) functions.
Thus $F$ is trivially closed. Now the infinite family \(\{f_i+V\}\) of
opens sets covers $F$, but if \(i\neq j\) then \(f_j\notin f_i+V\)
and there is no subcovering, in particulat no finite subcovering.
Hence $F$ is not compact.



%%%%%%%%%%%%%%
\begin{excopy}
Prove that the topology of \(C(\Omega)\) does not depend on the particular
choice of \(\{K_n\}\), as long as this sequence satisfies the conditions
specified in Section~1.44. Do the same for \(C^\infty(\Omega)\) (Section~1.46).
\end{excopy}

Let \(\{K'_i\}\) be an alternative set of compact sets with similar
requirements, that gives the corresponding \(\{{p'}_i\}\) seminorms.
Now since \(K_i = \inter{K_{i+1}}\) we actually have an open covering
\(\Omega = \cup_i \int K_i\). Hence for each compact \(K'_i\)
there exist some index \(j\geq1\) such that \(K'_i \subset K_j\).
Thus for any \(f\in C(\Omega)\), we have
\({p'}_i(f) \leq p_j(f)\). this shows that the (``origina'') topology
is finer or equal to that indeuced by \(\{K'_i\}\). By symmetrical argument
we show the opposite inclusion of topologies, hence the topologies
are independent of the choice of the compact increasing covering.

Now for the \(C^\infty(\Omega)\). We generlize the set of seminorms
from the original 
\begin{equation} \label{eq:pn:orig}
 p_N(f) = \max\{|D^\alpha f(x)|: x\in K_N, |\alpha| \leq N\}
\end{equation}
as apears in section 1.46 of \cite{RudinFA79}, into
\begin{equation*}
 p_{M,N}(f) = \max\{|D^\alpha f(x)|: x\in K_N, |\alpha| \leq M\}
\end{equation*}
Now the set of seminorm is richer, but still induces the same topology
since for any \(f\in C^\infty\) we have
\begin{equation*}
 p_{m,n}(f) \leq p_{\max(m,n),\max(m,n)} (f) =  p_{\max(m,n)}(f).
\end{equation*}
The last, right expression should be referred as the ``original'' seminorm
as in (\ref{eq:pn:orig}).

Now we can vary the compact sets as was done 
in the beginning of this exercise with \(C(\Omega)\)
\emph{without} effecting the order of \(\alpha\) derivations.
This will show that the choice of compact sets has 
has no effect on the topology of \(C^\infty(\Omega)\) as well.

%%%%%%%%%%%%%%
\begin{excopy}
In the setting of Section~1,46, prove that \(f\rightarrow D^\alpha f\)
is a continuous mapping of 
\(C^\infty(\Omega)\) into \(C^\infty(\Omega)\)  and also of
\(\scrD_K\) into \(\scrD_K\),
for every multi-index \(\alpha\).
\end{excopy}

Almost by definition. Let \(a=|\alpha\) and look at some base open set
\begin{equation*}
 V = \{f\in C^\infty(\Omega): p_n(f)< 1/n\}
\end{equation*}
(Note that \(V_n\) are getting finer as $n$ increases).
We need to show that the inverse image set \(W = (D^\alpha)^{-1}(V)\) is open.

But actually
\begin{eqnarray*}
 W 
 &=&  (D^\alpha)^{-1}(V) \\
 &=&  \{f\in C^\infty(\Omega): D^\alpha f \in V\} \\
 &=&  \{f\in C^\infty(\Omega): p_{n+a}(D^\alpha f)< 1/(n+a)\}
\end{eqnarray*}
which is another base open set.

Thus \(\alpha\)-derivation is continuous on \(C^\infty(\Omega)\)
and the same goes for \(\scrD_K\) for it is merely
a restriction. we just need to observer that 
\((D^\alpha)^{-1}(\scrD_K) \subset \scrD_K)\).



%%%%%%%%%%%%%%
\begin{excopy}
The seminorms \label{ex:1:18}
\begin{equation*}
 p_n(f) = \sup\{|f(x)|: -n \leq x \leq n\}
\end{equation*}
induce the metric
\begin{equation*}
  d(f,g) = \sum_{n=1}^\infty \frac{2^{-n} p_n(f-g)}{1 +  p_n(f-g)}
\end{equation*}
in the space \(C(R)\); compare Section~1.46 and remark~(c) of
Section~1.38.
Define
\begin{equation*}
 f(x) = \max(0,1-|x|), \qquad g(x) = 100f(x-2), \qquad 2h = f+g,
\end{equation*}
and compute that
\begin{equation*}
 d(f,0)=\frac{1}{2}, \qquad d(g,0) = \frac{50}{100}, 
                     \qquad d(h,0) = \frac{1}{6}+\frac{50}{102}.
\end{equation*}
The balls with radius \(\frac{1}{2}\) are therefore not convexity, although $d$
is compatible  with the usual locally convex topology of \(C(R)\).
\end{excopy}

% Here the compacts are \(K_n = [-n,n]\).
The balls (\(0<r<1\)) are \emph{not} convex.
Let $N$ be such that 
\begin{equation} \label{eq:1:r:2N}
2^{-N} \leq r \leq 2^{-(N-1)}.
\end{equation}
We construct monotonic functions $f$, $g$ such that
\begin{equation*}
p_n(f) = p_n(g) = f(n), \qquad d(g,0) \leq d(f,0) = r
\end{equation*}
but for \(h=(f+g)/2\), we have \(d(h,0)> r\).

Define
\begin{equation*}
f(x) = \left\{\begin{array}{lc}
              0               & x \leq N - 1 \\
              \alpha(x-(N-1)) & N -1 \leq x \leq N  \\
              \alpha          & x \geq N 
              \end{array}\right..
\end{equation*}
Where \(\alpha = r/(2^{-N+1} - r)\) 
and so 
\begin{equation*}
d(f,0) = \sum_{n=N}^\infty 2^{-n}\alpha/(1+\alpha) = 2^{-N+1}\alpha(1+\alpha)
       = r.
\end{equation*}

The function $g$ will be defined in two stages. First
\begin{equation*}
g(x) = \left\{\begin{array}{lc}
              0               & x \leq N - 1 \\
              \beta(x-(N-1)) & N -1 \leq x \leq N  \\
              \end{array}\right.
\end{equation*}
Where \(\beta\) satisfies the equation
\begin{equation*}
 2^{-N}\beta/(1+\beta) = r - 2^{-N}
\end{equation*}
Which must exist by (\ref{eq:1:r:2N}).

We can already have the estimation
\begin{equation*}
d(g,0) \leq 2^{-N}\beta/(1+\beta) + 2^{-N} \leq r.
\end{equation*}

The definition of $g$ will be completed  for \(x\geq N\) once  $h$
is defined. 
We require \(h=(f+g)/2\) that is \(g(x)=2h(x)-\alpha\) for \(x\geq N\).

For \(x\leq N\) we should define 
\begin{equation*}
h(x) = (f(x)+g(x))/2.
\end{equation*}

Now \(r - 2^{-N} < r/2\) and so \(\beta < (\alpha+\beta)/2 < \alpha\)
(since \(x \rightarrow x/(1+x)\) is increasing).
Hence, \(\beta < h(N) = (\alpha+\beta)/2\), and so 
\begin{equation*}
\epsilon = 2^{-N}h(N)/(1+h(N)) - (r - 2^{-N}) > 0.
\end{equation*}
Let \(\gamma\) satisfy 
\begin{equation*}
 \sum_{n=N+1}^\infty 2^{-n} \gamma/(1+\gamma) = 2^{-N} \gamma/(1+\gamma) 
 > 2^{-N} - \epsilon/2.
\end{equation*}
Now we complete the define of $h$
\begin{equation*}
 h(x) = \left\{\begin{array}{lc}
               (\gamma - \beta)(x-N) + \beta &  N \leq x \leq N+1 \\
               \gamma                        & x \geq N + 1
               \end{array}\right..
\end{equation*}

This guarantee that \(d(h,0) \geq r\) and the ball \(B(0,r)\) is not convex.

\textbf{Computations:}

\begin{itemize}

 \item 
 We observe that \(p_n(f) = 1\) for all $n$.
 \begin{equation*}
 d(f,0) = \sum_{n=1}^\infty 2^{-n} 1/(1 + 1) = \sum_{n=0}^\infty 2^{-n} = 1/2
 \end{equation*}

 \item
 We observe that \(p_1(g) = 0\) and \(p_n(g) = 100\) for \(n\geq 1\).
 \begin{equation*}
 d(g,0) = \sum_{n=2}^\infty 2^{-n} 100/(100 + 1) = 
      (1/2)\cdot (100/101) = 50/101.
 \end{equation*}


 \item
 We observe that \(p_1(h) = 1/2\) and \(p_n(h) = 50\) for \(n\geq 2\).
 \begin{eqnarray*}
 d(h,0) 
  &=& (1/2)\cdot\frac{1/2}{1+1/2} + \sum_{n=2}^\infty 2^{-n} \cdot 50/(50+1) \\
  &=& (1/2)\cdot(1/3) + (1/2)\cdot(50/51) = 1/6 + 50/102 = 67/102.
 \end{eqnarray*}

\end{itemize}


%%%%%%%%%%%%%% 19
\begin{excopy}
Suppose $M$ is a dense subspace of a topological vector space $X$,
$Y$ is an $F$-space, and \(\Lambda:M\rightarrow Y\) is a continuous
(relative to the topology that $M$ inherits from $X$) and linear.
Prove that \(\Lambda\) has a continuous linear extension 
\(\tilde{\Lambda}: X\rightarrow Y\).

\qquad \emph{Suggestion:} Let \(V_n\) be balanced neighborhoods of $0$ in $X$
such that \(V_n + V_n \subset V_{n-1}\) and such that 
\(d(0,\Lambda x) < 2^{-n}\) if \(x\in M\cap V_n\).
If \(x\in X\) and \(x_n \in (x+V_n)\cap M\), show that
\(\{\Lambda x_n\}\) is a Cauchy \index{Cauchy!sequence} sequence in $Y$, 
and define 
\(\tilde{\Lambda} x\) to be its limit.
Show that \(\tilde\Lambda\) is well defined, that 
\(\tilde{\Lambda} x = \Lambda x\) if \(x\in M\), and that \(\tilde\Lambda\)
is linear and continuous.
\end{excopy}

Using the suggestion. 
For any \(n > 0\) we can find a and open balanced  neighborhood \(V_n\)
such that (distance of point to set) \(d(0,\Lambda V_n) < 2^{-n}\)
and (for \(n>1\)) we also have \(V_n + V_n \subset V_{n-1}\).

Let \(x\in X\setminus M\). By density of $M$
we can find a sequence \(\{x_n\}\) such that
\(x_n \in (x+V_n)\cap M\). 
% We will show that \(\{\Lambda x_n\}\) is a Cauchy sequence.
Assume \(m<n\), then 
\begin{equation*}
x_m - x_n \in (x + V_m) - (x + V_n) = V_m - V_n = V_m + V_n
 \subset V_m + V_m \subset V_{m-1}.
\end{equation*}
Hence \(d(\Lambda x_m, \Lambda x_n) = d(0, \Lambda (x_m - x_n)) < 2^{m-1}\)
and \(\{\Lambda x_n\}\) is a Cauchy sequence.
We define \(\tilde{\Lambda}x\) to be the limit. 
If \(\{w_n\}\) is another Cauchy sequence, such that \(w_n\in x+V_n\)
then it is easy to see that the comobined sequence:
\begin{equation*}
 \Lambda x_1, \Lambda w_1, \Lambda x_2, \Lambda w_2, \ldots 
 \Lambda x_n, \Lambda w_n, \ldots
\end{equation*}
is a Cauchy sequence as well. Since its limit is the same as the limit
of any of its subsequences, the extension \(\tilde{\Lambda}\) is well defined.
Continuity is immediate from the density of $M$.
Since the addition in $X$ and scalar muliplication are continuous,
if \(x_m\rightarrow x\) and \(y_n \rightarrow y\) 
where \(x_m,y_n\in M\) then
it is easy to see that 
\begin{equation*}
\lim_{m,n\rightarrow \infty} \Lambda(ax_m + by_n) 
= \lim_{m,n\rightarrow \infty} a\Lambda x_m + b \Lambda y_n
= a \lim_{m\rightarrow \infty} \Lambda x_m + 
  b \lim_{n\rightarrow \infty} \Lambda y_n.
\end{equation*}
and linearity of the extension is established.
 

%%%%%%%%%%%%%% 20
\begin{excopy}
For each real number $t$ and each integer $n$, define \(e_n(t) = e^{int}\),
and define 
\begin{equation*}
  f_n = e_{-n} + ne_n \qquad (n=1,2,3,\ldots).
\end{equation*}
Regard these functions as members of \(L^2(-\pi,\pi)\).
Let \(X_1\) be the smallest closed subspace of \(L^2\) that contains 
\(e_0,e_1,e_2,\ldots\), and let \(X_2\) be the smallest closed subspace
of \(L^2\) that contains \(f_1,f_2,f_3,\ldots\). Show that \(X_1+X_2\)
is dense in \(L^2\) but not closed.
For instance the vector
\begin{equation*}
 x = \sum_{n=1}^\infty n^{-1}e_{-n}
\end{equation*}
is in \(L^2\) but not in \(X_1+X_2\). (Compare with Theorem~1.42.)
\end{excopy}

For \(n>0\), we have \(e^{i(-n)t} = f_n - n e_n\) and so
the subspace \(X_1+X_2\) is dense in \(L^2\) since it contains 
all of \(\{e^{int}\}\).


The fact that \(x\in L^2\) is application of the 
\index{Riesz-Fischer}
\emph{Riesz-Fischer} theorem (see \cite{RudinRCA80} 4.17, 4.26).
The analysis there also shows that \(\{e_n\}_{n\in\Z}\) are
linearly independent.


Let us now show that \(X_1\cap X_2 = \{0\}\).
Let \(g(t)\in X_1\cap X_2\) and \(g(t)\neq 0\).
We have a unique representation \(g=\sum_{n\in\Z} a_n e_n\),
where as we know 
\(a_n = \langle g,e_n\rangle = \int_{-\pi}^{+\pi} g(t)e^{int}dt\).
By negation, \(x\neq 0\) and so there must be at least some \(n\in\Z\)
such that \(a_n\neq 0\).
There are three cases:

\textbf{Case (i)}:  There is some \(j<0\) for which \(a_j\neq 0\).
For any finite linear combination of \(X_1\)'s basis \(\{e_k\}_{k\geq 0}\),
that is \(u = \sum_{k=0}^m u_i e_i\) we have

\begin{eqnarray*}
\|g - u\|
&=& \left\|\sum_{n\in Z} a_i e_n - \sum_{k=0}^m u_k e_k\right\| \\
&=& \left\|\left(\sum_{n<0} a_i e_n + \sum_{n\geq 0} a_i e_n\right)
          - \sum_{k=0}^m u_k e_k\right\| \\
&\geq & \left\| \sum_{n<0} a_n e_n\right\| \\
&\geq & | a_j | > 0
\end{eqnarray*}

Thus \(d(g,X_1) \geq |a_j| > 0\) contradicts the assumption of \(g\in X_1\).

\textbf{Case (ii)}:  
For all \(n<0\), we have \(a_n=0\), but \(a_0\neq 0\).
For any finite linear combination of \(X_2\)'s basis \(\{f_k\}_{k\geq 1}\),
that is \(w = \sum_{k=1}^m w_k f_k\) we have
we have:
\begin{equation*}
\|g - w\| \geq |a_0| > 0.
\end{equation*}
Thus \(d(g,X_2) \geq |a_0| > 0\) which contradicts 
the assumption of \(g\in X_2\).

\textbf{Case (iii)}:  
For all \(n\leq0\), we have \(a_n=0\). Let \(j \geq 1\) be the minimal
such that \(a_j \neq 0\).
Similarly, For any finite linear combination of \(\{f_k\}_{k\geq 1}\),
that is \(w = \sum_{k=1}^m w_k f_k\) we have

\begin{eqnarray}
\|g - w\| 
&=& \left\|\sum_{n\in Z} a_i e_n - \sum_{k=1}^m w_k f_k\right\| \notag \\
&=& \left\|\sum_{n\geq 0} a_i e_n - 
           \sum_{k=1}^m (w_k e_{-k} + kw_k e_k)\right\|  \notag \\
&\geq& \left\| a_j e_j - (w_j e_{-j} + jw_j e_j)\right\| \label{eq:justj} \\
&=&  \left( (a_j - jw_j)^2 + w_j^2\right)^{1/2}  \label{eq:g-w:last}
\end{eqnarray}

We get the inequality (\ref{eq:justj}) by restricting our attention
to the projection over \(\langle e_j \rangle\).
We observe the last expression with the sqaure root.
As a function of \(w_j\), it is a quadratic polynomial.
\begin{equation*}
p(w_j) = (j^2+1)w_j^2 + 2ja_j w_j + a_j^2
\end{equation*}
whose determinant is 
\begin{equation*}
\Delta(p) = 4j^2a_j^2 - 4(j^2+1)a_j^2 = -4a_j^2 < 0.
\end{equation*}
Thus $p$ has no real solutions, and 
since \(p(0) = a_j^2 > 0\), the last term in (\ref{eq:g-w:last})
is positive and has a lower bound independent of \(w_j\).
Again showing \(d(g,X_2) > 0\) and contradicting the assumption
of \(g\in X_2\).

Now that we establish that \(L^2\) is an (inner) direct sum
of \(X_1 + X_2\) we proceed with a lemma.

\begin{llem}
Let $X$ be a  topological vector space
and \(X_1\), \(X_2\) closed subspaces such that
\(X_1 + X_2 = X\) is an inner direct sum.
Then for \(i=1,2\), the mappings \(P_i: X \rightarrow X_i\)
defined by \(P_i(x) = x_i\) where \(x = x_1 + x_2\) 
\textnormal{(unique, by \(X_1+X_2\) being a direct sum)}
are continuous.
\end{llem}

\textbf{Proof.}
The mapping \(P_1\) gives
the algebraic isomorphism \(\pi_1: X/X_2 \cong  X_1\).
By the definition of the quotient topology of \(X/X_2\)
\(\pi_1\) is also a topological isomorphism.
By Exercise~\ref{ex:1:9}, the mapping \(P_1\) is continuous.
Analogically, \(P_2\) is continuous too.
\proofend

Now we put \(x_m = \sum_{n=1}^m n{-1}e_{-n}\)
Clearly \(x_m \rightarrow x\). If by negation \(X_2\) is closed,
then \(P_1\) --- defined as in the above lemma ---
is continuous. But \(n^{-1}e_{-n} = n^{-1}f_n - e_n\), thus
\begin{equation*}
x_m = \sum_{n=1}^m n{-1}e_{-n}
  = \sum_{n=1}^m n^{-1}f_n - e_n
\end{equation*}
and
\begin{equation*}
P_1(x_m)
  = P_1\left(\sum_{n=1}^m n^{-1}f_n \right) +
    P_1\left(\sum_{n=1}^m e_n \right) = \sum_{n=1}^m e_n.
\end{equation*}
Surely the last expression does not converge as \(m\rightarrow \infty\)
contradiction to \(P_1\) being continuous.


%%%%%%%%%%%%%%
\begin{excopy}
Let $V$ be a neighborhood of $0$ in a topological vector space $X$.
Prove that there is a real continuous functions $F$ on $X$ such that
\(f(0) = 0\) and \(f(x)=1\) outside $V$.
(Thus $X$ is a 
\index{completely regular}
\emph{completely regular} topological space).
\emph{Suggestion:} Let \(V_n\) be balanced neighborhoods of $0$ 
such that \(V_1+V_1 \subset V\) and
such that \(V_{n+1}+V_{n+1} \subset V_n\).
Construct $f$ as in the proof of Therefore~1.24.
Show that $f$ is continuous at $0$ and that
\begin{equation*}
 |f(x) - f(y)| \leq f(x-y).
\end{equation*}
\end{excopy}

We follow the suggestion. 
The existence of such neighborhoods \(\{V_i\}_{n=1}^\infty\)
is provided by \cite{RudinFA79} Sections~1.6 and~1.10.

We define 
\begin{equation*}
D = \left\{r\in \Q: 
      r = \sum_{n=1}^m c_n(r)2^{-n},\; \textrm{where}\;c_i(r)=0,1,\;
      1\leq m < \infty \right\}.
\end{equation*}


We put \(V_0 = V\) and define absorbing ``neighborhood'' function
\(A: D \rightarrow P(X)\):
\begin{equation*}
A(r) = \left\{\begin{array}{l@{\qquad}c}
  X & r = 1 \\
  \sum_{i=1}^m c_i V_i & r = \sum_{n=1}^m c_n(r)2^{-n} < 1
  \end{array}\right..
\end{equation*}

We now define the desired \(f: X \rightarrow [0,1]\) as:
\begin{equation*}
f(x) = \inf\{r\in D: x\in A(r)\}.
\end{equation*}

Clearly \(f(X) \subset [0,1]\) and
\(f(x) = 1\) if \(x\notin V\) and \(f(0) = 0\).


Let \(\epsilon > 0\), take $n$ such that \(2^{-n} < \epsilon\).
Now $f$ is continuous at $0$ because \(f(V) \subset (-\epsilon, +\epsilon)\).

Now we show the inequality. If \(f(x) < f(y)\) then it si trivial,
so we can now assume \(f(x)\geq f(y)\).
Since \(f(x) - f(y) < 1 - 0 = 1\),
we can exclude the trivial case of  \(f(x-y) \geq 1\) and
further assume that \(f(x-y) < 1\).

Now for any \(r,s\in D\) such that 
\(f(x-y)\leq r\) and
\(f(y)\leq s\) we have
\(x-y \in  A(r)\) and
\(y \in  A(s)\) and so \(x=(x-y)+y \in A(r) + A(s) \subset A(r+s)\)
which shows that \(f(x) \leq r + s\). Since $D$ is dense in  \([0,1]\)
we have \(f(x) \leq f(x-y) + f(y)\).

With the inequality established, we can show that $f$ is continuous
at any \(x_0\in X\) since if \(f(V) \subset (-\epsilon, \epsilon)\)
for a balanced neighborhood $V$ of $0$ 
then for \(x \in x_0 + V\) we have
\begin{equation*}
|f(x)- f(x_0)| \leq f(x - x_0) < \epsilon.
\end{equation*}


%%%%%%%%%%%%%% 22
\begin{excopy}
If $f$ is a complex  function defined on the compact interval 
\(I=[0,1]\subset \R\), define
\begin{equation*}
 \omega_\delta(f) = \sup \{|f(x) - f(y)|: |x-y|\leq \delta, x\in I, y\in I\}
\end{equation*}
If \(0<\alpha \leq 1\), the corresponding 
\index{Lipschitz space}
\emph{Lipschitz space} \(\Lip \alpha\) consists of all $f$ for which
\begin{equation*}
 \|f\| = |f(0)| + \sup\{\delta^{-\alpha} \omega_\delta(f): \delta > 0\}
\end{equation*}
is finite. Define
\begin{equation*}
 \lip \alpha = \{f\in \Lip \alpha: 
           \lim_{\delta\rightarrow 0} \{\delta^{-\alpha} \omega_\delta(f) = 0\}
\end{equation*}
Prove that \(\Lip \alpha\) is a 
\index{Banach}
Banach space and that \(\lip \alpha\) is a closed subspace of \(\Lip \alpha\).
\end{excopy}

The triangle inequality is trivial
from 
\begin{equation*}
|(f+g)(x) - (f+g)(y)| = |f(x) + g(y) - f(x) - f(y)| 
 \leq |f(x) - f(y)| + |g(x) - g(y)|
\end{equation*}
followed by
\begin{eqnarray*}
 \omega_\delta(f+g)
 &=& \sup \{|(f+g)(x) - (f+g)(y)|: |x-y|\leq \delta,\, x,y\in I\} \\
 &\leq& \sup \{|f(x)-f(y)| + |g(x)-g(y)|: |x-y|\leq \delta,\, x,y\in I\}\\
 &\leq&   \sup \{|f(x)-f(y)|: |x-y|\leq \delta,\, x,y\in I\}
        + \sup \{|g(x)-g(y)|: |x-y|\leq \delta,\, x,y\in I\}\\
 &=& \omega_\delta(f) + \omega_\delta(g).
\end{eqnarray*}
Thus
\begin{eqnarray*}
\|f+g\|
&=&    |(f+g)(0)| + \sup\{\delta^{-\alpha} \omega_\delta(f+g): \delta > 0\} \\
&\leq&    |f(0)| + \sup\{\delta^{-\alpha} \omega_\delta(f): \delta > 0\} +
          |g(0)| + \sup\{\delta^{-\alpha} \omega_\delta(g): \delta > 0\} \\
&\leq& \|f\| + \|g\|.
\end{eqnarray*}

Now for any \(f\in \C^{[0,1]}\) there exists \(x\in[0,1]\) such that
for \(0<\alpha \leq 1\),
\begin{equation*}
\|f\|_\infty = |f(x)| \leq 
 |f(0)| + |f(x) - f(0)| \leq |f(0)| + \omega_x(f) \leq 
 |f(0)| + x^{-\alpha} \omega_x(f) \leq \|f\|_\alpha
\end{equation*}
Therefore, the topology induced by this norm is stronger than 
the supremum norm and completeness is easily established.

For the last claim, let \(f_n\in \lip\alpha\)
such that 
\begin{equation*}
f_n \xrightarrow{\|\cdot\|_\alpha} f \in \Lip \alpha.
\end{equation*}
we have
\begin{eqnarray*}
\lim_{\delta\rightarrow 0} \delta^{-\alpha} \omega_\delta(f_n - f)
 &\leq& \sup_{\delta > 0} \delta^{-\alpha} \omega_\delta(f_n - f) \\
 &\leq&  \left| \sup_{\delta > 0} \delta^{-\alpha} \omega_\delta(f_n) -
                \sup_{\delta > 0} \delta^{-\alpha} \omega_\delta(f) \right|
       \xrightarrow{n\rightarrow\infty} 0.
\end{eqnarray*}
Hence \(f\in \lip\alpha\).

%%%%%%%%%%%%%%
\begin{excopy}
Let $X$ be a vector space of all continuous functions on the open segment
\((0,1)\). For 
\(f\in X\) and \(r>0\), let \(V(f,r)\) consist of all \(g\in X\) such that
\(|f(x)-g(x)|<r\) for all \(x\in(0,1)\).
Let \(\tau\) be the topology on $X$ that these sets \(V(f,r)\) generate.
Show that addition is \(\tau\)-continuous but scalar multiplication is not.
\end{excopy}

We have 
\begin{equation*}
V(f,r) + V(g,s) \subset V(f+g,r+s)
\end{equation*}
From which implies the continuity of the addition.

To show that scalar multiplication is not continuous, let's look
at the function \(f(x) = 1/x\) and look for a decreasing sequence
\(a_n \rightarrow 1\) such that \(a_n f \notin V(f,\epsilon)\) for any 
\(\epsilon > 0\).
Let \(a_n = (n+1)/n\). Now \(a_nf(x)-f(x) = 1/(nx)\)
Clearly for any \(n<\infty\) we can find \(x_n\in(0,1)\)
such that 
\(x_n < 1/(n\epsilon)\). Now 
\(1/(nx_n) > \epsilon\) and thus \(a_nf(x_n)-f(x_n) > \epsilon\)
and so \(a_nf \notin V(f,\epsilon)\).


%%%%%%%%%%%%%% 24
\begin{excopy}
Show that the set $W$ that occurs in the proof of Theorem~1.14 need
not be convex, and that $A$ need not be balanced unless $U$ is convex.
\end{excopy}


The balls of the space in Exercise~\ref{ex:1:18} are balanced
but not convex. so are their interiors which form 
a local base for the topology, $W$ could be such a ball.

Consider the space \(\C^2\).
For \(n=1,2\) define
\begin{equation*}
 R_n = \left\{(z,0) \in \C^2: |z| = n\right\}.
\end{equation*}
Let $V$ be a \((0,0)\)-neighborhood such that
\begin{equation*}
 R_1 \cap V = \emptyset \qquad \textnormal{and} \qquad R_2 \subset V.
\end{equation*}
Now we ``symmetrize'' $V$ by \(U = \cup_{|\alpha|=1} V\).
We can see that also for $U$
\begin{equation*}
 R_1 \cap U = \emptyset \qquad \textnormal{and} \qquad R_2 \subset U.
\end{equation*}
Clearly \(A=\cap_{|\alpha|=1} U = U\), but $U$ is not balanced,
Since \(R_1 \subset (1/2)U\).

%%%%%%%%%%%%%%%
\end{enumerate}
%%%%%%%%%%%%%%%

 %%%%%%%%%%%%%%%%%%%%%%%%%%%%%%%%%%%%%%%%%%%%%%%%%%%%%%%%%%%%%%%%%%%%%%%%
%%%%%%%%%%%%%%%%%%%%%%%%%%%%%%%%%%%%%%%%%%%%%%%%%%%%%%%%%%%%%%%%%%%%%%%%
%%%%%%%%%%%%%%%%%%%%%%%%%%%%%%%%%%%%%%%%%%%%%%%%%%%%%%%%%%%%%%%%%%%%%%%%
\chapterTypeout{Completeness}

%%%%%%%%%%%%%%%%%%%%%%%%%%%%%%%%%%%%%%%%%%%%%%%%%%%%%%%%%%%%%%%%%%%%%%%%
%%%%%%%%%%%%%%%%%%%%%%%%%%%%%%%%%%%%%%%%%%%%%%%%%%%%%%%%%%%%%%%%%%%%%%%%
%%%%%%%%%%%%%%%%%%%%%%%%%%%%%%%%%%%%%%%%%%%%%%%%%%%%%%%%%%%%%%%%%%%%%%%%
%%%%%%%%%%%%%%%%%%%%%%%%%%%%%%%%%%%%%%%%%%%%%%%%%%%%%%%%%%%%%%%%%%%%%%%%
\section{Exercises} % pages 36-40

%%%%%%%%%%%%%%%%%
\begin{enumerate}
%%%%%%%%%%%%%%%%%

%%%%%%%%%%%%%% 1
\begin{excopy}
If $X$ is an infinite-dimensional topological vector space which is the union
of countably many finite-dimensional subspaces, 
prove that $X$ is the 
\index{first category}
\index{category!first}
first category in itself.
Prove that therefore no inifinite-dimensional $F$-space has a countable
Hamel basis.

(A set \(\beta\) is a 
\index{Hamel basis}
\emph{Hamel basis} for a vector space $X$ if \(\beta\) is a maximal linearly 
independent subset of $X$. Alternatively, \(\beta\) is a Hamel basis if every
\(x\in X\) has a unique representation as a \emph{finite} linear combination
of elements of \(\beta\).)
\end{excopy}

A finite subspace is always closed and if it is a proper subspace
it has empty interior. Thus a union of finite-dimensional subscapes
of $X$ is of first category. An $F$-space is complete and by
\index{Baire}
Baire category theorem~2.2 is of second category, and therefore
cannot be a countable union of finite-dimensional subspaces.

%%%%%%%%%%%%%% 2
\begin{excopy}
Sets of first and second category 
\index{second category}
\index{category!second}
are ``small'' and ``large'' in a topological sense.
These notions are different when
``small'' and ``large'' are understood in the sense of measure,
even when the measure is initimately related to the topology.
To see this, construct a subset of the unit interval which is the
first category but whose Lebesgue measure is $1$.
\end{excopy}

Following \cite{Gelb1996} chapter~8, examples~4, 19 and~20.

First, for any \(\alpha\in(0,1)\)
we will construct a Cantor-like set \(C_\alpha\)
that is closed (even perfect), nowhere dense and \(m(C_\alpha)=\alpha\).
The construction is made of steps. 
Put \(\beta = 1 - \alpha\).
In step-$0$, 
we start from the closed unit interval \(K_0 = [0,1]\).
For each step $n$ we have a compact \(K_n\) which 
consists of \(2^n\) closed intervals of equal length.
In the $n$th step we remove from each closed interval $I$,
an open sub-interval $J$ such that the centers of $J$ and $I$ are the same
and \(m(J) = 2^{-2n}\beta\). Thus the resulted removal is
\begin{equation*}
m(K_n) - m(K_{n+1}) = m(K_n \setminus K_{n+1}) 
 = 2^n \cdot 2^{-2n}\beta = 2^{-n}\beta.
\end{equation*}
We define
\begin{equation*}
 C_\alpha \eqdef \bigcap_{n=1}^\infty K_n.
\end{equation*}
Clearly \(C_n\) is closed since each \(K_n\) is closed and is nowhere dense
since any open interval of size greater than \(2^{-n}\) cannot be contained
in \(K_{n+1}\). Now
\begin{equation*}
m(C_\alpha) = m([0,1]) - \sum_{n=1}^\infty m(K_n \setminus K_{n+1}) 
   = 1 - \sum_{n=1}^\infty 2^{-n}\beta = 1 - \beta = \alpha.
\end{equation*}

Now let \(\alpha_n = 1 - 1/n\),
or any sequence monotonically increasing and converging to $1$.
Define
\begin{equation*}
A \eqdef \bigcup_{n=1}^\infty C_{\alpha_n}
\end{equation*}
Clearly $A$ is of first category and \(m(A)=1\). 
Since \([0,1]\) is of second category, the complement 
\begin{equation*}
B = [0,1] \setminus A
\end{equation*}
is of second category and \(m(B) = 0\).


%%%%%%%%%%%%%% 3
\begin{excopy}
Put \(K=[-1,1]\); define \(\scrD_K\) as in Section~1.46
(with \(\R\) in place of \(\R^n\)).
Suppose \(\{f_n\}\) is a sequence of Lebesgue integrable functions such that
\begin{equation*}
 \Lambda \phi = \lim_{n\to\infty}\int_{-\infty}^\infty f_n(t)\phi(t)\,dt
\end{equation*}
exists for every \(\phi\in\scrD_K\).
Show that \(\Lambda\) is a continuous functional on \(\scrD_K\).
Show that there is a positive integer $p$ and a number \(M<\infty\) such that
\begin{equation*}
 \left| \int_{-\infty}^\infty f_n(t)\phi(t)\,dt\right| \leq M\|D^p\phi\|_\infty
\end{equation*}
for all $n$. For example, if \(f_n(t) = n^2\) on \([0,1/n]\) and $0$ elsewhere,
show that this can be done with \(p=1\).
Construct an example where it can be done with \(p=2\).
but not with \(p=1\).
\end{excopy}

The \(\Lambda\) is clearly a linear mapping.
The mappings 
\begin{equation*}
\Lambda_n \phi \eqdef \int_{-\infty}^\infty f_n(t)\phi(t)\,dt
\end{equation*}
are clearly
linear and continuous. The discussion of section~1.46 in \cite{RudinFA79}
shows that \(\scrD_K\) is an $F$-space. Combining with 
theorem~2.8 \cite{RudinFA79}, \(\Lambda\) is continuous.

Applying theorem~2.6 \cite{RudinFA79} shows that \(\{\Lambda_n\}\)
are equicontinuous. Thus there is a base neighborhood $V$ of $0$ in \(\scrD_K\)
such that
\begin{equation*}
\Lambda_n(V) \subset \{z\in\C: |z|\leq 1\}.
\end{equation*}
for all $n$. Looking at the seminorms of \(\scrD_K\)
there exists some $p$ and \(\epsilon > 0\) such that 
\begin{equation*}
W \eqdef \{\phi\in\scrD_K: \|D^p \phi\|_\infty < \epsilon\} \subset V.
\end{equation*}
By linearity for any \(\phi \in \scrD_K\), we have 
\(\epsilon\phi/\|D^p \phi\|_\infty \in W\) or 
\(\|D^p \phi\|_\infty = 0\) and in any case
\begin{equation*}
\Lambda_n(\phi) \in \{z\in\C: |z|\leq \|D^p \phi\|_\infty/\epsilon\}
\end{equation*}
or equivalently
\begin{equation*}
|\Lambda_n(\phi)| \leq  (1/\epsilon)\cdot \|D^p \phi\|_\infty.
\end{equation*}

\paragraph{Exercise Error!?} 
Assuming \(f_n = n^2\chi_{[0,1/n]}\). Pick some \(\phi\in\scrD_K\)
such that \(\phi(0)\neq 0\). By looking at \(\phi/\phi(0)\)
we may assume \(\phi(0)=1\). For arbitrary \(\epsilon>0\)
we can find some $n$ such that \(|\phi(x)| > 1 - \epsilon\)
for \(x\in (-1/n,1/n)\). Now
\begin{equation*}
      \int_{-\infty}^{\infty} f_n(t)\phi(t)\,dt
 =  \int_0^{1/n} n^2\phi(t)\,dt
\end{equation*}
and so 
\begin{equation*}
      \left|\int_{-\infty}^{\infty} f_n(t)\phi(t)\,dt\right|
 \geq  \int_0^{1/n} n^2(1-\epsilon)\,dt = n(1-\epsilon) 
\end{equation*}
contradiction to the requirement of converging to zero!


\iffalse %%%%%%%%%%%%%%%%%%%%%%%%%%%%%%%%
Assuming \(f_n = n^2\chi_{[0,1/n]}\), we compute
\begin{equation*}
\int_{-\infty}^\infty f_n(t)\phi(t)\,dt 
= \int_0^{1/n} n^2\phi(t)\,dt 
= - \int_0^{1/n} n^2t \phi'(t)\,dt.
\end{equation*}

we define \(F_n:\R\to\R\) 
\begin{equation*}
F_n(x) \eqdef   \left\{\begin{array}{ll}
                        0     &\qquad x \leq 0 \\     
                        n^2x  &\qquad 0 \leq x \leq 1/n \\     
                        n     &\qquad 1/n \leq x
                       \end{array}\right..
\end{equation*}
A tedious computation shows that
\begin{equation*}
\int_{-\infty}^\infty f_n(t)\phi(t)\,dt 
 = - \int_0^1 F_n(t)\phi'(t)\,dt.
\end{equation*}
\fi %%%%%%%%%%%%%%%%%%%%%%%%%%%%%%%%%%%%%%


%%%%%%%%%%%%%% 4
\begin{excopy}
Let \(L^1\) and \(L^2\) be the usual Lebesgue spaces on the unit interval.
Prove that \(L^2\) is of the first category in \(L^1\) in three ways:
\begin{itemize}
 \itemch{a}
 Show that \(\{f: \int|f|^2\leq n\}\) is closed in \(L^1\)
 but has empty interior.
 \itemch{b}
 Put \(g_n=n\) on \([0,n^{-3}]\), and show that 
 \begin{equation*}
  \int fg_n \to 0
 \end{equation*}
 for every \(f\in L^2\)
 but not for every \(f\in L^1\).
 \itemch{c}
  Note that the inclusion map of \(L^2\) into  \(L^1\)
  is continuous but not onto.
\end{itemize}
 Do the same for \(L^p\) and \(L^q\) if \(p<q\).
\end{excopy}

Assume \(1\leq p<q < \infty\). Later we apply \(p=1\) and \(q=2\).
  For any \(f\in L^q([0,])\),
  separate intgeration over \(G = \{t\in[0,1]: |f(t)|\geq 1\}\)
  and \(H = [0,1]\setminus G\) 
  % shows that \(L^2([0,1]) \subset L^1([0,1])\)
  as follows:
  % \begin{eqnarray*}
  \begin{equation*}
  \|f\|_p^p
   = \int_0^1 |f|^p\,dm \\
   =    \int_G |f|^p\,dm + \int_H |f|^p\,dm \\
   \leq \int_G |f|^q\,dm + \int_H 1\,dm \\
   \leq \|g\|_q^q + m(H) < \infty
  \end{equation*}
  % \end{eqnarray*}
  Thus \(f\in L^p([0,1])\) and so
  \(L^q([0,1])\subset L^p([0,1])\).

\begin{itemize}
 \itemch{a}
  Let \(\{f_n\}_{n\in\N}\) be a sequence of functions in \(L^q([0,1])\)
  such that 
  \begin{equation*} \label{eq:ex2.3:fnn}
  \int_0^1 |f_n|^q\,dm \leq n
  \end{equation*}
  and
  \(f_n\to f\) in \(L^p([0,1])\).
  By negation assume  \(f\notin L^q([0,1])\). 
  It is easy to see that
  \(|f_n|\to |f|\) in \(L^p([0,1])\) and 
  \(|f|\notin L^q([0,1])\), thus we may assume \(f_n\geq 0\) and \(f\geq 0\).
  Define 
  \begin{equation*}
  g_n(x) = \liminf_{k\geq n} f_k(x).
  \end{equation*}
  Clearly 
  for all \(x\in[0,1]\), the convergences
  \(g_n^p(x) \to f^p(x)\) and 
  \(g_n^q(x) \to f^q(x)\) holds as well and 
  \(g_n^q(x)\leq g_{n+1}^q(x)\) for all $n$. 
  Hence by Lebesgue's monotone convergence theorem~(1.26 \cite{RudinRCA80}),
  \begin{equation*}
    \lim_{n\to\infty} \int_{[0,1]} g_n^q \,dm  = \int_{[0,1]} f^q \,dm.
  \end{equation*}
  By \eqref{eq:ex2.3:fnn} and thus \(\int_{[0,1]} f^q \,dm \leq n\)
  and so \(f\in L^q([0,1])\) and the set 
  \begin{equation*}
  F_n \eqdef \left\{f: \int|f|^q\leq n\right\}
  \end{equation*}
  is closed in \(L^p\).

  Pick arbitrary \(\epsilon > 0\).
  Observing
  \begin{equation*}
  \int_0^1 x^{ra}\,dx = x^{ra+1}/(ra+1)\big|_0^1
  \end{equation*}
  we look for $a$ such that \(pa+1 > 0 > qa+1\) or equivalently
  \begin{equation*}
  -1/p < a < -1/q
  \end{equation*}
  For example we can take \(a = -2/(p+q)\).
  Let 
  \begin{equation*}
   A = \left(\int_0^1 x^{pa}\,dx\right)^{1/p} < \infty
  \end{equation*}
   and define
  \(h(x) = \epsilon x^a/A\). Using the constraints inequalities, we can see that 
  \begin{eqnarray*}
  h &\in& L^p([0,1]) \setminus L^q([0,1]) \\
  \|h\|_p &\leq& \epsilon
  \end{eqnarray*}
  Now for any \(f\in L^q([0,1])\) we have
  \(f+h\in L^p([0,1]) \setminus L^q([0,1])\) thus \(L^q([0,1])\)
  and in particular \(F_n\) have empty interior in \(L^p([0,1])\)
  and since \(L^q([0,1]) = \cup_n F_n\) it is of first category.

 \itemch{b}
  Assume again \(1\leq p < q < \infty\). 
  Put \(q' = q/q-1\) the exponent conjugate.
  We want to find scalars \(\alpha\) and \(\beta\) such that if
  \begin{equation*}
    g_n(x) = \left\{\begin{array}{ll}
                     n & \qquad x \in [0,n^\alpha] \\
                     0 & \qquad x \in (n^\alpha, 1]
                    \end{array}\right.
  \end{equation*}
  and 
  \begin{equation*}
    \psi(x) = \left\{\begin{array}{ll}
                      x^\beta & \qquad x\in (0,1] \\
                      0       & \qquad x = 0
                    \end{array}\right.
  \end{equation*}
  then
  \begin{eqnarray}
   \forall f\in L^q([0,1])\qquad 
   \lim_{n\to \infty} \int_{[0,1]} fg_n\,dm &=& 0
                         \label{eq:ex2.4b:i} \\
   \psi &\in&  L^p([0,1]) \setminus L^p([0,1])
                         \label{eq:ex2.4b:psi} \\
   \lim_{n\to \infty} \int_{[0,1]} \psi g_n\,dm &=& \infty
                         \label{eq:ex2.4b:iii}
  \end{eqnarray}

  To ensure \eqref{eq:ex2.4b:i}, we use H\"older inequality
  \begin{equation*}
  \left|\int_{[0,1]} fg_n\,dm \right| \leq \|f\|_q \|g_n\|_{q'}
  \end{equation*}
  and require that
  \(\lim_{n\to\infty} \|g_n\|_{q'}^{q'} = 0.\)
  Computing
  \begin{equation*}
   \|g_n\|_{q'}^{q'} = \int_0^{n^\alpha} n^{q'}\,dm 
   = n^\alpha \cdot n^{q'} = n^{\alpha + q/(q-1)}.
  \end{equation*}
  Thus 
  \begin{equation} \label{eq:ex2.4b:alpha}
  \alpha < -q/(q-1)
  \end{equation}
  must hold.

  To ensure \eqref{eq:ex2.4b:psi}, we require that 
  \(p\beta + 1 > 0\)
  \(q\beta + 1 < 0\), or equivalently
  \begin{equation} \label{eq:ex2.4b:beta1}
   -1 \leq -1/p < \beta < -1/q < 0.
  \end{equation}

  We compute
  \begin{equation*}
   \int_{[0,1]} \psi g_n\,dm 
   = \int_0^{n^\alpha} n x^\beta = n^{\alpha(\beta+1) + 1} / (\beta+1)
  \end{equation*}
   and since \(\beta+1 > 0\)
   to ensure \eqref{eq:ex2.4b:iii},  we require that 
   \begin{equation}  \label{eq:ex2.4b:beta2}
     \alpha(\beta+1) + 1 \geq 0.
   \end{equation}

   Combining the requirements of 
   \eqref{eq:ex2.4b:alpha}, 
   \eqref{eq:ex2.4b:beta1}, and 
   \eqref{eq:ex2.4b:beta2}
   \begin{eqnarray} 
     -1/(\beta + 1) \leq &\alpha& < -q/(q-1) \label{eq:ex2.4b:s1} \\
           -1/p     <    &\beta& < -1/q .    \label{eq:ex2.4b:s2}
   \end{eqnarray}
   We will define \(\beta = -1/p + \epsilon\) 
   for sufficiently small \(\epsilon > 0\) 
   so \eqref{eq:ex2.4b:s2} holds which is trivial,
   but also allow for \eqref{eq:ex2.4b:s1} to hold for some \(\alpha\),
   as we now show.
   We first note that the mapping
   \(t \to -t(t-1) = -1/(t-1) -1\) is increasing.
   We deal with two similar cases.
   \paragraph{Case 1.} Assume \(p>1\). 
   Since
   \begin{equation*}
     -1/((-1/p) + 1) = -p/(p-1) < -q/(q-1),
   \end{equation*}
   For sufficiently small \(\epsilon>0\) 
   \begin{equation*}
     -p/(p-1) <  -1\big/\bigl(((1/p)+\epsilon)\, + 1\bigr) < -q/(q-1)
   \end{equation*}
   and thus we can find \(\alpha\) so \eqref{eq:ex2.4b:s1} holds.
   \paragraph{Case 2.} Assume \(p=1\). The convergence
   \begin{equation*}
   \lim_{0<\epsilon\to 0} -1\big/\bigl((-1/p)+\epsilon + 1\bigr) = -\infty
   \end{equation*}
   Shows that for sufficiently small \(\epsilon > 0\)
   \begin{equation*}
      -1\big/\bigl((-1/p)+\epsilon + 1\bigr) < -q/(q-1)
   \end{equation*}
   and again we have \(\beta\) and \(\alpha\) so \eqref{eq:ex2.4b:s1} holds.

  Hence,
  \(\lim_{n\to\infty} \int_0^1 \psi g_n\,dm > 0\).
  By Banach Steinhaus
  \index{Banach~Steinhaus}
  theorem~2.5 (\cite{RudinFA79}) \(L^q([0,1])\) cannot be of second category
  in \(L^p([0,1])\).

 \itemch{c}
  To show that the inclusion map of \(L^q\to L_p\) is continuous, assume
  \(f_n\to f\) in \(L_q\).
  Then \(\{f_n\}\) is a Cauchy sequence in \(L^q\).
  By chapter~3 exercise~5 of \cite{RudinRCA80},
  \(\|g\|_p \leq \|g\|_q\) for any \(g\in L^q\).
  Thus \(\{f_n\}\) is a Cauchy sequence in \(L^p\) as well
  and by its completeness it has a limit that must coincide with $f$.

  By the open-mapping theorem~2.11 (\cite{RudinFA79}), 
  \(L^q([0,1])\) cannot be of second category
  in \(L^p([0,1])\), since otherwise the inclusion mapping would be onto.
\end{itemize}


%%%%%%%%%%%%%% 5
\begin{excopy}
Prove results analogous to those of Exercise~4 for the spaces \(\ell^p\),
where \(\ell^p\) is the Banach space of all
complex functions $x$ on \(\{0,1,2,\ldots\}\) whose norm 
\begin{equation*}
\|x\|_p = \left\{ \sum_{n=0}^\infty |x(n)|^p\right\}^{1/p}
\end{equation*}
is finite.
\end{excopy}

Assume \(1\leq p < q < \infty\).
We first show that \(\ell^p \subsetneq \ell^q\).
Note that this is analogically \emph{opposite} to exercise~4, because
of different characteristics of the measure space.
Let \(x\in \ell^p\).
Clearly, there exists an integer $n$, such that \(|x(j)| < 1\) 
for all \(j>n\). 
Now
\begin{eqnarray*}
\|x\|_q^q
&=& \sum_{j=0}^\infty |x(j)|^q \\
&=& \sum_{j=0}^n |x(j)|^q + \sum_{j=n+1}^\infty |x(j)|^q \\
&\leq& \sum_{j=0}^n |x(j)|^q + \sum_{j=n+1}^\infty |x(j)|^p \\
&\leq& \sum_{j=0}^n |x(j)|^q + \|x\|_p^p \\
&<& \infty.
\end{eqnarray*}
Thus \(x\in \ell^q\) and so \(\ell^p \subset \ell^q\). 
Let's look at 
\begin{equation} \label{eq:ex2.5:u}
u(n) \eqdef n^{-1/p}.
\end{equation}
Note that \(-q/p < -1\) and compute
\begin{eqnarray*}
\|u\|_q^q &=& \sum_{n=0}^\infty n^{-q/p} < \infty \\
\|u\|_p^p &=& \sum_{n=0}^\infty n^{-1} = \infty.
\end{eqnarray*}
Therefore \(u\in \ell^q \setminus \ell^p\) and so \(\ell^p \subsetneq \ell^q\).
We will show that \(\ell^p\) is of first category in \(\ell^q\)
in three ways sinilar to previous exercise.
\begin{itemize}
 \itemch{a}
   Let \(F_n \eqdef \{x\in \ell^p: \|x\|_q \leq n\}\).
   We will show that \(F_n\) is closed in \(\ell_q\).
   Let \(\{x_k\}_{k\in\N}\) a sequence in \(F_n\) that converges to $x$ 
   in \(\ell_p\). Formally, \(\lim_{k\to\infty} \|x-x_k\|_q = 0\), equivalently
   \(\lim_{k\to\infty} \|x-x_k\|_q^q = 0\).
   Clearly 
   \begin{equation} \label{eq:ex2.5a:1}
   \lim_{k\to\infty} x_k(j) = x(j) 
   \end{equation}
   in \C\ for each $k$.
   If by negation \(x\notin F_n\) then 
   \(\|x\|_p > n\) 
   or equivalently, there exists some $m$ such that
   \begin{equation} \label{eq:ex2.5a:2}
   h \eqdef \left(\sum_{j=1}^m |x(j)|^p\right) - n^q > 0.
   \end{equation}
   By \eqref{eq:ex2.5a:1} we can find some integer $K$, such that 
   \(|x_k(j) - x(j)|^p < h/m\) for any $j$ and for every \(k\geq K\)
   which contradicts \eqref{eq:ex2.5a:2}. Hence \(F_n\) is closed
   but it has empty interior, since 
   for any \(\epsilon > 0\), 
   for any \(x\in F_n\) and any \(\epsilon > 0\)
   using \eqref{eq:ex2.5:u} put \(v \eqdef \epsilon u/\|u\|_q\)
   and clearly \(v \in l^q\) but \(x+v \notin l^p\) for any \(x\in F_n\)
   (even \(x\in l^p\)). Noting that \(l^p = \cup_n F_n\)
   shows it is of first category in \(l^q\).
   
 \itemch{b}
  Take some $r$ such that \(1\leq p<r<q<\infty\).
  Denote the respective exponent-conjugates
  \begin{equation*}
    \infty 
    \geq\; p' \eqdef p/(p-1)
    >\;    r' \eqdef r/(r-1) 
    >\;    q' \eqdef q/(q-1) 
    > 1.
  \end{equation*}
  Define \(g_n:\N\to\R\) for each \(n\in\N\) 
  \begin{equation*}
  g_n(k) \eqdef \left\{\begin{array}{ll}
                       0          & \qquad 1 \leq k < n \\
                       k^{-1/r'}  & \qquad n \leq k  \\
                       \end{array}\right.\;.
  \end{equation*}
  Since \(-p'/r' < -1\),
  \begin{equation*}
  \sum_{k=1}^\infty k^{-p'/r'} < \infty
  \end{equation*}
  and so we have
  \begin{equation*}
  \lim_{n\to\infty} \|g_n\|_{p'} 
  = \lim_{n\to\infty} \sum_{k=n}^\infty k^{-p'/r'} = 0.
  \end{equation*}
  Hence, for each \(x\in l^p\) by H\"older inequality
  \begin{equation*}
  \lim_{n\to\infty} \left| \sum_{k=1}^\infty x(k)g_n(k) \right|
  \leq \lim_{n\to\infty} \|x\|_p \cdot \|g_n\|_{p'}
  = \|x\|_p \cdot \lim_{n\to\infty} \|g_n\|_{p'}
  = 0.
  \end{equation*}

  On the other hand take \(w(k) \eqdef k^{-1/r}\), clearly
  \(w \in l^q \setminus l^r\). Compute
  \begin{equation*}
       \sum_{k=1}^\infty w(k)g_n(k) 
  =    \sum_{k=n}^\infty (1/k)^{1/r} \cdot (1/k)^{1/r'}
  =    \sum_{k=n}^\infty (1/k)
  = \infty.
  \end{equation*}

  Thus as a linear operator, 
  \(\{g_n\}_{n\in\N}\) are bounded for each \(x\in l^p\),
  but not on \(w\in l^q\).
  Hence, by Banach Steinhaus
  \index{Banach~Steinhaus}
  theorem~2.5 (\cite{RudinFA79}) \(l^p\) cannot be of second category
  in \(l^q\).

 \itemch{c}
  The inclusion mapping of \(l^p\) into \(l^q\) is continuous,
  since if \(\lim_{k\to\infty}\|x_k - x\|_p = 0\) 
  then for sufficiently large $k$, clearly
  \(|x_k - x\|_p \geq |x_k - x\|_q\) and so 
  \(\lim_{k\to\infty}\|x_k - x\|_q = 0\).

  Again, by the open-mapping theorem~2.11 (\cite{RudinFA79}), 
  \(l^q\) cannot be of second category
  in \(l^p\), since otherwise the inclusion mapping would be onto.

\end{itemize}



%%%%%%%%%%%%%% 6
\begin{excopy}
Define the Fourier coefficients \(\hat{f}(n)\) of a function
\(f\in L^2(T)\) ($T$ is the unit circle) by
\begin{equation*}
\hat{f}(n) 
 = \frac{1}{2\pi} \int_{-\pi}^{\pi} f(e^{i\theta})e^{-in\theta}\,d\theta
\end{equation*}
for all \(n\in\Z\) (the integers). Put
\begin{equation*}
 \Lambda_n f = \sum_{k= -n}^n \hat{f}(k)\,.
\end{equation*}
Prove that 
 \(\{f\in L^2(T): \lim_{n\to\infty} \Lambda_n f\; \textrm{exists}\}\)
is a dense subspace of \(L^2(T)\) of the first category.
\end{excopy}

For any trigonometric polynomial, 
\begin{equation*}
 P(\theta) = \sum_{k= -m}^m a_k e^{i\theta}
\end{equation*}
we have \(\hat{P}(k) = a_k\)
for each $k$ such that \(|k| \leq m\).
Thus \(\Lambda_n f = f\) for any \(n\geq m\), 
and in particular \(\Lambda_n f\) converges in  \(L^2(T)\).
The trigonometric (finite, continuous) polynomials
are dense in \(C(T)\) and thus dense in \(L^2(T)\).
Now that density was shown, we need to show first category.

Assume by negation that the set 
\begin{equation*}
F = \{f\in L^2(T): \lim_{n\to\infty} \Lambda_n f\; \textrm{exists}\}
\end{equation*}
is of second category, then so would be the superset
\begin{equation*}
B = \{f\in L^2(T): \limsup_{n\to\infty} |\Lambda_n f| < \infty\}.
\end{equation*}
By
\index{Banach Steinhaus}
Banach~Steinhaus theorem~2.5 (\cite{RudinFA79})
\(\{\Lambda_n\}\) are equicontinuous, but this contradicts
the results of the 
discussion in section~5.11 of \cite{RudinRCA80} that shows that 
the functionals \(\Lambda_n\) are not equicontinuous.


%%%%%%%%%%%%%% 7
\begin{excopy}
Let \(C(T)\) be the set of all continuous complex functions
on the unit circle $T$.
Suppose \(\{\gamma_n\}\) (\(n\in\Z\)) is a complex sequence that associates
to each \(f\in C(T)\) a function \(\Lambda f \in C(T)\) whose 
Fourier coefficients  are
\begin{equation*}
 (\Lambda f){\,\hat{}\,}(n) = \gamma_n\hat{f}(n) \qquad (n\in\Z).
\end{equation*}
(The notation is as in Exercise~6.) Prove that \(\{\gamma_n\}\)
has this multiplier property if and only if there is a complex Borel measure
\(\mu\) on $T$ such that 
\begin{equation*}
 \gamma_n = \int e^{-in\theta}\,d\mu(\theta) \quad (n\in\Z)
\end{equation*}
\emph{Suggestion:} With the supremum norm, \(C(T)\) is a Banach space.
Apply the closed graph theorem. Then consider the functional
\begin{equation*}
 f \to (\Lambda f)(1) = \sum_{-\infty}^{\infty} \gamma_n \hat{f}(n)
\end{equation*}
and apply the Riesz representation theorem ([23], Th.~6.19)
(The above series may not converge;
use it only for trigonometric of polynomials.)
\end{excopy}

Assume there exists a Borel measure \(\mu\)
determining such \(\{\gamma_n\}_{n\in\Z}\).
Define \(\Lambda : C(T) \to C(T)\) by
\begin{equation*}
(\Lambda f)(t) \eqdef \sum_{k\in\Z} \gamma_n e^{ikt}.
\end{equation*}
For each \(k in\Z\) we define a base function \(b_k(t) = e^{ikt}\).
Clearly \(\posthat{b_k}(n) = 1\) iff \(k=n\) and 
\(\posthat{b_k}(n) = 0\) otherwise.
Compute
\begin{eqnarray*}
\posthat{(\Lambda b_n)}(n) 
&=& \frac{1}{2\pi} \int_{-\pi}^\pi%
       \left(\sum_{k\in\Z} \gamma_n e^{ikt}\right) e^{-int}\,dt \\
&=& \frac{1}{2\pi} \sum_{k\in\Z} \int_{-\pi}^\pi \gamma_n e^{i(k-n)t}\,dt \\
&=& \frac{1}{2\pi} \int_{-\pi}^\pi \gamma_n \cdot 1\,dt \\
&=& \gamma_n
\end{eqnarray*}
For any \(f\in C(T)\) we have
\(f = \sum_{k\in\Z} \posthat{f}(k)\cdot b_k\).
Hence
\begin{eqnarray*}
\posthat{(\Lambda f)}(n)
&=& \posthat{(\Lambda \sum_{k\in\Z} \posthat{f}(k)\cdot b_k )}(n) \\
&=& \sum_{k\in\Z} \posthat{(\Lambda \posthat{f}(k)\cdot b_k )}(n) \\
&=& \posthat{(\Lambda \posthat{f}(n)\cdot b_n)}(n) \\
&=& \posthat{f}(n)\cdot \posthat{(\Lambda b_n)}(n) \\
&=& \gamma_n \cdot \posthat{f}(n).
\end{eqnarray*}

Conversely,
assume that \(\{\gamma_n\}_{n\in\Z}\) satisfies the abive multiplier property.
For any \(f\in C(T)\), if \(\hat{f}(n) = 0\) for every \(n\in\Z\)
then \(f=0\). 
This can be derived from the discussion in section~4.26 of \cite{RudinRCA80}.
By this, we can easily see that \(\Lambda\) is linear.
Now we will show that the graph \(\{(f,\Lambda f): f\in C(T)\}\)
is closed in \(C(T)^2\).
Since it is a normed space, it is sufficient to verify
for (enumerable) sequences.
\iffalse
Every \(f\in C(T)\) satisfies
\begin{equation*}
f(t) = \sum{n\in\Z} \hat{f}(n)e^{int}
\end{equation*}
\fi
Let \(\lim_{m\to\infty} f_m = f\)
and \(\lim_{m\to\infty} \Lambda f_m = g\)
both in \(C(T)\) with the supremum norm.
For any \(n\in\Z\)
\begin{eqnarray*}
 (\Lambda f){\,\hat{}\,}(n) 
 &=& \gamma_n \hat{f}(n) \\
 &=& \frac{\gamma_n}{2\pi}%
     \int_{-\pi}^{\pi} f(e^{i\theta})e^{-in\theta}\,d\theta \\
 &=&  \frac{\gamma_n}{2\pi} \int_{-\pi}^{\pi}
       \lim_{m\to\infty} f_m(e^{i\theta})e^{-in\theta}\,d\theta \\
 &=& \gamma_n\lim_{m\to\infty}\frac{1}{2\pi}%
     \int_{-\pi}^{\pi} f_m(e^{i\theta})e^{-in\theta}\,d\theta \\
 &=& \gamma_n\lim_{m\to\infty} \widehat{f_m}(n) \\
 &=& \lim_{m\to\infty} \gamma_n \widehat{f_m}(n) \\
 &=& \lim_{m\to\infty} \posthat{(\Lambda {f_m})}(n) \\
 &=& \hat{g}(n)
\end{eqnarray*}
Note that the last inequality is justified by the fact that 
\(\phi \to \hat{\phi}(n)\) is a continuous functional.

By the closed graph theorem~2.15 (\cite{RudinFA79}), \(\Lambda\)
is continuous and so is the functional \(f \to (\Lambda f)(1)\).
By Riesz representation theorem (cited in the above suggestion),
there is a Borel complex measure \(\mu\) on $T$
such that \((\Lambda f)(1) = \int_T f\,d\mu\) for every \(f\in C(T)\).
In particular if \(f(t) = e^{ikt}\) then 
\(\hat{f}(n) = 1\) iff \(n=k\) and 
\(\hat{f}(n) = 0\) otherwise. 
Thus \(\posthat{(\Lambda f)}(n) = \gamma_n\) iff \(n=k\) and 
\(\posthat{(\Lambda f)}(n) = 0\) otherwise.
The desired equality can now be computed
\begin{equation*}
\int_T e^{ikt}\,d\mu
 = (\Lambda f)(1)
 = \sum_{n\in\Z} \posthat{(\Lambda f)}(n) e^{in\cdot 0}
 = \posthat{(\Lambda f)}(k)
 = \gamma_n.
\end{equation*}


%%%%%%%%%%%%%% 8
\begin{excopy}
Define functionals \(\Lambda_m\) on \(\ell^2\) (see Exercise~5) by
\begin{equation*}
 \Lambda_m x = \sum_{n=1}^m n^2x(n) \qquad (m=1,2,3,\ldots).
\end{equation*}
Define \(x_n\in\ell^2\) by \(x_n(n) = 1/n\), \(x_n(i) = 0\) if \(i\neq n\).
Let \(K\subset \ell^2\) consist of \(0,x_1,x_2,x_3,\ldots\).
Prove that $K$ is compact.
Compute \(\Lambda_m x_n\).
Show that \(\{\Lambda_m x\}\) is bounded for each \(x\in K\)
but \(\{\Lambda_m x_m\}\) is not.
Convexity can therefore not be omitted from the hypothesis of Theorem~2.9.

Choose \(c_n > 0\) so that 
\(\sum c_n = 1\), 
\(\sum n c_n = \infty\).
Take \(x = \sum c_n  x_n\). Show that $x$ lies in the closed convex hull of $K$
(by definition, this is the closure of the convex hull)
and that \(\{\Lambda_m x\}\) is not bounded.

Show that the convex hull of $K$ is not closed.
\end{excopy}

For any open covering \(\Omega\) of $K$, we pick some $0$-neighborhood
\(V_0\in\Omega\). Since \(x_n \to 0\), there exist some $N$ such that
\(x_n\in V_0\) for all \(n>N\). For each \(n\leq N\) we pick \(v_n\)
such that \(x_n\in V_n\in\Omega\). Now \(\{V_n\}_{n=0}^N\)
is a finite sub-covering. Thus $K$ is compact.

Compute
\begin{equation*}
\Lambda_m x_n = \left\{%
 \begin{array}{ll}
   n & \qquad n \leq m \\
   0 & \qquad m < n
 \end{array}\right..
\end{equation*}
Thus \(\Lambda_m x_n \leq m\) 
therefore \(\{\Lambda_m x\}\) is bounded for each \(x\in K\),
while \(\Lambda_m x_m = m\}\) which is not bounded.

Define \(s_m = \sum_{n=1}^m c_n x_n\) and let 
\(\gamma_{m} = 1 - \sum_{n=1}^m c_n\).
Thus 
\begin{equation*}
s_m = \gamma_{m}\cdot 0 = \sum_{n=1}^m c_n x_n \in \hull(K).
\end{equation*}
Since \(\lim_{m\to\infty} s_m = x\) we have \(x\in \overline{\hull(K)}\).
But 
\begin{equation*}
\Lambda_m x 
= \sum_{n=1}^m n^2c_n(1/n)
= \sum_{n=1}^m n c_n = \infty.
\end{equation*}



%%%%%%%%%%%%%% 9
\begin{excopy}
Suppose $X$, $Y$, $Z$ are Banach spaces and 
\begin{equation*}
B: X\times Y \to Z
\end{equation*}
is a bilinear and continuous. Prove that there exists \(M<\infty\) such that
\begin{equation*}
 \| B(x,y) \| \leq M \|x\| \|y\| \qquad (x\in X, y\in Y).
\end{equation*}
Is completeness needed here?
\end{excopy}

By continuity, there is a neighborhood $V$ of \((0,0)\in X\times Y\)
such that 
\begin{equation*}
B(V) \subset Z_1 \eqdef \{z\in Z: |z| < 1\}.
\end{equation*}
From definition of the topology of \(X\times Y\)
there are scalars \(0<a,b<\infty\) such that
\begin{equation*}
U \eqdef \{(x,y)\in X\times Y: |x| < a\;\wedge\; |y|<b\} \subset V.
\end{equation*}
Let \(M = \max(a, b)^2\).
Take arbitrary \((x,y)\in X\times Y\).
If \(x=0\) or \(y=0\) then \(B(x,y) = 0\) and the inequality
trivially holds.
Otherwise, 
\((x/a\|x\|, y/b\|y\|) \in U\) and so 
\begin{equation*}
\|B(x/a\|x\|, y/b\|y\|)\| \leq 1
\end{equation*}
or equivalently,
\begin{equation*}
\|B(x, y)\| \leq ab\|x\|\cdot\|y\| \leq M\|x\|\cdot\|y\|.
\end{equation*}
We used only the fact that Banach space is normed.
Completeness was not necessary.


%%%%%%%%%%%%%% 10
\begin{excopy}
Prove that a bilinear mapping is continuous if it is continuous 
at the origin \((0,0)\).
\end{excopy}

Suppose $X$, $Y$, $Z$ are topological vector spaces and 
let \(B: X\times Y \to Z\) be a bilinear mapping which is continuous
in \((0,0)\). Pick some \((x,y)\in X\times Y\)
and let \(z = B(x,y)\) and let \(z + W'\) a neighborhood of \(z\in Z\).
We can find a neighborhood $W$ of \(0\in Z\) such that 
\begin{equation*}
W + W + W \subset W'.
\end{equation*}

For each \(x,a\in X\) and \(y,b\in Y\) we have by bilinearity
\begin{equation*}
B(x+a,y+b) = B(x,y) + B(x,b) + B(a,y) + B(a,b).
\end{equation*}
We can find neighborhoods of the origins
\(V_X \subset X\) and \(V_Y \subset Y\) such that 
\begin{itemize}
 \item \(B(V_X \times V_Y) \subset W\) --- By continuity at the origin of $B$.
 \item \(B(\{x\}\times V_Y) \subset W\) --- By bilinearity (\(B(0,y)=0\)).
 \item \(B(V_X\times \{y\}) \subset W\) --- By bilinearity  (\(B(x,0)=0\)).
\end{itemize}
Putting \(V = V_X \times V_Y \subset X\times Y\), we get
\begin{equation*}
 B(V) \subset W + W + W \subset W'.
\end{equation*}
Thus $B$ is continuous.


%%%%%%%%%%%%%% 11
\begin{excopy}
Define \(B(x_1,x_2;y) = (x_1y, x_2y)\). 
Show that $B$ is a bilinear continuous mapping of 
\(\R^2\times\R\) onto \(\R^2\) which is not open at \((1,1;0)\).
Find all points where this $B$ is open.
\end{excopy}

\textbf{Note:} We refer to cases of table~\ref{tbl:B:open}.
\newcommand{\tbcase}[2]{\textbf{{\textsf{#1}}#2}}

We will use the following open interval notation:
\begin{equation*}
V(x,h) \eqdef \{t\in\R: x-h < t < x+h\}.
\end{equation*}


Put  \(P= (1,1;0)\). We have \(B(P) = (0,0)\). Pick a neighborhood of $P$
\begin{equation*}
V \eqdef \bigl\{(x_1,x_2;y) \in \R^2\times\R : 
           |x_1-1| < 1/2 \;\wedge\;
           |x_2-1| < 1/2 \;\wedge\;
           |y| < 1/2\bigr\}.
\end{equation*}
Clearly, for any \(\epsilon > 0\), we have
%\begin{equation*}
 \((\epsilon, 0) \notin B(V)\)
%\end{equation*}
and so \(B(V)\) is not open.
Similarly, $B$ is not open at 
\(\{(x_1,x_2;0)\in \R^2\times\R: x_1\neq 0 \neq x_2\}\)
(case \tbcase{TTF}{F}).

The other cases
\begin{itemize}

\item \textbf{Case} \tbcase{FFF}{T}.
If \(x_1 = x_2 = y = 0\), then for any \(\epsilon>0\), 
then the neighborhood \(U_\epsilon\) of
\((0,0;0)\) defined as 
\(U_\epsilon 
  \eqdef \bigl(V(0,\epsilon) \times  V(0,\epsilon)\bigr) \times V(0,\epsilon)\)
we can pick \(\delta = \epsilon^2\) and clearly
\begin{equation*}
 W_\delta \eqdef V(0,\delta) \times V(0,\delta) \subset B(U_\epsilon).
\end{equation*}
Hence, $B$ is open at \((0,0;y)\).

\item \textbf{Case} \tbcase{FFT}{T}.
If \(Q = (0,0;y)\) where \(y\neq 0\) then \(B(Q)=(0,0)\).
Pick \(\epsilon>0\) and let 
\begin{equation*}
U_\epsilon> \eqdef (V(0,\epsilon)\times V(0,\epsilon)\times)\times V(y,\epsilon)
\end{equation*}
a neighborhood of $Q$.
Take \(\delta = |y|\epsilon\) and let 
\(W_\delta = V(0,\delta)\times V(0,\delta)\).
If \((x_1,x_2)\in W_\delta\) then
\(|x_i/y| < \epsilon\) for \(i=1,2\) and 
so \((x_1/y,x_2/y;y)\in U_\epsilon\) and $B$ is open at $Q$.

\item \textbf{Cases} \tbcase{TFF}{F}, \tbcase{FTF}{F}.
If \(Q = (x_1,0;0)\) where \(x_1\neq 0\) then \(B(Q)=(0,0)\).
For any 
% \(0<\epsilon < |x_1|\), 
\(\epsilon > 0\), 
we can see that
\((0,\epsilon)\notin B(V)\), for sufficiently small neighborhood $V$ of $Q$.
Hence, using symmetry, $B$ is not open at
\(\{(x_1,x_2;0)\in \R^2\times\R: (x_1=0) \not\Leftrightarrow (x_2=0)\}\)


\item \textbf{Cases} \tbcase{FTT}{T}, \tbcase{TFT}{T}.
If \(Q = (x_1,0;y)\) where \(x_1y\neq 0\) then \(B(Q)=(x_1y,0)\).
Pick arbitrary \(\epsilon>0\) 
then the neighborhood \(U_\epsilon\) of $Q$ defined as 
\(U_\epsilon 
  \eqdef \bigl(V(x_1,\epsilon) \times  V(0,\epsilon)\bigr) \times V(y,\epsilon)\)
we can pick \(\delta = \epsilon/y\).
For any
\begin{equation*}
(w_1,w_2) \in W \eqdef V(x_1y, \delta) \times V(0,\delta)
\end{equation*}
we put 
\begin{equation*}
Q' \eqdef (u_1,u_2;y) \eqdef (w_1/y, w_2/y;y)
\end{equation*}
and clearly \(Q'\in U_\epsilon\) and therefore $B$ is open at $Q$.
Similarly the same result hold for \((0,x_2;y\) where \(x_2y\neq 0\).

\item \textbf{Case} \tbcase{TTT}{T}.
If \(x_1 x_2 y \neq 0\), then for any \(\epsilon>0\), 
the neighborhood \(U_\epsilon\) of
\((x_1,x_2;y)\) defined as 
\(U_\epsilon 
  \eqdef \bigl(V(x_1,\epsilon) \times  V(x_2,\epsilon)\bigr) 
         \times V(y,\epsilon)\)
we can find \(\delta\) such that 
\begin{equation*}
 W_\delta \eqdef V(x_1y,\delta) \times V(x_2y,\delta) \subset B(U_\epsilon).
\end{equation*}
For example we can pick
\(\delta = \min_{i=1,2} |x_i|\epsilon\)
and so $B$ is open at \((x_1,x_2;y)\).

\end{itemize}

To summerize
\begin{table}[ht] \label{tbl:B:open}
\begin{center}
\newcommand{\bfsf}{\bfseries\sffamily}
\begin{tabular}{|>{\bfsf}c|>{\bfsf}c|>{\bfsf}c|>{\bfseries}c|}
\hline
 \(x_1\neq 0\) &  \(x_2\neq 0\)  & \(y\neq 0\) & $B$ open at \((x_1,x_2;y)\) \\
\hline
  F & F & F & T \\ \hline
  F & F & T & T \\ \hline
  F & T & F & F \\ \hline
  F & T & T & ? \\ \hline
  T & F & F & F \\ \hline
  T & F & T & ? \\ \hline
  T & T & F & F \\ \hline
  T & T & T & T \\ \hline
\end{tabular}
\caption{Sorted cases where $B$ is open mapping}
\end{center}
\end{table}


%%%%%%%%%%%%%% 12
\begin{excopy}
Let $X$ be the normed space of all real polynomials in one variable, with
\begin{equation*}
 \|f\| = \int_0^1 |f(t)|\,dt\,.
\end{equation*}
Put \(B(f,g) = \int_0^1f(t)g(t)\,dt\),
and show that $B$ is a bilinear functional on \(X\times X\)
which is separately continuous but is not continuous.
\end{excopy}

First note that
\begin{equation*}
  \int_0^1 |f_n(t)g(t) - f(t)g(t)|\,dt
 \leq \max_{t\in[0,1]}|g(t)| \int_0^1 |f_n(t) - f(t)|\,dt.
\end{equation*}
If $g$ is fixed and \(\lim_{n\to\infty} f_n = f\) in the \(\|\cdot\|_1\) norm,
then 
\begin{equation*}
\lim_{n\to\infty} \int_0^1 f_n(t)g(t)\,dt = \int_0^1 f(t)g(t)\,dt 
\end{equation*}
by the above inequality. Thus $B$ is separately continuous.

Define a sequences \(\{f_n\}_{n\in\N}\) in \(C[0,1]\)
\begin{equation*}
f_n(x) = \left\{%
  \begin{array}{ll}
  n^2 - n^5x & \qquad 0 \leq x \leq  n^{-3} \\
  0          & \qquad n^{-3} \leq x \leq 1
  \end{array}\right.
\end{equation*}
Clearly \(f_n\) is continuous. The square is
\(f_n^2(x) = n^{10}x^2 - 2n^7x + n^4\). Integrations are
\begin{eqnarray*}
\int_0^1 f_n(x)\,dx &=& n^2\cdot n^{-3}/2 = 1/2n \\ 
\int_0^1 f_n^2(x)\,dx 
  &=& (n^{10}x^2/2 - n^7x^2 + n^4x)\bigm|_0^{n^{-3}} 
   =  n^4/2 - n + n = n^4/2
\end{eqnarray*}
By 
\index{Stone Weierstrass}
Stone Weierstrass theorem~(7.26 \cite{RudinPMA85})
we can find real polynomials \(p_n\) such that \(\|p_n - f_n\|_\infty < 1/n\)
and so 
\begin{eqnarray*}
\lim_{n\to\infty} \int_0^1 |p_n(x)|\,dx &=& 0 \\
\lim_{n\to\infty} \int_0^1 p_n^2(x)\,dx &=& \infty \\
\end{eqnarray*}
Thus \(p_n \xrightarrow{n\to\infty} 0\) but \(B(p_n,p_n) \not\to 0\).


%%%%%%%%%%%%%% 13
\begin{excopy}
Suppose $X$ is a topological vector space which 
is of a second category in itself.
Let $K$ be a closed, convex, absorbing subset of $X$.
Prove that $K$ contains a neighborhood of $0$.\\
\emph{Suggestion:} Show first that \(H= K\cap(-K)\) is absorbing.
By a category argument, $H$ has interior. Then use
\begin{equation*}
  2H = H + H = H - H.
\end{equation*}
Show that the result is false without convexity of $K$, 
even if \(X = \R^2\).
Show that the result is false if \(X = L^2\) topologized by the \(L^1\)-norm
(as in Exercise~4).
\end{excopy}

Following the suggestion. 
We will show that $H$ is absorbing.
Since $K$ is absorbing, so is \(-K\) 
and we have \(0\in K\) and \(0\in (-K)\) and so 
\(0\in H\). Now pick arbitrary \(x\in X\)
so there exist some  \(m_0\) and \(m_1\) such that 
\(x\in m_0 K\) and \(x\in m_1(-K)\).
Hence \(x/m_0 \in K\) and \(x/m_1 \in (-K)\).
If \(1/m_0 \leq 1/m_1\) the convexity of \(-K\) implies \(x/m_0 \in (-K)\).
Otherwise, \(1/m_1 \leq 1/m_0\) and similarly,
the convexity of $K$ implies \(x/m_1 \in K\).
In both cases \(x/\max(m_0,m_1) \in K\cap(-K)\),
equivalently, \(x\in \max(m_0,m_1) H\) and $H$ is absorbing.

Now, since \(X = \cup_{n\in\N} nH\)
since each \(nH\) is closed, and $X$ is of second category, then
some \(nH\) must have non empty interior, thus
$H$ must have non empty interior and so \emph{each} \(nH\).
 By definition, \(H= -H\)
and so \(H + H = H - H\). 
Also \(2H \subset H+H\) (actually for any set $H$), 
but by convexity \(2H = H+H\).
Let \(x_0\in V \subset H\), where $V$ is a neighborhood of \(x_0\in H\)
that exist since $H$ has non empty interior. Clearly 
\(V-\{x_0\}\) is a neighborhood of $0$ in \(H-H = 2H\).
Hence $W$ defined by 
\begin{equation*}
W \eqdef \frac{1}{2}\cdot\bigl(V-\{x\}\bigr) \subset H \subset K
\end{equation*}
is a neighborhood of $0$ in $K$.

Without convexity, the result is false as the following 
subset $K$ --- unit disc with ``sharp-arc-cavity'' in its positive quadrant ---
shows.
\begin{equation*}
K = \left\{(x,y)\in\R^2: x^2+y^2\leq 1 \; \wedge 
      (\; x\leq 0 \vee y\leq 0 \vee (x-1)^2+y^2 \leq 0\;)
      \right\}.
\end{equation*}
It is easy to see that $K$ is closed and absorbing, but has no $0$-neighborhood
and indeed is not convex.

%%%%%%%%%%%%%% 14
\begin{excopy}
\begin{itemize}
 \itemch{a}
  Suppose $X$ and $Y$ are topological vector spaces,
  \(\{\Lambda_n\}\) is an equicontinuous sequence of linear mappings
  of $X$ into $Y$, and $C$ is the set of all $x$ at which 
  \(\{\Lambda_n(x)\}\) is a Cauchy sequence in $Y$.
  Prove that $C$ is a closed subspace of $X$.
 \itemch{b}
  Assume, in addition to the hypothesis of \ich{a}, that $Y$ is an $F$-space
  and that \(\{\Lambda_n(x)\}\) converges in some dense subset of $X$.
  Prove that then
  \begin{equation*}
    \Lambda(x) = \lim_{n\to \infty} \Lambda_n(x)
  \end{equation*}
  exists for every \(x\in X\) and that \(\Lambda\) is continuous.
\end{itemize}
\end{excopy}

\begin{itemize}
 \itemch{a}
  First we show that $C$ is a subscpace.
  Let \(x\in C\), then
  \(\{\Lambda_n x\}\) is a Cauchy sequence. 
  Pick a scalar \(\alpha\). If \(\alpha=0\) then  \(\{\Lambda_n 0x\}\)
  is a trivial Cauchy sequence.
  Otherwise, for any neighborhood $V$ of \(0\in Y\), we look at 
  \((1/\alpha)V\) and there exists some $N$ such that 
  \((\Lambda_m x - \Lambda_n x) \in (1/\alpha)V\) for all \(m,n\geq N\).
  Equivalently,
  \((\Lambda_m \alpha x - \Lambda_n \alpha x) \in V\) for all \(m,n\geq N\)
  and therefore \(\alpha x \in C\).
  If \(x_1,x_2\in C\) then for any neighborhood $V$ of \(0\in Y\)
  we look at sub-neighborhood $W$ such that \(W+W\subset U\)
  and for \(j=1,2\) find \(N_j\) such that 
  \(\Lambda_m x_j - \Lambda_n x_j) \in W\).
  We can see that for \(N=\max(N_1,N_2\)
  \begin{equation*}
   \Lambda_m (x_1+x_2) - \Lambda_n (x_1+x_2)
   = 
     (\Lambda_m x_1 - \Lambda_n x_1) +
     (\Lambda_m x_2 - \Lambda_n x_2)
   \in W + W \subset U.
  \end{equation*}
  Hence $C$ is a linear subscpace of $X$.
  
  Let \(x\in \overline{C}\).
  Pick arbitrary neighborhood $V$ of \(0\in Y\).
  We take a symmetric sub-neighborhood $W$ such that \(W + W + W \subset V\).
  Since \(\{\Lambda_n\}\) are equicontinuous, there exists a neighborhood
  $U$ of \(0\in X\) such that \(\Lambda_n(U) \subset W\) for all $n$.
  Since \(x\in\overline{C}\) there exists \(x'\in C\) such that
  \(x-x'\in U\). By definition of $C$, there exists \(m_1\)
  such that \(\Lambda_{n_1}(x') - \Lambda_{n_2}(x') \in W\)
  for all \(n_1,n_2\geq m_1\). Now
  \begin{equation*}
   \Lambda_{n_1}(x) - \Lambda_{n_2}(x) =
      \bigl(\Lambda_{n_1}(x) - \Lambda_{n_1}(x')\bigr)
    +  \bigl(\Lambda_{n_1}(x') - \Lambda_{n_2}(x')\bigr)
    +  \bigl(\Lambda_{n_2}(x') - \Lambda_{n_2}(x)\bigr)
      \in W + W + W \subset V.
  \end{equation*}
  We have shown that \(\{\Lambda_n(x)\}\) is a Cauchy sequence.
  Hence \(x\in C\) and C is closed.
  
 \itemch{b}

  Since $Y$ is a complete sapce, every Cauchy sequence converges.
  Thus the assumption implies that $C$ is dense in $X$. By \ich{b}
  $C$ is closed and so  \(C=X\) and \(\Lambda\) is well defined.

  \emph{Note:} To show continuity,
  we cannot apply Theorems~2.7(b) or~2.8 (\cite{RudinFA79})
  since we \emph{cannot} assume $X$ is an F-Space.

  Pick an arbitrary \(x\in X\) and let $y+V$ 
  be a neighborhood  \(y=\Lambda(x)\).
  Take a symmetric neighborhood $W$ of \(0\in Y\) such that
  \(W + W + W \subset V\).
  By definition of \(\Lambda\) there exists $n$ such that
  \(\Lambda(x) - \Lambda_n(x) \in W\).
  By \(\{\Lambda_n\}\) being continuous, there exists a neighborhood $U$
  of $x$ such that  \(\Lambda_n(x) - \Lambda_n(x') \in W\) for all \(x'\in x+U\).
  We now have
  \begin{equation*}
  \Lambda(x) - \Lambda(x') = 
      \bigl(\Lambda(x) - \Lambda_n(x)\bigr)
   +  \bigl(\Lambda_n(x) - \Lambda_n(x')\bigr)
   +  \bigl(\Lambda_n(x') - \Lambda(x')\bigr)
   \in W + W + W \subset V.
  \end{equation*}
  Hence \(\Lambda\) is continuous.

\end{itemize}

%%%%%%%%%%%%%% 15
\begin{excopy}
Suppose $X$ is an $F$-space and $Y$ is a subscpace $X$ whose complement
is of the first category. Prove \(Y=X\).
\emph{Hint:} $Y$ must intersect \(x+Y\) for every \(x\in X\).
\end{excopy}

Pick an arbitrary \(x\in X\).
Look at 
\begin{equation*}
X \setminus \bigl(Y \cap (x+Y)\bigr)
\subset (X\setminus Y) \cup \bigl(X \setminus (x+Y)\bigr)
= (X\setminus Y) \cup \bigl(x+(X \setminus Y)\bigr).
\end{equation*}
Hence this complement is a union of two sets of first category,
Thus \(Y \cap (x+Y)\) is of second category since $X$ is a complete metric space.
In particular \(Y \cap (x+Y)\neq \emptyset\), and tus there exists
\(y_1,y_2 \in Y\) such that \(y_1 = y_2 + x\)
and \(x = y_1 - y_0 \in Y\). Therefore \(X\subset Y\) and so \(X=Y\).

%%%%%%%%%%%%%% 16
\begin{excopy}
Suppose $X$ and $K$ are metric spaces, that $K$ is compact,
and that the graph of \(f: X\to K\) is a closed subscpace of \(X\times K\).
Prove that $f$ is continuous.
(This is an analogue of Theorem~2.15 but is much easier.)
Show that compactness of $K$ cannot be omitted from the hypothesis,
even when $X$ is compact.
\end{excopy}

Because these are metric spaces, it is sufficient
to show continuity for sequences, that is let 
\((x_j)_{j\in\N}\) be a sequence in $X$ such that \(\lim_{n\to\infty} x_n = x\)
we need to show that 
\begin{equation} \label{eq:ex2.16}
\lim_{n\to\infty} f(x_n) = f(x).
\end{equation}
Put \(y_n = f(x_n\) for \(n\in\N\) and \(Y=\{y_j: j\in N\}\).
The sequence \((y_j)_{j\in\N}\) in $K$ must have at least 
one accumulation point $y$.

Assume by negation that  \((y_j)_{j\in\N}\) has 
another accumulation point \(y' \neq y\).
The graph~$G$ of~$f$ contains \(\{(x_n,y_n):n\in\N\}\)
and so both 
\begin{equation*}
(x,y),(x,y') \,\in \overline{G},
\end{equation*}
but by being a graph of a function it is impossible for $G$
to have both \((x,y)\) and \((x,y')\). 
Hence $y$ is the unique accumulation point.

Now, the (compact) sets 
\begin{equation*}
\left(K\setminus B(y;1/n)\right)\cap Y
\end{equation*}
must be finite, otherwise one of them would have another accumulation point
of \((y_j)_{j\in\N}\),
Hence $y$ is actually a limit point, \(y=\lim_{n\to\infty} y_n\)
and so \eqref{eq:ex2.16} holds and $f$ is continuous.


For a counterexample showing that compactness of $K$ is necessary,
look at \(f:[0,2]\to \R\) where \(f(x)=\tan(x)\) if \(x\neq \pi/2\)
and \(f(x)=0\) othewise. The graph of~$f$ is closed but clearly $f$
is not continuous.

%%%%%%%%%%%%%%%
\end{enumerate}
%%%%%%%%%%%%%%%

 %%%%%%%%%%%%%%%%%%%%%%%%%%%%%%%%%%%%%%%%%%%%%%%%%%%%%%%%%%%%%%%%%%%%%%%%
%%%%%%%%%%%%%%%%%%%%%%%%%%%%%%%%%%%%%%%%%%%%%%%%%%%%%%%%%%%%%%%%%%%%%%%%
%%%%%%%%%%%%%%%%%%%%%%%%%%%%%%%%%%%%%%%%%%%%%%%%%%%%%%%%%%%%%%%%%%%%%%%%
\chapterTypeout{Convexity}

%%%%%%%%%%%%%%%%%%%%%%%%%%%%%%%%%%%%%%%%%%%%%%%%%%%%%%%%%%%%%%%%%%%%%%%%
%%%%%%%%%%%%%%%%%%%%%%%%%%%%%%%%%%%%%%%%%%%%%%%%%%%%%%%%%%%%%%%%%%%%%%%%
\section{Notes}

\subsection{Missing \texorpdfstring{$t$}{t}}

In Theorem~3.2 Assumption \ich{b}, the condition misses some $t$,
it should be
\begin{equation*}
\ldots \qquad\textnormal{and}\qquad p(tx) = tp(x)
\end{equation*}


\subsection{Lowering Case \texorpdfstring{$a_0$}{a0}}

The proof of Theorem~3.4\ich{a} starts with:
\begin{quote}
\ich{a} Fix \(A_0\in A\),
\end{quote}
It should be:
\begin{quote}
\ich{a} Fix \(a_0\in A\),
\end{quote}


\subsection{\texorpdfstring{$a$}{a} (not alpha)}

The proof of Theorem~3.4\ich{a} has a paragraph that starts with
\begin{quote}
If now \(\alpha\in A\) and
\end{quote}
It should be:
\begin{quote}
If now \(a\in A\) and
\end{quote}

\subsection{\texorpdfstring{$y$ (not $\gamma$)}{y (not gamma)}}

In page~87, exercise 10, instead of:
\begin{quote}
\ldots\; of all real functions \(\gamma\) on $S$
\end{quote}
It should be:
\begin{quote}
\ldots\; of all real functions $y$ on $S$
\end{quote}

\subsection{Holomorphic Functions}

In the proof of Theorem~3.31 equality~(5) can be deduced as follows:
\begin{align*}
\frac{\Lambda f)(z) - (\Lambda f)(0)}{z}
&= \frac{1}{2\pi iz}\int_\Gamma 
   \left(\frac{(\Lambda f)(\zeta)}{\zeta - z} - \frac{(\Lambda f)(\zeta)}{\zeta}
   \right)d\zeta
 = \frac{1}{2\pi i}\int_\Gamma 
   \frac{\zeta(\Lambda f)(\zeta) - (\zeta - z)(\Lambda f)(\zeta)}{
         \zeta(\zeta - z)z} \,d\zeta \\
&= \frac{1}{2\pi i}\
   \int_\Gamma \frac{(\Lambda f)(\zeta)}{\zeta(\zeta - z)} \,d\zeta .
\end{align*}


%%%%%%%%%%%%%%%%%%%%%%%%%%%%%%%%%%%%%%%%%%%%%%%%%%%%%%%%%%%%%%%%%%%%%%%%
%%%%%%%%%%%%%%%%%%%%%%%%%%%%%%%%%%%%%%%%%%%%%%%%%%%%%%%%%%%%%%%%%%%%%%%%
\section{Exercises} % pages 85-91

%%%%%%%%%%%%%%%%%
\begin{enumerate}
%%%%%%%%%%%%%%%%%

%%%%%%% 1
\begin{excopy}
Call a set \(H\subset \R^n\) a
\index{hyperplane}
\emph{hyperplane} if there exist real numbers \seqn{a}, $c$
(with \(a_i\neq 0\) for at least one~$i$) such that~$H$ consists of all points
\(x=(x_1,\ldots,x_n)\) that satisfy \(\sum a_ix_i = c\).

Suppose $E$ is a convex set in \(\R^n\), with nonempty interior,
and $y$ is a boundary point of $E$. Prove that there is a hyperplane plane~$H$
such that \(y\in H\) and~$E$ lies entirely on one side of~$H$.
(state the conclusion more precisely.)
\emph{Suggestion:} Suppose $0$ is an interior point of~$E$,
let $M$ be the one dimensional subspace that contains~$y$, and apply
Theorem~3.2.
\end{excopy}

We are looking for $H$ defined as above, such that
\(y\in H\) and \(\sum a_ix_i \geq c\) for all \(x\in E\).

\Wlogy\ \(0\in \Int(E)\), otherwise we simply shift $X$ by some \(-v\)
where \(v\in\Int(E)\).
Let \(V=B(0;r)\subset \Int(E)\).
We will show that the half open segment \([0,y) = \{ty: 0\leq t<1\} \subset E\).
By negation, there exists
some \(0 < a < 1\) such that \(ay \notin E\).
Let \(\{y_n\}_{n\in\N}\) be a sequence in $E$ such that
\(\lim_{n\to\infty} y_n = y\) and \(\|y_n - y\| < 1-a\).
The last inequality ensures \(0\neq y_n \neq ay\).
For each \(y_n\) look at the line \(\ell_n\) determined by \(y_n\) and \(ay\).
Let \(x_n\in \ell_n\) be (the closest to the origin) such that
\(\|x_n\| = \min(\{\|x\|: x\in\ell_n\})\).
If \(x_n = ay + t(y_n - ay)\) where \(t\in\R\) then we know that
\(X_n\) is perpendicular to \(\ell_n\), hence
\begin{equation*}
0 = \langle x_n, y_n-ay\rangle
  = \langle ay, y_n-ay\rangle + t \|y_n -ay\|_2^2.
\end{equation*}
It is easy to see that \(\lim_{n\to\infty} \|x_n\| = 0\)
and so \(x_n \in V\) for some $n$. Now \(ay\) is a convex combination
of \(\{x_n,y_n\}\) which is a contradiction.

Since $E$ contains the neighborhood $V$ of $0$, it is an absorbing set and
by Theorem~1.35 the
\index{Minkowski functional}
Minkowski functional \(\mu_E(x) = \inf\{t\geq 0: x\in tE\}\)
is a seminorm.

Since \([0,y)\subset E\), \(y\in tE\) whenever \(t>1\)
and so \(\mu_E(y)\leq 1\).

If by negation \(\mu_E(y) < 1\), we can find \(a \in(0,1)\) such that
\(y\in aE\). Hence \(y/a \in aE\).
We look at the homeomorphic mapping
\(h(x) = (1-a)x+a(y/a)\)
restricted to $V$.
It maps each \(v\in V\) to a convex combination of $v$ and \(y/a\),
hence \(h(V) \subset E\) and is open since $h$ is homeomorphic.
But this gives the contradiction \(h(0) = y \in h(V) \Int(E)\).
We have shown
\begin{equation*}
\mu_E(y) = 1.
\end{equation*}

Let $M$ be the subspace generated by $y$ and define
\(f(ay) = a\) for each \(a\in\R\).
By Theorem~3.2 $f$ can be extended to a functional \(\Lambda:\R^n\to\R\)
such that \(-\mu_E(-x) \leq \Lambda x \leq \mu_E(x)\) for all \(x\in\R^n\).
By looking at the standard base of \(\R^n\)
there exist \(\{a_j\in\R: 1\leq j \leq n\}\) such that
\(\Lambda((x_{j=1}^n)) = \sum a_ix_i\). Let \(c=1\)
and now \(\Lambda(x) \leq 1\) for all \(c\in E\).


The condition of $E$ having nonempty interior is actually not necessary.
\Wlogy, we may assume \(0\in E\).
Now look at the vector sub-space \(V\subset\R^n\) spanned by $E$.
Unless the trivial case where \(|E|<2\) then because of convexity
$E$ has nonempty interior relative to $V$.
Now we can apply the initial result.

%%%%%%% 2
\begin{excopy}
Suppose \(L^2=L^2([-1,1])\). with respect to Lebesgue measure.
For each scalar \(\alpha\) let \(E_\alpha\) be the set of all continuous
functions $f$ on \([-1,1]\) such that \(f(0)=\alpha\).
Show that each \(E_\alpha\) is convex and that each is dense in \(L^2\).
thus \(E_\alpha\) and \(E_\beta\) are disjoint convex sets
(if \(\alpha \neq \beta\)) which cannot be separated by any continuous linear
functional \(\Lambda\) on \(L^2\). \emph{Hint:} What is \(\Lambda(E_\alpha)\)?
\end{excopy}

Clearly, every linear combination of
members from \(E_\alpha\) are in \(E_\alpha\).
In particular convex combinations. Hence \(E_\alpha\) is convex.

Let \(f\in E_\alpha\) and \(\epsilon>0\).
We can find some continuous \(g\in C([0,1])\)
such that \(\|f-g\|_2<\epsilon/2\).
Let \(M = \max|g|\) and for \(\delta>0\) define \(h_\delta\in E_\alpha\) by
\begin{equation*}
h_\delta(t) = \left\{\begin{array}{ll}
 g(t) \quad & |t-\alpha|\geq\delta \\
 (t-\alpha)g(\max(0,t-\delta))/\delta \quad &
     \max(0,\alpha-\delta) \leq t \leq \alpha \\
 (t-\alpha)g(\min(1,t+\delta))/\delta \quad &
     \alpha \leq t < \min(1,\alpha+\delta)
     \end{array}\right.
\end{equation*}
Since \(\|g-h\|_2 \leq \sqrt{2\delta} M\)
it is easy to find \(\delta>0\) such that \(\|g-h_\delta\|_2 < \epsilon/2\).
Hence \(E_\alpha\) is dense in \(L^2\).

The image \(\Lambda(E_\alpha)\) must be convex.
Assume \(\alpha\neq\beta\) and by negation that \(\Lambda\) is a functional
such that \(\Lambda(E_\alpha)\cap \Lambda(E_\beta) = \emptyset\).
By \index{Reisz} Reisz representation theorem
(\cite{RudinRCA87} Theorem~2.14) there exists a Borel measure \(\mu\)
such that \(\Lambda(f) = \int_{[-1,1]}f\,d\mu\) for all \(f\in L^2([-1,1])\).
We know that \(\Lambda\neq 0\) and thus \(\mu\neq 0\). By looking
at \(\supp(\mu)\setminus[-\epsilon,\epsilon]\)
is is clear that we can find \(f\in E_\lambda\)
such that \(\Lambda(f)\) is positively and negatively large as we need. Hence
\(\Lambda(E_\alpha)\) and \(\Lambda(E_\beta)\) are unbounded.
Hence \(\Lambda(E_\alpha)=\Lambda(E_\beta)=\R\)
and in particular these images are \emph{not} disjoint.


%%%%%%% 3
\begin{excopy}
Suppose $X$ is a real vector space (without topology).
Call a point \(x_0\in A\subset X\) an \emph{internal} point of~$A$ if
\(A-x_0\) is an absorbing set.
\begin{itemize}
\itemch{a} Suppose $A$ and $B$ are disjoint convex sets in $X$, and $A$
has an internal point.
Prove that there is a nonconstant linear functional \(\Lambda\)
on $X$ such that \(\Lambda(A)\cap \Lambda(B)\) contains at most one point.
(The proof is similar to that of Theorem~3.4)
\itemch{b} Show (with \(X=\R^2\), for example) that it may not be possible
to have \(\Lambda(A)\) and \(\Lambda(B)\) disjoint, under the hypothesis
of \ich{a}.
\end{itemize}
\end{excopy}

\paragraph{\ich{a}}
Let \(a\in A\) be an internal point of $A$ and pick arbitrary \(b\in B\).
Consider the set \(C = A - B + (b-a)\).
Since \(0\in -B+b\) we have \(A-a\subset C\) and so $C$ is absorbing.
It is wasy to see that $C$ is convex, since $A$ and $B$ are.
Look at the
\index{Minkowski} Minkowski functional \(\mu_C\) (See Section~1.33).
By Theorem~1.35\ich{c} \(p=\mu_C\) is a seminorm.
Since \(A\cap B=\emptyset\),
we see that \(x\notin C\), and so \(p(x_0)\geq 1\).

Define \(f(tx_0) = t\) on the subspace \(M=\R x_0\).
By Theorem~3.2, $f$ extends to a linear functional \(\Lambda\) on $X$
such that \(|\Lambda| \leq p\). In particular, \(\Lambda \leq 1\) on $C$.
If now \(a\in A\) and \(b\in B\), we have
\begin{equation*}
\Lambda(a) - \Lambda(b) + 1
= \Lambda(a - b + x_0) \leq p(a-b + x_0) \leq 1.
\end{equation*}
{\small[Note: here it is different from the proof of Theorem~3.4, since
 $C$ (and $A$ as well) is not necessary open.]}'
Thus \(\Lambda(a) \leq \Lambda(b)\).
Therefore \(\sup_{a\in A} \Lambda(a) \leq \inf_{b\in B} \Lambda(b)\)
and \(\Lambda(A)\cap\Lambda(B)\) have one value in common at most.


\paragraph{\ich{b}}
Here is the required example:
\begin{align*}
A &= \{(x,y)\in\R^2: x>0 \;\vee\; x = 0 \wedge y > 0\} \\
B &= \{(x,y)\in\R^2: x<0 \;\vee\; x = 0 \wedge y < 0\}.
\end{align*}


%%%%%%% 4
\begin{excopy}
Let \(\ellinf\) be the space of all real bounded functions $x$ on
positive integers. Let \(\tau\) be the translation operator defined on
\ellinf\ by the equation
\begin{equation*}
(\tau x)(n) = x(n+1) \qquad (n=1,2,3,\ldots).
\end{equation*}
Prove that there exists a linear functional \(\Lambda\)
on \ellinf\ (called a
\index{Banach limit}
\emph{Banach limit}) such that
\begin{itemize}
\itemch{a} \(\Lambda\tau x = \Lambda x\), and
\itemch{b}
 \(\liminf_{n\to\infty} x(n) \leq \Lambda x \leq \limsup_{n\to\infty} x(n)\)
 for every \(x\in\ellinf\).

\end{itemize}
\emph{Suggestion:} Define
\begin{align*}
\Lambda_n(x) &= \frac{x(1) + \cdots + x(n)}{n} \\
M &= \{x\in\ellinf: \lim_{n\to\infty} \Lambda_n x = \Lambda x\;
    \textnormal{exists}\} \\
p(x) &= \limsup_{n\to\infty} \Lambda_n x
\end{align*}
and apply Theorem~3.2.
\end{excopy}

Following the suggestion, except for correcting the definition
\begin{equation*}
p(x) = \limsup_{n\to\infty} |\Lambda_n x|
\end{equation*}
It is clear that \(\Lambda_n\in (\ellinf)^*\)
and that $M$ is a (proper) subspace.

Let \(f(x) = \lim_{n\to\infty} \Lambda_n(x)\) for all \(x\in M\).
Clearly $f$ is linear and \(-p(x)\leq f(x) \leq p(x)\).
Now Theorem~3.2 gives an extension \(\Lambda\) of $f$
such that \(-p(x)\leq \Lambda(x) \leq p(x)\) for all \(x\in \ellinf\).

It is clear that for all \(x\in \ellinf\)
\begin{equation*}
\liminf_{n\to\infty} x(n)
= \liminf_{n\to\infty} x(n+1)
\qquad
 \limsup_{n\to\infty} x(n)
= \limsup_{n\to\infty} x(n+1)
\end{equation*}
and for all \(x\in M\)
\begin{equation*}
\liminf_{n\to\infty} x(n)
 \leq \lim_{n\to\infty}\Lambda(x)
 = \lim_{n\to\infty}\Lambda(\tau x)
 \leq \limsup_{n\to\infty} x(n)
\end{equation*}

Pick \(x\in \ellinf\) and put \(y = x - \tau x\). Now
\begin{equation*}
\Lambda_n(y) = \left(\Lambda(x_1) - \Lambda(x_{n+1})\right)/n
\end{equation*}
hence \(\lim_{n\to\infty} \Lambda_n(y) = 0\)
and so \(y\in M\) and \(\Lambda(y) = 0\).
Thus \ich{a} holds for all \(x\in\ellinf\).


%%%%%%% 5
\begin{excopy}
For \(0<p<\infty\), let \ellp\ be the space of all functions $x$,
(real or complex, as the case may be) on the positive integers, such that
\begin{equation*}
\sum_{n=1}^\infty |x(n)|^p < \infty.
\end{equation*}
For \(1\leq p<\infty\), define \(\|x\|_p = \left\{\sum|x(n)|^p\right\}^{1/p}\),
and define \(\|x\|_\infty = \sup_n|x(n)|\).

\begin{itemize}
%%%%%%%%
\itemch{a}
Assume \(1\leq p<\infty\). Prove that \(\|x\|_p\) and \(\|x\|_\infty\)
make \ellp\ and \ellinf\ into Banach spaces.
If \(p^{-1} + q^{-1} = 1\), prove that \((\ellp)^* = \ell^q\), in the following
sense: There is a one-to-one correspondence \(\Lambda \leftrightarrow y\)
between \((\ellp)^*\) and \(\ell^q\), given by
\begin{equation*}
\Lambda x = \sum x(n)y(n) \qquad (x\in \ellp).
\end{equation*}

%%%%%%%%
\itemch{b}
Assume \(1< p<\infty\) and prove that \(\ellp\) contains sequences that
converge weakly but not strongly.
%%%%%%%%
\itemch{c}
On the other hand, prove that every weakly convergent sequence in \ellone\ %
converges strongly, in spite of th fact that the weak topology of \ellone\ %
is difference from its strongly topology (which is induced by the norm).
%%%%%%%%
\itemch{d}
If \(0<p<1\), prove that \ellp\ metrized by
\begin{equation*}
d(x,y) = \sum |x(n) - y(n)|^p,
\end{equation*}
is a locally bounded $F$-space which is not locally convex
but that \((\ellp)^*\) nevertheless separates points in \ellp,
(Thus there are many convex open sets in \ellp\ but not enough to form a base
for its topology.)
Show that \((\ellp)^* = \ellinf\) in the sense as in \ich{a}.
Show also that the set of all $x$ with \(\sum|x(n)|<1\)
is weakly bounded but not originally bounded.
%%%%%%%%
\itemch{e}
For \(0<p\leq 1\), let \(\tau_p\) be the weak \upstar-topology
induced on \ellinf\ by \ellp; see \ich{a} and \ich{d}.
If \(0<p<r\leq 1\), show that \(\tau_p\) and \(\tau_q\) are \emph{different}
topologies (is one weaker that the other?)
but that they induce the same topology on each norm-bounded subset of \ellinf.
\emph{Hint:} The norm-closed unit ball of \ellinf\ is weak \upstar-compact.
\end{itemize}
\end{excopy}

\begin{itemize}
%%%%%%%%
\itemch{a}
By particular case of Theorems~3.8 and~6.16 of \cite{RudinRCA87}.

%%%%%%%%
\itemch{b}
Take \(e_n\in \ell^p\), that is (e\(_n(m) = \delta_{m,n}\)).
Now \(\|e_n\|_p = 1\), but for all \(x\in\ell^q\)
\begin{equation*}
\Lambda_n(e_n) = x(n) \xrightarrow {n\to\infty}  0.
\end{equation*}

%%%%%%%%
\itemch{c}
Assume
\begin{equation} \label{eq:ex3.5:weakly}
\lim_{j\to\infty}x_j = 0 \quad \textnormal{weakly}.
\end{equation}
If by negation
\(x_j \stackrel{\|\cdot\|_1}{\nrightarrow} 0\)
it implies
\begin{equation*}
\exists \alpha>0,\,\forall m,\,\exists j>m,\; \|x_j\|>\alpha.
\end{equation*}
Hence, by changing \((x_j)_{j\in\N}\) into a subsequence and multiplying
by a constant factor, we may assume
\begin{equation} \label{eq:ex3.5:geq1}
\forall j\in\N,\; \|x_j\|_1 \geq 1.
\end{equation}

Define an increasing sequences of natural numbers
\(s_j\) (subsequence indices)
and
\(b_j\) (blocks)
by induction on $j$.
Let \(s_1=1\) and let \(b_1\) be such that \(\sum_{j<b_1} |x_1(j)| > 1/6\).
Say \(b_n\) and \(s_n\) were defined for \(n\leq k\).
By \eqref{eq:ex3.5:weakly}, intuitively the
``finite initial \(b_{k}\)-blocks'' of
\((x_j)_{j\in\N}\) converge to zero. Formally we can say that
there exists \(m\in\N\) such that
\begin{equation*}
\forall s>m,\; \sum_{j=1}^{k-1} |x_s(j)| < 1/6.
\end{equation*}
With \eqref{eq:ex3.5:geq1} it implies
\begin{equation*}
\forall s>m,\; \sum_{j=k}^\infty |x_s(j)| \geq 5/6.
\end{equation*}

Define \(s_{k+1} = \max(m,s_k)+1\)
and let \(b_{k+1}\) be such that
\begin{equation*}
\sum_{j = b_{k+1}}^\infty |x_{s_{k+1}}(j)| < 1/6
\end{equation*}
which also implies
\begin{equation*}
% \sum_{b_k \leq j < b_{k+1}} |x_{s_{k+1}}(j)|
\sum_{j=b_k}^{b_{k+1}-1} |x_{s_{k+1}}(j)|
\geq \|x_{s_{k+1}}\|_1
     - \sum_{j < b_k} |x_{s_{k+1}}(j)|
     - \sum_{j \geq b_{k+1}} |x_{s_{k+1}}(j)|
\geq 1 - 2/6 = 2/3
\end{equation*}

Now that the sequences
\(s_j\), \(b_j\) are defined, let us
define \(u\in\ellinf\) by blocks.

For each \(j\in \N\)
there is a unique $k$ such that \(b_k \leq j < b_{k+1}\).
Let \(u(j)\) be such that
\begin{equation*}
u(j)\cdot x_{s_k}(j) = |x_{s_k}(j)| \quad\textnormal{and}\quad |u(j)|=1.
\end{equation*}
Considering \(u\in(\ellone)^*\) we now have
\begin{align*}
u(x_{s_k})
&= \left|\sum_{j=1}^\infty u(j) \cdot x_{s_k}(j)\right|
 \geq \left|\sum_{j=b_k}^{b_{k+1}-1} u(j) \cdot x_{s_k}(j)\right|
     - \sum_{j=1}^{b_k-1} |x_{s_k}(j)|
     - \sum_{j=b_{k+1}}^\infty |x_{s_k}(j)| \\
&\geq 2/3 - 1/6 - 1/6 = 1/3
\end{align*}
This is a contradiction to the
weakly convergence \eqref{eq:ex3.5:weakly} assumption.


%%%%%%%%
\itemch{d}
For \(0<p><1\) the space \ellp\ is locally bounded $F$-space
by similar arguments of section~1.47.
The balls \(B_r\) are not convex. To show this take
\begin{equation*}
r' = r^{1/p}, \qquad v_i = r'e_j \quad (j=1,2), \qquad v_{12} = (v_1+v_2)/2.
\end{equation*}
Now \(\|v_i\|_p = r\) and \(\|v_{12}\|_p = 2(r'/2)^p = (2/2^p)r > r\)x.

The equality \((\ellp)^* = \ellinf\) follows again from the
\emph{proofs} of Theorems \cite{RudinRCA87}~3.8 and~6.16, since in these
proofs wherever \(p=1\) the same could be done with \(0 < p \leq 1\).

There is an error in some editions of the ook here.
The unbounded but weakly bound set shoule be
\begin{equation*}
E := \{x\in\ellp: \sum|x(n)|<1\}.
\end{equation*}
Looking on \ellone, it is easy to see that
\begin{equation*}
\forall \Lambda\in (\ellp)^* \cong \ellinf,\,\exists \gamma,\,|\Lambda x|<\gamma,
\end{equation*}
hence $E$ is weakly bounded.
Suppose by negation that $E$ is bounded. Then for
\begin{equation*}
V := \{x\in\ellp: \sum|x(n)|^p < 1\}
\end{equation*}
\(\exists t<\infty,\,E\subset tV\).
In particular
\begin{align}
            & \forall n\in\N,\, \frac{1}{n}\sum_{j=1}^n e_j \in E \subset tV
 \notag \\
\Rightarrow & \forall n\in\N,\, \frac{1}{nt}\sum_{j=1}^n e_j \in E \subset V
 \notag \\
\Rightarrow & \forall n\in\N,\, n\cdot\left(\frac{1}{nt}\right)^p < 1
 \notag \\
\Rightarrow & \forall n\in\N,\, t^p > n^{1-p}. \label{eq:ex3.5d:contra}
\end{align}
But \(\lim_{n\to\infty} n^{1-p} = \infty\)
which contradicts \eqref{eq:ex3.5d:contra}.

%%%%%%%%
\itemch{e}
We show \(\ellp \subset \ell^r\). Let \(v\in \ellp\). Hence
\begin{align*}
& \exists n_0\in\N,\,\forall n>n_0,\,|v(n)|<1 \\
\Rightarrow & \forall n>n_0,\, |v(n)|^r < |v(n)|^p \\
\Rightarrow & v\in \ell^r.
\end{align*}
Therefore \(\ell^r\) induces a stronger topology, so \(\tau_p \subseteq \tau_r\).
By Theorem~3.10 if \(\tau_p = \tau_r\) then \(\ellp = \ell^r\),
but take \(x(n)=n^{-p}\) then \(x\notin \ellp\)
but
\begin{equation*}
\sum_{n\in\N} |x()|^r \leq \int_0^\infty (1/t)^{r/p}\,dt  < \infty
\end{equation*}
hence \(x\in\ell^r\setminus\ellp\), so \(\tau_p \subsetneq \tau_r\).

Let \(E\subset \ellinf\) be a norm bounded set.
So \(\exists M\geq 1,\,\forall x\in E,\,\|x\|_\infty \leq M\).
Let
\begin{equation*}
V_p := \{y\in\ellp: \sum_{n\in\N} |y(n)|<1/M\}.
\end{equation*}
Clearly \(V_p \subset \{v\in\ellinf: \|v\|_\infty \leq 1\}\).
Now  \(\forall\, y\in V_p,\, \forall x\in E\) satisfy:
\begin{align*}
& |\Lambda_x y| = \left|\sum_{n\in\N} x(n)\cdot y(n)\right|
                = \|x\|_\infty \left|\sum_{n\in\N} y(n)\right|
                \leq M \sum_{n\in\N} |y(n)|^p \\
\Rightarrow &
  E \subset K_p :=
  \{\Lambda \in (\ellp)^*\cong \ellinf: \forall y\in V_p,\, |\Lambda y|\leq 1\}.
\end{align*}
The set \(K_p\) is \(\tau_p\)-compact by
\index{Banach-Alaoglu}
\index{Alaoglu}
Banach-Alaoglu Theorem~3.15.
Hence also \(\tau_r\)-compact since it is a weaker topology.
On \(K_p\) both topologies are \(T_2\)-compact, so they coincide, particularly
on~$E$.
\end{itemize}


%%%%%%% 6
\begin{excopy}
Put \(f_n(t)=e^{ixt}\) (\(-\pi \leq t \leq \pi\)); Let
\(L^p = L^p(-\pi,\pi)\), with respect to Lebesgue measure.
If \(1\leq p < \infty\), prove that \(f_n\to 0\) weakly in \(L^p\),
but not strongly.
\end{excopy}

Clearly \(\|f_n\|_p = 1\). But for every trigonometric polynomial $g$
\begin{equation*}
\forall n \geq \deg(g)\,\; \int_{-\pi}^\pi g f_n = 0,
\end{equation*}
so \(\lim_{n\to\infty}\int_{-\pi}^{\pi} gf_n = 0\).

Let $q$ such that \(1/p+1/q=1\) (\(q=\infty\) if \(p=1\)).
We want to show that trigonometric polynomial are densed in \(L^q([-\pi,\pi]\),
By Theorem~4.25 in \cite{RudinRCA87} the trigonometric polynomials
are dense in \(C([-\pi,\pi])\) and by Theorem~3.14 \cite{RudinRCA87}
\(C([-\pi,\pi])\) is dense in \(L^q([-\pi,\pi])\).

Hence for any \(h\in L^q([-\pi,\pi]) \subset L^1([-\pi,\pi])\)
and any \(\epsilon>0\) there exists a trigonometric polynomial $g$
such that \(\int_{-\pi}^\pi |g-h|^q<\epsilon\) and so
\begin{equation*}
\left|\int_{-\pi}^\pi hf_n\right|
\leq \int_{-\pi}^\pi \left|(g - h)f_n\right| 
  + \left|\int_{-\pi}^\pi  gf_n\right|
\stackrel{n\to\infty}{\longrightarrow} < \epsilon.
\end{equation*}
Since \(\epsilon\) was chosen arbitrarily, so
\(\lim_{n\to\infty}\int_{-\pi}^\pi  hf_n = 0\) and
\(f_n \stackrel{n\to\infty}{\longrightarrow} 0\) weakly.

If the convergence were strongly, the it must be to $0$ as well, 
contradiction to \(\|f_n\|_p = 1\).


%%%%%%% 7
\begin{excopy}
\(L^\infty([0,1])\) has its norm topology
\(\|f\|_\infty\) is the essential supremum of \(|f|\) and
its weak \upstar-topology as the dual of \(L^1\).
Show that $C$, the space of all continuous functions on \([0,1]\),
is dense in \(L^\infty\) in one of these topologies but not in the other.
(Compare with the corollaries to Theorem~3.12.)
Show the same with ``closed'' in place of ``dense''.
\end{excopy}

Since \(C \subsetneq L^\infty\),
in any topology $C$ cannot be both closed and dense.
In the \(\|\cdot\|_\infty\)-topology $C$ is closed by
Theorem~7.12 of \cite{RudinPMA85}.
Now we will show that $C$ it is dense in the weak topology induced by \(L^1\).
But first, we will prove the the following assertion.
\begin{llem}
Given a measurable space \((X,\mu,\frakM)\).
If \(\mu(X)<\infty\) and \(f\in L^1(\mu)\) then
\begin{equation*}
\forall \epsilon>0,\, \exists \delta>0,\, \forall A\in\frakM, \;
  \mu(A) < \delta \;\Rightarrow\; \int_A|f|\,d\mu < \epsilon.
\end{equation*}
\end{llem}
\begin{thmproof}
For \(n\in\N\), let
\begin{align*}
X_n &= \{x\in X: n-1 \leq |f|<n\} \\
S_n &= \cup_{j\leq n} X_m \\
T_m &= \cup_{j>n} X_m = X \setminus S_n.
\end{align*}
We have
\begin{equation*}
\sum_{n\in\N} n\mu(X_n) \leq \int_X(|f|+1)\,d\mu \leq \|f\|_1+\mu(X) < \infty.
\end{equation*}
Let \(\epsilon>0\) \(\exists n\, \sum_{m>n} m\mu(X_m) < \epsilon/2\).
Put \(\delta = \epsilon/(2n)\).
Now if \(\mu(A)<\delta\), then
\begin{equation*}
\int_A|f|\,d\mu
\leq \int_{T_n} |f|\,d\mu +  \int_{A\cap S_n} |f|\,d\mu
\leq \sum_{m>n} n\mu(X_n) + \delta n < \epsilon.
\end{equation*}
\end{thmproof}

To establish the denisty, we need to show
\begin{equation*}
\forall f\in L^\infty,\,\epsilon>0,\,h\in L^1,\,\exists g\in C,
\; \int h(f-g)<\epsilon.
\end{equation*}
Pick such arbitrary $f$, \(\epsilon\), $h$.
NOw apply the above lemma with $h$ instead of $f$,
and \(\epsilon/(2\|f\|_\infty\) instead of \(\epsilon\) and get the appropriate
\(\delta>0\). Now by
\index{Lusin}
Lusin Theorem~2.23 in \cite{RudinRCA87}
\begin{equation*}
\exists g\in C(I),\, m\bigl(\{x\in I: g(x)\neq f(x)\}\bigr) < \delta
\qquad \textnormal{and} \qquad \|g\|_\infty \leq \|f\|_\infty.
\end{equation*}
and so
\begin{equation*}
\left|\int h(f-g)\right|
\leq \int |h|\cdot|f-g| \leq 2\|f\|\cdot \int_{\{x: g(x)\neq f(x)\}} |h| < \epsilon.
\end{equation*}

%%%%%%% 8
\begin{excopy}
Let $C$ be the Banach space of all complex continuous functions on \([0,1]\),
with the supremum norm.
Let $B$ be the closed unit ball of $C$.
Show that there exist continuous linear functionals \(\Lambda\) on $C$
for which \(\Lambda(B)\) is an \emph{open} subset of the
complex plane; in particular, \(|\Lambda|\) attains no maximum on $B$.
\end{excopy}

Put \(\Q\cap I\) in a sequences \(\{q_n\}_{n\in\N}\). Put
\begin{equation*}
\Lambda f := \sum_{n-1}^\infty (-1)^n f(q_n).
\end{equation*}
\(\|\Lambda\| \leq 1\) trivially.
For all $n$ \(\exists f_n \in B,\,\forall 1\leq j\leq n,\, f_n(q_j) = (-1)^j\),
so \(\lim_{n\to\infty} \Lambda f_n = 1\), and \(\|\Lambda\|=1\).
But if \(\forall n\in\N,\, h(q_n) = u(-1)^n\) and \(|u|=1\)
then \(h\notin C\), so
\begin{equation*}
\Lambda(B) = \{z\in\C: |z|<1\}.
\end{equation*}

%%%%%%% 9
\begin{excopy}
Let  \(E \subset L^2(-\pi,\pi)\) be the set of all functions
\begin{equation*}
f_{m,n}(t) = e^{imt} + me^{int},
\end{equation*}
where \(m,n\) are integers amd \(0\leq m \leq n\). Let \(E_1\) be the set
of all \(g\in L^2\) such that some sequence in $E$ converges weakly to $g$.
(\(E_1\) is called the
\index{weak sequential closure}
\emph{weak sequential closure} of $E$.)
\begin{itemize}
\itemch{a} Find all \(g\in E_1\).
\itemch{b} Find all $g$ in the weak closure \(\overline{E}_w\) of $E$.
\itemch{c}
Show that \(0\in \overline{E}_w\) bit $0$ is not in \(E_1\),
although $0$ lies in the weak sequential closure of \(E_1\).
\end{itemize}
This example shows that a weak sequential closure need not be
weak sequentially closed. The passage from a set to its weak sequential
closure is therefore not a closure operation, in the sense in which that term
is usually used in topology. (see also Exercise~28.)
\end{excopy}

Abbreviate \(L^2 = L^2(-\pi,\pi)\).
\begin{itemize}
\itemch{a}
Since
\(\lim_{n\to\infty} e^{int} = 0\) weakly, we
clearly have
\begin{equation*}
\lim_{n\to\infty} e^{imt} + me^{int} = e^{int} \qquad\textnormal{weakly}^*.
\end{equation*}
Hence \(\forall m\in\Z^+,\,e^{imt} \in E_1\).

We need to show that these are all the non-trivial weakly sequential limits.
Assume
\begin{equation*}
g_j = f_{m_j,n_j} = e^{im_jt} + m_j e^{in_jt} \qquad (j\in\N)
\end{equation*}
is a sequence with a weak sequential limit \(g\in L^2\).
If \(\{m_j\}_{j\in\N}\) is bounded we can find a subsequence
of \(\{g_j\}_{j\in\N}\) with constant \(\overline{m}=m_j\) and it converges
to \(e^{i\overline{m}t}\).
Otherwise \(\{m_j\}_{j\in\N}\) is unbounded and by moving to a subsequence,
we may assume \(m_j \leq n_j < m_{j+1}\) for all \(j\in\N\).
Note that this also implues \(n_j < n_{j+1}\) for all \(j\in\N\).
Consider now the
function $h$ defined by
\begin{equation*}
h(t) = \sum_{k\in\N} e^{-in_k t} / m_k.
\end{equation*}
Clearly \(h\in L^2\) since
\begin{equation*}
\|h\|_2 = \sum_{k\in\N} m_k^{-2} \leq \sum_{k\in\N} k^{-2}
\leq 1 + \sum_{k=2}^\infty \int_{k-1}^{k} x^{-2}\,dx
= 1 + \int_1^\infty x^{-2}\,dx = 2 < \infty.
\end{equation*}
Now
\begin{align*}
\int_{-\pi}^\pi h(t)\cdot g_j(t)\,dt
&= \int_{-\pi}^\pi \left(\sum_{k\in\N} e^{-in_k t} / m_k\right)
                \cdot \left(e^{im_jt} + m_j e^{in_jt}\right)\,dt
  \\
&= 2\pi \sum_{k\in\N} \delta(n_k,m_k)/m_k + 1 = \infty
\end{align*}
in contradiction, since \(h \cdot g_j\) cannot converge in \(L^2\).

\itemch{b}
Since \(e^{imt} \in E_1\) by what we saw in \ich{a} we have
\(0 \in \overline{E}_w\setminus E_1\).
Put \(E_2 := E_1 \cup \{0\}\).
We will show that actually \(\overline{E}_w = E_2\).
Pick
\(f\in L^2) \setminus E_2\). Take the representation
\begin{equation*}
f(t) = \sum_{k\in\Z} a_je^{ikt}
\qquad \textnormal{where}\quad
\sum_{k\in\Z} |a_k|^2 < \infty
\end{equation*}
We need to show that there is a weak neighborhood $V$ of $f$
such that \(V\cap E_2 = \emptyset\).
Since \(L^2\) is a Hilbert place
we may identify all \(g\in L^2\) with \(\Lambda_g\in (L^2)^*\)
via
\begin{equation*}
\Lambda_g(f) = \langle f, g\rangle
 = \frac{1}{2\pi} \int_{-\pi}^\pi f(t)\overline{g(t)}\,dt.
\end{equation*}
For each such $f$ we will find some \(g\in L^2\) and a neighborhood
\begin{equation*}
V = \Lambda_g^{-1}\bigl(B(\Lambda_g(f);\delta)\bigr)
  = \{h\in L^2: |\Lambda_g(h-f)| < \delta\}.
\end{equation*}
Consider cases by order.
Each case implicitly  assumes previous cases do not hold.
\paragraph{Case 1.} There exists some \(k\in\Z\) such that \(a_k\notin\{0,1\}1\).
We take \(g(t)=e^{-ikt}\) and
\(V = \Lambda_g^{-1}\left(B(a_k;\min(|a_k|,|1-a_k|)/2)\right)\).

\paragraph{Case 2.} There exists some \(k<0\) such that \(a_k=1\).
We take \(g(t)=e^{-ikt}\) and \(V = \Lambda_g^{-1}(B(1;1/2))\).

\paragraph{Case 2.} There is a finite set \(K\subset \Z^+\)
such that \(a_k = 1\) iff \(k\in K\) and \(a_k = 0\) otherwise.
That is \(f(t) = \sum_{k\in K} e^{ikt}\).
Note that the case \(K=\{1,n\}\) with \(1<n\) is impossible
since \(f\notin E\).
We define
\begin{align*}
g_k(t) &= e^{-ikt} \\
V_k &= \Lambda_{g_k}^{-1}(B(1;1/2)) \\
V &= \cap_{k\in K} V_k
\end{align*}
Now $V$ is neighborhood of $f$ and for each \(h\in E_2\)
there exists some \(k\in K\) such that \(h\notin V_k\subset V\).

\itemch{c}
Shown in previous item.
\end{itemize}

%%%%%%% 10
\begin{excopy}
Representation \ellone\ as the space off all real functions $x$ on
\(S = \{(m,n): m\leq 1,\; n \leq 1\}\), such that
\begin{equation*}
\|x\|_1 = \sum |x(m,n)| < \infty.
\end{equation*}
Let \(c_0\) be the space of all real functions $y$ on $S$ such that
\(y(m,n) \to 0\) as \(m+n\to\infty\),
with the norm \(\|y\|_\infty = \sup |y(m,n)|\).

Let $M$ be the subspace of \ellone\ consisting of all \(x\in\ellone\)
that satisfy the equations
\begin{equation*}
mx(m, 1) = \sum_{n=2}^\infty x(m,n) \qquad (m=1,2,3,\ldots).
\end{equation*}
\begin{itemize}
\itemch{a}
Prove that \(\ellone = (c_0)^*\). (see also Exercise~24, Chapter~4.)
\itemch{b}
Prove that $M$ is a norm-closed subspace of \ellone.
\itemch{c}
Prove that $M$ is a weak \upstar-dense in \ellone\ [relative to the
weak \upstar-topology given by \ich{a}].
\itemch{d}
Let $B$ be the norm-closed unit ball of \ellone. In spite of \ich{c},
prove that the weak \upstar-closure of \(M\cap B\) contains no ball.
\emph{Suggestion:} If \(\delta>0\) and \(m > 2/\delta\), then
\begin{equation*}
|x(m,1)| \leq \frac{\|x\|}{m} < \frac{\delta}{2}
\end{equation*}
if \(x\in M \cap B\) although  \(x(m,1) = \delta\) for some \(x\in \delta B\).
Thus \(\delta B\) is not in the weak \upstar-closure of \(M\cap B\).
Extend this to balls with other centers.
\itemch{e}
Put \(x_0(m,1) = m^{-2}\), \(x_0(m,n)=0\) when \(n\geq 2\).
Prove that no \emph{sequence} in $M$ is weak \upstar-convergent to \(x_0\),
in spite of \ich{c}.
\emph{Hint:} Weak \hbox{\upstar-convergence} of
\(\{x_j\}\) to \(x_0\) implies
that \mbox{\(x_j(m,n)\to x_0(m,n)\)} for all \(m,n\), as \(j\to\infty\),
and that \(\{\|x_j\|_1\}\) is bounded.
\end{itemize}
\end{excopy}
\begin{itemize}

\itemch{a}
Identifying $S$ with \N, this was show in \cite{RudinRCA87} Chapter~4,
Exercise~9\ich{a}.

\itemch{b}
% Let \(M_1\) be the weak closure of $M$ in \ellone.
Assume by negation \(x \in \overline{M}\setminus M\).
Then we can find some \(m\in\N\) such that
\begin{equation*}
mx(m, 1) \neq \sum_{n=1}^\infty x(m,n).
\end{equation*}
Let
\begin{equation*}
\delta = \left| mx(m, 1) - \sum_{n=1}^\infty x(m,n) \right| > 0.
\end{equation*}
Now the open ball \(B(x;\delta)\) is a norm neighborhood of $X$
disjoint from $M$, a contradiction to the assumption.
Hence \(\overline{M} = M\).

\itemch{c}
Pick arbitrary \(\xi\in\ellone\) and a base weak\upstar- neighborhood of
it $V$, by picking arbitrary
where \(\seq{v}{k}\in c_0\) and \(\epsilon > 0\) and
defined by
\begin{equation*}
V = \left\{ x\in\ellone: \forall j\in\N_k\;
      \sum_{m,n\in\N}
        \left| v_j(m,n) \cdot \bigl(x(m,n) - \xi(m,n)\bigr) \right|
      < \epsilon
    \right\}.
\end{equation*}
To show that $M$ is weak\upstar\ dense, we need to find some \(x\in M\cap V\).
Define the ``row errors''
\begin{equation*}
e_m = mx(m,1) - \sum_{n=2}^\infty x(m,n) \qquad (m \in \N).
\end{equation*}
We have the following finite values
\begin{align*}
\sup_{m\in\N} |e_m|/m &\leq \sum_{m\in\N} |e_m|/m
   \leq \sum_{m\in\N} \left(|x(m,1)| + (1/m)\sum_{n=2}^\infty |x(m,n)|\right) \\
  &\leq \sum_{m\in\N} \left(|x(m,1)| + \sum_{n=2}^\infty |x(m,n)|\right)
   = 2\|x\|_1 < \infty \\
\|\xi\|_\infty &= \max_{m,n\in\N} |\xi(m,n)| \leq \|\xi\|_1 < \infty.
\end{align*}
Since \(\seq{v}{k}\in c_0\), for all \(j\in\N_k\).
\begin{itemize}
\item
There exists some \(\mu<0\) such that
\begin{equation*}
\left(\sup_{m\in\N} |e_m|/m\right)\sum_{m=\mu+1}^\infty |v_j(m,1)| < \epsilon/2.
\end{equation*}
\item
For each \(m\in\N\)
(just \(m\in\N_\mu\) is needed)
we can find some \(\nu_m\in\N\)
such that
\begin{equation*}
|v_j(m,\nu_m)\cdot e_m| < 2^{-m-1}\epsilon.
\end{equation*}
\end{itemize}
Now we define \(x\in\ellone\) as follows
\begin{equation*}
x(m,n) =
\left\{
 \begin{array}{ll}
 \xi(m,n) \quad &
    \textnormal{if}\; m \leq \mu \;\textnormal{or}\; n \neq \nu_m \\
 \xi(m,n) + e_m \quad &
    \textnormal{if}\;  m \leq \mu \;\textnormal{and}\;n = \nu_m \\
 \xi(m,1) - e_m/m \quad &
    \textnormal{if}\;  m > \mu
 \end{array}
\right.
\end{equation*}
Now
\begin{equation*}
\|x\|_1
 \leq \|\xi\|_1 + \sum_{m=1}^\mu |e_m| + \sum_{m=\mu+1}^\infty |\xi(m,1)|
 \leq 2\|\xi\|_1 + \sum_{m=1}^\mu |e_m|
 < \infty
\end{equation*}
and so \(x\in\ellone\) and directly by definition \(x\in M\).
Finally for each \(j\in\N_k\) we have
\begin{align*}
d_j
&=
\sum_{m,n\in\N} \left| v_j(m,n) \cdot \bigl(x(m,n) - \xi(m,n)\bigr) \right| \\
&=   \sum_{m=1}^\mu | v_j(m,\nu_m)\cdot e_m |
  + \sum_{m=\mu+1}^\infty  |v_j(m,1) \cdot e_m/m | \\
&\leq \sum_{m=1}^\mu  2^{-m-1}\epsilon
    + \left(\sup_{m\in\N} |e_m|/m\right)\sum_{m=\mu+1}^\infty  |v_j(m,1)|
 \leq \epsilon/2 + \epsilon/2 = \epsilon.
\end{align*}
Hence \(x\in V\) and thus $M$ is a weak \upstar-dense in \ellone.

\itemch{d}
Put \(G = M \cap B\) and $H$ is the weak\upstar closure of $G$.
The suggestion shows that $H$ does not contain any non trivial balls
 centered at~$0$.

By negation assume \(C := \overline{B(x;r)} \subset H\) for some \(r>0\)
and clearly \(x\in B = \overline{B(0;1)}\).
Since \ellone\ is locally convex and $C$ is clearly convex,
Corollary~\ich{a} of Theorem~3.12, that $C$ is also weak\upstar\ closed.
Since \(c_0\) separates points in \ellone, hence any singleton \ellone\
is weak\upstar\ closed. Thus \(G\cap C\) contains at lease two distinct points.
Being a vector space, $M$ is also convex, hence $C$ contains some
internal point \(y\in M\), and some \(s>0\) such that
\(B(y;s) \subset B(x,r)\).
Translation is homeomorphism in both topologies, \(M + y = M\) and
so \(B(0;s) \subset M \cap B\), which contradicts the initial result
shown by the suggestion.

\itemch{e}
We begin by proving the claim in the hint by steps.

\begin{llem}
Let \(\{x_j\}\) to \(x_0\) weakly in \ellone\ then
  \(\{\|x_j\|_1\}\) is bounded.
\end{llem}
\begin{proof}
{\nullfont dum}
\newline
\textbf{Claim-I.}
For any \(m,n>0\) the set \(\{|x_j(m,n)|: j\in\N\}\)  is bounded.
This is trivial by looking at \(y\in c_0\) defined by \(y(j,k)=1\)
if \((j,k)=(m,n)\) and \(y(j,k)=0\) otherwise.

\textbf{Claim-II.}
For any finite set $F$ of pairs
\(m,n>0\) the set
\begin{equation*}
\{|x_j(m,n)|: j\in\N\;\land\; (m,n)\in F\}
\end{equation*}
is bounded.
\newline
The bound is a simple maximum taken on the finite sets of
bounds each of a fixed pair \((m,n)\in F\).

Now by negation assume that the set  \(\{\|x_j\|_1:j\in\N\}\) is not bounded.
We will construct \(y\in c_0\) such that the set
\(\{|y(x_j)|:\,j\in\N\}\) is not bounded, that is by changing to
sub-sequence \((s(j))_{j\in\N}\) we can have
\begin{equation*}
\lim_{j\to\infty}|y(x_{s(j)})|
 = \lim_{j\to\infty} \left| \sum_{(m,n)\in\N^2} y(m,n)\cdot x_{s(j)}(m,n) \right|
 = \infty
\end{equation*}
which contradicts the assumption of this lemma.

We start by constructing mutually disjoints finite
sets \(D_n \subset \N^2\) by induction.
We put
\begin{alignat*}{2}
E_0 &= 0    &  d_0 &= 0 \\
E_j &= E_{j-1} \cup D_j   \qquad &   d_j &= \max\{m, n: (m,n)\in E_j\}
   \qquad (\forall j\in\N)
\end{alignat*}
Let \(D_0 = \emptyset\).
Assume \(D_j\) are defined for all \(j < k\).
Let $b$ be a bound (provided by Claim~II) of
\begin{equation*}
\{|x_j(m,n)|: \, j\in\N \land (m,n) \in E_{k-1}\}.
\end{equation*}
pick an index \(s(k)\) such that \(x_{s(k)}\) satisfy the following
\begin{align*}
\sum_{(m,n) \in \N^2 \setminus E_{k-1}} |x(m,n)| > k(b + k + 1)
\end{align*}
Since \(x_j\in \ellone\) we can find some \(d_k\) such that
\begin{align*}
\sum_{(m,n > d_k) \in \N^2 \setminus E_{k-1}} |x(m,n)| < 1
\end{align*}
We define \(\lrcorner\)-like set
\begin{equation*}
D_k = \{(m,n)\in \N^2: m,n \leq d_k\} \setminus D_{k-1}
\end{equation*}
Clearly \(\N^2 = \disjunion_{j\in\N} D_j\), so we can define
\(y(m,n) = e^{i\theta(m,n)}/s(k)\) iff \((m,n)\in D_k\)
setting the argument \(\theta(m,n)\)
so that \(e^{i\theta(m,n)} x_{s(k)}(m,n) = |x_{s(k)}(m,n)|\).
Using the abbreviation
\begin{equation*}
c(m,n) =  y(m,n)\cdot x_{s(k)}(m,n)
\end{equation*}
 we have
\begin{align*}
|y(x_{s_k})|
&= \left|
   \left(\sum_{(m,n)\in E_{k-1}} c(m,n)\right)
   + \left(\sum_{(m,n)\in D_k} c(m,n)\right)
   + \left(\sum_{(m,n)\in \N^2 \setminus E_k} c(m,n)\right)
   \right| \\
&\geq
   \left|\sum_{(m,n)\in D_k} c(m,n)\right|
  - \left|\sum_{(m,n)\in E_{k-1}} c(m,n)\right|
  - \left|\sum_{(m,n)\in \N^2 \setminus E_k} c(m,n)\right| \\
&\geq
   \left(\sum_{(m,n)\in D_k} x_{s_k}(m,n)\right)\bigm/ k
  - \sum_{(m,n)\in E_{k-1}} |c(m,n)|
  - \sum_{(m,n)\in \N^2 \setminus E_k} |c(m,n)| \\
&\geq (b+k+1) - b - 1 = k.
\end{align*}
\end{proof}


With the lemma established, assume by negation that the sequence \(\{x_j\}\)
where \(x_j\in M\),
converges \upstar-weakly to \(x_0\).
Let $M$ be the bound of \(\{\|x_j\|_1:j\in\N\}\) provided by the lemma.
Let $K$ be such that \(\sum_{j=1}^K 1/j > M + 1\).
%Let \(v\in c_0\) defined by \(v(m,n)=1\) iff \(1\leq m \leq K\) and \(n=1\)
% and  \(v(m,n)=0\) otherwise.
By convergence, for each \(\epsilon>0\) there must exists some $k$ such that
if \(y = x_j \in M\) then
\begin{equation*}
|y(m,1) - x_0(m,1)| = |y(m,1) - m^{-2}| < \epsilon
  \qquad (1 \leq m \leq K)
\end{equation*}
But then,
\begin{align*}
\|y\|_1
&= \sum_{m=1}^\infty \left(|y(m,1)| + \sum_{j=2}^\infty|y(m,j)|\right)
 \geq \sum_{m=1}^\infty
    \left(|y(m,1)| + \left|\sum_{j=2}^\infty|y(m,j)\right|\right)   \\
&= \sum_{m=1}^\infty |y(m,1)| +m|y(m,1)|
 = \sum_{m=1}^\infty (m+1)|y(m,1)| \\
&\geq \sum_{m=1}^K (m+1)(m^{-2} - \epsilon)
 \geq \left(\sum_{m=1}^K m^{-1}\right) (K+1)\epsilon.
\end{align*}
But \(\epsilon\) can be arbitrary small, thus
we get the contradiction
\(\|y\|_1 = \|x_j\|_1 > M\).
\end{itemize}

%%%%%%% 11
\begin{excopy}
Let $X$ be an infinite-dimensional
\index{Frechet@Fr\'echet}
Fr\'echet space. Prove that \(X^*\), with its weak\upstar-topology is
of the first category in itself.
\end{excopy}

Let $d$ be the invariant metric on $X$.
Let
\begin{align*}
V &= \{x\in X: d(x,0) < 1\} \\
B_n &= \left\{\Lambda\in X^*: \sup_{x\in V} |\Lambda(x)| \leq n\right\}.
\end{align*}
\index{Banach-Alaoglu}
Clearly \(B_n = nB_1\).
By Banach-Alaoglu Theorem~3.15 \(B_n\) are weak\upstar-compact sets.
and thus are closed.
Assume by negation that \(B_n\)  contains non empty interior for some $n$.
Then by the remark
at the end of section~3.12 (using tranlation to the origin)
applied to \(X^*\) which is obviously infinite-dimensional,
\(B_n\) is not weak\upstar\ bounded.
But this contradicts \(B_n\)'s compactness.
Now
\(X^* = \cup_{n\in\N} B_n\) hence a countable union of nowhere desnse sets.
Hence \(X^*\) is of first category in its weak\upstar-topology.

%%%%%%% 12
\begin{excopy}
Show that the norm-closed unit ball of \(c_0\) is not weakly compact.
recall that \((c_0)^* = \ellone\) (Exercise~10).
\end{excopy}

Let \(u_n\in c_0\) be defined by \(u_n(j) = \delta_{n,j}\).
Clearly  \(u_n\) are in the unit ball. This sequence
has no weak-accumulation point
(and in particular, no accumulation in the unit ball). Thus the unit ball
is not weakly compact.

%%%%%%% 13
\begin{excopy}
Put \(f_N(t) = N^{-1} \sum_{n=1}^{N^2} e^{int}\). Prove that \(f_n\to 0\) weakly
in \(L^2(-\pi,\pi)\).

By Theorem~3.13, some sequence of convex combinations of the \(f_N\)
converges to $0$ in the \(L^2\) norm.
Find such a sequence.
Show that \(g_N = N^{-1}(f_1 + \cdots + f_N)\) will not do.
\end{excopy}

\iffalse
We first solve the following
(similar Exercise~7 of Chapter~3 in \cite{RudinPMA85})

\begin{llem}
If \(\sum_{n=1}^\infty a_n^2 < \infty\) where \(a_n \geq 0\)
then the series \sum_{n=1}^\infty a_n/n\) converges.
\end{llem}
\begin{proof}
\Wlogy\ we assume \(a_n \geq 0\).
Using the root ratio test (Theorem~3.33 in \cite{RudinPMA85}.
\begin{align*}
\limsup_{n\to\infty} \sqrt[n]{a_n/n}
=       \left(\limsup_{n\to\infty} \sqrt[n]{a_n}\right)
  \cdot \left(\lim_{n\to\infty} \sqrt[n]{1/n}\right) \\
&= \limsup_{n\to\infty} \sqrt[n]{a_n}
\end{align*}
\end{proof}
\fi

We assume
 \(\langle f,g \rangle = \frac{1}{2\pi}\int_{-\pi}^\pi f(t)\overline{g(t)}\,dt\).

To show weakly convergence to~0, we pick arbitrary \(h\in L^2(-\pi,\pi)\)
such that \(h(t) = \sum_{n\in\Z} a_n e^{int}\)
with norm  and \(\|h\|_2 = \sum_{n\in\Z} |a_n|^2 < \infty\).
Note that by looking at the cases
\(n|a_n| \leq 1\) or \(n|a_n| \geq 1\) we always have
\(|a_n|/n \leq \max(1/n^2, |a_n|^2)\).


\iffalse
% Take an arbitrary \(h \in  \(L^2 = L^2(-\pi,\pi) \equiv (L^2)^*\).
% We have the representation \(h(t) = \sum_{n=-\infty}^\infty a_n e^{int}\)
% with norm \(\|h\|_2 = \sum{n=-\infty}^\infty |a_n|^2 < \infty\).
Using the root ratio test (Theorem~3.33 in \cite{RudinPMA85}
we know that
\begin{equation*}
\limsup_{n\to\infty} \sqrt[n]{|a_n|^2} < 1.
\end{equation*}
\fi

Pick arbitrary \(\epsilon>0\).
There exists \(t\) such that
both \(\sum_{n>t}  1/n^2 < \epsilon\)
and \(\sum_{n>t}  |a_n|^2 < \epsilon\).
Pick \(T \geq t\) such that
\begin{equation*}
T^{-1} \sum_{n=1}^{t^2} |a_n| < \epsilon.
\end{equation*}
Now for \(N \geq T\)
\begin{align*}
\left|\langle h, f_N \rangle\right|
&= \frac{1}{2\pi} \left|\int_{-\pi}^\pi h(t)\overline{f_N(t)}\,dt\right|
 = N^{-1} \left| \sum_{n=1}^{N^2} a_n \right| \\
&\leq N^{-1} \sum_{n=1}^{N^2} |a_n|
  \leq N^{-1} \sum_{n=1}^{T^2} |a_n| + N^{-1} \sum_{n=t^2+1}^\infty |a_n|  \\
 &\leq T^{-1} \sum_{n=1}^{t^2} |a_n| + \sum_{n=t^2+1}^\infty \max(1/n^2, |a_n|^2)
  \leq \epsilon + \sum_{n=t^2+1}^\infty 1/n^2 + \sum_{n=t^2+1}^\infty |a_n|^2 \\
 &\leq 3\epsilon
\end{align*}
Hence \(\lim_{N\to\infty} \langle  h, f_N \rangle = 0\).
(This is similar to Exercise~7 of Chapter~3 in \cite{RudinPMA85}.)

Compute \(L^2\) norms.
\begin{align}
 e_n(t) &= e^{int} \label{eq:en:eint} \\
\|e_n\|_2 &= 1
  % \qquad \textnormal{conventional division by}\; 2\pi
  \notag \\
\|f_n\|_2^2 &=  n^{-2} \sum_{j=1}^{n^2} \|e_n\|_2^2 = 1 \notag \\
g_N &=  N^{-1}\sum_{n=1}^N f_n
    = N^{-1}\sum_{n=1}^N \left(\sum_{k=n}^N 1/k\right)
                       \left(\sum_{k=(n-1)^2+1}^{n^2} e_k\right) \notag \\
\|g_N\|_2^2 &=  N^{-2}\sum_{n=1}^N \left(\sum_{k=n}^N 1/k\right)^2
                                \left(n^2 - (n-1)^2\right)
            =  N^{-2}\sum_{n=1}^N (2n-1)\left(\sum_{k=n}^N 1/k\right)^2
  \label{eq:ex3.13:gN2}
\end{align}

In order to analyze the behavior of \(\|g_N\|_2\) we diverge
to show some equalities related to harmonic numbers.

Given the finite sequence
\((a_j)_{j=l}^{h}\) and
\((b_j)_{j=l}^{h}\),
we have the  \emph{summation by parts} formula
\begin{equation} \label{eq:sumbyparts}
\sum_{j=l}^{h-1} (a_{j+1} - a_j)b_j
 = a_{h} b_{h} - a_l b_l - \sum_{j=l}^{h-1} a_{j+1}(b_{j+1} - b_j).
\end{equation}
which can be easily proved by induction on \(h-l\).

Define the harmonic numbers:
\begin{equation} \label{eq:harmonic:numbers}
H_n = \sum_{k=1}^n 1/k
\end{equation}
Clearly \(\sum_{k=l}^h = H_h - H_{l-1}\).

We will use \eqref{eq:sumbyparts} with \(l=1\), \(h=N+1\), that is:
\begin{equation}   \label{eq:sumbyparts:1N}
\sum_{n=1}^N n \sum_{k=n}^N 1/k
= a_{N+1}b_{N+1} - a_1 b_1 - \sum_{n=1}^N a_{n+1}(b_{n+1} - b_n)
\end{equation}
with some special cases for
\((a_n)_{n=1}^{N+1}\) and \((b_n)_{n=1}^{N+1}\).

\textbf{Case 1.}
\begin{alignat*}{2}
a_n &= (n^2-n)/2  \qquad  &           b_n &= \sum_{k=n}^N 1/k \\
a_{n+1} - a_n &= n  &  \qquad b_{n+1} - b_n &= -1/n \\
a_1 &= 0           &               b_{N+1} &= 0.
\end{alignat*}
Now \eqref{eq:sumbyparts:1N} becomes
\begin{align}
\sum_{n=1}^N n \sum_{k=n}^N 1/k
&= a_{N+1}\cdot 0 - 0 \cdot b_1 - \sum_{n=1}^N \left((n^2+n)/2\right)(-1/n)
= \frac{1}{2} \sum_{n=1}^N (n+1)
\notag \\
&= (N(N+1)/2+N)/2
% = (n(n+1)+2n)/4
 = N(N+3)/4 \label{eq:harmonic:case1}
\end{align}


\textbf{Case 2.}
\begin{alignat*}{2}
a_n &= (n-1)^2      &           b_n &= \sum_{k=n}^N 1/k \\
a_{n+1} - a_n &= 2n-1 \qquad &  \qquad b_{n+1} - b_n &= -1/n \\
a_1 &= 0              &               b_{N+1} &= 0.
\end{alignat*}
Now \eqref{eq:sumbyparts:1N} becomes
\begin{equation*}
\sum_{n=1}^N (2n-1) \sum_{k=n}^N 1/k
= a_{N+1}\cdot 0 - 0 \cdot b_1 - \sum_{n=1}^N n^2(-1/n)
= \frac{1}{2} \sum_{n=1}^N n
= N(N+1)/2
\end{equation*}


\textbf{Case 3.}
\begin{alignat*}{2}
a_n &= (n-1)^2  \qquad &           b_n &= \left(\sum_{k=n}^N 1/k\right)^2 \\
a_{n+1} - a_n &= 2n-1 \qquad & \qquad
  b_{n+1} - b_n &=
    \left(2\left(\sum_{k=n}^N 1/k\right) - 1/n\right)\cdot(-1/n) \\
a_1 &= 0           &               b_{N+1} &= 0.
\end{alignat*}
Now using  \eqref{eq:sumbyparts:1N} and \eqref{eq:harmonic:case1} we get
\begin{align}
\sum_{n=1}^N (2n-1) \left(\sum_{k=n}^N 1/k\right)^2
&= a_{N+1}\cdot 0 - 0 \cdot b_1
   - \sum_{n=1}^N n^2
     \cdot
     \left(2\left(\sum_{k=n}^N 1/k\right) - 1/n\right)\cdot(-1/n) \notag \\
&= \sum_{n=1}^N n \left(2\left(\sum_{k=n}^N 1/k\right) - 1/n\right)
 = \left(2\sum_{n=1}^N n \sum_{k=n}^N 1/k\right)
    - \sum_{n=1}^N n(1/n) \notag \\
&= 2N(N+3)/4 - N
 = N(N+1)/2 \label{eq:harmonic:for:gN}
\end{align}

We now can show that \(g_N\) cannot converge to zero.
By \eqref{eq:ex3.13:gN2} and \eqref{eq:harmonic:for:gN} we have
\begin{equation*}
\|g_N\|_2^2 = N(N+1)/(2N^2) \geq 1/2.
\end{equation*}
Actually \(\lim_{N\to\infty} \|g_N\|_2 = (1/2)^{1/2} > 0\).
Thus clearly \(g_n\) will not do.

\paragraph{Converging to $0$}.
Finally we show that the following, using \eqref{eq:harmonic:numbers},
sequence of convex combinations of \(\{f_j: j\in\N\}\) converges to~$0$.
\begin{equation*}
g_N = (1/H_N)\sum_{n=1}^N f_n /n
\end{equation*}
Hence
\begin{equation*}
H_N \cdot g_N = \sum_{n=1}^N f_n/n.
\end{equation*}
With \eqref{eq:en:eint} we note that
\begin{equation*}
H_N \cdot g_N = \sum_{n=1}^N \left(\sum_{k=n}^N 1/k\right)\cdot
   \left(\sum_{k=(n-1)^2+1}^{n^2} e_k/k\right)
\end{equation*}
and
\begin{equation*}
\sum_{k=m}^n 1/k^2 \leq \int_m^{n+1} x^{-2}\,dx
% = 1/(n+1) - 1/m
= \frac{1}{n+1} - \frac{1}{m}.
\end{equation*}
Now using summation by parts
with \(a_n=(n-1)^2\) and \(b_n = \sum_{k=n}^N 1/k^2\), \(b_{N+1} = 0\)
we get
\begin{align*}
\|H_N \cdot g_N\|_2^2
&= \sum_{n=1}^N \left(n^2 - (n-1)^2\right) \sum_{k=n}^N 1/k^2 \\
% \leq \sum_{n=1}^N (2n - 1)(1/(n+1) - 1/m) \\
&= a_{N+1} \cdot 0 - 0 \cdot b_0
  - \sum_{n=1}^N n^2
    \left(
      \left(\sum_{k=n+1}^N \frac{1}{k^2}\right)^2
      -
      \left(\sum_{k=n}^N \frac{1}{k^2}\right)^2
    \right) \\
&= \sum_{n=1}^N n^2
   \left(
     \left(2\sum_{k=n}^N \frac{1}{k^2}\right) - \frac{1}{n^2}
   \right)
   \cdot \left(\frac{1}{n^2}\right)
 = \sum_{n=1}^N \left(2\sum_{k=n}^N \frac{1}{k^2}\right) - \frac{1}{n^2} \\
&\leq 2 \sum_{n=1}^N \sum_{k=n}^N \frac{1}{k^2}
 \leq 2 \sum_{n=1}^N \bigl(1/n - 1/(N+1)\bigr)
 \leq 2 \sum_{n=1}^N 1/n  = 2H_N.
\end{align*}
Hence
\(\|g_N\|_2^2 \leq 2/H_N \)
and
\(\lim_{N\to\infty} \|g_N\|_2 = 0\).


%%%%%%% 14
\begin{excopy}
\begin{itemize}
%%
\itemch{a}
Suppose \(\Omega\) is a locally compact Hausdorff space.
For each compact \(K\subset \Omega\) define a seminorm \(p_K\) on
\(C(\Omega)\), the space of all complex continuous functions on \(\Omega\), by
\begin{equation*}
p_K(f) = \sup \{|f(x)|: x\in K\}.
\end{equation*}
Give \(C(\Omega)\) the topology induced by this collection of seminorms.
Prove that to every \(\Lambda \in C(\Omega)^*\) correspond a compact
 \(K\subset \Omega\) and a complex Borel  measurable \(\mu\) on $K$ such that
\begin{equation*}
\Lambda f = \int_K f\,d\mu \qquad [f \in C(\Omega).]
\end{equation*}
%%
\itemch{b}
Suppose \(\Omega\) is an open set in \(\C\).
Find a countable collection \(\Gamma\) of measures with compact support in
\(\Omega\) such that \(H(\Omega)\) (the space of all holomorphic functions
in \(\Omega\)) consists of exactly those \(f\in C(\Omega)\) which satisfy
\(\int f\,d\mu = 0\) for every \(\mu\in\Gamma\).
\end{itemize}
\end{excopy}

\begin{itemize}
%%
\itemch{a}
Pick \(\Lambda \in C(\Omega)^*\) and let 
\begin{equation*}
V = \{f \in C(\Omega): |\Lambda f| < 1\}
\end{equation*}
be a neighborhood of the origin. 
It must contain some base neighborhood. In other words
there exists finite number of compact sets \(\{K_j: 1 \leq j \leq n\}\)
such that \(K_j \subset \Omega\) and \(\delta_j > 0\) for \(j \in \N_n\)
and
\begin{equation*}
U := \bigcap_{j=1}^n \{f \in C(\Omega): p_{K_j}(f) < \delta_j\} \subset V.
\end{equation*}
If we put \(K = \cup _{j=1}^n K_j\) and \(\delta = \min\{\delta_j: j\in \N_n\}\)
then 
\begin{equation*}
U' =  \{f \in C(\Omega): p_K(f) < \delta\} \subset U \subset V.
\end{equation*}
Now if \(f \in C(\Omega)\) if \(f_{\restriction \Omega \setminus K} \equiv 0\)
then \(af \in U'\) for all \(a\in \C\) and so 
\(|a\Lambda f| < \delta\)  for all \(a\in \C\), hence \(\Lambda f = 0\).
Thus we can define a functional \(\Lambda_K : C(K) \to \C\)
by \(\Lambda_K f = \Lambda \tilde{f}\)
where \(\tilde{f}\) is some continuous extension of \(f\in C(K)\)
to \(C(\Omega)\) (for example, using Urysohn's lemma).
Now the topology of \(C(K)\) induced by that of \(C(\Omega)\)
is the same as that of the supremum-norm \(\|\cdot\|_\infty\) in $K$.
Applying Riesz representation theorem 
(Theorem~6.19 \cite{RudinRCA87}) gives the desired result.

%%
\itemch{b}
For each closed triangle \(T\subset \Omega\) let \(\gamma_S\)
be its parametrized piecewise differntable boundary
\(\gamma_S : [0,1] \to \boundary{S}\).
Define the regular complex integration for each \(f\in C(\Omega)\)
\begin{equation*}
\int_\Omega  f\,d\mu_S = \frac{1}{2\pi i}\int_0^4 f(\gamma_S(t))\gamma_S'(t)\,dt
\end{equation*}
Now let \(\Gamma\) be the collection of all such \(\mu_S\).
By Morera's Theorem \index{Morera's Theorem} (\cite{RudinRCA87} 10.17)
this \(\Gamma\) with the condition above characterize \(H(\Omega)\).
\end{itemize}

%%%%%%% 15
\begin{excopy}
Let $X$ be a topological vector space on which \(X^*\) separates points.
Prove that the weak \upstar-topology of \(X^*\) is metrizable if and only if
$X$ has a finite or countable Hamel base.
(See Exercise~1, Chapter~2 for the definition.)
\end{excopy}

Assume \(\{u_n: n\in\N\}\) is a Hamel base for $X$. Then
\begin{equation*}
d(\phi,\psi) = \sum_{n=1}^\infty 2^{-n}\cdot |\phi(u_n) - \psi(u_n)|
\end{equation*}
is a metric for \(X^*\). The fact that \(X^*\) separates points
gives \(d(\phi,\psi)=0\) iff \(\phi = \psi\).

(Inspired by \cite{Munkres2000} 2.\S{}21 \textsc{Example}~2.)\newline
Conversely assume $X$ is metrizable with a metric $d$
and \(\{x_\alpha: \alpha \in J\}\)
is a Hamel basis for $X$. By negation assume $J$ is uncountable.
Each \(x\in X\) has a unique representation
\begin{equation*}
x = \sum_{\alpha \in F_x} a_\alpha x_\alpha \qquad
   (a_\alpha \in \C \;\wedge\; |F_x| < \infty)
\end{equation*}
Each \(\Lambda \in X^*\) is determined by the values
 \(\{\Lambda x_\alpha: \alpha \in J\}\).
Now define the set \(A \subset X^*\) such that 
it is ``almost always''~$1$ on the basis, formally
\begin{equation*}
A = \bigl\{ \Lambda\in X^*: 
      \{\alpha \in J: \Lambda x_\alpha \neq 1\} < \infty 
    \bigr\}.
\end{equation*}
We will show that \(0 \in \overline{A}\).
Pick a base neighborhood of \(0\in X^*\) by selecting some finte 
set \(F \subset J\) and \(\delta>0\) and define
\begin{equation*}
V = \{\Lambda \in X^*: \forall \alpha\in F\; |\Lambda x_\alpha| < \delta\}.
\end{equation*}
Define \(T\in X^*\) by 
\begin{equation*}
\T x_\alpha = 
\left\{
 \begin{array}{ll}
 0 \quad& \alpha \in F \\
 1 \quad& \alpha \notin F
 \end{array}
\right.
\end{equation*}
Clearly \(T \in V\), hence \(0 \in \overline{A}\).

By the assumed existence of $d$, we define the open balls
\begin{equation*}
B_n = \{\Lambda \in X^*: d(0,\Lambda) < 1/n\}.
\end{equation*}
Since \(0 \in \overline{A} \subset X^*\), we can find
\(a_n \in B_n \cap A\) thus 
\begin{equation} \label{eq:metriz:lim:a_n:0}
\lim_{n\to\infty} a_n = 0\,.
\end{equation}
Define ``non-$1$'' sets
\begin{equation*}
F_n = \{\alpha \in J: a_n(x_\alpha) \neq 1\}.
\end{equation*}
By definition of $A$, we have \(|F_n| < \infty\)
and so \(|\cup_{n\in\N} F_n|< \aleph_0\).
But $J$ is uncountable (the Hamel basis) and hence there exists
\(\gamma \in J \setminus \cup_{n\in\N} F_n\).
Thus \(a_n(x_\gamma) = 1\) for all \(n\in\N\), 
which contradicts \eqref{eq:metriz:lim:a_n:0}.


%%%%%%% 16
\begin{excopy}
Prove that the close unit ball of \(L^1\) (relative to the Lebesgue measure
on the unit interval) has no extreme points but that every point on the
"surface" of the unit ball in \(L^p\) (\(1<p<\infty\)) is an extreme
point of the ball.
\end{excopy}

\paragraph{No exterme.}
Given \(f\in L^2([0,1])\) such that \(\|f\|_1 = 1\),
% Define \(a:[0,1] \to \{z\in\C: |z|=1\}\) by \(a(t) = f(t)/|f(t)|\) if
% \(f(t)\neq 0\) and \(a(t)=1\) otherwise.
by regularity of the Lebesgue measure, there exists \(b\in[0,1]\)
such that
\begin{equation*}
\int_0^b |f(t)|\,dt = \int_b^1 |f(t)|\,dt = 1/2.
\end{equation*}
Now define \(f_1,f_2\in L^1([0.1])\) by
\begin{equation}  \label{eq:f1f2:conv}
\bigl(f_1(t),f_2(t)\bigr) =
\left\{
\begin{array}{ll}
\bigl(0,2f(t)\bigr) \quad & 0 \leq t \leq b \\
\bigl(2f(t),0\bigr) \quad & b \leq t \leq 1
\end{array}
\right.
\end{equation}
Clearly \(\|f_1\|_1 = \|f_2\|_1 = 1\)
and \(f = (f_1 + f_2)/2\). Hence the unit ball of \(L^1\)
has no extreme points.

\paragraph{All exterme.}
Suppose \(f\in L^p([0,1])\) where \(0<p<1\) and \(\int_0^1 |f|^p = 1\).
By regularity of the Lebesgue measure, there exists \(a \in (0,1)\) such that 
\begin{equation*}
\int_0^a |f|^p = \int_a^1 |f|^p = 1/2.
\end{equation*}
Define
\begin{equation*}
f_1(t) = \left\{
  \begin{array}{ll}
  2f(t) \quad & 0 \leq t < a \\
  0 \quad & a \leq t \leq 1
  \end{array}
\right.
\qquad
f_2(t) = \left\{
  \begin{array}{ll}
  0 \quad & 0 \leq t < a \\
  2f(t) \quad & a \leq t \leq 1
  \end{array}
\right.
\end{equation*}
It is clear that we have a convex combination \(f = (f_1 + f_2)/2\) and
\begin{equation*}
\int_0^1 |f_j|^p\,dm 
= \int_{\{0,a\}}^{\{a,1\}} |2f|^p\,dm
= 2^p \int_{\{0,a\}}^{\{a,1\}} |f|^p\,dm 
= 2^{p-1} < 1.
\end{equation*}
for \(j=1,2\). Hence \(f_1\),\(f_2\) are in the unit ball of \(L^p\)
and $f$ is not an extreme point there.


%%%%%%% 17
\begin{excopy}
Determine the extreme points of the closed unit ball of $C$, the space of all
continuous functions on the unit interval, with the supremum norm.
(The answer depends on the choice of the scalar field.)
\end{excopy}

When the scalar field is \R\ then the extreme points are
\(f=1\) and \(f= -1\).
When the scalar field is \C\ then the set of extreme points is
\begin{equation*}
\{f\in C(I): \forall t\in I,\, |f(t)|=1\}.
\end{equation*}


%%%%%%% 18
\begin{excopy}Let $K$ be the smallest convex set in \(\R^3\) that contains
the points
\((1,0,1)\)
\((1,0,-1)\), and \((\cos \theta, \sin \theta, 0)\)
for \(0\leq \theta \leq 2\pi\).
Show that $K$ is compact but that the set of all extreme points of $K$ is not
compact. Does such an example exist in \(\R^2\)?
\end{excopy}

Look at the set
\begin{equation*}
G = \{((1,0,1), (1,0,-1)\} \cup
    \{(\cos \theta, \sin \theta, 0): t \in [0,2\pi]\}.
\end{equation*}
and the mapping
\begin{align*}
\varphi: G \times G \times [0,1] &\to E \\
\varphi(u,v,t) &= tu + (1-t)v\,.
\end{align*}
Now $K$ is the image of the continuous mapping with the compact domain
 \(G^2\times [0,1]\) and thus $K$ is compact.
But the set of extreme points is \(E = G \setminus \{(1, 0, 0)\}\)
which is \emph{not} closed and thus not compact.

Such situation cannot happen in \(\R^2\). By negation let \(K\subset \R^2\)
be a compact convex set and \(E(K)\) be not compact.
Thus we have a sequence \((x_n)_{n\in\N}\) in $E$ such that
\(x = \lim_{n\to\infty} x_n \in K\setminus E\).
Now since $x$ is not an extreme point \(x \in (a,b)\) where \([a,b]\subset K\).
\Wlogy, we may assume (by changing to a sub-sequenc) that 
\((x_n)_{n\in\N}\) lie on one side of \([b,c]\).
Pick a sufficently small neighborhood $V$ of $x$ such that \(a,b\notin V\).
Now thee must be some \(j>1\) such that \(x_j \in V\).
Now \(x_j\) is internal to the triangle \(\triangle(x_1,a,b)\),
a~contradiction to \(x_j \in E\).


%%%%%%% 19
\begin{excopy}
Suppose $K$ is a compact convex set in \(\R^n\). Prove that every \(x\in K\)
is a convex combination of at most \(n+1\) extreme points of $K$.
\emph{Suggestion:} Use induction on $n$. Draw a line from some extreme point
of $K$ through $x$ to where it leaves $K$. Use Exercise~1.
\end{excopy}

If there are $m$ extreme points in $K$ and
\(m < n+1\), then the hyperplane
generated by them is isomorphic with isometry is to \(\R^k\)
with \(k \leq m-1\) and we can reduce the problem to \(\R^k\).
Hence we may assume there are at least \(n+1\) extreme points.

\iffalse
If \(b \in \partial K\) is a non-extreme boundary point of $K$, define its
\emph{face} \(F_b\) by
\begin{equation*}
F_b = \{tv + (1-t)w:
   v,w\in K \wedge 0\leq t \leq 1 \wedge\;
   \exists \tau \in(0,1),\, b = \tau v + (1-\tau)u\}.
\end{equation*}

We want to show that \(F_b\) is comact, convex and lies with a hyperplane
of dimension less that that of $K$.
\fi

Following the suggestion. If \(n=1\) then $K$ must be a simple segment
and the result easily follows. Now assume the claim holds for all \(n < m\).
Now assume \(n=m\). Pick some extreme point $v$ and let ray $r$ coming from $v$
towards $x$, that is \(r=\{v + t(x-v): t\in\R^+\}\).
This ray intersects $K$ as a line segment whose end points are
the extreme point $v$ and another boundary point $b$ of $K$.
We pass a hyperplane $H$ in $b$ such that $K$ lies on of of its sides.
Now \(K' = H \cap K\) is clearly compact and convex and can be viewed
(by isometry) as a compact convex set in \(R^{m-1}\).

We claim that the extreme points of \(K'\) are extreme points of $K$.
Otherwise by negation, we have an extreme point \(y \in K'\) which lies in
an interior of line segment
\begin{equation*}
y \in (y_0,y_1) \subsetneq [y_0,y_1] \subset K.
\end{equation*}
But \([y_0,y_1]\) lies within one side of $H$ and since it intersects $H$
in $b$, both \(y_0,y_1 \in H\) contradiction to $y$ being extreme in \(K'\).

Now by induction $b$ can be represented as a convex combination
of less than \(m+1\) extreme points. Consequently, since \(x \in [b,v]\),
there exists a convex combination of
\(m+1\) extreme points (or less) for $x$.

%%%%%%% 20
\begin{excopy}
Let \(\{u_1, u_2, u_3,\ldots\}\) be a sequence of pairwise orthogonal unit
vectors in a Hilbert space.
Let $K$ consist of the vectors $0$ and \(n^{-1}u_n\) (\(n\geq 1\)).
Show that
\ich{a} $K$ is compact;
\ich{b} \(\co(K)\) is bounded.
\ich{c}~\(\co(K)\) is not closed.
Find all extreme points of \(\overline{\co}(K)\).
\end{excopy}

\begin{itemize}
\itemch{a}
Let \(G = \cup_{j\in J}V_j\) be a open cover of $K$.
Pick some \(j_0\in J\) such that \(0\in V_{j_0}\).
Now for some \(N<\infty\) we have \(u_n/n \in V_j\) for all \(n > N\).
Take \(V_{j_0}\) and completeness a finite subfamily of G by picking
\(\{j_n\in J: 1\leq n \leq N\}\) such that \(u_n/n \in V_{j_n}\)
whenever \(1\leq n \leq N\).
\itemch{b}
For each \(v\in \co(K)\) clearly \(\|x\|\leq 1\) hence \(\co(K)\) is bounded.
\itemch{c}
The points \(v_n = \sum_{j=1}^n 2^{-j}u_j \in \co(K)\)
but the limit
   \(\sum_{j=1}^\infty 2^{-j}u_j \in \overline{\co(K)} \setminus \co(K)\).
Hence \(\co(K)\) is not closed.
\end{itemize}
The extreme points of \(\overline{\co(K)}\) are $0$, \(\{u_n:n\in\N\}\)
and all ``infinite convex combinations'' points of the form
\(\sum_{j=1}^\infty a_j u_j\) such that
\begin{gather*}
\forall j\in\N,\;0 \leq a_j < 1 \\
\sum_{j=1}^\infty a_j = 1\\
a_j > 0 \qquad \textnormal{for infinitely many }\; j\in\N.
\end{gather*}


%%%%%%% 21
\begin{excopy}
If \(0<p<1\), every \(f\in L^p\) (except \(f=0\)) is the arithmetic mean
of two functions whose distance from~$0$ is less than that of $f$.
(See Section~1.47.)
Use this to construct an explicit example of a countable compact set $K$
in \(L^p\) (with $0$ as its only limit point) which has no extreme point.
\end{excopy}

Assume \(0<p<1\).
Pick \(f \in L^p([0,1]) \setminus \{0\}\).
Let \(b\in[0,1]\) such that 
\begin{equation*}
\int_0^b |f(t)|^p\,dt = \int_b^1 |f(t)|^p\,dt.
\end{equation*}
Define \(f_1\), \(f_2\) as in~\eqref{eq:f1f2:conv}.
Clearly \(f = (f_1 + f_2)/2\), and
\begin{equation*}
d(0,f_1) 
= \int_0^b |f(t)|^p\,dt 
= 2^p \int_0^b |f(t)|\,dt 
= 2^p (d(0,f)/2)
= 2^{p-1} d(0,f) < d(0,f).
\end{equation*}
And similarly, \(d(0,f_2) < f(0,f)\).

To construct the example, let \(u == 1 \in L^p([0,1]\)
and let \(S_0 = \{f\}\).
By induction, Given a set \(S_k\) of functions in \(L^p([0,1])\)
define \(S_{k+1}\) as the set of functions \(f_1\), \(f_2\) 
defined in~\eqref{eq:f1f2:conv} where \(f\in S_k\)
(thus \(|S_k| = n^k\)).
Clearly if \(f\in S_k\) then \(d(0,f) = 2^{n(p-1)}\)
and actualy there exists some \(0 < j \leq 2^k\) such that 
\(f(t)=1\) if \(t \in (2^{-k}(j-1), 2^{-k}j)\) and 
\(f(t)=0\) if \(t \in [0,1] \setminus [2^{-k}(j-1), 2^{-k}j]\).

Let \(U = \cup_{k=\N} S_k\). Now $U$ has no extreme point, 
since each \(f\in S_k\) is th average of two functions
from \(S_{k+1}\). If \(f_k\in S_k\) then \(\lim_{k\to\infty} d(0,f_k) = 0\), 
hence \(0 \in \overline{U}\) thus if 
\(U^- = \{-f: f\in U\}\), then
\begin{equation*}
K = U \cup \{0\} \cup U^-
\end{equation*}
is the desired compact set with no extreme point.

Notes:
\begin{itemize}
\item
 Eac open neighborhood of $0$ leaves out just a finite 
 number of elements of $K$, hence $K$ is compact.
\item
 \(U \cup \{0\}\) is compact but $0$ is extreme point there.
\end{itemize}


%%%%%%% 22
\begin{excopy}
If \(0<p<1\), show that \ellp\ contains a compact set $K$ whose convex hull is
unbounded. This happens in spite of the fact that \((\ellp)^*\)
separates points in \ellp; see Exercise~5.
\emph{Suggestion:} Define \(x_n \in \ellp\) by
\begin{equation*}
x_n(n) = n^{p-1}, \qquad x_n(m) = 0 \qquad \textnormal{if}\; m\neq n.
\end{equation*}
Let $K$ consist of \(0, x_1, x_2, x_3,\dots\) If
\begin{equation*}
y_N = N^{-1}(x_1 + \cdots + x_N),
\end{equation*}
show that \(\{y_n\}\) is unbounded in \ellp.
\end{excopy}

Clearly \(\lim_{n\to\infty} d(0,x_n) = n^{p(p-1)} = 0\).
Hence single limit point ($0$) and $K$ is compact.
Now
\begin{align*}
d(0,y_N) 
&= \sum_{k=1}^N \left(N^{-1} k^{p-1}\right)^p
 = N^{-p} \sum_{k=1}^N  k^{p(p-1)}
 \geq N^{-p} \int_2^N x^{p(p-1)}\,dx \\
&= N^{-p} \frac{1}{p(p-1)+1}\left(x^{p(p-1)+1}\right)\bigm|_2^N
\end{align*}
But
\begin{align*}
\lim_{N\to\infty} N^{-p}N^{p(p-1)+1}
= \lim_{N\to\infty} N^{p(p-1)+1-p}
= \lim_{N\to\infty} N^{(p-1)^2} = \infty.
\end{align*}
Hence \(\lim_{N\to\infty} d(0,y_N) = \infty\).


%%%%%%% 23
\begin{excopy}
Suppose \(\mu\) is a Borel probability measure on a compact Hausdorff
space $Q$. $X$ is a Fr\'echet space, and \(f:Q\to X\) is continuous.
A \emph{partition} of $Q$ is, by definition, a finite collection of disjoint
Borel subsets of $Q$ whose union is $Q$. Prove that to every neighborhood
$V$ of $0$ in $X$ there corresponds a partition \(\{E_i\}\) such that the
difference
\begin{equation*}
z = \int_Q f\,d\mu - \sum_i \mu(E_i)f(s_i)
\end{equation*}
lies in $V$ for every choice of \(s_i \in E_i\).
(this exhibits the integral as a strong limit of ``Riemann sums.'')
\emph{Suggestion:} Take $V$ convex and balanced.
If \(\Lambda \in X^*\) and if \(|\Lambda x| \leq 1\) for every \(x \in V\),
then \(|\Lambda x| \leq 1\), provided that the sets \(E_i\) are chosen so
that \(f(s) - f(t) \in V\) wherever $s$ and $t$ lie in the same \(E_i\).
\end{excopy}

Let $V_0$ be an arbitrary neighborhood of $0$.
Pick a balanced neighborhood $V$ of \(0\in X\) 
such that \(\overline{V} \subset V_0\).
For each \(q\in Q\) find a neighborhood \(U_q \ni q\) such that 
\(f(U_q) \subset f(q) + V/2 = \{x\in X: 2(x - f(q)) \in V\}\).
By compactness of $Q$, there is 
a~finite set \(F = \{q_1,\ldots,q_n\} \subset Q\)
such that \(Q = \cup_{q\in F} U_q\).

Define \(E_1 = U_{q_1}\) and \(E_j = U_{q_j} \setminus \cup_{k<j} E_k\)
for \(1 < j \leq n\).
If \(s_1,s_2\in E_j\) then
\(f(s_k) - f(q) \in V/2\) for \(k=1,2\) and so 
\(f(s_1) - f(s_2) \in V\).

If \(\Lambda \in X^*\) then
\begin{align*}
\Lambda z
 &= \Lambda\left( \int_Q f\,d\mu - \sum_i \mu(E_i)f(s_i) \right)
  = \int_Q \Lambda f\,d\mu - \sum_i \mu(E_i) \Lambda f(s_i)  \\
 &= \sum_i \int_{E_i} \bigl(\Lambda f -  \Lambda f(s_i)\bigr)\,d\mu
  = \sum_i \int_{E_i} \Lambda \bigl(f - f(s_i)\bigr)\,d\mu.
\end{align*}

Assume \(|\Lambda x| \leq 1\) for every \(x \in V\). Now
\begin{equation}  \label{eq:lambdaz:leq1}
|\Lambda z| 
 \leq \sum_i \int_{E_i} \Lambda \bigl(f - f(s_i)\bigr)\,d\mu
 \leq \sum_i \int_{E_i} 1,d\mu
 = \sum_i \mu(E_i) = 1.
\end{equation}

If by negation \(z\notin V_0\) then by Theorem~3.7
 there exists \(\Lambda\in X^*\)
such that \(|\Lambda x| \leq 1\) for all \(x \in V\)
and \(\Lambda z > 1\). This contradicts \eqref{eq:lambdaz:leq1}.


%%%%%%% 24
\begin{excopy}
In addition to the hypothesis of Theorem~3.27, assume that $T$ us a continuous
linear mapping of $X$ into a topological vector space $Y$ on which \(Y^*\)
separates points and prove that
\begin{equation*}
T\int_Q f\,d\mu = \int_Q (Tf)\,d\mu.
\end{equation*}
\emph{Hint:} \(\Lambda T \in X^*\) for every \(\Lambda \in Y^*\).
\end{excopy}

We use the hint and the result of previous exercise, namely approximation 
of the intergrals via Riemann sums.

Pick arbitrary neighborhood \(U \subset Y\) of \(o\in Y\).
Let \(V = T^{-1}(U)\) which is clearly a neighborhood of \(0\in X\).
For each \(P = \{E_j: j \in N_n\}\) 
and a selection $S$ of \(s_j \in E_j\) for each \(j\in \N_n\) 
we note that
\begin{equation*}
r(P,S) = T\left(\sum_{j=1}^n \mu(E_i)f(s_j)\right) = \sum_{j=1}^n \mu(E_i)(Tf)(s_j).
\end{equation*}

Let \(P = \{E_j: j \in N_n\}\) be a Borel disjoint partition of $X$ such that if
\begin{align*}
z &= \int_Q f\,d\mu - \sum_{j=1}^n \mu(E_i)f(s_j) \in X \\
w &= \int_Q Tf\,d\mu - \sum_{j=1}^n \mu(E_i)(Tf)(s_j) \in Y
\end{align*}
then \(z \in V\) and \(w \in U\) for every choice of $S$.
A mutual partition can be taken by a ``cross intersection'' of two partitions
such that each condition is satisfied by one partition.

Now, such $P$ gives
\begin{align*}
\xi
&= T\int_Q f\,d\mu - \int_Q (Tf)\,d\mu
 = T\left(z + \sum_{j=1}^n \mu(E_i)f(s_j)\right) 
   - w - \sum_{j=1}^n \mu(E_i)(Tf)(s_j) \\
&= Tz - w + r(P,S) - r(P,S) = Tz - w \in U + U.
\end{align*}
Since $U$ was an arbitrary neighborhood of \(0 \in Y\), we must have \(\xi = 0\).


%%%%%%% 25
\begin{excopy}
Let $E$ be the set of all extreme points of a compact set $K$ in a topological
vector space $X$ on which \(X^*\) separates points.
Prove that to every \(y \in K\) corresponds a regular Borel probability measure
\(\mu\) on \(Q = \overline{E}\) such that
\begin{equation*}
y = \int_Q x\,d\mu(x).
\end{equation*}
\end{excopy}

We start with some generalization of 
the 
Banach-Alaoglu
\index{Banach-Alaoglu} Theorem~3.15.
The theorem
 shows that
if $V$ is a neighborhood of \(0\in X\) then the polar
\begin{equation}  \label{eq:polar:sup:nbd}
P^*(V) = \{\Lambda \in X^*: \forall x\in V,\;|\Lambda x| \leq 1 \}
\end{equation}
is weak\upstar-compact.
Following the proof one can see that it holds also when
$V$ is any subset that \emph{contains} a neighborhood of \(0\in X\)
(but $V$ is not necessarily open).


The set $E$ may not be compact, but it is locally compact
since \(E \subset \overline{E} \subset K\).
Since  \(Q \subset K\) it is also compact.
We look at the space \(C(Q) = C_c(Q)\) and its dual space \(C(Q)^*\).

The set \(\conv(E)\) consists of finite convex combinations of points of $E$.
Thus for each \(v \in \conv(E)\) we have
\begin{align*}
v &= \sum_{k=1}^n a_k w_k \qquad (0 < a_k \leq 1, \; w_k \in Q) \\
1 &= \sum_{k=1}^n  a_k
\end{align*}
and we assoicate \(\Lambda_v \in C(K)^*\) by
\begin{equation*}
\Lambda_v (f) = \sum_{k=1}^n a_k f(w_k) \qquad (f \in C(K)).
\end{equation*}

By Riesz representation Theorem (\cite{RudinRCA80})
there exists a (not necessarily unique) measure \(\mu_v\) such that 
\(\Lambda_v(f) = \int_Q f(x)\,d\mu_v(x)\),
but we can simply define \(\mu_v\) by \(\mu(\{w_k\}) = a_k\)
and \(\mu_v(A) = 0\) for any Borel subset \(A\subset Q\)
such that \(\forall k,\; w_k\notin A\).
Now for \(f(x)=x\) we have
\begin{equation} \label{eq:conv:measure:x:v}
\int_Q x\,d\mu_v(x) = \sum_{k=1}^n a_k w_k = v.
\end{equation}

Let \(F\subset C(X)^*\) be all the functionals \(\Lambda_v\)
where for each \(v\in \conv(Q)\) we take all choices of \(\Lambda_v\).
If \(U = \{f\in C(K): \|f\|_\infty \leq 1\}\), then  clearly
\(F \subset \in P^*(U)\) (as defined in \eqref{eq:polar:sup:nbd}).

Now \(\overline{F} \subset P^*(U)\), hence it is weak\upstar-compact.
If we look at the identity function \(\Id(x)=x\)
as a functional on \(C(X)^*\) then the image of \(\Id(\overline{F})\)
is compact. Since we saw that \(\conv{Q} \subset \Id(\overline{F})\)
then  \(\overline{\conv{Q}} \subset \Id(\overline{F})\) as well
(weak\upstar-closure).
By Krein-Milman
\index{Krein-Milman}
Theorem~3.23 \(K = \overline{\conv(Q)}\).
Hence, for every \(y \in K\) there exists some 
\(\Lambda_y \in \overline{F}\) such that \(Id(\Lambda) = y\).
Hence by the Riesz  representation Theorem (\cite{RudinRCA80})
\eqref{eq:conv:measure:x:v} also for all \(y \in K\)
for some Borel measure \(\mu_y\).
Being in the closure of probability measures, such measures
is a probability measure as well.


%%%%%%% 26
\begin{excopy}
Suppose \(\Omega\) is a region in \C, $X$ is a Fr\'echet space, and
\(f:\Omega\to X\) is holomorphic.
\begin{itemize}
\itemch{a}
State and prove a theorem concerning the power series representation of $f$,
that is, concerning the formula \(f(z) = \sum(z-a)^n c_n\), where \(c_n\in X\).
\itemch{b}
Generalize Morera's theorem to $X$-valued holomorphic functions.
\itemch{c}
For a sequence of complex holomorphic functions in \(\Omega\),
uniform convergence on a compact subsets of \(\Omega\) implies that the limit
is holomorphic.
Does this generalize to $X$-valued holomorphic functions?
\end{itemize}
\end{excopy}

\begin{itemize}
\itemch{a}
With the assumptions above,
If \(f:\Omega \to X\) is holomorphic, then for each open circle $U$
with center $a$, such that \(\overline{U} \subset \Omega\)
there exists a sequence \((c_n)_{n\in\Z^+}\) in $X$ such that 
\begin{equation*}
f(z) = \sum(z-a)^n c_n
\end{equation*}
holds for all \(z\in U\).

Let \(\Gamma\) be a parmetrized boundary of $U$.
By Theorem~3.31\ich{b}
\begin{equation*}
f(z) = \frac{1}{2\pi i}\int_\Gamma (\zeta -z)f(\zeta)\,d\zeta.
\end{equation*}
Using ideas as in Theorems~10.7 and 10.16 of \cite{RudinRCA80}
we note that
\begin{equation*}
\frac{1}{\zeta - z} = \sum_{n=0}^\infty \frac{(z-a)^n}{(\zeta - a)^{n+1}}
\end{equation*}
and define
\begin{equation*}
c_n = \dtwopii \int_\Gamma (\zeta - z)^{-n-1}f(\zeta)\,d\zeta
\end{equation*}
By Theorem~3.31\ich{b}
\begin{align}
f(z) 
&= \frac{1}{2\pi i}\int_\Gamma (\zeta -z)^{-1} f(\zeta)\,d\zeta
 = \frac{1}{2\pi i}\int_\Gamma 
   \left(\sum_{n=0}^\infty \frac{(z-a)^n}{(\zeta - a)^{n+1}}\right) 
   f(\zeta)\,d\zeta \notag \\
&= \frac{1}{2\pi i}\int_\Gamma 
   \left(\sum_{n=0}^\infty (z-a)^n(\zeta - a)^{-n-1}\right) f(\zeta)\,d\zeta
   \notag\\
&= \label{eq:holo:tvs:sigmaintegral}
   \frac{1}{2\pi i}
   \sum_{n=0}^\infty
     \left(\int_\Gamma (\zeta - a)^{-n-1} f(\zeta)\,d\zeta \right)
     (z-a)^n \notag \\
&= \sum(z-a)^n c_n
\end{align}
The equality \eqref{eq:holo:tvs:sigmaintegral} is justified
by the uniform convergence of the infinite sum
and the fact that \(\Lambda f\) is bounded on \(\Gamma^*\)
for each \(\Lambda \in X^*\).

\itemch{b}
With the assumptions above,
if for any closed path \(\gamma\) such that \(\gamma^* \subset \Omega\)
the equality
\begin{equation*}
\int_\gamma f(z) = \int_0^1 f(\gamma(t))\gamma'(t)\,dt = 0
\end{equation*}
holds, then $f$ is holomorphic.

\itemch{c}
Yes it does generalize.
Say \(f_n:\Omega \to X\) is such a sequence of holomorphic functions.
Let \(f(z) = \lim_{n\to\infty} f_n(z)\).
Now for any triangle \(T \subset \Omega\) clearly 
\begin{equation*}
\int_{\boundary{T}} f(z)\,dz
= \int_{\boundary{T}} \left(\lim_{n\to\infty} f_n(z)\right) dz 
= \lim_{n\to\infty} \int_{\boundary{T}} f_n(z) dz 
= 0
\end{equation*}
Now by \ich{b} $f$ is holomorphic.
\end{itemize}


%%%%%%% 27
\begin{excopy}
Suppose \(\{\alpha_i\}\) is a bounded set of distinct complex numbers,
\(f(x) = \sum_0^\infty c_n z^n\) is an entire function with every \(c_n\neq 0\)
and
\begin{equation*}
g_i(z) = f(\alpha_iz).
\end{equation*}
Prove that the vector space generated by the functions \(g_i\) is dense in
the Fr\'echet space \(H(\C)\) defined in Section~1.45.

\emph{Suggestion:} Assume \(\mu\) is a measure with compact support such that
\(\int g_i\,d\mu = 0\) for all $i$. Put
\begin{equation*}
\phi(w) = \int f(wz)\,d\mu(z) \qquad (w\in\C).
\end{equation*}
Prove that \(\phi(w) = 0\) for all $w$. Deduce that \(\int z^n\,d\mu(z) = 0\)
for \(n=1,2,3,\ldots\). Use Exercise~14.

Describe the closed subspace \(H(\C)\) generated by the functions \(g_i\),
if some of the \(c_n\) are~$0$.
\end{excopy}

\textbf{Note.} The assumption should require explicitly that the 
set  \(\{\alpha_i\}\) is infinite. Otherwise there are trivial examples
that make the claim false.

Following the suggestion.
\begin{equation*}
\phi(w)
= \int \left(\sum_0^\infty c_n (wz)^n\right)d\mu(z)
= \sum_0^\infty c_n  \left(\int z^n\,d\mu(z)\right) w^n.
\end{equation*}
Hence \(\phi\) is holomorphic.
Since \(\phi(\alpha_j) = 0\) and \(\{\alpha_j:j\in J\}\)
has a limit point in \C,
thus \(\phi\) is identically zero by Theorem~10.14~\cite{RudinRCA80}.
But this shows that 
\begin{equation*}
c_n  \int z^n\,d\mu(z) = 0
\end{equation*}
for all \(n\in \N^+\),
and also
\begin{equation} \label{eq:integral:mu:zn:eq0:HC}
\forall n\in\N+\; \int z^n\,d\mu(z) = 0
\end{equation}
 since \(c_n \neq 0\) for all \(n\in \N^+\).

The functions \(z\to z^n\) span a vector space that is dense in
\(C(K)\) for any compact \(K \subset \C\).
By Exercise~14 \(\int f\,d\mu = 0\) for all \(f \in H(\C)\).

If by negation the vector space $V$ generated by \(\{g_j\}\)
were not dense in \(H(\C\) then by Hahn-Banach Theorem~3.5
and Riesz representation Theorem~3.16 \cite{RudinRCA80}
we could find some Borel measure \(\mu\) with compact support
such that  \(\int f\,d\mu = 0\) for all \(f\in \overline{V}\)
but still \(\int f\,d\mu \neq 0\)
for some \(f \in H(\C) \setminus \overline{V}\).
But this contradicts \eqref{eq:integral:mu:zn:eq0:HC}.

If some \(c_n = 0\) then the closed subspace generated by the
the functions \(\{g_j\}\) has co-dimension 1.


%%%%%%% 28
\begin{excopy}
Suppose $X$ is a Fr\'echet space (or, more generally, a metrizable locally
convex space). Prove the following statements:
\begin{itemize}
\itemch{a}
\(X^*\) is the union of countably many weak\upstar-compact sets \(E_n\).
\itemch{b}
If $X$ is separable, each \(E_n\) is metrizable. The weak\upstar-topology of
\(X^*\) is therefore separable, and some countable subsets of \(X^*\) separates
points on~$X$.
(Compare with Exercise~15.)
\itemch{c}
If $K$ is a weakly compact subset of $X$ and if \(x_0\in K\) is a  weak limit
point of some countable set \(E\subset K\), then there is a sequence
\(\{x_n\}\) in $E$ which converges weakly to \(x_0\).
\emph{Hint:} Let $Y$ be the smallest closed subspace of $X$ that contains $E$.
Apply \ich{b} to $Y$ to conclude that the weak topology of \(K \cap Y\) is
metrizable.

\emph{Remark:} The point of \ich{c} is the existence of convergent
\emph{subsequences} rather than \emph{subnets}. Note that there exist compact
Hausdorff spaces in which no sequence of distinct points converges.
For an example, see Exercise~18, Chapter~11.
\end{itemize}
\end{excopy}

\begin{itemize}
\itemch{a}
By definition, there is a metric $d$ on $X$, such that
\begin{equation*}
U = \{x\in X: d(x,0) \leq 1\}
\end{equation*}
is convex.
Let 
\begin{equation*}
K_n = \{\Lambda \in X^*: \forall x\in U,\, |\Lambda x| \leq n\}\
\qquad (n\in\N).
\end{equation*}
Clearly \(K_n = \{n\Lambda \in X^*: \Lambda \in K_1\}\) and
by
\index{Banach-Alaoglu}
\index{Alaoglu}
Banach-Alaoglu Theorem~3.15 the sets \(K_n\) are weak\upstar-compact
and clearly \(X^* \cup_{n\in\N} K_n\).

\itemch{b}
By Theorem~3.16 each \(E_n\) is metrizable.
For each \(m\in\N\) there is a finite covering of \(E_n\)
with balls of radius \(1/m\). Thus \(E_n\) separable
and consequently so is \(X^* = \cup_{n\in\N} E_n\).

Let \(A \subset X^*\) be a countable weak\upstar\ dense set.
By negation let \(x_1,x_2\in X\) and \(x_1\neq x_2\)
and \(\Lambda x_1 = \Lambda x_2\) for all \(\Lambda \in A\).
By Hahn-Banach Theorem~3.4 
(with \(A=\{x_1\}\) and \(B=\{x_2\}\))
there exists \(\Lambda \in X^*\)
such that \(\Lambda x_1 \neq \Lambda x_2\).
% Let \(z_j = \Lambda x_j\) for \(j=1,2\).
Let
% \(\epsilon = |z_1 - z_2|/2\) and
\(V_j = \{T\in X^*: |T x_j - \Lambda x_j| < \epsilon/2\}\)
for \(j=1,2\).
Clearly \(V_1 \cap A = \emptyset\) or \(V_2 \cap A = \emptyset\).
Therefore $A$ separates points  in $X$.

\itemch{c}
Following the hint.
The subspace $Y$ is originally closed, and by Corollary~\ich{a}
of Theorem~3.12 it is also weak\upstar\ closed.
Thus \(K \cap Y\) is weak\upstar\ compact.
The set
\begin{equation*}
A = \{(q+ir)v \in Y: v\in E \;\wedge\; q,r \in \Q\}
\end{equation*}
is countble and dense in $Y$, thus $Y$ is separable.
and so is \(K \cap Y\).
By \ich{b} there exists a countble set \(S \subset Y^*\) that separates
points in $Y$

By remark \ich{c} of Section~3.8
(where $A$ should be extended to sufficiently rich real valued functions)
we see that  \(K \cap Y\) is metrizable.
With such metric $d$, we give \(x_0\) a countable local base
 \(\{B_d(x_0;1/n): n\in\N\}\).
By choosing \(x_n \in B_d(x_0;1/n) \cap E\) we have 
\begin{equation*}
{\lim}_{n\to\infty}^{\textnormal{w}^*} x_n = x_0
\end{equation*}
as desired.
\end{itemize}


%%%%%%% 29
\begin{excopy}
Let \(C(K)\) be the Banach space of all continuous complex functions on the
compact Hausdorff space $K$, with the supremum norm.
For \(p\in K\), define \(\Lambda_p\in C(K)^*\) by \(\Lambda_p f = f(p)\).
Show that \(p \to \Lambda_p\) is a homeomorphism of $K$ into \(C(K)^*\),
equipped with its weak\upstar-topology. Part \ich{c} of Exercise~28
can therefore not be extended to weak\upstar-compact sets.
\end{excopy}

Call the mapping  \(\varphi\), and
by Urysohn's lemma (\cite{RudinRCA87} 2.12)
it is injective.
We will form a base neighborhood of \(\Lambda_p\).
Pick a finite \(F = \{f_j\in C(K): 1 \leq j \leq n\}\) set 
and \(\epsilon > 0\). Now
\begin{equation*}
V = \{T \in C(X)^*: \forall j\in\N_n,\; |T(f_j) - \Lambda_p(f_j)| < \epsilon\}.
\end{equation*}
By continuity of \(f_j\in F\) there exists some neighborhood \(U \in K\)
of $p$ such that \(|f_j(x) - f_j(p)| < \epsilon\)
for all \(x\in U\) and \(j\in\N_n\).
Hence \(\Lambda_x \in V\) for all \(x\in U\).
Thus the mapping is continuous (and so \(\varphi(K)\) is compact).
The induced topology \(\tau_1\), of \(C(K)^*\) on \(\varphi(K)\)
cannot be richer than that of induced by \(\varphi\), that is
\begin{equation*}
\tau_2 = \{\varphi(G): G \;\textnormal{open in}\; K\}.
\end{equation*}
By continuity of \(\varphi\) we have \(\tau_1 \subset \tau_2\).
Since \(\tau_1\) is clearly Hausdorff, by the remark~\ich{a}
of Section~3.8 therefore \(\varphi\) is a homeomorphism.


%%%%%%% 30
\begin{excopy}
Suppose $p$ is an extreme point of some convex set $K$, and that
\(p = t_1x_1 + \cdots t_n x_n\), where \(\sum t_i = 1\), \(t_i > 0\)
and \(x_i \in K\) for all $i$.
Prove that \(x_i = p\) for all $i$.
\end{excopy}

By induction. The case \(n=1\) is trivial and the case \(n=2\)
is immediate from the definition of extreme point.
Assume the claim holds whenever \(n < k\),
and now assume \(n=k\).
put \(a = 1 - t_n\) and \(y = (p - t_nx_n)/a\). Clearly
\(y = \sum_{i=1}^{n-1} (t_i/a) x_i\) and \(y \in \co(K)\) since
\(\sum_{i=1}^{n-1} (t_i/a) = 1\).
Now \(p = ay + t_nx_n\). By the trivial case of \(n=2\)\
we have \(p = y = x_n\).
By symmetry or induction, we also have  \(p = x_i\) for \(1 \leq i < n\).


%%%%%%% 31
\begin{excopy}
Suppose that \(A_1,\ldots,A_n\) are convex sets in a vector space $X$.
Prove that every \(x \in \co(A_1 \cup \cdots \cup A_n)\) can be represented
in the form
\begin{equation*}
x = t_1 a_1 + \cdots + t_n a_n,
\end{equation*}
with \(a_i \in A_i\) and \(t_i \geq 0\) for all $i$, \(\sum t_i = 1\).
\end{excopy}

Let \(x = \sum_{j=1}^m = s_j u_j\) where \(u_j \in \cup_{k=1}^n A_n\) and
\(\sum_{j=1}^m s_j = 1\) and \(s_j \in [0,1]\) for all \(j\in\N_m\).
Clearly all such $x$ combinations form \(\co\left(\cup_{k=1}^n A_k\right)\).
Now we reorder \((s_j u_j)_{j=1}^m\) such that 
we have increasing sequence of indices satisfying
\begin{gather*}
p_0 = 0
 \qquad
 p_{k-1} \leq p_k \;\textnormal{for}\; (k \in \N_n)
 \qquad
 p_n = m
 \\
 \forall k\in\N_n\; p_{k-1} < j \leq p_k \;\Rightarrow\; u_j \in A_k.
\end{gather*}
Putting
\begin{equation*}
t_k = \sum_{j=p_{k-1}+1}^{p_k} s_j \qquad (k \in \N_n)
\end{equation*}
gives
\begin{equation*}
x
 = \sum_{j=1}^m = s_j u_j
 = \sum_{k=1}^n \sum_{j=p_{k-1}+1}^{p_k} s_j u_j
 = \sum_{k=1}^n t_k \sum_{j=p_{k-1}+1}^{p_k} (s_j/t_k) u_j
\end{equation*}
which shows the desired representation, since
\(a_k = \sum_{j=p_{k-1}+1}^{p_k} (s_j/t_k) u_j \in A_k\) for all \(k\in\N_n\)
by the convexity of \(A_k\).

%%%%%%% 32
\begin{excopy}
Let $X$ be an infinite-dimensional Banach space and let
\(S = \{x \in X: \|x\| = 1\}\) be the unit sphere of $X$. We want to cover $S$
with finitely many closed balls, none of which contains the origin of $X$.
Can this be done in \ich{a} every $X$, \ich{b} some $X$,  \ich{c} no $X$ ?
\end{excopy}

It can never be done.

First let's establish simple algebraic lemma about finite co-dimensions
within infinite-dimensional vector space.

\begin{llem} \label{llem:infiker}
Let $X$ be a infinite-dimensional vector space, and \(S_j \subset X\)
subspaces such that \(X/S_j\) are finite-dimensional vector spaces
for \(j\in\N_n\). If \(S = \cap_{j=1}^n S_j\) then \(X/S\) is finite-dimensional.
\end{llem}
\begin{proof}
let \(\pi_j : X \to X/S_j\) be the projection mappings (\(j\in\N_n\).
Define the map:
\begin{align*}
p: X &\to \prod_{j=1}^n (X/S_j) \\
p(x) &= (\pi_j(x))_{j=1}^n.
\end{align*}
Clearly \(S = \Ker p\)
 and
\begin{equation*}
\dim(\Image p) = \prod_{j=1}^n(\dim(X/S_j)) < \infty.
\end{equation*}
Since \(\dim X = \dim(\Ker p) + \dim(\Image p)\)
we must have \(\,\dim S = \infty\).
\end{proof}

By negation, let \(\{B_j: j\in\N_n\}\) be such covering set, 
where \(B_j = \overline{B(c_j;r_j)}\).
Clearly \(r_j < \|c_j\|\) for all \(j\in\N_n\).

By Corollary to Theorem~3.3 there exists \(\Lambda_j \in X^*\) 
such that 
\begin{align*}
\Lambda_j c_j &= \|c_j\| \\
\forall x\in X,\; |\Lambda_j x| &\leq \|x\|
\end{align*}
for all \(j\in\N_n\).  Put
\begin{equation*}
N = \cap_{j=1}^n \Ker \Lambda_j.
\end{equation*}
By the above local lemma~\ref{llem:infiker} \(\dim N = \infty\),
we actually need just \(\dim N > 0\).
Thus there exists some \(u \in N\) such that \(\|u\| = 1\).
We will show the contradiction \(u \notin \cup_{j=1}^n B_j \).

For each \(j\in\N_n\) we have
\begin{align*}
\|c_j - u\|
 &\geq |\Lambda_j(c_j - u)|
 = |\Lambda_j(c_j) - \Lambda(u)|
 = |\Lambda_j(c_j)| = \|c_j\| > r_j.
\end{align*}
Hence \(u \notin B_j\).

See also \cite{Fabian2001} Lemma~6.15.




%%%%%%% 33
\begin{excopy}
Let \(C(I)\) be the Banach space of all continuous complex functions on
the closed unit interval $I$, with the supremum norm.
Let \(M = C(I)^*\), the space of all complex Borel measures on $I$.
Give $M$ the weak\upstar-topology induced by \(C(I)\).

For each \(t \in I\), let \(e_t \in M\) be the ``evaluation functional''
defined by \(e_t f = f(t)\), and define \(\Lambda \in M\) by
\(\Lambda f = \int_o^1 f(s)\,ds\).
\begin{itemize}
\itemch{a}
Show that \(t \to e_t\) is a continuous map from $I$ into $M$ 
and that \(K = \{e_t: t\in I\}\) is a compact set in $M$.
\itemch{b} Show that \(\Lambda \in \overline{\co}(K)\).
\itemch{c} Find all \(\mu \in \overline{\co}(K)\).
\itemch{d} Let $X$ be the subspace of $M$ consisting of all finite linear 
combinations 
\begin{equation*}
c_0 \Lambda + c_1 e_{t_1} + \cdots + c_n e_{t_n}
\end{equation*}
with complex coefficients \(c_j\). Note that \(\co(K) \subset K\) and that 
\(X \cap \overline{\co}(K)\) is the closed convex hull of $K$ 
within $X$. Prove that \(\Lambda\) is an extreme point 
of \(X \cap \overline{\co}(K)\), even though \(\Lambda\) is not in $K$.
\end{itemize}
\end{excopy}

\begin{itemize}
\itemch{a}
A special case of Exercise~29.

\itemch{b}
Answered by \ich{c}

\itemch{c}
Let 
\begin{equation*}
F = \{\mu: \mu \geq 0 \;\wedge\; \|\mu\| = 1\}.
\end{equation*}
We calim that  \(\overline{\co}(K) = F\),
that is the set of all positive measures \(\mu\)
such that \(\|\mu\| = 1\) where
\begin{equation*}
\|\mu\|  = \sup 
  \bigl\{\int_I |f|\,d\mu: f\in C(I) \;\wedge\; \|f\|_\infty \leq 1\bigr\}.
\end{equation*}
Since \(\mu \geq 0\) we simply have
 \(F = \{\mu\in C(I)^*: \mu\geq 0 \wedge \mu(I)=1\}\).

By Banach-Alaoglu
\index{Alaoglu}
Theorem~3.15, $F$ is weak\upstar\ compact and it is easy to see
that $F$ is convex. We just need to show that \(\ext(F) = K\).

Assume for some 
\begin{equation*}
e_t = a_1\mu_1 + a_2 \mu_2 \qquad
  (a_1,a_2 \in [0,1],\; \mu_j \geq 0,\; \|\mu\| = 1).
\end{equation*}
Clearly \(a_1 \mu_1(\{t\}) + a_2 \mu_2(\{t\})\)
and so \(\mu_j(\{t\}) = 1\) for \(j=1,2\) and since \(\|\mu_j\| = 1\)
we have \(\mu_j(E) = 0\) for any measurable \(E \subset I \setminus \{t\}\).
Hence \(\mu_1 = \mu_2 = e_t\) which is exterme point of $K$.

Now assume by negation \(\nu \in \ext(F) \setminus K\).
Thus we have a partition \(I = A \disjunion B\) and \(0 < \nu(A), \nu(B) < 1\).
Put
\begin{equation*}
a = \nu(A) \qquad b = \nu(B)
\end{equation*}
and define
\begin{equation*}
\nu_s(E) 
=   \frac{(a+b)\nu(E \cap A)}{a}(1 - t)
  + \frac{(a+b)\nu(E \cap B)}{b}t  \qquad ( 0 \leq s \leq 1).
\end{equation*}
for each Borel measurable \(E\subset I\). Clearly
\begin{gather*}
\forall\in[0,1]\; \nu_s \in A. \\
\nu_0 \neq \nu \neq \nu_1.\\
s = b/(a+b) \;\Rightarrow\; \nu_s = \nu.
\end{gather*}
Hence \(\nu \notin \ext(F)\).
Our claim is finally proven by applying Krein-Milman
\index{Krein-Milman}
Theorem~3.23.

\itemch{d}
Enumerate the numbers of \(I\cap \Q\) as \((q_j)_{j\in\N}\).
Define
\begin{equation*}
\Lambda_n = \frac{1}{n}\sum_{j=1}^n e_t.
\end{equation*}
Clearly \(\Lambda_n \in \co(K)\) and 
\(\lim_{n\to\infty} = \Lambda\) in \(C(I)^*\).
Hence \(\Lambda \in X \cap \overline{\co}(K)\).
Now if
\begin{equation*}
\mu = c_0\Lambda + \sum_{j=1}^n c_j e_{t_j} \qquad
  (c_j \neq 0\;\wedge\; t_j < t_{j+1})
\end{equation*}
then
\begin{equation*}
\|\mu\| = \sum_{j=0}^n |c_j|
\end{equation*}
this is justified by picking arbitrary \(\epsilon>0\) 
picking intervals \(J_j\) such that \(t_j\in J_j\) for \(j\in\N_n\)
and 
\begin{equation*}
\sum_{j=1}^n m(J_j) < \epsilon/\max\{|c_j|:j\in\N_n\}
\end{equation*}
and forming a function 
\(f\in C(I)\) such that 
\begin{gather*}
\|f\|_\infty \leq 1 \\
\forall j\in\N_n,\; f(t_j) = \overline{c_j}/|c_j| \\
c_0 \neq 0 \;\Rightarrow\; f(x) = \overline{c_0}/|c_0|
\end{gather*}

Note that if \(T = c_0\Lambda + \sum_{j=1}^n c_je_{t_j}\)
and \(t_j < t_{j+1}\) then \(\T\| = \sum_{j=0}^n |c_j|\).

Now assume
\begin{align*}
\mu &= b_0\Lambda + \sum_{j=1}^n b_j e_{t_j} \\
\nu &= c_0\Lambda + \sum_{j=1}^n c_j e_{u_j} 
\end{align*}
and both
 \(\|\mu\| = \|\nu\| = 1\).
if \(\Lambda = \mu (1-t) + \nu t\)
then by looking at all Borel measurable sets
 \(E \subset I \setminus 
     \left(
     \left(\cup_{j=m} \{t_j\}\right)
     \cup
     \left(\cup_{j=n} \{u_j\}\right)
     \right)\)
it is easy to see that \(\Lambda(E) =  b_0\Lambda(E) + c_0 \Lambda(E)\)
Thus \(b_0 + c_0 = 1\) and \(b_j = c_j = 0\) for all \(j>0\).
Hence it is a trivial convex combination for representating \(\Lambda\)
which is exterme as desired.
\end{itemize}




%%%%%%%%%%%%%%%
\end{enumerate}
%%%%%%%%%%%%%%%

\else
\fi


%%%%%%%%%%%%%%%%%%%%%%%%%%%%%%%%%%%%%%%%%%%%%%%%%%%%%%%%%%%%%%%%%%%%%%%%
%%%%%%%%%%%%%%%%%%%%%%%%%%%%%%%%%%%%%%%%%%%%%%%%%%%%%%%%%%%%%%%%%%%%%%%%
%%%%%%%%%%%%%%%%%%%%%%%%%%%%%%%%%%%%%%%%%%%%%%%%%%%%%%%%%%%%%%%%%%%%%%%%
% \bibliographystyle{plain}
\bibliographystyle{alpha}
\bibliography{rudinfa}

%%%%%%%%%%%%%%%%%%%%%%%%%%%%%%%%%%%%%%%%%%%%%%%%%%%%%%%%%%%%%%%%%%%%%%%%
%%%%%%%%%%%%%%%%%%%%%%%%%%%%%%%%%%%%%%%%%%%%%%%%%%%%%%%%%%%%%%%%%%%%%%%%
%%%%%%%%%%%%%%%%%%%%%%%%%%%%%%%%%%%%%%%%%%%%%%%%%%%%%%%%%%%%%%%%%%%%%%%%
\printindex


%%%%%%%%%%%%%%%%%%%%%%%%%%%%%%%%%%%%%%%%%%%%%%%%%%%%%%%%%%%%%%%%%%%%%%%%
%%%%%%%%%%%%%%%%%%%%%%%%%%%%%%%%%%%%%%%%%%%%%%%%%%%%%%%%%%%%%%%%%%%%%%%%
%%%%%%%%%%%%%%%%%%%%%%%%%%%%%%%%%%%%%%%%%%%%%%%%%%%%%%%%%%%%%%%%%%%%%%%%
\end{document}
